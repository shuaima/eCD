\section{Preliminaries}
\label{sec-preliminary}

In this section, we first introduce basic concepts for piece-wise line based trajectory compression.
We then describe the optimal \lsa algorithm and the \textit{sector intersection} mechanism, and show how this mechanism can be used to speed up the \lsa algorithms using \ped and why it cannot work with \sed.
Finally, we illustrate a convex polygon intersection algorithm, which serves as one of the fundamental components of our local synchronous distance checking method.
%
Notations used are summarized in Table~\ref{tab:notations}.



%\textcolor[rgb]{1.00,0.00,0.00}{A discussion and summary is presented in the last.}



\subsection{Basic Notations}
\label{subsec-notation}

We first introduce basic notations.

\begin{table}
	\renewcommand{\arraystretch}{1.35}
	\caption{\small Summary of notations}
%	\vspace{-1ex}
	\centering
	\footnotesize
	%\scriptsize
	\begin{tabular}{|c|l|}
		\hline
		{\bf Notations}& {\bf Semantics}   \\
		\hline %\hline
		$P$ & a data point \\
		\hline
		$\dddot{\mathcal{T}}$ & a trajectory $\dddot{\mathcal{T}}$ is a sequence of data points\\
		\hline
		$\overline{\mathcal{T}}$&  {a piece-wise line representation of a }	\\
								& trajectory $\dddot{\mathcal{T}}$		\\
		\hline
		$\mathcal{L}$ & a directed line segment  \\


\hline
		$\vv{P_{s}P_{e}}$ & a directed line segment with the start point $P_s$ \\
        & and the end point $P_e$	\\
        \hline
$|\mathcal{L}|$ &  the length of $\mathcal{L}$ in the x-y 2D space  \\


		\hline
		$ped(P, \mathcal{L})$ &  {the perpendicular Euclidean distance of }	\\
								& point $P$ to line segment $\mathcal{L}$	\\
		\hline
		$sed(P, \mathcal{L})$ & {the synchronous Euclidean distance of }	\\
								& point $P$ to line segment $\mathcal{L}$	\\
		\hline
		$\epsilon$ & the error bound \\
		\hline
		\sector{} & a sector \\
		\hline
		$\vv{A} \times \vv{B}$ & the cross product of (vectors) $\vv{A}$ and $\vv{B}$\\
		\hline
		$\mathcal{H}(\mathcal{L})$ & The open half-plane to the left of $\mathcal{L}$ \\
		\hline
		$\mathcal{R}$& a convex polygon \\
		\hline
		$\mathcal{R}^*$ & the intersection of convex polygons \\
		\hline
		$m$ & the maximum number of edges of a polygon\\
		\hline
		$E^j$ & a group of edges labeled with $j$\\
		\hline
		$g(e)$ & the label of an edge $e$ of polygons \\
		\hline
		\circle{} & a synchronous circle\\
		\hline
		\cone{} & a spatio-temporal cone \\
		\hline
		\pcircle{} & a cone projection circle \\
		\hline
		$\bigsqcap$ & intersection of geometries\\
		\hline
		$G$ &	the reachability graph of a trajectory\\
		\hline
	\end{tabular}
	\label{tab:notations}
	\vspace{-1ex}
\end{table}


\stitle{Points ($P$)}. A data point is defined as a triple $P(x, y, t)$, which represents that a moving object is located at {\em longitude} $x$ and {\em latitude} $y$ at {\em time} $t$. Note that data points can be viewed as points in a three-dimension Euclidean space.

\stitle{Trajectories ($\dddot{\mathcal{T}}$)}. A trajectory $\dddot{\mathcal{T}}[P_0, \ldots, P_n]$ is a sequence of data points in a monotonically increasing order of their associated time values (\ie $P_i.t < P_j.t$ for any $0\le i<j\le n$). Intuitively, a trajectory is the path (or track) that a moving object follows through space as a function of time~\cite{physics-trajectory}.


\stitle{Directed line segments ($\mathcal{L}$)}. A directed line segment (or line segment for simplicity) $\mathcal{L}$ is defined as $\vv{P_{s}P_{e}}$, which represents the closed line segment that connects the start point $P_s$ and the end point $P_e$.
Note that here $P_s$ or $P_e$ may not be a point in a trajectory $\dddot{\mathcal{T}}$.

%, and hence, we also use notation $\mathcal{R}$ instead of $\mathcal{L}$ when both $P_s$ and $P_e$ belong to $\dddot{\mathcal{T}}$.

\textcolor{blue}{For the projection of a directed line segment $\mathcal{L}$ on a \emph{x-y} 2D space, where $x$ and $y$ are the longitude and latitude, respectively, we also use $|\mathcal{L}|$ and $\mathcal{L}.\theta\in [0, 2\pi)$ to denote the length of $\mathcal{L}$ in the x-y 2D space}, and its angle with the $x$-axis of the coordinate system $(x, y)$.
 That is, the projection of a directed line segment $\mathcal{L}$ = $\vv{P_{s}P_{e}}$ on a \emph{x-y} 2D space is treated as a triple $(P_s, |\mathcal{L}|, \mathcal{L}.\theta)$.


\stitle{Piecewise line representation ($\overline{\mathcal{T}}$)}. A piece-wise line representation $\overline{\mathcal{T}}[\mathcal{L}_0, \ldots , \mathcal{L}_m]$ ($0< m \le n$) of a trajectory $\dddot{\mathcal{T}}[P_0, \ldots, P_n]$ is a sequence of continuous directed line segments $\mathcal{L}_{i}$ = $\vv{P_{s_i}P_{e_i}}$ ($i\in[0,m]$) of $\dddot{\mathcal{T}}$  such that $\mathcal{L}_{0}.P_{s_0} = P_0$, $\mathcal{L}_{m}.P_{e_m} = P_n$ and  $\mathcal{L}_{i}.P_{e_i}$ = $\mathcal{L}_{i+1}.P_{s_{i+1}}$ for all $i\in[0, m-1]$. Note that each directed line segment in $\overline{\mathcal{T}}$ essentially represents a continuous sequence of data points in $\dddot{\mathcal{T}}$.

%%%%%%%%%%%%%%%%%%%%%%%%%%%%%%%%%%%%%%%%%%%%%%%%%%%%%%%%%%%%%%%%%%%%%%%%%%%%%%
\eat{
\subsubsection{Notations of error metrics}

{For line simplification, there are distance based and shape based error metrics\cite{Shi:Survey} that measure the errors between the original trajectory and the simplified trajectory.
However, for trajectory simplification, the distance based metrics are definitely the distinct metrics.}

\stitle{Included angles ($\angle$)}. Given two directed line segments $\mathcal{L}_1$ = $\vv{P_{s}P_{e_1}}$ and $\mathcal{L}_2$ = $\vv{P_{s}P_{e_2}}$ with the same start point $P_s$, the included angle from $\mathcal{L}_1$ to $\mathcal{L}_2$, denoted as $\angle(\mathcal{L}_1, \mathcal{L}_2)$,  is $\mathcal{L}_2.\theta - \mathcal{L}_1.\theta$. For convenience, we also represent the included angle  $\angle(\mathcal{L}_1, \mathcal{L}_2)$ as $\angle{P_{e_1}P_sP_{e_2}}$.
}
%%%%%%%%%%%%%%%%%%%%%%%%%%%%%%%%%%%%%%%%%%%%%%%%%%%%%%%%%%%%%%%%%%%%%%%%%%%%%%

%comments: 20171126
%\stitle{Perpendicular Euclidean Distance (\ped)}. Given a data point $P$ and a directed line segment $\mathcal{L}$ = $\vv{P_{s}P_{e}}$, the perpendicular Euclidean distance (or simply perpendicular distance) $ped(P, \mathcal{L})$ of $P$ to $\mathcal{L}$ is the Euclidean distance of $P$ to line $\overline{P_{s}P_{e}}$, adopted by many trajectory simplification methods, \eg~\cite{Douglas:Peucker, Hershberger:Speeding, Liu:BQS, Williams:Longest, Sklansky:Cone, Dunham:Cone, Zhao:Sleeve, Lin:Operb}.


\stitle{Perpendicular Euclidean Distance (\ped)}. Given a data point $P$ and a directed line segment $\mathcal{L}$ = $\vv{P_{s}P_{e}}$, the perpendicular Euclidean distance (or simply perpendicular distance) $ped(P, \mathcal{L})$ of point $P$ to line segment $\mathcal{L}$ is $\min\{|\vv{PQ}|\}$ for any point $Q$ on $\vv{P_{s}P_{e}}$.



%ll as does not support applications like spatio-temporal query

\stitle{Synchronized points \cite{Meratnia:Spatiotemporal}}. Given a sub-trajectory $\dddot{\mathcal{T}}_s[P_s$, $\ldots, P_e]$, the synchronized point $P'$ of a data point  $P \in \dddot{\mathcal{T}}_s$,~\wrt line segment $\vv{P_sP_e}$ is defined as follows:
(1) $P'.x$ = $P_s.x +  w\cdot(P_e.x - P_s.x)$,
(2) $P'.y$ = $P_s.y +  w\cdot(P_e.y - P_s.y)$ and
(3) $P'.t$ = $P.t$, where $w= \frac{P.t-P_s.t}{P_e.t-P_s.t}$.

\textcolor{blue}{Intuitively, a synchronized data point $P'_i$ of $P_i$ ($s<i<e$) is a point on $\vv{P_sP_{e}}$ satisfying $|\vv{P_sP_e}|>0$ and $\frac{|\vv{P_sP'_i}|}{|\vv{P_sP_e}|} = \frac{P_i.t - P_s.t}{P_e.t - P_s.t}$, which means that the object is moving from $P_s$ to $P_e$ at an average speed $\frac{|\vv{P_sP_e}|}{P_e.t-P_s.t}$, and its position at time $t_i$ is point $P'_i$ on $\overrightarrow{P_sP_{e}}$ having a distance of $\frac{P_i.t - P_s.t}{P_e.t - P_s.t}\cdot|\vv{P_sP_e}|$ to point $P_s$~\cite{Meratnia:Spatiotemporal, Chen:Fast, Zhang:Evaluation}.}
%\textcolor{blue}{This definition assumes that the sub-trajectory $[P_s, ..., P_e]$ is to be represented by or simplified to a line segment $\overrightarrow{P_sP_e}$ and the object is moving from $P_s$ to $P_e$ linearly in a \textit{uniform speed}. This assumption is followed by the subsequent trajectory simplification works\cite{Meratnia:Spatiotemporal, Chen:Fast, Zhang:Evaluation}.}
%from the start point to the end point of a trajectory .

\stitle{Synchronous Euclidean Distance (\sed) \cite{Meratnia:Spatiotemporal}}. Given a data point $P$ and a directed line segment $\mathcal{L}$ = $\vv{P_{s}P_{e}}$, the synchronous Euclidean distance (or simply synchronous distance) $sed(P, \mathcal{L})$ of $P$ to $\mathcal{L}$ is $|\vv{PP'}|$ that is the Euclidean distance from $P$ to its synchronized point $P'$ \wrt $\mathcal{L}$. %, adopted by recent trajectory simplification methods~\cite{Meratnia:Spatiotemporal, Potamias:Sampling, Chen:Fast, Muckell:Compression, Popa:Spatio}.

\textcolor{blue}{The use of \ped brings better compression ratios, but the temporal information of data points is not available \cite{Meratnia:Spatiotemporal}, which makes \ped not suitable for spatio-temporal queries.}
\textcolor{blue}{In contrast, \sed takes both spatial and temporal information of data points into consideration \cite{Meratnia:Spatiotemporal}. Hence \sed is more suitable for spatio-temporal queries.}


We illustrate these notations with examples.


\begin{example}
\label{exm-notations}
Consider Figure~\ref{fig:notations} again, in which

\sstab(1) $\dddot{\mathcal{T}}[P_0$, $\ldots, P_{10}]$ is a trajectory having 11 data points,

\sstab(2) the set of two continuous line segments $\{\vv{P_0P_4}$, $\vv{P_4P_{10}}$\} (Left) and the set of four continuous line segments $\{\vv{P_0P_2}$, $\vv{P_2P_4}$, $\vv{P_4P_7}$, $\vv{P_7P_{10}}$\} (Right) are two piecewise line representations of trajectory $\dddot{\mathcal{T}}$,

\sstab(3) $ped(P_4, \vv{P_0P_{10}})=|\vv{P_4P^*_4}|$, where $P^*_4$ is the perpendicular point of $P_4$ \wrt line segment $\vv{P_0P_{10}}$,

\sstab(4) \textcolor{blue}{$P'_4$ is the synchronized point of $P_4$ \wrt line segment $\vv{P_0P_{10}}$, satisfying $\frac{|\vv{P_0P'_4}|}{|\vv{P_0P_{10}}|} = \frac{P_4.t - P_0.t}{P_{10}.t-P_0.t} = \frac{4-0}{10-0}= \frac{2}{5}$, and}

\sstab(5) $sed(P_4, \vv{P_0P_{10}})= |\vv{P_4P'_4}|$, $sed(P_2, \vv{P_0P_{4}})= |\vv{P_2P'_2}|$ and $sed(P_7, \vv{P_4P_{10}})$ $=$ $|\vv{P_7P'_7}|$,
where points $P'_4$, $P'_2$ and $P'_7$ are the synchronized points of $P_4$, $P_2$ and $P_7$ \wrt line segments $\vv{P_0P_{10}}$, $\vv{P_0P_{4}}$ and $\vv{P_4P_{10}}$, respectively. \eop
\end{example}

\stitle{Error bounded algorithms}. Given a trajectory \trajec{T} and its compression  algorithm $\mathcal{A}$ using \sed (respectively \ped) that produces another trajectory \trajec{T'},
we say that algorithm $\mathcal{A}$ is error bounded by $\epsilon$ if  for each point $P$ in \trajec{T}, there exist points $P_j$ and $P_{j+1}$ in \trajec{T'} such that such that $P_j.t$ $\le$ $P.t$ $\le$ $P_{j+1}.t$ and $sed(P, \mathcal{L}(P_j,P_{j+1}))\le \epsilon$ (respectively $ped(P, \mathcal{L}(P_j,P_{j+1}))\le \epsilon$).



\subsection{The Optimal \lsa Algorithm}
\label{subsec-optimal}

Given a trajectory \trajec{T}${[P_0, \ldots, P_n]}$ and an error bound $\epsilon$, the optimal trajectory simplification problem, as formulated by Imai and Iri in \cite{Imai:Optimal}, can be solved in two steps: (1) construct a reachability graph $G$ of \trajec{T}, and (2) search a shortest path from $P_0$ to $P_{n}$ in graph $G$.

The reachability graph of a trajectory \trajec{T}${[P_0, \ldots, P_n]}$ \wrt a bound $\epsilon$ is an unweighted graph $G(V, E)$, where (1) $V = \{P_0, \ldots, P_n\}$, and (2) for any nodes $P_s$ and $P_{s+k} \in V$ ($s\ge 0, k>0, s+k\le n$), edge $(P_s, P_{s+k}) \in E$ if and only if the distance of each point $P_{s+i}$ $(i\in[0,k])$ to line segment $\vv{P_sP_{s+k}}$ is not greater than $\epsilon$.

Observe that in the reachability graph $G$, (1) a path from nodes $P_0$ to $P_{n}$ is a representation of trajectory \trajec{T}. The path also reveals the subset of points of \trajec{T} used in the approximate trajectory, (2) the path length corresponds to the number of line segments in the approximate trajectory, and
(3) a shortest path is an optimal representation of trajectory \trajec{T}.

Constructing the reachability graph $G$ needs to check for all pairs of points $P_s$ and $P_{s+k}$ whether the distances of all points $P_{s+i}$  ($0<i<k$) to the line segment $\vv{P_sP_{s+k}}$ are less than $\epsilon$.
There are $O(n^2)$ pairs of points in the trajectory and checking the error of all points $P_{s+i}$ to a line segment $\vv{P_sP_{s+k}}$ takes $O(n)$ time.
Thus, the construction step takes $O(n^3)$ time.
Finding shortest paths on unweighted graphs can be done in linear time. Hence, the brute-force algorithm takes $O(n^3)$ time in total.

%
Though the brute-force algorithm was initially developed using \ped, it can be used for \sed.
As pointed out in \cite{Chan:Optimal}, the construction of the reachability graph $G$ using \ped can be implemented in $O(n^2)$ time using the \textit{sector intersection} mechanism (see Section \ref{sub-ci-ped}). However, the \textit{sector intersection} mechanism cannot work with \sed.  Hence, the construction of the reachability graph $G$ using \sed remains in $O(n^3)$ time, and the brute-force algorithm using \sed remains in  $O(n^3)$ time.


%%%%%%%%%%%%%%%%%%%%%%%%%%%%%%%%%%%%%%%%%%%%%%%%%%%%%%%%%
% Cone Intersection
%%%%%%%%%%%%%%%%%%%%%%%%%%%%%%%%%%%%%%%%%%%%%%%%%%%%%%%%%
\begin{figure*}[tb!]
	\centering
	\includegraphics[scale=0.78]{figures/Fig-sleeve.png}
	%\vspace{-1ex}
	\caption{\small Trajectory $\dddot{\mathcal{T}}[P_0, \ldots, P_{10}]$ in Figure~\ref{fig:notations} is compressed into two line segments by the Sector Intersection algorithm  \cite{Williams:Longest, Sklansky:Cone}. Each circle in the figure has a radius of $\epsilon/2$, which is used to define the sector. }
	\vspace{-1ex}
	\label{fig:sleeve}
\end{figure*}


\subsection{Sector Intersection based Algorithms using \ped}
\label{sub-ci-ped}





The sector intersection (\cia) algorithm \cite{Williams:Longest, Sklansky:Cone} was developed for graphic and pattern recognition in the late 1970s, for the approximation of arbitrary planar curves by linear segments or finding a polygonal approximation of a set of input data points in a 2D Cartesian coordinate system. The \sleeve algorithm \cite{Zhao:Sleeve} in the cartographic discipline essentially applies the same idea as the \cia algorithm.
Further, \cite{Dunham:Cone}  optimized algorithm \cia by considering the distance between a potential end point and the initial point of a line segment. It is worth pointing out that all these \cia based algorithms use the perpendicular Euclidean distances.

%Note that these notable algorithms are not introduced to the discipline of trajectory compression.
%though these notable algorithms are famous in disciplines of graphic and pattern recognition,

Given a sequence of data points $[P_{s}, P_{s+1}, \ldots, P_{s+k}]$ and an error bound $\epsilon$, the \cia based algorithms process the input points one by one in order, and produce a simplified polyline.  Instead of using the distance threshold $\epsilon$ directly, the \cia based algorithms convert the distance tolerance into a variable angle tolerance for testing the successive data points.

For the start data point $P_s$, any point $P_{s+i}$ and $|\vv{P_sP_{s+i}}|>\epsilon$ ($i\in[1, k]$), there are two directed lines $\vv{P_sP^u_{s+i}}$ and $\vv{P_sP^l_{s+i}}$ such that $ped(P_{s+i}, \vv{P_sP^u_{s+i}})$ $=$ $ped(P_{s+i}, \vv{P_sP^l_{s+i}}) = \epsilon$ and either ($\vv{P_sP^l_{s+i}}.\theta < \vv{P_sP^u_{s+i}}.\theta ~and~\vv{P_sP^u_{s+i}}.\theta - \vv{P_sP^l_{s+i}}.\theta <\pi$) or ($\vv{P_sP^l_{s+i}}.\theta > \vv{P_sP^u_{s+i}}.\theta ~and~ \vv{P_sP^u_{s+i}}.\theta - \vv{P_sP^l_{s+i}}.\theta < -\pi)$. Indeed, they form a \emph{sector} \sector{(P_s, P_{s+i}, \epsilon)} that takes $P_s$ as the center point and $\vv{P_sP^u_{s+i}}$ and $\vv{P_sP^l_{s+i}}$ as the border lines.
% , where $P_s$ is the center point, and $\vv{P_sP^l_{s+i}}$ and $\vv{P_sP^l_{s+i}}$ are two bound lines.
Then there exists a data point $Q$ such that for any data point $P_{s+i}$ ($i \in [1, ... k]$), its perpendicular Euclidean distance to
directed line $\overline{P_sQ}$ is not greater than the error bound $\epsilon$ if and only if the $k$ sectors \sector{(P_s, P_{s+i}, \epsilon)} ($i\in[1,k]$) share common data points other than $P_s$, \ie $\bigsqcap_{i=1}^{k}$\sector{(P_s, P_{s+i}, \epsilon)} $\ne \{P_s\}$ \cite{Williams:Longest, Sklansky:Cone,Zhao:Sleeve}.

The point $Q$ may not belong to $\{P_{s}, P_{s+1},$ $\ldots, P_{s+k}\}$.
However, if $P_{s+i}$ ($1\le i\le k$) is chosen as $Q$, then
for any data point $P_{s+j}$ ($j \in [1, ... i]$), its perpendicular Euclidean distance to
line segment $\overline{P_sP_{s+i}}$ is not greater than the error bound $\epsilon$ if and only if $\bigsqcap_{j=1}^{i}$\sector{(P_s, P_{s+j}, \epsilon/2)} $\ne \{P_s\}$, as pointed out in \cite{Zhao:Sleeve}.

That is, {\em these \cia based algorithms can be easily adopted for trajectory compression using \ped although they have been overlooked by existing trajectory simplification studies}.
The \cia based algorithms run in $O(n)$ time and $O(1)$ space and are one-pass algorithms.

We next illustrate how the \cia based algorithms can be used for trajectory compression with an example.

\begin{example}
\label{exm-alg-sleeve}
Consider Figure~\ref{fig:sleeve}. An \cia based simplification algorithm takes as input a trajectory $\dddot{\mathcal{T}}[P_0, \ldots, P_{10}]$, and returns a simplified ployline consisting of two line segments $\vv{P_0P_4}$ and  $\vv{P_4P_{10}}$. Initially, point $P_0$ is the start point.

\sstab(1) Point $P_1$ is firstly read, and the sector \sector{(P_0, P_{1}, \epsilon/2)} of $P_1$ is created as shown in Figure~\ref{fig:sleeve}.(1).

\sstab(2) Then $P_2$ is read, and the sector \sector{(P_0, P_{2}, \epsilon/2)}  is created for $P_2$. The intersection of sectors \sector{(P_0, P_{1}, \epsilon/2)} and  \sector{(P_0, P_{2}, \epsilon/2)} contains data points other than $P_0$,  which has an up border line $P_0P_2^u$ and a low border line $P_0P_1^l$, as shown in Figure~\ref{fig:sleeve}.(2).

Similarly, points $P_3$ and $P_4$ are processed, as shown in Figures~\ref{fig:sleeve}.(3) and \ref{fig:sleeve}.(4), respectively.

\sstab(3) When point $P_5$ is read,  line segment $\vv{P_0P_4}$ is produced, and point $P_4$ becomes the start point, as $\bigsqcap_{i=1}^{4}$\sector{(P_0, P_{s+i}, \epsilon/2)} $\ne\{P_0\}$ and $\bigsqcap_{i=1}^{5}$\sector{(P_0, P_{s+i}, \epsilon/2)} $=\{P_0\}$ as shown in Figure~\ref{fig:sleeve}.(5).


\sstab(4) Points $P_5, \ldots, P_{10}$ are processed similarly one by one in order, and finally the algorithm outputs another line segment $\vv{P_4P_{10}}$ as shown in Figure~\ref{fig:sleeve}.(5). \eop
\end{example}


However, if we use \sed instead of \ped, then  ``the $k$ sectors \sector{(P_s, P_{s+i}, \epsilon)} ($i\in[1,k]$) sharing common data points other than $P_s$" cannot ensure ``there exists a data point $Q$ such that for any data point $P_{s+i}$ ($i \in [1, ... k]$), its synchronized Euclidean distance to directed line $\overline{P_sQ}$ is not greater than the error bound $\epsilon$''.
Hence, the \cia mechanism is \ped specific, and not applicable for \sed.







\subsection{Intersection Computation of Convex Polygons}
\label{subsec-cpi}

We also employ and revise a convex polygon intersection algorithm developed in~\cite{ORourke:Intersection}, whose basic idea is straightforward.
Assume \kwlog that the edges of polygons $\mathcal{R}_1$ and $\mathcal{R}_2$ are oriented counterclockwise, and $\vv{A} = (P_{s_A}, P_{e_A})$ and $\vv{B} = (P_{s_B}, P_{e_B})$ are two (directed) edges on $\mathcal{R}_2$ and $\mathcal{R}_1$, respectively.


The algorithm has $\vv{A}$ and $\vv{B}$ ``chasing'' one another, \ie moves $\vv{A}$ on $\mathcal{R}_2$ and $\vv{B}$ on $\mathcal{R}_1$ counter-clockwise step by step under certain rules, so that they meet at every crossing of $\mathcal{R}_1$ and $\mathcal{R}_2$.
%
The rules, called \emph{advance rules}, are carefully designed depending on geometric conditions of $\vv{A}$ and $\vv{B}$.
%If $\vv{B}$ ``aims toward" the line containing $\vv{A}$, but does not cross it, then we want to advance $\vv{B}$ in order to ``close in" on a possible intersection with A.
Let $\vv{A} \times \vv{B}$ be the cross product of (vectors) $\vv{A}$ and $\vv{B}$, and $\mathcal{H}(\vv{A})$ be the open half-plane to the left of $\vv{A}$, the rules are as follows:

\sstab {\em Rule (1)}: If $\vv{A} \times \vv{B} < 0$ and $P_{e_A} \not \in \mathcal{H}(\vv{B})$, or $\vv{A} \times \vv{B} \ge 0$ and $P_{e_B} \in \mathcal{H}(\vv{A})$, then $\vv{A}$ is advanced a step.

For example, in Figure~\ref{fig:c-poly-inter}.(1) and~\ref{fig:c-poly-inter}.(2), $\vv{A}$ moves forward a step as  $\vv{A} \times \vv{B} > 0$ and $P_{e_B} \in \mathcal{H}(\vv{A})$.

\sstab {\em Rule (2)}: If $\vv{A} \times \vv{B} \ge 0$ and $P_{e_B} \not \in \mathcal{H}(\vv{A})$, or $\vv{A} \times \vv{B} < 0$ and $P_{e_A} \in \mathcal{H}(\vv{B})$, then  $\vv{B}$ is advanced a step.


For example, in Figure~\ref{fig:c-poly-inter}.(6) and~\ref{fig:c-poly-inter}.(7), $\vv{B}$ moves forward a step as $\vv{A} \times \vv{B} < 0$ and $P_{e_A} \in \mathcal{H}(\vv{B})$.


\begin{figure}[tb!]
	\centering
	\includegraphics[scale=0.92]{figures/Fig-convex-poly-inter.png}
%	\vspace{-1ex}
	\caption{\small A running example of convex polygon intersection.}
	\vspace{-1ex}
	\label{fig:c-poly-inter}
\end{figure}


\stitle{Algorithm \cpia}. The complete algorithm is shown in Figure~\ref{alg:c-poly-inter}.
Given two convex polygons $\mathcal{R}_1$ and $\mathcal{R}_2$, algorithm \cpia first arbitrarily sets directed edge $\vv{A}$ on $\mathcal{R}_2$ and directed edge $\vv{B}$ on $\mathcal{R}_1$, respectively (line 1).
%
It then checks the intersection of edges $\vv{A}$ and $\vv{B}$. If $\vv{A}$ intersects $\vv{B}$ (line 3), then the algorithm checks for some special termination conditions (\eg if $\vv{A}$ and $\vv{B}$ are overlapped and, at the same time, polygons $\mathcal{R}_1$ and $\mathcal{R}_2$ are on the opposite sides of the overlapped edges, then the process is terminated) (line 4), and records the inner edge, which is a boundary segment of the intersection polygon (line 5).
After that, the algorithm moves on $\vv{A}$ or $\vv{B}$ one step under the advance rules (lines 6--11).
The above processes repeated, until both $\vv{A}$ and $\vv{B}$ completely cycle their polygons (line 12).
%
Next, the algorithm handles three special cases of the polygons $\mathcal{R}_1$ and $\mathcal{R}_2$, \ie $\mathcal{R}_1$ is inside of $\mathcal{R}_2$, $\mathcal{R}_2$ is inside of $\mathcal{R}_1$, and $\mathcal{R}_1 \bigsqcap \mathcal{R}_2 = \emptyset$ (line 13).
%
At last, it returns the intersection polygon (line 14).





%%%%%%%%%%%%%%%%%%%%%Baseline Algorithm
\begin{figure}[tb!]
	\begin{center}
		{\small
			\begin{minipage}{3.3in}
				\myhrule
				\vspace{-1ex}
				\mat{0ex}{
					{\bf Algorithm}~\cpia($\mathcal{R}_1$, $\mathcal{R}_2$) \\
					\bcc \hspace{1ex}\=  \SET $\vv{A}$ and $\vv{B}$ {arbitrarily} on $\mathcal{R}_2$ and $\mathcal{R}_1$, respectively;\\
					\icc \hspace{1ex}\= \Repeat \\
					\icc \>\hspace{2ex} \If $\vv{A} \bigsqcap \vv{B} \ne \emptyset$ \Then \\
					\icc \>\hspace{4ex} {Check for termination;} \\
					\icc \>\hspace{4ex} Update an inside flag;\\
					\icc \> \hspace{2ex} \If ($\vv{A}\times\vv{B} < 0$ \And $P_{e_A} \not \in \mathcal{H}(\vv{B})$) \Or \\
					\icc \> \hspace{4ex} ($\vv{A} \times \vv{B} \ge 0$ \And $P_{e_B} \in \mathcal{H}(\vv{A})$) \Then \\
					\icc \> \hspace{4ex} advance $\vv{A}$ one step; \\
					\icc \> \hspace{2ex} \ElseIf ($\vv{A} \times \vv{B} \ge 0$ \And $P_{e_B} \not \in \mathcal{H}(\vv{A})$) \Or\\
					\icc \> \hspace{4ex}  ($\vv{A} \times \vv{B} < 0$ \And $P_{e_A} \in \mathcal{H}(\vv{B})$) \Then \\
					\icc \> \hspace{4ex} advance $\vv{B}$ one step; \\
					\icc \hspace{0ex} \Until both $\vv{A}$ and $\vv{B}$ cycle their polygons \\
					\icc \hspace{0ex} \Handle $\mathcal{R}_1 \subset \mathcal{R}_2$ and $\mathcal{R}_2 \subset \mathcal{R}_1$ and $\mathcal{R}_1 \bigsqcap \mathcal{R}_2 = \emptyset$ cases; \\
					\icc \hspace{0ex} \Return $\mathcal{R}_1 \bigsqcap \mathcal{R}_2$.
				}
				\vspace{-2ex}
				\myhrule
			\end{minipage}
		}
	\end{center}
	\vspace{-2ex}
	\caption{\small Algorithm for convex polygon intersection~\cite{ORourke:Intersection}.}
	\label{alg:c-poly-inter}
	\vspace{-2ex}
\end{figure}
%%%%%%%%%%%%%%%%%%%%%%%%%%%%%%%%%%%%%





\begin{example}
Figure~\ref{fig:c-poly-inter} shows a running example of the convex polygon intersection algorithm \cpia.

\sstab(1) Initially, directed edges $\vv{A}$ and $\vv{B}$ are on polygons $\mathcal{R}_2$ and $\mathcal{R}_1$, respectively, such that $\vv{A} \bigsqcap \vv{B} = \{P_1\}$, \ie $\vv{A}$ and $\vv{B}$ intersect on point $P_1$, as shown in Figure~\ref{fig:c-poly-inter}.(1).

\sstab(2) Then, because $\vv{A} \times \vv{B} > 0$ and $P_{e_B} \in \mathcal{H}(\vv{A})$, by the advance rule (1), edge $\vv{A}$ moves on a step and makes $\vv{A} \bigsqcap \vv{B} = \emptyset$ as shown in Figure~\ref{fig:c-poly-inter}.(2).
After 7 steps of moving of edge $\vv{A}$ or $\vv{B}$, each by an advance rule, $\vv{A}$ and $\vv{B}$ intersect on $P_2$, as shown in Figure~\ref{fig:c-poly-inter}.(6).

\sstab(3) Next, because $\vv{A} \times \vv{B} < 0$ and $P_{e_A} \in \mathcal{H}(\vv{B})$, by the advance rule (2), edge $\vv{B}$ moves on a step, and makes $\vv{A} \bigsqcap \vv{B} = \emptyset$, as shown in Figure~\ref{fig:c-poly-inter}.(7).

\sstab(4) After 6 steps of moving of edge $\vv{B}$ or $\vv{A}$ one by one, both edges $\vv{A}$ and $\vv{B}$ have finished their cycles as shown in Figure~\ref{fig:c-poly-inter}.(8).

\sstab(5) The algorithm finally returns the intersection polygon as shown in Figure~\ref{fig:c-poly-inter}.(9). \eop
\end{example}


Algorithm  \cpia has a time complexity of $O(|\mathcal{R}_1| + |\mathcal{R}_2|)$, where $|\mathcal{R}|$ is the number of edges of polygon $\mathcal{R}$.
It is also worth pointing out that $|\mathcal{R}_1 \bigsqcap \mathcal{R}_2| \le (|\mathcal{R}_1| + |\mathcal{R}_2|)$.







\eat{%%%%%%%%%%%%%%%%%%%%%%%%%%%%%%%%%
%%%%%%%%%%%%%%%%%%%%%%%%%%%%%%%%%%%%%%%%%%%%%%%%%%%%%%%%%%%%%%%%%%%%%%%%%%%%%%%%%%%%%%%%%%%
%\vspace{-3ex}
\subsection{Problem Definition}
This paper focus on the \emph{min-$\#$ problem} \cite{Chan:Optimal, Imai:Optimal,Pavlidis:Segment} of trajectory simplification.
Given a trajectory \trajec{T}$[P_0, \dots, P_n]$ and a pre-specified constant $\epsilon$, a trajectory simplification algorithm $\mathcal{A}$ approximates \trajec{T} by $\overline{\mathcal{T}}[\mathcal{L}_0, \ldots , \mathcal{L}_m]$ ($0< m \le n$), where on
each of them the data points $[P_{s_i}, \dots, P_{e_i}]$ are approximated by a line segment $\mathcal{L}_i = \vv{P_{s_i}P_{e_i}}$ with the maximum error of  $ped(P_j, \mathcal{L}_i)$ or $sed(P_j, \mathcal{L}_i)$, $s_i < j<e_i$,  less than $\epsilon$.
The optimal methods that find the minimal $m$, having the time complexity of $O(n^2)$\cite{Chan:Optimal},
making it impractical for large inputs\cite{Heckbert:Survey}.
Hence, this paper evaluates the distinct sub-optimal methods.



\begin{figure}[tb!]
\label{fig:scope}
\centering
\includegraphics[scale=0.8]{figures/Fig-scope.png}
\vspace{-1ex}
\caption{Example \emph{scopes} of Synchronous points. The Synchronous point $P'$ has (1) a circle scope  when using \sed, and (2) a rectangle scope when using \ded.}

\vspace{-2ex}
\end{figure}

%Given a line $\overline{P_sP_e}$, a Synchronous point $P'$ on line $\overline{P_sP_e}$ and a error bound $\epsilon_s$ of ~\sed (or error bounds $\epsilon_p$ and $\epsilon_r$ of \ded), all points potentially be compressed to a Synchronous point $P'$ forms a \emph{scope} of $P'$ (Fig.\ref{fig:scope}). The \emph{scope} of $P'$ in \sed is a circle around $P'$ whose radius is less than the \sed error bound $\epsilon_s$, and the \emph{scope} of $P'$ in \ded is a rectangle, paralleling to the line $\overline{P_sP_e}$, taking $P'$ as the central point and whose height and width are less than the \ded error bounds $\epsilon_p$ and $\epsilon_r$ respectively. With he help of \ded, one can set the value of $\epsilon_p$ and $\epsilon_r$ separately according to varied application requirements, \eg a smaller $\epsilon_p$ to limit the perpendicular deviation of the car to the road while a bigger $\epsilon_r$ to ensure a better compression ratio.
}%%%%%%%%%%%%%%%%%%%%%%%%%%%%%%%%%
