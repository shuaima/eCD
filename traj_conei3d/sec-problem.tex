\section{Preliminaries}
In this section, we introduce basic concepts and cone intersection based one-pass algorithms using \textcolor[rgb]{0.00,0.07,1.00}{\ped} for trajectory compression.
%\textcolor[rgb]{1.00,0.00,0.00}{A discussion and summary is presented in the last.}


%%%%%%%%%%%%%%%%%%%%%%%%%%%%%%%%%%%%%%%%%%%%%%%%%%%%%%%%%
% Cone Intersection
%%%%%%%%%%%%%%%%%%%%%%%%%%%%%%%%%%%%%%%%%%%%%%%%%%%%%%%%%
\begin{figure*}[tb!]
\centering
\includegraphics[scale=0.8]{figures/Fig-sleeve.png}
\vspace{-2.5ex}
\caption{\small The trajectory $\dddot{\mathcal{T}}[P_0, \ldots, P_{10}]$ is compressed by the \conei algorithm to two line segments.}
\vspace{-2ex}
\label{fig:sleeve}
\end{figure*}



\subsection{Basic Notations}

We first introduce basic notations.

\stitle{Points ($P$)}. A data point is defined as a triple $P(x, y, t)$, which represents that a moving object is located at {\em longitude} $x$ and {\em latitude} $y$ at {\em time} $t$. Note that data points can be viewed as points in a three-dimension Euclidean space.

\stitle{Trajectories ($\dddot{\mathcal{T}}$)}. A trajectory $\dddot{\mathcal{T}}[P_0, \ldots, P_n]$ is a sequence of data points in a monotonically increasing order of their associated time values (\ie $P_i.t < P_j.t$ for any $0\le i<j\le n$). Intuitively, a trajectory is the path (or track) that a moving object follows through space as a function of time~\cite{physics-trajectory}.


\stitle{Directed line segments ($\mathcal{L}$)}. A directed line segment (or line segment for simplicity) $\mathcal{L}$ is defined as $\vv{P_{s}P_{e}}$, which represents the closed line segment that connects the start point $P_s$ and the end point $P_e$.
Note that here $P_s$ or $P_e$ may not be a point in a trajectory $\dddot{\mathcal{T}}$, and hence, we also use notation $\mathcal{R}$ instead of $\mathcal{L}$ when both $P_s$ and $P_e$ belong to $\dddot{\mathcal{T}}$.

We also use $|\mathcal{L}|$ and $\mathcal{L}.\theta\in [0, 2\pi)$ to denote the length of a directed line segment $\mathcal{L}$, and its angle with the $x$-axis of the coordinate system $(x, y)$, where $x$ and $y$ are the longitude and latitude, respectively.
That is, a directed line segment $\mathcal{L}$ = $\vv{P_{s}P_{e}}$ can be treated as a triple $(P_s, |\mathcal{L}|, \mathcal{L}.\theta)$.

\stitle{Piecewise line representation ($\overline{\mathcal{T}}$)}. A piece-wise line representation of a trajectory $\dddot{\mathcal{T}}[P_0, \ldots, P_n]$ is denoted as $\overline{\mathcal{T}}[\mathcal{L}_0, \ldots , \mathcal{L}_m]$ ($0< m \le n$), a sequence of continuous directed line segments $\mathcal{L}_{i}$ = $\vv{P_{s_i}P_{e_i}}$ of $\dddot{\mathcal{T}}$ ($i\in[0,m]$)  such that $\mathcal{L}_{0}.P_{s_0} = P_0$, $\mathcal{L}_{m}.P_{e_m} = P_n$ and  $\mathcal{L}_{i}.P_{e_i}$ = $\mathcal{L}_{i+1}.P_{s_{i+1}}$ for all $i\in[0, m-1]$. Note that each directed line segment in $\overline{\mathcal{T}}$ essentially represents a continuous sequence of data points in $\dddot{\mathcal{T}}$.

%%%%%%%%%%%%%%%%%%%%%%%%%%%%%%%%%%%%%%%%%%%%%%%%%%%%%%%%%%%%%%%%%%%%%%%%%%%%%%
\eat{
\subsubsection{Notations of error metrics}

{For line simplification, there are distance based and shape based error metrics\cite{Shi:Survey} that measure the errors between the original trajectory and the simplified trajectory.
However, for trajectory simplification, the distance based metrics are definitely the distinct metrics.}

\stitle{Included angles ($\angle$)}. Given two directed line segments $\mathcal{L}_1$ = $\vv{P_{s}P_{e_1}}$ and $\mathcal{L}_2$ = $\vv{P_{s}P_{e_2}}$ with the same start point $P_s$, the included angle from $\mathcal{L}_1$ to $\mathcal{L}_2$, denoted as $\angle(\mathcal{L}_1, \mathcal{L}_2)$,  is $\mathcal{L}_2.\theta - \mathcal{L}_1.\theta$. For convenience, we also represent the included angle  $\angle(\mathcal{L}_1, \mathcal{L}_2)$ as $\angle{P_{e_1}P_sP_{e_2}}$.
}
%%%%%%%%%%%%%%%%%%%%%%%%%%%%%%%%%%%%%%%%%%%%%%%%%%%%%%%%%%%%%%%%%%%%%%%%%%%%%%


\stitle{Perpendicular Euclidean Distances (\ped)}. Given a point $P$ and a directed line segment $\mathcal{L}$ = $\vv{P_{s}P_{e}}$, the distance of $P$ to $\mathcal{L}$, denoted as $ped(P, \mathcal{L})$, is the Euclidean distance from $P$ to $\vv{P_sP_e}$, commonly adopted by most existing trajectory simplification methods, \eg~\cite{Douglas:Peucker, Hershberger:Speeding, Keogh:online, Chen:Fast, Liu:BQS, Williams:Longest, Sklansky:Cone, Dunham:Cone, Zhao:Sleeve, lin:operb}.


\stitle{Synchronized points}. Given a sub trajectory $\dddot{\mathcal{T}}_s[P_s, \ldots, P_e]$, the synchronized point $P'$ of a data point  $P(x, y, t) \in \dddot{\mathcal{T}}_s$, ~\wrt the line segment $\vv{P_sP_e}$ is $P'(x', y', t)$ such that $x' = P_s.x +  \frac{P.t-P_s.t}{P_e.t-P_s.t}(P_e.x - P_s.x)$ and $y' = P_s.y +  \frac{P.t-P_s.t}{P_e.t-P_s.t}(P_e.y - P_s.y)$.


\stitle{Synchronous Euclidean Distances (\sed)}. The \sed of a data point $P$ to a line segment $\mathcal{L} = \vv{P_{s}P_{e}}$, denoted as $sed(P, \mathcal{L})$, is $|\vv{PP'}|$, is the perpendicular Euclidean distance from $P$ to the synchronized data point $P'$ \wrt $\mathcal{L}$. %, adopted by recent trajectory simplification methods~\cite{Meratnia:Spatiotemporal, Potamias:Sampling, Chen:Fast, Muckell:Compression, Popa:Spatio}.

%\stitle{Radial Euclidean Distances ($\red$)}. The \red of $P$ \wrt $\mathcal{L}$, denoted as $red(P, \mathcal{L})$, is the distance from the perpendicular point $P^*$ of $P$ to the Synchronous point $P'$ of $P$.


%\stitle{Dual Euclidean distances $(\ped, ~\red)$}. A \ded is a two-tuples $(\ped, ~\red)$, whereas, \ped and \red either has its error bound ($\epsilon_p$ of \ped and $\epsilon_r$ of \red), and is checked separately. Note that the distance checking approach of the original \lsa algorithms, whose \red error bound $\epsilon_r$ is fixed with $\infty$, is a special case of \ded.

We next illustrate these notations with examples.

\begin{example}
\label{exm-notations}
Consider Figure~\ref{fig:notations}, in which

\ni(1) $\dddot{\mathcal{T}}[P_0$, $\ldots, P_{10}]$ is a trajectory having eleven data points,

\ni (2) the set of two continuous line segments $\{\vv{P_0P_4}$, $\vv{P_4P_{10}}$\} (Left) and the set of four continuous line segments $\{\vv{P_0P_2}$, $\vv{P_2P_4}$, $\vv{P_4P_7}$, $\vv{P_7P_{10}}$\} (Right) are two piecewise line representations of trajectory $\dddot{\mathcal{T}}$,

\ni (3) $ped(P_4, \vv{P_0P_{10}})=|\vv{P_4P^*_4}|$, where $P^*_4$ is the perpendicular point of $P_4$ \wrt line segment $\vv{P_0P_{10}}$, and

\ni  (4) $sed(P_4, \vv{P_0P_{10}})= |\vv{P_4P'_4}|$, $sed(P_2, \vv{P_0P_{4}})= |\vv{P_2P'_2}|$ and $sed(P_7, \vv{P_4P_{10}})$ $=$ $|\vv{P_7P'_7}|$,
where points $P'_4$, $P'_2$ and $P'_7$ are the synchronized points of points $P_4$, $P_2$ and $P_7$ \wrt line segments $\vv{P_0P_{10}}$, $\vv{P_0P_{4}}$ and $\vv{P_4P_{10}}$, respectively.
\end{example}







\subsection{Cone Intersection Algorithms using \ped}


The Cone Intersection (\cia) algorithm \cite{Williams:Longest, Sklansky:Cone} was developed in disciplines of graphic and pattern recognition in the late 1970s, for the approximation of arbitrary planar curves by linear segments or finding a polygonal approximation of a set of input data points in a 2D-space. The \sleeve algorithm \cite{Zhao:Sleeve} in the cartographic discipline essentially indeed applies the same idea as the cone intersection algorithm.
Further, \cite{Dunham:Cone}  optimized algorithm \cia by considering the distance between a potential end point and the initial point of a line segment. It is worth pointing out that all these algorithms use the perpendicular Euclidean distances.

%Note that these notable algorithms are not introduced to the discipline of trajectory compression.
%though these notable algorithms are famous in disciplines of graphic and pattern recognition,

Given a sequence of data points $[P_{s}, P_{s+1}, \ldots, P_{s+k}]$ and an error bound $\epsilon$, the \cia algorithms process the input points one by one in order and produce a simplified poly-line.  Instead of using the distance threshold $\epsilon$ directly, \cia converts the distance tolerance into a variable angle tolerance for testing the points.

For the start point $P_s$ and any point $P_{s+i}$, $1\le i\le k$, there are two different lines $\vv{P_sP^u_{s+i}}$ and $\vv{P_sP^l_{s+i}}$, satisfying $ped(P_i, \vv{P_sP^u_{s+i}}) = \epsilon$, $ped(P_i, \vv{P_sP^l_{s+i}}) = \epsilon$ and $\vv{P_sP^l_{s+i}}.\theta < \vv{P_sP^u_{s+i}}.\theta$. Indeed, they forms a \emph{cone} $\mathcal{C}$($P_s$, $P_{s+i}$, $\epsilon$) (also named a \emph{Sector} in the \sleeve algorithm) taking $P_s$ as the center point and $\vv{P_sP^u_{s+i}}$ and $\vv{P_sP^l_{s+i}}$ as the border lines.
% , where $P_s$ is the center point, and $\vv{P_sP^l_{s+i}}$ and $\vv{P_sP^l_{s+i}}$ are two bound lines.
Then there exists a point $Q$ such that all points $P_{s+i}$, $i \in [1, ... k]$, have perpendicular distances to
line $\mathcal{L}(P_s,Q)$ within the given \ped tolerance $\epsilon$ if and only if $\bigcap_{i=1}^{k}\mathcal{C}(P_s, P_{s+i}, \epsilon) \ne \phi$.
Note that in this case, point $Q$ may not belong to $[P_{s}, P_{s+1}, \ldots, P_{s+k}]$.
However, if the last point $P_{s+k}$ is chosen as the end point of the approximating line segment, then a narrow \emph{cone} with a distance tolerance $\epsilon/2$ could be employed to ensure the algorithm be \ped error bounded by $\epsilon$ \cite{Zhao:Sleeve}.

Algorithm \cia runs in $O(n)$ time and takes $O(1)$ space.

Note that it runs in a 2D plane, and the \sed is not supported in the algorithm.



\begin{example}
\label{exm-alg-sleeve}
Figure~\ref{fig:sleeve} is a running example of the narrow \emph{cone} taking a distance tolerance $\epsilon/2$. At the beginning, $P_0$ is the first start point.

\ni (1) The point $P_1$ is read, and the narrow cone $\mathcal{C}$($P_0$, $P_{1}$, $\epsilon/2$) of $P_1$ is created.

\ni  (2) The narrow cone $\mathcal{C}$($P_0$, $P_{2}$, $\epsilon/2$) is created for $P_2$. The intersection of $\mathcal{C}$($P_0$, $P_{1}$, $\epsilon/2$) and $\mathcal{C}$($P_0$, $P_{2}$, $\epsilon/2$) is not $\phi$. It has a up border line $P_0P_2^u$ and a low border line $P_0P_1^l$.

\ni  (3) The point $P_3$ is read, and the border lines of the intersection cone are update to $P_0P_3^u$ and $P_0P_3^l$.

\ni  (4) The point $P_4$ is processed as above.

\ni  (5) The point $P_5$ is read. Because $\bigcap_{i=1}^{4}\mathcal{C}(P_0, P_{s+i}, \epsilon/2) \ne \phi$ and $\bigcap_{i=1}^{5}\mathcal{C}(P_0, P_{s+i}, \epsilon/2) = \phi$, $\vv{P_0P_4}$ is output and $P_4$ becomes the start point of the next section.

At last, the algorithm outputs two continuous line segments, \ie $\vv{P_0P_4}$ and $\vv{P_4P_{10}}$.
\end{example}




\subsection{Intersection of convex polygons}

In the work, we also implement a fast polygon intersect algorithm which is a customization of the distinct convex polygon algorithm presented in ~\cite{ORourke:Intersection}. We next introduce the algorithm of~\cite{ORourke:Intersection}, whose structure is shown in Figure~\ref{alg:c-poly-inter}.

%%%%%%%%%%%%%%%%%%%%%Baseline Algorithm
\begin{figure}[tb!]
\begin{center}
{\small
\begin{minipage}{3.36in}
\myhrule
\vspace{-1ex}
\mat{0ex}{
	{\bf Algorithm} ~$ConvexPolyInter$($\mathcal{G}_1$, $\mathcal{G}_2$) \\
	\bcc \hspace{1ex}\=  Choose $A$ and $B$ {arbitrarily} \\
	\icc \hspace{1ex}\= Repeat \\
	\icc \>\hspace{3ex} If $A$ intersects $B$ Then \\
	\icc \>\hspace{6ex} {Check for termination.} \\
	\icc \>\hspace{6ex} Update an inside flag. \\
	\icc \> \hspace{3ex} Moves on either $A$ or $B$.\\
	\icc \hspace{1ex} Until both $A$ and $B$ cycle their polygons \\
	\icc \hspace{1ex} Handle $G_1 \subset G_2$ and $G_2 \subset G_1$ and $G_1 \bigcap G_2 = \phi$ cases \\
    \icc \hspace{1ex} Return $G_1 \bigcap G_2$
}
\vspace{-2ex}
\myhrule
\end{minipage}
}
\end{center}
\vspace{-2ex}
\caption{\small Intersection of Convex polygons.}
\label{alg:c-poly-inter}
\vspace{-2ex}
\end{figure}
%%%%%%%%%%%%%%%%%%%%%%%%%%%%%%%%%%%%%

The basic idea of the algorithm is straightforward. Assume the boundaries of the two polygons $G_1$ and $G_2$ are oriented counterclockwise, and let $A$ and $B$ be directed edges on each. The algorithm has $A$ and $B$ ``chasing" one another, step by step on the polygons, so that they meet at every crossing of $G_1$ and $G_2$.
%
(1) During each step, it checks the intersection of $A$ and $B$. If $A$ intersects $B$ (line 3), then the algorithm checks for some special termination conditions (\eg if $A$ and $B$ are overlapped, and at the same time, $G_1$ and $G_2$ are on the opposite sides of the overlapped line segment, then the process is terminated) (line 4), and records the inner segment, which is a boundary segment of the intersection polygon (line 5).
%
(2) The algorithm moves on $A$ or $B$ one step under some carefully designed rules, depending on geometric conditions of $A$ and $B$ (line 6).
The above processes repeated, until both $A$ and $B$ cycle their polygons (line 7).
%
After that, the algorithm handles three special cases of the two polygons, \ie $G_1 \subset G_2$, $G_2 \subset G_1$ and $G_1 \bigcap G_2 = \phi$ cases (line 8).
%
At last, it returns the intersection polygon (line 9).

The algorithm has a time complexity of $O(|G_1| + |G_2|)$, where $|G|$ is the number of edges of polygon $G$.





\begin{example}
Figure~\ref{fig:c-poly-inter} is a running example of the convex polygon intersection algorithm. 

\ni (1) initially, directed edges $A$ and $B$ are on polygons $G_2$ and $G_1$ separately;

\ni (2) $A \bigcap B = \phi$. $A$ moves forward a step;

\ni (3) $B$ moves forward a step;

\ni (4) after some steps of moving, $A$ and $B$ intersect on $P_1$;

\ni (5) after some steps of moving, $A$ and $B$ intersect on $P_2$;

\ni (6) both $A$ and $B$ cycle their polygons;

At last, it returns the intersection polygon.
\end{example}


\begin{figure}[tb!]
\centering
\includegraphics[scale=0.88]{figures/Fig-convex-poly-inter.png}
\vspace{-1ex}
\caption{\small A running example of convex polygons intersection.}
\vspace{-2ex}
\label{fig:c-poly-inter}
\end{figure}





\eat{%%%%%%%%%%%%%%%%%%%%%%%%%%%%%%%%%
%%%%%%%%%%%%%%%%%%%%%%%%%%%%%%%%%%%%%%%%%%%%%%%%%%%%%%%%%%%%%%%%%%%%%%%%%%%%%%%%%%%%%%%%%%%
%\vspace{-3ex}
\subsection{Problem Definition}
This paper focus on the \emph{min-$\#$ problem} \cite{Chan:Optimal, Imai:Optimal,Pavlidis:Segment} of trajectory simplification.
Given a trajectory \trajec{T}$[P_0, \dots, P_n]$ and a pre-specified constant $\epsilon$, a trajectory simplification algorithm $\mathcal{A}$ approximates \trajec{T} by $\overline{\mathcal{T}}[\mathcal{L}_0, \ldots , \mathcal{L}_m]$ ($0< m \le n$), where on
each of them the data points $[P_{s_i}, \dots, P_{e_i}]$ are approximated by a line segment $\mathcal{L}_i = \vv{P_{s_i}P_{e_i}}$ with the maximum error of  $ped(P_j, \mathcal{L}_i)$ or $sed(P_j, \mathcal{L}_i)$, $s_i < j<e_i$,  less than $\epsilon$.
The optimal methods that find the minimal $m$, having the time complexity of $O(n^2)$\cite{Chan:Optimal},
making it impractical for large inputs\cite{Heckbert:Survey}.
Hence, this paper evaluates the distinct sub-optimal methods.



\begin{figure}[tb!]
\label{fig:scope}
\centering
\includegraphics[scale=0.8]{figures/Fig-scope.png}
\vspace{-1ex}
\caption{Example \emph{scopes} of Synchronous points. The Synchronous point $P'$ has (1) a circle scope  when using \sed, and (2) a rectangle scope when using \ded.}

\vspace{-2ex}
\end{figure}

%Given a line $\overline{P_sP_e}$, a Synchronous point $P'$ on line $\overline{P_sP_e}$ and a error bound $\epsilon_s$ of ~\sed (or error bounds $\epsilon_p$ and $\epsilon_r$ of \ded), all points potentially be compressed to a Synchronous point $P'$ forms a \emph{scope} of $P'$ (Fig.\ref{fig:scope}). The \emph{scope} of $P'$ in \sed is a circle around $P'$ whose radius is less than the \sed error bound $\epsilon_s$, and the \emph{scope} of $P'$ in \ded is a rectangle, paralleling to the line $\overline{P_sP_e}$, taking $P'$ as the central point and whose height and width are less than the \ded error bounds $\epsilon_p$ and $\epsilon_r$ respectively. With he help of \ded, one can set the value of $\epsilon_p$ and $\epsilon_r$ separately according to varied application requirements, \eg a smaller $\epsilon_p$ to limit the perpendicular deviation of the car to the road while a bigger $\epsilon_r$ to ensure a better compression ratio.
}%%%%%%%%%%%%%%%%%%%%%%%%%%%%%%%%%
