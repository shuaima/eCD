\section{Local Synchronous Distance Checking}
\label{sec-localcheck}

%In this section, we first analyze the properties of synchronized points and synchronous Euclidean distances, then we present a local one-pass \sed checking method by extending the cone intersection for \ped checking in Section~\ref{sub-ci-ped}.
%Finally, we provide a one-pass, error-bounded and \sed enabled trajectory simplification algorithm.

In this section, we develop a local synchronous distance checking approach such that each point in a trajectory is checked only once in $O(1)$ time during the entire process of trajectory simplification, by substantially extending the \textit{sector intersection} method in Section~\ref{sub-ci-ped} from a 2D space to a Spatio-Temporal 3D space, which lays down the key for the one-pass trajectory simplification algorithms using \sed (Section~\ref{sec-alg}).

We consider a sub-trajectory $\dddot{\mathcal{T}}_s[P_s, \ldots, P_{s+k}]$, an error bound $\epsilon$, and a 3D Cartesian coordinate system whose origin, $x$-axis, $y$-axis and $t$-axis  are $P_s$, longitude, latitude and time, respectively.

\subsection{Spatio-Temporal Cone Intersection}

We first present the \textit{spatio-temporal cone intersection} method in a 3D Cartesian coordinate system, which extends the \textit{sector intersection} method~\cite{Williams:Longest, Sklansky:Cone, Zhao:Sleeve}. %in a 2D Cartesian coordinate system.

%Recall that the \sed of point $P_i$ to a directed line segment $\mathcal{L} = \vv{P_{s}P_{e}}$, denoted as $sed(P_i, \mathcal{L})$, is $|P_iP_i'|$, the distance from $P_i$ to its synchronized point $P_i' (x_i', y_i', t_i)$ \wrt $\mathcal{L}$.




\stitle{Synchronous Circles (\circle{})}. The synchronous circle of a data point $P_{s+i}$ ($1\le i\le k$) in $\dddot{\mathcal{T}}_s$ \wrt an error bound $\epsilon$, denoted as \circle{(P_{s+i}, \epsilon)}, or \circle{_{s+i}} in short, is a circle on the plane $P.t-P_{s+i}.t = 0$ such that $P_{s+i}$ is its center and $\epsilon$ is its radius.

Figure~\ref{fig:cis} shows two synchronous circles, \circle{(P_{s+i}, \epsilon)} of point $P_{s+i}$ and \circle{(P_{s+k}, \epsilon)} of point $P_{s+k}$.
It is easy to know that for any point in the area of a circle \circle{(P_{s+i}, \epsilon)}, its distance to $P_{s+i}$ is no greater than $\epsilon$.


%Note that in this definition, a \emph{synchronous circle} $\mathcal{O}(P_i, \epsilon)$ is only defined by a central point $P_i$ and a constant $\epsilon$. Indeed, it is nothing to do with any start point $P_s$ or end point $P_e$.



\stitle{Spatio-temporal cones (\cone{})}. The spatio-temporal cone (or simply \textit{cone}) of a data point $P_{s+i}$ ($1\le i\le k$) in $\dddot{\mathcal{T}}_s$ \wrt a point $P_s$ and an error bound $\epsilon$, denoted as \cone{(P_s, \mathcal{O}(P_{s+i}, \epsilon))}, or \cone{_{s+i}} in short, is an oblique circular cone such that point $P_s$ is its apex and the synchronous circle $\mathcal{O}(P_{s+i}, \epsilon)$ of point $P_{s+i}$ is its base.

Figure~\ref{fig:cis} also illustrates two example spatio-temporal cones: \cone{(P_s, \mathcal{O}(P_{s+i}, \epsilon))} {(purple)} and \cone{(P_s, \mathcal{O}(P_{s+k}, \epsilon))} (red), with the same apex $P_s$ and error bound $\epsilon$.



\begin{figure}[tb!]
	\centering
	\includegraphics[scale=0.66]{figures/Fig-CIS.png}
%	\vspace{-1ex}
	\caption{\small Examples of spatio-temporal cones in a 3D Cartesian coordinate system, where (1) $P_s$, $P_{s+i}$ and $P_{s+k}$ are three points, (2) \circle{_{s+i}} and \circle{_{s+k}} are two synchronous circles, (3) \cone{_{s+i}} and \cone{_{s+k}} are two spatio-temporal cones, (4) $Q$ is a point in synchronous circle \circle{_{s+k}}, and (5) $P'_{s+i}$ is the intersection point of line $\overline{P_sQ}$ and synchronous circle \circle{_{s+i}}.}
	%\vspace{-1ex}
	\label{fig:cis}
\end{figure}
%Note that the projection circle $\mathcal{O}(P^i_{i+1}, r^i_{i+1})$ is on the plane $t=t_i$.

%\begin{figure}[tb!]
%\centering
%\includegraphics[scale=0.56]{figures/Fig-SEDin3D.png}
%\vspace{-1ex}
%\caption{\small Trajectory simplification in the Spatio-temporal space. Note the synchronous circle $\mathcal{O}(P_i, \epsilon)$ is on plane $t=t_i$.} %
%\vspace{-3ex}
%\label{fig:sed3d}
%\end{figure}




%\textcolor[rgb]{1.00,0.00,0.00}{The \sed is demonstrated in 2D space in the previews works~\cite{Meratnia:Spatiotemporal, Chen:Fast, Muckell:Compression, Popa:Spatio}. However,  the trajectory and the \sed are also capable presented in 3D space.}

\begin{prop}
\label{prop-3d-syn-point}
Given a sub-trajectory $[P_s, \ldots, P_{s+k}]$ and a point $Q$ in the area of synchronous circle \circle{(P_{s+k}, \epsilon)}, the intersection point $P'_{s+i}$ of the directed line segment $\vv{P_sQ}$ and the plane $P.t - P_{s+i}.t = 0$ is the synchronized point of $P_{s+i}$ ($1\le i\le k$) \wrt  $\vv{P_sQ}$, and the distance $|\vv{P_{s+i}P'_{s+i}}|$ from $P_{s+i}$ to $P'_{s+i}$ is the synchronous distance of $P_{s+i}$ to $\vv{P_sQ}$.
%in the spatio-temporal space.
\end{prop}

\begin{proof}\ It suffices to show that $P'_{s+i}$ is indeed a synchronized point $P_{s+i}$ \wrt $\vv{P_sQ}$.
%
The intersection point $P'_{s+i}$ satisfies that $P'_{s+i}.t = P_{s+i}.t$ and
$\frac{P'_{s+i}.t - P_{s}.t}{Q.t - P_{s}.t}$ = $\frac{P_{s+i}.t - P_{s}.t}{Q.t - P_{s}.t}$  =
$\frac{|\vv{P_sP'_{s+i}}|}{|\vv{P_sQ}|}$ =
$\frac{P'_{s+i}.x - P_{s}.x}{Q.x - P_{s}.x}$ = $\frac{P'_{s+i}.y - P_{s}.y}{Q.y - P_{s}.y}$.
%
Hence, by the definition of synchronized points, we have the conclusion. \eop
\end{proof}






\begin{prop}
\label{prop-3d-ci}
Given a sub-trajectory $[P_s,...,P_{s+k}]$ and an error bound $\epsilon$, there exists a point $Q$ such that $Q.t = P_{s+k}.t$ and $sed(P_{s+i}, \vv{P_sQ})\le \epsilon$ for each $i \in [1,k]$ if and only if $\bigsqcap_{i=1}^{k}$\cone{(P_s, \mathcal{O}(P_{s+i}, \epsilon))} $\ne \{P_s\}$.
\end{prop}

\begin{proof}\  
Let $P'_{s+i}$ ($i\in[1, k]$) be the intersection point of line segment $\vv{P_sQ}$ and the plane $P.t - P_{s+i}.t$ = $0$.
By Proposition~\ref{prop-3d-syn-point}, $P'_{s+i}$ is the synchronized point of $P_{s+i}$ \wrt $\vv{P_sQ}$.

Assume first that $\bigsqcap_{i=1}^{k}$\cone{(P_s, \mathcal{O}(P_{s+i}, \epsilon))} $\ne \{P_s\}$. Then there must exist a point $Q $ in the area of the  synchronous circle \circle{(P_{s+k}, \epsilon)} such that $\vv{P_sQ}$ passes through all the cones \cone{(P_s, \mathcal{O}(P_{s+i}, \epsilon))} $i\in[1, k]$. Hence,  $Q.t = P_{s+k}.t$.
We also have $sed(P_{s+i}, \vv{P_sQ}) = |\vv{P'_{s+i}P_{s+i}}| \le \epsilon$ for each $i \in [1, k]$  since $P'_{s+i}$  is in the area of circle  \circle{(P_{s+i}, \epsilon)}.

Conversely, assume that there exists a point $Q$ such that $Q.t = P_{s+k}.t$ and $sed(P_{s+i}, \vv{P_sQ})\le\epsilon$ for all $P_{s+i}$ ($i \in [1,k]$). Then $|\vv{P'_{s+i}P_{s+i}}| \le \epsilon$ for all $i \in [1, k]$. Hence, we have  $\bigsqcap_{i=1}^{k}$\cone{(P_s, \mathcal{O}(P_{s+i}, \epsilon))} $\ne \{P_s\}$. \eop
\end{proof}

By Proposition~\ref{prop-3d-ci}, we now have a spatio-temporal cone intersection method in a 3D Cartesian coordinate system, which extends the sector intersection method~\cite{Williams:Longest, Sklansky:Cone, Zhao:Sleeve} {from a 2D space to a Spatio-Temporal 3D space}.


\subsection{Circle Intersection}
\label{subsec-ProjectionCircle}

For spatio-temporal cones with the same apex $P_s$, the checking of their intersection can be computed by a much simpler way, \ie the checking of intersection of cone projection circles on a plane, as follows.

\stitle{Cone projection circles}. The projection of a cone \cone{(P_s, \mathcal{O}(P_{s+i}, \epsilon))} on a plane $P.t- t_c = 0$ ($t_c > P_s.t$) is a circle \pcircle{(P^c_{s+i}, r^c_{s+i})}, or \pcircle{_{s+i}} in short, such that
%
(1) $P^c_{s+i}.x = P_s.x +  c\cdot(P_{s+i}.x- P_{s}.x)$,
%
(2) $P^c_{s+i}.y = P_s.y +  c\cdot(P_{s+i}.y- P_{s}.y)$,
%
(3) $P^c_{s+i}.t = t_c$ and
%
(4) $r^c_{s+i} =c\cdot\epsilon$, where $c=\frac{t_c - P_s.t}{P_{s+i}.t - P_s.t}$ .

\vspace{.5ex}


In Figure~\ref{fig:pcircle}, the green dashed circles \pcircle{(P^c_{s+i}, r^c_{s+i})} and \pcircle{(P^c_{s+k}, r^c_{s+k})} on plane ``$P.t-t_{c}=0$" are the projection circles of cones \cone{(P_s, \mathcal{O}(P_{s+i}, \epsilon))} and \cone{(P_s, \mathcal{O}(P_{s+k}, \epsilon))} on the plane.




\begin{prop}
\label{prop-circle-intersection}
Given a sub-trajectory $[P_s,\ldots, P_{s+k}]$, an error bound $\epsilon$, and any $t_c > P_s.t$, there exists a point $Q$ such that $Q.t = P_{s+k}.t$ and $sed(P_{s+i}, \vv{P_sQ})\le \epsilon$ for all points $P_{s+i}$ ($i \in [1,k]$) if and only if $\bigsqcap_{i=1}^{k}$\pcircle{(P^c_{s+i}, r^c_{s+i})} $\ne \emptyset$.
\end{prop}


\begin{proof}\ 
By Proposition~\ref{prop-3d-ci}, it suffices to show that $\bigsqcap_{i=1}^{k}$ \pcircle{(P^c_{s+i}, r^c_{s+i})} $\ne \emptyset$ if and only if $\bigsqcap_{i=1}^{k}$\cone{(P_s, \mathcal{O}(P_{s+i}, \epsilon))}$\ne \{P_s\}$, which is obvious. Hence, we have the conclusion. \eop
\end{proof}

%These oblique circular cones have the same vertex $P_s$ and their bottom circles are parallel, thus,
% ``$\bigcap_{i=1}^{e}{\mathcal{O}(P^c_i, r^c_i)} \ne \phi$, $t_c > t_s$" is naturally equivalent to ``$\bigcap_{i=s+1}^{e}{\mathcal{C}(P_s, P_i, \epsilon)} \ne \{P_s\}$".
%For example, in Figure~\ref{fig:sed3d}, $\mathcal{O}(P^i_{i+1}, r^i_{i+1})$ is the projection circle of synchronous circle $\mathcal{O}(P_{i+1}, \epsilon)$ on the plane $t=t_i$, then ``${\mathcal{O}(P^i_{i+1}, r^i_{i+1})} \bigcap{\mathcal{O}(P_i, \epsilon)} \ne \emptyset$" is equal to ``$\mathcal{C}(P_s, P_i, \epsilon) \bigcap {\mathcal{C}(P_s, P_{i+1}, \epsilon)} \ne \{P_s\}$". Hence, by Theorem~\ref{prop-3d-ci}, we have the conclusion.


Proposition~\ref{prop-circle-intersection} tells us that the intersection checking of spatio-temporal cones can be reduced to simply check the intersection of cone projection circles on a plane. %\textcolor{blue}{which can be implemented in ${O(n\log n)}$ time ~\cite{Shamos:Circle}.}

\begin{figure}[tb!]
	\centering
	\includegraphics[scale=0.7]{figures/Fig-project-circle.png}
%	\vspace{-1ex}
	\caption{\small Cone projection circles.}
	\vspace{-1ex}
	\label{fig:pcircle}
\end{figure}
%\vspace{1ex}


\subsection{Inscribed Regular Polygon Intersection}
\label{subsec-RPI}

Finding the common intersection of $n$ circles on a plane has a time complexity of ${O(n\log n)}$~\cite{Shamos:Circle}, which cannot be used for designing one-pass trajectory simplification algorithms using \sed.
%
However, we can approximate a circle with its $m$-edge inscribed regular polygon, whose intersection can be computed more efficiently.




\stitle{Inscribed regular polygons ($\mathcal{R}$)}.
Given a cone projection circle \pcircle{(P, r)}, its inscribed $m$-edge regular polygon is denoted as $\mathcal{R}(V, E)$,
where (1) $V=\{v_1, \ldots, v_{m}\}$ is the set of vertexes that are defined by a polar coordinate system, whose origin is the center $P$ of \pcircle{}, as follows:


\vspace{-2ex}
\begin{equation*}
\label{equ-regular-polygon}
%\hspace{-1.5ex}
    \begin{aligned}
        \hspace{5ex}  v_j = (r, \frac{(j-1)}{m}2\pi), ~j \in [1, m], \\
    \end{aligned}
\end{equation*}
%
\ni and (2) $E= \{\vv{v_mv_1}\} \bigcup \{\vv{v_jv_{j+1}}\ |\ j\in [1, m-1]\}$ is the set of edges that are labeled with the subscript of their start points.

%\stitle{intersection polygons ($\mathcal{R}^*$)}.

Figure~\ref{fig:polygons}.(1) illustrates the inscribed regular octagon ($m=8$) of a cone projection circle \pcircle{(P, r)}.

Let $\mathcal{R}_{s+i}$ ($1\le i \le k$) be the inscribed regular polygon of the cone projection  circle \pcircle{(P^c_{s+i}, r^c_{s+i})},
$\mathcal{R}^*_l$ ($1\le l\le k$) be the intersection $\bigsqcap_{i=1}^{l}\mathcal{R}_{s+i}$,
and $E^j$ ($1\le j \le m$) be the group of $k$ edges labeled with $j$ in all $\mathcal{R}_{s+i}$ ($i\in[1, k]$).
%
It is easy to verify that all edges in the same edge groups $E^j$ ($1\le j\le m$) are in parallel (or overlapping) with each other by the above definition of inscribed regular polygons, as illustrated in Figure~\ref{fig:polygons}.(2).


\begin{prop}
\label{prop-rp-intersection}
The intersection $\mathcal{R}^*_{l} \bigsqcap \mathcal{R}_{s+l+1}$ ($ 1\le l< k$) has at most $m$ edges, \ie at most one from each edge group.
\end{prop}



\eat{%%%%%%%%%%%
\begin{theorem}
\label{prop-rp-intersection}
If $\mathcal{R}_i$, $i \in [1, k]$, are M-edges regular polygons on a plane which are built by equation (3), then the intersection polygon
$\mathcal{R}^*_k$ of all $\mathcal{R}_i$ includes at most one edge from an edge group, \eg the $j^{th}$ edge group.
\end{theorem}
}%%%%%%%%%%%%%


\begin{proof}\ 
We shall prove this by contradiction.
Assume that $\mathcal{R}^*_{l} \bigsqcap \mathcal{R}_{s+l+1}$ has two distinct edges $\vv{A_i}$ and $\vv{A_{i'}}$  with the same label $j$ $(1\le j \le m)$, originally from
$\mathcal{R}_{s+i}$ and $\mathcal{R}_{s+i'}$  ($1\le i< i' \le l+1$).
%
Note that here $\mathcal{R}_{s+i} \bigsqcap \mathcal{R}_{s+i'} \ne \emptyset$ since $\mathcal{R}^*_l \bigsqcap \mathcal{R}_{s+l+1} \ne \emptyset$.
%
However, when $\mathcal{R}_{s+i} \bigsqcap \mathcal{R}_{s+i'} \ne \emptyset$, the intersection $\mathcal{R}_{s+i} \bigsqcap \mathcal{R}_{s+i'}$ cannot have
both edge $\vv{A_i}$ and edge $\vv{A_{i'}}$, which contradicts the assumption. \eop
\end{proof}


\begin{figure}[tb!]
	\centering
	\includegraphics[scale=0.85]{figures/Fig-polygons.png}
	\vspace{0ex}
	\caption{\small Regular octagons and their intersections ($m =8$).}
	\vspace{-1ex}
	\label{fig:polygons}
\end{figure}

Figure~\ref{fig:polygons}.(2) shows the intersection polygon (red lines) of $\mathcal{R}_1$, $\mathcal{R}_2$ and $\mathcal{R}_3$ with $7$ edges, and here edges labeled with $7$ have no contributions to the resulting intersection polygon.

\begin{prop}
\label{prop-cpi-time}
The intersection of $\mathcal{R}^*_l$ and $\mathcal{R}_{s+l+1}$ ($ 1\le l< k$) can be done in $O(1)$
time.
\end{prop}

\begin{proof}\ 
The inscribed regular polygon $\mathcal{R}_{s+l+1}$ has $m$ edges, and intersection polygon $\mathcal{R}^*_l$ has at most $m$ edges by Proposition~\ref{prop-rp-intersection}.
As the intersection of two $m$-edge convex polygons can be computed in $O(m)$ time~\cite{ORourke:Intersection}, the intersection of polygons $\mathcal{R}^*_l$ and $\mathcal{R}_{s+l+1}$ can be done in $O(1)$ time for a fixed $m$. \eop
\end{proof}

%By Theorem~\ref{prop-rp-intersection}, the intersection polygon $\mathcal{R}^*_k$ of all approximate polygon $\mathcal{R}_i$, $i \in [1, k]$, has no more than $M$ edges. We know the convex polygon intersection algorithm, \ie \cpia, of Figure~\ref{alg:c-poly-inter} has a time complexity of $O(|\mathcal{R}^*_k| + |\mathcal{R}_{k+1}|)$, which is $O(2M)$ here, thus, the computing of the intersection of polygons $\mathcal{R}^*_k$ and $\mathcal{R}_{k+1}$ can be implemented in a constant time, \ie $O(M)$, by the \cpia algorithm.



%%%%%%%%%%%%%%%%%%%%%%%%%%%%%%%%%%%%%%%%%%%%%%%%%%%%%%%
\subsection{Speedup Inscribed Regular Polygon Intersection}
\label{subsec-fastRPI}


Observe that algorithm \cpia in Figure~\ref{alg:c-poly-inter} is for general convex polygons,
while the inscribed regular polygons $\mathcal{R}_{s+i}$ ($i\in[1, k]$) of the cone projection circles are constructed in a unified way,
which allows us to develop a fast method to compute their intersection.

Let $\vv{A} = (P_{s_A}, P_{e_A})$ and $\vv{B} = (P_{s_B}, P_{e_B})$  be two directed edges on polygons $\mathcal{R}_{s+l+1}$ and $\mathcal{R}^*_{l}$, respectively.
Again edges $\vv{A}$ and $\vv{B}$ are moved counter-clockwise. Note that $\vv{A}$ and $\vv{B}$ are advanced step by step each time by the two advancing rules of algorithm \cpia.
%
However, it is possible to advance $\vv{A}$ or $\vv{B}$ multiple steps each time.
%
For example, in Figure~\ref{fig:c-poly-inter}.(1)--(5), edge $\vv{A}$ successively moves four steps, each under the advance rule (1) ``($\vv{A} \times \vv{B} < 0$ and $P_{e_A} \not \in \mathcal{H}(\vv{B})$) or ($\vv{A} \times \vv{B} \ge 0$ and $P_{e_B} \in \mathcal{H}(\vv{A})$)'' of algorithm \cpia.
Alternatively, we can directly move $A$ from Figure~\ref{fig:c-poly-inter}.(1) to Figure~\ref{fig:c-poly-inter}.(5), by reducing four steps to one step only.



\begin{prop}
\label{prop-rule1}
If either $(\vv{A} \bigsqcap \vv{B} \ne \emptyset$ \And $\vv{A} \times \vv{B} < 0$ \And $P_{e_A} \not \in \mathcal{H}(\vv{B}))$ or $(\vv{A} \bigsqcap \vv{B} \ne \emptyset$ \And $\vv{A} \times \vv{B} \ge 0$ \And $P_{e_B} \in \mathcal{H}(\vv{A}))$ holds, then $\vv{A}$ advances $s$ steps such that

\vspace{-1ex}
\begin{equation*}
\label{equ-rule1}
\small
    \hspace{2ex} s =  \left\{
    \begin{aligned}
        & 2\times(g(\vv{B}) - g(\vv{A}))  \hspace{5ex}~~if  ~g(\vv{B}) > g(\vv{A}) \\
        & {1}              \hspace{21ex}~if  ~g(\vv{A}) = g(\vv{B}) \\
        & 2\times(m+g(\vv{B}) - g(\vv{A})) ~~if  ~g(\vv{B}) < g(\vv{A}), \\
    \end{aligned}
    \right.       \hspace{6ex}{}
\end{equation*}
in which $g(e)$ denotes the label of edge $e$.
\end{prop}



\begin{proof}\ 
We first explain how the edge $\vv{A}$ advances.
Indeed, $\vv{A}$ is moved from its original position to its symmetric edge on $\mathcal{R}_{s+l+1}$ \wrt the symmetric line that is perpendicular to $\vv{B}$  on $\mathcal{R}^*_{l}$.
For example, in Figure~\ref{fig:r-poly-rule1}.(1), there is $\vv{A} \bigsqcap \vv{B} \ne \emptyset$ \And $\vv{A} \times \vv{B} \ge 0$ \And $P_{e_B} \in \mathcal{H}(\vv{A})$, hence $\vv{A}$ moves on. As $g(\vv{B})=3 > 1=g(\vv{A})$, $\vv{A}$ moves forward $2\times(g(\vv{B}) - g(\vv{A}))$ = $2\times(3-1)= 4$ steps.
Here, the label of edge $\vv{A}$ is changed to $5$, its symmetric edge $1$ on $\mathcal{R}_{s+l+1}$ \wrt the symmetric line that is perpendicular to $\vv{B}$ labeled with $3$  on $\mathcal{R}^*_{l}$.


We then present the proof.
If ($\vv{A} \bigsqcap \vv{B} \ne \emptyset$ \And $\vv{A} \times \vv{B} < 0$ \And $P_{e_A} \not \in \mathcal{H}(\vv{B})$) or ($\vv{A} \bigsqcap \vv{B} \ne \emptyset$ \And $\vv{A} \times \vv{B} \ge 0$ \And $P_{e_B} \in \mathcal{H}(\vv{A})$), then as all edges in the same edge groups $E^j$ ($1\le j\le m$) are in parallel with each other and by the geometric properties of regular polygon $\mathcal{R}_{s+k+1}$, it is easy to find that, for each position of $\vv{A}$ between its original to its opposite positions, we have (1) $\vv{A} \bigsqcap \vv{B} = \emptyset$, and (2) either $P_{e_A} \not \in \mathcal{H}(\vv{B})$ or $P_{e_B} \in \mathcal{H}(\vv{A})$. Hence, by the advance rule (1) of algorithm \cpia in Section~\ref{subsec-cpi}, edge $\vv{A}$ is always moved forward until it reaches the opposite position of its original one. From this, we have the conclusion. \eop
\end{proof}



%%%%%%%%%%%%%%%%%%%%%%%%%%%%%%%%%%%%%%%%
\begin{prop}
\label{prop-rule2}
If either ($\vv{A} \bigsqcap \vv{B} \ne \emptyset$ \And $\vv{A} \times \vv{B} \ge 0$ \And $P_{e_B} \not \in \mathcal{H}(\vv{A})$) or ($\vv{A} \bigsqcap \vv{B} \ne \emptyset$ \And $\vv{A} \times \vv{B} < 0$ \And $P_{e_A} \in \mathcal{H}(\vv{B})$) holds, then edge $\vv{B}$ is directly moved to the edge after the one having the same edge group as edge $\vv{A}$.
\end{prop}

\begin{proof}\ 
We first explain how the edge $\vv{B}$ is moved forward.
For example, in Figure~\ref{fig:r-poly-rule1}.(2), $\vv{A} \bigsqcap \vv{B} \ne \emptyset$ \And $\vv{A} \times \vv{B} < 0$ \And $P_{e_A} \in \mathcal{H}(\vv{B})$, hence $\vv{B}$ is moved forward. As the edge $\vv{A}$ is labeled with 7,
$\vv{B}$ moves to the edge labeled with 8 on $\mathcal{R}^*_{l}$, which is the next of the edge labeled with 7 on $\mathcal{R}^*_{l}$.
Note that if the edge labeled with 8 were not actually existing in the intersection polygon $\mathcal{R}^*_{l}$, then $\vv{B}$ should repeatedly move on until it reaches the first ``real" edge on $\mathcal{R}^*_{l}$.

We then present the proof.
If ($\vv{A} \bigsqcap \vv{B} \ne \emptyset$ \And $\vv{A} \times \vv{B} \ge 0$ \And $P_{e_B} \not \in \mathcal{H}(\vv{A})$) or ($\vv{A} \bigsqcap \vv{B} \ne \emptyset$ \And $\vv{A} \times \vv{B} < 0$ \And $P_{e_A} \in \mathcal{H}(\vv{B})$), then it is also easy to find that, for each position of $\vv{B}$ between its original to its target positions (\ie the edge after the one having the same edge group as $\vv{A}$), we have (1) $\vv{A} \bigsqcap \vv{B} = \emptyset$, and (2) either $P_{e_B} \not \in \mathcal{H}(\vv{A})$ or $P_{e_A} \in \mathcal{H}(\vv{B})$. Hence, by the advance rule (2) of algorithm \cpia in Section~\ref{subsec-cpi}, edge $\vv{B}$ is always moved forward until it reaches the target position. From this, we have the conclusion. \eop
\end{proof}



%%%%%%%%%%%%%%%%%%%%%example of intersection of regular polygons
\begin{figure}[tb!]
	\centering
	%\vspace{1ex}
	\includegraphics[scale=0.82]{figures/Fig-r-poly-rule1.png}
	\vspace{-1ex}
	\caption{\small Examples of fast advancing rules.}
	\vspace{-1ex}
	\label{fig:r-poly-rule1}
\end{figure}


%%%%%%%%%%%%%%%%%%%%%%%%%%%%%%%%%%%%%%%%



\eat{%%%%%%%%%%%%%%%%%%%%%%%%%%%%%%%%%%%
%%%%%%%%%%%%%%%%%%%%%%%%%%%%%%%%%%%%%%%%
\vspace{1ex}
\ni \emph{\underline{Rule 3}:
If $A \bigcap B = \emptyset$ and advances $A$, then moves $A$ and $B$ forward ``$a$ and $b$" steps respectively, where}
%\vspace{-1ex}
\begin{equation*}
\label{equ-rule3}
\small
    \hspace{0ex} (a,b) =  \left\{
    \begin{aligned}
        & (2,0), ~if~ g(next(A)) = g(B) ~and~ {outside}(A)  \\
        & (1,1), ~if~ g(next(A)) = g(B) ~and~ {inside}(A)\\
        & (1,0), ~else
    \end{aligned}
    \right.       \hspace{2ex}(5)
    %\vspace{-1ex}
\end{equation*}
%\vspace{-1ex}
\emph{and procedure $inside()$ or $outside()$ is a checking of the ``{\emph{inside}}" flag (line 5) of the \cpia algorithm (Figure~\ref{alg:c-poly-inter}).}
\vspace{1ex}


For example, in Figure~\ref{fig:r-poly-inter}-(4), $A \bigcap B = \emptyset$ and advances $A$, hence, rule 3 is applied. Because $A$ is inside and $g(next(A))=3 = g(B)$, $A$ and $B$ both move forward one step (Figure~\ref{fig:r-poly-inter}-(5)).



%%%%%%%%%%%%%%%%%%%%%%%%%%%%%%%%%%%%%%%%
\vspace{1ex}
\ni \emph{\underline{Rule 4}:
If $A \bigcap B = \phi$ and advances $B$, then moves $A$ and $B$ forward ``$a$ and $b$" steps respectively, where}
%\vspace{-1ex}
\begin{equation*}
\label{equ-rule4}
\small
    \hspace{0ex} (a,b) =  \left\{
    \begin{aligned}
        & (0,2), ~if~ g(next(B)) = g(A) ~and~ outside(B)\\
        & (1,1), ~if~ g(next(B)) = g(A) ~and~ inside(B)\\
        & (0,1), ~else
    \end{aligned}
    \right.       \hspace{2ex}(6)
\end{equation*}
\vspace{-1ex}

}%%%%%%%%%%%%%%%%%%%%%%%%%%End of Eat






\stitle{Algorithm \rpia}.
The presented regular polygon intersection algorithm, \ie\ \rpia, is the optimized version  of the convex polygon intersection algorithm \cpia, by Propositions \ref{prop-rule1} and \ref{prop-rule2}. We also save vertexes of a polygon in a fixed size array, which is different from \cpia  that saves polygons in linked lists.
Considering the regular polygons each having a fixed number of vertexes/edges, marked from $1$ to $m$, this policy allows us to quickly address an edge or vertex by its label.
%, as well as avoid the creating and removing of link nodes.

Given intersection polygon $\mathcal{R}^*_{l}$ of the preview $l$ polygons and the next approximate polygon $\mathcal{R}_{s+l+1}$, the algorithm \rpia returns $\mathcal{R}^*_{l+1}=\mathcal{R}^*_{l}  \bigsqcap \mathcal{R}_{s+l+1}$.
%The sketch of the algorithm is shown in Figure~\ref{alg:r-poly-inter}.
It runs the similar routine as the \cpia algorithm, except that (1) it saves polygons in arrays, and (2) the advance strategies are partitioned into two parts, \ie $\vv{A} \bigsqcap \vv{B} \ne \emptyset$ and $\vv{A} \bigsqcap \vv{B} = \emptyset$, where the former applies Propositions \ref{prop-rule1} and \ref{prop-rule2}, and the later remains the same as algorithm \cpia.





\eat{%%%%%%%%%%%%%%%%%%%%%%
\begin{example}
Figure~\ref{fig:r-poly-inter} is a running example of algorithm \rpia. The input is the same as Figure~\ref{fig:c-poly-inter}.

\ni (1) Initially, directed edges $\vv{A}$ and $\vv{B}$ are on polygons $\mathcal{R}_{k+1}$ and $\mathcal{R}^*_{k}$ separately. $\vv{A} \bigcap \vv{B} = P_1$ and $\vv{A}$ moves on.
\ni (2) $\vv{A}$ moves forward a 4-steps, directly from $2^{th}$ to $6^{th}$, under rule 2. Then, $\vv{A} \bigcap \vv{B} = \emptyset$ and $\vv{B}$ moves on.
\ni (3) After 4 steps of moving (in turn), $\vv{A} \bigcap \vv{B} = P_2$ and $\vv{B}$ moves on.
\ni (4) $\vv{B}$ advances a 2-steps, from $6^{th}$ to $3^{th}$, under rule 1. Then $\vv{A} \bigcap \vv{B} = \emptyset$ and $\vv{A}$ moves on.
\ni (5) $\vv{A}$ advances a step. Then $\vv{A} \bigcap \vv{B} = \emptyset$ and $\vv{B}$ moves on.
\ni (6) After 3 steps of moving, both $\vv{A}$ and $\vv{B}$ cycle their polygons. The intersection polygon, the same as the result of \cpia (also see Figure~\ref{fig:c-poly-inter}), is returned.
\end{example}


%%%%%%%%%%%%%%%%%%%%%example of intersection of regular polygons
\begin{figure}[tb!]
\centering
\includegraphics[scale=0.88]{figures/Fig-r-poly-inter.png}
\vspace{-1ex}
\caption{\small A running example of intersection of polygons.}
\vspace{-2ex}
\label{fig:r-poly-inter}
\end{figure}
}


\eat{%%%%%%%%%%%%%%%%%%%%%Algorithm
\begin{figure}[tb!]
\begin{center}
{\small
\begin{minipage}{3.36in}
\myhrule
\vspace{-1ex}
\mat{0ex}{
	{\bf Algorithm} ~\rpia ($\mathcal{R}^*_k$, $\mathcal{R}_{k+1}$) \\
	\bcc \hspace{2ex}\=  Set $\vv{A}$ and $\vv{B}$ {arbitrarily} on $\mathcal{R}^*_k$ and $\mathcal{R}_{k+1}$\\
	\icc \>\hspace{0ex}\= Repeat \\
	\icc \>\hspace{3ex} If $\vv{A} \bigcap \vv{B} \ne \phi$ Then \\
	\icc \>\hspace{6ex} {Check for termination}. \\
	\icc \>\hspace{6ex} Update an {\emph{inside}} flag for $\vv{A}$ or $\vv{B}$. \\
	\icc \>\hspace{6ex} {\emph{Moves on either $\vv{A}$ or $\vv{B}$ under rule 1 or 2.}}\\
	\icc \>\hspace{3ex} Else \\
	\icc \>\hspace{6ex} {{Moves on either $\vv{A}$ or $\vv{B}$.}}\\
	\icc \hspace{1ex} Until both $\vv{A}$ and $\vv{B}$ cycle their polygons \\
	\icc \hspace{0ex} Handle $\mathcal{R}^*_k \subset \mathcal{R}_{k+1}$ and $\mathcal{R}^*_k \subset \mathcal{R}_{k+1}$ and $\mathcal{R}^*_k \bigcap \mathcal{R}_{k+1} = \phi$ cases \\
    \icc \hspace{0ex} Return $\mathcal{R}^*_k \bigcap \mathcal{R}_{k+1}$
}
\vspace{-2ex}
\myhrule
\end{minipage}
}
\end{center}
\vspace{-2ex}
\caption{\small Intersection of Regular polygons.}
\label{alg:r-poly-inter}
\vspace{-2ex}
\end{figure}
}%%%%%%%%%%%%%%%%%%%%%%%%%%%%%%%%%%%%%




\vspace{0.5ex}
\stitle{Correctness and complexity analyses.}
Observe that algorithm \rpia basically has the same routine as algorithm \cpia, except that it fastens the advancing speed of directed edges $\vv{A}$ and $\vv{B}$ under certain circumstances as shown by Propositions \ref{prop-rule1} and \ref{prop-rule2}, which together ensure the correctness of \rpia. Moreover, algorithm \rpia runs in $O(1)$ time
by Proposition~\ref{prop-cpi-time}.



%%%%%%%%%%%%%%%%%%%%%%%%%%%%%%%%%%%%%%%%%%%%%%%%%%%%%%%



