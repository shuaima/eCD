\documentclass{letter}
\usepackage{geometry}

% duan
\usepackage{xspace}
\usepackage{color}
\usepackage{amsfonts}
\usepackage{cite}

\newcommand{\marked}[1]{\textcolor{red}{#1}}

\newcommand{\kw}[1]{{\ensuremath {\mathsf{#1}}}\xspace}

\geometry{left=2.0cm, right=2.0cm, top=2.5cm, bottom=2.5cm}
\newcommand{\ie}{\emph{i.e.,}\xspace}
\newcommand{\eg}{\emph{e.g.,}\xspace}
\newcommand{\wrt}{\emph{w.r.t.}\xspace}
\newcommand{\aka}{\emph{a.k.a.}\xspace}
\newcommand{\kwlog}{\emph{w.l.o.g.}\xspace}
\newcommand{\etal}{\emph{et al.}\xspace}
\newcommand{\sstab}{\rule{0pt}{8pt}\\[-2.4ex]}

\newcommand{\topk}[1]{\kw{top}--\kw{#1}}
\newcommand{\topdown}{\kw{topDown}}
\newcommand{\extsubgraph}{\kw{compADS^+}}
\newcommand{\drfds}{\kw{FIDES^+}}
\newcommand{\extsubgraphold}{\kw{compADS}}
\newcommand{\findtimax}{\kw{maxTInterval}}
\newcommand{\findtimin}{\kw{minTInterval}}
\newcommand{\meden}{\kw{MEDEN}}

\newcommand{\tranformgraph}{\kw{convertAG}}
\newcommand{\mergecc}{\kw{strongMerging}}
\newcommand{\strongpruning}{\kw{strongPruning}}
\newcommand{\boundedprobing}{\kw{boundedProbing}}

\newcommand{\AFPR}{\kw{AFP}-\kw{reduction}}
\newcommand{\nwm}{{\sc nwm}\xspace}


\newcommand{\cone}[1]{{$\mathcal{C}{#1}$}}
\renewcommand{\circle}[1]{{$\mathcal{O}{#1}$}}
\newcommand{\pcircle}[1]{{$\mathcal{O}^c{#1}$}}

\newcommand{\vv}{\overrightarrow}

\begin{document}



Prof. {Ren{\'{e}}e J. Miller} \\
Editor-in-Chief		\\
The VLDB Journal	\\



Dear Prof. Miller,

Attached please find a revised version of our submission to
the VLDB Journal, \emph{One-Pass Trajectory Simplification Using the Synchronous Euclidean Distance}.


{The paper has been revised according to the comments of Reviewer \#4. In particular, we have pointed out that the  ``optimal'' LS algorithm is compression optimal and added the explanation about this algorithm in Section 5.2.2.

We would like to thank all the referees for their thorough reading of our paper and for their valuable comments.

Below please find our detailed responses to the comments.



%******************* reviewer 1 ***********************************************
\line(1,0){500}

\textbf{Response to the comments of Reviewer 4.}

\line(1,0){100}


\textbf{[R4C1]} \emph{The authors have mostly addressed my comments. However there is one important remaining issues:
%
In Section 5.2.2 ``Evaluation of Average Errors", the ``Optimal LS Algorithm" is in fact worse than almost all other algorithms except for CISED-W.
%
The ``Optimal LS Algorithm", as explained in Section 2.2, seems to be an exhaustive brute force algorithm that optimizes compression ratio given an error bound.
So given this context the author should consider renaming the ``Optimal LS Algorithm" in order to reflect its objective, which is compression not accuracy.
%
Also, some explanation in Section 5.2.2 about the behaviour of this algorithm is necessary.}

Yes, given an error bound, the ``Optimal LS Algorithm" optimizes compression ratio, not accuracy.
To clarify this, we have renamed ``Optimal LS Algorithm (Optimal in short)" as ``Compression Optimal LS algorithm (C-Optimal in short)".
We have also added explanations about this algorithm in Section 5.2.2.

Thanks for pointing this out!

\line(1,0){500}



Your sincerely,

Xuelian Lin, Jiahao Jiang, Shuai Ma, Yimeng Zuo and Chunming Hu



%\bibliographystyle{abbrv}
%\bibliography{sec-ref}


\end{document}
