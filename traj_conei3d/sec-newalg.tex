



\section{One-Pass Trajectory Simplification}

Following \cite{Trajcevski:DDR,Lin:Operb}, we consider two classes of trajectory simplification.
The first one, referred to as \emph{strong simplification}, that takes input a trajectory \trajec{T} and an error bound $\epsilon$ as inputs, and produces   
a simplified trajectory \trajec{T'} such that all data points in \trajec{T'} belongs to \trajec{T}.
The second one, referred to as \emph{weak simplification}, that takes input a trajectory \trajec{T} and an error bound $\epsilon$ as inputs, and produces
a simplified trajectory \trajec{T'} such that some data points in \trajec{T'} may not belong to \trajec{T}. That is,  weak simplification allows data interpolation,
commonly used in polygonal curve approximation \cite{Williams:Bounded, Heckbert:Survey, Zhao:Sleeve}, time series approximation \cite{ORourke:Fitting, Elmeleegy:Stream, Xie:Stream, Luo:Streaming} and trajectory simplification \cite{Lin:Operb}.


The main result here is stated as follows.

\begin{theorem}
\label{prop-cist-complexity}
There exists an one-pass error bounded trajectory simplification algorithm using the synchronous Euclidean Distance for either strong or weak trajectory simplification.
\end{theorem}

We shall prove this by providing such algorithm for both strong and weak trajectory simplification, by employing the techniques developed .

We now are ready to present two one pass, error bounded and \sed enabled trajectory simplification algorithms, namely \cist and \cista, implementing the above spatio-temporal cone intersection approach.



%%%%%%%%%%%%%%%%%%%%%%%%%%%%%%%%%%%%%%%%%%%%%%%%%%%%%%%
%\subsection{{One-pass algorithm}}
\subsection{Strong Trajectory Simplification}
The first algorithm, \cist, is a \emph{strong simplification}\cite{Trajcevski:DDR} algorithm, which outputs a subset of the original points.
Recall in Theorem~\ref{prop-3d-ci}, the point $Q$ may not be in the input sub trajectory $[P_s,...,P_e]$.
If we mandatory require $Q=P_e$, the end point of the sub trajectory, then a \emph{narrow} spatio-temporal cone with a threshold of $\epsilon/2$ is helpful.

\begin{theorem}
\label{prop-3d-ci-half}
Given a sub trajectory $\{P_s, \ldots, P_e\}$ and a constant $\epsilon$, if $\bigcap_{i=s+1}^{e}{\mathcal{C}(P_s, P_i, \epsilon/2)} \ne \{P_s\}$ then the sub trajectory can be represented by the line segment $\vv{P_sP_e}$.
\end{theorem}

\begin{proof}
If $\bigcap_{i=s+1}^{e}{\mathcal{C}(P_s, P_i, \epsilon/2)} \ne \{P_s\}$, then by Theorem~\ref{prop-3d-ci}, there exists a point $Q$, $Q.t = P_e.t$, such that $sed(P_i, \vv{P_sQ})<\epsilon/2$ for all $P_i$, $i \in [s,e]$. Hence, $sed(P_i, \vv{P_sP_e})\le \epsilon/2 + |P_eQ| < \epsilon/2 + \epsilon/2 = \epsilon$.
\end{proof}



Figure~\ref{alg:CI3d} is the P-codes of algorithm \cist. It takes as input a trajectory $\dddot{\mathcal{T}}$, an error bound $\epsilon$, the number $M$ of edges of a regular polygon and a time $t_c>t_s$, and returns the simplified trajectory $\overline{\mathcal{T}}$.

%%%%%%%%%%%%%%%%%%%%%
\begin{figure}[tb!]
\begin{center}
{\small
\begin{minipage}{3.36in}
\myhrule
\vspace{-1ex}
\mat{0ex}{
	{\bf Algorithm} ~\cist$(\dddot{\mathcal{T}}[P_0,\ldots,P_n], ~\epsilon, ~M, ~t_c)$\\
%	\sstab
	\bcc \hspace{1ex}\= $P_s := P_0$;  ~$P_e := P_0$;  ~$\overline{\mathcal{T}} := \phi$;  ~$\mathcal{G}^* := \phi$;  ~$i := 1$\\
%         \hspace{2ex}     $intersection = \emptyset$;   \\
	\icc \hspace{1ex}\= \While $i \le n$ \Do \\
	\icc \>\hspace{3ex} $\mathcal{G}$ := \kw{getRegularPolygon}($P_s$, $P_i$, $\epsilon/2$, $M$, $t_c$) \\
	\icc \>\hspace{3ex} \If $\mathcal{G}^* = \phi$ \Then \\
	\icc \>\hspace{6ex} $\mathcal{G}^* :=\mathcal{G}$ \\
	\icc \>\hspace{3ex} \Else \\
	\icc \>\hspace{6ex} $\mathcal{G}^* := {\rpia}(\mathcal{G}^*, ~\mathcal{G})$ \\
	\icc \>\hspace{3ex} \If $\mathcal{G}^* \ne \phi$ \Then \\
	\icc \> \hspace{6ex} $P_e := P_i$ \\
	\icc \> \hspace{6ex} $i := i+1$ \\
	\icc \>\hspace{3ex} \Else\\
	\icc \> \hspace{6ex} $\overline{\mathcal{T}} := \overline{\mathcal{T}}\cup \{\mathcal{L}(P_s,P_e)\}$ \\
	\icc \> \hspace{6ex} $P_s := P_e$;  ~~$\mathcal{G}^* := \phi$ \\
	\icc \hspace{1ex}\Return $\overline{\mathcal{T}}$ \\
\\
	{\bf Procedure} ~\kw{getRegularPolygon}$(P_s,~P_i,~r,~M,~t_c)$ \\
%	\bcc \hspace{1ex} \textcolor[rgb]{0.00,0.07,1.00}{Transform $P_s$ and $P_i$ to points in Cartesian coordinates} \\
	\bcc \hspace{1ex} $k := (t_c-t_s)/(P_i.t - P_s.t)$ \\
	\icc \hspace{1ex} $P_i.x := P_s.x + k*(P_i.x-P_s.x)$ \\
	\icc \hspace{1ex} $P_i.y := P_s.y + k*(P_i.y-P_s.y)$ \\
	\icc \hspace{1ex} \For $(j := 1;j \le M;j++)$ \\
	\icc \> \hspace{3ex} $\theta :=  (2 * j + 1)*\pi /M $ \\
	\icc \> \hspace{3ex} $\mathcal{G}.vertex[j].x := P_i.x + k*r*\cos\theta$\\
	\icc \> \hspace{3ex} $\mathcal{G}.vertex[j].y := P_i.y + k*r*\sin\theta$\\
	\icc \hspace{1ex} \Return $\mathcal{G}$
}
\vspace{-2ex}
\myhrule
\end{minipage}
}
\end{center}
\vspace{-2ex}
\caption{\small One-pass strong trajectory  simplification algorithm.}
\label{alg:CI3d}
\vspace{-2ex}
\end{figure}
%%%%%%%%%%%%%%%%%%%%%%%%%%%%%%%%%%%%%




The algorithm first initializes the start and end points, $P_s$ and $P_e$, to $P_0$, the intersection polygon $\mathcal{G}^*$ to $\phi$, the output set $\overline{\mathcal{T}}$ to $\phi$ and the index $i$ of point to $1$ (line 1).

The algorithm sequentially processes each point in the trajectory (line 2).
It gets the regular polygon of the current point $P_i$ (line 3).
This is performed by calling procedure $getRegularPolygon()$. The procedure $getRegularPolygon()$ is an implementation of equations (1)-(3). It first gets the projection circle for the current point $P_i$, defined by equations (1) and (2); then it builds and returns an inscribed regular polygon $\mathcal{G}$, whose vertexes are defined by equation (3), of the projection circle.
Note that the computing of $\theta$, $r*\sin\theta$ and $r*\cos\theta$ can be optimized to run one time only during the processing of a trajectory.

After that, it gets the intersection polygon of the current regular polygon $\mathcal{G}$ with the polygon $\mathcal{G}^*$, which is formed by all the preview regular polygons, and saves it in $\mathcal{G}^*$ (lines 4-7).
This is performed by calling procedure $\rpia()$, as shown in Figure~\ref{alg:r-poly-inter}.
% which returns the intersection polygon of input polygons $\mathcal{G}^*$ and $\mathcal{G}$.

If the intersection polygon is not empty, then the process goes on to the next point (lines 8-10).
Otherwise, the current section is terminated and a new section is started (lines 11-13).
%
At last, the algorithm returns $\overline{\mathcal{T}}$ (line 14).




%%%%%%%%%%%%%%%%%%%%%%%%%%%%%%%%%%%%%%%%%%%%%%%example of Algorithm CISED
\begin{figure*}[tb!]
\centering
\includegraphics[scale=0.8]{figures/Fig-ex-conest.png}
\vspace{-1ex}
\caption{\small A running example of the \cist algorithm. The points and the oblique circular cones are projected on an x-y space. The trajectory $\dddot{\mathcal{T}}[P_0, \ldots, P_{10}]$ is compressed to four line segments.}
\vspace{-2ex}
\label{fig:exm-const}
\end{figure*}
%%%%%%%%%%%%%%%%%%%%%%%%%%%%%%%%%%%%%%%%%%%%%%%%%%%%%%%%%%%%%%%%%%%%%%%%%%%




\begin{example}
\label{exm-alg-conest}
Figure~\ref{fig:exm-const} is a running example of the \cist algorithm taking as input the same trajectory as shown in Figure~\ref{fig:notations} and Figure~\ref{fig:sleeve}.
%, and a distance tolerance $\epsilon$
%At the beginning, $P_0$ is the start point.
For convenience, here, we project the points and the oblique circular cones on a x-y space.

\ni (1) After initialization, the \cist algorithm reads point $P_1$ and builds a narrow \emph{oblique circular cone} $\mathcal{C}$($P_0$, $P_{1}$, $\epsilon/2$), taking $P_0$ as the vertex and $\mathcal{O}(P_1, \epsilon/2)$ as the bottom circle (green dash). The \emph{circular cone} is projected on the plane $t=t_1$, and the inscribe regular polygon $\mathcal{G}_1$ of the projection circle is returned. The intersection of $\mathcal{G}_1$ and $\mathcal{G}^*$ is $\mathcal{G}_1 \ne \phi$.

\ni (2) The algorithm reads $P_2$ and builds $\mathcal{C}$($P_0$, $P_{2}$, $\epsilon/2$) (red dash). The \emph{circular cone} is also projected on the plane $t=t_1$ and the inscribe regular polygon $\mathcal{G}_2$ of the projection circle is returned. The intersection of $\mathcal{G}_2$ and $\mathcal{G}^*$ is not $\phi$.

\ni (3) For point $P_3$, the algorithm runs the same routing as $P_2$. However, the intersection of $\mathcal{G}_3$ and $\mathcal{G}^*$ is $\phi$. Thus, line segment $\vv{P_0P_2}$ is output, and a new section is started, taking $P_2$ as the new start point.

\ni (4) At last, the algorithm outputs four continuous line segments, \ie $\vv{P_0P_2}$, $\vv{P_2P_4}$, $\vv{P_4P_{7}}$ and $\vv{P_7P_{10}}$.
\end{example}





%%%%%%%%%%%%%%%%%%%%%%%%%%%%%%%%%%%%%%%%%%%%%%%%%%%%%%%%%%%%%%%%%%%
\subsection{Weak Trajectory Simplification}


The second algorithm, named \cista, allows outputting points $\overline{\mathcal{T}}$ that may not belong to the original points $\dddot{\mathcal{T}}$.
This method, known as the ``interpolation" approach, is used in disciplines of polygonal curve approximation \cite{Williams:Bounded, Heckbert:Survey, Zhao:Sleeve}, time series approximation \cite{ORourke:Fitting, Elmeleegy:Stream, Xie:Stream, Luo:Streaming} and trajectory simplification \cite{Lin:Operb}.
The ``interpolation" approach is also called a \emph{weak simplification} in \cite{Trajcevski:DDR}, and will usually bring better compression ratios.

Algorithm \cista is the directly result of Corollary~\ref{prop-circle-intersection}.
Given a sub trajectory $[P_s,...,P_e]$, an error bound $\epsilon$, a constant $M$ and a time $t_c > t_s$, \cista outputs point $Q$ of Corollary~\ref{prop-circle-intersection}. The point $Q$ is got by the following strategies.

\vspace{1ex}
\ni (1) if $G^*_e \ne \phi$ and $P^c_e \in G^*_e$, meaning the projection point of $P_e$ on the plane ``$t=t_c$", \ie $P^c_e$, is in the intersection polygon of regular polygons \{$\mathcal{G}_s, \dots, \mathcal{G}_e$\} of all projection circles \{$\mathcal{O}(P^c_s, \epsilon^c_s), \ldots, \mathcal{O}(P^c_e, \epsilon^c_e)$\}, then the point $P_e$ is a candidate point of $Q$ and the algorithm outputs $Q=P_e$.

\vspace{1ex}
\ni (2) if $G^*_e \ne \phi$ and $P^c_e \notin G^*_e$, then it gets the central point of intersection polygon $G^*_e$, projects it on the plane ``$t=t_e$", and outputs the projection point $Q$.

\vspace{1ex}
%Figure~\ref{alg:ciseda} is the P-codes of algorithm \cista.
Algorithm \cista runs the similar routine as algorithm \cist, except that in line 3 of Figure~\ref{alg:CI3d}, $\epsilon/2$ is replaced by $\epsilon$; and in lines 12-13 of Figure~\ref{alg:CI3d}, $P_e$ is replace by $Q$ which is got by the above strategies.



\begin{example}
\label{exm-alg-conesta}
Figure~\ref{fig:exm-consta} is a running example of the \cista algorithm taking as input the same trajectory as shown in above.
The points and the oblique circular cones are projected on a x-y space.

\ni (1) After initialization, the \cista algorithm reads point $P_1$ and builds a \emph{oblique circular cone} $\mathcal{C}$($P_0$, $P_{1}$, $\epsilon$), and projects it on the plane $t=t_1$. The inscribe regular polygon $\mathcal{G}_1$ of the projection circle is returned. The intersection of polygon $\mathcal{G}_1$ and polygon $\mathcal{G}^*$ is $\mathcal{G}_1 \ne \phi$.

\ni (2) $P_2$, $P_3$ and $P_4$ are processed in turn. The intersection polygons are not empty.

\ni (3) For point $P_5$, the intersection of $\mathcal{G}_5$ and $\mathcal{G}^*$ is $\phi$. Thus, line segment $\vv{P_0Q} =\vv{P_0P'_4}$ is output, and a new section is started, taking $P'_4$ as the new start point.

\ni (4) At last, the algorithm outputs 3 continuous line segments, \ie $\vv{P_0P'_4}$, $\vv{P'_4P_8}$ and $\vv{P_8P_{10}}$. Where, $P_0$, $P_8$ and $P_{10}$ are original points.
\end{example}



\eat{%%%%%%%%%%%%%%%%%%%%% Algorithm
\begin{figure}[tb!]
\begin{center}
{\small
\begin{minipage}{3.36in}
\myhrule
\vspace{-1ex}
\mat{0ex}{
	{\bf Algorithm} ~$\cista(\dddot{\mathcal{T}}[P_0,\ldots,P_n], ~\epsilon, ~M, ~t_c)$\\
%	\sstab
	\bcc \hspace{1ex}\= $P_s := P_0$;  ~$P_e := P_0$;  ~$\overline{\mathcal{T}} := \phi$;  ~$\mathcal{G}^* := \phi$;  ~$i := 1$\\
%         \hspace{2ex}     $intersection = \emptyset$;   \\
	\icc \hspace{1ex}\= \While $i \le n$ \Do \\
	\icc \>\hspace{3ex} $\mathcal{G} := {getRegularPolygon}$($P_s$, $P_i$, $\epsilon$, $M$, $t_c$) \\
	\icc \>\hspace{3ex} \If $\mathcal{G}^* = \phi$ \Then \\
	\icc \>\hspace{6ex} $\mathcal{G}^* :=\mathcal{G}$ \\
	\icc \>\hspace{3ex} \Else \\
	\icc \>\hspace{6ex} $\mathcal{G}^* := {\rpia}(\mathcal{G}^*, ~\mathcal{G})$ \\
	\icc \>\hspace{3ex} \If $\mathcal{G}^* \ne \phi$ \Then \\
	\icc \> \hspace{6ex} $P_e := P_i$ \\
	\icc \> \hspace{6ex} $i := i+1$ \\
	\icc \>\hspace{3ex} \Else\\
	\icc \> \hspace{6ex} $\overline{\mathcal{T}} := \overline{\mathcal{T}}\cup \{\mathcal{L}(P_s,Q)\}$ \\
	\icc \> \hspace{6ex} $P_s := Q$;  ~~$\mathcal{G}^* := \phi$ \\
	\icc \hspace{1ex}\Return $\overline{\mathcal{T}}$
%
%	{\bf procedure} $getPolygon$($P_s$, $P_i$, $\epsilon/2$) \\
%
%	{\bf procedure} $getIntersection(ipolygon, \mathcal{G})$ \\
}
\vspace{-2ex}
\myhrule
\end{minipage}
}
\end{center}
\vspace{-2ex}
\caption{\small Aggressive spatio-temporal cone intersection algorithm (\cista).}
\label{alg:ciseda}
\vspace{-2ex}
\end{figure}
}%%%%%%%%%%%%%%%%%%%%%%%%%%%%%%%%%%%%%




%%%%%%%%%%%%%%%%%%%%%%%%%%%%%%%%%%%%%%%%%%%%%%%%%%%%%%%%%%%%%%%%%%%
%A running example of the \cista algorithm
%%%%%%%%%%%%%%%%%%%%%%%%%%%%%%%%%%%%%%%%%%%%%%%%%%%%%%%%%%%%%%%%%%%
\begin{figure*}[tb!]
\centering
\includegraphics[scale=0.8]{figures/Fig-conesta.png}
\vspace{-1ex}
\caption{\small A running example of the \cista algorithm. The points and the oblique circular cones are projected on an x-y space. The trajectory $\dddot{\mathcal{T}}[P_0, \ldots, P_{10}]$ is compressed to three line segments.}
\vspace{-2ex}
\label{fig:exm-consta}
\end{figure*}
%%%%%%%%%%%%%%%%%%%%%%%%%%%%%%%%%%%%%%%%%%%%%%%%%%%%%%%%%%%%%%%%%%%%



\subsection{Correctness and complexity analysis}

The correctness of algorithm \cist follows from Theorem~\ref{prop-3d-ci-half} and Corollary~\ref{prop-circle-intersection}, and the correctness of algorithm \cista follows from Corollary~\ref{prop-circle-intersection}. For the circles intersection of Corollary~\ref{prop-circle-intersection}, we replace it by the regular polygons intersection.
Meanwhile, given time $t_c$ and projection circles $\mathcal{O}(P^c_i, r^c_i)$, $i \in (s, e]$, we get a inscribed regular polygon from each projection circle.
If these regular polygons are intersecting, then $sed(P_i, \vv{P_sP_e})<\epsilon$ for all points $P_i$, $i \in [s,e]$.
This ensures the correctness of the algorithms.

{For space complexity, the algorithms need $O(M)$ space.}
For time complexity, we have follows.


\begin{proof}
For each data point in the input trajectory, the procedures $getRegularPolygon()$ and $\rpia()$ are both called one time.
(1) Procedure $getRegularPolygon()$ completes in $O(M)$ time, and
(2) Procedure $\rpia(\mathcal{G}^*, \mathcal{G})$ also completes in $O(M)$ time.
%completes in $O(|\mathcal{G}^*| + |\mathcal{G}|)$ time. By Theorem~\ref{prop-rp-intersection}, we know that $|\mathcal{G}^*| \le M$ and $|\mathcal{G}|\le M$, where $M$ is the number of vertices in a regular polygon. Hence, $getIntersectPolygon()$
$M$ is a constant, thus, the process of a point needs a constant time.
Hence, algorithms \cist and \cista both have a time complexity of $O(N)$, where $N$ is the size of the input trajectory.
This finishes the proof.
\end{proof}



