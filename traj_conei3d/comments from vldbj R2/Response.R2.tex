\documentclass{letter}
\usepackage{geometry}

% duan
\usepackage{xspace}
\usepackage{color}
\usepackage{amsfonts}
\usepackage{cite}

\newcommand{\marked}[1]{\textcolor{red}{#1}}

\newcommand{\kw}[1]{{\ensuremath {\mathsf{#1}}}\xspace}

\geometry{left=2.0cm, right=2.0cm, top=2.5cm, bottom=2.5cm}
\newcommand{\ie}{\emph{i.e.,}\xspace}
\newcommand{\eg}{\emph{e.g.,}\xspace}
\newcommand{\wrt}{\emph{w.r.t.}\xspace}
\newcommand{\aka}{\emph{a.k.a.}\xspace}
\newcommand{\kwlog}{\emph{w.l.o.g.}\xspace}
\newcommand{\etal}{\emph{et al.}\xspace}
\newcommand{\sstab}{\rule{0pt}{8pt}\\[-2.4ex]}

\newcommand{\topk}[1]{\kw{top}--\kw{#1}}
\newcommand{\topdown}{\kw{topDown}}
\newcommand{\extsubgraph}{\kw{compADS^+}}
\newcommand{\drfds}{\kw{FIDES^+}}
\newcommand{\extsubgraphold}{\kw{compADS}}
\newcommand{\findtimax}{\kw{maxTInterval}}
\newcommand{\findtimin}{\kw{minTInterval}}
\newcommand{\meden}{\kw{MEDEN}}

\newcommand{\tranformgraph}{\kw{convertAG}}
\newcommand{\mergecc}{\kw{strongMerging}}
\newcommand{\strongpruning}{\kw{strongPruning}}
\newcommand{\boundedprobing}{\kw{boundedProbing}}

\newcommand{\AFPR}{\kw{AFP}-\kw{reduction}}
\newcommand{\nwm}{{\sc nwm}\xspace}


\newcommand{\cone}[1]{{$\mathcal{C}{#1}$}}
\renewcommand{\circle}[1]{{$\mathcal{O}{#1}$}}
\newcommand{\pcircle}[1]{{$\mathcal{O}^c{#1}$}}

\newcommand{\vv}{\overrightarrow}

\begin{document}



Prof. {Ren{\'{e}}e J. Miller} \\
Editor-in-Chief		\\
The VLDB Journal	\\



Dear Prof. Miller,

Attached please find a revised version of our submission to
the VLDB Journal, \emph{One-Pass Trajectory Simplification Using the Synchronous Euclidean Distance}.


\textcolor{blue}{The paper has been substantially revised according to the referees' comments. In particular, we have (a) added the intuition and key ideas of our solution, (b) elaborated the proofs and clarified the necessities of propositions, (c) added a set of new experiments to compare SED with PED, (d) moved all the proofs to an appendix, and (e) refined the entire paper to improve its readability.}

We would like to thank all the referees for their thorough reading of our paper and for their valuable comments.

Below please find our detailed responses to the comments.


\line(1,0){500}


%******************* reviewer 1 ***********************************************
\line(1,0){500}

\textbf{Response to the comments of Reviewer 1.}

\line(1,0){100}


\textbf{[R1Ca]} \emph{The scholarship in the paper remains very sloppy and I do not find that \textcolor{blue}{R2W1} or \textcolor{blue}{R3C1} were addressed adequately in the revision only rather begrudgingly in the response letter. Without this, the paper remains a delta off their PED work [15]. The novelty rests on an adequate justification for SED over PED.}



\textbf{[R1Cb]} \emph{Too many of the reviewer comments were addressed exclusively in response letter, not in the paper itself (e.g., \textcolor{blue}{R3C4-2}).}





%******************* reviewer 2 ***********************************************
\line(1,0){500}

\textbf{Response to the comments of Reviewer 2.}

\line(1,0){100}


\textbf{[R2Ca]} \emph{The reviewers have pointed out that the paper lacks intuition from over-formalization. Reading the revised paper, I still see many places where intuition is missing.}
\emph{(1) When introducing SED in the introduction (3rd paragraph), instead of defining SED using notation, it would be better to define it using plain English.}
\emph{(2) In Sec. 2.1 (2nd paragraph on Page 4), the author sate ``Intuitively..." but what's following is not intuitive at all.}
\emph{Both are pointed out by \textcolor{blue}{R3C2}.}

Yes, we have rewritten the 3rd paragraph of Sec.1 to better explain SED. We have introduced SED using plain English, instead of defining SED using notation, and a new figure (Fig.1 in the new submission) has been added to help understand the meaning of SED.
{In Sec. 2.1 (2nd paragraph on Page 4), we also introduce SED using plain English in the first, followed by the formal explaination using notations.} 
Thanks very much for pointing out this.


\textbf{[R2Cb]} \emph{\textcolor{blue}{R3C1} pointed out that the example in the introduction is not convincing because it is really difficult to tell the difference between PED and SED. In the response, the authors explained the motivation of SED, however, did not directly address the comment. From the example (Fig. 1), it is still hard to tell the difference. Ideally, I wish the motivating example could illustrate the ``unbounded" error of using PED (claimed at the end of 2nd paragraph in Sec. 1), and allow the readers to observe that SED provides a huge improvement over PED.}

Yes, we have add a new figure (Fig.1 in the new submission) to explain, and also compare, PED and SED. Moreover, the 2nd paragraph of Sec.1 is rewritten to explain the charactor of PED and why is not suitable for Spatio-temporal queries, \ie the ``unbounded" error of using PED in Spatio-temporal queries. And in the revised 3rd paragraph of Sec.1, we have also explained why SED is suitable for Spatio-temporal queries.
Thanks for your great suggestions.

\textbf{[R2Cc]} \emph{The experiments are comprehensive, however, still seem to lack meaningful discussion. For example, I wish to understand why in Fig.16, the average error increases with larger $m$, as this is somewhat counter-intuitive as larger $m$ leads to longer running time (Fig.20). The discussion in Exp-2.1 does not offer an explanation but simply describes what's in the plot. If applicable, I suggest the authors to remove the pervasive ``listings" of numerical results (e.g., ``The average errors of algorithms CISED-S and CISED-W are on average $(119.3\%, 127.7\%, 119.9\%, 138.0\%)$"), as they are already given in the plots, and add more in-depth discussion on why the results are what they are by linking back to the algorithms. }

Yes, we have added ``more in-depth discussion on why the results are what they are by linking back to the algorithms" in Sec.5, majorly in Exp\_1.1-(2), Exp\_1.2-(1)(3)(4), Exp\_2.1-(2), Exp\_2.2-(1)(2), Exp\_3.1-(2)(3), Exp\_4.1, Exp\_4.2 and Exp\_4.3. 
To the example you mentioned above,  ``why in Fig.16, the average error increases with larger $m$", it is clarified in Exp\_2.1-(2), \ie ``the increase of edge number $m$ of a polygon lets the polygon more closely approximate its corresponding circle, which means that some points with larger SED are also included to a line segment, which further leads to a larger average error." \textcolor{blue}{Besides, the average error does not seem to have a significant relationship with the running time, so I realy do not catch why it is ``counter-intuitive".}

The numerical results are to be reserved as it is convenient for some readers to refer them. Thanks! 


\line(1,0){500}



Your sincerely,

Xuelian Lin, Jiahao Jiang, Shuai Ma, Yimeng Zuo and Chunming Hu



%\bibliographystyle{abbrv}
%\bibliography{sec-ref}


\end{document}
