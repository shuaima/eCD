\documentclass{letter}
\usepackage{geometry}

% duan
\usepackage{xspace}
\usepackage{color}
\usepackage{amsfonts}
\usepackage{cite}

\newcommand{\marked}[1]{\textcolor{red}{#1}}

\newcommand{\kw}[1]{{\ensuremath {\mathsf{#1}}}\xspace}

\geometry{left=2.0cm, right=2.0cm, top=2.5cm, bottom=2.5cm}
\newcommand{\ie}{\emph{i.e.,}\xspace}
\newcommand{\eg}{\emph{e.g.,}\xspace}
\newcommand{\wrt}{\emph{w.r.t.}\xspace}
\newcommand{\aka}{\emph{a.k.a.}\xspace}
\newcommand{\kwlog}{\emph{w.l.o.g.}\xspace}
\newcommand{\etal}{\emph{et al.}\xspace}
\newcommand{\sstab}{\rule{0pt}{8pt}\\[-2.4ex]}

\newcommand{\topk}[1]{\kw{top}--\kw{#1}}
\newcommand{\topdown}{\kw{topDown}}
\newcommand{\extsubgraph}{\kw{compADS^+}}
\newcommand{\drfds}{\kw{FIDES^+}}
\newcommand{\extsubgraphold}{\kw{compADS}}
\newcommand{\findtimax}{\kw{maxTInterval}}
\newcommand{\findtimin}{\kw{minTInterval}}
\newcommand{\meden}{\kw{MEDEN}}

\newcommand{\tranformgraph}{\kw{convertAG}}
\newcommand{\mergecc}{\kw{strongMerging}}
\newcommand{\strongpruning}{\kw{strongPruning}}
\newcommand{\boundedprobing}{\kw{boundedProbing}}

\newcommand{\AFPR}{\kw{AFP}-\kw{reduction}}
\newcommand{\nwm}{{\sc nwm}\xspace}


\newcommand{\cone}[1]{{$\mathcal{C}{#1}$}}
\renewcommand{\circle}[1]{{$\mathcal{O}{#1}$}}
\newcommand{\pcircle}[1]{{$\mathcal{O}^c{#1}$}}

\newcommand{\vv}{\overrightarrow}

\begin{document}



Prof. {Ren{\'{e}}e J. Miller} \\
Editor-in-Chief		\\
The VLDB Journal	\\



Dear Prof. Miller,

Attached please find a revised version of our submission to
the VLDB Journal, \emph{One-Pass Trajectory Simplification Using the Synchronous Euclidean Distance}.


{The paper has been substantially revised according to the referees' comments. In particular, we have (a) re-introduced SED using plain English, and showed the significant difference between SED and PED, (b) revised the paper to meet those comments that had ever been addressed majorly in the previous response letter rather than in the paper itself, and (c) added more in-depth discussion on experimental results.}

We would like to thank all the referees for their thorough reading of our paper and for their valuable comments.

Below please find our detailed responses to the comments.



%******************* reviewer 1 ***********************************************
\line(1,0){500}

\textbf{Response to the comments of Reviewer 1.}

\line(1,0){100}


\textbf{[R1Ca]} \emph{The scholarship in the paper remains very sloppy and I do not find that {R2W1} or {R3C1} were addressed adequately in the revision only rather begrudgingly in the response letter. Without this, the paper remains a delta off their PED work [15]. The novelty rests on an adequate justification for SED over PED.}

Yes, as you have pointed out, the novelty of the work rests on an adequate justification for SED over PED. On this issue, referees pointed out that we did not adequately tell the difference between PED and SED (\eg R3C1), which further hindered readers to catch the contributions of the work (\eg R2W1). Though we had addressed these problems in the previous response letter and the paper itself from our points of view, it seems that this revision is still inadequate. For convenience, we first list R3C1 and R2W1 here.

\begin{itemize}
  \item {{[R3C1]} \emph{From the example in Fig. 1, it is really difficult to tell the difference between PED and SED. It seems that SED will use more line segments due to the given constraint enforced by epsilon. However, why is the compression using SED better than that using PED? If PED is good enough, why do we care about SED? Essentially, the motivation of leaning towards SED is never clear from a practical perspective. I think it all depends on the tolerance at the application level. So I would like the authors to evaluate some real applications on top of the compressed trajectory data, to demonstrate the benefits of using SED. At least, the authors should give some convincing arguments on that.}}
  \item {{[R2W1]} \emph{The contribution of the paper seems to be limited; the key idea is to approximate circles intersection with polygons intersection to determine when to start a new line segment in the summarized trajectory.}}
\end{itemize}

Now, we have further revised the paper for the above two comments, as below:

(1) For R3C1, we have rewritten the 2nd and 3rd paragraphs of Section 1 to a large extent to better explain SED and show the benefits of using SED compared with PED.
More specifically, in the 2nd paragraph, we have introduced PED and explained why it is not suitable for applications like spatio-temporal queries; then in the 3rd paragraph, we have introduced the key idea of SED and explained why it is more suitable for spatio-temporal queries.
We have also added a new figure (Fig.1 in the revised paper) to help readers understand them.

In the 5th and 6th paragraphs of Section 1, we have further summarized the key difference between one-pass line simplification algorithms using SED and PED, \ie PED works in a 2D space while SED works in a spatio-temporal 3D space, which make it a brand new work to design a one-pass line simplification algorithm using SED.

(2) For R2W1, we have rewritten the contributions of the paper in Section 1, and merged the responses of {R2W1} into the revised contributions (contributions 1 and 2). Note the above new introduction and comparison of SED and PED also reveal the significance and innovativeness of the work.

Thanks very much for your advices!


\textbf{[R1Cb]} \emph{Too many of the reviewer comments were addressed exclusively in response letter, not in the paper itself (e.g., {R3C4-2}).}

We have checked the comments and responses in the previous response letter, and confirmed that most comments had been addressed in both response letter and the paper except three comments, \ie~{R2W1}, {R2C7} and {R3C4-2}, that were majorly addressed in the response letter only. For R2W1, please refer to R1Ca. For R2C7 and R3C4-2, we first list them as below:
\begin{itemize}

  \item {{[R2C7]} \emph{Theorem 8 is not really needed.}}

  \item {{[R3C4-2]} \emph{Also, I am not convinced by just using compression ratio as the single metric for measuring effectiveness. As I mentioned, I recommend using the compressed data to evaluate some real applications.}
}
\end{itemize}

Now, we have further revised the paper according to the two comments:

(1) For {R2C7}, we have clarified the importance of Theorem 8 (Theorem 7 in the revised version) at the begin of Section 4, where the response to R2C7 in the previous response letter is absorbed.

(2) For {R3C4-2}, we have clarified the reason we use compression ratio, error and running time as the major metrics at the end of Section 1 (before \emph{Organization}).

Thanks for pointing out this!





%******************* reviewer 2 ***********************************************
\line(1,0){500}

\textbf{Response to the comments of Reviewer 2.}

\line(1,0){100}


\textbf{[R2Ca]} \emph{The reviewers have pointed out that the paper lacks intuition from over-formalization. Reading the revised paper, I still see many places where intuition is missing.}
\emph{(1) When introducing SED in the introduction (3rd paragraph), instead of defining SED using notation, it would be better to define it using plain English.}
\emph{(2) In Sec. 2.1 (2nd paragraph on Page 4), the author sate ``Intuitively..." but what's following is not intuitive at all.}
\emph{Both are pointed out by {R3C2}.}

Yes, we have rewritten the 3rd paragraph of Section 1 to a large extent to better explain SED. We have introduced SED using plain English, instead of defining SED using notation, and a new figure (Fig.1 in the new submission) has been added to help readers understand the meaning of SED.
{In Section 2.1 (2nd paragraph on Page 4), we have also introduced SED using plain English in the first, followed by the formal explanation using notations.}
Thanks very much for pointing out this!


\textbf{[R2Cb]} \emph{{R3C1} pointed out that the example in the introduction is not convincing because it is really difficult to tell the difference between PED and SED. In the response, the authors explained the motivation of SED, however, did not directly address the comment. From the example (Fig. 1), it is still hard to tell the difference. Ideally, I wish the motivating example could illustrate the ``unbounded" error of using PED (claimed at the end of 2nd paragraph in Sec. 1), and allow the readers to observe that SED provides a huge improvement over PED.}

We have add a new figure (Fig.1 in the new submission) to explain and compare PED and SED. More specifically, we have rewritten the 2nd paragraph of Section 1 to introduce the character of PED and to explain why it is not suitable for spatio-temporal queries, \ie the ``unbounded" error of using PED in spatio-temporal queries. And in the revised 3rd paragraph of Section 1, we have also explained why SED is more suitable for spatio-temporal queries.
Thanks for your great suggestions!

\textbf{[R2Cc]} \emph{The experiments are comprehensive, however, still seem to lack meaningful discussion. For example, I wish to understand why in Fig.16, the average error increases with larger $m$, as this is somewhat counter-intuitive as larger $m$ leads to longer running time (Fig.20). The discussion in Exp-2.1 does not offer an explanation but simply describes what's in the plot. If applicable, I suggest the authors to remove the pervasive ``listings" of numerical results (e.g., ``The average errors of algorithms CISED-S and CISED-W are on average $(119.3\%, 127.7\%, 119.9\%, 138.0\%)$"), as they are already given in the plots, and add more in-depth discussion on why the results are what they are by linking back to the algorithms. }

Yes, we have added ``more in-depth discussion on why the results are what they are by linking back to the algorithms" in Exp\_1.1-(2), Exp\_1.2-(1)(3)(4), Exp\_2.1-(2), Exp\_2.2-(1)(2), Exp\_3.1-(2)(3), Exp\_4.1, Exp\_4.2 and Exp\_4.3 in Section 5.
For the example you mentioned above,  \ie ``why in Fig.16, the average error increases with larger $m$", we have clarified it in Exp\_2.1-(2), \ie ``the increase of edge number $m$ of a polygon lets the polygon more closely approximate its corresponding circle, meaning that some points with larger SED are also included into a line segment, which further leads to a larger average error." {By the way, the average error does not seem to have a significant relationship with the running time, so I really do not catch why it is ``counter-intuitive".}

In addition, the numerical results are to be reserved as it is convenient for readers to refer them. Thanks!


\line(1,0){500}



Your sincerely,

Xuelian Lin, Jiahao Jiang, Shuai Ma, Yimeng Zuo and Chunming Hu



%\bibliographystyle{abbrv}
%\bibliography{sec-ref}


\end{document}
