\section*{{Appendix: Proofs}}


\stitle{Proof of Proposition~\ref{prop-3d-syn-point}}:
It suffices to show that $P'_{s+i}$ is indeed a synchronized point $P_{s+i}$ \wrt $\vv{P_sQ}$.
%
The intersection point $P'_{s+i}$ satisfies that
\textcolor{blue}{
(1) $P'_{s+i}.t = P_{s+i}.t$,
(2) $P'_{s+i}.x$ = $P_s.x +  c\cdot(Q.x - P_s.x)$, and
(3) $P'_{s+i}.y$ = $P_s.y +  c\cdot(Q.y - P_s.y)$,
where $c= \frac{P'_{s+i}.t - P_{s}.t}{Q.t - P_{s}.t}= \frac{P_{s+i}.t-P_s.t}{Q.t-P_s.t}$.
}%
Hence, by the definition of synchronized points, we have the conclusion. \eop


\stitle{Proof of Proposition~\ref{prop-3d-ci}}:
Let $P'_{s+i}$ ($i\in[1, k]$) be the intersection point of line segment $\vv{P_sQ}$ and the plane $P.t - P_{s+i}.t$ = $0$.
By Proposition~\ref{prop-3d-syn-point}, $P'_{s+i}$ is the synchronized point of $P_{s+i}$ \wrt line segment $\vv{P_sQ}$.

Assume first that $\bigsqcap_{i=1}^{k}$\cone{(P_s, \mathcal{O}(P_{s+i}, \epsilon))} $\ne \{P_s\}$. Then there must exist a point $Q $ in the area of the  synchronous circle \circle{(P_{s+k}, \epsilon)} such that $\vv{P_sQ}$ passes through all the cones \cone{(P_s, \mathcal{O}(P_{s+i}, \epsilon))} $i\in[1, k]$. Hence,  $Q.t = P_{s+k}.t$.
We also have $sed(P_{s+i}, \vv{P_sQ}) = |\vv{P'_{s+i}P_{s+i}}| \le \epsilon$ for each $i \in [1, k]$  since $P'_{s+i}$  is in the area of circle  \circle{(P_{s+i}, \epsilon)}.

Conversely, assume that there exists a point $Q$ such that $Q.t = P_{s+k}.t$ and $sed(P_{s+i}, \vv{P_sQ})\le\epsilon$ for all $P_{s+i}$ ($i \in [1,k]$). Then $|\vv{P'_{s+i}P_{s+i}}| \le \epsilon$ for all $i \in [1, k]$. Hence, we have  $\bigsqcap_{i=1}^{k}$\cone{(P_s, \mathcal{O}(P_{s+i}, \epsilon))} $\ne \{P_s\}$. \eop


\stitle{Proof of Proposition~\ref{prop-circle-intersection}}:
By Proposition~\ref{prop-3d-ci}, it suffices to show that $\bigsqcap_{i=1}^{k}$ \pcircle{(P^c_{s+i}, r^c_{s+i})} $\ne \emptyset$ if and only if $\bigsqcap_{i=1}^{k}$\cone{(P_s, \mathcal{O}(P_{s+i}, \epsilon))}$\ne \{P_s\}$, which is obvious. Hence, we have the conclusion. \eop



\stitle{Proof of Proposition~\ref{prop-rp-intersection}}:
We shall prove this by contradiction.
Assume first that $\mathcal{R}^*_{l} \bigsqcap \mathcal{R}_{s+l+1}$ has more than $m$ edges. Then it must have two distinct edges $\vv{A_i}$ and $\vv{A_{i'}}$  with the same label $j$ $(1\le j \le m)$, originally from
$\mathcal{R}_{s+i}$ and $\mathcal{R}_{s+i'}$  ($1\le i< i' \le l+1$).
%
Note that here $\mathcal{R}_{s+i} \bigsqcap \mathcal{R}_{s+i'} \ne \emptyset$ since $\mathcal{R}^*_l \bigsqcap \mathcal{R}_{s+l+1} \ne \emptyset$.
%
However, when $\mathcal{R}_{s+i} \bigsqcap \mathcal{R}_{s+i'} \ne \emptyset$, the intersection $\mathcal{R}_{s+i} \bigsqcap \mathcal{R}_{s+i'}$ cannot have
both edge $\vv{A_i}$ and edge $\vv{A_{i'}}$, as  all edges with the same label are in parallel (or overlapping) with each other by the above definition of inscribed regular polygons. This contradicts with the assumption. \eop


\stitle{Proof of Proposition~\ref{prop-cpi-time}}:
The inscribed regular polygon $\mathcal{R}_{s+l+1}$ has $m$ edges, and intersection polygon $\mathcal{R}^*_l$ has at most $m$ edges by Proposition~\ref{prop-rp-intersection}.
As the intersection of two $m$-edges convex polygons can be computed in $O(m)$ time~\cite{ORourke:Intersection}, the intersection of polygons $\mathcal{R}^*_l$ and $\mathcal{R}_{s+l+1}$ can be done in $O(1)$ time for a fixed $m$. \eop


\stitle{Proof of Proposition~\ref{prop-rule1}}:
We first explain how the edge $\vv{A}$ advances.
Indeed, $\vv{A}$ is moved from its original position to its symmetric edge on $\mathcal{R}_{s+l+1}$ \wrt the symmetric line that is perpendicular to $\vv{B}$  on $\mathcal{R}^*_{l}$.
For example, in Figure~\ref{fig:r-poly-rule1}.(1), there is $\vv{A} \bigsqcap \vv{B} \ne \emptyset$ \And $\vv{A} \times \vv{B} \ge 0$ \And $P_{e_B} \in \mathcal{H}(\vv{A})$, hence $\vv{A}$ moves on. As $g(\vv{B})=3 > 1=g(\vv{A})$, $\vv{A}$ moves forward $2\times(g(\vv{B}) - g(\vv{A}))$ = $2\times(3-1)= 4$ steps.
Here, the label of edge $\vv{A}$ is changed to $5$, its symmetric edge $1$ on $\mathcal{R}_{s+l+1}$ \wrt the symmetric line that is perpendicular to $\vv{B}$ labeled with $3$  on $\mathcal{R}^*_{l}$.


We then present the proof.
If ($\vv{A} \bigsqcap \vv{B} \ne \emptyset$ \And $\vv{A} \times \vv{B} < 0$ \And $P_{e_A} \not \in \mathcal{H}(\vv{B})$) or ($\vv{A} \bigsqcap \vv{B} \ne \emptyset$ \And $\vv{A} \times \vv{B} \ge 0$ \And $P_{e_B} \in \mathcal{H}(\vv{A})$), then as all edges in the same edge groups $E^j$ ($1\le j\le m$) are in parallel with each other and by the geometric properties of regular polygon $\mathcal{R}_{s+k+1}$, it is easy to find that, for each position of $\vv{A}$ between its original to its opposite positions, we have (1) $\vv{A} \bigsqcap \vv{B} = \emptyset$, and (2) either $P_{e_A} \not \in \mathcal{H}(\vv{B})$ or $P_{e_B} \in \mathcal{H}(\vv{A})$. Hence, by the advance rule (1) of algorithm \cpia in Section~\ref{subsec-cpi}, edge $\vv{A}$ is always moved forward until it reaches the opposite position of its original one. From this, we have the conclusion. \eop




\stitle{Proof of Proposition~\ref{prop-rule2}}:
We first explain how the edge $\vv{B}$ is moved forward.
For example, in Figure~\ref{fig:r-poly-rule1}.(2), $\vv{A} \bigsqcap \vv{B} \ne \emptyset$ \And $\vv{A} \times \vv{B} < 0$ \And $P_{e_A} \in \mathcal{H}(\vv{B})$, hence $\vv{B}$ is moved forward. As the edge $\vv{A}$ is labeled with 7,
$\vv{B}$ moves to the edge labeled with 8 on $\mathcal{R}^*_{l}$, which is the next of the edge labeled with 7 on $\mathcal{R}^*_{l}$.
Note that if the edge labeled with 8 were not actually existing in the intersection polygon $\mathcal{R}^*_{l}$, then $\vv{B}$ should repeatedly move on until it reaches the first ``real" edge on $\mathcal{R}^*_{l}$.

We then present the proof.
If ($\vv{A} \bigsqcap \vv{B} \ne \emptyset$ \And $\vv{A} \times \vv{B} \ge 0$ \And $P_{e_B} \not \in \mathcal{H}(\vv{A})$) or ($\vv{A} \bigsqcap \vv{B} \ne \emptyset$ \And $\vv{A} \times \vv{B} < 0$ \And $P_{e_A} \in \mathcal{H}(\vv{B})$), then it is also easy to find that, for each position of $\vv{B}$ between its original to its target positions (\ie the edge after the one having the same edge group as $\vv{A}$), we have (1) $\vv{A} \bigsqcap \vv{B} = \emptyset$, and (2) either $P_{e_B} \not \in \mathcal{H}(\vv{A})$ or $P_{e_A} \in \mathcal{H}(\vv{B})$. Hence, by the advance rule (2) of algorithm \cpia in Section~\ref{subsec-cpi}, edge $\vv{B}$ is always moved forward until it reaches the target position. From this, we have the conclusion. \eop



\stitle{Proof of Proposition~\ref{prop-3d-ci-half}}:
If $\bigsqcap_{i=s+1}^{e}{\mathcal{C}(P_s, P_{s+i}, \epsilon/2)} \ne \{P_s\}$, then by Proposition~\ref{prop-3d-ci}, there exists a point $Q$, $Q.t = P_{s+k}.t$, such that $sed(P_{s+i}, \vv{P_sQ})\le \epsilon/2$ for all $i \in [1,k]$. By the triangle inequality essentially, $sed(P_{s+i}, \vv{P_sP_{s+k}})\le  sed(P_{s+i}, \vv{P_sQ}) + |\vv{QP_{s+k}}| \le  \epsilon/2+\epsilon/2 = \epsilon$. \eop


\stitle{Proof of Proposition~\ref{prop-cist-Q}}:
By Proposition~\ref{prop-circle-intersection} and the nature of inscribed regular polygon, it is easy to find that for any point $Q \in \mathcal{R}^*_k$  \wrt plane $t_c=P_{s+k}.t$, there is $sed(P_{s+i}, \vv{P_sQ})\le \epsilon$ for all points $P_{s+i}$ ($i \in [1,k]$).
From this, we have the conclusion. \eop

%\begin{proof}\ 
%\end{proof}