\subsection{Approximate circles intersection}

The checking of the $N$ circles intersection has a time complexity of ${O(N\log N)}$~\cite{Shamos:Circle}, which is not the linear time.
To develop a linear time complexity spatio-temporal simplification algorithm, we provide an approximate circles intersection method.
That is, we further replace a circle $\mathcal{O}(P, r)$ to a $M$-edges inscribed regular polygon $\mathcal{G}(V, E)$, where $V=\{v_1,
\ldots, v_{M}\}$ is the set of vertexes and $E= \{\overline{v_jv_{j+1}}| j\in [1,M-1]\} \bigcup \{\overline{v_Mv_1}\}$ is the set of edges.
Furthermore, if we let the center point $P$ of the circle be the origin of polar coordinate system, then each vertex satisfies:

\vspace{-2ex}
\begin{equation*}
\label{equ-regular-polygon}
%\hspace{-1.5ex}
    \begin{aligned}
        \hspace{10ex}  v_j = (r, \frac{2(j-1)}{M}\pi), ~j \in [1, M]    \hspace{15ex} (3)\\
    \end{aligned}
\end{equation*}
\vspace{-2ex}




For example, Figure~\ref{fig:polygons}-(1) is a regular octagon whose vertexes satisfying the Equation (3).

%
Then we check the intersection of the regular polygons instead of the checking of circles intersection.
%The intersection of two M-edges convex polygons has a time complexity of $O(M)$ \cite{ORourke:Intersection}.
The regular polygons built by equation (3) have elegant properties, which is helpful to derive an algorithm of linear time and constant space
for the intersection of $N$ regular polygons.


\begin{figure}[tb!]
\centering
\includegraphics[scale=0.9]{figures/Fig-polygons.png}
\vspace{-1ex}
\caption{\small Regular octagons and their intersections.}
\vspace{-3ex}
\label{fig:polygons}
\end{figure}



\begin{theorem}
\label{prop-rp-intersection}
If $\mathcal{G}_i$, $i \in [1, k]$, are M-edges regular polygons on a plane which are built by equation (3), then the intersection polygon
$\mathcal{G}^*$ of all $\mathcal{G}_i$ has no more than $M$ edges.
\end{theorem}


\begin{proof}
Let $e_{i,j} = \overline{v_{i,j}v_{i,j+1}}$, $j\in [1,M)$, or $e_{i,j} = \overline{v_{i,M}v_{i,1}}$, $j = M$, be an edge of regular polygons
$\mathcal{G}_i$, then all $e_{i,j}$, $i\in [1, k]$ form an \emph{edge group}, namely the $j^{th}$ edge group (see
Figure~\ref{fig:polygons}-(2)).
We first prove that the intersection polygon $\mathcal{G}^*$ includes at most one edge from an edge group, \eg the $j^{th}$ edge group.
We prove this by contradiction.

Suppose $\mathcal{G}^*$ have two different edges, $e_{h,j}$ and $e_{l,j}$, $h\ne l$, belonging to the $j^{th}$ edge group. Obviously, edge
$e_{h,j}$ belongs to regular polygon $\mathcal{G}_h$, edge $e_{l,j}$ belongs to regular polygon $\mathcal{G}_l$ and $e_{h,j}$ is parallel to
$e_{l,j}$.

If $\mathcal{G}_l \bigcap \mathcal{G}_h = \phi$, then $\mathcal{G}^*=\phi$, which conflicts with the hypothetic premises.

If $\mathcal{G}_l \bigcap \mathcal{G}_h \ne \phi$, then the intersection polygon of $\mathcal{G}_h$ and $\mathcal{G}_l$ must include at most
one edge of $e_{h,j}$ and $e_{l,j}$, which is also the contradiction with the hypothetic premises.

Hence, the intersection polygon $\mathcal{G}^*$ includes at most one edge of an edge group.
%
There are at most $M$ edge groups, thus, the intersection polygon $\mathcal{G}^*$ has no more than $M$ edges.
\end{proof}

Note that the intersection of regular polygons is only a sufficient condition of the intersection of the corresponding circles , and
algorithms employ this policy may loss some compression ratio. Obviously, a larger $M$ leads to a better compression ratio and a poorer
efficiency.

