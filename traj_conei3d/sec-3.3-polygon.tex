\subsection{Approximate circles intersection}

%\textcolor[rgb]{1.00,0.00,0.00}{Todo....}
The checking of the $N$ circles intersection has a time complexity of \textcolor[rgb]{1.00,0.00,0.00}{${O(N\log N)}$~\cite{Shamos:Circle}}, which is not a linear time.
To develop a linear time complexity spatio-temporal simplification algorithm, we provide an approximate circles intersection method.
That is, we further replace a circle $\mathcal{O}(P, r)$ to a $M$-edges inscribed regular polygon, 
then we check the intersection of the regular polygons instead of the checking of circles intersection.
Note that the intersection of regular polygons is only a sufficient condition of the intersection of the corresponding circles , and
algorithms employ this policy may loss some compression ratio. Obviously, a larger $M$ leads to a better compression ratio and a poorer
efficiency.

\stitle{Approximate polygons}.
Given a circle $\mathcal{O}(P, r)$, its approximate polygons is a $M$ edges inscribed regular polygon $\mathcal{G}(V, E)$, 
where $V=\{v_1, \ldots, v_{M}\}$ is the set of vertexes and 
$E= \{\overline{v_jv_{j+1}}| j\in [1,M-1]\} \bigcup \{\overline{v_Mv_1}\}$ is the set of edges.
Furthermore, let the center point $P$ of the circle be the origin of a polar coordinate system, then each vertex satisfies:

\vspace{-2ex}
\begin{equation*}
\label{equ-regular-polygon}
%\hspace{-1.5ex}
    \begin{aligned}
        \hspace{10ex}  v_j = (r, \frac{2(j-1)}{M}\pi), ~j \in [1, M]    \hspace{10ex} (3)\\
    \end{aligned}
\end{equation*}
\vspace{-2ex}

For example, Figure~\ref{fig:polygons}-(1) is a regular octagon whose vertexes satisfying the Equation (3).

%The intersection of two M-edges convex polygons has a time complexity of $O(M)$ \cite{ORourke:Intersection}.
The regular polygons built by equation (3) have elegant properties, which is helpful to derive an algorithm of linear time and constant space
for the intersection of $N$ regular polygons.


\stitle{Edge groups}.
Let $e_{i,j} = \overline{v_{i,j}v_{i,j+1}}$, $j\in [1,M)$, or $e_{i,j} = \overline{v_{i,M}v_{i,1}}$, $j = M$, be an edge of an approximate polygon
$\mathcal{G}_i$, then all $e_{i,j}$, $i\in [1, k]$, form an \emph{edge group}, namely the $j^{th}$ edge group.
For example, in Figure~\ref{fig:polygons}-(2), all the $1^{th}$ edges form the $1^{th}$ edge group.




\begin{figure}[tb!]
\centering
\includegraphics[scale=0.9]{figures/Fig-polygons.png}
\vspace{-1ex}
\caption{\small Regular octagons and their intersections.}
\vspace{-3ex}
\label{fig:polygons}
\end{figure}



\begin{theorem}
\label{prop-rp-intersection}
If $\mathcal{G}_i$, $i \in [1, k]$, are M-edges regular polygons on a plane which are built by equation (3), then the intersection polygon
$\mathcal{G}^*_k$ of all $\mathcal{G}_i$ includes at most one edge from an edge group, \eg the $j^{th}$ edge group.
\end{theorem}


\begin{proof}
We prove this by contradiction.

Suppose $\mathcal{G}^*_k$ have two different edges, $e_{h,j}$ and $e_{l,j}$, $h\ne l$, belonging to the $j^{th}$ edge group. Obviously, edge
$e_{h,j}$ belongs to regular polygon $\mathcal{G}_h$, edge $e_{l,j}$ belongs to regular polygon $\mathcal{G}_l$ and $e_{h,j}$ is parallel to
$e_{l,j}$.

If $\mathcal{G}_l \bigcap \mathcal{G}_h = \phi$, then $\mathcal{G}^*_k=\phi$, which conflicts with the hypothetic premises.

If $\mathcal{G}_l \bigcap \mathcal{G}_h \ne \phi$, then the intersection polygon of $\mathcal{G}_h$ and $\mathcal{G}_l$ must include at most
one edge of $e_{h,j}$ and $e_{l,j}$, which is also the contradiction with the hypothetic premises.

Hence, the intersection polygon $\mathcal{G}^*_k$ includes at most one edge of an edge group.
%
%There are at most $M$ edge groups, thus, the intersection polygon $\mathcal{G}^*$ has no more than $M$ edges.
\end{proof}



%\begin{cor}
%\label{prop-rp-intersection}
%If $\mathcal{G}_i$, $i \in [1, k]$, are M-edges regular polygons on a plane which are built by equation (3), then the intersection polygon
%$\mathcal{G}^*$ of all $\mathcal{G}_i$ has no more than $M$ edges.
%\end{cor}


By Theorem~\ref{prop-rp-intersection}, the intersection polygon $\mathcal{G}^*_k$ of all approximate polygon $\mathcal{G}_i$, $i \in [1, k]$, has no more than $M$ edges, thus, the computing of the intersection polygon of polygon $\mathcal{G}^*_k$ and approximate polygon $\mathcal{G}_{k+1}$ can be implemented in a constant time, \ie time $O(M)$, by the distinct convex polygon intersection algorithm presented in~ \cite{ORourke:Intersection}.
%
Moreover, observe that the convex polygon intersection algorithm of \cite{ORourke:Intersection} is a algorithm for general convex polygons, while the approximate polygons are special polygons. We further develop an even fast approximate polygon intersection algorithm based on the properties of approximate polygons.
  
\subsubsection{Approximate polygon intersection}







