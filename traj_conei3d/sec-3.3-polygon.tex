
\subsection{A fast regular polygon intersection method}


Observe that the algorithm of ~\cite{ORourke:Intersection} is for general convex polygons, while the approximate polygons are special polygons. We further develop an even fast approximate polygon intersection method based on the properties of approximate polygons.
%
The presented approximate polygon intersection algorithm is indeed a customized version of the convex polygon intersection algorithm of ~\cite{ORourke:Intersection}. The major optimization is it allows the algorithm to move $A$ or $B$ multiple steps one time.
%


Let $A$ and $B$ be directed edges on $\mathcal{G}_{k+1}$ and $\mathcal{G}^*_k$ respectively. The algorithm has $A$ and $B$ ``chasing" one another in the counterclockwise order like the algorithm ~\cite{ORourke:Intersection}.
%
Here, the advancing rules are four categories, \ie $A \bigcap B \ne \phi$ and advances $A$, $A \bigcap B \ne \phi$ and advances $B$, $A \bigcap B = \phi$ and advances $A$, and $A \bigcap B = \phi$ and advances $B$.
%
In each of these four cases, it allows $A$ or $B$ moves on multiple steps, as follows.



\emph{Rule 1: 
If $A \bigcap B \ne \phi$ and advances $B$, then moves on $B$ to the next edge of $g(A)$.}




\emph{Rule 2: 
If $A \bigcap B \ne \phi$ and advances $A$, then moves on $A$ ``s" steps, where}
\vspace{-1ex}
\begin{equation*}
\small
\label{equ-rule1}
    \hspace{2ex} s =  \left\{
    \begin{aligned}
        & 2*(g(A) - g(B)),  \hspace{5ex} ~if  ~g(A) > g(B) \\
        & \textcolor[rgb]{1.00,0.00,0.00}{1}, \hspace{21ex}\textcolor[rgb]{1.00,0.00,0.00}{if  ~g(A) = g(B)} \\
        & 2*(M+g(A) - g(B)), ~if  ~g(A) < g(B) \\
    \end{aligned}
    \right.       \hspace{6ex}(4)
\end{equation*}
\vspace{-1ex}


\emph{Rule 3:
If $A \bigcap B = \phi$ and advances $A$, then moves on $B$ ``" steps.}


\emph{Rule 4:
If $A \bigcap B = \phi$ and advances $B$, then moves on $B$ ``" steps.}



%%%%%%%%%%%%%%%%%%%%%Baseline Algorithm
\begin{figure}[tb!]
\begin{center}
{\small
\begin{minipage}{3.36in}
\myhrule
\vspace{-1ex}
\mat{0ex}{
	{\bf Algorithm} ~$RegularPolyInter$($\mathcal{G}^*_k$, $\mathcal{G}_{k+1}$) \\
	\bcc \hspace{2ex}\=  Set $A$ and $B$ {arbitrarily} on $\mathcal{G}^*_k$ and $\mathcal{G}_{k+1}$\\
	\icc \>\hspace{0ex}\= Repeat \\
	\icc \>\hspace{3ex} If $A$ intersects $B$ Then \\
	\icc \>\hspace{6ex} {Check for termination.} \\
	\icc \>\hspace{6ex} Update an inside flag. \\
	\icc \>\hspace{6ex} \textcolor[rgb]{0.00,0.00,1.00}{Moves on either $A$ or $B$ under rule 1 or 2.}\\
	\icc \>\hspace{3ex} Else \\
	\icc \>\hspace{6ex} \textcolor[rgb]{0.00,0.07,1.00}{Moves on either $A$ or $B$ under rule 3 or 4.}\\
	\icc \hspace{1ex} Until both $A$ and $B$ cycle their polygons \\
	\icc \hspace{0ex} Handle $\mathcal{G}^*_k \subset \mathcal{G}_{k+1}$ and $\mathcal{G}^*_k \subset \mathcal{G}_{k+1}$ and $\mathcal{G}^*_k \bigcap \mathcal{G}_{k+1} = \phi$ cases \\
    \icc \hspace{0ex} Return $\mathcal{G}^*_k \bigcap \mathcal{G}_{k+1}$
}
\vspace{-2ex}
\myhrule
\end{minipage}
}
\end{center}
\vspace{-2ex}
\caption{\small Intersection of Regular polygons.}
\label{alg:r-poly-inter}
\vspace{-2ex}
\end{figure}
%%%%%%%%%%%%%%%%%%%%%%%%%%%%%%%%%%%%%


\stitle{\textcolor[rgb]{0.00,0.07,1.00}{Regular polygon intersection algorithm}}
Given intersection polygon $\mathcal{G}^*_k$ of the preview $k$ polygons and the next approximate polygon $\mathcal{G}_{k+1}$, the algorithm returns $\mathcal{G}^*_{k+1} = \mathcal{G}^*_k  \bigcap \mathcal{G}_{k+1}$.
The sketch of the algorithm is shown in Figure~\ref{alg:r-poly-inter}.

%Let $A$ and $B$ be directed edges on $\mathcal{G}^*_k$ and $\mathcal{G}_{k+1}$ respectively. The algorithm has $A$ and $B$ ``chasing" one another. During the chasing, $A$ or $B$ moves on step by step in the counterclockwise order, following some rules designed depending on geometric conditions, as described in ~\cite{ORourke:Intersection}.






\begin{figure}[tb!]
\centering
\includegraphics[scale=0.88]{figures/Fig-r-poly-inter.png}
\vspace{-1ex}
\caption{\small A running example of intersection of polygons.}
\vspace{-2ex}
\label{fig:r-poly-inter}
\end{figure}




\begin{example}
Figure~\ref{fig:r-poly-inter} is a running example of the approximate polygon intersection algorithm.
Initially, the candidate edges are marked and directed edges $A$ and $B$ are on polygons $\mathcal{G}_{k+1}$ and $\mathcal{G}^*_{k}$ separately

\ni (1) Initially, directed edges $A$ and $B$ are on polygons $\mathcal{G}_{k+1}$ and $\mathcal{G}^*_{k}$ separately. $A \bigcap B = P_1$.

\ni (2) $A$ moves forward a 4-steps. $A \bigcap B = \phi$.

\ni (3) After 4 steps of moving, $A \bigcap B = P_2$

\ni (4) $B$ moves forward a 2-steps. $A \bigcap B = \phi$.

\ni (5) After 5 steps of moving, both $A$ and $B$ cycle their polygons;

\ni (6) At last, it returns the intersection polygon.
\end{example}






