\subsection{Speedup Inscribed Regular Polygon Intersection}
\label{subsec-fastRPI}


Observe that algorithm \cpia in Figure~\ref{alg:c-poly-inter} is for general convex polygons,
while the inscribed regular polygons $\mathcal{R}_{s+i}$ ($i\in[1, k]$) of the cone projection circles are constructed in a unified way,
which allows us to develop a fast method to compute their intersection.

Let $\vv{A} = (P_{s_A}, P_{e_A})$ and $\vv{B} = (P_{s_B}, P_{e_B})$  be two directed edges on polygons $\mathcal{R}_{s+l+1}$ and $\mathcal{R}^*_{l}$, respectively.
Again edges $\vv{A}$ and $\vv{B}$ are moved counter-clockwise. Note that $\vv{A}$ and $\vv{B}$ are advanced step by step each time by the two advancing rules of algorithm \cpia.
%
However, it is possible to advance $\vv{A}$ or $\vv{B}$ multiple steps each time.
%
For example, in Figure~\ref{fig:c-poly-inter}.(1)--(5), edge $\vv{A}$ successively moves four steps, each under the advance rule (1) ``($\vv{A} \times \vv{B} < 0$ and $P_{e_A} \not \in \mathcal{H}(\vv{B})$) or ($\vv{A} \times \vv{B} \ge 0$ and $P_{e_B} \in \mathcal{H}(\vv{A})$)'' of algorithm \cpia.
Alternatively, we can directly move $A$ from Figure~\ref{fig:c-poly-inter}.(1) to Figure~\ref{fig:c-poly-inter}.(5), by reducing four steps to one step only.



\begin{prop}
\label{prop-rule1}
If either $(\vv{A} \bigsqcap \vv{B} \ne \emptyset$ \And $\vv{A} \times \vv{B} < 0$ \And $P_{e_A} \not \in \mathcal{H}(\vv{B}))$ or $(\vv{A} \bigsqcap \vv{B} \ne \emptyset$ \And $\vv{A} \times \vv{B} \ge 0$ \And $P_{e_B} \in \mathcal{H}(\vv{A}))$ holds, then $\vv{A}$ advances $s$ steps such that

\vspace{-1ex}
\begin{equation*}
\label{equ-rule1}
\small
    \hspace{2ex} s =  \left\{
    \begin{aligned}
        & 2\times(g(\vv{B}) - g(\vv{A}))  \hspace{5ex}~~if  ~g(\vv{B}) > g(\vv{A}) \\
        & {1}              \hspace{21ex}~if  ~g(\vv{A}) = g(\vv{B}) \\
        & 2\times(m+g(\vv{B}) - g(\vv{A})) ~~if  ~g(\vv{B}) < g(\vv{A}), \\
    \end{aligned}
    \right.       \hspace{6ex}{}
\end{equation*}
in which $g(e)$ denotes the label of edge $e$.
\end{prop}



\begin{proof}\ 
We first explain how the edge $\vv{A}$ advances.
Indeed, $\vv{A}$ is moved from its original position to its symmetric edge on $\mathcal{R}_{s+l+1}$ \wrt the symmetric line that is perpendicular to $\vv{B}$  on $\mathcal{R}^*_{l}$.
For example, in Figure~\ref{fig:r-poly-rule1}.(1), there is $\vv{A} \bigsqcap \vv{B} \ne \emptyset$ \And $\vv{A} \times \vv{B} \ge 0$ \And $P_{e_B} \in \mathcal{H}(\vv{A})$, hence $\vv{A}$ moves on. As $g(\vv{B})=3 > 1=g(\vv{A})$, $\vv{A}$ moves forward $2\times(g(\vv{B}) - g(\vv{A}))$ = $2\times(3-1)= 4$ steps.
Here, the label of edge $\vv{A}$ is changed to $5$, its symmetric edge $1$ on $\mathcal{R}_{s+l+1}$ \wrt the symmetric line that is perpendicular to $\vv{B}$ labeled with $3$  on $\mathcal{R}^*_{l}$.


We then present the proof.
If ($\vv{A} \bigsqcap \vv{B} \ne \emptyset$ \And $\vv{A} \times \vv{B} < 0$ \And $P_{e_A} \not \in \mathcal{H}(\vv{B})$) or ($\vv{A} \bigsqcap \vv{B} \ne \emptyset$ \And $\vv{A} \times \vv{B} \ge 0$ \And $P_{e_B} \in \mathcal{H}(\vv{A})$), then as all edges in the same edge groups $E^j$ ($1\le j\le m$) are in parallel with each other and by the geometric properties of regular polygon $\mathcal{R}_{s+k+1}$, it is easy to find that, for each position of $\vv{A}$ between its original to its opposite positions, we have (1) $\vv{A} \bigsqcap \vv{B} = \emptyset$, and (2) either $P_{e_A} \not \in \mathcal{H}(\vv{B})$ or $P_{e_B} \in \mathcal{H}(\vv{A})$. Hence, by the advance rule (1) of algorithm \cpia in Section~\ref{subsec-cpi}, edge $\vv{A}$ is always moved forward until it reaches the opposite position of its original one. From this, we have the conclusion. \eop
\end{proof}



%%%%%%%%%%%%%%%%%%%%%%%%%%%%%%%%%%%%%%%%
\begin{prop}
\label{prop-rule2}
If either ($\vv{A} \bigsqcap \vv{B} \ne \emptyset$ \And $\vv{A} \times \vv{B} \ge 0$ \And $P_{e_B} \not \in \mathcal{H}(\vv{A})$) or ($\vv{A} \bigsqcap \vv{B} \ne \emptyset$ \And $\vv{A} \times \vv{B} < 0$ \And $P_{e_A} \in \mathcal{H}(\vv{B})$) holds, then edge $\vv{B}$ is directly moved to the edge after the one having the same edge group as edge $\vv{A}$.
\end{prop}

\begin{proof}\ 
We first explain how the edge $\vv{B}$ is moved forward.
For example, in Figure~\ref{fig:r-poly-rule1}.(2), $\vv{A} \bigsqcap \vv{B} \ne \emptyset$ \And $\vv{A} \times \vv{B} < 0$ \And $P_{e_A} \in \mathcal{H}(\vv{B})$, hence $\vv{B}$ is moved forward. As the edge $\vv{A}$ is labeled with 7,
$\vv{B}$ moves to the edge labeled with 8 on $\mathcal{R}^*_{l}$, which is the next of the edge labeled with 7 on $\mathcal{R}^*_{l}$.
Note that if the edge labeled with 8 were not actually existing in the intersection polygon $\mathcal{R}^*_{l}$, then $\vv{B}$ should repeatedly move on until it reaches the first ``real" edge on $\mathcal{R}^*_{l}$.

We then present the proof.
If ($\vv{A} \bigsqcap \vv{B} \ne \emptyset$ \And $\vv{A} \times \vv{B} \ge 0$ \And $P_{e_B} \not \in \mathcal{H}(\vv{A})$) or ($\vv{A} \bigsqcap \vv{B} \ne \emptyset$ \And $\vv{A} \times \vv{B} < 0$ \And $P_{e_A} \in \mathcal{H}(\vv{B})$), then it is also easy to find that, for each position of $\vv{B}$ between its original to its target positions (\ie the edge after the one having the same edge group as $\vv{A}$), we have (1) $\vv{A} \bigsqcap \vv{B} = \emptyset$, and (2) either $P_{e_B} \not \in \mathcal{H}(\vv{A})$ or $P_{e_A} \in \mathcal{H}(\vv{B})$. Hence, by the advance rule (2) of algorithm \cpia in Section~\ref{subsec-cpi}, edge $\vv{B}$ is always moved forward until it reaches the target position. From this, we have the conclusion. \eop
\end{proof}



%%%%%%%%%%%%%%%%%%%%%example of intersection of regular polygons
\begin{figure}[tb!]
	\centering
	%\vspace{1ex}
	\includegraphics[scale=0.82]{figures/Fig-r-poly-rule1.png}
	\vspace{-1ex}
	\caption{\small Examples of fast advancing rules.}
	\vspace{-1ex}
	\label{fig:r-poly-rule1}
\end{figure}


%%%%%%%%%%%%%%%%%%%%%%%%%%%%%%%%%%%%%%%%



\eat{%%%%%%%%%%%%%%%%%%%%%%%%%%%%%%%%%%%
%%%%%%%%%%%%%%%%%%%%%%%%%%%%%%%%%%%%%%%%
\vspace{1ex}
\ni \emph{\underline{Rule 3}:
If $A \bigcap B = \emptyset$ and advances $A$, then moves $A$ and $B$ forward ``$a$ and $b$" steps respectively, where}
%\vspace{-1ex}
\begin{equation*}
\label{equ-rule3}
\small
    \hspace{0ex} (a,b) =  \left\{
    \begin{aligned}
        & (2,0), ~if~ g(next(A)) = g(B) ~and~ {outside}(A)  \\
        & (1,1), ~if~ g(next(A)) = g(B) ~and~ {inside}(A)\\
        & (1,0), ~else
    \end{aligned}
    \right.       \hspace{2ex}(5)
    %\vspace{-1ex}
\end{equation*}
%\vspace{-1ex}
\emph{and procedure $inside()$ or $outside()$ is a checking of the ``{\emph{inside}}" flag (line 5) of the \cpia algorithm (Figure~\ref{alg:c-poly-inter}).}
\vspace{1ex}


For example, in Figure~\ref{fig:r-poly-inter}-(4), $A \bigcap B = \emptyset$ and advances $A$, hence, rule 3 is applied. Because $A$ is inside and $g(next(A))=3 = g(B)$, $A$ and $B$ both move forward one step (Figure~\ref{fig:r-poly-inter}-(5)).



%%%%%%%%%%%%%%%%%%%%%%%%%%%%%%%%%%%%%%%%
\vspace{1ex}
\ni \emph{\underline{Rule 4}:
If $A \bigcap B = \phi$ and advances $B$, then moves $A$ and $B$ forward ``$a$ and $b$" steps respectively, where}
%\vspace{-1ex}
\begin{equation*}
\label{equ-rule4}
\small
    \hspace{0ex} (a,b) =  \left\{
    \begin{aligned}
        & (0,2), ~if~ g(next(B)) = g(A) ~and~ outside(B)\\
        & (1,1), ~if~ g(next(B)) = g(A) ~and~ inside(B)\\
        & (0,1), ~else
    \end{aligned}
    \right.       \hspace{2ex}(6)
\end{equation*}
\vspace{-1ex}

}%%%%%%%%%%%%%%%%%%%%%%%%%%End of Eat






\stitle{Algorithm \rpia}.
The presented regular polygon intersection algorithm, \ie\ \rpia, is the optimized version  of the convex polygon intersection algorithm \cpia, by Propositions \ref{prop-rule1} and \ref{prop-rule2}. We also save vertexes of a polygon in a fixed size array, which is different from \cpia  that saves polygons in linked lists.
Considering the regular polygons each having a fixed number of vertexes/edges, marked from $1$ to $m$, this policy allows us to quickly address an edge or vertex by its label.
%, as well as avoid the creating and removing of link nodes.

Given intersection polygon $\mathcal{R}^*_{l}$ of the preview $l$ polygons and the next approximate polygon $\mathcal{R}_{s+l+1}$, the algorithm \rpia returns $\mathcal{R}^*_{l+1}=\mathcal{R}^*_{l}  \bigsqcap \mathcal{R}_{s+l+1}$.
%The sketch of the algorithm is shown in Figure~\ref{alg:r-poly-inter}.
It runs the similar routine as the \cpia algorithm, except that (1) it saves polygons in arrays, and (2) the advance strategies are partitioned into two parts, \ie $\vv{A} \bigsqcap \vv{B} \ne \emptyset$ and $\vv{A} \bigsqcap \vv{B} = \emptyset$, where the former applies Propositions \ref{prop-rule1} and \ref{prop-rule2}, and the later remains the same as algorithm \cpia.





\eat{%%%%%%%%%%%%%%%%%%%%%%
\begin{example}
Figure~\ref{fig:r-poly-inter} is a running example of algorithm \rpia. The input is the same as Figure~\ref{fig:c-poly-inter}.

\ni (1) Initially, directed edges $\vv{A}$ and $\vv{B}$ are on polygons $\mathcal{R}_{k+1}$ and $\mathcal{R}^*_{k}$ separately. $\vv{A} \bigcap \vv{B} = P_1$ and $\vv{A}$ moves on.
\ni (2) $\vv{A}$ moves forward a 4-steps, directly from $2^{th}$ to $6^{th}$, under rule 2. Then, $\vv{A} \bigcap \vv{B} = \emptyset$ and $\vv{B}$ moves on.
\ni (3) After 4 steps of moving (in turn), $\vv{A} \bigcap \vv{B} = P_2$ and $\vv{B}$ moves on.
\ni (4) $\vv{B}$ advances a 2-steps, from $6^{th}$ to $3^{th}$, under rule 1. Then $\vv{A} \bigcap \vv{B} = \emptyset$ and $\vv{A}$ moves on.
\ni (5) $\vv{A}$ advances a step. Then $\vv{A} \bigcap \vv{B} = \emptyset$ and $\vv{B}$ moves on.
\ni (6) After 3 steps of moving, both $\vv{A}$ and $\vv{B}$ cycle their polygons. The intersection polygon, the same as the result of \cpia (also see Figure~\ref{fig:c-poly-inter}), is returned.
\end{example}


%%%%%%%%%%%%%%%%%%%%%example of intersection of regular polygons
\begin{figure}[tb!]
\centering
\includegraphics[scale=0.88]{figures/Fig-r-poly-inter.png}
\vspace{-1ex}
\caption{\small A running example of intersection of polygons.}
\vspace{-2ex}
\label{fig:r-poly-inter}
\end{figure}
}


\eat{%%%%%%%%%%%%%%%%%%%%%Algorithm
\begin{figure}[tb!]
\begin{center}
{\small
\begin{minipage}{3.36in}
\myhrule
\vspace{-1ex}
\mat{0ex}{
	{\bf Algorithm} ~\rpia ($\mathcal{R}^*_k$, $\mathcal{R}_{k+1}$) \\
	\bcc \hspace{2ex}\=  Set $\vv{A}$ and $\vv{B}$ {arbitrarily} on $\mathcal{R}^*_k$ and $\mathcal{R}_{k+1}$\\
	\icc \>\hspace{0ex}\= Repeat \\
	\icc \>\hspace{3ex} If $\vv{A} \bigcap \vv{B} \ne \phi$ Then \\
	\icc \>\hspace{6ex} {Check for termination}. \\
	\icc \>\hspace{6ex} Update an {\emph{inside}} flag for $\vv{A}$ or $\vv{B}$. \\
	\icc \>\hspace{6ex} {\emph{Moves on either $\vv{A}$ or $\vv{B}$ under rule 1 or 2.}}\\
	\icc \>\hspace{3ex} Else \\
	\icc \>\hspace{6ex} {{Moves on either $\vv{A}$ or $\vv{B}$.}}\\
	\icc \hspace{1ex} Until both $\vv{A}$ and $\vv{B}$ cycle their polygons \\
	\icc \hspace{0ex} Handle $\mathcal{R}^*_k \subset \mathcal{R}_{k+1}$ and $\mathcal{R}^*_k \subset \mathcal{R}_{k+1}$ and $\mathcal{R}^*_k \bigcap \mathcal{R}_{k+1} = \phi$ cases \\
    \icc \hspace{0ex} Return $\mathcal{R}^*_k \bigcap \mathcal{R}_{k+1}$
}
\vspace{-2ex}
\myhrule
\end{minipage}
}
\end{center}
\vspace{-2ex}
\caption{\small Intersection of Regular polygons.}
\label{alg:r-poly-inter}
\vspace{-2ex}
\end{figure}
}%%%%%%%%%%%%%%%%%%%%%%%%%%%%%%%%%%%%%




\vspace{0.5ex}
\stitle{Correctness and complexity analyses.}
Observe that algorithm \rpia basically has the same routine as algorithm \cpia, except that it fastens the advancing speed of directed edges $\vv{A}$ and $\vv{B}$ under certain circumstances as shown by Propositions \ref{prop-rule1} and \ref{prop-rule2}, which together ensure the correctness of \rpia. Moreover, algorithm \rpia runs in $O(1)$ time
by Proposition~\ref{prop-cpi-time}.


