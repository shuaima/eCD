\documentclass{letter}
\usepackage{geometry}

% duan
\usepackage{xspace}

\geometry{left=2.0cm, right=2.0cm, top=2.5cm, bottom=2.5cm}
\newcommand{\ie}{\emph{i.e.,}\xspace}
\newcommand{\eg}{\emph{e.g.,}\xspace}
\newcommand{\wrt}{\emph{w.r.t.}\xspace}
\newcommand{\aka}{\emph{a.k.a.}\xspace}
\newcommand{\kwlog}{\emph{w.l.o.g.}\xspace}
\newcommand{\etal}{\emph{et al.}\xspace}
\newcommand{\sstab}{\rule{0pt}{8pt}\\[-2.4ex]}



\begin{document}


Dear Editors,

I would like to request you to consider the attached manuscript entitled ``One-Pass Trajectory Simplification Using the Synchronous Euclidean Distance'' for publication in VLDB Journal as a research article.
 
It is known that raw trajectory data seriously wastes the storage, network band and computing resource. To attack this issue, one-pass line simplification (LS) algorithms are developed, by compressing data points in a trajectory to a set of continuous line segments.  However, these algorithms (including our previous work ``One-Pass Error Bounded Trajectory Simplification'' published in VLDB 2017) adopt the perpendicular Euclidean distance (PED), and none of them uses the synchronous Euclidean distance (SED), and cannot support spatio-temporal queries.  Further, it is more challenging to design an one pass LS algorithm using SED than using PED. To our knowledge, no one-pass LS algorithms using SED have been developed in the community.

In this work, we propose two one-pass error bounded LS algorithms using SED for compressing trajectories in an efficient and effective way, by utilizing a novel local synchronous distance checking approach (spatio-temporal Cone Intersection) proposed in this study.  Using four real-life trajectory dataset, we also conduct an
extensive experimental study to demonstrate the advantages of our methods compared with the state-of-the-art existing methods.


I believe that the findings of this study are relevant to the scope of your journal, and will be of interests to the corresponding readers.
No conflicts of interest exits in the submission of this manuscript, and this manuscript is approved by all authors for publication. I would like to declare on behalf of my co-authors that the work described was original research that has not been published previously, and not under consideration for publication elsewhere, in whole or in part. All the authors listed have approved the manuscript that is enclosed.



Looking forward to hearing from you.


Your sincerely,

Shuai Ma


Email: mashuai@buaa.edu.cn\\
Homepage: http://mashuai.buaa.edu.cn/
\end{document}
