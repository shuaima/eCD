%\section{One-pass Spatio-temporal data simplification}
\section{One-pass \sed checking method}

In this section, we first discuss the properties of synchronized data points and \sed, then we present a novel \sed checking method by extending the idea of cone intersection.
Finally, we provide a one-pass, error-bounded and \sed enabled trajectory simplification algorithm.



\subsection{Properties of \sed} %and synchronous points     %Spatio-temporal Data
Recall in its definition, the \sed of point $P_i$ to a directed line segment $\mathcal{L} = \vv{P_{s}P_{e}}$, denoted as $sed(P_i, \mathcal{L})$, is $|P_iP_i'|$, the distance from $P_i$ to its synchronized point $P_i' (x_i', y_i', t_i)$ \wrt $\mathcal{L}$.
We first demonstrate the \sed and the synchronized points in the spatio-temporal space, as shown in Figure~\ref{fig:sed3d}.
From the spatio-temporal space, we have the following findings.




%\textcolor[rgb]{1.00,0.00,0.00}{The \sed is demonstrated in 2D space in the previews works~\cite{Meratnia:Spatiotemporal, Chen:Fast, Muckell:Compression, Popa:Spatio}. However,  the trajectory and the \sed are also capable presented in 3D space.}

\begin{lemma}
\label{prop-3d-syn-point}
Given a sub trajectory $[P_s, \ldots, P_e]$ and a point $P_i$, $s<i<e$, the intersection point $P'_i$ of the plane $t=t_i$ and the directed line segment $\mathcal{L} = \vv{P_sP_e}$ is the synchronized data point of $P_i$ \wrt $\mathcal{L}$ in the spatio-temporal space.
\end{lemma}

\begin{proof}
As shown in Figure~\ref{fig:sed3d}, the point $P'_i (x'_i, y'_i, t'_i)$ is the intersection point of the plane $t=t_i$ and the directed line segment $\mathcal{L} = \vv{P_sP_e}$, thus, $t'_i = t_i$ and $\frac{t'_i - t_s}{t_e - t_s}$ = $\frac{t_i - t_s}{t_e - t_s}$  = $\frac{|P_sP'_i|}{|P_sP_e|}$ = $\frac{x'_i - x_s}{x_e - x_s}$ = $\frac{y'_i - y_s}{y_e - y_s}$. Hence, $x'_i = x_s +  \frac{t_i-t_s}{t_e - t_s}(x_e - x_s)$ and $y'_i = y_s +  \frac{t_i - t_s}{t_e - t_s}(y_e - y_s)$, which satisfies the definition of the synchronized data point $P_i$ \wrt $\vv{P_sP_e}$.
\end{proof}




%\begin{definition}
%\label{def:scircle}
\stitle{Synchronous Circles ($\mathcal{O})$}. The synchronous circle of point $P_i$, denoted as $\mathcal{O}(P_i, \epsilon)$, is a circle on the plane $t=t_i$, taking $P_i$ as the center point and having radius of $\epsilon$. %(see Figure~\ref{fig:sed3d}).
%\end{definition}

For example, in Figure~\ref{fig:sed3d}, the point $P_i$ has a synchronous circles $\mathcal{O}(P_i, \epsilon)$. All points in the circle have distances (\ie \sed) less than $\epsilon$ to $P_i$.

With Lemma~\ref{prop-3d-syn-point} and the notion of \emph{synchronous circle}, and from the geometrical perspective, we have Theorem~\ref{prop-3d-sed-sim}.

\begin{theorem}
\label{prop-3d-sed-sim}
Given a sub trajectory $[P_s, \ldots, P_e]$ and a constant $\epsilon$, the sub trajectory can be represented by the line segment $\vv{P_sP_e}$ if and only if the line segment passes through all synchronous circles $\mathcal{O}(P_i, \epsilon)$ for $i \in (s, e)$.
\end{theorem}

\begin{proof}
As shown in Figure~\ref{fig:sed3d}, let $P'_i$ be the intersection point of the line segment $\vv{P_sP_e}$ and the plane $t=t_i$.
By Lemma~\ref{prop-3d-syn-point}, the point $P'_i$ is the synchronized data point of $P_i$ \wrt line segment $\vv{P_sP_e}$.

(1) If the line segment $\vv{P_sP_e}$ passes through all synchronous circles $\mathcal{O}(P_i, \epsilon)$ for $i \in (s, e)$, then by the definition of synchronous circle, it must pass through the circle which is on the plane $t=t_i$ and around point $P_i$, and has a radius of $\epsilon$, thus, $|P'_iP_i| <\epsilon$. Furthermore, the property is hold for each $i \in (s, e)$, hence, the sub trajectory $[P_s, \ldots, P_e]$ can be represented by the line segment $\vv{P_sP_e}$.

(2) If the sub trajectory $[P_s, \ldots, P_e]$ can be represented by the line segment $\vv{P_sP_e}$, then $|P'_iP_i| <\epsilon$ for each $i \in (s, e)$, \ie $P'_i$ is in the synchronous circles $\mathcal{O}(P_i, \epsilon)$. Thus, the line segment $\vv{P_sP_e}$ passes through all synchronous circles $\mathcal{O}(P_i, \epsilon)$ for $i \in (s, e)$.

We have the conclusion.
\end{proof}


\begin{figure}[tb!]
\centering
\includegraphics[scale=0.6]{figures/Fig-SEDin3D.png}
\vspace{-1ex}
\caption{\small Trajectory simplification in the Spatio-temporal space. Note the synchronous circle $\mathcal{O}(P_i, \epsilon)$ is on the plane $t=t_i$.}
\vspace{-3ex}
\label{fig:sed3d}
\end{figure}


%A sub trajectory $\{P_s, \ldots, P_e\}$ is represented by a line segment $\overline{P_sP_e}$
%which passes through the circle on the plane $t=t_i$ and around point $P_i$, for each $s<i<e$.}


\subsection{Spatio-temporal cone intersection}

In this section, we expand the \emph{cone intersection} method from a 2D space, \ie the $x$-$y$ space, to a 3D space, \ie the spatio-temporal space.
In the spatio-temporal space, the notion \emph{cone} of the cone intersection \cite{Williams:Longest, Sklansky:Cone} (or \emph{sector} of sleeve \cite{Zhao:Sleeve}) algorithm is extended to a new notion, \ie the \emph{spatio-temporal cone}.

\stitle{Spatio-temporal cones ($\mathcal{C}$)}. Given a sub trajectory $\{P_s,...,P_e\}$ and a constant $\epsilon$, the spatio-temporal cone of point $P_i$ \wrt point $P_s$, denoted as $\mathcal{C}(P_s, P_i, \epsilon)$, is an oblique circular cone which takes the start point $P_s$ as its vertex and the synchronous circle $\mathcal{O}(P_i, \epsilon)$ as its bottom circle.

For example, in Figure~\ref{fig:cis}, the points $P_s$ and $P_i$ forms an spatio-temporal cone $\mathcal{C}(P_s, P_i, \epsilon)$, and the points $P_s$ and $P_{i+1}$ forms another spatio-temporal cone $\mathcal{C}(P_s, P_{i+1}, \epsilon)$.

Based on the notion of spatio-temporal cone, we further use a novel 3D cone intersection method to check whether there exists a line start from $P_s$ approximating these points.

\begin{theorem}
\label{prop-3d-ci}
Given a sub trajectory $[P_s,...,P_e]$ and a constant $\epsilon$, there exists a point $Q$, $Q.t = P_e.t$, such that $sed(P_i, \vv{P_sQ})<\epsilon$ for all $P_i$, $i \in [s,e]$, if and only if $\bigcap_{i=s+1}^{e}{\mathcal{C}(P_s, P_i, \epsilon)} \ne \{P_s\}$.
\end{theorem}

\begin{proof}
Let $P'_i$ be the intersection point of the line segment $\vv{P_sQ}$ and the plane $t=t_i$.
By Lemma~\ref{prop-3d-syn-point}, the point $P'_i$ is the synchronized data point of $P_i$ \wrt the line segment $\vv{P_sQ}$.


(1) If $\bigcap_{i=s+1}^{e}{\mathcal{C}(P_s, P_i, \epsilon)} \ne \{P_s\}$, then there must exits a point $Q = (x,y,t_e)$ within the circle $\mathcal{O}(P_e, \epsilon)$ of $P_e$ such that $\vv{P_sQ}$ passes through the intersection of all the cones. Hence it pass through all circles $\mathcal{O}(P_i, \epsilon)$ for $i \in (s, e]$, and the intersection point, $P'_i$, is sure within the circle $\mathcal{O}(P_i, \epsilon)$. Thus, $sed(P_i, \vv{P_sQ}) = |P'_iP_i| < \zeta$, for $i \in (s, e]$.

(2) If there exists a point $Q$, $Q.t = P_e.t$, such that $sed(P_i, \vv{P_sQ})<\epsilon$ for all $P_i$, $i \in [s,e]$, then $|P'_iP_i| < \zeta$ for all $i \in (s, e]$. Hence, $\bigcap_{i=s+1}^{e}{\mathcal{C}(P_s, P_i, \epsilon)} \ne \{P_s\}$.

We have the conclusion.
\end{proof}


\begin{figure}[tb!]
\centering
\includegraphics[scale=0.6]{figures/Fig-cis.png}
%\vspace{-1ex}
\caption{\small Examples of spatio-temporal cones. A spatio-temporal cone is an oblique circular cone whose bottom circle and projection circle are parallel to the x-y plane. }
\vspace{-2ex}
\label{fig:cis}
\end{figure}
%Note that the projection circle $\mathcal{O}(P^i_{i+1}, r^i_{i+1})$ is on the plane $t=t_i$.

The checking of intersection of these cones could be enforced by a more simple way, \ie the checking of the intersection of circles on a plane.

\stitle{Projection circles}. Given time $t_c > t_s$, the projection of the synchronous circle $\mathcal{O}(P_i, \epsilon)$ of point $P_i$ on the plane $t=t_c$ is a circle, denoted as $\mathcal{O}(P^c_i, r^c_i)$, where $P^c_i$ =  $(x^c_i, y^c_i, t_c)$ satisfying:


\vspace{-1ex}
\begin{equation*}
\label{equ-proj-circle-xy}
\hspace{10ex} \left\{
    \begin{aligned}
        & x^c_i = x_s +  \frac{t_c - t_s}{t_i - t_s}(x_i - x_s) \\
        & y^c_i = y_s +  \frac{t_c - t_s}{t_i - t_s}(y_i - y_s) \\
    \end{aligned}
    \right.       \hspace{12ex}(1)
\end{equation*}
\vspace{-1ex}

and

\vspace{-2ex}
\begin{equation*}
\label{equ-proj-circle-r}
    \begin{aligned}
        \hspace{17ex}  r^c_i =\frac{t_c-t_s}{t_i-t_s}{\epsilon}   \hspace{20ex}(2) \\
    \end{aligned}
\end{equation*}
\vspace{-1ex}



For example, in Figure~\ref{fig:cis}, the red dash circle on plane $t=t_i$, \ie $\mathcal{O}(P^i_{i+1}, r^i_{i+1})$, is the projection circle of the synchronous circle $\mathcal{O}(P_{i+1}, \epsilon)$ on plane $t=t_i$.

\vspace{1ex}

\begin{cor}
\label{prop-circle-intersection}
Given time $t_c > t_s$, there exists a point $Q$, $Q.t = P_e.t$, such that $sed(P_i, \vv{P_sQ})<\epsilon$ for all points $P_i$, $i \in [s,e]$, if and only if $\bigcap_{i=s+1}^{e}{\mathcal{O}(P^c_i, r^c_i)} \ne \phi$.
\end{cor}


\begin{proof}
These oblique circular cones have the same vertex $P_s$ and their bottom circles are parallel, thus,
 ``$\bigcap_{i=1}^{e}{\mathcal{O}(P^c_i, r^c_i)} \ne \phi$, $t_c > t_s$" is naturally equivalent to ``$\bigcap_{i=s+1}^{e}{\mathcal{C}(P_s, P_i, \epsilon)} \ne \{P_s\}$".
For example, in Figure~\ref{fig:sed3d}, $\mathcal{O}(P^i_{i+1}, r^i_{i+1})$ is the projection circle of synchronous circle $\mathcal{O}(P_{i+1}, \epsilon)$ on the plane $t=t_i$, then ``${\mathcal{O}(P^i_{i+1}, r^i_{i+1})} \bigcap{\mathcal{O}(P_i, \epsilon)} \ne \phi$" is equal to ``$\mathcal{C}(P_s, P_i, \epsilon) \bigcap {\mathcal{C}(P_s, P_{i+1}, \epsilon)} \ne \{P_s\}$".
Hence, by Theorem~\ref{prop-3d-ci}, we have the conclusion.
\end{proof}




%%%%%%%%%%%%%%%%%%%%%%%%%%%%%%%%%%%%%%%%%%%%%%%%%%%%%%%
\subsection{Approximate circles intersection}

%\textcolor[rgb]{1.00,0.00,0.00}{Todo....}
The checking of the $N$ circles intersection has a time complexity of \textcolor[rgb]{1.00,0.00,0.00}{${O(N\log N)}$~\cite{Shamos:Circle}}, which is not a linear time.
To develop a linear time complexity spatio-temporal simplification algorithm, we provide an approximate circles intersection method.
That is, we further replace a circle $\mathcal{O}(P, r)$ to a $M$-edges inscribed regular polygon,
then we check the intersection of the regular polygons instead of the checking of circles intersection.
Note that the intersection of regular polygons is only a sufficient condition of the intersection of the corresponding circles , and
algorithms employ this policy may loss some compression ratio. Obviously, a larger $M$ leads to a better compression ratio and a poorer
efficiency.

\stitle{Proximate polygons ($\mathcal{G}$)}.
Given a circle $\mathcal{O}(P, r)$, its approximate polygons is a $M$ edges inscribed regular polygon $\mathcal{G}(V, E)$,
where $V=\{v_1, \ldots, v_{M}\}$ is the set of vertexes and
$E= \{\overline{v_jv_{j+1}}| j\in [1,M-1]\} \bigcup \{\overline{v_Mv_1}\}$ is the set of edges.
Furthermore, let the center point $P$ of the circle be the origin of a polar coordinate system, then each vertex satisfies:

\vspace{-2ex}
\begin{equation*}
\label{equ-regular-polygon}
%\hspace{-1.5ex}
    \begin{aligned}
        \hspace{10ex}  v_j = (r, \frac{2(j-1)}{M}\pi), ~j \in [1, M]    \hspace{10ex} (3)\\
    \end{aligned}
\end{equation*}
\vspace{-1ex}

%\stitle{intersection polygons ($\mathcal{G}^*$)}.

For example, Figure~\ref{fig:polygons}-(1) is a regular octagon whose vertexes satisfying the Equation (3).

%The intersection of two M-edges convex polygons has a time complexity of $O(M)$ \cite{ORourke:Intersection}.
The regular polygons built by equation (3) have elegant properties, which is helpful to derive an algorithm of linear time and constant space
for the intersection of $N$ regular polygons.


\stitle{Edge groups}.
Let $e_{i,j} = \overline{v_{i,j}v_{i,j+1}}$, $j\in [1,M)$, or $e_{i,j} = \overline{v_{i,M}v_{i,1}}$, $j = M$, be an edge of an approximate polygon
$\mathcal{G}_i$, then all $e_{i,j}$, $i\in [1, k]$, form an \emph{edge group}, namely the $j^{th}$ edge group.
For example, in Figure~\ref{fig:polygons}-(2), all the $1^{th}$ edges form the $1^{th}$ edge group.




\begin{figure}[tb!]
\centering
\includegraphics[scale=0.88]{figures/Fig-polygons.png}
\vspace{-1ex}
\caption{\small Regular octagons and their intersections.}
\vspace{-3ex}
\label{fig:polygons}
\end{figure}



\begin{theorem}
\label{prop-rp-intersection}
If $\mathcal{G}_i$, $i \in [1, k]$, are M-edges regular polygons on a plane which are built by equation (3), then the intersection polygon
$\mathcal{G}^*_k$ of all $\mathcal{G}_i$ includes at most one edge from an edge group, \eg the $j^{th}$ edge group.
\end{theorem}


\begin{proof}
We prove this by contradiction.

Suppose $\mathcal{G}^*_k$ have two different edges, $e_{h,j}$ and $e_{l,j}$, $h\ne l$, belonging to the $j^{th}$ edge group. Obviously, edge
$e_{h,j}$ belongs to regular polygon $\mathcal{G}_h$, edge $e_{l,j}$ belongs to regular polygon $\mathcal{G}_l$ and $e_{h,j}$ is parallel to
$e_{l,j}$.

If $\mathcal{G}_l \bigcap \mathcal{G}_h = \phi$, then $\mathcal{G}^*_k=\phi$, which conflicts with the hypothetic premises.

If $\mathcal{G}_l \bigcap \mathcal{G}_h \ne \phi$, then the intersection polygon of $\mathcal{G}_h$ and $\mathcal{G}_l$ must include at most
one edge of $e_{h,j}$ and $e_{l,j}$, which is also the contradiction with the hypothetic premises.

Hence, the intersection polygon $\mathcal{G}^*_k$ includes at most one edge of an edge group.
%
%There are at most $M$ edge groups, thus, the intersection polygon $\mathcal{G}^*$ has no more than $M$ edges.
\end{proof}



For example, in Figure~\ref{fig:polygons}-(2), the intersection polygon (red lines) of $G_1$, $G_2$ and $G_3$ has 7 edges, marked 1, 2, 3, 5-8, each from a different edge group. Note the $4^{th}$ edge group has no contribution to the intersection polygon.


By Theorem~\ref{prop-rp-intersection}, the intersection polygon $\mathcal{G}^*_k$ of all proximate polygon $\mathcal{G}_i$, $i \in [1, k]$, has no more than $M$ edges. 
We know the convex polygon intersection algorithm of~\cite{ORourke:Intersection} has a time complexity of $O(|\mathcal{G}^*_k| + |\mathcal{G}_{k+1}|)$, which is $O(2M)$ here, thus, the computing of the intersection polygon of polygon $\mathcal{G}^*_k$ and proximate polygon $\mathcal{G}_{k+1}$ can be implemented in a constant time, \ie time $O(M)$, by the distinct polygon intersection algorithm of~\cite{ORourke:Intersection}.
%
Moreover, observe that the algorithm of ~\cite{ORourke:Intersection} is for general convex polygons, while the proximate polygons are special polygons. We further develop an even fast proximate polygon intersection method based on the properties of proximate polygons.

\subsubsection{Proximate polygons intersection}


\begin{figure}[tb!]
\centering
\includegraphics[scale=0.88]{figures/Fig-poly-edges.png}
\vspace{-1ex}
\caption{\small Samples of Candidate intersection edges.}
\vspace{-2ex}
\label{fig:poly-edges}
\end{figure}

The presented proximate polygon intersection algorithm is indeed a customized version of the convex polygon algorithm of ~\cite{ORourke:Intersection}.
%
Given intersection polygon $\mathcal{G}^*_k$ of the preview $k$ polygons and the next proximate polygon $\mathcal{G}_{k+1}$, the algorithm returns $\mathcal{G}^*_{k+1} = \mathcal{G}^*_k  \bigcap \mathcal{G}_{k+1}$.

The basic idea of the algorithm is a two steps processing, \ie \emph{Filtering} and \emph{Intersecting}:
 (1) \emph{Filtering}, which picks out $M$ candidate intersection edges, one edge of an edge group, from $\mathcal{G}^*_k $ and $\mathcal{G}_{k+1}$;
 (2) \emph{Intersecting}, which tries to form the intersection polygon, \ie $\mathcal{G}^*_{k+1}$, from these candidate edges.



\stitle{Candidate intersection edges}.
Let $e_{i,j}$, $1 \le i \le k+1$ and $1 \le j \le M$, be an edge of $\mathcal{G}_{i}$. The line containing the edge partitions the full plane into two half plane, the half plane that contains the central point $P_i$ of $\mathcal{G}_i$ be the \emph{inner half plane} of $e_{i,j}$, denoted as $\mathcal{H}(e_{i,j})$.
%
Then, the candidate intersection edge of the $j^{th}$ edge group is $e_{i,j}$, satisfying,

\vspace{-2ex}
\begin{equation*}
\label{equ-inter-edge}
%\hspace{-1.5ex}
    \begin{aligned}
        \hspace{2ex}  \forall l \in [1, M] (e_{i,j} \in \mathcal{H}(e_{l,j}) ~or~ \mathcal{H}(e_{i,j}) = \mathcal{H}(e_{l,j}) )    \hspace{5ex} (4)\\
    \end{aligned}
\end{equation*}
\vspace{-2ex}


If there are more than one edges of an edge group satisfying Equation (4), then we random choose one edge from them so as to guarantee the unique of the candidate intersection edge of an edge group. Note that not all candidate edges have contributions to intersection polygon.


%\stitle{Picks out $M$ candidate edges from $\mathcal{G}^*_k $ and $\mathcal{G}_{k+1}$.}
By this definition, the selection of candidate edges can be implemented in an incremental way:

\vspace{-2ex}
\begin{equation*}
\label{equ-inter-edge}
%\hspace{-1.5ex}
    \begin{aligned}
        \hspace{1ex}  e^*_{k+1, j} = \left\{
            \begin{aligned}
                & e_{k+1, j},   \hspace{8ex}if~~ e_{k+1,j} \in \mathcal{H}(e^*_{k,j}) \\
                & e^*_{k, j},   \hspace{10ex}if~~ e^*_{k,j} \in \mathcal{H}(e_{k+1,j})      \hspace{6ex} (5)  \\
                & e_{k+1, j} ~or~ e^*_{k, j}, ~if~ \mathcal{H}(e^*_{k,j}) = \mathcal{H}(e_{k+1,j}) \\
            \end{aligned}
        \right.
    \end{aligned}
\end{equation*}
\vspace{-1ex}

where,  $e^*_{1, j} = e_{1, j}$, $j \in [1, M]$ and $k\ge 1$.



\begin{example}
In Figure~\ref{fig:poly-edges}(1)-(6), the eight candidate intersection edges of $\mathcal{G}^*_{k+1}$ are:

(1) $e^*_{k,1}$, $e^*_{k,2}$, $e^*_{k,3}$, $e^*_{k,4}$, $e_{k+1,5}$, $e_{k+1,6}$, $e_{k+1,7}$ and $e_{k+1,8}$;

(2) $e_{k+1,1}$, $e_{k+1,2}$, $e_{k+1,3}$, $e_{k+1,4}$, $e^*_{k,5}$, $e^*_{k,6}$, $e^*_{k,7}$ and $e^*_{k,8}$;

(3) $e_{k+1,1}$ or $e^*_{k,1}$, $e_{k+1,2}$, $e_{k+1,3}$ or $e^*_{k,3}$, $e_{k+1,4}$, $e^*_{k,5}$, $e^*_{k,6}$ or $e_{k+1,6}$, $e_{k+1,7}$ and $e^*_{k,8}$;

(4) $e_{k+1,j}$, $j \in [1, 8]$;

(5) $e^*_{k,j}$, $j \in [1, 8]$;

(6) $e^*_{k,1}$, $e^*_{k,2}$, $e^*_{k,3}$, $e_{k+1,4}$, $e_{k+1,5}$, $e_{k+1,6}$, $e_{k+1,7}$ and $e^*_{k,8}$.


\end{example}





\begin{figure}[tb!]
\centering
\includegraphics[scale=0.88]{figures/Fig-poly-inter.png}
\vspace{-1ex}
\caption{\small A running example of intersection of polygons.}
\vspace{-2ex}
\label{fig:poly-inter}
\end{figure}


\stitle{\textcolor[rgb]{0.00,0.07,1.00}{polygon intersection algorithm}}
Given two proximate polygons $\mathcal{G}^*_k$ and $\mathcal{G}_{k+1}$, and the candidate intersection edges, the algorithm returns the intersection polygon (may be null).

The sketch of the algorithm is shown in Figure~\ref{alg:poly-inter}.

Get the candidate edges

Handle $G_1 \subset G_2$ and $G_2 \subset G_1$ cases.
From the candidate intersection edges, one can straightly conclude whether $\mathcal{G}_{k+1}$ belongs to $\mathcal{G}^*_{k}$ (Figure~\ref{fig:poly-edges}-(4)) or $\mathcal{G}^*_{k}$ belongs to $\mathcal{G}_{k+1}$ (Figure~\ref{fig:poly-edges}-(5)).

%

Handle the other cases.
Let $A$ and $B$ be directed edges on $\mathcal{G}^*_k$ and $\mathcal{G}_{k+1}$ respectively. The algorithm has $A$ and $B$ ``chasing" one another. During the chasing, $A$ or $B$ moves on step by step in the counterclockwise order, following some rules designed depending on geometric conditions, as described in ~\cite{ORourke:Intersection}.


%%%%%%%%%%%%%%%%%%%%%Baseline Algorithm
\begin{figure}[tb!]
\begin{center}
{\small
\begin{minipage}{3.36in}
\myhrule
\vspace{-1ex}
\mat{0ex}{
	{\bf Algorithm} ~$PolyInter$($\mathcal{G}^*$, $\mathcal{G}$, \textcolor[rgb]{1.00,0.00,0.00}{CandidateEdges of $\mathcal{G}^*$})\\
	\bcc \hspace{1ex}\= \textcolor[rgb]{1.00,0.00,0.00}{Get candidate edges} \\
	\bcc \hspace{1ex}\= \textcolor[rgb]{1.00,0.00,0.00}{Handle $G_1 \subset G_2$ and $G_2 \subset G_1$ cases, return?}  \\
	\bcc \hspace{1ex}\= Choose A and B \textcolor[rgb]{0.00,0.07,1.00}{arbitrarily} \\
	\icc \hspace{1ex}\= repeat \\
	\icc \>\hspace{3ex} if A intersects B then \\
	\icc \>\hspace{6ex} \textcolor[rgb]{0.00,0.07,1.00}{Check for termination} \\
	\icc \>\hspace{6ex} Update an inside flag. \\
	\icc \> \hspace{3ex} Moves on either A or B \textcolor[rgb]{0.00,0.07,1.00}{on candidate edges}.\\
	\icc \hspace{1ex} until both A and B cycle their polygons \\
	\icc \hspace{1ex} return the intersection polygon \\
}
\vspace{-2ex}
\myhrule
\end{minipage}
}
\end{center}
\vspace{-2ex}
\caption{\small Intersection of proximate polygons.}
\label{alg:poly-inter}
\vspace{-2ex}
\end{figure}
%%%%%%%%%%%%%%%%%%%%%%%%%%%%%%%%%%%%%






\begin{example}

\end{example}

%%%%%%%%%%%%%%%%%%%%%%%%%%%%%%%%%%%%%%%%%%%%%%%%%%%%%%%



