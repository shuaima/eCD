%%%%%%%%%%%%%%%%%%%%%%%%%%%%%%%%%%%%%%%%%%%%%%%%%%%%%%%%%%%%%%%%%%%%%%%%%%%%%%
\section{Experimental Study} %
\label{sec-exp}
%%%%%%%%%%%%%%%%%%%%%%%%%%%%%%%%%%%%%%%%%%%%%%%%%%%%%%%%%%%%%%%%%%%%%%%%%%%%%%
In this section, we present an extensive experimental study of the spatio-temporal cone intersection algorithms (\cist and \cista) and algorithms of \dpa and \squishe on trajectory data sets.
Using four real-life datasets, we conducted three sets of experiments to evaluate:
(1) the compression ratios of these algorithms,
(2) the average errors of these algorithms,
(3) the execution time of these algorithms, and
(4) the impacts of algorithm \rpia and the edge number $M$ of the regular polygons to the effectiveness and efficiency of \cist and \cista.


\subsection{Experimental Setting}
%We first introduce the settings of our experimental study.

\stitle{Real-life Trajectory Datasets}.
We use four real-life datasets shown in Table~\ref{tab:datasets} to test our solutions.

\ni \emph{(1) Truck trajectory data}, referred to as \truck, is the GPS trajectories collected by \eat{10,368} trucks equipped with GPS sensors in China
during a period from Mar. 2015 to Oct. 2015. The sampling rate varied from 1s to 60s.
%Trajectories mostly have around $50$ to $90$ thousand data points.

\ni \emph{(2) Service car trajectory data}, referred to as \sercar,  is the GPS trajectories collected by a car rental company.
We chose $1,000$ cars from them, during Apr. 2015 to Nov. 2015. The sampling rate was one point per $3$--$5$ seconds, and
each trajectory has around $114.1K$ data points.

\ni \emph{(3) GeoLife trajectory data}, referred to as \geolife, is the GPS trajectories collected in GeoLife project~\cite{Zheng:GeoLife} by 182 users in a period from Apr. 2007 to Oct. 2011. These trajectories have a variety of sampling rates, among which 91\% are logged in each 1-5 seconds or each 5-10 meters per point.
%The longest trajectory has 2,156,994 points.

\ni \emph{(4) Private car trajectory data}, referred to as \pricar, is a small set of high sampling (the sampling rate is fixed with one point per second) GPS trajectories collected by our team members in 2017. There are 10 trajectories and each trajectory has around 11.8K points.

%{This dataset contains 182 trajectories, one trajectory for each user, with a total distance of about 1.2 million kilometers. }




%The details of these datasets are shown in Table~\ref{tab:dataset}.

\stitle{Algorithms and implementation}.
We implement four line simplification algorithms, \ie our \cist and \cista, \sed equipped \dpa ~\cite{Douglas:Peucker} (\dpa-\sed in short) which has the outstanding compression ratios, and \squishe~\cite{Muckell:Compression} which is the most current \sed enabled online trajectory simplification algorithm.
We also implement polygon intersection algorithms, \cpia and our \rpia.
All algorithms were implemented with Java.
All tests were run on an {x64-based  PC with 8 Intel(R) Core(TM) i7-6700 CPU @ 3.40GHz and 8GB of memory, and each test was repeated
over 3 times and the average is reported here}.


\begin{table}
\caption{\small Real-life trajectory datasets}
\vspace{-1ex}
\centering
\footnotesize
\begin{tabular}{|l|c|c|c|r|}
\hline
\kw{Data}& \kw{Number\ of}     &\kw{Sampling}   &\kw{Points Per}    &\kw{Total} \\
\kw{Sets} & \kw{Trajectories}   &\kw{Rates (s)}  &\kw{Trajectory (K)}&\kw{points}\\
\hline\hline
%\truck	&10,368	    &1-60	    &$\sim71.9$     &746M \\
\truck	&1,000	    &1-60	    &$\sim132.7$     &132.7M \\
\hline
%\sercar	&11,000	    &3-5	    &$\sim119.1$   &1.31G\\
\sercar	&1,000	    &3-5	    &$\sim114.1$   &114M\\
\hline
\geolife &182	    &1-5	    &$\sim132.8$   &24.2M\\
\hline
\pricar	& 10	    &1	        &$\sim11.8$      &118K \\
\hline
\end{tabular}
\label{tab:datasets}
\vspace{-3ex}
\end{table}




%%%%%%%%%%%%%%%%%%%%%%%%%%%%%%%%%%%%%%%%%%%%%%%%%%%%%%%%%%%%%%%%%%%%%%%%%%%%%%
\subsection{Experimental Results}
%%%%%%%%%%%%%%%%%%%%%%%%%%%%%%%%%%%%%%%%%%%%%%%%%%%%%%%%%%%%%%%%%%%%%%%%%%%%%%


%%%%%%%%%%%%%%%%%%%%%%%%%%%%%%%%%%%%%%%%%%%%%%%%%%%%%%%%%%%%%%%%%%%%%%%%%%%%%%
\subsubsection{Evaluation of Compression Ratios}
%%%%%%%%%%%%%%%%%%%%%%%%%%%%%%%%%%%%%%%%%%%%%%%%%%%%%%%%%%%%%%%%%%%%%%%%%%%%%%

%%%%%%%%%%%%%%%%%%%%%%%%%%%%%%%%%%%%%%%%%%%%%%%%%%%%%%%%%%%%%%%%%%%%%%%%%%%%%%%%%%%%%%%%%%%%%
%%%%%%%%%%%%%%%%%%%%%%%%%%%%Compression Ratios


\begin{figure*}[tb!]
\centering
\includegraphics[scale = 0.245]{figures/Exp-M-e-20-CR-truck.png}
\includegraphics[scale = 0.245]{figures/Exp-M-e-20-CR-service.png}
\includegraphics[scale = 0.245]{figures/Exp-M-e-20-CR-geolife.png}
\includegraphics[scale = 0.245]{figures/Exp-M-e-20-CR-private.png}
\vspace{-2ex}
\caption{\small Evaluation of compression ratios: fixed error bound with $\epsilon=20m$ and varying $M$.}
\label{fig:m-cr-e20}
\vspace{-1ex}
\end{figure*}

%\eat{%%%%%%%%%%%%%%%%%%%%%%%%%%%%%%%%%%%
\begin{figure*}[tb!]
\centering
\includegraphics[scale = 0.245]{figures/Exp-M-e-200-CR-truck.png}
\includegraphics[scale = 0.245]{figures/Exp-M-e-200-CR-service.png}
\includegraphics[scale = 0.245]{figures/Exp-M-e-200-CR-geolife.png}
\includegraphics[scale = 0.245]{figures/Exp-M-e-200-CR-private.png}
\vspace{-2ex}
\caption{\small Evaluation of compression ratios: fixed error bound with $\epsilon=200m$ and varying $M$.}
\label{fig:m-cr-e200}
\vspace{-1ex}
\end{figure*}
%}


%%%%%%%%%%%%%%%%%%%%%%%%%%%%%%%%%%%%%%%
\eat{
\begin{figure*}[tb!]
\centering
\includegraphics[scale = 0.245]{figures/Exp-cr-epsilon-truck.png}
\includegraphics[scale = 0.245]{figures/Exp-cr-epsilon-service.png}
\includegraphics[scale = 0.245]{figures/Exp-cr-epsilon-geolife.png}
\includegraphics[scale = 0.245]{figures/Exp-cr-epsilon-private.png}
\vspace{-3ex}
\caption{\small Evaluation of compression ratios: fixed with $M=10$ and varying error bound $\epsilon$.}
\label{fig:cr-m10}
\vspace{-1.0ex}
\end{figure*}
}%%%%%%%%%%%%%%%%%%%%%%%%%%%%%%%%%%%%%%



\begin{figure*}[tb!]
\centering
\includegraphics[scale = 0.245]{figures/Exp-cr-epsilon-truck.png}
\includegraphics[scale = 0.245]{figures/Exp-cr-epsilon-service.png}
\includegraphics[scale = 0.245]{figures/Exp-cr-epsilon-geolife.png}
\includegraphics[scale = 0.245]{figures/Exp-cr-epsilon-private.png}
\vspace{-3ex}
\caption{\small Evaluation of compression ratios: fixed with $M=20$ and varying error bound $\epsilon$.}
\label{fig:cr-m20}
\vspace{-2.0ex}
\end{figure*}


In this set of tests, we first evaluate the impacts of polygon intersection algorithm \rpia and parameter $M$ on the compression ratios of algorithms \cist and \cista, then compare the compression ratios of our algorithms \cist and \cista with \dpa and \squishe.

Given a set of trajectories $\{\dddot{\mathcal{T}_1}, \ldots, \dddot{\mathcal{T}_M}\}$ and their piecewise line representations $\{\overline{\mathcal{T}_1}, \ldots, \overline{\mathcal{T}_M}\}$, the compression ratio of an algorithm is $(\sum_{j=1}^{M} |\overline{\mathcal{T}}_j |)/(\sum_{j=1}^{M} |\dddot{\mathcal{T}}_j |)$.
By the definition, \emph{algorithms with lower compression ratios are better}.




\stitle{Exp-1.1: Impacts of algorithm \rpia and parameter $M$ on compression ratios}.
To evaluate the impacts of algorithm \rpia and parameter $M$ on compression ratios of algorithms \cist and \cista, we fixed the error bounds \textcolor[rgb]{1.00,0.00,0.00}{$\epsilon =10m$} and $\epsilon =200m$, and varied $M$ from $4$ to $40$.
The results are reported in Figures~\ref{fig:m-cr-e20} and~\ref{fig:m-cr-e200}.


\ni(1) The polygon intersection algorithms \rpia and \cpia have the same impacts on compression ratios on all datasets. \eg the compression ratio of \rpia equipped simplification algorithm \cist, \ie \cist-\rpia, is the same as the \cpia equipped simplification algorithm, \ie \cist-\cpia, for each $M$.


\ni(2) When varying $M$, the compression ratios of algorithms \cist and \cista decrease with the increase of $M$ on both small error bounds \textcolor[rgb]{1.00,0.00,0.00}{(\eg $\epsilon = 20m$)} and large error bounds (\eg $\epsilon = 200m$) and all datasets.
More specially, the compression ratio decreases (a) fast when $M < 10$, (b) slow when $M \in [10, 20]$, and (c) very slow when $M  > 20$.


\ni(3) \emph{The region $[10, 20]$ is the candidate region for $M$ in terms of compression ratio.}
Where, the compression ratio of $M=10$ is average \textcolor[rgb]{1.00,0.00,0.00}{$102\%$} of $M=20$.
%The former is average \textcolor[rgb]{1.00,0.00,0.00}{$102\%$} of the later.


\stitle{Exp-1.2: Impacts of the error bound $\epsilon$ to compression ratios}.
To evaluate the impacts of $\epsilon$ on compression ratios of these algorithms, we fixed $M=20$, and varied $\epsilon$ from $10m$ to $200m$ on the entire four datasets, respectively.
The results are reported in Figure~\ref{fig:cr-m20} .
%Figure~\ref{fig:cr-m10}.  and


\ni (1) When increasing $\epsilon$, the compression ratios decrease.

%For example, in \sercar, the compression ratios are greater than $38\%$ when $\epsilon$ = $10m$, but are less than $5.4\%$ when $\epsilon$ = $200m$.

\ni (2) \pricar has the lowest compression ratios, compared with \truck, \sercar and \geolife, due to its highest sampling rate,
\truck has the highest compression ratios due to its lowest sampling rate, and \sercar and \geolife have the compression ratios in the middle accordingly.

\ni \textcolor[rgb]{1.00,0.00,0.00}{(3)} Algorithm \cist has \textcolor[rgb]{1.00,0.00,0.00}{comparable} compression ratios with \dpa, and is better than \squishe on all datasets.
For all $\epsilon$, the compression ratios of \cist are on average ($114.7\%$, $110.6\%$, $109.0\%$, $110.7\%$) and ($92.3\%$, $80.0\%$, $72.4\%$, {$73.5\%$}) of \dpa and \squishe on (\truck, \sercar, \geolife, \pricar), respectively.
For example, when $\epsilon = 40m$, the compression ratios are ($26.9\%$, $14.7\%$, $5.4\%$, $3.4\%$), ($30.3\%$, $16.3\%$, $5.9\%$, $3.6\%$) and ($30.3\%$, $16.3\%$, $5.9\%$, $3.6\%$) of \dpa, \cist and \squishe on (\truck, \sercar, \geolife, \pricar), respectively.

\ni \textcolor[rgb]{1.00,0.00,0.00}{(4)} Algorithm \cista has \textcolor[rgb]{1.00,0.00,0.00}{the best} compression ratios on all datasets.
For all $\epsilon$, the compression ratios of \cista are on average ($114.7\%$, $110.6\%$, $109.0\%$, $110.7\%$) and ($92.3\%$, $80.0\%$, $72.4\%$, {$73.5\%$}) of \dpa and \squishe on (\truck, \sercar, \geolife, \pricar), respectively.
For example, when $\epsilon = 40m$, the compression ratios of \cista are ($26.9\%$, $14.7\%$, $5.4\%$, $3.4\%$) on (\truck, \sercar, \geolife, \pricar), respectively.


%The results also show that \cist has better compression ratios than \squishe on datasets with higher sampling rates.





%%%%%%%%%%%%%%%%%%%%%%%%%%%%%%%%%%%%%%%%%%%%%%%%%%%%%%%%%%%%%%%%%%%%%%%%%%%%%%
\subsubsection{Evaluation of Average Errors}
%%%%%%%%%%%%%%%%%%%%%%%%%%%%%%%%%%%%%%%%%%%%%%%%%%%%%%%%%%%%%%%%%%%%%%%%%%%%%%
In this set of tests, we first evaluate the impacts of polygon intersection algorithm \rpia and parameter $M$ on the average errors of algorithms \cist and \cista, then compare the average errors of our algorithms \cist and \cista with \dpa and \squishe.

Given a set of trajectories $\{\dddot{\mathcal{T}_1}, \ldots, \dddot{\mathcal{T}_M}\}$ and their piecewise line representations $\{\overline{\mathcal{T}_1}, \ldots, \overline{\mathcal{T}_M}\}$, and point $P_{j,i}$ denoting
a point in trajectory $\dddot{\mathcal{T}}_j$ contained in a line segment $\mathcal{L}_{l,i}\in\overline{\mathcal{T}_l}$ ($l\in[1,M]$),
then the average error is $\sum_{j=1}^{M}\sum_{i=0}^{M} d(P_{j,i},
\mathcal{L}_{l,i})/\sum_{j=1}^{M}{|\dddot{\mathcal{T}}_j |}$.


%%%%%%%%%%%%%%%%%%%%%%%%%%%%%%%%%%%%%%%%%%%%%%%%%%%%%%%%%%%%%%%%%%%%%%%%%%%%
%Average error
%%%%%%%%%%%%%%%%%%%%%%%%%%%%%%%%%%%%%%%%%%%%%%%%%%%%%%%%%%%%%%%%%%%%%%%%%%%%


\begin{figure*}[tb!]
\centering
\includegraphics[scale = 0.245]{figures/Exp-M-e-20-error-truck.png}
\includegraphics[scale = 0.245]{figures/Exp-M-e-20-error-service.png}
\includegraphics[scale = 0.245]{figures/Exp-M-e-20-error-geolife.png}
\includegraphics[scale = 0.245]{figures/Exp-M-e-20-error-private.png}
\vspace{-2ex}
\caption{\small Evaluation of average errors: fixed error bound with $\epsilon = 20m$ and varying $M$.}
\label{fig:m-error-e20}
\vspace{-1ex}
\end{figure*}

%\eat{%%%%%%%%%%%%%%%%%%%%%%%%%%%%%%%%%%%
\begin{figure*}[tb!]
\centering
\includegraphics[scale = 0.245]{figures/Exp-M-e-200-error-truck.png}
\includegraphics[scale = 0.245]{figures/Exp-M-e-200-error-service.png}
\includegraphics[scale = 0.245]{figures/Exp-M-e-200-error-geolife.png}
\includegraphics[scale = 0.245]{figures/Exp-M-e-200-error-private.png}
\vspace{-2ex}
\caption{\small Evaluation of average errors: fixed error bound with $\epsilon = 200m$ and varying $M$.}
\label{fig:m-error-e200}
\vspace{-1ex}
\end{figure*}
%}




%%%%%%%%%%%%%%%%%%%%%%%%%%%%%%%%%%%%%%%%%%%%%%%%%%%%%%%%%%%%%%
\eat{
\begin{figure*}[tb]
\centering
\includegraphics[scale = 0.245]{figures/Exp-error-epsilon-truck.png}
\includegraphics[scale = 0.245]{figures/Exp-error-epsilon-service.png}
\includegraphics[scale = 0.245]{figures/Exp-error-epsilon-geolife.png}
\includegraphics[scale = 0.245]{figures/Exp-error-epsilon-private.png}
\vspace{-3ex}
\caption{\small Evaluation of average errors: fixed with $M=10$ and varying error bound $\epsilon$.}
\label{fig:ae-m10}
\vspace{-1ex}
\end{figure*}
}
%%%%%%%%%%%%%%%%%%%%%%%%%%%%%%%%%%%%%%%%%%%%%%%%%%%%%%%%%%%%%%



\begin{figure*}[tb]
\centering
\includegraphics[scale = 0.245]{figures/Exp-error-epsilon-truck.png}
\includegraphics[scale = 0.245]{figures/Exp-error-epsilon-service.png}
\includegraphics[scale = 0.245]{figures/Exp-error-epsilon-geolife.png}
\includegraphics[scale = 0.245]{figures/Exp-error-epsilon-private.png}
\vspace{-3ex}
\caption{\small Evaluation of average errors: fixed with $M=20$ and varying error bound $\epsilon$.}
\label{fig:ae-m20}
\vspace{-2ex}
\end{figure*}






%%%%%%%%%%%%%%%%%%%%%%%%%%%%%%%%%%%%%%%%%%%%%%%%%%%%%%%%%%%%%%%%%%%%%%%%%
\stitle{Exp-2.1: Impacts of algorithm \rpia and parameter $M$ on average errors}.
To evaluate the impacts of algorithm \rpia and parameter $M$ on average errors of algorithms \cist and \cista, we fixed the error bounds \textcolor[rgb]{1.00,0.00,0.00}{$\epsilon =10m$} and $\epsilon =200m$, and varied $M$ from $4$ to $40$.
The results are reported in Figures~\ref{fig:m-error-e20} and ~\ref{fig:m-error-e200}.


\ni(1) The polygon intersection algorithms \rpia and \cpia have the same impacts on average errors on all datasets. \eg the average error of \rpia equipped simplification algorithm \cist, \ie \cist-\rpia, is the same as the \cpia equipped algorithm, \ie \cist-\cpia, for each $M$.


\ni(2) When varying $M$, the average errors of algorithms \cist and \cista increase with the increase of $M$ on both small error bound \textcolor[rgb]{1.00,0.00,0.00}{(\eg $\epsilon = 20m$)} and large error bound (\eg $\epsilon = 200m$) and all datasets.
Similar as compression ratios, the average errors increases (a) fast when $M < 10$, (b) slow when $M \in [10, 20]$, and (c) very slow when $M  > 20$.


\ni(3) \emph{The region $[10, 20]$ is also the candidate region for $M$ in terms of average errors.}
Where, the average error of $M=10$ is average \textcolor[rgb]{1.00,0.00,0.00}{$98\%$} of $M=20$.




%%%%%%%%%%%%%%%%%%%%%%%%%%%%%%%%%%%%%%%%%%%%%%%%%%%%%%%%%%%%%%%%%%%%%%%%%
\stitle{Exp-2.2: Impacts of the error bound $\epsilon$ to average errors}.
To evaluate the average errors of these algorithms, we fixed $M=20$, and varied $\epsilon$ from $10m$ to $200m$ on the entire \truck, \sercar, \geolife and \pricar, respectively.
The results are reported in Figure~\ref{fig:ae-m20}. %~\ref{fig:ae-m10} and

\ni(1) Average errors obviously increase with the increase of $\epsilon$.

\ni(2) Algorithms \cist and \cista have lager average errors compared with \dpa and \squishe.

%\ni(2) All datasets have the similar average error in every $\epsilon$.
%\ni(3) Algorithm \squishe has lower average errors than algorithms \dpa and \cist on all datasets and all $\epsilon$ values.






\eat{ %%%%%%%%%%%%%%%%%%%%%%%%%%%%%%%%%%%%%%%%%%%%%%%%%%%%%%%%%%%%%%

%%%%%%%%%%%%%%%%%%%%%%%%%%%%%%%%%%%%%%%%%%%%%%%%%%%%%%%%%%%%%%
\stitle{Exp-2.3: Impacts of the error bound $\epsilon$ to end points}.
%%%%%%%%%%%%%%%%%%%%%%%%%%%%%%%%%%%%%%%%%%%%%%%%%%%%%%%%%%%%%%

\begin{figure*}[tb]
\centering
\includegraphics[scale = 0.245]{figures/Exp-error-epsilon-truck.png}
\includegraphics[scale = 0.245]{figures/Exp-error-epsilon-service.png}
\includegraphics[scale = 0.245]{figures/Exp-error-epsilon-geolife.png}
\includegraphics[scale = 0.245]{figures/Exp-error-epsilon-private.png}
\vspace{-3ex}
\caption{\small Effectiveness evaluation: Original points in \cista.(\cista  $\times$ $M = 10, {20}$ )}
\label{fig:orig-points}
\vspace{-1ex}
\end{figure*}

To evaluate the number of original points are saved in algorithm \cista, we varied $\epsilon$ from $10m$ to $200m$ on the entire \truck, \sercar, \geolife and \pricar, respectively.
The results are reported in Figure~\ref{fig:orig-points}.
}
%%%%%%%%%%%%%%%%%%%%%%%%%%%%%%%%%%%%%%%%%%%%%%%%%%%%%%%%%%%%%%





%%%%%%%%%%%%%%%%%%%%%%%%%%%%%%%%%%%%%%%%%%%%%%%%%%%%%%%%%%%%%%%%%%%%%%%%%%%%%%
\subsubsection{Evaluation of Efficiency}
%%%%%%%%%%%%%%%%%%%%%%%%%%%%%%%%%%%%%%%%%%%%%%%%%%%%%%%%%%%%%%%%%%%%%%%%%%%%%%


In the set of tests, we compare the efficiency of polygon intersection algorithms \rpia and \cpia, the efficiency (execution time) of our approaches \cist and \cista with algorithms \dpa and \squishe, and test the impacts of parameter $M$, \ie the edge number of a regular polygon, to the efficiency of algorithms \cist and \cista.
%
%The execution time is the running time of the compressing process.
%For a small size trajectory, we repeat compress it tens of times and accumulation the total running time so as to get the average compression time.



%%%%%%%%%%%%%%%%%%%%%%%%%%%%%%%%%%%%%%%%%%%%%%%%%%%%%%%%%%%%%%
\stitle{Exp-3.1: Impacts of algorithm \rpia on efficiency.}
To compare the execution time of polygon intersection algorithms \rpia and \cpia, we fixed $\epsilon =60m$ and varying $M$ from $4$ to $40$.
%
The results are reported in Figure~\ref{fig:m-poly-time}.


\ni(1) When varying $M$, the execution time of algorithms \rpia and \cpia increase linearly.

\ni(2) The execution time of algorithm \rpia is average \textcolor[rgb]{1.00,0.00,0.00}{$20\%$ }less than algorithm \cpia on all the datasets and parameter $M$.




\begin{figure*}[tb!]
\centering
\includegraphics[scale = 0.245]{figures/Exp-M-poly-time-truck.png}
\includegraphics[scale = 0.245]{figures/Exp-M-poly-time-service.png}
\includegraphics[scale = 0.245]{figures/Exp-M-poly-time-geolife.png}
\includegraphics[scale = 0.245]{figures/Exp-M-poly-time-private.png}
\vspace{-2ex}
\caption{\small Evaluation of running time of polygon intersection algorithms: fixed error bound with $\epsilon=60m$, and varying $M$.}
\label{fig:m-poly-time}
\vspace{-1ex}
\end{figure*}

\begin{figure*}[tb!]
\centering
\includegraphics[scale = 0.245]{figures/Exp-M-e-60-time-truck.png}
\includegraphics[scale = 0.245]{figures/Exp-M-e-60-time-service.png}
\includegraphics[scale = 0.245]{figures/Exp-M-e-60-time-geolife.png}
\includegraphics[scale = 0.245]{figures/Exp-M-e-60-time-private.png}
\vspace{-2ex}
\caption{\small Evaluation of running time of \lsa algorithms: fixed error bound with $\epsilon=60m$, and varying $M$.}
\label{fig:m-time-e20}
\vspace{-1ex}
\end{figure*}

\eat{%%%%%%%%%%%%%%%%%%%%%%%%%%%%%%%%%%%
\begin{figure*}[tb!]
\centering
\includegraphics[scale = 0.245]{figures/Exp-M-e-200-time-truck.png}
\includegraphics[scale = 0.245]{figures/Exp-M-e-200-time-service.png}
\includegraphics[scale = 0.245]{figures/Exp-M-e-200-time-geolife.png}
\includegraphics[scale = 0.245]{figures/Exp-M-e-200-time-private.png}
\vspace{-2ex}
\caption{\small Evaluation of running time of \lsa algorithms: fixed error bound with $\epsilon=200m$, and varying $M$.}
\label{fig:m-time-e200}
\vspace{-1ex}
\end{figure*}
}



\begin{figure*}[tb!]
\centering
\includegraphics[scale = 0.245]{figures/Exp-time-epsilon-truck.png}
\includegraphics[scale = 0.245]{figures/Exp-time-epsilon-service.png}
\includegraphics[scale = 0.245]{figures/Exp-time-epsilon-geolife.png}
\includegraphics[scale = 0.245]{figures/Exp-time-epsilon-private.png}
\vspace{-3ex}
\caption{\small Evaluation of running time of \lsa algorithms: fixed with $M=20$ and varying error bounds $\epsilon$.}
\label{fig:time-epsilon}
\vspace{-1ex}
\end{figure*}



\begin{figure*}[tb!]
\centering
\includegraphics[scale = 0.245]{figures/Exp-time-size-truck.png}
\includegraphics[scale = 0.245]{figures/Exp-time-size-service.png}
\includegraphics[scale = 0.245]{figures/Exp-time-size-geolife.png}
\includegraphics[scale = 0.245]{figures/Exp-time-size-private.png}
\vspace{-3ex}
\caption{\small Evaluation of running time of \lsa algorithms: fixed with $M=20$ and $\epsilon=60m$, and varying the size of trajectories.}
\label{fig:time-size}
\vspace{-2ex}
\end{figure*}





%%%%%%%%%%%%%%%%%%%%%%%%%%%%%%%%%%%%%%%%%%%%%%%%%%%%%%%%%%%%%%
\stitle{Exp-3.2: Impacts of algorithm \rpia and parameter $M$ on efficiency.}
To evaluate the impacts of algorithm \rpia and parameter $M$ on execution time of line simplification algorithms \cist and \cista, we fixed $\epsilon =60m$ and varying $M$ from $4$ to $40$.
%
The results are reported in Figure~\ref{fig:m-time-e20}. % and ~\ref{fig:m-time-e200}.

\ni(1) When varying $M$, the execution time of algorithms \cist-\rpia, \cist-\cpia, \cista-\rpia and \cista-\cpia increases approximately linear with the increase of $M$.

\ni(2) Parameter $M$ has the similar impacts on all the datasets.

\ni(3) The execution time of algorithms \cist-\rpia and \cista-\rpia is average \textcolor[rgb]{1.00,0.00,0.00}{$10\%$} less than \cpia equipped algorithms.

\ni(4) The running time of $M=10$ is average \textcolor[rgb]{1.00,0.00,0.00}{$60\%$} of $M=20$.


%%%%%%%%%%%%%%%%%%%%%%%%%%%%%%%%%%%%%%%%%%%%%%%%%%%%%%%%%%%%%%
\stitle{{Exp-3.3}:  Impacts of the error bound $\epsilon$}.
To evaluate the impacts of $\epsilon$, we fixed $M=20$ and varying $\epsilon$  from $10m$ to $200m$ on the entire \truck, \sercar, \geolife and \pricar, respectively.
The results are reported in Figure~\ref{fig:time-epsilon}.

\ni(1) All algorithms are not very sensitive to $\epsilon$, but their running time all decreases a little bit with the increase of $\epsilon$, as the increment of $\epsilon$ decreases the number of directed line segments in the output.
Further, algorithm \dpa is more sensitive to $\epsilon$ than the other three algorithms.

\ni(2) Algorithms \cist and \cista are obviously faster than \dpa and \squishe in all cases.
\cist and \cista are on average ($13.5$, $9.6$, $11.8$, {$2.8$}) times faster than \dpa, and {($1.7$, $1.8$, $2.2$, {$1.7$}) times faster} than \squishe on (\truck, \sercar, {\geolife}, \pricar), respectively.


%%%%%%%%%%%%%%%%%%%%%%%%%%%%%%%%%%%%%%%%%%%%%%%%%%%%%%%%%%%%%%
\stitle{{Exp-3.4}: Impacts of the sizes of trajectories}.
To evaluate the impacts of the number of data points in a trajectory (\ie the size of a trajectory) on the execution time,
we chose {$10$} trajectories from \truck, \sercar,\geolife and \pricar, respectively,
fixed $M=20$ and $\epsilon = 60m$, and varying the size \trajec{|T|} of trajectories from $1K$ points to $10K$ points.
%
The results are reported in Figure~\ref{fig:time-size}.

\ni(1) Algorithms \cist and \cista scale well with the increase of the size of trajectories on all datasets,
and have a linear running time, while algorithm \dpa does not.
This is consistent with their time complexity analyses.

\ni(2) Algorithms \cist and \cista are the fastest \sed enabled \lsa algorithms, and are {($6.3$--$10.1$, $4.0$--$6.7$, $2.4$--$7.0$, $3.5$--$6.3$)} times faster than \dpa,
and {($1.0$--$1.4$, $1.3$--$1.6$, $1.2$--$1.9$, $1.9$--$2.3$)} times faster than \squishe on the selected $1K$ to $10K$ points data sets (\truck, \sercar, \geolife, \pricar), respectively.
The advantages of \cist and \cista increase with the increase of the size of trajectories.



%%%%%%%%%%%%%%%%%%%%%%%%%%%%%%%%%%%%%%%%%%%%%%%%%%%%%%%%%%%%%%%%%%%%%%%%%%%%%%
\stitle{Summary}.
%%%%%%%%%%%%%%%%%%%%%%%%%%%%%%%%%%%%%%%%%%%%%%%%%%%%%%%%%%%%%%%%%%%%%%%%%%%%%%
From these tests we find the following.

\vspace{0.5ex}
\ni\emph{(1) Polygon intersection Algorithms}. \rpia runs fast than \cpia on all datasets, and it has the same effectiveness as \cpia.

\vspace{0.5ex}
\ni\emph{(2) Parameter $M$}. The compression ratio decreases with the increase of $M$, and the running time increases approximately linear with the increase of $M$. In practical, region $[10, 20]$ is the candidate region of parameter $M$.

\vspace{0.5ex}
\ni\emph{(3) Compression ratios}. \cist is comparable {with \dpa}. Its compression ratios are on average ($92.3\%$, $80.0\%$, $72.4\%$, {$73.5\%$}) of \squishe and ($114.7\%$, $110.6\%$, $109.0\%$, $110.7\%$) of \dpa on (\truck, \sercar, \geolife, \pricar), respectively, and \cist has a better performance on trajectories with higher sampling rates.

\vspace{0.5ex}
\ni\emph{(4) Average errors}. {Algorithm \cist has higher average errors than the other algorithms.}

\vspace{0.5ex}
\ni\emph{(5) Running time}. \cist is the fastest algorithms, which are on average ($13.5$, $9.6$, $11.8$, {$2.8$}) times faster than \dpa, and {($1.7$, $1.8$, $2.2$, {$1.7$}) times faster than \squishe on (\truck, \sercar, \geolife, \pricar), respectively. The efficiency advantage of algorithm \cist enlarges with the increase of the size of trajectories.


%%********************************* The End **********************************


