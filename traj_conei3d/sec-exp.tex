%%%%%%%%%%%%%%%%%%%%%%%%%%%%%%%%%%%%%%%%%%%%%%%%%%%%%%%%%%%%%%%%%%%%%%%%%%%%%%
\section{Experimental Study} %
\label{sec-exp}
%%%%%%%%%%%%%%%%%%%%%%%%%%%%%%%%%%%%%%%%%%%%%%%%%%%%%%%%%%%%%%%%%%%%%%%%%%%%%%
In this section, we present an extensive experimental study of the spatio-temporal cone intersection algorithms (\cist and \cista) and algorithms of \dpa and \squishe on trajectory data sets.
Using four real-life datasets, we conducted four sets of experiments to evaluate:
(1) the impacts of algorithm \rpia and the edge number $M$ of the regular polygons to the effectiveness and efficiency of \cist and \cista,
(2) the compression ratios of these algorithms,
(3) the average errors of these algorithms, and
(4) the execution time of these algorithms.



\subsection{Experimental Setting}
%We first introduce the settings of our experimental study.

\stitle{Real-life Trajectory Datasets}.
We use four real-life datasets shown in Table~\ref{tab:datasets} to test our solutions.

\eat{
%\sstab{\bf(1) Taxi trajectory data}, referred to as \taxi, is the GPS trajectories collected by $12,727$ taxies equipped with GPS sensors in Beijing during a period from Nov. 1, 2010 to Nov. 30, 2010. The sampling rate was one point  per 60s, and \taxi has $39,100$ data points on average per trajectory.
}

\ni \emph{(1) Truck trajectory data}, referred to as \truck, is the GPS trajectories collected by \eat{10,368} trucks equipped with GPS sensors in China
during a period from Mar. 2015 to Oct. 2015. The sampling rate varied from 1s to 60s.
%Trajectories mostly have around $50$ to $90$ thousand data points.

\ni \emph{(2) Service car trajectory data}, referred to as \sercar,  is the GPS trajectories collected by a car rental company.
We chose $1,000$ cars from them, during Apr. 2015 to Nov. 2015. The sampling rate was one point per $3$--$5$ seconds, and
each trajectory has around $114.1K$ data points.

\ni \emph{(3) GeoLife trajectory data}, referred to as \geolife, is the GPS trajectories collected in GeoLife project~\cite{Zheng:GeoLife} by 182 users in a period from Apr. 2007 to Oct. 2011. These trajectories have a variety of sampling rates, among which 91\% are logged in each 1-5 seconds or each 5-10 meters per point.
%The longest trajectory has 2,156,994 points.

\ni \emph{(4) Private car trajectory data}, referred to as \pricar, is a small set of high sampling (the sampling rate is fixed with one point per second) GPS trajectories collected by our team members in 2017. There are 10 trajectories and each trajectory has around 11.8K points.

%{This dataset contains 182 trajectories, one trajectory for each user, with a total distance of about 1.2 million kilometers. }


\begin{table}
\caption{\small Real-life trajectory datasets}
\vspace{-1ex}
\centering
\footnotesize
\begin{tabular}{|l|c|c|c|r|}
\hline
\kw{Data}& \kw{Number\ of}     &\kw{Sampling}   &\kw{Points Per}    &\kw{Total} \\
\kw{Sets} & \kw{Trajectories}   &\kw{Rates (s)}  &\kw{Trajectory (K)}&\kw{points}\\
\hline\hline
%\truck	&10,368	    &1-60	    &$\sim71.9$     &746M \\
\truck	&1,000	    &1-60	    &$\sim132.7$     &132.7M \\
\hline
%\sercar	&11,000	    &3-5	    &$\sim119.1$   &1.31G\\
\sercar	&1,000	    &3-5	    &$\sim114.1$   &114M\\
\hline
\geolife &182	    &1-5	    &$\sim132.8$   &24.2M\\
\hline
\pricar	& 10	    &1	        &$\sim11.8$      &118K \\
\hline
\end{tabular}
\label{tab:datasets}
\vspace{-3ex}
\end{table}

%The details of these datasets are shown in Table~\ref{tab:dataset}.

\stitle{Algorithms and implementation}.
We implement three algorithms, \ie our \cist, \dpa-\sed, \ie \sed equipped \dpa ~\cite{Douglas:Peucker}, the distinct line simplification algorithm having the outstanding compression ratio, and \squishe~\cite{Muckell:Compression}, the most current \sed enabled online trajectory simplification algorithm.
All algorithms were implemented with Java.
All tests were run on an \textcolor[rgb]{1.00,0.00,0.00}{x64-based  PC with 4 Intel(R) Core(TM) i5-4570 CPU @ 3.20GHz  and 16GB of memory, and each test was repeated
over 3 times and the average is reported here}.


%%%%%%%%%%%%%%%%%%%%%%%%%%%%%%%%%%%%%%%%%%%%%%%%%%%%%%%%%%%%%%%%%%%%%%%%%%%%%%
\subsection{Experimental Metrics}
%%%%%%%%%%%%%%%%%%%%%%%%%%%%%%%%%%%%%%%%%%%%%%%%%%%%%%%%%%%%%%%%%%%%%%%%%%%%%%

From here to the end, we use the following metrics:

\vspace{0.5ex}
\ni{\bf(1) Compression ratio.} Given a set of trajectories $\{\dddot{\mathcal{T}_1}, \ldots, \dddot{\mathcal{T}_M}\}$ and their piecewise line representations $\{\overline{\mathcal{T}_1}, \ldots, \overline{\mathcal{T}_M}\}$, the compression ratio of an algorithm is $(\sum_{j=1}^{M} |\overline{\mathcal{T}}_j |)/(\sum_{j=1}^{M} |\dddot{\mathcal{T}}_j |)$.
Note that by the definition, algorithms with \emph{lower compression ratios are better}.

\vspace{0.5ex}
\ni{\bf(2) Average error.} Given a set of trajectories $\{\dddot{\mathcal{T}_1}, \ldots, \dddot{\mathcal{T}_M}\}$ and their piecewise line representations $\{\overline{\mathcal{T}_1}, \ldots, \overline{\mathcal{T}_M}\}$, and point $P_{j,i}$ denoting
a point in trajectory $\dddot{\mathcal{T}}_j$ contained in a line segment $\mathcal{L}_{l,i}\in\overline{\mathcal{T}_l}$ ($l\in[1,M]$),
then the average error is $\sum_{j=1}^{M}\sum_{i=0}^{M} d(P_{j,i},
\mathcal{L}_{l,i})/\sum_{j=1}^{M}{|\dddot{\mathcal{T}}_j |}$.

\vspace{1ex}
\ni{\bf(3) Execution time.} It is the running time of the compressing process.
For a small size trajectory, we repeat compress it tens of times and accumulation the total running time so as to get the average compression time.






%%%%%%%%%%%%%%%%%%%%%%%%%%%%%%%%%%%%%%%%%%%%%%%%%%%%%%%%%%%%%%%%%%%%%%%%%%%%%%%%%%%%%%%%%%%%%
%%%%%%%%%%%%%%%%%%%%%%%%%%%%Cinfigurations
\begin{figure*}[tb!]
\centering
\includegraphics[scale = 0.24]{figures/Exp-M-time-truck.png}
\includegraphics[scale = 0.25]{figures/Exp-M-time-service.png}
\includegraphics[scale = 0.25]{figures/Exp-M-time-geolife.png}
\includegraphics[scale = 0.25]{figures/Exp-M-time-private.png}
\vspace{-2ex}
\caption{\small Evaluation of configurations: Running time of polygon intersection algorithms \rpia and \cpia. The running time of \rpia is less than \cpia on all the datasets and parameter $M$. ($\epsilon = 20, 80, 200$)}
\label{fig:m-poly-time}
\vspace{-1ex}
\end{figure*}

\begin{figure*}[tb!]
\centering
\includegraphics[scale = 0.25]{figures/Exp-M-time-truck.png}
\includegraphics[scale = 0.25]{figures/Exp-M-time-service.png}
\includegraphics[scale = 0.25]{figures/Exp-M-time-geolife.png}
\includegraphics[scale = 0.25]{figures/Exp-M-time-private.png}
\vspace{-2ex}
\caption{\small Evaluation of configurations: Running time of algorithm \cist. (\rpia, \cpia $\times$ $\epsilon = 20, 80, 200$)}
\label{fig:m-time-cist}
\vspace{-1ex}
\end{figure*}

%\eat{%%%%%%%%%%%%%%%%%%%%%%%%%%%%%%%%%%%
\begin{figure*}[tb!]
\centering
\includegraphics[scale = 0.25]{figures/Exp-M-time-truck.png}
\includegraphics[scale = 0.25]{figures/Exp-M-time-service.png}
\includegraphics[scale = 0.25]{figures/Exp-M-time-geolife.png}
\includegraphics[scale = 0.25]{figures/Exp-M-time-private.png}
\vspace{-2ex}
\caption{\small Evaluation of configurations: Running time of algorithms \cista.(\rpia, \cpia $\times$ $\epsilon = 20, 80, 200$)}
\label{fig:m-time-cista}
\vspace{-1ex}
\end{figure*}
%}



\begin{figure*}[tb!]
\centering
\includegraphics[scale = 0.25]{figures/Exp-M-cr-truck.png}
\includegraphics[scale = 0.25]{figures/Exp-M-cr-service.png}
\includegraphics[scale = 0.25]{figures/Exp-M-cr-geolife.png}
\includegraphics[scale = 0.25]{figures/Exp-M-cr-private.png}
\vspace{-2ex}
\caption{\small Evaluation of configurations: Compression Ratio of algorithms \cist.  (\rpia, \cpia $\times$ $\epsilon = 20, 80, 200$)}
\label{fig:m-cr-cist}
\vspace{-1ex}
\end{figure*}

%\eat{%%%%%%%%%%%%%%%%%%%%%%%%%%%%%%%%%%%
\begin{figure*}[tb!]
\centering
\includegraphics[scale = 0.25]{figures/Exp-M-cr-truck.png}
\includegraphics[scale = 0.25]{figures/Exp-M-cr-service.png}
\includegraphics[scale = 0.25]{figures/Exp-M-cr-geolife.png}
\includegraphics[scale = 0.25]{figures/Exp-M-cr-private.png}
\vspace{-2ex}
\caption{\small Evaluation of configurations : Compression Ratio of algorithms \cista.(the same as above)}
\label{fig:m-cr-cista}
\vspace{-1ex}
\end{figure*}
%}


\begin{figure*}[tb!]
\centering
\includegraphics[scale = 0.25]{figures/Exp-M-error-truck.png}
\includegraphics[scale = 0.25]{figures/Exp-M-error-service.png}
\includegraphics[scale = 0.25]{figures/Exp-M-error-geolife.png}
\includegraphics[scale = 0.25]{figures/Exp-M-error-private.png}
\vspace{-2ex}
\caption{\small Evaluation of configurations : Average error of algorithms \cist.(\rpia, \cpia $\times$ $\epsilon = 20, 80, 200$)}
\label{fig:m-error-cist}
\vspace{-1ex}
\end{figure*}

%\eat{%%%%%%%%%%%%%%%%%%%%%%%%%%%%%%%%%%%
\begin{figure*}[tb!]
\centering
\includegraphics[scale = 0.25]{figures/Exp-M-error-truck.png}
\includegraphics[scale = 0.25]{figures/Exp-M-error-service.png}
\includegraphics[scale = 0.25]{figures/Exp-M-error-geolife.png}
\includegraphics[scale = 0.25]{figures/Exp-M-error-private.png}
\vspace{-2ex}
\caption{\small Evaluation of configurations: Average error of algorithms \cista.( \rpia, \cpia $\times$ $\epsilon = 20, 80, 200$)}
\label{fig:m-error-cista}
\vspace{-1ex}
\end{figure*}
%}


%%%%%%%%%%%%%%%%%%%%%%%%%%%%%%%%%%%%%%%%%%%%%%%%%%%%%%%%%%%%%%%%%%%%%%%%%%%%%%
\subsection{Experimental Results: Configurations}
%%%%%%%%%%%%%%%%%%%%%%%%%%%%%%%%%%%%%%%%%%%%%%%%%%%%%%%%%%%%%%%%%%%%%%%%%%%%%%
%We next present our findings.

In the set of tests, we compare the efficiency of (regular) polygon intersection algorithms \rpia and \cpia, and test the impacts of the parameter $M$, \ie the number of edges of regular polygons, to the efficiency and effectiveness of algorithms \cist and \cista.

%%%%%%%%%%%%%%%%%%%%%%%%%%%%%%%%%%%%%%%%%%%%%%%%%%%%%%%%%%%%%%%%%%%%%%%%%%%%%
%\subsubsection{Evaluation of the polygon edge number $M$}
%%%%%%%%%%%%%%%%%%%%%%%%%%%%%%%%%%%%%%%%%%%%%%%%%%%%%%%%%%%%%%%%%%%%%%%%

\stitle{Exp-1: Impacts of algorithm \rpia on efficiency.}
To compare the execution time of polygon intersection algorithms \rpia and \cpia, we varied $M$ from $4$ to $64$.
The results are reported in Figure~\ref{fig:m-poly-time}.


\ni(1) When varying $M$, the execution time of algorithms \rpia and \cpia increase linear?

\ni(2) The execution time of algorithm \rpia is less than algorithm \cpia on all the datasets and parameter $M$.


\stitle{Exp-2: Impacts of algorithm \rpia and parameter $M$ on efficiency.}
To evaluate the impacts of algorithm \rpia and parameter $M$ on execution time of line simplification algorithms \cist and \cista, we varied $M$ from $4$ to $64$.
The results are reported in Figure~\ref{fig:m-time-cist} and ~\ref{fig:m-time-cista}.

\ni(1) When varying $M$, the execution time of algorithms \cist - \rpia, \cist - \cpia, \cista- \rpia and \cista- \cpia increases approximately linear with the increase of $M$.

\ni(2) The execution time of algorithms \cist - \rpia and \cista- \rpia is less than algorithms \cist - \cpia and \cista- \cpia

\ni(3) Parameter $M$ has the similar impacts on all the datasets.

\stitle{Exp-3: Impacts of algorithm \rpia and parameter $M$ on compression ratios}.
To evaluate the impacts of algorithms \rpia and parameter $M$ on compression ratios of algorithms \cist and \cista, we varied $M$ from $4$ to $64$.
The results are reported in Figures~\ref{fig:m-cr-cist} and~\ref{fig:m-cr-cista}.


\ni(1) The compression ratio and average error of \rpia equipped simplification algorithms are the same as algorithm \cpia equipped algorithms.

\ni(2) When varying $M$, the compression ratio decreases with the increase of $M$.
The compression ratio decreases (a) fast when $M \in [4,8]$, and (b) slowly when $M \in [16, 64]$.


\stitle{Exp-4: Impacts of algorithm \rpia and parameter $M$ on average errors}.
To evaluate the impacts of algorithms \rpia and parameter $M$ on average errors of algorithms \cist and \cista, we varied $M$ from $4$ to $64$.
The results are reported in Figures~\ref{fig:m-error-cist} and ~\ref{fig:m-error-cista}.


\ni(1) When varying $M$, the average error increases with the increase of $M$.
The average error increases (a) fast when $M \in [4,8]$, and (b) slowly when $M \in [16, 64]$.

\ni(2) Parameter $M$ has the similar impacts on all the datasets.

\emph{The region $[8, 20]$ is the candidate region for $M$ where it has nearly the best compress ratio and a less running time.}








%%%%%%%%%%%%%%%%%%%%%%%%%%%%%%%%%%%%%%%%%%%%%%%%%%%%%%%%%%%%%%%%%%%%%%%%%%%%%%
%%%%%%%%%%   Effectiveness
%%%%%%%%%%%%%%%%%%%%%%%%%%%%%%%%%%%%%%%%%%%%%%%%%%%%%%%%%%%%%%%%%%%%%%%%%%%%%%
\begin{figure*}[tb!]
\centering
\includegraphics[scale = 0.25]{figures/Exp-cr-epsilon-truck.png}
\includegraphics[scale = 0.25]{figures/Exp-cr-epsilon-service.png}
\includegraphics[scale = 0.25]{figures/Exp-cr-epsilon-geolife.png}
\includegraphics[scale = 0.25]{figures/Exp-cr-epsilon-private.png}
\vspace{-3ex}
\caption{\small Effectiveness evaluation: varying the error bound $\epsilon$.(\cist, \cista, \dpa, \squishe  $\times$ $M = 10, {20}$ )}
\label{fig:cr}
\vspace{-1.0ex}
\end{figure*}


\begin{figure*}[tb]
\centering
\includegraphics[scale = 0.25]{figures/Exp-error-epsilon-truck.png}
\includegraphics[scale = 0.25]{figures/Exp-error-epsilon-service.png}
\includegraphics[scale = 0.25]{figures/Exp-error-epsilon-geolife.png}
\includegraphics[scale = 0.25]{figures/Exp-error-epsilon-private.png}
\vspace{-3ex}
\caption{\small Evaluation of average errors: varying the error bound $\epsilon$.(\cist, \cista, \dpa, \squishe  $\times$ $M = 10, {20}$ )}
\label{fig:ae}
\vspace{-1ex}
\end{figure*}


\begin{figure*}[tb]
\centering
\includegraphics[scale = 0.25]{figures/Exp-error-epsilon-truck.png}
\includegraphics[scale = 0.25]{figures/Exp-error-epsilon-service.png}
\includegraphics[scale = 0.25]{figures/Exp-error-epsilon-geolife.png}
\includegraphics[scale = 0.25]{figures/Exp-error-epsilon-private.png}
\vspace{-3ex}
\caption{\small Evaluation of original points: varying the error bound $\epsilon$.(\cista  $\times$ $M = 10, {20}$ )}
\label{fig:orig-points}
\vspace{-1ex}
\end{figure*}



%%%%%%%%%%%%%%%%%%%%%%%%%%%%%%%%%%%%%%%%%%%%%%%%%%%%%%%%%%%%%%%%%%%%%%%%%%%%%%
\subsection{Experimental Results: Effectiveness}
%%%%%%%%%%%%%%%%%%%%%%%%%%%%%%%%%%%%%%%%%%%%%%%%%%%%%%%%%%%%%%%%%%%%%%%%%%%%%%
In the second set of tests, we compare the compression ratios and averages of our algorithms \cist and \cista with \dpa and \squishe.



\stitle{Exp-2.1: Impacts of the error bound $\epsilon$ to compression ratios}.
To evaluate the impacts of $\epsilon$ on compression ratios of these algorithms, we varied $\epsilon$ from $10m$ to $200m$ on
 the entire four datasets, respectively.
The results are reported in Figure~\ref{fig:cr}.

\ni (1) When increasing $\epsilon$, the compression ratios decrease.

%For example, in \sercar, the compression ratios are greater than $38\%$ when $\epsilon$ = $10m$, but are less than $5.4\%$ when $\epsilon$ = $200m$.

\ni (2) \pricar has the lowest compression ratios, compared with \truck, \sercar and \geolife, due to its highest sampling rate,
\truck has the highest compression ratios due to its lowest sampling rate, and \sercar and \geolife have the compression ratios in the middle accordingly.

\ni (3) Algorithm \cist has \textcolor[rgb]{1.00,0.00,0.00}{comparable} compression ratios with \dpa.
For all $\epsilon$, the compression ratios of \cist are on average ($114.7\%$, $110.6\%$, $109.0\%$, $110.7\%$) of \dpa on (\truck, \sercar, \geolife, \pricar), respectively.
For example, when $\epsilon = 40m$, the compression ratios are ($26.9\%$, $14.7\%$, $5.4\%$, $3.4\%$) of \dpa and ($30.3\%$, $16.3\%$, $5.9\%$, $3.6\%$) of \cist on (\truck, \sercar, \geolife, \pricar), respectively.

\ni (4) \cist is better than  \squishe on all datasets.
For all $\epsilon$, the compression ratios of \cist are on average ($92.3\%$, $80.0\%$, $72.4\%$, {$73.5\%$}) of \squishe on (\truck, \sercar, \geolife, \pricar), respectively.
%For example, when $\epsilon = 40m$, the compression ratios are ($31.3\%$, $20.0\%$, $8.0\%$, $4.9\%$)of \squishe, and ($30.3\%$, $16.3\%$, $5.9\%$, $3.6\%$) of \cist on (\truck, \sercar, \geolife, \pricar), respectively.
The results also show that \cist has better compression ratios than \squishe on datasets with higher sampling rates.




%%%%%%%%%%%%%%%%%%%%%%%%%%%%%%%%%%%%%%%%%%%%%%%%%%%%%%%%%%%%%%%%%%%%%%%%%%%%%%
%\subsubsection{Evaluation of Average Errors}
%%%%%%%%%%%%%%%%%%%%%%%%%%%%%%%%%%%%%%%%%%%%%%%%%%%%%%%%%%%%%%%%%%%%%%%%%%%%%%
\stitle{Exp-2.2: Impacts of the error bound $\epsilon$ to average errors}.
%%%%%%%%%%%%%%%%%%%%%%%%%%%%%%%%%%%%%%%%%%%%%%%%%%%%%%%%%%%%%%
To evaluate the average errors of these algorithms, we varied $\epsilon$ from $10m$ to $200m$ on the entire \truck, \sercar, \geolife and \pricar, respectively.

The results are reported in Figure~\ref{fig:ae}.

\ni(1) Average errors obviously increase with the increase of $\epsilon$.

\ni(2) All datasets have the similar average \sed error in every $\epsilon$.

\ni(3) Algorithm \squishe has lower average errors than algorithms \dpa and \cist on all datasets and all $\epsilon$ values.

\ni(4) Algorithms \cist have lager average errors compared with \dpa and \squishe.



%%%%%%%%%%%%%%%%%%%%%%%%%%%%%%%%%%%%%%%%%%%%%%%%%%%%%%%%%%%%%%
\stitle{Exp-2.3: Impacts of the error bound $\epsilon$ to end points}.
%%%%%%%%%%%%%%%%%%%%%%%%%%%%%%%%%%%%%%%%%%%%%%%%%%%%%%%%%%%%%%
To evaluate the number of original points are saved in algorithm \cista, we varied $\epsilon$ from $10m$ to $200m$ on the entire \truck, \sercar, \geolife and \pricar, respectively.
The results are reported in Figure~\ref{fig:orig-points}.






%%%%%%%%%%%%%%%%%%%%%%%%%%%%%%%%%%%%%%%%%%%%%%%%%%%%%%%%%%%%%%%%%%%%%%%%%%%%%%
%%%%%%%%%%   Efficiency
%%%%%%%%%%%%%%%%%%%%%%%%%%%%%%%%%%%%%%%%%%%%%%%%%%%%%%%%%%%%%%%%%%%%%%%%%%%%%%
\begin{figure*}[tb!]
\centering
\includegraphics[scale = 0.25]{figures/Exp-time-epsilon-truck.png}
\includegraphics[scale = 0.25]{figures/Exp-time-epsilon-service.png}
\includegraphics[scale = 0.25]{figures/Exp-time-epsilon-geolife.png}
\includegraphics[scale = 0.25]{figures/Exp-time-epsilon-private.png}
\vspace{-3ex}
\caption{\small Efficiency evaluation: varying the error bound $\epsilon$. (\cist, \cista, \dpa, \squishe  $\times$ $M = 10, {20}$ )}
\label{fig:time-epsilon}
\vspace{-1ex}
\end{figure*}


\begin{figure*}[tb!]
\centering
\includegraphics[scale = 0.25]{figures/Exp-time-size-truck.png}
\includegraphics[scale = 0.25]{figures/Exp-time-size-service.png}
\includegraphics[scale = 0.25]{figures/Exp-time-size-geolife.png}
\includegraphics[scale = 0.25]{figures/Exp-time-size-private.png}
\vspace{-3ex}
\caption{\small Efficiency evaluation: varying the size of trajectories. (\cist, \cista, \dpa, \squishe  $\times$ $M = 10, {20}$ )}
\label{fig:time-size}
\vspace{-1ex}
\end{figure*}

%%%%%%%%%%%%%%%%%%%%%%%%%%%%%%%%%%%%%%%%%%%%%%%%%%%%%%%%%%%%%%%%%%%%%%%%%%%%%%
\subsection{Experimental Results: Efficiency}
%%%%%%%%%%%%%%%%%%%%%%%%%%%%%%%%%%%%%%%%%%%%%%%%%%%%%%%%%%%%%%%%%%%%%%%%%%%%%%
%%%%%%%%%%%%%%%%%%%%%%%%%%%%%%%%%%%%%%%%%%%%%%%%%%%%%%%%%%%%%%%%%%%%%%%%%%%%%%
%\subsubsection{Evaluation of Compression Efficiency}
%%%%%%%%%%%%%%%%%%%%%%%%%%%%%%%%%%%%%%%%%%%%%%%%%%%%%%%%%%%%%%%%%%%%%%%%%%%%%%
$\newline$
In this set of tests, we compare the efficiency (execution time) of our approach \cist with algorithms \dpa and \squishe.

%%%%%%%%%%%%%%%%%%%%%%%%%%%%%%%%%%%%%%%%%%%%%%%%%%%%%%%%%%%%%%
\stitle{{Exp-4.1}:  Impacts of the error bound $\epsilon$}.
To evaluate the impacts of $\epsilon$, we varied $\epsilon$ from $10m$ to $200m$ on the entire \truck, \sercar, \geolife and \pricar, respectively.
The results are reported in Figure~\ref{fig:time-epsilon}.

\ni(1) All algorithms are not very sensitive to $\epsilon$, but their running time all decreases a little bit with the increase of $\epsilon$,
as the increment of $\epsilon$ decreases the number of directed line segments in the output.
Further, algorithm \dpa is more sensitive to $\epsilon$ than the other three algorithms.

\ni(2) Algorithm \cist is obviously faster than \dpa and \squishe in all cases.
\cist is on average ($13.5$, $9.6$, $11.8$, {$2.8$}) times faster than \dpa, and {($1.7$, $1.8$, $2.2$, {$1.7$}) times faster} than \squishe on (\truck, \sercar, {\geolife}, \pricar), respectively.


%%%%%%%%%%%%%%%%%%%%%%%%%%%%%%%%%%%%%%%%%%%%%%%%%%%%%%%%%%%%%%
\stitle{{Exp-4.2}: Impacts of the sizes of trajectories}.
To evaluate the impacts of the number of data points in a trajectory (\ie the size of a trajectory) on the execution time,
we chose {$10$} trajectories from \truck, \sercar,\geolife and \pricar, respectively,
and varied the size \trajec{|T|} of trajectories from $1K$ points to $10K$ points, while fixed $\epsilon = 40$ meters.

The results are reported in Figure~\ref{fig:time-size}.

\ni(1) Algorithm \cist scales well with the increase of the size of trajectories on all datasets,
and shows a linear running time, while algorithm \dpa does not.
This is consistent with their time complexity analyses.

\ni(2) Algorithm \cist is the fastest \lsa algorithms, and are {($6.3$--$10.1$, $4.0$--$6.7$, $2.4$--$7.0$, $3.5$--$6.3$)} times faster than \dpa,
and {($1.0$--$1.4$, $1.3$--$1.6$, $1.2$--$1.9$, $1.9$--$2.3$)} times faster than \squishe on the selected 1K to 10K points data sets (\truck, \sercar, \geolife, \pricar), respectively.
The advantages of \cist increase with the increase of the size of trajectories.



%%%%%%%%%%%%%%%%%%%%%%%%%%%%%%%%%%%%%%%%%%%%%%%%%%%%%%%%%%%%%%%%%%%%%%%%%%%%%%
\stitle{Summary}.
%%%%%%%%%%%%%%%%%%%%%%%%%%%%%%%%%%%%%%%%%%%%%%%%%%%%%%%%%%%%%%%%%%%%%%%%%%%%%%
From these tests we find the following.

\vspace{1ex}
\emph{\ni{(1) Parameter $M$}}. The compression ratio decreases with the increase of $M$, and the running time increases approximately linear with the increase of $M$. The region $[8, 16]$ is the candidate region for $M$, and \emph{$M$ = $10$} is a good choice in practical.

\vspace{1ex}
\emph{\ni{(2) Compression ratios}}. \cist is comparable {with \dpa}. Its compression ratios are on average ($92.3\%$, $80.0\%$, $72.4\%$, {$73.5\%$}) of \squishe and ($114.7\%$, $110.6\%$, $109.0\%$, $110.7\%$) of \dpa on (\truck, \sercar, \geolife, \pricar), respectively, and \cist has a better performance on trajectories with higher sampling rates.

\vspace{1ex}
\emph{\ni{(3) Average errors}}. {Algorithm \cist has higher average errors than the other algorithms.}

\vspace{1ex}
\emph{\ni{(4) Running time}}. \cist is the fastest algorithms, which are on average ($13.5$, $9.6$, $11.8$, {$2.8$}) times faster than \dpa, and {($1.7$, $1.8$, $2.2$, {$1.7$}) times faster than \squishe on (\truck, \sercar, \geolife, \pricar), respectively. The efficiency advantage of algorithm \cist enlarges with the increase of the size of trajectories.


%%********************************* The End **********************************


