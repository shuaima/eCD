%%%%%%%%%%%%%%%%%%%%%%%%%%%%%%%%%%%%%%%%%%%%%%%%%%%%%%%%%%%%%%%%%%%%%%%%%%%%%%
\section{Conclusions} % and Future Work}
%%%%%%%%%%%%%%%%%%%%%%%%%%%%%%%%%%%%%%%%%%%%%%%%%%%%%%%%%%%%%%%%%%%%%%%%%%%%%%

We have proposed \cist and \cista, two \sed enabled, one-pass and error bounded trajectory simplification algorithms.
%
We have developed a novel spatio-temporal cone intersection approach and a fast circle intersection checking method, based on which we then have designed \cist and \cista.
%
We have theoretically proved that both \cist and \cista have linear time and constant space complexities.
%
We have also experimentally verified that algorithms \cist and \cista are of three times faster than \squishe, the most current \sed enabled \lsa algorithm with improved runtime performance,
and in terms of compression ratio, algorithm \cist is {comparable} with \dpa, the existing \lsa algorithm with the best compression ratio, and is $21.1\%$ better than \squishe on average; and algorithm \cista is {$19.6\%$} and {$42.4\%$} better than \dps and \squishe, respectively.
