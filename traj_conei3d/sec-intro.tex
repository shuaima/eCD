%%%%%%%%%%%%%%%%%%%%%%%%%%%%%%%%%%%%%%%%%%%%%%%%%%%%%%
%\vspace{-0.5ex}
\section{Introduction}
\label{sec-intro}
%%%%%%%%%%%%%%%%%%%%%%%%%%%%%%%%%%%%%%%%%%%%%%%%%%%%%

Various mobile devices, such as smart-phones, on-board diagnostics, personal navigation devices, and wearable smart devices, use their sensors to collect massive trajectory data of moving objects at a certain sampling rate (e.g., a data point every $5$ seconds), which is transmitted to cloud servers for various applications such as location based services and trajectory mining.
%
Transmitting and storing raw trajectory data consumes too much network bandwidth and storage capacity \cite{Chen:Trajectory, Meratnia:Spatiotemporal,Shi:Survey, Lin:Operb, Liu:BQS, Liu:Amnesic, Muckell:survey, Muckell:Compression,Cao:Spatio, Popa:Spatio,Nibali:Trajic}. %Schmid:Semantic, Richter:Semantic, Long:Direction,
%Further, we find that the online transmitting of raw trajectories also seriously aggravates several other issues such as out-of-order and duplicate data points in our experiences when implementing an online vehicle-to-cloud data transmission system.
%These issues can be resolved or greatly alleviated by trajectory compression techniques via removing redundant data points of trajectories \cite{Douglas:Peucker, Hershberger:Speeding, Meratnia:Spatiotemporal, Liu:BQS, Muckell:survey, Muckell:Compression, Chen:Trajectory, Chen:Fast,Cao:Spatio, Shi:Survey, Nibali:Trajic}, % Keogh:online, Richter:Semantic, Long:Direction, Song:PRESS
%
%Trajectory compression techniques remove redundant data points of trajectories,
%, or replace original data points in a trajectory with other places of interests, such as roads and shops.
%A large number of trajectory compression techniques have been developed,
% among which the piece-wise line {simplification}  technique is widely used \cite{Douglas:Peucker, Hershberger:Speeding, Liu:BQS, Muckell:Compression, Chen:Trajectory, Chen:Fast, Cao:Spatio, Shi:Survey}, due to its distinct advantages: (a) simple and easy to implement, (b) no need of extra knowledge and suitable for freely  moving  objects \cite{Popa:Spatio}, and (c) bounded errors with good compression ratios.
%
It is known that these issues can be resolved or greatly alleviated by trajectory compression techniques via removing redundant data points of trajectories \cite{Douglas:Peucker, Hershberger:Speeding, Meratnia:Spatiotemporal,Lin:Operb, Liu:BQS, Liu:Amnesic,  Muckell:Compression, Chen:Trajectory, Cao:Spatio,  Nibali:Trajic, Long:Direction, Popa:Spatio, Han:Compress, Chen:Fast}, among which the piece-wise line simplification technique is widely used \cite{Douglas:Peucker, Meratnia:Spatiotemporal,  Muckell:Compression, Chen:Trajectory, Cao:Spatio, Liu:BQS, Liu:Amnesic, Lin:Operb, Chen:Fast}, due to its distinct advantages: (a) simple and easy to implement, (b) no need of extra knowledge and suitable for freely  moving  objects, and (c) bounded errors with good compression ratios \cite{Popa:Spatio,Lin:Operb}.


%%%%%%%%%%%%%%%PED%%%%%%%%%%%%%%%%%%
Originally, line simplification (\lsa) algorithms adopt the \emph{perpendicular Euclidean distance} (\ped) as a metric to compute the errors.
\textcolor{blue}{Suppose a sub-trajectory $[P_s, ..., P_e]$ is represented by a line segment $\vv{P_sP_e}$  simplified by an error bounded \lsa algorithm using \ped, then for any point $P \in [P_s, ..., P_e]$, its \emph{perpendicular Euclidean distance} to the line segment $\vv{P_sP_e}$  is the shortest distance from point $P$ to the line segment.}
\textcolor{blue}{Indeed, after the redundant points are removed, information about those points are lost except that they are located in a zone composed by a rectangle and two half circles  as shown in Figure~\ref{fig:distances}(a), and each point in the zone has a \ped not more than the pre-defined error bound $\epsilon$ to $\vv{P_sP_e}$.}
\textcolor{blue}{Hence, a spatio-temporal query, \eg ``\emph{the position $P$ of a moving object at time $t$}", on the compressed trajectories is inappropriate. This query, in best practice, returns a point $P'$ selected from the line segment $\vv{P_sP_e}$. No matter how to choose $P'$, the distance $|PP'|$ is possible larger than error bound $\epsilon$ as point $P$ may be located in any position of the zone. }
%as the zone or the line segment usually has a bigger geographic range or a larger length compared with the error bound
%Thus, the simplified trajectory using \ped is not suitable for the spatio-temporal query.

\eat{
\eg $|\vv{P_4P^*_4}|$ is the \ped of data point $P_4$ to line segment $\vv{P_0P_{10}}$ in Figure~\ref{fig:notations} (left).
Line simplification algorithms using \ped have good compression ratios~ \cite{Douglas:Peucker, Hershberger:Speeding,Lin:Operb, Liu:BQS, Muckell:Compression, Chen:Trajectory, Cao:Spatio, Shi:Survey}.
However, when using \ped, a spatio-temporal query, \eg ``\emph{the position of a moving object at time $t$}", on the compressed trajectories will return an approximate point $P'$ whose distance to the actual position $P$ of the moving object at time $t$ is unbounded.
}


%For instance,
%And a query for the position of the moving object at time $P_7.t$ will return the synchronized data point $P'_7$ whose distance to the actual position $P_7$ is great than the bound $\epsilon$.}


\eat{, of a data point to a proposed generalized line (\eg in Figure~\ref{fig:notations} (left), $|P_4P^*_4|$ is the \ped of $P_4$ to the line $\overline{P_0P_{10}}$) as the condition to discard or retain that data point.
\lsa algorithms using \ped have good compression ratios and are error bounded on \ped, hence they are widely used in scenarios that compression ratio is the most concerned factor. However, when using \ped, the temporal information of trajectory points is lost. Thus, a temporal-spatio query, \eg ``\emph{the position $P$ of a moving object at time $t$}", on trajectories compressed by \lsa algorithms using \ped returns an approximated point $P'$ whose distance to the actual position $P$ at time $t$ is unbounded. For example, a query at time $t_7$ returns an approximated point $P'_7$ whose distance to the point $P_7$ is great than the threshold $\epsilon$.

,  and implemented it in Douglas-Peucker (\dpa)~\cite{Douglas:Peucker} algorithm
}


%%%%%%%%%%%%%%%SED%%%%%%%%%%%%%%%%%%
\eat{
The \emph{synchronous Euclidean distance} (\sed) was then introduced for trajectory compression to support the above spatio-temporal queries, where the approximate point $P'$ was referred to as the approximate temporally \emph{synchronized data point} \cite{Meratnia:Spatiotemporal}.
Intuitively, for a sub-trajectory $[P_s$, $\ldots, P_e]$, a synchronized data point $P'_i$ ($s<i<e$) is a point on line segment $\vv{P_sP_{e}}$ satisfying $|\vv{P_sP_e}|>0$ and $\frac{|\vv{P_sP'_i}|}{|\vv{P_sP_e}|} = \frac{P_i.t - P_s.t}{P_e.t - P_s.t}$, and \sed is the Euclidean distance of a data point $P_i$ to its \emph{synchronized data point $P'_i$} on the line segment $\vv{P_sP_{e}}$.
%
For example, in Figure~\ref{fig:notations}, $P'_4$ is the \emph{synchronized data point} of point $P_4$ \wrt line segment $\vv{P_0P_{10}}$, satisfying $\frac{|\vv{P_0P'_4}|}{|\vv{P_0P_{10}}|} = \frac{P_4.t - P_0.t}{P_{10}.t-P_0.t}$. The \sed of point $P_4$ to line segment $\vv{P_0P_{10}}$ is $|\vv{P_4P'_4}|$.
%
Indeed, the \sed of a point to a line segment is always not less than the \ped of the point to the line segment, thus, \lsa algorithms using \sed may lead to more line segments.
However, the use of \sed ensures that the Euclidean distance between a data point and its synchronized point \wrt the corresponding line segment is limited within a distance bound $\epsilon$. Hence, the above spatio-temporal query over the trajectories compressed by \sed enabled approaches returns the synchronized point $P'$ of a data point $P$ within the bound $\epsilon$.
}


The \emph{synchronous Euclidean distance} (\sed) was then introduced for trajectory compression to support the above spatio-temporal queries.
\textcolor{blue}{The definition of \sed highly depends on a notion named \emph{synchronized data point} \cite{Meratnia:Spatiotemporal}. Intuitively, the \emph{synchronized data point} $P'$ of a point $P$ at time $t$ \wrt the line segment $\vv{P_sP_e}$  is the position of the moving object on $\vv{P_sP_e}$ at time $t$ if the object moved from $P_s$ to $P_e$ with an uniform speed.}
\textcolor{blue}{Then the \sed of point $P$ to line segment $\vv{P_sP_e}$ is the distance between $P$ and its \emph{synchronized data point} $P'$  \wrt the line segment $\vv{P_sP_e}$, as shown in Figure~\ref{fig:notations}(b).}
\textcolor{blue}{After the sub-trajectory is simplified by an error bounded \lsa algorithm using \sed to line segment $\vv{P_sP_e}$, those redundant points are removed, however, we can still infer that their are located inside of the circles around their \emph{synchronized data points} with a radius of $\epsilon$, respectively.}
\textcolor{blue}{Hence, the above spatio-temporal query over the trajectories compressed by \sed enabled approaches returns the synchronized point $P'$ of data point $P$ \wrt the corresponding line segment, and it ensures that the Euclidean distance between the data point and its synchronized point is limited within the distance bound $\epsilon$.}
%
Note the \sed of a point to a line segment is always not less than the \ped of the point to the line segment, thus, \lsa algorithms using \sed may lead to more line segments.


\begin{figure}[tb!]
\centering
%\vspace{-1ex}
\includegraphics[scale=1.2]{figures/Fig-Distances.png}
%\vspace{-1ex}
\caption{\small A sub-trajectory $[P_s, \ldots, P_e]$ is simplified using \ped and \sed, respectively.}
\vspace{-2ex}
\label{fig:distances}
\end{figure}

%
%

%%%%%%%%%%%%%%%%%%%Min-# Problem%%%%%%%%%%%%%%%
The problem of finding the minimal number of line segments to represent the original polygonal lines \wrt an error bound $\epsilon$ is known as the ``min-\#" problem\cite{Imai:Optimal,Chan:Optimal}, and there exists an optimal \lsa algorithm using \sed that runs in $O(n^3)$~\cite{Imai:Optimal} (originally designed for \ped ),  where $n$ is the number of the original points.
Due to this high time complexity, sub-optimal \lsa algorithms using \sed have been developed for trajectory compression, including batch algorithms (\eg Douglas-Peucker based algorithm \dpsed \cite{Meratnia:Spatiotemporal}) and online algorithms (\eg\ \squishe \cite{Muckell:Compression}).
However, these methods still have high time and/or space complexities, which hinders their utilities in resource-constrained devices. %\cite{Lin:Operb}


Observe that one-pass \lsa algorithms using \ped \cite{Williams:Longest, Sklansky:Cone, Dunham:Cone, Zhao:Sleeve, Lin:Operb} have been developed, and they are more efficient for resource-constrained devices.
%
The key idea to achieve one-pass processing is by local distance checking for a single data point in constant time, \eg the \textit{sector intersection} mechanism used in \cite{Williams:Longest, Sklansky:Cone, Dunham:Cone, Zhao:Sleeve} and the \textit{fitting function} approach used in our preview work \cite {Lin:Operb}.
Unfortunately, these techniques are designed specifically for \ped, and can hardly be applied for \sed.


Indeed, it is even more challenging to design one-pass \lsa algorithms using \sed than using \ped.
To our knowledge,  no one-pass \lsa algorithms using \sed have been developed in the community yet.



\eat{%%%%%%%%%%%%%%%%%%
%No matter using \ped or \sed,
\textcolor{blue}{The problem of finding the minimal number of line segments to represent the original polygonal lines \wrt an error bound $\epsilon$ is known as the ``min-\#" problem\cite{Imai:Optimal,Chan:Optimal}.}
%
\textcolor{blue}{An optimal $O(n^3)$  \lsa algorithm was firstly developed in \cite{Imai:Optimal}, where $n$ is the number of the original points. The high time and space complexities of the optimal \lsa algorithm make it impractical.}
%
\textcolor{blue}{Sub-optimal \lsa algorithms were then developed or used for trajectory compression, including the batch algorithms (\eg Douglas-Peucker \cite{Douglas:Peucker, Meratnia:Spatiotemporal, Cao:Spatio} and Theo~Pavlidis \cite{Pavlidis:Segment}), the online algorithms (\eg \bqsa~\cite{Liu:BQS} and \ \squishe \cite{Muckell:Compression}) and most recently, the one-pass algorithms (\eg sector intersection \cite{Williams:Longest, Sklansky:Cone, Dunham:Cone, Zhao:Sleeve} and \operb \cite {Lin:Operb}). }
%
\textcolor{blue}{Among these algorithms, the one-pass algorithms have linear time and constant space complexities, hence, they are more efficient and suitable for resource-constrained devices that frequently collect and store GPS data points.
However, the current one-pass algorithms are all \ped specific, and to our knowledge, no one-pass \lsa algorithm using \sed has been developed in the community.}


\textcolor{blue}{As have been pointed out in \cite {Lin:Operb}, the key idea to achieve a one-pass algorithm is by \emph{local distance checking} in constant time, \eg the \textit{sector intersection} mechanism used in \cite{Williams:Longest, Sklansky:Cone, Dunham:Cone, Zhao:Sleeve} and the \textit{fitting function} approach used in our preview work \cite {Lin:Operb}, however, the design of an effective local synchronous distance checking approach is non-trivial.}
}%%%%%%%%%%%%%%%%%%


\stitle{{Contributions}}. To this end, we propose two fast one-pass error bounded \lsa algorithms using \sed for compressing trajectories with good compression ratios. % error bounded

\sstab {(1)} We first develop a novel local synchronous distance checking approach, \ie spatio-temporal \underline{C}one \underline{I}ntersection using the \underline{S}ynchronous \underline{E}uclidean \underline{D}istance (CISED).
%, by extending the sector intersection method \cite{Williams:Longest, Sklansky:Cone, Zhao:Sleeve}\eat{(Section~\ref{sec-localcheck})}.
We further approximate the intersection of spatio-temporal cones with the intersection of a special class of  regular polygons, and develop a fast regular polygon intersection algorithm, such that each data point in a trajectory is checked in $O(1)$ time during the entire process of trajectory simplification.

%\sstab {(2)} \textcolor{blue}{We then show that an optimal trajectory simplification algorithm using \sed can achieve $O(n^2 \log n)$ time by using our local synchronous distance checking technique, and provide a \textit{near optimal} trajectory simplification algorithm \cisto that achieves $O(n^2)$ time and $O(n)$ space.}

\sstab {(2)} We next develop two one-pass trajectory simplification algorithms \cist and \cista, achieving $O(n)$ time complexity and $O(1)$ space complexity, based on our local synchronous distance checking technique.
Algorithm \cist belongs to strong simplification that only has original points in its outputs, while algorithm \cista belongs to weak simplification that allows interpolated data points in its outputs.


\sstab (3) Using four real-life trajectory datasets (\sercar, \geolife, \mopsi, \pricar),
we finally conduct an extensive experimental study, by comparing our methods \cist and \cista  with the optimal \lsa algorithm using \sed, \dps~\cite{Meratnia:Spatiotemporal} (the existing sub-optimal \lsa algorithm using \sed having the best compression ratios) and \squishe \cite{Muckell:Compression} (the fastest existing \lsa algorithm using \sed).

For running time,
algorithms \cist and \cista are on average $15.0$, $3.2$ and $14345.0$ times faster than \dps, \squishe and the optimal \lsa algorithm on the test datasets, respectively.
%
For compression ratios, algorithm \cist is better than \squishe and close to \dps. The output sizes of \cist are on average $74.4\%$, $110.4\%$ and $137.9\%$ of \squishe, \dps and the optimal \lsa algorithm on the test datasets, respectively.
Moreover, algorithm \cista is on average $54.9\%$ and $81.6\%$ better than \squishe and \dps on the test datasets, respectively.

It is worth pointing out that trajectory data is collected by mobile devices from GPS sensors, and these devices have range errors, which leads to data quality issues of  trajectory data~\cite{PfoserJ99,ZufleTPRRLDE17}. However, the problem is beyond the scope of this study, and we focus on lossy simplification of trajectory data only.


\stitle{{Organization}}.
The remainder of the article is organized as follows.
Section \ref{sec-preliminary} introduces the basic concepts and techniques.
Section \ref{sec-localcheck} presents our local synchronous distance checking method.
Section \ref{sec-alg} presents our one-pass trajectory simplification algorithms.
Section \ref{sec-exp} reports the experimental results, followed by related work in
Section \ref{sec-related} and conclusion in Section \ref{sec-conclusion}.
All proofs are provided in the Appendix.
%The appendix covers additional related work and experimental results.





