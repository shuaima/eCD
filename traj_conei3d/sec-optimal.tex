\section{\textcolor{blue}{Near Optimal Trajectory Simplification}}
\label{sec-optimal}

Based on the Spatio-temporal cone technique, we develop a fast near optimal algorithm using \sed. Our near optimal algorithm achieves a $O(n^2)$ time which is much faster than the optimal algorithm using \sed having a time complexity of $O(n^3)$ \cite{Imai:Optimal}.


\subsection{Optimal algorithm using \sed}

Given a trajectory \trajec{T}${[P_0, \ldots, P_n]}$ and an error bound $\epsilon$, the optimal trajectory simplification problem can be solved in two steps. 
The first step is to construct a reachability graph $G$ of \trajec{T} and the second step is to find the shortest path in $G$ from $P_0$ to $P_{n}$ \cite{Imai:Optimal, Chan:Optimal}.

\stitle{Reachability Graph ($G$)}. The reachability graph of a trajectory \trajec{T}${[P_0, \ldots, P_n]}$ \wrt an error bound $\epsilon$ is $G$
= \{\trajec{T}, $E\}$, where, for vertexes $P_i$ and $P_k \in$ \trajec{T} $(0 \le i<k\le n)$, edge $e_{ik}=(P_i, P_k) \in E$ if and only if the \sed of each point $P_j (i<j<k)$, to line segment $\vv{P_iP_k}$ is no great than $\epsilon$, i.e., $sed(P_j, \vv{P_iP_k}) \le \epsilon$.
%, and (2) $w$ is a weighting function such that $w(e_{ik}) = k-i$.

Then, the shortest path in $G$ from point $P_0$ to point $P_{n}$ is the optimal representation of trajectory \trajec{T}, and the path length will correspond to the number of line segments in the approximation trajectory. 
%Moreover, the shortest path also reveals the subset of points of \trajec{T} used in the approximate trajectory.


The brute-force algorithm of constructing reachability graph $G$ is to check for each pair of points $P_i$ and $P_k$ whether $sed(P_j, \vv{P_iP_k}) \le \epsilon$. 
There are $O(n^2)$ pairs of points in the trajectory and checking the error of data points to a line segment takes $O(n)$ time. 
Thus,this method takes $O(n^3)$ time \cite{Imai:Optimal}. 
Since finding the shortest path takes no more than $O(n^2)$ time, the brute-force algorithm takes $O(n^3)$ time.
%
Though the authors of \cite{Chan:Optimal} proved that the construction of a reachability graph $G$ using \ped could further be implemented in $O(n^2)$ time with the help of the \textit{sector intersection} mechanism, the \textit{sector intersection} mechanism can not work with \sed (see Section \ref{sub-ci-ped}), hence, the construction of a graph  $G$ using \sed remains $O(n^3)$.

Thus, the crucial part of of the algorithm is still the construction of $G$. 
In the following, we shall discuss how a near optimal $G$ can be constructed in $O(n^2)$ time.


\subsection{Speed up graph constructing.} % by using spatio-temporal cones
In this section, we speed up the construction of reachability graph $G$ by using the spatio-temporal cone.

Firstly, we give the necessary condition for the inclusion of an edge to the reachability graph $G$.

\begin{prop}
\label{prop-edge-check}
Edge $(P_{s}, P_{s+k})$ can be included in reachability graph $G$ if and only if the intersection of line segment $\vv{P_{s}P_{s+k}}$ and 
the intersection of preview spatio-temporal cones $\bigsqcap_{i=1}^{k-1}$ \cone{(P_{s}, \mathcal{O}(P_{s+i}, \epsilon))} is not $\{P_s\}$.
% and $\bigsqcap_{j=i+1}^{k-1}$\cone{(P_i, \mathcal{O}(P_{j}, \epsilon))} $\ne \{P_i\}$. 
\end{prop}

\textcolor{red}{todo...}  %{prop-3d-ci}

\begin{proof}
If the directed line segment $\vv{P_sP_r} \in$ $\bigsqcap_{i=1}^{r-1}$\cone{(P_s, \mathcal{O}(P_{s+i}, \epsilon))}, the
intersection point $P'_{s+i}$ of the directed line segment $\vv{P_sP_r}$ and the
plane $P.t - P_{s+i}.t = 0$  is  in the area of the  synchronous circle
\circle{(P_{s+i}, \epsilon)} for each $i \in [1,r - 1]$ . By Proposition~\ref{prop-3d-syn-point}, the
synchronous distance of $P_{s+i}$ to $\vv{P_sP_r}$ is error bouned for each $i \in [1,r - 1]$ . Thus,Edge
$(v_s,v_r)$ can be included in the $\epsilon - graph$  
\end{proof}


%\textcolor{blue}{How?}
Then we propose the speedup graph constructing by using spatio-temporal cones.

From Proposition~\ref{prop-edge-check}, we can determine whether $(v_s,v_r) \in
G$ by testing whether $\vv{P_sP_r} \in $ $\bigsqcap_{i=1}^{r - 1}$\cone{(P_s,
  \mathcal{O}(P_{s+i}, \epsilon))}. For any r, the algorithm would consider all
pairs of vertices  $(v_s,v_r)$ with  r ranging from s + 1 to n.  

The checking of whether a line segment lies inside a spatio-temporal cone can be
reduced to checking whether its projection point lies inside the projection
circle on a plane which can be done in constant time. Furthermore, as we have
proposed before, the compution of intersection of spatio-temporal cones can be
done in constant time. Thus, checking for each pair of vertices can be done in
constant time.

Since we use inscribed regular polygon to approximate the projection circle,
this method is near optimal.

\subsection{Near Optimal algorithm using \sed.}

%\textcolor{blue}{code, algorithm desc and example.}

We now present algorithm \cisto. It takes as input a trajectory \trajec{T}${[P_0, \ldots, P_n]}$, an error bound $\epsilon$ and the number $m$ of edges for inscribed regular polygons, and returns a simplified  trajectory $\overline{\mathcal{T}}$ of $\dddot{\mathcal{T}}$.

The algorithm first  constructs the $\epsilon graph$ for the trajectory. It
checks for each point $P_s$, whether the points ``after'' it , i.e., $P_r,r > s$
can form an edge in the graph with the starting point. 

Given a trajectory \trajec{T}${[P_0, \ldots, P_n]}$, an error bound $\epsilon$ and the number $m$ of edges for inscribed
regular polygons, it returns a simplified trajectory,


%%%%%%%%%%%%%%%%%%%%% Algorithm
\begin{figure}[tb!]
	\begin{center}
		{\small
			\begin{minipage}{3.36in}
				\myhrule
				\vspace{-1ex}
				\mat{0ex}{
					{\bf Algorithm} ~$\cisto(\dddot{\mathcal{T}}[P_0,\ldots,P_n], ~\epsilon, ~M, ~t_c)$\\
					%	\sstab
					\bcc \hspace{1ex}\= $P_s := P_0$;  ~$P_e := P_0$; ~$E = \phi$;
          ~$\overline{\mathcal{T}} := \phi$;\\
					%         \hspace{2ex}     $intersection = \emptyset$;   \\
					\icc \hspace{1ex}\= \For $s = 0$  \To $n - 2$ \Do \\
          \icc \>\hspace{3ex}\= $\mathcal{G}^* := {getRegularPolygon}$($P_s$, $P_{s+1}$, $\epsilon$, $M$, $t_c$) \\
					\icc \>\hspace{3ex}\= \For $r = s + 2$  \To $n$ \Do \\
					\icc \>\hspace{6ex} \If ${isInside}$($\mathcal{G}^*$, $P_r$) \Then \\
					\icc \>\hspace{9ex} ${addEdge}$($P_s$,$P_r$,$E$) \\
					\icc \>\hspace{6ex} $\mathcal{G} := {getRegularPolygon}$($P_s$, $P_r$, $\epsilon$, $M$, $t_c$) \\
					\icc \>\hspace{6ex} $\mathcal{G}^* := {\rpia}(\mathcal{G}^*, ~\mathcal{G})$ \\

					\icc \>\hspace{6ex} \If $\mathcal{G}^* = \phi$ \Then \\
					\icc \>\hspace{9ex} \Break \\
					% \icc \>\hspace{3ex} \Else \\
					% \icc \>\hspace{3ex} \If $\mathcal{G}^* \ne \phi$ \Then \\
					% \icc \> \hspace{6ex} $P_e := P_i$ \\
					% \icc \> \hspace{6ex} $i := i+1$ \\
					% \icc \>\hspace{3ex} \Else\\
					% \icc \> \hspace{6ex} $\overline{\mathcal{T}} := \overline{\mathcal{T}}\cup \{\mathcal{L}(P_s,Q)\}$ \\
					% \icc \> \hspace{6ex} $P_s := Q$;  ~~$\mathcal{G}^* := \phi$ \\

					\icc \hspace{1ex} $\overline{\mathcal{T}} = {Dijkstra}(E)$\\
					\icc \hspace{1ex}\Return $\overline{\mathcal{T}}$
					%
					%	{\bf procedure} $getPolygon$($P_s$, $P_i$, $\epsilon/2$) \\
					%
					%	{\bf procedure} $getIntersection(ipolygon, \mathcal{G})$ \\
				}
				\vspace{-2ex}
				\myhrule
			\end{minipage}
		}
	\end{center}
	\vspace{-2ex}
	\caption{\small Aggressive spatio-temporal cone intersection algorithm (\cista).}
	\label{alg:ciseda}
	\vspace{-2ex}
\end{figure}
%%%%%%%%%%%%%%%%%%%%%%%%%%%%%%%%%%%%%


%%%%%%%%%%%%%%%%%%%%%%%%%%%%%%%%%%%%%%%%%%%%%%%example of Algorithm CISED-O
\begin{figure*}[tb!]
	\centering
	\includegraphics[scale=0.8]{figures/Fig-CISED-O.png}
	%\vspace{-2ex}
	\caption{\small A running example of the \cisto algorithm. The points and the oblique circular cones are projected on an x-y space. The trajectory $\dddot{\mathcal{T}}[P_0, \ldots, P_{10}]$ is compressed into four line segments.}
	\vspace{-1ex}
	\label{fig:exm-const}
\end{figure*}
%%%%%%%%%%%%%%%%%%%%%%%%%%%%%%%%%%%%%%%%%%%%%%%%%%%%%%%%%%%%%%%%%%%%%%%%%%%

