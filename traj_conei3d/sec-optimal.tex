\section{\textcolor{blue}{Near Optimal Trajectory Simplification}}
\label{sec-optimal}

Based on the Spatio-temporal cones and their fast intersection techniques, we develop a fast near optimal trajectory simplification algorithm using \sed. Our near optimal algorithm achieves a $O(n^2)$ time which is much faster than the optimal algorithm using \sed having a time complexity of $O(n^3)$.


\subsection{The Optimal Algorithm Using \sed}

Given a trajectory \trajec{T}${[P_0, \ldots, P_n]}$ and an error bound $\epsilon$, the optimal trajectory simplification problem, as formulated by Imai and Iri in \cite{Imai:Optimal}, can be solved in two steps: construct a reachability graph $G$ of \trajec{T} and then search a shortest path from $P_0$ to $P_{n}$ in graph $G$.

\stitle{Reachability Graph ($G$)}. The reachability graph of a trajectory \trajec{T}${[P_0, \ldots, P_n]}$ \wrt an error bound $\epsilon$ is $G$
= (\trajec{T}, $E)$, where, for vertexes $P_s$ and $P_{s+k} \in$ \trajec{T} $(k>0)$, edge $(P_s, P_{s+k}) \in E$ if and only if the distance of each point $P_{s+i} (0<i<k)$ to line segment $\vv{P_sP_{s+k}}$ is no great than $\epsilon$.
%, i.e., $sed(P_j, \vv{P_iP_k}) \le \epsilon$.
%, and (2) $w$ is a weighting function such that $w(e_{ik}) = k-i$.

Then, the shortest path in $G$ from point $P_0$ to point $P_{n}$ is the optimal representation of trajectory \trajec{T}, and the path length will correspond to the number of line segments in the approximation trajectory. 
%Moreover, the shortest path also reveals the subset of points of \trajec{T} used in the approximate trajectory.


The brute-force algorithm of constructing reachability graph $G$ using \sed is to check for each pair of points $P_s$ and $P_{s+k}$ whether $sed(P_{s+i}, \vv{P_sP_{s+k}}) \le \epsilon$. 
There are $O(n^2)$ pairs of points in the trajectory and checking the error of all data points $P_{s+i}$ to a line segment $\vv{P_sP_{s+k}}$ takes $O(n)$ time. 
Thus, this method takes $O(n^3)$ time \cite{Imai:Optimal}. 
Since finding the shortest path takes no more than $O(n^2)$ time, the brute-force algorithm takes $O(n^3)$ time.
%
Though the authors of \cite{Chan:Optimal} proved that the construction of a reachability graph $G$ using \ped could further be implemented in $O(n^2)$ time with the help of the \textit{sector intersection} mechanism, the \textit{sector intersection} mechanism can not work with \sed (see Section \ref{sub-ci-ped}), hence, the construction of a graph  $G$ using \sed remains $O(n^3)$ time.

Thus, the crucial part of of the algorithm is still the construction of $G$. 
In the following, we shall discuss how a near optimal $G$ can be constructed in $O(n^2)$ time.


\subsection{Speed Up Graph Constructing} % by using spatio-temporal cones
In this section, we speed up the construction of reachability graph $G$ by using the spatio-temporal cone.
Firstly, we give the necessary and sufficient condition for the inclusion of an edge to the reachability graph $G$.

\begin{prop}
\label{prop-edge-check}
Edge $(P_{s}, P_{s+k})$ can be included in reachability graph $G$ if and only if the intersection of line segment $\vv{P_{s}P_{s+k}}$ and 
the common area of preview spatio-temporal cones, \ie $\bigsqcap_{i=1}^{k-1}$ \cone{(P_{s}, \mathcal{O}(P_{s+i}, \epsilon))}, is not $\{P_s\}$.
% and $\bigsqcap_{j=i+1}^{k-1}$\cone{(P_i, \mathcal{O}(P_{j}, \epsilon))} $\ne \{P_i\}$. 
\end{prop}

\begin{proof}
This proposition is a directed result of Proposition \ref{prop-3d-ci}, where point $Q$ is the original point $P_{s+k}$. 
%Hence, edge $(P_{s}, P_{s+k})$ can be included in reachability graph $G$.
\end{proof}


From Proposition~\ref{prop-edge-check}, we can determine whether edge $(P_s, P_{s+k}) \in E$ by testing whether the intersection of $\vv{P_sP_{s+k}}$ and $\bigsqcap_{i=1}^{k - 1}$\cone{(P_s,  \mathcal{O}(P_{s+i}, \epsilon))} is $\{P_i\}$. 
It must be noted that, even if edge $(P_s, P_{s+k})$ is not in $E$, edge $(P_s, P_{s+k+1})$ is still possible in $E$ as long as $\bigsqcap_{i=1}^{k}$ \cone{(P_{s}, \mathcal{O}(P_{s+i}, \epsilon))}$\ne \{P_s\}$.

As mentioned above, the checking of intersection of cones could be approximately replaced by the checking of intersection of regular polygons. 
If we use the regular polygon intersection method, then we get another reachability graph $G'=$ (\trajec{T}, $E')$.

\begin{prop}
	\label{prop-near-opt-graph}
	Reachability graph $G'=$ (\trajec{T}, $E')$ is a sub-graph of $G=$(\trajec{T}, $E)$ and $\lim_{m \to \infty}{G'=G}$, where $m$ is the number of edges of a regular polygon.
\end{prop}

\begin{proof}
	Firstly, we show that any reachability graph $G'=$ (\trajec{T}, $E')$ is a sub-graph of its corresponding $G=$(\trajec{T}, $E)$.
	For a edge $(P_s, P_{s+k}) \in E'$, we have $sed(P_{s+i}, \vv{P_sP_{s+k}}) \le \epsilon$ for all $0<i<k$ when using the intersection of regular polygons. 
	Because a regular polygon is a subset of of a projection circle, we also have $sed(P_{s+i}, \vv{P_sP_{s+k}}) \le \epsilon$ for all $0<i<k$ when using the intersection of projection circles.
	Thus, edge $(P_s, P_{s+k}) \in E$ and reachability graph $G'=$ (\trajec{T}, $E'$) is a sub-graph of $G=$(\trajec{T}, $E$).
	
	Secondly, when $m \to \infty$, then the regular polygon is infinitely close to its projection circle, thus, for any edge $(P_s, P_{s+k}) \in E$, it is in $E'$ with an infinite possibility. In the other word, $\lim_{m \to \infty}{G'=G}$.
\end{proof}

Proposition~\ref{prop-near-opt-graph} tells us that a reachability graph $G'$ can be infinitely close to the optimal graph $G$.

\begin{prop}
	\label{prop-near-opt-graph-construction}
	A near optimal reachability graph $G'$ can be constructed in $O(n^2)$ time.
\end{prop}

\begin{proof}
For any $P_s (0\le s \le n)$, the construction of a reachability graph $G'$ need consider all pairs of vertices $(P_s,P_{s+k}) (1\le k \le n)$ and compute the intersection of $\vv{P_sP_{s+k}}$ and $\bigsqcap_{i=1}^{k - 1}$\cone{(P_s,  \mathcal{O}(P_{s+i}, \epsilon))}. 
The computing of the common area of cones, \ie $\bigsqcap_{i=1}^{k - 1}$\cone{(P_s,  \mathcal{O}(P_{s+i}, \epsilon))}, could be implement in an incremental way, and each approximate intersection of two cones could be checked by the intersection of two polygons, which can be achieved in a constant time, hence, it needs $O(n)$ time to check all edges $(P_s,P_{s+k})$ start form point $P_s$. 

There are $n$ points in the trajectory, thus, the construction of a near optimal graph $G'$ needs $O(n^2)$ time.
\end{proof}

%Proposition~\ref{{prop-near-opt-graph}} and Proposition~\ref{prop-near-opt-graph-construction} tell us we can construct a near optimal reacha

%From Proposition~\ref{prop-edge-check}, we can determine whether $(v_s,v_r) \in G$ by testing whether $\vv{P_sP_r} \in $ $\bigsqcap_{i=1}^{r - 1}$\cone{(P_s,  \mathcal{O}(P_{s+i}, \epsilon))}. For any r, the algorithm would consider all pairs of vertices  $(v_s,v_r)$ with  r ranging from s + 1 to n.  

%The checking of whether a line segment lies inside a spatio-temporal cone can be reduced to checking whether its projection point lies inside the projection circle on a plane which can be done in constant time. 
%Furthermore, as we have proposed before, the compution of intersection of spatio-temporal cones can be done in constant time. Thus, checking for each pair of vertices can be done in constant time.

%Since we use inscribed regular polygon to approximate the projection circle, this method is near optimal.

\subsection{Near Optimal Algorithm Using \sed.}

%\textcolor{blue}{code, algorithm desc and example.}

We now present algorithm \cisto. The full algorithm is shown in Figure~\ref{alg:cisedo}.

\stitle{Procedure \kw{getRegularPolygon}}.
We first present procedure \kw{getRegularPolygon} that, given a cone projection circle, generates its inscribed $m$-edge regular polygon,  following the definition in Section~\ref{subsec-RPI}.

The parameters $P_s$, $P_i$, $r$ and $t_c$ together form the projection circle \pcircle{(P^c_i, r^c_i)} of the spatio-temporal cone \cone{(P_s, \mathcal{O}(P_{i}, r))} of point $P_{i}$ \wrt point $P_s$ on the plane $P.t - t_c$ = $0$. Firstly, $P^c_i.x$ and $P^c_i.y$ are computed (lines 1--3), and $r^c_i = c\cdot r$.
Then it builds and returns an $m$-edge inscribed regular polygon $\mathcal{R}$ of \pcircle{(P^c_i, r^c_i)} (lines 4--8), by transforming a polar coordinate system
into a Cartesian one. Note that here $\theta$, $r\cdot\sin\theta$ and $r\cdot\cos\theta$ only need to be computed once during the entire processing of a trajectory.

\stitle{Algorithm \cisto}. 
We then present algorithm \cisto. 
It takes as input a trajectory \trajec{T}${[P_0, \ldots, P_n]}$, an error bound $\epsilon$ and the number $m$ of edges for inscribed regular polygons, and returns a simplified  trajectory $\overline{\mathcal{T}}$ of $\dddot{\mathcal{T}}$.


%%%%%%%%%%%%%%%%%%%%% Algorithm
\begin{figure}[tb!]
	\begin{center}
		{\small
			\begin{minipage}{3.36in}
				\myhrule
				\vspace{-1ex}
				\mat{0ex}{
					{\bf Algorithm} ~\cisto$(\dddot{\mathcal{T}}[P_0,\ldots,P_n], ~\epsilon, ~m)$\\
					%	\sstab
					\bcc \hspace{1ex}\= $E' := \phi$;   ~$\overline{\mathcal{T}} := \phi$;   \\
					\icc \hspace{1ex}\= \For $s = 0$  \To $n - 2$ \Do \\
					\icc \>\hspace{3ex}\= $t_c$ := $P_{s+1}.t$;	\\
					\icc \>\hspace{3ex}\= $\mathcal{R}^* := {getRegularPolygon}$($P_s$, $P_{s+1}$, $\epsilon$, $m$, $t_c$); \\
					\icc \>\hspace{3ex}\= $E' = E' \cup (P_s,P_{s+1})$; \\
					\icc \>\hspace{3ex}\= \For $k = s + 2$  \To $n$ \Do \\
					\icc \>\hspace{6ex} \If ${\textcolor{red}{isInside}}$($\mathcal{R}^*$, $P_k$) \Then $E' = E' \cup (P_s, P_k)$; \\
					\icc \>\hspace{6ex} $\mathcal{R} := {getRegularPolygon}$($P_s$, $P_k$, $\epsilon$, $m$, $t_c$); \\
					\icc \>\hspace{6ex} $\mathcal{R}^* := {\rpia}(\mathcal{R}^*, ~\mathcal{R})$; \\
					\icc \>\hspace{6ex} \If $\mathcal{R}^* = \phi$ \Then \Break; \\
					\icc \hspace{0ex} $\overline{\mathcal{T}} = {shortestPath}(G')$;\\
					\icc \hspace{0ex} \Return $\overline{\mathcal{T}}$; \\
					%
					\\
					{\bf Procedure} ~\kw{getRegularPolygon}$(P_s,~P_i,~r,~m,~t_c)$ \\
					%	\bcc \hspace{1ex} \textcolor[rgb]{0.00,0.07,1.00}{Transform $P_s$ and $P_i$ to points in Cartesian coordinates} \\
					\bcc \hspace{1ex} $c := (t_c-t_s)/(P_i.t - P_s.t)$; \\
					\icc \hspace{1ex} $x := P_s.x + c\cdot(P_i.x-P_s.x)$; \\
					\icc \hspace{1ex} $y := P_s.y + c\cdot(P_i.y-P_s.y)$; \\
					\icc \hspace{1ex} \For $(j := 1;j \le m;j++)$ \Do \\
					\icc \> \hspace{2ex} $\theta :=  (2j + 1)*\pi /m $; \\
					\icc \> \hspace{2ex} $\mathcal{R}.v_j.x := x + c\cdot r\cdot\cos\theta$;\\
					\icc \> \hspace{2ex} $\mathcal{R}.v_j.y := y + c\cdot r\cdot\sin\theta$;\\
					\icc \hspace{1ex} \Return $\mathcal{R}$.

				}
				\vspace{-2ex}
				\myhrule
			\end{minipage}
		}
	\end{center}
	\vspace{-2ex}
	\caption{\small Near optimal spatio-temporal cone intersection algorithm (\cisto).}
	\label{alg:cisedo}
	\vspace{-1.5ex}
\end{figure}
%%%%%%%%%%%%%%%%%%%%%%%%%%%%%%%%%%%%%

The algorithm first initializes the start point $P_s$ to $P_0$, the edge set $E$ of
the reachability graph $G'$ to $\emptyset$ and the result trajectory $\overline{\mathcal{T}}$ to $\emptyset$, respectively (line 1).
%
Then the algorithm takes each data point of the trajectory as a start point $P_s$ (line 2). 
For each start point $P_{s}$, it initializes $t_c$ (line 3), sets the intersection of regular polygons $\mathcal{R}^*$ to the $m$-inscribed regular polygon \wrt the next point $P_{s+1}$ by calling procedure $\kw{getRegularPolygon}$ (line 4) and includes the edge $(P_s, P_{s+1})$ to $E$ (line 5).
%
Next, it checks for each point $P_{k} (k>s+1)$ in the trajectory whether the edge ($P_{s}, P_{k}$) can be included into the edge set $E'$ by calling procedure \textit{isInside}() (lines 6--7). %of reachability graph $G'$
It gets the regular polygon $\mathcal{R}$ of $P_{k}$ by calling procedure \textit{getRegularPolygon}()
and updates the intersection $\mathcal{R}^*$ by calling procedure $\rpia()$ introduced in Section~\ref{subsec-fastRPI} (lines 8--9).
If $\mathcal{R}^* = \emptyset$, the checking for the current start point $P_s$ is terminated (line 10) and the process goes on to $P_{s+1}$.
This process repeats until all possible edges have been included.
%
Finally, it computes the shortest path from point $P_{0}$ to point $P_{n}$ (line 11), and returns an optimal  piece-wise line representation $\overline{\mathcal{T}}$ of trajectory \trajec{T}${[P_0, \ldots, P_n]}$ (line 12).




%%%%%%%%%%%%%%%%%%%%%%%%%%%%%%%%%%%%%%%%%%%%%%%example of Algorithm CISED-O
\begin{figure*}[tb!]
	\centering
	\includegraphics[scale=0.75]{figures/Fig-CISED-O.png}
	%\vspace{-2ex}
	\caption{\small A running example of the \cisto algorithm. The points and the oblique circular cones are projected on an x-y space. The trajectory $\dddot{\mathcal{T}}[P_0, \ldots, P_{10}]$ is compressed into three line segments.}
	\vspace{-1ex}
	\label{fig:exm-consto}
\end{figure*}
%%%%%%%%%%%%%%%%%%%%%%%%%%%%%%%%%%%%%%%%%%%%%%%%%%%%%%%%%%%%%%%%%%%%%%%%%%%


\begin{example}
\label{exm-alg-conesto}
Figure~\ref{fig:exm-consto}  shows a running example of algorithm \cisto for compressing the trajectory \trajec{T} in Figure~\ref{fig:notations} again.


\sstab (1) After initialization, the \cisto algorithm reads point $P_1$ and
builds an \emph{oblique circular cone} \cone{(P_0, \mathcal{O}(P_{1},
  \epsilon))},  and projects it on the plane $P.t-P_1.t=0$. The inscribed
regular polygon $\mathcal{R}_1$ of the projection circle  is returned and the intersection $\mathcal{R}^*$ is set to $\mathcal{R}_1$.
Since the intersection of $\vv{P_0P_{1}}$ and \cone{(P_0,  \mathcal{O}(P_{1}, \epsilon))} is not $\{P_0\}$, edge
$(P_0,P_{1})$ is added to the graph $G$. 


\sstab (2) $P_2$, $P_3$ and $P_4$ are processed in turn. The intersection
polygons $\mathcal{R}^*$ are not empty and edges $(P_0,P_{2})$,$(P_0,P_{3})$,$(P_0,P_{4})$ are added to the graph $G$. 


\sstab (3) For point $P_5$, the intersection of $\vv{P_0P_{5}}$ and $\bigsqcap_{i=1}^{
  4}$\cone{(P_0,  \mathcal{O}(P_{i}, \epsilon))} is $\{P_0\}$, edge
$(P_0,P_{5})$ is not added to the graph $G$. 
And the intersection of $\mathcal{R}_5$ and $\mathcal{R}^*$ is $\emptyset$.
Thus, the checking from point $P_0$ is terminated and  a new  checking cycle is started such that $Q=P_1$ is the new start point and plane $P.t-P_2.t=0$ is the new projection plane.

\sstab (4) At last, the algorithm outputs 3 continuous line segments, \ie $\vv{P_0P'_4}$, $\vv{P'_4P_7}$ and $\vv{P_7P_{10}}$, in which $P'_4$ is an interpolated data points not in \trajec{T}.
\end{example}



\stitle{Correctness and complexity analyses}.
The correctness of algorithm \cisto follows from Propositions~\ref{prop-circle-intersection}, ~\ref{prop-edge-check} and~\ref{prop-near-opt-graph}.
It is also easy to see that this algorithm takes $O(n^2)$ time and $O(n^2)$ space.
