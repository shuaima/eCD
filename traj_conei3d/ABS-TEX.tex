Various mobile devices have been used to collect, store and transmit tremendous trajectory data, and it is known that raw trajectory data seriously wastes the storage, network band and computing resource. Line simplification algorithms are an effective approach to attacking this issue by compressing data points in a trajectory to a set of continuous line segments, and are commonly used in practice. However, although there exist one-pass line simplification algorithms appropriate for resource-constrained devices, none of them uses the synchronous Euclidean distance (SED), and cannot support spatio-temporal queries. In this study, we develop two one-pass error bounded trajectory simplification algorithms (CISED-S and  CISED-W) using the synchronous Euclidean distance, based on a novel spatio-temporal cone intersection technique. Using four real-life trajectory datasets, we experimentally show that our approaches are both efficient and effective. In terms of running time, algorithms CISED-S and  CISED-W are on average $3$ times faster than SQUISH-E (the most efficient existing LS algorithm using SED). In terms of compression ratios, algorithms CISED-S and  CISED-W are comparable with and $19.6\%$ better than DPSED (the most effective existing LS algorithm using SED) on average, respectively, and are $21.1\%$ and $42.4\%$ better than SQUISH-E on average, respectively. 