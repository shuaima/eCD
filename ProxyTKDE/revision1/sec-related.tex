\section{Related Work}
\label{sec-related}

Algorithms for shortest  paths and distances have been extensively studied since 1950's, and fall into different categories in terms of different criteria:

 \bi
 \item exact distances~\cite{WuXDCZZ12,Dijkstra59,FredmanT84,LubyR89,GeisbergerSSD08,SankaranarayananSA09,SandersS05,GoldbergH05,MozesS12,ChengKCC12,MozesS12,ChanL07,SaundersT07,WagnerW07,bast2014route,delling2014robust,arz2013transit,zhu2013shortest,klein2010shortest,fakcharoenphol2006planar,gupta2004roads}
     and approximate distances~\cite{PotamiasBCG09,SarmaGNP10,ThorupZ05,SankaranarayananS10},


 \item memory-based~\cite{PotamiasBCG09,SarmaGNP10,WuXDCZZ12,Dijkstra59,FredmanT84,LubyR89,GeisbergerSSD08,Wei10,SankaranarayananSA09,SandersS05,
ThorupZ05,MozesS12,SaundersT07,WagnerW07,bast2014route,delling2014robust,arz2013transit,SankaranarayananS10,zhu2013shortest,klein2010shortest,gupta2004roads} and disk-based algorithms~\cite{ChengKCC12,ChanL07},


 \item for unweighted~\cite{PotamiasBCG09,SarmaGNP10,Wei10,bast2014route,delling2014robust,arz2013transit} and weighted graphs~\cite{WuXDCZZ12,Dijkstra59,FredmanT84,LubyR89,GeisbergerSSD08,SankaranarayananSA09,GoldbergH05,MozesS12,SandersS05,ChengKCC12,ThorupZ05,MozesS12,ChanL07,SaundersT07,WagnerW07,bast2014route,delling2014robust,arz2013transit,SankaranarayananS10,zhu2013shortest,klein2010shortest,fakcharoenphol2006planar,gupta2004roads},
     and


 \item for directed~\cite{SaundersT07,GoldbergH05,MozesS12,bast2014route,delling2014robust,arz2013transit,zhu2013shortest,klein2010shortest,fakcharoenphol2006planar} and undirected graphs~\cite{PotamiasBCG09,SarmaGNP10,WuXDCZZ12,Dijkstra59,FredmanT84,LubyR89,GeisbergerSSD08,Wei10,SankaranarayananSA09,SandersS05,ChengKCC12,ThorupZ05,MozesS12,ChanL07,WagnerW07,bast2014route,delling2014robust,arz2013transit,SankaranarayananS10,gupta2004roads}.
 \ei


In this work, we study the memory-based (exact) shortest path and shortest distance problem on weighted undirected large graphs.
%
We next introduce those methods that fall into this category from two aspects: on general graphs and on road networks.

\stitle{Approaches for general graphs}.
The classic solution for shortest path and distance queries is Dijkstra's algorithm \cite{Dijkstra59}. It visits nodes in an ascending order of their distances from the source node, and all the nodes that are closer to the source node than the target node are visited. Thus, many techniques are proposed to reduce the search space. Among these, (1) bidirectional Dijkstra search \cite{LubyR89} was proposed to search from both source and target node; (2) \alt \cite{GoldbergH05} selects a set of nodes (reffered to as landmarks), and pre-computes the distances from each node to landmarks, which are used to prune unnecessary nodes during the search process; (3) An edge labeling method named \arcflag \cite{MohringSSWW05} cuts graphs into partitions to reduce the search space;
(4) An indexing and query processing scheme named \tedi \cite{Wei10} decomposes a graph into a tree, and uses the tree index to process shortest path queries; And (5) a 2-hop labeling based exact algorithm was proposed in \cite{delling2014robust} to deal with large networks.



\eat{
Though these studies improve the efficiency for answering shortest path and distance queries, they are inefficient for large scale graphs. Hence, many techniques have been proposed \cite{ChengKCC12, delling2014robust, PotamiasBCG09, Wei10}. Cheng et al. \cite{ChengKCC12} propose a novel disk-based index for processing single-source shortest path or distance queries. Wei \cite{Wei10} and Delling et al.\cite{delling2014robust} study the problem of computing exact query on large scale graphs.
}


\stitle{Approaches for road networks}.
 A lot of work focuses on processing shortest path and distance queries on road networks. Different from general graphs, the shortest paths on road networks are often spatially coherent. Path oracles have been proposed for spatial networks~\cite{SankaranarayananSA09}. Transit Node Routing (\tnr) \cite{arz2013transit} is a fast and exact distance oracle for road networks. Both of them utilize the property of spatial coherence, i.e. spatial positions of both source and destination vertices and the shortest paths between them which facilitates the aggregation of source and destination vertices into groups that share common vertices or edges on the shortest path between them. Moreover, road networks are also often assumed to be planar graphs with non-negetive weights\cite{fakcharoenphol2006planar,gupta2004roads,klein2010shortest,MozesS12}. A hierarchical index structure is used by many techniques\cite{SandersS05, GeisbergerSSD08, zhu2013shortest}. For example, Sanders et al. propose a route planning method named Highway Hierarchies (HH)\cite{SandersS05}. They construct a highway hierarchy such that only high level edges need to be considered to compute the path and distance from a source to a far target. Inspired by HH, Geisberger et al.\cite{GeisbergerSSD08} propose a road network index named Contraction Hierarchies (CH), which is a extreme case of HH. Zhu et al. further propose Arterial Hierarchy (AH)\cite{zhu2013shortest} that narrows the gap between theory and practice in answering shortest path and distance queries.

Note that to achieve a better query efficiency, except for (bidirectional) Dijkstra, all aforementioned approaches require preprocessing to answer a shortest path or distance query. The phase of query answering is highly dependent on the phase of preprocessing. Thus, it is hard to classify the techniques into techniques used for preprocessing and techniques used for query answering. These techniques are different from our proxy in the following two aspects. First of all, these techniques rely on different indices. As a result, applying one technique could very well preclude the other from being applied. However, our proxy is a general approach that can be easily combined with the other methods. As a data reduction method, it can be used as a pre-processing step before these techniques are applied. As we Specifically, since the routing proxies can represent their DRAs, we can reduce graph sizes by removing all the nodes in the DRAs. Then we can process the query as we have discussed in Section~\ref{sec-query}. Secondly, these techniques performs a trade-off between preprocessing and query answering. To achieve a better query efficiency, these techniques usually incurs high preprocessing time and space overhead. However, our proxy is a light-weight optimization technique which scales well to large networks.


Most close to our work is the study of 1-dominator sets in~\cite{SaundersT07}. Different from the aforementioned techniques, it is proposed for shortest path queries on nearly acyclic directed graphs rather than undirected graphs. When an undirected graph is converted to an equivalent directed graph, each undirected edge is replaced by a pair of inverse directed edges. Hence, 1-dominator sets~\cite{SaundersT07} are not applicable for undirected graphs. However, routing proxies and deterministic routing areas proposed in this study  are for undirected graphs,  and  significantly different from 1-dominator sets (from definitions to analyses to algorithms).








