\documentclass[11pt]{letter}

\usepackage{times,,xspace,amsmath,amssymb,color}
\usepackage[english]{babel}
\parindent=0cm
\parskip=0.28cm
\setlength{\textwidth}{17cm}
\setlength{\oddsidemargin}{-0.24cm}
\setlength{\evensidemargin}{-0.24cm}
\setlength{\topmargin}{0cm}
\setlength{\headheight}{0cm}
\setlength{\textheight}{24cm}
\setlength{\headsep}{0cm}
\newcommand{\eat}[1]{}
\date{}

\newcommand{\sstab}{\rule{0pt}{8pt}\\[-3.4ex]}
\newcommand{\stab}{\rule{0pt}{8pt}\\[-2.4ex]}
\newcommand{\vs}{\vspace{1ex}}
\newcommand{\svs}{\vspace{0.36ex}}
\newcommand{\kw}[1]{{\ensuremath {\mathsf{#1}}}\xspace}
\newcommand{\dist}{\kw{dist}}
\newcommand{\pred}{\kw{pred}}
\newcommand{\desc}{\kw{desc}}
\newcommand{\pSim}{\kw{Match}}
\newcommand{\at}[1]{\protect\ensuremath{\mathsf{#1}}\xspace}

\newcommand{\NP}{\kw{NP}}
\newcommand{\DAGs}{{\sc dag}s\xspace}
\newcommand{\NC}{\kw{NC}\xspace}
\newcommand{\coNP}{co\kw{NP}\xspace}
\newcommand{\PTIME}{\kw{PTIME}}
\newcommand{\PSPACE}{\kw{PSPACE}}
\newcommand{\EXPTIME}{\kw{EXPTIME}\xspace}
\newcommand{\NPSPACE}{\kw{NPSPACE}\xspace}


\newcommand{\bi}{\begin{itemize}}
\newcommand{\ei}{\end{itemize}}
\newcommand{\be}{\begin{enumerate}}
\newcommand{\ee}{\end{enumerate}}
\newcommand{\im}{\item}
\newenvironment{tbi}{\begin{itemize}
        \setlength{\topsep}{1.5ex}\setlength{\itemsep}{0ex}\vspace{-0.5ex}}
        {\end{itemize}\vspace{-0.5ex}}
\newenvironment{tbe}{\begin{enumerate}
        \setlength{\topsep}{0ex}\setlength{\itemsep}{-0.7ex}\vspace{-1ex}}
        {\end{itemize}\vspace{-1ex}}

\newcommand{\eps}{\prec}
\newcommand{\deps}{\prec_{D}}
\newcommand{\leps}{\prec_L}
\newcommand{\dleps}{\prec_{D}^{L}}
\newcommand{\iso}{\lhd}
\newcommand{\bieps}{\sim}
\newcommand{\embed}{\lessdot}
\newcommand{\neps}{\ntrianglelefteq}
\newcommand{\ees}{\preceq_{(e,e)}}
\newcommand{\nees}{\not\preceq_{e,e}}
\newcommand{\Reps}{S}
\newcommand{\bcp}{{\sc bcp}\xspace}
\newcommand{\ie}{\emph{i.e.,}\xspace}
\newcommand{\eg}{\emph{e.g.,}\xspace}
\newcommand{\wrt}{\emph{w.r.t.}\xspace}
\newcommand{\aka}{\emph{a.k.a.}\xspace}
\newcommand{\kwlog}{\emph{w.l.o.g.}\xspace}


\definecolor{gray}{rgb}{0.5,0.5,0.5}
\newcommand{\added}[1]{\textcolor{blue}{#1}}
\newcommand{\changed}[1]{\textcolor{red}{#1}}
\newcommand{\removed}[1]{\textcolor{gray}{#1}}

\newcommand{\ball}[1]{\hat{G}[#1]}
\newcommand{\amazon}{\kw{Amazon}}
\newcommand{\Amazon}{\kw{Amazon}}
\newcommand{\youtube}{\kw{YouTube}}
\newcommand{\YouTube}{\kw{YouTube}}


\newcommand{\match}{\kw{Match}}
\newcommand{\optmatch}{\kw{Match^+}}
\newcommand{\dismatch}{\kw{dMatch}}
\newcommand{\optdismatch}{\kw{dMatch^+}}
\newcommand{\minq}{\kw{minQ}}
\newcommand{\graphsim}{\kw{Sim}}
%\newcommand{\subiso}{\kw{SubIso}}
%\newcommand{\dissubiso}{\kw{dSubIso}}
\newcommand{\metis}{{\sc Metis}\xspace}
\newcommand{\vf}{\kw{VF2}}
\newcommand{\tale}{\kw{TALE}}
\newcommand{\mcs}{\kw{MCS}}
\newcommand{\dsim}{\kw{dSim}}
\newcommand{\dissubiso}{\kw{dVF2}}
\newcommand{\dvf}{\kw{dVF2}}

\newcommand{\stitle}[1]{\vspace{0.5ex} \noindent{\bf #1}}
\newcommand{\etitle}[1]{\vspace{1ex}\noindent{\underline{\em #1}}}

\newcommand{\ah}{{\sc ah}\xspace}
\newcommand{\alt}{{\sc alt}\xspace}
\newcommand{\tedi}{{\sc tedi}\xspace}
\newcommand{\arcflag}{{\sc arcFlag}\xspace}
\newcommand{\tnr}{{\sc tnr}\xspace}
\newcommand{\dra}{{\sc dra}\xspace}
\newcommand{\dras}{{\sc dra}s\xspace}



\begin{document}



\noindent
Prof. Jian Pei,\\
Editor-in-Chief,\\
IEEE Transactions on Knowledge and Data Engineering

\vspace{0.3cm}
\noindent
Dear Prof. Pei,


Attached please find a revised version of our submission to IEEE Transactions on Knowledge and Data Engineering,
{\em Proxies for Shortest Path and Distance Queries}.

The paper has been substantially revised according to the referees' comments. In particular,
%
(1) we have reorganized our paper, by adding a separate section for query answering with routing proxies (Section 5), a separate section for related work (Section 7) and an appendix section for proofs, and by adjusting the section on the properties of proxies and DRAs (Section 3.2) and the experimental study section (Section 6),
%
(2) we have shown the benefits of using proxies for a new approach Arterial Hierarchy(\ah) [38] and added the space overhead evaluation (Table 2) in the experimental study,
%
(3) we have added the discussions of three new references in the related work,
%
(4) we have rewritten and improved several proofs, and, finally,
%
(5) we have also taken this opportunity to rewrite several parts of the paper to improve the presentation.



We would like to thank all the referees for their thorough reading of our
paper and for their valuable comments.

Below please find our responses to the comments by the referees.

%%%%%%%%%%%%%%%%%%%%%
\vspace{3.6ex}
\hrule
\vspace{0.6ex}

\vspace{2ex} \stitle{Response to the comments of Referee 1}.

{\em
{\bf [R1a]} The theory behind the reduction technique is very clear and the theoretical results around the technique are good. The claim is that the reduction technique is very general
(in the sense that any existing algorithm can be applied on top of it).
But, is it efficient for denser graphs? I would expect there will be only a few nodes in DRAs in case of dense graphs. Then one might get very small improvement in efficiency. Considering
the fact that there will be preprocessing time, it would be good to have more insights about the effect of the reduction as the sparsity changes. Also, in the experiments the datasets are very sparse.
The insight about sparsity is missing in the experiments. Definitely, there is a trade-off between the amount of reduction in size of the graph by the proposed technique and the sparsity of the graph.
So, the point (2) in summary might need some justification.
}
\svs

Thanks for your suggestion. In our revised version, we add a new set of experiments studying the effect of graph density on the performance of our method. Specifically, we adopt the graph-tool to generate random graphs. For a fixed number of nodes, we vary the average degree of the graph to evaluate the performance of our algorithm. The results show that less nodes can be captured by DRAs when the density of the graph is high. This is because: (1) There are less cut-nodes in graphs with higher density, thus less proxies can be found. (2) For a proxy $u$, the set of nodes $A_u^i$ that presented by $u$ is more likely to exceed the size constraint. Therefore, there are less nodes in its \dra.


\noindent
{\em
{\bf [R1b]}   AH is very well known baseline. Given that there is a preprocessing time, the reduction technique could only improve 1\% of the efficiency. Also, the described space overhead is tricky. Because the proposed technique also claims extra space in time of pre-processing.}
\svs

(1) As a preprocessing step, it only needs to be done once, no matter how many shortest distance/path queries are invoked later on. That is, we can still gain benefit.

(2) The extra space is already included in the report (Table 1), and the space overhead of Proxy+\ah (including the extra space) is smaller than \ah (about 18\% smaller).



\noindent
{\em{\bf[R1c]} I like the fact that the reduction technique can be applied as a pre-step of any existing algorithm. I described my concern in (a).}
\svs

Thanks.


\noindent{\em{\bf[R1d]} I still feel the experimental section could be improved. The experiments look very repetitive. Table 1 is extremely helpful. Some of the figures could be replaced by a table.}
\svs

Thanks. We have replaced the figures by tables.


\noindent{\em{\bf[R1e]} In Example 4, the merging step is missing. (It is not necessary, but it would be better to have for sake of completeness).}
\svs

Thanks. We have fixed the problem.

\noindent{\em{\bf[R1f]}  Page 2, first paragraph: "bidirectional" is misspelled.}
\svs

Fixed! Thanks!

\noindent{\em{\bf[R1g]} Page 7, last paragraph: ``AH is one of the state-of-art method" $\rightarrow$ "AH is one of the state-of-art methods''.}
\svs


Fixed! Thanks!



\vspace{2.8ex}
\hrule
\vspace{0.6ex}
{\bf Response to the comments of Referee 2.}



\vs
\noindent
{\em
{\bf [R2C1]}
1. Please mention before Prop 1 that all proofs are in the appendix.}
\svs

This has been clarified in the end of the first paragraph of Section 3.

\vs
\noindent
{\em
{\bf [R2C2]} Why is proposition 2 required - please verify it is indeed necessary to further the understanding of the paper. If not, you may remove it.}
\svs

Indeed, `each proxy has one and only one DRA' is a way to show that proxies and DRAs are well defined. That is, uniqueness is a way to show a well-defined concept. This has been clarified right before Proposition 2.

\vs
\noindent
{\em
{\bf [R2C3]} The BC-Sketch graph is also known as BLOCK GRAPH (see the graph theory book by Diestel). Why do you introduce a new name, and even if you want to do that because of slight differences in definition, please refer to a graph theory book to put the thing in the correct context.}
\svs

Indeed, we were NOT aware of the concept of BLOCK GRAPH at all. This has been clarified after the introduction of Example 3. Thanks for providing us with this valuable information!


\vs
\noindent
{\em{\bf[R2C4]}  In the algorithm "computeDRAs" (fig 4) line 15 is confusing : you say $A^+_{v'}$ of proxy v' = X' but X' is a set of BCC's, while the DRA is a set of nodes; that is unless, and I think this is what it is : you want not only the DRA as a set of nodes but also its breakup into its constituent BCC's $A^1_{v'}$, $A^2_{v'}$, ... -- indeed you seem to be using this breakup in your query answering algorithm in page 11, see the paragraph on "Query answering" part I. Can this be clarified a bit?}
\svs

Yes, your observation is absolutely right, and we have further clarified by adding a note at line 15 in algorithm computeDRAs (Figure 4).
We also fixed a typo at line 15 by changing `the DRA $A^+_{v'}$' to `the $A^+_{v'}$'. Indeed, a \dra is a subgraph (see its definition at page 2). Thanks!


\vs
\noindent
{\em{\bf[R2C5]} Algorithm computeDRA's is reasonably clear and intuitive, but its proof needs to be explained more. It is definitely believable but the proof should be explained better using lemmas etc. I believe it is correct though, but its presentation can be improved.}
\svs




\vspace{3.6ex}
\hrule
\vspace{3.6ex}
\closing{Your sincerely,}

\vspace{-8ex}
Shuai Ma, Kaiyu Feng, Jianxin Li, Haixun Wang, Gao Cong, and Jinpeng Huai
\end{document}
