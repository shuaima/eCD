%\vspace{-0.5ex}
\section{Conclusions}
\label{sec-conclude}
%\vspace{-1ex}

We have proposed \pCFDs and \pCINDs, which further extend \CFDs and \CINDs,
respectively,  by allowing patterns on data values to be expressed in
terms of $\ne, <, \le, >$ and
$\ge$ predicates. We have shown that \pCFDs and \pCINDs
are more powerful than \CFDs and \CINDs for detecting errors
in real-life data. In addition, the
satisfiability and implication problems for \pCFDs and \pCINDs have
the same complexity bounds as their counterparts for \CFDs and
\CINDs, respectively. We have also provided automated methods to
generate \SQL queries for detecting errors based on \pCFDs and
\pCINDs. These provide commercial \rdms with an immediate capability to
capture errors commonly found in real-world data.


One topic for future work is to develop a dependency language that
is capable of expressing various extensions of \CFDs (\eg \pCFDs,
e\CFDs~\cite{icde08} and {{\sc cfd$^c$}{\small s}~\cite{ChenFM09}),
without increasing the complexity of static analyses. Second, we are
developing effective algorithms for discovering \pCFDs and \pCINDs,
along the same lines as \cite{CM08,divesh08,icde09}. Third, we plan
to extend the methods of~\cite{sigmod05,repair} to repair data based
on \pCFDs and \pCINDs, instead of using \CFDs~\cite{repair},
traditional \FDs and \INDs~\cite{sigmod05}, denial
constraints~\cite{BertossiBFL08,ChomickiM05}, and aggregate
constraints~\cite{FlescaFP05}.
