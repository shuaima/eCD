%%% Local Variables:
%%% mode: latex
%%% TeX-master: "gis18"
%%% End:
\section{evaluation}
\label{sec-exp}

In this section, we present an extensive experimental study of our one-pass trajectory tracking algorithms (\citt, \sitt and \bitt) compared with the
existing algorithms of \ldrh and \grts on trajectory datasets. Using three real-life trajectory datasets, we conducted sets of experiments to evaluate:
(1) the number of messages (including data points and velocities),
(2) the compression ratios,
(3) the average errors, and
(4) the running time of algorithms \citt, \sitt and \bitt vs. \ldrh and \grts. 
Among them, the impacts of error bounds and distance metrics on messages, errors and running time of these algorithms are evaluated. 

\subsection{Experimental setting}

datasets

algorithms

metrics


\subsection{Experimental Results}

\subsubsection{Evaluation of messages}
This section evaluates the number of messages (including data points and velocities) of algorithms \citt, \sitt and \bitt vs. \ldrh and \grts.

\stitle{Exp1: impacts of the shape of rectangle (\ped vs. \sed) on messages of \bitt.}

show 6 curves (1:2, 1:4, 1:8, 1:16, 1:32, 1:64) in a figure? or a ped-sed-z 3d figure.

x: threshold (\ped) from 10 to 200

y: total messages (percentage), data points (percentage) and velocities (percentage)


\stitle{Exp2: impacts of distance metrics and error bounds on messages.}

show five algorithms in a figure, 

x: thresholds from 10 to 200

y: total messages (percentage), data points (percentage) and velocities (percentage)


\subsubsection{Evaluation of compression ratios}

\subsubsection{Evaluation of errors}

\subsubsection{Evaluation of running time}