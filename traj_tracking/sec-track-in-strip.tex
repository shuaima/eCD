
%%% Local Variables:
%%% mode: latex
%%% TeX-master: "www2019"
%%% End:

\section{One-pass tracking in strip areas}
\label{sec:strip}


Position tracking and trajectory tracking both use \sed as the distance metric to check data points that confirms the actual position of a moving object at time $t$ lives in a circular area around the excepting position of the moving object at that time.
Though \sed is currently the only distance metric used in tracking moving objects, we argue that another distance metric, \ped that is wildly used in trajectory simplification and always has remarkable better compression ratios than \sed, is also important in tracking moving objects that confirms moving objects lives in strip areas, \eg a school boy is expected on his way home no matter the speed of his movement.
This section introduces novel methods that use \ped to track moving object in strip areas.


%\subsection{Position tracking in strip areas}
%based on ldr, a direction is needed, no speed;
%  --  distance checking during computing, ped: from a point to a ray.
  

\subsection{Sector and its use in trajectory simplification}

\emph{Sector intersection} \cite{Williams:Longest, Sklansky:Cone, Dunham:Cone, Zhao:Sleeve} is developed in fields of computational geometry and cartographic, where a \emph{sector} is largely a simplified version of a spatio-temporal cone in an x--y 2D space that the temporal information is ignored, that converts the \ped distance tolerance into an angle tolerance for efficiently checking data points.
%
Given a sequence of points $[P_{s}, P_{s+1}, \ldots, P_{s+k}]$ and an error bound $\epsilon$,
for the start data point $P_s$, any point $P_{s+i}$ and $|\vv{P_sP_{s+i}}|>\epsilon$ ($i\in[1, k]$), there are two directed lines $\vv{P_sP^u_{s+i}}$ and $\vv{P_sP^l_{s+i}}$ such that $ped(P_{s+i}, \vv{P_sP^u_{s+i}})$ $=$ $ped(P_{s+i}, \vv{P_sP^l_{s+i}}) = \epsilon$ and either ($\vv{P_sP^l_{s+i}}.\theta < \vv{P_sP^u_{s+i}}.\theta ~and~\vv{P_sP^u_{s+i}}.\theta - \vv{P_sP^l_{s+i}}.\theta <\pi$) or ($\vv{P_sP^l_{s+i}}.\theta > \vv{P_sP^u_{s+i}}.\theta ~and~ \vv{P_sP^u_{s+i}}.\theta - \vv{P_sP^l_{s+i}}.\theta < -\pi)$. Indeed, they form a \emph{sector} \sector{(P_s, P_{s+i}, \epsilon)} that takes $P_s$ as the center point and $\vv{P_sP^u_{s+i}}$ and $\vv{P_sP^l_{s+i}}$ as the borderlines.

The sector intersection approach is the key of one-pass trajectory simplification algorithms using \ped. Like the checking of the common intersection of spatio-temporal cones in \cised, these algorithms check the common intersection of sectors, \ie $\bigsqcap_{i=1}^{k}$\sector{(P_s, P_{s+i}, \epsilon)} $\ne \{P_s\}$ \cite{Williams:Longest, Sklansky:Cone,Zhao:Sleeve}, to know whether the sub-trajectory can be represented by a line segment $\overline{P_sQ}$ \wrt error bound $\epsilon$, where $Q$ is a point that may not belong to the sub-trajectory. However, if $Q$ must be $P_{s+k}$, the last point of the sub-trajectory, then (1) the full-$\epsilon$ sectors should be replaced by half-$\epsilon$ sectors, and (2) $P_{s+k}$ should be one of the points have the furthest distances to $P_s$, \ie $|P_sP_{s+k}| > max\{|P_sP_{s+i}|\} - \epsilon$ for each $i \in (0, k)$. 
%

\todo{figure}

\begin{example}
	\label{exm-sleeve}
	\todo.
\end{example}

\subsection{Tracking by sectors}

\todo{sectors can be used in trajectory tracking. }

Because the half-$\epsilon$ sectors may limit its compression performance, we also extend it to full-$\epsilon$ sectors for a better performance.


%, that is, for any data point $P_{s+i}$ ($i \in [1, ... k]$), its perpendicular Euclidean distance to line segment $\overline{P_sP_{s+k}}$ is not greater than the error bound $\epsilon$.




\begin{theorem}
	\label{theo-full-sector}
	Given a sub-trajectory $[P_s,...,P_{s+k}]$ and an error bound $\epsilon$, $ped(P_{s+i}, \vv{P_sP_{s+k}})\le \epsilon$ for each $i \in [1,k]$ if line segment $\vv{P_sP_{s+i}}$ passes through $\bigsqcap_{j=1}^{i-1}$\sector{(P_s, P_{s+j}, \epsilon)} - \{$P_s$\} ~and~ $|P_sP_{s+i}| > max\{|P_sP_{s+j}|\} - \epsilon$ for each $j \in (0, i)$.
\end{theorem}

\begin{proof}
	\todo.
\end{proof}

Theorem \ref{theo-full-sector} tells that full-$\epsilon$ sectors with some constrains can also be used in trajectory simplification and/or tracking. 

\begin{theorem}
	\label{theo-sector-vs}
	Given a sub-trajectory $[P_s,...,P_{s+k}]$ and an error bound $\epsilon$, if $\bigsqcap_{i=1}^{k}$\sector{(P_s, P_{s+i}, \epsilon/2)} $\ne \{P_s\}$, then for each $i \in [2, k]$, line segment $\vv{P_sP_{s+i}}$ passes through $\bigsqcap_{j=1}^{i-1}$\sector{(P_s, P_{s+j}, \epsilon)}-$\{P_s\}$; and the opposite is not always true.
\end{theorem}

\begin{proof}
	\todo.
\end{proof}

Theorem \ref{theo-sector-vs} tells that the full-$\epsilon$ sector approach also brings a better effectiveness than the half-$\epsilon$ sector in trajectory simplification and/or tracking.
We next present a {\em trajectory tracking} algorithm based on full-$\epsilon$ sectors (\sitt) that runs in a similar routine as \citt. 

%%%%%%%%%%%%%%%%%%%%%%%%%%%%%%%%%%%%%%%%%%%%%%%%%%%%%%%%%%%%%%%%%%%%%%%%%%
% Algorithm: Traj tracking based on section intersection using full sectors.
\begin{figure}[tb!]   
	\begin{center}
		{\small
			\begin{minipage}{3.3in}
				\myhrule
				%\vspace{-1ex}
				\mat{0ex}{
					{\bf Algorithm}~\sitt $(\dddot{\mathcal{T}}[P_0,\ldots,P_n],~\epsilon)$\\
					%	\sstab
					\bcc \hspace{1ex}\= $P_s := P_0$; ~~~~$\mathcal{S}^*$ := \kw{getSector}($P_s$, $P_{s+1}$, $\epsilon$); ~~~~$l_{max} = |P_sP_{s+1}|$;\\
					\icc \hspace{1ex}\= $\vv{v}.\theta:=\vv{P_{s}P_{s+1}}.\theta$;  \\
					\icc \hspace{1ex}\= update ($P_{s}, \vv{v}$); 	\\
					\icc \hspace{1ex}\= $i:= 2$;  	\\
					\icc \hspace{1ex}\= while $i \le n$ do \\
					\icc \>\hspace{3ex} if $\vv{P_sP_{i}}$ ~does not pass through~ $\mathcal{S}^*$ or $|P_sP_{i}| \le l_{max} - \epsilon$ then \\ % // updates velocity and location \\
					\icc \>\hspace{7ex}    $P_s := P_{i-1}$; ~~~~$\mathcal{S}^*$ := $\emptyset$; ~~~~$l_{max} = |P_sP_{i}|$;\\
					\icc \>\hspace{7ex}    $\vv{v}.\theta:=\vv{P_{s}P_{i}}.\theta$;  \\
					\icc \>\hspace{7ex}    update ($P_{s}, \vv{v}$); 	\\
					\icc \>\hspace{3ex} else if $\vv{P_sP_{i}}$ ~passes through~ $\mathcal{S}^*$ ~and~ $|P_sP_{i}| > l_{max} - \epsilon$ \\ \hspace{9ex} ~and~ $ped(P_i, \vv{v}) \ge \epsilon $ then  \\ %~// updates velocity only 
					\icc \>\hspace{7ex}    $\vv{v}.\theta:=\vv{P_sP_{i}}.\theta$; \\
					\icc \>\hspace{7ex}    update ($\vv{v}$); \\
					\icc \>\hspace{3ex} if $\mathcal{S}^*=\emptyset$ then $\mathcal{S}^*:=$ \kw{getSector}($P_s$, $P_{i}$, $\epsilon$); \\
					\icc \>\hspace{3ex} else $\mathcal{S}^*$ := $\mathcal{S}^*\bigsqcap$ \kw{getSector}($P_s$, $P_{i}$, $\epsilon$); $l_{max} = \max\{|P_sP_{i}|, l_{max}\}$;\\
					\icc \>\hspace{3ex} $i$ := $i +1$;\\
					\icc \>\hspace{0ex} update ($P_{n}$); 
				}
				\vspace{-2ex}
				\myhrule
			\end{minipage}
		}
	\end{center}
	\vspace{-1ex}
	\caption{\small Trajectory tracking based on sector.}
	\label{alg:sitt}
	\vspace{-1ex}
\end{figure}
%%%%%%%%%%%%%%%%%%%%%%%%%%%%%%%%%%%%%

\subsection{Implementation}

%proposition 2: full-sector is better than half sector.

\stitle{Algorithm \sitt.}
\todo{description; and tips for those points having distances to $P_s$ smaller than $\epsilon$ ...}

\begin{example}
	\todo.
\end{example}


\stitle{Correctness and complexity.} 
\todo.


\eat{%%%%%%%%
	\begin{example}
		\label{exm-alg-sleeve}
		Figure~\ref{fig:sleeve} is a running example of algorithm \siped($\frac{\epsilon}{2}$) taking as input the same trajectory $\dddot{\mathcal{T}}[P_0, \ldots, P_{10}]$. At the beginning, $P_0$ is the first start point, and points $P_1$, $P_2$, $P_3$, etc., each has a narrow \emph{sector}.
		For example, the narrow \emph{sector} $\mathcal{S}$($P_0$, $P_{3}$, $\epsilon/2$) takes $P_0$ as the center point and $\vv{P_0P^u_{3}}$ and $\vv{P_0P^l_{3}}$ as the borderlines.
		Because $\bigsqcap_{i=1}^{4}\mathcal{S}(P_0, P_{0+i}, \epsilon/2) \ne \{P_0\}$ and $\bigsqcap_{i=1}^{5}\mathcal{S}(P_0, P_{0+i}, \epsilon/2) = \{P_0\}$, $\vv{P_0P_4}$ is output and $P_4$ becomes the start point of the next section.
		At last, the algorithm outputs two continuous line segments $\vv{P_0P_4}$ and $\vv{P_4P_{10}}$.
	\end{example}
	
	\begin{figure}[tb!]
		\centering
		\includegraphics[scale=0.66]{Figures/Fig-sleeve.jpg}
		\vspace{-2ex}
		\caption{\small The trajectory $\dddot{\mathcal{T}}$ is compressed by the sector intersection algorithm using \ped to two line segments.}
		\vspace{-1ex}
		\label{fig:sleeve}
	\end{figure}
}%%%%%%%%
