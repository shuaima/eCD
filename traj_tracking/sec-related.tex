
%%% Local Variables:
%%% mode: latex
%%% TeX-master: "gis18"
%%% End:
\section{related work}
\label{sec-related}

In this section, we summarize position tracking and trajectory tracking methods. For trajectory simplification, please refer to \cite{Zhang:Evaluation, Lin:Cised} for more details.

\subsection{\textcolor{blue}{Position Tracking}}
In the paper [6], the authors discussed different location tracking protocols and compared the effectiveness and efficiency of these protocols. Different location tracking algorithms are based on different metrics, for example, these algorithms can be divided into distance-based algorithms and road-based algorithms.
The representative algorithm in distance-based position tracking algorithms is LDR. In the paper [9], LDR algorithm is used to model the management of moving objects. The advantages of the LDR algorithm are simple implementation and fast processing speed. But the result of this kind of algorithm is not very good, it needs to spend a lot of network bandwidth. 
In the papers [13, 15], the authors proposed different adaptive dead reckoning algorithms. These adaptive dead reckoning algorithms improve the performance of plain dead reckoning algorithms, but there are still some efficiency problems.
Another type of algorithm is the road-based algorithm. The update policy of road-based algorithm is different from the distance-based algorithm. It assumes that the movement of the moving object is related to the road on the map and uses this to track the location. For example, in the paper [14], the authors proposed a deviation location update policy. The policy predicts that after the location is updated, the moving object will continue to move on the same street. And when the position of the moving object exceeds a given threshold from the predicted position, it will also be updated. Therefore, the algorithm ensures that the position error of the moving object is bounded. They found that when the threshold is 0.05 miles, the performance is 43% better than the distance policy. Their algorithm assumes that map matching is always valid. If this is not the case, the algorithm will fail. In the paper [2], the authors modified the road network and introduced acceleration on the basis of the predecessors to improve the performance of the road-based algorithm.
\subsection{\textcolor{blue}{trajectory tracking}}
In the paper [11], the authors proposed the LDRH algorithm. This algorithm modifies the LDR algorithm and makes it applicable to trajectory tracking. The LDRH algorithm is a one-pass algorithm with good running speed, but the compression effect is not ideal.
Another representative trajectory tracking algorithm is GRTS[4, 5]. GRTS has good compression ratio, and can be used in combination with different compression algorithms to directly balance the computational cost and compression ratio. But the GRTS algorithm has certain disadvantages: 1. GRTS algorithm requires a buffer, which brings storage limitations. 2. GRTS separates compression and position tracking. 

why not integrate one-pass algorithms with GRTS? 1. One pass algorithm does not need a buffer, but GRTS requires a buffer. 2. A one-pass algorithm runs in a manner similar with LDR, which inspires us to develop effective and efficient trajectory tracking methods that combine compression and position tracking into a consistent algorithm, and run in one-pass manner.



