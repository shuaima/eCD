
%%% Local Variables:
%%% mode: latex
% TeX-master: "gis18"
%%% End:


\documentclass[sigconf,nonacm]{acmart}
\special{papersize=8.5in,11in}

% \usepackage{polyglossia}
\usepackage{booktabs} % For formal tables


\usepackage{graphicx}
\usepackage{enumerate}
\usepackage{amsfonts}
\usepackage{amsmath}
\usepackage{amssymb}
\usepackage{color}
\usepackage{colortbl}
\usepackage{epsfig}
\usepackage{xspace}
\usepackage{esvect} % for arrows
\usepackage{subcaption}
% \usepackage{subfigure}
\usepackage{balance}
% \usepackage{cite}
\usepackage[english]{babel}

% algorithms
\usepackage{algorithm}
\usepackage[noend]{algpseudocode}

\DeclareMathOperator*{\argmax}{argmax} % no space, limits underneath in displays

%%%%%%%%%%%%%%%%%%%%%%%%%%%%%%%%%%%%%
%% DO NOT DELETE!!
%%%%%%%%%%%%%%%%%%%%%%%%%%%%%%%%%%%%%
%\usepackage{tikz}
%\usetikzlibrary{trees}

\usepackage{multirow}
\usepackage{url}

\newcommand{\imp}{\vdash_{\cal I}}


%%%%%%%%%%%%%%%%%%%%%%%%%%%%%%%%%%%%%%%%%%
% Enumerate and Itemize modifications
%\usepackage{enumitem}
%\setlist{topsep=0pt,noitemsep} \setitemize[1]{label=$\circ$}
%%%%%%%%%%%%%%%%%%%%%%%%%%%%%%%%%%%%%%%%%%%

\sloppy
\newcommand{\rtable}[1]{\ensuremath{\mathsf{#1}}}
\newcommand{\ratt}[1]{\ensuremath{\mathit{#1}}}
\newcommand{\at}[1]{\protect\ensuremath{\mathsf{#1}}\xspace}
\newcommand{\myhrule}{\rule[.5pt]{\hsize}{.5pt}}
\newcommand{\oneurl}[1]{\texttt{#1}}
\newcommand{\eat}[1]{}
\newcommand{\stab}{\rule{0pt}{8pt}\\[-1.6ex]}
\newcommand{\sttab}{\rule{0pt}{8pt}\\[-2ex]}
%\newcommand{\sstab}{\rule{0pt}{8pt}\\[-2.4ex]}
\newcommand{\tabstrut}{\rule{0pt}{4pt}\vspace{-0.07in}}
\newcommand{\vs}{\vspace{1ex}}
\newcommand{\exa}[2]{{\tt\begin{tabbing}\hspace{#1}\=\+\kill #2\end{tabbing}}}
\newcommand{\ra}{\rightarrow}
\newcommand{\la}{\leftarrow}
\newcommand{\bi}{\begin{itemize}}
\newcommand{\ei}{\end{itemize}}
\newenvironment{tbi}{\begin{itemize}
        \setlength{\topsep}{1.5ex}\setlength{\itemsep}{0ex}\vspace{-0.5ex}}
        {\end{itemize}\vspace{-0.5ex}}
\newenvironment{tbe}{\begin{enumerate}
        \setlength{\topsep}{0ex}\setlength{\itemsep}{-0.7ex}\vspace{-1ex}}
        {\end{itemize}\vspace{-1ex}}

\newcommand{\mat}[2]{{\begin{tabbing}\hspace{#1}\=\+\kill #2\end{tabbing}}}
\newcommand{\m}{\hspace{0.05in}}
\newcommand{\ls}{\hspace{0.1in}}
\newcommand{\be}{\begin{enumerate}}
\newcommand{\ee}{\end{enumerate}}
\newcommand{\beqn}{\begin{eqnarray*}}
\newcommand{\eeqn}{\end{eqnarray*}}
\newcommand{\card}[1]{\mid\! #1\!\mid}
\newcommand{\fth}{\hfill $\Box$}
\newcommand{\AND}{\displaystyle{\bigwedge_{i=1}^{n}}}
%\newcommand{\U}[1]{\displaystyle{\bigcup_{#1}}}
\newcommand{\Sm}[1]{\displaystyle{\sum_{#1}}}
\newcommand{\stitle}[1]{\vspace{1ex}\noindent{\bf #1}}
\newcommand{\etitle}[1]{\vspace{0.5ex}\noindent{\em \underline{#1}}}
\renewcommand{\t}{\tau}
\newcommand{\Inh}[1]{\$#1}
\renewcommand{\r}[1]{{\it rule}(#1)}
\newcommand{\pa}{\parallel}
\newcommand{\LHS}{\kw{{\small LHS}}}
\newcommand{\RHS}{\kw{RHS}}
\newcommand{\len}{\kw{len}}
\newcommand{\kop}{\kw{op}}
%\newcommand{\st}{\emph{s.t.}\xspace}
\newcommand{\ie}{\emph{i.e.,}\xspace}
\newcommand{\eg}{\emph{e.g.,}\xspace}
\newcommand{\wrt}{\emph{w.r.t.}\xspace}
\newcommand{\aka}{\emph{a.k.a.}\xspace}
\newcommand{\kwlog}{\emph{w.l.o.g.}\xspace}

\newcommand{\VNM}{\kw{VNM}}
\newcommand{\VNMs}{\kw{VNM}}
\newcommand{\VN}{\kw{VN}}
\newcommand{\SN}{\kw{SN}}

%%%%%%%%%%%%%%%%%%%%%%%%%%%%%%%%%%%%%%%%%%%%%%%%%%%%%%%%%%%%%%%%
%                  Relation Algebra operators
%%%%%%%%%%%%%%%%%%%%%%%%%%%%%%%%%%%%%%%%%%%%%%%%%%%%%%%%%%%%%%%%

\newcommand{\RS}{{\small S}\xspace}
\newcommand{\RP}{{\small P}\xspace}
\newcommand{\RJ}{{\sc j}\xspace}
\newcommand{\RC}{{\small C}\xspace}
\newcommand{\RSJ}{{\small SJ}\xspace}
\newcommand{\RSC}{{\small SC}\xspace}
\newcommand{\RSP}{{\small SP}\xspace}
\newcommand{\RPJ}{{\small PJ}\xspace}
\newcommand{\RPC}{{\small PC}\xspace}
\newcommand{\RSPJ}{{\sc spj}\xspace}
\newcommand{\RSPC}{{\small SPC}\xspace}
\newcommand{\RSPJU}{{\sc spju}\xspace}
\newcommand{\RSPCU}{{\small SPCU}\xspace}
\newcommand{\RSPJUN}{{\small SPJU$^N$}\xspace}
\newcommand{\RSPCUN}{{\small SPCU$^N$}\xspace}
%%%%%%%%%%%%%%%%%%%%%%%%%%%%%%%%%%%%%%%%%%%%%%%%%%%%%%%%%%%%%%%%%%%%%%%%%%%%%%
% ALGORITHMS
%%%%%%%%%%%%%%%%%%%%%%%%%%%%%%%%%%%%%%%%%%%%%%%%%%%%%%%%%%%%%%%%%%%%%%%%%%%%%%%

\newcommand{\kw}[1]{{\ensuremath {\mathsf{#1}}}\xspace}

\newcounter{ccc}
\newcommand{\bcc}{\setcounter{ccc}{1}\theccc.}
\newcommand{\icc}{\addtocounter{ccc}{1}\theccc.}
\newcommand{\checking}{{\mbox{\small\sf Checking}\xspace}}
\newcommand{\preProcessing}{{\mbox{\small\sf preProcessing}\xspace}}
\newcommand{\CFDconsistency}{{\mbox{\small\sf CFD\_Checking}\xspace}}
\newcommand{\MCS} {\kw{MCS}}
\newcommand{\templateDB}{{\mbox{\small\sf templateDB}\xspace}}
\newcommand{\ChaseChecking}{{\mbox{\small\sf RandomChecking}\xspace}}
\newcommand{\chase}{{\mbox{\small\sf Chase}\xspace}}
\newcommand{\SAT}{{\mbox{\small\sf SAT}\xspace}}
\newcommand{\kSAT}{{\mbox{\small 3SAT}\xspace}}
\newcommand{\PropCFDSPC}{\kw{Prop{\small CFD\_SPC}}}
\newcommand{\PropCFDSPCU}{\kw{Prop{\small CFD\_SPCU}}}
\newcommand{\UnionEQs}{\kw{UnionEQs}}
\newcommand{\UnionCFDs}{\kw{UnionCFDs}}
\newcommand{\EQ}{\kw{EQ}}
\newcommand{\eq}{\kw{eq}}
\newcommand{\key}{\kw{key}}
\newcommand{\rep}{\kw{rep}}
\newcommand{\PEQ}{\kw{EQ2CFD}}
\newcommand{\Drop}{\kw{Drop}}
%\newcommand{\Res}{\kw{Res}}
\newcommand{\CFD}{{\small CFD}\xspace}
\newcommand{\CFDs}{{\small CFD}{\small s}\xspace}
\newcommand{\CIND}{{\sc cind}\xspace}
\newcommand{\cind}{{\small \sf CIND}}
\newcommand{\cfd}{{\small \sf CFD}}
\newcommand{\CINDp}{{\sc cind}$^+$\xspace}
\newcommand{\CINDn}{{\sc cind}$^-$\xspace}
\newcommand{\CINDs}{{\sc cind}{\small s}\xspace}
\newcommand{\FD}{{\small FD}\xspace}
\newcommand{\FDs}{{\small FD}{\small s}\xspace}
\newcommand{\IND}{{\sc ind}\xspace}
\newcommand{\INDs}{{\sc ind}{\small s}\xspace}
\newcommand{\TGDs}{{\sc tgd}{\small s}\xspace}
\newcommand{\NP}{{\small NP}\xspace}
\newcommand{\DTIME}{{\small DTIME}\xspace}
\newcommand{\NPO}{{\small NPO}\xspace}
\newcommand{\APX}{{\small APX}\xspace}
\newcommand{\DAGs}{{\sc dag}s\xspace}
\newcommand{\NC}{{\sc nc}\xspace}
\newcommand{\coNP}{co{\small NP}\xspace}
\newcommand{\PTIME}{{\small PTIME}\xspace}
\newcommand{\PSPACE}{{\sc pspace}\xspace}
\newcommand{\EXPTIME}{{\sc exptime}\xspace}
\newcommand{\NPSPACE}{{\sc npspace}\xspace}
\newcommand{\dom}{\protect\ensuremath{\mathsf{dom}}\xspace}
\newcommand{\atset}{\protect\ensuremath{\mathsf{attr}}\xspace}
\newcommand{\attr}[1]{\protect\ensuremath{\mathsf{#1}}\xspace}
\newcommand{\attrset}{\protect\ensuremath{\mathsf{attr}}\xspace}
\newcommand{\finatset}{\protect\ensuremath{\mathsf{finattr}}\xspace}
\newcommand{\pvar}{\protect\ensuremath{\mathsf{var\%}}\xspace}
\newcommand{\lLHS}{\protect\ensuremath{\mathsf{{\small LHS}}}\xspace}
\newcommand{\RA}{{\small RA}\xspace}
\newcommand{\RBR}{\kw{RBR}}
\newcommand{\SQL}{{\sc sql}\xspace}
\newcommand{\XSLT}{{\sc xslt}\xspace}
\newcommand{\DBMS}{{\sc dbms}\xspace}
\newcommand{\ATG}{{\sc atg}\xspace}
\newcommand{\ATGs}{{\sc atg}{\small s}\xspace}
\newcommand{\EBI}{{\sc ebi}\xspace}
\newcommand{\GO}{{\sc go}\xspace}
\newcommand{\VEC}[1]{{\sc vec}(#1)}
\newcommand{\DAG}{{\sc dag}\xspace}
\newcommand{\XQ}{{\sc xq}\xspace}
\newcommand{\XQwc}{{\sc xq}$^{\scriptscriptstyle[*]}$\xspace}
\newcommand{\XQdes}{{\sc xq}$^{\scriptscriptstyle[//]}$\xspace}
\newcommand{\XQfull}{{\sc xq}$^{\scriptscriptstyle[*,//]}$\xspace}
\newcommand{\vect}[1]{$\langle$ #1 $\rangle$}
\newcommand{\sem}[1]{[\![#1]\!]}
\newcommand{\NN}[2]{#1\sem{#2}}
\newcommand{\e}[2]{{\mathit (#1,#2)}}
\newcommand{\ep}[2]{{\mathit (#1,#2)+}}
\newcommand{\brname}{\ensuremath{{\mathsf{N}}}}
\newcommand{\budrel}[1]{\ensuremath{{\brname_{#1}}}}
\newcommand{\budgen}[2]{\ensuremath{Q^\brname_\e{#1}{#2}}}
\newcommand{\budcut}[2]{\ensuremath{Q_\e{#1}{#2}}}
\newcommand{\eop}{\hspace*{\fill}\mbox{$\Box$}}     % End of proof
\newcounter{example}%[section]
%\newcommand{\theexample}{\arabic{example}}
\newenvironment{example}{
         \vspace{1.5ex}
         \refstepcounter{example}
         {\noindent\bf Example \theexample:}}{
         \eop\vspace{1.5ex}}
\def\copyrightspace{}
\renewcommand{\ni}{\noindent}
\newcommand{\comlore}[1]{\begin{minipage}{3in}\fbox{\fbox{\parbox[t]{3in}{{\vspace{2mm}\noindent \bf COMM(LORE):~
{ #1}\hfill  END.}}}}\end{minipage}\\}
\newcommand{\comwenfei}[1]{\begin{minipage}{3in}\fbox{\fbox{\parbox[t]{3in}{{\vspace{2mm}\noindent \bf COMM(WENFEI):~
{ #1}\hfill  END.}}}}\end{minipage}\\}
\newcommand{\comshuai}[1]{\begin{minipage}{3in}\fbox{\fbox{\parbox[t]{3in}{{\vspace{2mm}\noindent \bf COMM(SHUAI):~
{ #1}\hfill  END.}}}}\end{minipage}\\}
\newcommand{\nthesection}{\arabic{section}}
%\newcounter{problem}
%\newenvironment{problem}{\begin{em}
%        \refstepcounter{problem}
%        {\vspace{1.5ex} \noindent\bf Problem \theproblem:}}{
%        \end{em}\eop\vspace{1.5ex}}
\newcounter{prop}[section]
%\renewcommand{\theprop}{\arabic{theorem}}
%\newcounter{lemma}[section]
%\renewcommand{\thelemma}{\arabic{theorem}}
%\newcounter{cor}[section]
%\renewcommand{\thecor}{\arabic{theorem}}
\newenvironment{ttheorem}{\begin{em}
         \refstepcounter{theorem}
         {\vspace{1.5ex} \noindent\bf  Theorem  \thetheorem:}}{
        \end{em}\eop\vspace{1.5ex}} %\hspace*{\fill}\vspace*{1ex}}
\newenvironment{pprop}{\begin{em}
        \refstepcounter{theorem}
        {\vspace{1.5ex}\noindent \bf Proposition \thetheorem:}}{
        \end{em}\eop\vspace{1.5ex}}%\hspace*{\fill}\vspace*{1ex}}
\newenvironment{llemma}{\begin{em}
         \refstepcounter{theorem}
        {\vspace{1.5ex}\noindent\bf Lemma \thetheorem:}}{
         \end{em}\eop\vspace{1.5ex}} %\hspace*{\fill}\vspace*{1ex}}
\newenvironment{cor}{\begin{em}
        \refstepcounter{theorem}
        {\vspace{1.5ex}\noindent\bf Corollary \thetheorem:}}{
        \end{em}\eop\vspace{1.5ex}} %\hspace*{\fill}\vspace*{1ex}}

%\newcounter{definition}
%\renewcommand{\thedefinition}{\arabic{definition}}
%\newenvironment{definition}{
%        \vspace{1.5ex}
%        \refstepcounter{definition}
%        {\noindent\bf Definition {\bf \thedefinition}:}}{\eop\vspace{1.5ex}
%}
\newcounter{alg}[section]
\renewcommand{\thealg}{\nthesection.\arabic{alg}}
\newenvironment{alg}[1]{
        \refstepcounter{alg}
        {\vspace{1ex}\noindent\bf Algorithm \thealg:\, #1}}{
        \vspace*{1ex}}
\newcounter{arule}
\renewcommand{\thearule}{\arabic{arule}}
\newenvironment{arule}{
        \vspace{0.6ex}
        \refstepcounter{arule}
        {\noindent \em Rule \thearule:}}{
        }
\newcounter{claim}
\renewcommand{\theclaim}{\arabic{claim}}
\newenvironment{claim}{
        \vspace{0.6ex}
        \refstepcounter{claim}
        {\noindent\em Claim \theclaim:}}{%--{ Wenfei Fan}\\
        }
\renewenvironment{proof}{
%\newenvironment{proof}{
        \vspace{0ex}
        {\noindent\bf Proof:}}{\eop\vspace{1ex}}
\newenvironment{proofS}{
        \vspace{1ex}
        {\noindent\bf Proof sketch:\ }}{\eop\vspace{1ex}}

\newcommand{\dist}{\kw{ldist}}
\newcommand{\pSim}{\kw{JoinMatch}}
\newcommand{\spSim}{\kw{SplitMatch}}
\newcommand{\gpq}{\kw{PQ}}
\newcommand{\gpqs}{\kw{PQs}}
\newcommand{\rrq}{\kw{RQ}}
\newcommand{\rrqs}{\kw{RQs}}
\newcommand{\rpe}{\kw{RPE}}
\newcommand{\rpes}{\kw{RPEs}}

\newcommand{\eps}{\trianglelefteq}
\newcommand{\neps}{\ntrianglelefteq}
\newcommand{\ees}{\preceq_{(e,e)}}
\newcommand{\nees}{\not\preceq_{e,e}}
\newcommand{\Reps}{S}

\newcommand{\added}[1]{\textcolor{blue}{#1}}
\newcommand{\changed}[1]{\textcolor{red}{#1}}
\newcommand{\removed}[1]{\textcolor{gray}{#1}}

\newcommand{\ret}{\kw{ret}}
\newcommand{\remv}{\kw{premv}}
\newcommand{\presim}{\kw{amat}}
\newcommand{\prev}{\kw{prev}}
\newcommand{\subiso}{\kw{SubIso}}

\newcommand{\ssim}{\kw{mat}}
\newcommand{\join}{\kw{Join}}
\newcommand{\nor}{\kw{Normalize}}
\renewcommand{\split}{\kw{Split}}
\newcommand{\sccg}{\kw{Sccgraph}}
\newcommand{\rmv}{\kw{rmv}}
\newcommand{\block}{{\cal B}}
\newcommand{\rel}{\kw{rel}}
\newcommand{\partition}{\kw{par}}
\newcommand{\cpath}{{\em c}-path\xspace}
\newcommand{\cpaths}{{\em c}-paths\xspace}
\newcommand{\psimset}{\kw{Psim}}



\newcommand{\vn}{\kw{VN}}
\newcommand{\vns}{\kw{VNs}}
\newcommand{\sns}{\kw{SNs}}
\newcommand{\vm}{\kw{VM}}
\newcommand{\vms}{\kw{VMs}}
\newcommand{\vmp}{\kw{VMP}}
\newcommand{\sn}{\kw{SN}}
\newcommand{\vne}{\kw{VNE}}

\newcommand{\buildAug}{\kw{compAuxGraph}}
\newcommand{\minVN}{\kw{minVN}}
\newcommand{\compMap}{\kw{compVNM}}
\newcommand{\compMapNS}{\kw{compVNM_{NS}}}
\newcommand{\PTAS}{{\small PTAS}\xspace}
\newcommand{\APTAS}{{\small APTAS}\xspace}
\newcommand{\VM}{\kw{VM}}
\newcommand{\vine}{\kw{ViNE}}
\newcommand{\vineNS}{\kw{ViNE_{NS}}}
\newcommand{\rwsp}{\kw{RW}-\kw{SP}}
\newcommand{\lvb}{\{\!|}
\newcommand{\rvb}{|\!\}}
%% APPENDIX

\newcommand{\gap}{\kw{GAP}}
\newcommand{\rgap}{\kw{RGAP}}
\newcommand{\subgIso}{\kw{Subgraph} \kw{Isomorphism}}
\newcommand{\xtc}{\kw{X3C}}
\newcommand{\binpack}{\kw{Bin} \kw{Packing}}
\newcommand{\parti}{\kw{PARTITION}}
\newcommand{\mwsat}{\kw{Minimum} \kw{Weight} \kw{3SAT}}
\newcommand{\edp}{\kw{EDP}}
\newcommand{\att}{\SIM}
\newcommand{\swsf}{\kw{SWSF\_FP}}

\newcommand{\warn}[1]{\textcolor{red}{#1}}
\newcommand{\revise}[1]{\textcolor{blue}{#1}}
\newcommand{\marked}[1]{\revise{#1}}


%%%%%%%%%%%%%%%%%%%%%%%%%%%%%%%Data sets%%%%%%%%%%%%%%%%%
\newcommand{\taxi}{\kw{Taxi}}
\newcommand{\sercar}{\kw{ServiceCar}}
\newcommand{\pricar}{\kw{PrivateCar}}
\newcommand{\geolife}{\kw{GeoLife}}
\newcommand{\mopsi}{\kw{Mopsi}}
\newcommand{\didi}{\kw{Didi}}
\newcommand{\pubdata}{\kw{Public Data}}

%%%%%%%%%%%%%%%%%%%%%%%%%%%%%%% algorithms %%%%%%%%%%%%%%%%%
\newcommand{\ped}{\kw{PED}} %perpendicular Euclidean distance (PED).
\newcommand{\sed}{\kw{SED}} %synchronous Euclidean distance (SED).
\newcommand{\red}{\kw{RED}} %radial Euclidean distance (RED).
%\newcommand{\dad}{\kw{DAD}} %Direction-Aware Distance (DAD).
\newcommand{\bed}{\kw{BED}} %Binary Euclidean distance (BED).


\newcommand{\sector}[1]{{$\mathcal{S}{#1}$}}
\newcommand{\cone}[1]{{$\mathcal{C}{#1}$}}
\renewcommand{\circle}[1]{{$\mathcal{O}{#1}$}}
\newcommand{\pcircle}[1]{{$\mathcal{O}^c{#1}$}}

\newcommand{\cised}{\kw{CISED}}
\newcommand{\siped}{\kw{SIPED}}
\newcommand{\citt}{\kw{CITT}}
\newcommand{\citts}{\kw{CITT}-\kw{S}}
\newcommand{\cittsh}{\kw{CITT}-\kw{SH}}
\newcommand{\cittsf}{\kw{CITT}-\kw{SF}}
\newcommand{\cittw}{\kw{CITT}-\kw{W}}
\newcommand{\sitt}{\kw{SITT}}
\newcommand{\bitt}{\kw{BITT}}

\newcommand{\ldr}{\kw{LDR}}
\newcommand{\ldrh}{\kw{LDRH}}
\newcommand{\grts}{\kw{GRTS}}


\newcommand{\trajec}[1]{$\dddot{\mathcal{#1}}$}
\newcommand{\ffunc}[1]{{\mathbb{#1}}}
\newcommand{\sstab}{\vspace{0.5ex}\noindent}

\newcommand{\myfig}[1]{\textcolor{blue}{Figure~\ref{#1}}}
\newcommand{\todo}[1]{\textcolor{red}{Todo...#1}}
\newcommand{\myred}[1]{\textcolor{red}{#1}}
\newcommand{\myblue}[1]{\textcolor{blue}{#1}}



%%%%%%%%%%%%%%%%%%%vldb commands
\newcommand\vldbdoi{XX.XX/XXX.XX}
\newcommand\vldbpages{XXX-XXX}
% issue-specific
\newcommand\vldbvolume{14}
\newcommand\vldbissue{1}
\newcommand\vldbyear{2020}
% should be fine as it is
\newcommand\vldbauthors{\authors}
\newcommand\vldbtitle{\shorttitle}
% leave empty if no availability url should be set
\newcommand\vldbavailabilityurl{http://vldb.org/pvldb/format_vol14.html}
% whether page numbers should be shown or not, use 'plain' for review versions, 'empty' for camera ready
\newcommand\vldbpagestyle{plain}
%%%%%%%%%%%%%%%%%%%%%%vldb end

\begin{document}

%\title{One-pass Tracking Moving Objects in Circular, Strip and Combined Areas}
%\title{Effectively/Efficiently Tracking Moving Objects in Customized Regions}
%\title{One-pass Trajectory Tracking in Circular, Strip and Combined Areas}
%\title{One-pass Trajectory Tracking in Circular and Strip Areas}
%\title{One-pass Trajectory Tracking in Circular, Strip and Rectangle-like Areas}
%\title{Effectively and Efficiently Tracking Moving Objects in Circular and Rectangle-like Areas}
\title{One-pass Trajectory Tracking in Discs and Beams}

%\titlenote{Produces the permission block, and copyright information}
% \subtitle{Extended Abstract}
%\subtitlenote{The full version of the author's guide is available as  \texttt{acmart.pdf} document}

\eat{%%%%%%%%%%%%%%%%for anony
 \author{Xuelian Lin, Yihao Fu, Yanchen Hou and Shuai Ma$^*$}
 \affiliation{%
   \institution{State Key Laboratory of Software Development Environment, Beihang University}
   \streetaddress{37th XueYuan Road}
   \city{Beijing}
   \country{China}
   \postcode{100191}
 }
 \email{{linxl, fuyh, houyc, mashuai}@buaa.edu.cn}
}%%%%%%%%%%%%%End eat


\author{Anonymous Author(s)}
\affiliation{%
  \institution{Institution}
  \streetaddress{Street address}
  \city{City}
  \country{Country}
  \postcode{postcode}
}
\email{emails}



\pagestyle{empty} % removes running headers

% The default list of authors is too long for headers.
%\renewcommand{\shortauthors}{J. Jiang et al.}
%\renewcommand{\shortauthors}{XXX et al.}


\begin{abstract}
Trajectory tracking is a method that tracks the current position of a moving object and meanwhile simplifies its trajectory. It is a combination of two fundamental techniques of moving objects databases, position tracking and trajectory simplification, in one routine such that only a small piece of position information is sent to and saved in the databases, and thus the network, storage and computing resources are saved.
%
There are some distinct trajectory tracking algorithms, such as \ldrh and \grts, have been developed. However, they still suffer in performance of effectiveness or efficiency, and more important, they only track a moving object in a circular area, unable to satisfy the varied requirements of trajectory tracking in areas beyond a circle. 
%
To solve these problems, this paper presents three novel one-pass trajectory tracking algorithms that effectively and efficiently track a moving object in a disc, infinite beam and finite beam, respectively, based on the techniques of sector intersection and spatio-temporal cone intersection.
%
Using three real-life trajectory datasets, we experimentally show that our approaches are both efficient and effective that outperform \ldrh and \grts, and are feasible to track a moving object in such an area.
\end{abstract}



%
% The code below should be generated by the tool at
% http://dl.acm.org/ccs.cfm
% Please copy and paste the code instead of the example below.
%
% \begin{CCSXML}
% <ccs2012>
%  <concept>
%   <concept_id>10010520.10010553.10010562</concept_id>
%   <concept_desc>Computer systems organization~Embedded systems</concept_desc>
%   <concept_significance>500</concept_significance>
%  </concept>
%  <concept>
%   <concept_id>10010520.10010575.10010755</concept_id>
%   <concept_desc>Computer systems organization~Redundancy</concept_desc>
%   <concept_significance>300</concept_significance>
%  </concept>
%  <concept>
%   <concept_id>10010520.10010553.10010554</concept_id>
%   <concept_desc>Computer systems organization~Robotics</concept_desc>
%   <concept_significance>100</concept_significance>
%  </concept>
%  <concept>
%   <concept_id>10003033.10003083.10003095</concept_id>
%   <concept_desc>Networks~Network reliability</concept_desc>
%   <concept_significance>100</concept_significance>
%  </concept>
% </ccs2012>
% \end{CCSXML}

% \ccsdesc[500]{Computer systems organization~Embedded systems}
% \ccsdesc[300]{Computer systems organization~Redundancy}
% \ccsdesc{Computer systems organization~Robotics}
% \ccsdesc[100]{Networks~Network reliability}


% \keywords{Map matching, trajectory compression, HMM}

\maketitle

%%% do not modify the following VLDB block %%
%%% VLDB block start %%%
\pagestyle{\vldbpagestyle}
\begingroup\small\noindent\raggedright\textbf{PVLDB Reference Format:}\\
\vldbauthors. \vldbtitle. PVLDB, \vldbvolume(\vldbissue): \vldbpages, \vldbyear.\\
\href{https://doi.org/\vldbdoi}{doi:\vldbdoi}
\endgroup
\begingroup
\renewcommand\thefootnote{}\footnote{\noindent
	This work is licensed under the Creative Commons BY-NC-ND 4.0 International License. Visit \url{https://creativecommons.org/licenses/by-nc-nd/4.0/} to view a copy of this license. For any use beyond those covered by this license, obtain permission by emailing \href{mailto:info@vldb.org}{info@vldb.org}. Copyright is held by the owner/author(s). Publication rights licensed to the VLDB Endowment. \\
	\raggedright Proceedings of the VLDB Endowment, Vol. \vldbvolume, No. \vldbissue\ %
	ISSN 2150-8097. \\
	\href{https://doi.org/\vldbdoi}{doi:\vldbdoi} \\
}\addtocounter{footnote}{-1}\endgroup
%%% VLDB block end %%%

%%% do not modify the following VLDB block %%
%%% VLDB block start %%%
%\ifdefempty{\vldbavailabilityurl}{}{
%\vspace{.3cm}
%\begingroup\small\noindent\raggedright\textbf{PVLDB Artifact Availability:}\\
%The source code, data, and/or other artifacts have been made available at \url{\vldbavailabilityurl}.
%\endgroup
%}
%%% VLDB block end %%%




%%% Local Variables:
%%% mode: latex
%%% TeX-master: "gis18"
%%% End:

\section{introduction}
\label{sec-intro}


\textit{Trajectory tracking} \cite{Lange:Tracking} is a combination of \textit{position tracking} \cite{Wolfson:PositionTracking,Leonhardi:Comparison} and \textit{trajectory simplification} \cite{Lin:Cised,Zhang:Evaluation} in one routine, where \textit{position tracking} is an approach that lets the moving objects database (MOD) server know the current position of a moving object effectively and efficiently, that is, it achieves the desired accuracy of the location information on the server by transmitting as few messages as possible \cite{Leonhardi:Comparison}. Linear dead reckoning (\ldr) \cite{Wolfson:PositionTracking} is such a widely used position tracking method, which is essentially an agreement between a given moving object and a MOD server such that the server could infer the current, excepted position of the moving object whose distance to the actual position of the object is bounded by a user specified threshold;
%
and \textit{trajectory simplification} \cite{Lin:Cised,Zhang:Evaluation} is to approximate a fine trajectory with a coarse one (whose corresponding data points are a subset of the original one), such that the size of the trajectory is reduced under a constrain that the maximum distance of the former to the latter is bounded by a user specified threshold. 
%Linear simplification \cite{Lin:Cised,Zhang:Evaluation} is such an effective and efficient approach that is also widely used in practice.
%
Position tracking and trajectory simplification both are the fundamental technologies of trajectory management and they also share some common target and strategy, \ie, reduce the number of messages or the size of trajectory data by discarding some location information that seems not that important, hence, researchers are trying to combine them in one routine and make it be suitable to run in resource constraint devices.

The authors of \cite{Trajcevski:LDRH} find that the position tracking algorithm \ldr with some tiny modifications is applicable to both track the positions of a moving object and simplify the trajectory built out of these positions. The modified \ldr,  called \ldrh in \cite{Lange:Tracking}, is the first trajectory tracking algorithm that combines position tracking and trajectory simplification into one consistent process. It is concise and efficient, and is suitable for mobile devices. However, it suffers in effectiveness in terms of compression ratio and communication cost, due to the nature of \ldr. 
%
Then, a framework, named the generic remote trajectory simplification (GRTS) \cite{Lange:GRTS,Lange:Tracking}, is developed to improve the effectiveness of trajectory tracking by separate position tracking and trajectory simplification into two sub-processes, where the positions of a moving object is also tracked by \ldr, and these positions are temporarily saved in a buffer and then simplified by some third-party line simplification algorithm. Indeed, it is more effective than \ldrh at a cost of weakening the conciseness and efficiency of \ldrh.
%



\stitle{\todo{Motivations}.}

\ni(1) Trajectory track algorithms are supposed to run in resource-constraint mobile devices, thus, besides good performance of efficiency and effectiveness, they should also be simple and light, \ie having low time and space complexities, otherwise, they are not suitable to run in those mobile devices. In response to these requirements, \ldrh is light, simple and efficient, but not effective; and \grts is effective, but not efficient and light enough. That is, neither of them is the ideal solution for trajectory tracking.
%The emerging of one pass trajectory simplification algorithms. These algorithms can be integrated into grts, however, it is not a natural way to implement a one-pass trajectory tracking algorithm like this way. Acutually, one pass position tracking + one pass trajectory simplification = one pass and effective trajectory tracking algorithm......co-design, like LDRH, yet more effective.


\ni(2) The current works, \ie~\ldrh and \grts, only compress a trajectory or track a moving object in circular areas, \ie the moving object is supposed to locate in a circular taking the expected position of the object as the center. However, in practical, there is a need to track moving objects in other areas, such as strip or rectangular-like areas. \todo{examples and figures of areas,}





\stitle{\todo{Contributions}.}
To the end, we design ways for trajectory tracking in varied areas, including strip and combined areas, and provide three novel one-pass algorithms tracking moving objects effectively and efficiently. 

1. one-pass tracking moving object in circular, citt, effectively and efficiently.

2. one-pass tracking in strips using ped. sitt.
a way that customize region by sed and ped. and implement it in position tracking LDR and trajectory tracking framework GRTS. advantage...

3. one-pass tracking in combined areas using sed and ped. bitt.  
A one-pass trajectory tracking algorithm supporting sed and ped, by a combination cone intersection and sector intersection, \ie co-design of position tracking and trajectory simplification, effective and low time and space complexity, suitable running in resource constraint devices.

4. experiments

\stitle{{Organization}}.
The remainder of the paper is organized as follows:
Section \ref{sec-pre} introduces the basic concepts and the basic HMM method,
Section \ref{sec-method} presents our trajectory simplification aware map-matching method,
Section \ref{sec-exp} reports the experimental results of these methods, followed by related works in Section \ref{sec-related} and conclusion in Section \ref{sec-conclusion}.





%%% Local Variables:
%%% mode: latex
%%% TeX-master: "gis18"
%%% End:



\section{Preliminaries}
\label{sec-pre}




In this section, we first introduce the concepts on simplified trajectories and map-matching, then we introduce the basic HMM-based map-matching method that serves as the fundamental of the work.
%, followed by statement the problem of map-matching on simplified trajectories.

\subsection{Notations}


\stitle{Points ($P$)}. A GPS point is defined as a triple $P(x, y, t)$,
which represents that a moving object is located at {\em longitude} $x$ and {\em
  latitude} $y$ at {\em time} $t$.

\stitle{Trajectories ($\dddot{\mathcal{T}}$)}. A trajectory
$\dddot{\mathcal{T}}[P_0, \ldots, P_n]$ is a sequence of data points in a
monotonically increasing order of their associated time values ($P_i.t <
P_j.t$ for any $0\le i<j\le n$). Intuitively, a trajectory is the path (or
track) that a moving object follows through space as a function of time~\cite{physics-trajectory}.


\eat{
\stitle{Simplified line segments ($\mathcal{L}$)}. A Simplified line segment (or
line segment for simplicity) $\mathcal{L}$ is  defined as $\vv{P_{s}P_{e}}$,
which represents the  closed line segment that connects the start point $P_s$ and the end point $P_e$.
There are also two attributes $\mathcal{L}.L_p$ and $\mathcal{L}.L_n$
representing the length of raw trajectory on each side of the simplified line
segment respectively.
}

\stitle{Simplified trajectories ($\overline{\mathcal{T}}$)}. A simplified trajectory $\overline{\mathcal{T}}[\mathcal{L}_0, \ldots , \mathcal{L}_m]$ ($0< m \le n$) of a trajectory $\dddot{\mathcal{T}}[P_0, \ldots, P_n]$ is a sequence of continuous directed line segments $\mathcal{L}_{i}$ = $\vv{P_{s_i}P_{e_i}}$ ($i\in[0,m]$) of $\dddot{\mathcal{T}}$  such that $\mathcal{L}_{0}.P_{s_0} = P_0$, $\mathcal{L}_{m}.P_{e_m} = P_n$ and  $\mathcal{L}_{i}.P_{e_i}$ = $\mathcal{L}_{i+1}.P_{s_{i+1}}$ for all $i\in[0, m-1]$.
Note that (1) each directed line segment in $\overline{\mathcal{T}}$ essentially represents a continuous sequence of data points in $\dddot{\mathcal{T}}$, and
(2) the simplified trajectories are referred to {as} error bounded if for each point $P$ in \trajec{T}, there exist points $P_j$ and $P_{j+1}$ in $\overline{\mathcal{T}}$ such that the distance from $P$ to $\mathcal{L}(P_j,P_{j+1}))$ is less than $\epsilon$.
%error bounded by $\epsilon$ if

\eat{
\stitle{Error bounded trajectory simplification}. Given a trajectory \trajec{T}, an error bound $\epsilon$ and a simplification algorithm $\mathcal{A}$ that produces another trajectory \trajec{T'},
we say that algorithm $\mathcal{A}$ is error bounded by $\epsilon$ if  for each point $P$ in \trajec{T}, there exist points $P_j$ and $P_{j+1}$ in \trajec{T'} such that the distance from $P$ to $\mathcal{L}(P_j,P_{j+1}))$ is less than $\epsilon$.
}



%\subsection{Terms on map-matching}



\stitle{Road segments ($r$)}. A road segment is defined as $r = (v_s,v_e)$, representing an edge directly connecting two ending
points in the map.



\eat{
\stitle{Candidate Road Sets ($C$)}. A candidate road set (candidate set in short) $C_i = \{r_i^1,r_i^2,\ldots,r_i^k\}$ of a GPS point $P_i$
is a set of road segments that are close to the point. The final
matched road segment is selected from the candidate set.
%In this paper, we set the search range as a circle centered at point $P_i$ with radius as 200 meters.
}

\stitle{Routes ($R$)}. A route $R = {[r_0, \ldots,r_m]}$ is a continuous sequence
of road segment such that $r_i.v_e = r_{i+1}.v_s$, $0\le i<m$.

\stitle{Road network ($G$)}. A road network is a directed graph $G(V,E)$ where $V$ is the set of junction points of roads and $E$
is the set of road segments between two junction points.

\stitle{Map-matching}. Given a (simplified) trajectory of a user and a road network, the goal of (trajectory simplification aware) map-matching is to find the most likely route in the road network that has been traveled by the user.




\subsection{HMM-based Map-matching}
In recent years, map-matching is always modeled as a sequence labeling problem and tackled using sequence models such as HMM.
The authors of \cite{Lamb1999Avoiding} first introduce HMM for map-matching, then a number of works \cite{Newson2009Hidden, Wang:eddy, Osogami:2013:IRL, yin:feature-based} follow this idea.
%
In the modeling of HMM-based map-matching, road segments are \emph{hidden states} and GPS points are \emph{observations}.
For example, in \myfig{fig:hmm-model-a}, GPS points $P_1,P_2,P_3$ are observations of the moving object at timestamps $T_1,T_2,T_3$, respectively,
and $r_1^1$ and $r_1^2$, two {candidate road segments} of point $P_1$, are the hidden states of the moving object at timestamp $T_1$.
Moreover, the likelihood of the GPS point residing in a road segment is described by \emph{emission probability} ($E$). For instance, in \myfig{fig:hmm-model-b}, the emission probability of point $P_1$ on road segment {$r_1^2$ is $E_1^2$}.

Then, the map-matching of a sub-trajectory to a road network is
modeled as a weighted directed graph (\myfig{fig:hmm-model-b}), where a vertex is a hidden state (candidate road segment), an edge is the transition from the previous hidden state to the next hidden state, and the weight of an edge, named \emph{transition probability} ($T$), is the probability that the moving object transitions from one road segment to another. For example, {$T_{2}^3$} is the transition probability from {$r_1^2$ to $r_2^3$}.

Finally, the probability of a sub-trajectory \trajec{T}$[P_s, \ldots, P_{s+u}]$ matched to a route $R$ is defined as the joint probability $J(\dddot{\mathcal{T}}, R) = \prod_{i=1}^u{T(r_{s+i}|r_{s+i-1})\cdot E(P_{s+i}|r_{s+i})}$, $P\in \dddot{\mathcal{T}}$ and $r\in R$, and a path in the graph with the highest joint probability is the matched route of the sub-trajectory.
Note that most HMM-based methods share the same model except that they have respective definitions of transition probabilities.
Our \stmm also follows this common model and has specific definition of
transition probability for simplified {trajectories}.

%\begin{equation}
%  \label{equ:joint-prob}
%  P(R,T) = \prod_{i=1}^n{P(r_i|r_{i-1})\cdot P(P_i|r_i)}
%\end{equation}


\begin{figure}[tb!]
  \centering
  \begin{subfigure}{0.4\textwidth}
    \includegraphics[width = \textwidth]{Figures/Fig-HMM-model-road.pdf}
    \caption{finding the candidate road segments.}\label{fig:hmm-model-a}
    \vspace{1ex}
  \end{subfigure}
  \begin{subfigure}{0.42\textwidth}
    \includegraphics[width = \textwidth, height = 0.6\textwidth]{Figures/Fig-HMM-model.pdf}
    \caption{finding the optimal route. }\label{fig:hmm-model-b}
  \end{subfigure}
  \vspace{-1ex}
  \caption{HMM-based map-matching.}
  \label{fig:hmm-model}
 \vspace{-4ex}
\end{figure}




%\subsection{Problem statement}
%Given a simplified trajectory and a road network($G(V,E)$), the goal of map-matching on simplified trajectories is to find the most likely route ($R$) in the road network that has been traveled by the user.




\section{Effective and efficient tracking in floating discs}
%\section{Tracking in a circular area}
\label{sec:circle}

{\ldrh is a one-pass trajectory tracking algorithm, having linear time and constant space complexities, and suffering in effectiveness (compression ratios).
Observing the recently trajectory simplification algorithm using \sed, named  \cised, is also one-pass (which is important for a trajectory tracking algorithm that is supposed to run on mobile devices), and at the same time, it has a good effectiveness (compression ratios), these inspire us to develop an efficient and effective trajectory tracking algorithm using \sed.}


%\subsection{Spatio-temporal cones and their usage in trajectory simplification}
\subsection{Spatio-temporal cones}
\cised uses a local synchronous distance checking approach based on a concept of \textit{spatio-temporal cone}, defined in a 3D Cartesian coordinate system whose $x$-axis, $y$-axis and $t$-axis are longitude, latitude and time, respectively, that converts the \sed distance tolerance into cones intersection for testing the successive points. This mechanism is potential to be extended for trajectory tracking with good effectiveness and efficiency.

\stitle{Spatio-temporal cone (\cone{}) \cite{Lin:Cised}}. 
Given a start point $P_s$ of sub-trajectory $\dddot{\mathcal{T}}_s[P_s, \ldots, P_{s+k}]$ and an error bound $\epsilon$, the spatio-temporal cone (or simply \textit{cone}) of a data point $P_{s+i}$ ($1\le i\le k$) in $\dddot{\mathcal{T}_s}$ \wrt $P_s$ and $\epsilon$, denoted as \cone{(P_s, P_{s+i}, \epsilon)}, or \cone{_{s+i}} in short, is an oblique circular cone such that point $P_s$ is its apex and the synchronous circle $\mathcal{O}(P_{s+i}, \epsilon)$ of point $P_{s+i}$, or \circle{_{s+i}} in short, a circle on the plane $P.t-P_{s+i}.t = 0$ such that $P_{s+i}$ is its center and $\epsilon$ is its radius, is its base (see Figure~\ref{fig:cis}).

\eat{%%%%%%%%%%%
	\begin{example}
		\label{exm-circles-cones}
		Figure~\ref{fig:cis} shows 
		(1) two synchronous circles, \circle{(P_{s+i}, \epsilon)} of point $P_{s+i}$ and \circle{(P_{s+k}, \epsilon)} of point $P_{s+k}$.
		It is easy to see that for any point in the area of a circle \circle{(P_{s+i}, \epsilon)}, its distance to $P_{s+i}$ is not greater than $\epsilon$, 
		and (2) two example spatio-temporal cones, \cone{(P_s, P_{s+i}, \epsilon)} {(purple)} and \cone{(P_s, P_{s+k}, \epsilon)} (red), with the same apex $P_s$ and error bound $\epsilon$. %\eop
	\end{example}
}%%%%%%%%%%%

%Note that in this definition, a \emph{synchronous circle} $\mathcal{O}(P_i, \epsilon)$ is only defined by a central point $P_i$ and a constant $\epsilon$. Indeed, it is nothing to do with any start point $P_s$ or end point $P_e$.


%\textcolor{blue}{We define \textit{synchronous circles and Spatio-temporal cones} in a \emph{x-y-t} 3D coordinate system, and build the connection between \textit{synchronous circles} and \textit{synchronous distances}.}




%, (4) $Q$ is a point in synchronous circle \circle{_{s+k}}, and (5) $P'_{s+i}$ is the intersection point of line $\protect\overline{P_sQ}$ and synchronous circle \circle{_{s+i}}

Based on the spatio-temporal cones, authors in \cite{Lin:Cised} prove that the \sed tolerance can be checked by finding the common intersection of half-$\epsilon$ spatio-temporal cones built from points of a sub-trajectory $[P_s,...,P_{s+k}]$, \ie ``{given a sub-trajectory $[P_s,...,P_{s+k}]$ and an error bound $\epsilon$, $sed(P_{s+i}, \overline{P_sP_{s+k}})\le \epsilon$ for each $i \in [1,k]$ if  $\bigsqcap_{i=1}^{k}$\cone{(P_s, P_{s+i}, \frac{\epsilon}{2})} $\ne \{P_s\}$}''.
In other words, if these half-$\epsilon$ cones do have a common intersection, then line segment $\overline{P_sP_{s+k}}$ is able to represent the sub-trajectory. For efficiency consideration, \cised projects those cones on some plane, \eg plane $t=P_{s+1}.t$, so as to convert the checking of cone intersection into a much simpler way, \ie~\textit{the intersection of projection circles of those cones on the plane} (see Figure~\ref{fig:cis}).
For the same reason, a circle is further approximated by its inscribe \emph{regular polygon} $\mathcal{R}$ and the intersecting of circles is approximated by the intersecting of these polygons, which can be computed in a linear time.
%
%Note that though the cone intersection approach outperform the counterpart of \ldr and \ldrh, it is still not introduced to trajectory tracking.

\eat{%%%%%%%%%%%%%%%%%% Delete because of the page limitation.
	Figure~\ref{fig:cised} is a running example of \cised. In this case, it outperforms \ldrh in terms of compression ratio.
	
	\begin{figure}[tb!]
		\centering
		\includegraphics[scale=0.9]{figures/Fig-CISED-SH.png}
		\vspace{-1ex}
		\caption{\small A running example of \cised. In the first section, cones are projected on plane $P_1.t$, where (1) the projection circles of \circle{_{1}},\circle{_{2}},\circle{_{3}} and \circle{_{4}} have common intersection, and (2) it does not intersected with the projection circle of \circle{_{5}}. Thus, $P_4$ is output, and it serves as the new start point of the next section. Finally, the same trajectory is simplified to four points $\{P_0, P_4, P_7, P_8\}$.}
		\vspace{-2ex}
		\label{fig:cised}
	\end{figure}
}%%%%%%%%%%%%%%%%%%%%%%


\subsection{Tracking with spatio-temporal cones}

In this section, we will first show that the counterpart of \ldrh is just a special case of, and has a worse effectiveness in terms of compression ratio than the approaches based on spatio-temporal cone (even it uses a half-$\epsilon$ cone), thus, the latter is obviously a more effective way to develop trajectory tracking algorithms. Then, we further extend the half-$\epsilon$ cone used in \cised to a full-$\epsilon$ cone so as to get an even better effectiveness.



\begin{proposition}
\label{theo-ldrh-cised}
Given a sub-trajectory $\dddot{\mathcal{T}}_s[P_s,...,P_{s+k}]$ and an error bound $\epsilon$, if $\dddot{\mathcal{T}}_s$ can be represented by line segment $\overline{P_sP_{s+k}}$ through algorithm \ldrh, then it can also be represented by approaches based on spatio-temporal cones.
\end{proposition}

\begin{proof}
If $\dddot{\mathcal{T}}_s$ can be represented by line segment $\overline{P_sP_{s+k}}$ by algorithm \ldrh, then we have $|P_{s+i}P'_{s+i}| < \epsilon/2$ for each $i \in (0, k)$, where $P'_{s+i}$ is the expected position (synchronized point) of $P_{s+i}$ \wrt the initial velocity $\vv{v}$ of \ldrh.
From the view of the ``x-y-t'' 3D space, $\vv{v}$ must be in the common intersection of  $\bigsqcap_{i=1}^{k}$\cone{(P_s, P_{s+i}, \epsilon/2)}, in other words, we have $\bigsqcap_{i=1}^{k}$\cone{(P_s, P_{s+i}, \epsilon/2)} $\ne \{P_s\}$, meaning this sub-trajectory can be represented by approaches based on the spatio-temporal cones.
\end{proof}

Proposition \ref{theo-ldrh-cised} tells that (1) {approaches based on the spatio-temporal cone are also applicable to do trajectory tracking in a circular area}, and (2) \ldrh is just a special case of and has a worse effectiveness than approaches based on spatio-temporal cones. \ldrh assumes an initial velocity before the simplification or tracking of a sub-trajectory, while the spatio-temporal cone based methods do not, instead, it has an ability to find out the potentially feasible velocities to fit the movement of the object when it simplifies the sub-trajectory. Thus, the latter is sure a more effective way to develop one-pass trajectory tracking algorithms. 
%
Moreover, observe that the half-$\epsilon$ cone in the above is still a little conservative, we next extend it to a full-$\epsilon$ cone.% plus certain constrain to get an even better effectiveness.

\begin{figure}[tb!]
	\centering
	\includegraphics[scale=0.88]{figures/Fig-CIS.png}
	\vspace{-2ex}
	\caption{\small Examples of spatio-temporal cones in a 3D Cartesian coordinate system taking point $P_s$ as the origin, where (1) $P_{s+i}$ and $P_{s+k}$ are two points, (2) \circle{_{s+i}} and \circle{_{s+k}} are two synchronous circles, (3) \cone{_{s+i}} and \cone{_{s+k}} are two spatio-temporal cones.}
	\vspace{-1ex}
	\label{fig:cis}
\end{figure}


\begin{proposition}
	\label{theo-full-cone}
	Given a sub-trajectory $[P_s,...,P_{s+k}]$ and an error bound $\epsilon$, $sed(P_{s+i}, \overline{P_sP_{s+k}})\le \epsilon$ for each $i \in [1,k]$ if~ $\overline{P_sP_{s+k}}$ passes through  the common intersection $\bigsqcap_{i=1}^{k-1}$\cone{(P_s, P_{s+i}, \epsilon)} - \{$P_s$\}.
\end{proposition}

\begin{proof}
Let $P'_{s+i}$ be the intersection point of line segment $\overline{P_sP_{s+k}}$ and plane $t = P_{s+i}.t,~i\in (0,k)$, indeed, $P_{s+i}$ is the synchronized point of $P_{s+i}$ \wrt line segment $\overline{P_sP_{s+k}}$. 
Because $\overline{P_sP_{s+k}}$ passes through $\bigsqcap_{i=1}^{k-1}$\cone{(P_s, P_{s+i}, \epsilon)} - \{$P_s$\}, $P'_{s+i}$ must be inside of the synchronous circle of $P_{s+i}$ on the plane. Thus we have $|P_{s+i}P'_{s+i}|<\epsilon$, \ie $sed(P_{s+i}, \overline{P_sP_{s+k}})|<\epsilon$.
\end{proof}

Proposition \ref{theo-full-cone} tells that full-$\epsilon$ spatio-temporal cones, with a constrain that the line segment $\overline{P_sP_{s+i}}$ passes through the common intersection of all the preview cones, can also be used in trajectory simplification and/or tracking. Thus we get two ways of trajectory tracking in a circle. The question is, which one is the better?

\begin{proposition}
	\label{theo-cone-vs}
	Given a sub-trajectory $[P_s,...,P_{s+k}]$ and an error bound $\epsilon$, if $\bigsqcap_{i=1}^{k}$\cone{(P_s, P_{s+i}, \epsilon/2)} $\ne \{P_s\}$, then $\overline{P_sP_{s+k}}$ passes through the common intersection $\bigsqcap_{i=1}^{k-1}$\cone{(P_s, P_{s+i}, \epsilon)}-$\{P_s\}$; and the opposite is not necessarily true.
\end{proposition}

\begin{proof}
(1) If $\bigsqcap_{i=1}^{k}$\cone{(P_s, P_{s+i}, \epsilon/2)} $\ne \{P_s\}$, then $sed(P_{s+i}, P_sP_{s+k}) <\epsilon$ for each $i \in [1,k]$. 
The intersection point  $P'_{s+i}$ of line segment $P_s P_{s+k}$ and plane $t = P_{s+i}.t$ is the synchronized point of $P_{s+i}$ \wrt $P_s P_{s+k}$, such that $|P_{s+i}P'_{s+i]}| < \epsilon$. Hence for each $i \in [1,k]$, $P'_{s+i}$ falls in the synchronous circle of $P_{s+i}$ on the plane, meaning $\overline{P_sP_{s+k}}$ passes through the common intersection of the preview cones $\bigsqcap_{i=1}^{k-1}$\cone{(P_s, P_{s+i}, \epsilon)}-$\{P_s\}$.
%
(2) If line segment $\overline{P_sP_{s+k}}$ passes through the common intersection $\bigsqcap_{i=1}^{k-1}$\cone{(P_s, P_{s+i}, \epsilon)} - \{$P_s$\}, then, suppose there is $i$ and $j$ ($i<j<k$) such that $P^{s+j}_{s+i}$ and $P^{s+k}_{s+i}$ are the intersection points of $\overline{P_sP_{s+j}}$ and $\overline{P_sP_{s+k}}$ with the plane $t = P_{s+i}.t$, respectively, we have $0 \le |P_{s+i}P^{s+k}_{s+i}|<\epsilon$ and $0 \le |P^{s+j}_{s+i}P^{s+k}_{s+i}|<\epsilon$, meaning $0 \le |P_{s+i}P^{s+j}_{s+i}|<2\epsilon$, \ie~$|P_{s+i}P^{s+j}_{s+i}|$ is not necessarily less than $\epsilon$. 
%
If it is greater than $\epsilon$, then the intersection of \cone{(P_s,P_{s+i},\epsilon/2)} and \cone{(P_s,P_{s+j},\epsilon/2)} is $\{P_s\}$, and $\bigsqcap_{i=1}^{k}$\cone{(P_s, P_{s+i}, \epsilon/2)} is also $\{P_s\}$.
%
Combine (1) and (2) we have the conclusion.
\end{proof}

Proposition \ref{theo-cone-vs} tells that, given the same sub-trajectory and start point, the full-$\epsilon$ cone approach has an ability to include more points into a line segment than the half-$\epsilon$ cone, which is in turn better than \ldrh, in trajectory simplification and/or tracking. Thus, it is better to use the full-$\epsilon$ cone to develop trajectory tracking algorithms.
%as shown in Proposition \ref{theo-ldrh-cised}

\subsection{Algorithm}
We then provide a one-pass trajectory tracking algorithm based on spatio-temporal cone, named \underline{C}one \underline{I}ntersection for \underline{T}rajectory \underline{T}racking (\citt). This algorithm uses full-$\epsilon$ spatio-temporal cones and can run on mobile devices. In a nutshell, when a distance deviation of position tracking occurs, unlike \ldr and \ldrh, algorithm \citt does not roughly send an update of a new start position $P_s$ and a new velocity $\vv{v}$, \ie a \emph{position-velocity-message} ($P_s$, $\vv{v}$), to the MOD server. Instead, it tries to finds out a new feasible velocity $\vv{v}$ by the intersection of spatio-temporal cones such that only the new velocity $\vv{v}$ is sent to the MOD server while the start point $P_s$ keeps the same. 
Thus, there are two kinds of messages in \citt, a) \emph{velocity-messages} that only report the new velocity information  $\vv{v}$ and b) \emph{position-velocity-messages} that report the new position and velocity information ($P_s$, $\vv{v}$) when no velocity is feasible for the current start position.
By this way, a line segment of \citt is potential to represent a longer sub-trajectory compared with \ldrh, therefor the storage and network bandwidth are saved.

\stitle{Algorithm \citt}. Given a trajectory $\dddot{\mathcal{T}}$ and an error bound $\epsilon$, \citt initializes the velocity $\vv{v}$ (including value and direction) and sends it coupling with the start position $P_s$ of the trajectory to the MOD server. Then, for each point $P_{s+i}$, $i>0$, it checks (1) the deviation of the actual position $P_{s+i}$ to its synchronized point $P'_{s+i}$ built from the start point $P_s$, velocity $\vv{v}$ and time $P_{s+i}.t$, for the purpose of position tracking, and (2) the common intersection of spatio-temporal cones built from points $P_{s+j}$, $j \in [1, i]$, for the purpose of trajectory simplification.
%
During the checking, (1) if the common intersection of cones is $\{P_s\}$, then an update of new $(P_s, \vv{v})$ is sent to the MOD server and the algorithm goes on to process the next sub-trajectory start from the new $P_s$;
%
(2) if the common intersection is not $\{P_s\}$ but the position deviation is lager than the given threshold $\epsilon$, meaning this sub-trajectory can be represented by a line segment and there is a more appropriate velocity $\vv{v}$ for position tracking, then an update of the new $\vv{v}$ is sent to the MOD server and the algorithm goes on processing the same sub-trajectory start from the $P_s$;
%
otherwise, (3) no deviation occurs, no update is sent and the algorithm goes on processing the same sub-trajectory. 
%
When setting a new velocity $\vec{v}$, any $\vec{v}$ satisfying $\vec{v}.\theta = \overline{P_sQ}.\theta$ and $|\vec{v}|=|P_sQ|/(Q.t-P_s.t)$ is applicable, where $Q$ is a point living in the common intersection of the spatio-temporal cones. For convenience, we roughly let $Q$ be the current point $P_{s+i}$.

%Note that like \cised, this algorithm also transforms the intersection of cones to the intersection of regular polygons.

\begin{figure}[tb!]   % full cones
	\begin{center}
		{\small
			\begin{minipage}{3.3in}
				\myhrule
				%\vspace{-1ex}
				\mat{0ex}{
					{\bf Algorithm}~\citt $(\dddot{\mathcal{T}}[P_0,\ldots,P_n],~\epsilon,~m)$\\
					%	\sstab
					\bcc \hspace{1ex}\= $P_s := P_0$; ~~~~$\mathcal{R}^*$ := \kw{getRPolygon}($P_s$, $P_{s+1}$, $\epsilon$, $m$, $P_{s+1}.t$); \\
					\icc \hspace{1ex}\= $|\vv{v}|:=\frac{|P_{s}P_{s+1}|}{P_{s+1}.t-P_s.t}$; ~~~~$\vv{v}.\theta:=\overline{P_{s}P_{s+1}}.\theta$; \\
					\icc \hspace{1ex}\= update ($P_{s}, \vv{v}$); \\
					\icc \hspace{1ex}\= $i := 2$; 	\\
					\icc \hspace{1ex}\= while $i \le n$ do \\
					\icc \>\hspace{3ex} if $\overline{P_sP_{i}}$ ~does not pass~ $\mathcal{R}^*$ then \\ % // updates velocity and location \\
					\icc \>\hspace{7ex}    $P_s := P_{i-1}$; ~~~~$\mathcal{R}^*$ := $\emptyset$; \\
					\icc \>\hspace{7ex}    $|\vv{v}|:=\frac{|P_{s}P_{i}|}{P_{i}.t-P_s.t}$; ~~~~~$\vv{v}.\theta:=\overline{P_{s}P_{i}}.\theta$; \\
					\icc \>\hspace{7ex}    update ($P_{s}, \vv{v}$); \\
					\icc \>\hspace{3ex} else if $sed(P_{i},\vv{v}) \ge \epsilon $ then  \\ %~$\overline{P_sP_{i}}$ ~passes~ $\mathcal{R}^*$ and
					\icc \>\hspace{7ex}    $|\vv{v}|:=\frac{|P_sP_{i}|}{P_{i}.t-P_s.t}$; ~~~~~$\vv{v}.\theta:=\overline{P_sP_{i}}.\theta$; \\
					\icc \>\hspace{7ex}    update ($\vv{v}$); \\
					\icc \>\hspace{3ex} if $\mathcal{R}^*=\emptyset$ then $\mathcal{R}^*:=$ \kw{getRPolygon}($P_s$, $P_{i}$, $\epsilon$, $m$, $P_{s+1}.t$); \\
					\icc \>\hspace{3ex} else $\mathcal{R}^*$ := $\mathcal{R}^*\bigsqcap$ \kw{getRPolygon}($P_s$, $P_{i}$, $\epsilon$, $m$, $P_{s+1}.t$); \\
					\icc \>\hspace{3ex} $i$ := $i +1$;	\\
					\icc \>\hspace{0ex} update $(P_{n}, 0)$; 
				}
				\vspace{-2ex}
				\myhrule
			\end{minipage}
		}
	\end{center}
	\vspace{-2ex}
	\caption{\small Trajectory tracking based on spatio-temporal cone.}
	\label{alg:citt-s-full}
	\vspace{-2ex}
\end{figure}
%%%%%%%%%%%%%%%%%%%%%%%%%%%%%%%%%%%%%


Figure~\ref{alg:citt-s-full} is the P-codes of \citt. It takes as input a trajectory \trajec{T}${[P_0, \ldots, P_n]}$, an error bound $\epsilon$ and the number $m$ of edges for an inscribed regular polygon (recall in \cised, a circle is approximated by its $m$--edges inscribe \emph{regular polygon} $\mathcal{R}$ and accordingly the intersecting of circles is approximated by the intersecting of these polygons), and outputs a set of velocities and a simplified  trajectory $\overline{\mathcal{T}}$ of $\dddot{\mathcal{T}}$.
%
The algorithm first initializes the start point $P_s$ to $P_0$, the common intersection of polygons $\mathcal{R}^*$ to the regular inner polygon of $P_1$ by calling procedure $\kw{getRPolygon}()$ \cite{Lin:Cised}, and the expected velocity $(|\vv{v}|, \vv{v}.\theta)$ to ($\frac{|P_{s}P_{s+1}|}{P_{s+1}.t-P_s.t},\overline{P_{s}P_{s+1}}.\theta$) (lines 1--2), then it sends its initial location $P_s$ and the velocity $\vv{v}$ to the MOD server (line 3), meaning that it is supposed to move from point $P_s$ along the direction of $\vv{v}.\theta$ at a speed of $|\vv{v}|$.
%, such that the expecting position of the object at time $t>P_s.t$ can be extrapolated from them as long as no subsequent update is sent to the MOD server.
%
This algorithm sequentially processes the rest points of the trajectory one by one (lines 4--15). 
For the current point $P_{i}$, if $\overline{P_sP_{i}}$ passes through the common intersection $\mathcal{R}^*$, meaning that it can not includes more points into a line segment, then a line segment $\overline{P_sP_{i-1}}$ and a new section started from  $P_{i-1}$ is generated, the common intersection of cones is set to null, and $P_{i-1}$ and a new velocity $\vv{v}$ are sent to the MOD server (lines 6--9).
%
Otherwise, it calculates the distance from the actual location $P_{i}$ to the expecting location $P'_{i}$ extrapolated from the initial location $P_s$ and the velocity $\vv{v}$, and checks whether $\overline{P_sP_{i}}$ passes through the common intersection $\mathcal{R}^*$ of the preview points.
If it passes through and the distance $|P_{i}P'_{i}| \ge \epsilon$, meaning that more points can be included into a line segment while the velocity $\vv{v}$ needs to be updated. Hence, it updates velocity $\vv{v}$ based on $P_s$ and $P_{i}$, and sends it to the MOD server (lines 10--12). 
%
Anyway, the algorithm gets the $m$-edge inscribed regular polygon \wrt the current point $P_{i}$ by calling procedure $\kw{getRPolygon()}$ \cite{Lin:Cised} and gets the common intersection $\mathcal{R}^*$ of the preview cones (lines 13--14). The process repeats until all points have been processed (line 15).
At last, it outputs the last point $P_{n}$ (line 16).
%





\begin{figure}[tb!]
	\centering
	\includegraphics[scale=1.0]{figures/Fig-CITT.png}
	\vspace{-2ex}
	\caption{\small A running example of trajectory tracking by \citt. }
	\vspace{-3ex}
	\label{fig:citt}
\end{figure}


\begin{example}
	Figure~\ref{fig:citt} is a running example of algorithm \citt. It takes the same input and sets the same start point and initial velocity as Figure~\ref{fig:ldr}, and uses full-$\epsilon$ cones to simplify the trajectory. Then, (1) $P_3$ lives in the common intersection of \cone{_{1}} and \cone{_{2}} and it has a distance larger than $\epsilon$ to its expected position $P'_3$ \wrt $\vec{v_1}$, thus, \citt updates the velocity from $\vec{v_1}$ to $\vec{v_3}$ and the process goes on, and (2) $P_5$ is outside of the common intersection of the preview cones, thus, $P_4$ serves as the new start point, and an update is triggered. Finally, this algorithm sends three points, $P_0, P_4$ and $P_8$ (not shown), and four velocities, $\vec{v_1}$, $\vec{v_3}$, $\vec{v_5}$ (not shown) and $\vec{v_8}$ (not shown), to the MOD server. Also note that, no matter during the tracking or after the simplification, this algorithm ensures that any removed point is located in a circle around its expected position \wrt a velocity of trajectory tracking or a line segment connecting two neighboring data points of the simplified trajectory. 
\end{example}


\stitle{Correctness and complexity.} 
The correctness of algorithm \citt follows from Propositions \ref{theo-ldrh-cised} and \ref{theo-full-cone}.
It is easy to find that every point is processed only once in \citt, and for each point, it needs $O(1)$ time as getRPolygon() \cite{Lin:Cised}, intersecting of polygons \cite{Lin:Cised} and {other operations} all have a time complexity of $O(1)$. Hence, \citt has a time complexity of $O(n)$, where $n$ is the number of data points.






\eat{%%%%%%%%%%%%%%%%%%%%%%%%%%%%%%%%%%%%%%%%%%%%%%%%
	
\subsection{Weak tracking}

-- theorem: if ldr is true, then the intersection of those cones must not be \{$P_s$\}.
means that, if ldr is true, it is sure those points can be represented by a line segment.
thus, we only need to check the intersection after ldr is false, during the process.

-- algorithm CITT-W(e)

-- example

-- correctness and complexity
}%%%%%%%%%%%%%%%%%%%%%%%%%%%%%%%%%%%%%%%%%%%%%%%%%%%%%


%%% Local Variables:
%%% mode: latex
%%% TeX-master: "www2019"
%%% End:

\section{Tracking in a rectangle-like area}
\label{sec:rectangle}

This section develops an effective and efficient way to track a moving object in a rectangle-like area. Though it is possible to do a tracking in a rectangular area, however, it is really hard to design such a shape and at the same time develop such an efficient algorithm. Thus, we alternatively define a rectangle-like area.


\subsection{Building the rectangle-like areas}

Like the circular area related to \sed and the strip area related to \ped, a rectangle-like area is also related to a Euclidean distance metric, namely the binary Euclidean distances of \sed and \ped.

\stitle{Binary Euclidean distances of \sed and \ped}, shortly \bed (\sed, \ped), is the combination of distance metrics \sed and \ped where their error bounds are set separately to $\epsilon_{sed}$ and $\epsilon_{ped}$, such that (1) if $\epsilon_{ped} \ge \epsilon_{sed}$, then it falls back to the \sed, \ie tracking in a circular area, otherwise, (2) it is the double effects of \ped and \sed, and forms a rectangle-like shape whose short sides are replaced by circular arcs of \sed as shown in {Figure~\ref{fig:areas}-(3)}. Obviously, if $\epsilon_{ped} << \epsilon_{sed}$, then its effect is actually approximate to a \ped.

%By combining \sed and \ped, we get a 
The shape of a rectangle-like area is controlled by two independent parameters, $\epsilon_{ped}$ and $\epsilon_{sed}$, thus, by carefully setting them, we not only build such rectangle-line areas that satisfy the needs of varied applications but also balance two performance metrics, the compression ratios and the errors of spatio-temporal queries, of trajectory simplification/tracking algorithms. 
For the same example of ``the school boy on his way home" mentioned in Section \ref{sec-intro}, if we only use \ped, then we get a simplified trajectory having a good compression ratio and a unbounded query error; if we only use \sed, then we get a poorer compression ratio and a bounded query error. However, if we combine them, \ie~use the \bed, then we could get a medium compression ratio and a bounded query error.



\subsection{Tracking by cone and sector}

Is there an effective and efficient trajectory algorithm implementing \bed that tracks a moving object in a rectangle-like area? Theorem \ref{theo-binary} is the answer to this question.
\begin{theorem}
	\label{theo-binary}
	Given a sub-trajectory $[P_s,...,P_{s+k}]$ and two error bounds $\epsilon_{sed}$ and $\epsilon_{ped}$, it can be tracked in rectangle-like areas by combining sectors and spatio-temporal cones.
\end{theorem}

To track the positions and simplify the trajectory at the same time in a rectangle-like area, it is important to make sure that during the processing of a sub-trajectory $[P_s,...,P_{s+k}]$, there are the same start point $P_s$ and the same velocity $\vec{v}$ for each technique of spatio-temporal cones, sectors, and position tracking of \ped and \sed, such that a strip and a circle \wrt a velocity $\vec{v}$ or a line segment $\overline{P_sP_{s+i}}$, $0<i\le k$, exactly form a rectangle-like area. This is the guideline to develop such a trajectory tracking algorithm. 



\todo{Could ``$|P_sP_{i}| \le l_{m} $" be replaced by a more relaxed constrain, by benefiting from \sed?}





%%%%%%%%%%%%%%%%%%%%%%%%%%%%%%%%%%%%%%%%%%%%%%%%%%%%%%%%%%%%%%%%%%%%%%%%%%
% Algorithm: Traj tracking based on section intersection using full sectors.
\begin{figure}[tb!]   
	\begin{center}
		{\small
			\begin{minipage}{3.3in}
				\myhrule
				%\vspace{-1ex}
				\mat{0ex}{
					{\bf Algorithm}~\bitt $(\dddot{\mathcal{T}}[P_0,\ldots,P_n], ~\epsilon_{sed}, m, ~\epsilon_{ped})$\\
					%	\sstab
					\bcc \hspace{1ex}\= $P_s := P_0$; ~~~~$\mathcal{R}^*$ := \kw{getRPolygon}($P_s$, $P_{s+1}$, $\epsilon_{sed}$, $m$, $P_{s+1}.t$); \\
					\icc \hspace{1ex}\= $\mathcal{S}^*$ := \kw{getSector}($P_s$, $P_{s+1}$, $\epsilon_{ped}$); ~~~~$l_{m} = |P_sP_{s+1}|$;\\
					\icc \hspace{1ex}\= $|\vv{v}|:=\frac{|P_{s}P_{s+1}|}{P_{s+1}.t-P_s.t}$; ~~~~$\vv{v}.\theta:=\overline{P_{s}P_{s+1}}.\theta$;  \\
					\icc \hspace{1ex}\= update ($P_{s}, \vv{v}$); 	\\
					\icc \hspace{1ex}\= $i:= 2$;  	\\
					\icc \hspace{1ex}\= while $i \le n$ do \\
					\icc \>\hspace{3ex} if $\overline{P_sP_{i}}$ ~does not pass~ $\mathcal{R}^*~or~\mathcal{S}^*$, or $|P_sP_{i}| < l_{m}$ then \\ % // updates velocity and location \\
					\icc \>\hspace{7ex}    $P_s := P_{i-1}$; ~~~~$\mathcal{R}^*$ := $\emptyset$;~~~~$\mathcal{S}^*$ := $\emptyset$; ~~~~$l_{m} = |P_sP_{i}|$;\\
					\icc \>\hspace{7ex}    $|\vv{v}|:=\frac{|P_sP_{i}|}{P_{i}.t-P_s.t}$; ~~~~$\vv{v}.\theta:=\overline{P_{s}P_{i}}.\theta$;  \\
					\icc \>\hspace{7ex}    update ($P_{s}, \vv{v}$); 	\\
					\icc \>\hspace{3ex} else if $sed(P_i, \vv{v}) \ge \epsilon_{sed}$ ~or~ $ped(P_i, \vv{v}) \ge \epsilon_{ped}$ then  \\ %$\overline{P_sP_{i}}$ ~passes ~ $\mathcal{R}^*$ and $\mathcal{S}^*$, $|P_sP_{i}| > l_{m} - \epsilon$ \\ \hspace{9ex} ~and~
					\icc \>\hspace{7ex}    $|\vv{v}|:=\frac{|P_sP_{i}|}{P_{i}.t-P_s.t}$; ~~~~$\vv{v}.\theta:=\overline{P_sP_{i}}.\theta$; \\
					\icc \>\hspace{7ex}    update ($\vv{v}$); \\
					\icc \>\hspace{3ex} if $\mathcal{S}^*=\emptyset$ then $\mathcal{R}^*:=$ \kw{getRPolygon}($P_s$, $P_{i}$, $\epsilon_{sed}$, $m$, $P_{s+1}.t$); \\
					\icc \>\hspace{7ex}    $\mathcal{S}^*:=$ \kw{getSector}($P_s$, $P_{i}$, $\epsilon_{ped}$); \\
					\icc \>\hspace{3ex} else $\mathcal{R}^*:=$ \kw{getRPolygon}($P_s$, $P_{i}$, $\epsilon_{sed}$, $m$, $P_{s+1}.t$); \\
					\icc \>\hspace{7ex}    $\mathcal{S}^*$ := $\mathcal{S}^*\bigsqcap$ \kw{getSector}($P_s$, $P_{i}$, $\epsilon_{ped}$); $l_{m} = \max\{|P_sP_{i}|, l_{m}\}$;\\
					\icc \>\hspace{3ex} $i$ := $i +1$;\\
					\icc \>\hspace{0ex} update ($P_{n}$); 
				}
				\vspace{-2ex}
				\myhrule
			\end{minipage}
		}
	\end{center}
	\vspace{-2ex}
	\caption{\small Trajectory tracking based on sector and cone.}
	\label{alg:bitt}
	\vspace{-2ex}
\end{figure}
%%%%%%%%%%%%%%%%%%%%%%%%%%%%%%%%%%%%%

\subsection{Implementation.}
We now present the algorithm of \underline{B}inary \underline{I}ntersection for \underline{T}rajectory \underline{T}racking (BITT) that tracks moving objects in a rectangle-like area of \bed, as shown in Figure~\ref{alg:bitt}. 
%
Indeed, it is a double check of cone intersection and sector intersection for each point for the purpose of trajectory simplification, and a double check of \sed and \ped distance deviations for position tracking. In addition to that, it uses a uniform velocity $\vv{v}$ for both position tracks of \sed and \ped such that either deviation of \ped or \sed distance will cause an update of velocity $\vv{v}$. 
%
\bitt is the super version of \citt and \sitt, that is, if $\epsilon_{sed} >> \epsilon_{ped}$, then ``$\overline{P_sP_{i}}$ does not pass $\mathcal{R}^*$" of line 7 and ``$sed(P_i, \vv{v}) \ge \epsilon_{sed}$" of line 11 are always false, thus \bitt falls back to \sitt. {Similarly, if $\epsilon_{sed} \le \epsilon_{ped}$, then it falls back to \citt.}




\begin{example}
	Figure~\ref{fig:bitt} is a running example of \bitt. It takes as inputs the same trajectory as the above, the same $\epsilon_{sed}$ as Figure~\ref{fig:citt} and an $\epsilon_{ped}$ of half that of Figure~\ref{fig:sitt}. Because, its $\epsilon_{sed}$ is the same as Figure~\ref{fig:citt}, its effectiveness is also the same as Figure~\ref{fig:citt}. For the purpose of clearness, we do not show those cones in the figure.
	%
	\bitt uses full-$\epsilon_{sed}$ cones and full-$\epsilon_{ped}$ sectors to simplify the trajectory. Initially, it sets the same start point and initial velocity as Figure~\ref{fig:citt}, 
	Then, (1) $P_3$ lives in the common intersection of the preview cones and sectors, and it has both \sed and \ped distances larger than $\epsilon_{sed}$ and $\epsilon_{ped}$, respectively, \ie $|P_3P'_3| \ge \epsilon_{sed}$ and $|P_3P^*_3| \ge \epsilon_{ped}$, thus, \bitt updates the velocity from $\vec{v_1}$ to $\vec{v_3}$ and the process goes on, (2) $P_5$ is outside of the common intersections of the preview cones and sectors, thus, $P_4$ serves as the new start point, and an update is triggered, and (3) $P_7$ lives in the common intersections of the preview cones and sectors, and it has a \ped distance larger than $\epsilon_{ped}$, thus, \bitt updates the velocity from $\vec{v_5}$ to $\vec{v_7}$ (not shown) and the process goes on. Finally, this algorithm sends three points, $P_0, P_4$ and $P_8$, and four velocities, $\vec{v_1}$, $\vec{v_3}$, $\vec{v_5}$ and $\vec{v_7}$, to the MOD server. 
	\bitt ensures that any removed point is located in a rectangle-like area around its expected position \wrt a velocity of trajectory tracking or a line segment connecting two neighboring data points of the simplified trajectory, \eg $P_2$ is in the rectangle-like area around its synchronized point $P'_2$ \wrt line segment $\overline{P_0P_4}$. 
\end{example}

\begin{figure}[tb!]
	\centering
	\includegraphics[scale=1.0]{figures/Fig-BITT.png}
	\vspace{-2ex}
	\caption{\small A running example of trajectory tracking by \bitt. In this case, the spatio-temporal cones and their intersections are the same as Figure~\ref{fig:citt}, thus they are not shown here for clearness.  }
	\vspace{-2ex}
	\label{fig:bitt}
\end{figure}

\stitle{Correctness and complexity.} 
The correctness of algorithm \bitt follows from Theorems \ref{theo-full-cone}, \ref{theo-full-sector} and \ref{theo-binary}.
It is also easy to find that it has a linear time complexity like \citt and \sitt.


%\stitle{Discuss.} Relations to position tracking and traj simplification. 

%%% Local Variables:
%%% mode: latex
%%% TeX-master: "gis18"
%%% End:
\section{Experimental Study}
\label{sec-exp}

\eat{
\begin{table*}[!ht]
	\renewcommand{\arraystretch}{1.20}
	\caption{\small Real-life Trajectory Datasets}
	\vspace{-1.5ex}
	\centering
	\footnotesize
	%\scriptsize
	\begin{tabular}{|l|c|c|c|r|}
		\hline
		\bf{ Data Sets}& \bf{Number\ of Trajectories}     &\bf {Sampling Rates\ (s)}   &\bf{Points Per Trajectory\ (K)}    &\bf {Total points} \\
		\hline
		\sercar	&1,000	    &3-5	    &$\sim114.0$   &114M\\
		\hline
		\geolife &182	    &1-5	    &$\sim131.4$   &24.2M\\
		\hline
		\mopsi &51	    	&2	    &$\sim153.9$     &7.9M\\
		\hline
	\end{tabular}
	\label{tab:datasets}
	\vspace{-2ex}
\end{table*}
}

\begin{table}[tb!]
	\renewcommand{\arraystretch}{1.20}
	\caption{\small Real-life Trajectory Datasets}
	\vspace{-1.5ex}
	\centering
	\footnotesize
	%\scriptsize
	\begin{tabular}{|l|c|c|r|}
		\hline
		\bf{ Properties of Data Sets} & \sercar      &\geolife   &\mopsi \\
		\hline
		{Number\ of Trajectories}	&1,000	    &182	    & 51  \\
		\hline
		 {Sampling Rates\ (s)} &3-5  & 1-5 & 2 \\
		\hline
		{Points Per Trajectory\ (K)}  &	$\sim114.0$    &$\sim131.4$	    & $\sim153.9$ \\
		\hline
		 {Total points (M)} &114   	    	&24.2    &7.9\\
		\hline
	\end{tabular}
	\label{tab:datasets}
	\vspace{-2ex}
\end{table}


In this section, we present an extensive experimental study of our one-pass trajectory tracking algorithms (\citt, \sitt and \bitt) compared with the
existing algorithms of \ldrh and \grts on trajectory datasets. Using three real-life trajectory datasets, we conducted sets of experiments to evaluate:
(1) the compression ratios,
(2) the number of messages (including data points and velocities),
(3) the max and average errors, and
(4) the running time of algorithms \citt, \sitt and \bitt vs. \ldrh and \grts. 
Among them, the impacts of error bounds and distance metrics on messages, errors and running time of these algorithms are evaluated. 




\begin{figure*}[tb!]
	\centering
	\includegraphics[scale = 0.56]{figures/Fig-BITT-mopsi-compression-ratio.png}\hspace{1ex}
	\includegraphics[scale = 0.56]{figures/Fig-BITT-sercar-compression-ratio.png}\hspace{1ex}
	\includegraphics[scale = 0.56]{figures/Fig-BITT-geolife-compression-ratio.png}\hspace{1ex}
	\vspace{-2ex}
	\caption{\small Evaluation of the compression ratios of \bitt: varying error bounds $\epsilon_{sed}$ and $\epsilon_{ped}$.}
	\label{fig:bitt-compression-ratio}
	\vspace{-1ex}
\end{figure*}


\begin{figure*}[tb!]
	\centering
	\includegraphics[scale = 0.565]{figures/Fig-BITT-mopsi-total-messages.png}\hspace{1ex}
	\includegraphics[scale = 0.565]{figures/Fig-BITT-sercar-total-messages.png}\hspace{1ex}
	\includegraphics[scale = 0.565]{figures/Fig-BITT-geolife-total-messages.png}\hspace{1ex}
	\vspace{-2ex}
	\caption{\small Evaluation of the total messages of \bitt: varying error bounds $\epsilon_{sed}$ and $\epsilon_{ped}$.}
	\label{fig:bitt-total-message}
	\vspace{-1ex}
\end{figure*}




\begin{figure*}[tb!]
	\centering
	\includegraphics[scale = 0.56]{figures/Fig-BITT-mopsi-sed-error.png}\hspace{1ex}
	\includegraphics[scale = 0.56]{figures/Fig-BITT-sercar-sed-error.png}\hspace{1ex}
	\includegraphics[scale = 0.56]{figures/Fig-BITT-geolife-sed-error.png}\hspace{1ex}
	\vspace{-2ex}
	\caption{\small Evaluation of the \sed errors of \bitt: varying error bounds $\epsilon_{sed}$ and $\epsilon_{ped}$.}
	\label{fig:bitt-sed-error}
	\vspace{-1ex}
\end{figure*}



\begin{figure*}[tb!]
	\centering
	\includegraphics[scale = 0.56]{figures/Fig-BITT-mopsi-ped-error.png}\hspace{1ex}
	\includegraphics[scale = 0.56]{figures/Fig-BITT-sercar-ped-error.png}\hspace{1ex}
	\includegraphics[scale = 0.56]{figures/Fig-BITT-geolife-ped-error.png}\hspace{1ex}
	\vspace{-2ex}
	\caption{\small Evaluation of the \ped errors of \bitt: varying error bounds $\epsilon_{sed}$ and $\epsilon_{ped}$.}
	\label{fig:bitt-ped-error}
	\vspace{-1ex}
\end{figure*}

\subsection{Experimental setting}

\stitle{Real-life Trajectory Datasets}. We use three reallife datasets ServiceCar, GeoLife and Mopsi shown in Table \ref{tab:datasets} to test our solutions.

\vspace{0.5ex}
\ni \emph{(1) Service car trajectory data} (\sercar) is the GPS trajectories collected by a Chinese car rental company during Apr. 2015 to Nov. 2015. The sampling rate was one point per $3$--$5$ seconds, and
each trajectory has around $114.1K$ points.

\vspace{0.5ex}
\ni \emph{(2) GeoLife trajectory data} (\geolife) is the GPS trajectories collected in GeoLife project by 182 users in a period from Apr. 2007 to Oct. 2011. These trajectories have a variety of sampling rates, among which 91\% are logged in each 1-5 seconds per point. %or each 5-10 meters

\vspace{0.5ex}
\ni \emph{(3) Mopsi trajectory data} (\mopsi) is the GPS trajectories collected in Mopsi project by 51 users in a period from 2008 to 2014. Most routes are in Joensuu region, Finland.
The sampling rate was one point per $2$ seconds, and each trajectory has around $153.9K$ points.

\stitle{Algorithms and implementation}.
We implement five tracking algorithms, \ie our \citt, \sitt and \bitt, \ldrh \cite{Trajcevski:LDRH} (the first and the most efficient trajectory tracking algorithm) and \grts~\cite{Lange:GRTS,Lange:Tracking} (the most effective tracking algorithm).
All algorithms were implemented with Java.
All tests were run on an {x64-based  PC with 8 Intel(R) Core(TM) i5-6500 CPU @ 3.20GHz and 8GB of memory.
	%, and each test was repeated over 3 times and the average is reported here.
	
\stitle{Metrics.}
Following the main stream \cite{Trajcevski:LDRH, Lange:GRTS, Lange:Tracking, Lin:Cised, Zhang:Evaluation}, we use \emph{compression ratio}, \emph{message ratio}, \emph{error} and \emph{running time} to evaluate algorithms.

 \ni \emph{(1) Compression ratio}. {It is defined as follows: Given a set of trajectories $\{\dddot{\mathcal{T}_1}, \ldots, \dddot{\mathcal{T}_M}\}$ and their piece-wise line representations $\{\overline{\mathcal{T}_1}, \ldots, \overline{\mathcal{T}_M}\}$, the compression ratio of an algorithm is $(\sum_{j=1}^{M} |\overline{\mathcal{T}}_j |)/(\sum_{j=1}^{M} |\dddot{\mathcal{T}}_j |)$.
	By the definition, \emph{algorithms with lower compression ratios are better}.}

 \ni \emph{(2) Message ratio}. It is ``the total number of messages'' divided by ``the total number of the original trajectory points''. Note there are totally three kinds of messages, \ie a) \emph{position-message} $P_s$ for \grts, b) \emph{velocity-message} $\vv{v}$ for \bitt (including \citt and \sitt) and c) \emph{position-velocity-message} ($P_s$, $\vv{v}$) for \ldrh, \grts and \bitt. 
 %For fair comparison, a \emph{position-velocity-message}
 

 \ni \emph{(3) Max and average errors}. Max (average) error is the maximal (average) value of the distances from every point of the original trajectories to its representing line segment of the simplified trajectories.
 
 \ni \emph{(4) Running time}. It is the essential execution time of an algorithm in processing a dataset.
 %It is the efficiency  of the algorithms.
 



\begin{figure*}[tb!]
	\centering
	\includegraphics[scale = 0.580]{figures/Fig-mopsi-compression-ratio.png}\hspace{-1ex}
	\includegraphics[scale = 0.580]{figures/Fig-sercar-compression-ratio.png}\hspace{-1ex}
	\includegraphics[scale = 0.580]{figures/Fig-geolife-compression-ratio.png}\hspace{0ex}
	\vspace{-1ex}
	\caption{\small Evaluation of compression ratios: varying error bounds $\epsilon_{sed}$ and $\epsilon_{ped}$.}
	\label{fig:compression-ratio}
	\vspace{-1ex}
\end{figure*}



\begin{figure*}[tb!]
	\centering
	\includegraphics[scale = 0.580]{figures/Fig-mopsi-total-messages.png}\hspace{-1ex}
	\includegraphics[scale = 0.580]{figures/Fig-sercar-total-messages.png}\hspace{-1ex}
	\includegraphics[scale = 0.580]{figures/Fig-geolife-total-messages.png}\hspace{0ex}
	\vspace{-1ex}
	\caption{\small Evaluation of total messages: varying error bounds $\epsilon_{sed}$ and $\epsilon_{ped}$.}
	\label{fig:total-message}
	\vspace{-1ex}
\end{figure*}

\eat{%%%%%%%%%%%%%%%%%%%%velocity messages
\begin{figure*}[tb!]
	\centering
	\includegraphics[scale = 0.580]{figures/Fig-mopsi-speed-messages.png}\hspace{-1ex}
	\includegraphics[scale = 0.580]{figures/Fig-sercar-speed-messages.png}\hspace{-1ex}
	\includegraphics[scale = 0.580]{figures/Fig-geolife-speed-messages.png}\hspace{0ex}
	\vspace{-1ex}
	\caption{\small Evaluation of velocity messages: varying error bounds $\epsilon_{sed}$ and $\epsilon_{ped}$.}
	\label{fig:speed-message}
	\vspace{-1ex}
\end{figure*}
}%%%%%%%%%%%%%%%%%%%%%%velocity messages



\subsection{Experimental Results}

\subsubsection{Evaluation of~algorithm \bitt}
%%%%%%%%%%%% messages
This section test the impacts of \ped and \sed (\ie the shapes of finite beams) on algorithm \bitt. We varied error bounds $\epsilon_{sed}$ and $\epsilon_{ped}$ from $10$ meters to $200$ meters on the entire three datasets, respectively. The results are reported in Figures~\ref{fig:bitt-compression-ratio}, \ref{fig:bitt-total-message}, \ref{fig:bitt-sed-error} and \ref{fig:bitt-ped-error}.

%\stitle{Message and compression ratios}. Figures~\ref{fig:bitt-total-message} and \ref{fig:bitt-compression-ratio} tell 

\ni (1) Both message and compression ratios decrease with the increase of $\epsilon_{sed}$ and $\epsilon_{ped}$, respectively. It is clear that when the tracking area becomes larger, \bitt is more tolerant of the distance deviation of a moving object, hence, fewer messages are transmitted and fewer data points are saved.

\ni (2) The velocity  $\vv{v}$ (see Figures~\ref{alg:citt-s-full} and ~\ref{alg:bitt}) is not frequently updated during the process of a sub-trajectory for all $\epsilon$ in all datasets. More specifically, a) when $\epsilon_{ped} \ge \epsilon_{sed}$, \ie~\bitt falls back to \citt, the \emph{position-velocity-messages} are on average $(39.59\%, 42.69\%, 47.20\%)$ of the total messages \wrt datasets (\mopsi, \sercar, \geolife), respectively, and b) when $\epsilon_{ped} << \epsilon_{sed}$, \ie~\bitt falls back to \sitt, the \emph{position-velocity-messages} are on average $(66.09\%, 71.29\%, 70.07\%)$ of the total messages \wrt datasets (\mopsi, \sercar, \geolife), respectively. Otherwise, \bitt has the number of \emph{position-velocity-messages} between \citt and \sitt.
%More than \myred{half} messages are \emph{position-velocity-messages}  for all $\epsilon$ in all datasets, meaning that, given a start point $P_s$ and an initial velocity $\vv{v}$ as the way shown in Figures~\ref{alg:citt-s-full} and ~\ref{alg:bitt}, 

%\ni (3) Dataset \sercar has the highest message and compression ratios, compared with datasets \mopsi and \geolife, due to its lowest sampling rate. 

\ni (3) Both average \ped and \sed errors increase with the increase of $\epsilon_{sed}$ and $\epsilon_{ped}$, respectively.

\ni (4) Given a $\epsilon_{sed}$, both the compression and message ratios, and the average \sed and \ped errors are constant for all $\epsilon_{ped}$ that are greater than $\epsilon_{sed}$, \eg $\epsilon_{sed}=10$ and $\epsilon_{ped} \ge 10$, showing that \bitt falls back to \citt in these cases.

%\ni (6) When $\epsilon_{sed} > \epsilon_{ped}$, the performance of algorithm \bitt will be similar to that of \sitt, as a result, the average \ped error will be much smaller than $\epsilon_{sed}$,
%the $\epsilon_{ped}$ is mainly in effect,

\begin{figure*}[tb!]
	\centering
	\includegraphics[scale = 0.580]{figures/Fig-mopsi-sed-error.png}\hspace{-1ex}
	\includegraphics[scale = 0.580]{figures/Fig-sercar-sed-error.png}\hspace{-1ex}
	\includegraphics[scale = 0.580]{figures/Fig-geolife-sed-error.png}\hspace{0ex}
	\vspace{-1ex}
	\caption{\small Evaluation of \sed errors: varying error bounds $\epsilon_{sed}$ and $\epsilon_{ped}$.}
	\label{fig:sed-error}
	\vspace{-1ex}
\end{figure*}

\begin{figure*}[tb!]
	\centering
	\includegraphics[scale = 0.580]{figures/Fig-mopsi-ped-error.png}\hspace{-1ex}
	\includegraphics[scale = 0.580]{figures/Fig-sercar-ped-error.png}\hspace{-1ex}
	\includegraphics[scale = 0.580]{figures/Fig-geolife-ped-error.png}\hspace{0ex}
	\vspace{-1ex}
	\caption{\small Evaluation of \ped errors: varying error bounds $\epsilon_{sed}$ and $\epsilon_{ped}$.}
	\label{fig:ped-error}
	\vspace{-1ex}
\end{figure*}

\begin{figure*}[tb!]
	\centering
	\includegraphics[scale = 0.580]{figures/Fig-mopsi-running-time.png}\hspace{-1ex}
	\includegraphics[scale = 0.580]{figures/Fig-sercar-running-time.png}\hspace{-1ex}
	\includegraphics[scale = 0.580]{figures/Fig-geolife-running-time.png}\hspace{0ex}
	\vspace{-1ex}
	\caption{\small Evaluation of running time: varying error bounds $\epsilon_{sed}$ and $\epsilon_{ped}$.}
	\label{fig:running-time}
	\vspace{-1ex}
\end{figure*}


\subsubsection{Comparing algorithms \bitt, \sitt and \citt with \ldrh and \grts.}
This section compares our algorithms \citt, \sitt and \bitt with algorithms \ldrh and \grts.
We varied the error bound (either $\epsilon_{sed}$ or $\epsilon_{ped}$) of \citt, \sitt, \ldrh and \grts from $10$ meters to $200$ meters on the entire three datasets, respectively. 
{For \bitt, its performance depends on the shape of the finite beam, \ie given the same area, it varies \wrt the ratio between $\epsilon_{ped}$ and $\epsilon_{sed}$. {Without losing generality}, we set its $\epsilon_{ped}$ to {$0.5$} times the $\epsilon_{ped}$ of \sitt and its $\epsilon_{sed}$ to {$1.6$} times the $\epsilon_{sed}$ of \citt, \grts and \ldrh, such that the area of the finite beam of \bitt is $3.147\times\epsilon_{sed}^2$, which is approximate to $3.142\times\epsilon_{sed}^2$, the area of the circular of algorithms \citt, \ldrh and \grts~\wrt the given $\epsilon_{sed}$.}
%
The results are reported in Figures~\ref{fig:compression-ratio}, \ref{fig:total-message}, \ref{fig:sed-error}, \ref{fig:ped-error} and \ref{fig:running-time}.


\stitle{Compression ratios.} We first report and analyze the compression ratios from Figure~\ref{fig:compression-ratio}.

\ni (1) When increasing $\epsilon_{sed}$ and $\epsilon_{ped}$, the compression ratios of all these algorithms decrease on all datasets.

\ni (2) \sitt has the best compression ratios, \ldrh is the worst, and \grts, \citt and \bitt are comparable on all datasets and for all $\epsilon$.
The compression ratios of \grts, \bitt,  \citt and \sitt are on average {($27.5\%$, $38.6\%$, $32.2\%$), ($34.9\%$, $37.3\%$, $36.7\%$), ($27.7\%$, $38.9\%$, $32.8\%$) and ($20.2\%$, $19.6\%$, $19.4\%$)} of \ldrh on datasets (\mopsi, \sercar, \geolife), respectively.
For example, when $\epsilon = 40$ meters, \ie~$\epsilon_{sed} = 40$ meters for \ldrh, \grts and \citt, $\epsilon_{ped} = 40$ meters for \sitt and {$(\epsilon_{ped}, \epsilon_{sed}) = (0.5\times 40, 1.6\times 40)=(20, 64)$} meters for \bitt, the compression ratios of \ldrh, \grts, \bitt, \citt and \sitt are
{($10.9\%$, $33.7\%$, $13.6\%$), ($3.0\%$, $13.3\%$, $4.5\%$), {($3.7\%$, $12.7\%$, $5.0\%$)}, ($3.0\%$, $13.2\%$, $4.4\%$) and ($2.1\%$, $6.1\%$, $2.6\%$)} on  {datasets (\mopsi, \sercar, \geolife)}, respectively. 

\ni (3) Datasets have impacts on compression ratios, \ie~datasets with higher sampling rates usually have better performance in terms of compression ratio.
	


\stitle{Message ratios.} We then report and and analyze the message ratios from Figure~\ref{fig:total-message}.

%To evaluate the impacts of distance metrics and error bounds on messages of \citt, \sitt and \bitt vs. \ldrh and \grts, we varied the error bound (either $\epsilon_{sed}$ or $\epsilon_{ped}$) from $10$ meters to $200$ meters on the entire three datasets, respectively. 

\ni (1) When increasing $\epsilon_{sed}$ and $\epsilon_{ped}$, the number of messages of all these algorithms decrease on all datasets.

\ni (2) The message numbers from the largest to the smallest are \ldrh, \grts, \citt, \bitt and \sitt. Among them, \ldrh has the largest messages because it has the worst compression ratios (thus it produces many \emph{position-velocity-messages}), and \sitt has the least messages because it has the best compression ratios (corresponding to the least \emph{position-velocity-messages}) and at the same time it seldom updates velocities during the process of a sub-trajectory (thus it only sends a small amount of \emph{velocity-messages}).

%(all of them are \emph{position-velocity-messages})
%\ni (2) Since \ldrh updates the position information every time the velocity is updated, the percentage of \emph{velocity-messages} is always $50\%$ of all messages.

\ni (3) \citt has a medium amount of messages (including \emph{position-velocity-messages} and \emph{velocity-messages}) and \bitt is between \citt and \sitt in this test.


\ni (4) \grts has more messages than \citt. Recall that \grts has \emph{position-messages} and \emph{position-velocity-messages}, while \citt has \emph{velocity-messages} and \emph{position-velocity-messages}. From Figure \ref{fig:compression-ratio} we know that \grts has a similar compression ratio as \citt, meaning \grts has the similar number of \emph{position-messages} as the \emph{position-velocity-messages} of \citt. Besides, the  number of \emph{position-velocity-messages} of \grts ({on average $(61.81\%, 59.28\%, 61.85\%)$} of the total messages \wrt datasets (\mopsi, \sercar, \geolife), respectively) is usually a bit larger than the \emph{velocity-messages} of \citt (on average $(60.41\%, 57.31\%, 52.80\%)$ of the total messages \wrt datasets (\mopsi, \sercar, \geolife), respectively). As a result, \grts has more total messages than \citt.

%This is because \grts only updates the velocities messages when the buffer is cleared, and mainly transmits position information.




\stitle{Errors.} We next report and analyze the max and average errors from Figures~\ref{fig:sed-error} and \ref{fig:ped-error} and Table \todo{X}.

%The results are reported in Figures~\ref{fig:sed-error} and Figure~\ref{fig:ped-error}.
\ni (1) Average errors increase with the increase of $\epsilon_{sed}$ and $\epsilon_{ped}$.

\ni (2) The average \sed errors of these algorithms from the largest to the smallest are \sitt, \grts (\citt) and \ldrh. Among them, the average \sed error of \citt is very close to \grts. \todo{\bitt}
The average \sed errors of algorithms \citt and \sitt are on average
($412.7\%$, $222.8\%$, $350.1\%$)
and ($842.0\%$, $857.5\%$, $1452.5\%$)
of \ldrh and ($103.1\%$, $98.4\%$, $100.5\%$) and
($209.9\%$, $442.9\%$, $414.0\%$)
of \grts on datasets (\mopsi, \sercar, \geolife), respectively.

\ni (3) The average \ped errors of these algorithms from the largest to the smallest are \sitt, \grts (\citt) and \ldrh. Among them, the average \ped errors of \citt is very close to \grts. \todo{\bitt}
The average \ped errors of algorithms \citt and \sitt are on average
($452.2\%$, $278.1\%$, $388.9\%$)
and ($550.5\%$, $517.0\%$, $589.0\%$)
of \ldrh and ($102.0\%$, $98.3\%$, $99.7\%$) and
($123.9\%$, $187.0\%$, $150.9\%$)
of \grts on datasets (\mopsi, \sercar, \geolife), respectively.




%%%%%%%%%%%%%%%%% running time
%In this part of experiments, we compare the running time of our algorithms \citt, \sitt and \bitt with \ldrh and \grts.
%The results are reported in Figure~\ref{fig:running-time}. 
\stitle{Running time.} We finally report and analyze the running time.
Since the running time of \grts is hundreds of times slower than other algorithms, it is not shown in Figure~\ref{fig:running-time}.

\ni (1) The error bound $\epsilon$ has few impacts on running time of \citt, \sitt, \bitt and \ldrh.

\ni (2) The running time of \bitt is approximately the sum of \citt and \sitt, because it combines the logic of \citt and \sitt.

\ni (3) The running time of these algorithms from the largest to the smallest are \grts, \bitt, \citt, \sitt and \ldrh on all datasets.
The average running time of algorithms \citt and \sitt is on average
($378.8\%$, $363.2\%$, $296.4\%$)
and ($331.5\%$, $294.1\%$, $256.4\%$)
of \ldrh and ($0.607\%$, $21.7\%$, $0.704\%$) and
($0.529\%$, $18.1\%$, $0.623\%$)
of \grts on datasets (\mopsi, \sercar, \geolife), respectively.



\subsubsection{Summary.} % and discuss
From these tests we find the followings.

\sstab\emph{(1) Compression ratios}. The optimal \sitt algorithm has the best compression ratios among all the algorithms. Algorithm \bitt and \citt are comparable with \grts.
They are all better than \ldrh.

\sstab\emph{(2) Message ratios}. The message numbers from the largest to the smallest are \ldrh, \grts, \citt, \bitt and \sitt.

\sstab\emph{(3) Average errors}. The average errors of these algorithms from the largest to the smallest are \sitt, \grts, \citt and \ldrh. Among them, the average error of \citt is very close to \grts.

\sstab\emph{(4) Running time}. Algorithm \ldrh is the fastest and \grts is the slowest. Moreover, the running time of \bitt is approximately the sum of \citt and \sitt.

In a conclusion, \ldrh runs the fastest and has the lowest average errors at a price of the poorest compression and message ratio. \sitt outperforms \grts in every metrics except average errors. \citt outperforms \grts in message ratio and running time, and is comparable with \grts in compression ratio and average errors. \bitt is comparable with \citt except that it has smaller average errors and longer running time.
\stitle{Related work}. We summarize related work as follows.
%
%Scholarly article ranking

Scholarly article ranking has shifted from citation count analysis~\cite{Garfield471,Hirsch15112005} to graph analysis~\cite{ChenXMR07,Zhou07-CoRank,Jiang12-MRank,Liang16AAAI,Li08TSRanking,Wang13AAAI,WalkerXKM07,sayyadi09,
Wang16TIST,Ng11KDD}.
Based on the information used, these methods are divided into four categories: (a) using the citation information only~\cite{Garfield471,Hirsch15112005,ChenXMR07,Ng11KDD}, (b) using the citation and temporal information~\cite{Li08TSRanking,WalkerXKM07}, (c) using the citation information and other heterogeneous information, \eg authors and venues of articles~\cite{Zhou07-CoRank,Jiang12-MRank,Liang16AAAI}, and (d) combining the citation, temporal and other heterogeneous information~\cite{sayyadi09,Wang16TIST,Wang13AAAI}.
Our work belongs to the last category aiming at fully employing information available for scholarly article ranking.


%\stitle{PageRank\&weighted PageRank algorithms}.

%PageRank \cite{Brin98:PageRank} and its extensions have been extensively used for citation analyses \cite{Waltman2014}. While PageRank equally propagates scores along outlinks, Weighted PageRank \cite{Xing04:WPR} extends PageRank by distributing scores based on the popularity of pages. Different from previous work, the Time-Weighted PageRank proposed in this work discriminately propagates scores in terms of citation statistics.

PageRank \cite{Brin98:PageRank} and its extensions have been extensively used for citation analyses \cite{Waltman2014}. While PageRank equally propagates scores along outlinks, Weighted PageRank extends PageRank by distributing scores based on certain criteria such as popularity of pages~\cite{Xing04:WPR} or authority of authors~\cite{Ding11}. Different from previous work, the Time-Weighted PageRank proposed in this work discriminately propagates scores in terms of citation statistics.






%\stitle{Dynamic algorithms}.

Dynamic algorithms have proven useful for various tasks by avoiding computing from scratch~\cite{RamalingamR93}.
% and only recomputing those affected by updates
%Dynamic algorithms have proven useful for graph analysis tasks, \eg incremental graph pattern matching~\cite{FanWW13} and  incremental simrank computation~\cite{YuLZ14}.
To our knowledge, little concern has been paid to dynamic scholarly article ranking except that~\cite{GhoshKHLL11} uses PageRank in dynamic citation networks. However, its solution is based on a strong and impractical assumption that there are no citations between articles in the same years.
Further, although there exist several studies on incremental PageRank computation~\cite{DesikanPSK05,AbiteboulPC03,WuR09} and on incremental PageRank approximation \cite{BahmaniCG10,BahmaniKMU12}, they are not designed for scholarly article ranking.
%
Different from previous work, we study scholarly article ranking in a dynamic environment in terms of
the citation characteristics of scholarly articles, which has never been exploited before.

%Our approach only makes the assumption that there are no mutual references within the citation network, which, we admit, violates xx\% of total citations on \magdata, and is significantly different (yy\% on \magdata) from~\cite{GhoshKHLL11}.  - move to Section 3

Ensemble methods use multiple learners to obtain better performance than could be obtained from a constituent learner alone~\cite{zhihua-book}.
%In this work, we leverage ensembles to produce better and robust results for scholarly article ranking~\cite{zhihua-book,wsdmcup,DuanAMHH16}.
In this work, we leverage  importance assembling  to produce better and robust results for scholarly article ranking~\cite{zhihua-book,wsdmcup,DuanAMHH16}.

\vspace{-1ex}
%%%%%%%%%%%%%%%%%%%%%%%%%%%%%%%%%%%%%%%%%%%%%%%%%%%%%%%%%%%%%%%%%%%%%%%%%%%%%%
\section{Conclusions}
%%%%%%%%%%%%%%%%%%%%%%%%%%%%%%%%%%%%%%%%%%%%%%%%%%%%%%%%%%%%%%%%%%%%%%%%%%%%%%

We have evaluated the state-of-the-art \lsa algorithms for trajectory compression, including \emph{both the optimal and the sub-optimal methods that use either \ped or \sed}. 
Using a variety of real trajectory datasets, we evaluated the performance of each technique.% in terms of its processing time, compression ratio and average error.
Our experimental results show that 
(1) the output sizes of algorithms using \sed are approximate $2$ times of using \ped, 
(2) the output sizes of sub-optimal algorithms are $130\%$--$160\%$ of the optimal algorithms, and 
(3) the one-pass algorithms \siped and \operb and \cised are tens of times faster than the batch algorithms and \textcolor{red}{$xxx$} times faster than online algorithms, while they still have comparable compression ratios with batch algorithms. Hence, they are more suitable for resource constraint mobile devices.
%\section*{{Appendix:  Proofs}}





\stitle{Proof of Theorem~\ref{theo-ldrh-cised}}:
\todo{1. $\vv{v}$ in 3D space; 2...}

If $|P_{s+i}P'_{s+i}|\le \epsilon/2$ for each $i \in [1,k]$, then $\vv{v}$ must live in the common intersection of half-$\epsilon$ cones $\bigsqcap_{i=1}^{k}$\cone{(P_s, P_{s+i}, \epsilon/2)}, where $P'_{s+i}$ is the synchronized point of $P_{s+i}$ \wrt velocity $\vv{v}$.
\eop

\stitle{Proof of Theorem~\ref{theo-full-cone}}:
\todo.
\eop

\stitle{Proof of Theorem~\ref{theo-cone-vs}}:
\todo.
\eop

\stitle{Proof of Theorem~\ref{theo-half-sector}}:
Trajectory tracking is the combination of trajectory simplification and position tracking.
%
(1) Trajectory simplification: It can be simplified in strip-like areas as shown in Section \ref{sec:sector-in-simp};
%
(2) Position tracking: If it can be represented by a line segment by the intersection of sectors, then there is sure a $\vv{v}$ living in the common intersection of sectors, \eg $\vv{v}$ on $P_sP_{s+i}$, such that it is applicable to track the position in strip areas.
%
Combine (1) and (2), we have the conclusion.
\eop

\stitle{Proof of Theorem~\ref{theo-full-sector}}:
\todo.
\eop

\stitle{Proof of Theorem~\ref{theo-sector-vs}}:
\todo.
\eop

\stitle{Proof of Theorem~\ref{theo-binary}}:
\todo.
1. traj simplification by cones and sectors.
2. position tracking: one velocity, from the intersection of cones.
\eop




\eat{%%%%%%%%%%%%%%%%
	\section*{Acknowledgments}
	This work is supported in part by NSFC (U1636210), NSFC ({61421003}) and SKLSDE (2020ZX-31).
}%%%%%%%%%%%%%%%%%%

\balance
\bibliographystyle{ACM-Reference-Format}
\bibliography{ref-traj-simp}

\end{document}
