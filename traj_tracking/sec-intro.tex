
%%% Local Variables:
%%% mode: latex
%%% TeX-master: "gis18"
%%% End:

\section{introduction}
\label{sec-intro}

moving object and global location service

position tracking:LDR
come from..., ldr is a agreement between moving object () and mod (), initial position and velocity, ...

trajectory tracking: 
traj tracking = position tracking + line simplification, where line simplification is ...
LDRH is the first work that find LDR with a small modification can be used to both track the position of moving objects and simplify the trajectories; GRTS formally defines the notion of traj tracking, and present a framework GRTS that has a better compression ratio than LDRH. GRTS ... separates simplification operation and tracking operation; simplify in buffer; track using ldr.


\stitle{Motivations.}

tracking moving objects in circles, limitations.

trajectory tracking = position tracking + trajectory compression/simplification, and trajectory compression supports three distance metrics, ped, sed and dad, while position tracking only uses sed.

the emerging of one pass trajectory simplification. in grts, one pass position tracking + one pass trajectory simplification = one pass and effective trajectory tracking algorithm? yes, but it is not good. co-design, like LDRH, yet more effective.

\stitle{Contributions.}
To the end, we proposed a novel 

a way that customize region by sed, plus ped and/or dad. and implement it in position tracking LDR and trajectory tracking framework GRTS. advantage...

a one-pass trajectory tracking algorithm supporting sed, ped and dad, by a combination cone intersection, sector intersection and interval intersection, \ie co-design of position tracking and trajectory simplification, effective and low time and space complexity, suitable running in resource constraint devices.

exper

\stitle{{Organization}}.
The remainder of the paper is organized as follows:
Section \ref{sec-pre} introduces the basic concepts and the basic HMM method,
Section \ref{sec-method} presents our trajectory simplification aware map-matching method,
Section \ref{sec-exp} reports the experimental results of these methods, followed by related works in Section \ref{sec-related} and conclusion in Section \ref{sec-conclusion}.



