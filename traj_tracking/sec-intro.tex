
%%% Local Variables:
%%% mode: latex
%%% TeX-master: "gis18"
%%% End:

\section{introduction}
\label{sec-intro}


\textit{Trajectory tracking} \cite{Lange:Tracking} is a combination of two fundamental technologies of moving objects databases (MOD), \textit{position tracking} \cite{Wolfson:PositionTracking,Leonhardi:Comparison} and \textit{trajectory simplification} \cite{Lin:Cised,Zhang:Evaluation}, in one routine, where \textit{position tracking} is an approach that lets the MOD server effectively and efficiently know the current position of a moving object, that is, it achieves the desired accuracy of the location information on the server by transmitting as few messages as possible \cite{Leonhardi:Comparison}. Linear dead reckoning (\ldr) \cite{Wolfson:PositionTracking} is such a widely used position tracking method, which is essentially an agreement between a moving object and a MOD server such that the server could infer the current, excepted position of the moving object whose distance to the actual position of the object is bounded by a user specified threshold;
%
and \textit{trajectory simplification} is to approximate a fine trajectory with a coarse one (whose corresponding data points are a subset of the original one), such that the size of the trajectory is reduced under a constrain that the maximum distance of the former to the latter is bounded by a user specified threshold \cite{Lin:Cised,Zhang:Evaluation}. 
%
Position tracking and trajectory simplification share some common target and strategy, \ie reduce the number of messages or the size of trajectory data by discarding some position information that seems not that important. Hence, researchers are trying to combine them in one routine, namely trajectory tracking, and make it suitable to run in resource constraint devices.

The authors of \cite{Trajcevski:LDRH} find that the position tracking algorithm \ldr with some small modifications is applicable to both track the positions of a moving object and simplify the trajectory built of these positions. The modified \ldr,  called \ldrh in \cite{Lange:Tracking}, is the first trajectory tracking algorithm that combines position tracking and trajectory simplification into one consistent process. It is concise and efficient, and is suitable for resource-constraint mobile devices. However, it suffers in effectiveness in terms of compression ratio compared with other trajectory simplification algorithms. %, due to the nature of \ldr. 
%
Then, a framework, named the generic remote trajectory simplification (\grts) \cite{Lange:GRTS,Lange:Tracking}, is developed to improve the effectiveness of trajectory tracking. \grts, retrograding to some extent, separates position tracking and trajectory simplification into two sub-processes, where the positions of a moving object are also tracked by \ldr and are temporarily saved in a buffer, then they are simplified by some third-party line simplification algorithm. \grts is more effective than \ldrh at an expense of less conciseness and efficiency.
%



\stitle{Motivations}. Consider the deployment environment and the varied application requirements of trajectory tracking, the current works, \ie~\ldrh and \grts, are far to be sufficient. Firstly, trajectory track algorithms are supposed to be deployed in resource-constraint mobile devices, thus, besides good performance of efficiency and effectiveness, they should also be simple and light, \ie having low time and space complexities, otherwise, they are not suitable to run in those mobile devices. In response to these characters, \ldrh is light, simple and efficient, but not effective; and \grts is effective, but not efficient or light. That is, neither of them is good enough for trajectory tracking in those devices.
%The emerging of one pass trajectory simplification algorithms. These algorithms can be integrated into grts, however, it is not a natural way to implement a one-pass trajectory tracking algorithm like this way. Acutually, one pass position tracking + one pass trajectory simplification = one pass and effective trajectory tracking algorithm......co-design, like LDRH, yet more effective.

Secondly, the current works only track a moving object in a circular area, \ie the moving object is supposed to locate in a circle taking the expected position of the object as its center (see Figure \ref{fig:areas}-(1)). However, in practical, there is a need of tracking a moving object in other areas, such as strip or rectangle-like areas, \eg a school boy is expected on a straight road home that his perpendicular deviation to the road is strict restricted, while his horizontal deviation on the road is totally unrestricted (meaning tracking him in a strip area as shown in Figure \ref{fig:areas}-(2)) or is not such strict restricted (meaning tracking him in a floating rectangle-like area as shown in Figure \ref{fig:areas}-(3)). The current position and/or trajectory tracking algorithms are fail to satisfy these varied requirements.



\begin{figure}[tb!]
	\centering
	\includegraphics[scale=1.0]{Figures/Fig-Areas.png}\vspace{-1ex}
	%\caption{\small A trajectory is simplified by algorithm \dpa using distance metrics \ped, \sed and \dad, respectively.}
	\caption{\small  Trajectory tracking in (1) a circular area, (2) a strip area, and (3) a rectangle-like area. Where $P_s$ is the start position of the sub-trajectory and $P'$ is the expected position of the moving object whose actually position at that time is $P$.}
	\vspace{-1ex}
	\label{fig:areas}
\end{figure}

\stitle{{Contributions}.}
To the end, we explore ways to track a moving object in circular, strip and rectangle-like areas, and design three novel one-pass trajectory tracking algorithms that effectively and efficiently run in those areas, respectively. 

\ni (1) We develop a one-pass trajectory tracking algorithm \citt based on the intersection of spatio-temporal cones that tracks a moving object in a circular area. \citt is as concise and efficient as \ldrh, and as effective as \grts, that is, it outperforms them (Section \ref{sec:circle}).

\ni (2) We develop a one-pass trajectory tracking algorithm \sitt based on the intersection of sectors that tracks a moving object in a strip area. \sitt is the first algorithm that tracks a moving in a strip area, and is as concise and efficient as \citt (Section \ref{sec:strip}). %, and has a better compression ratio than \citt with a cost of a poorer accuracy.

\ni (3) We provide a simple way to design a rectangle-like shape, and develop a one-pass trajectory tracking algorithm \bitt based on the intersections of sectors and spatio-temporal cones that tracks a moving object in such an area. \bitt is the first algorithm that tracks a moving object in a rectangle-like area, and it takes the advantages of both \citt and \sitt (Section \ref{sec:rectangle}). 

\ni (4) Using three real-life trajectory datasets (ServiceCar, GeoLife, Mopsi), we finally conduct an extensive experimental study that compares our methods \citt, \sitt and \bitt with  representative trajectory tracking algorithms \ldrh (the first and the most efficient) and \grts (the most effective). The experimental results show that our approaches are both efficient and effective that outperform \ldrh and \grts, and are feasible to track a moving object in circular, strip and rectangle-like areas (Section \ref{sec-exp}).

%\eat{%%%%%%%%%%%%%%%%
\stitle{{Organization}}.
The remainder of the paper is organized as follows:
Section \ref{sec-pre} introduces the basic concepts and the \ldr and \ldrh approaches,
Sections \ref{sec:circle}, \ref{sec:strip} and \ref{sec:rectangle} present three one-pass algorithms that track in circular, strip and rectangle-like areas, respectively,
Section \ref{sec-exp} reports the experimental results of these methods, followed by related works in Section \ref{sec-related} and conclusion in Section \ref{sec-conclusion}.
All proofs are provided in the Appendix.
%}%%%%%%%%%%%%%%%%eat


