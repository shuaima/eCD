
%%% Local Variables:
%%% mode: latex
%%% TeX-master: "gis18"
%%% End:

\section{introduction}
\label{sec-intro}

\eat{
The moving object and the wide use of global location service.
Tracking moving objects = position tracking and/or trajectory tracking.
1. position tracking.
comes from...
ldr is an agreement between moving objects (MOs) and moving objects databases (MODs), initial position and velocity, ...
}

Trajectory tracking \cite{Lange:Tracking} is a combination of position tracking \cite{Wolfson:PositionTracking} and trajectory simplification in one routine, where position tracking is an agreement between a moving object and a MOD server such that the MOD server can infer the current position of the object with an deviation to its actual position less than a user specified threshold,
and trajectory simplification is to approximate a fine trajectory with a coarse one (whose corresponding data points are a subset of the original one), such that the size of the trajectory is clearly reduced under a constrain that the maximum distance of the former to the latter is also bounded by a user specified threshold.
\todo{importance.}

LDRH is the first work that find LDR with a small modification can be used to both track the position of moving objects and simplify the trajectories. LDRH combines position tracking and line simplification into one consistent process. Limitation of LDRH: poor compression ratios. 
GRTS formally defines the notion of traj tracking, and presents a framework GRTS that has a better compression ratio than LDRH, by separates traj tracking process in to two sub-processes, simplification operation and position tracking operation. GRTS tracks positions of moving objects by ldr, and utilizes a third party algorithm to implement line simplification that simplify data points saved in a buffer.
\todo{traj tracking algorithm should be light, low time and space complexities that suitable to run on mobile devices}

\stitle{Motivations.}

1. The current works only compress trajectory or track moving objects in circular areas. vs application requirements that need to track moving objects in other areas, such as strips, or combinations of circulars and strips.
The necessity and possibility to track in circular and strip areas.

the distance metric used in tracking ``the distance between the expecting position and the actual position" ... it is SED in trajectory simplification. 

scope: It is  a CIRCLE around the expecting position. fig.

Limitations.

we can make these regions more flexible as soon as we include another distance metric, PED, in trajectory simplification with a little modification.

scope: fig.

advantages and applications.

besides, dad... tracking the direction deviation of a moving object.

%Trajectory tracking = position tracking + trajectory compression/simplification, and trajectory compression supports three distance metrics, ped, sed and dad, while position tracking only uses sed.

2. The emerging of one pass trajectory simplification algorithms. These algorithms can be integrated into grts, one pass position tracking + one pass trajectory simplification = one pass and effective trajectory tracking algorithm, however, it is not good... co-design, like LDRH, yet more effective...... 
The necessity and possibility to design efficient and effective track algorithm, at the same time supports the track in circular and strip areas.

\stitle{Contributions.}
To the end, we proposed a novel 

1. a way that customize region by sed, plus ped and/or dad. and implement it in position tracking LDR and trajectory tracking framework GRTS. advantage...

2. a one-pass trajectory tracking algorithm supporting sed, ped and dad, by a combination cone intersection, sector intersection and interval intersection, \ie co-design of position tracking and trajectory simplification, effective and low time and space complexity, suitable running in resource constraint devices.

3. experiments

\stitle{{Organization}}.
The remainder of the paper is organized as follows:
Section \ref{sec-pre} introduces the basic concepts and the basic HMM method,
Section \ref{sec-method} presents our trajectory simplification aware map-matching method,
Section \ref{sec-exp} reports the experimental results of these methods, followed by related works in Section \ref{sec-related} and conclusion in Section \ref{sec-conclusion}.



