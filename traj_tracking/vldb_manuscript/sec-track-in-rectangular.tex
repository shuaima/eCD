
%%% Local Variables:
%%% mode: latex
%%% TeX-master: "www2019"
%%% End:

\section{Tracking in beams}
\label{sec:rectangle}
Currently, position tracking and trajectory tracking methods both use \sed as the distance metric to check data points and confirm the actual position of a moving object at time $t$ lives inside a circle around the excepting position of the moving object at that time.
{However, as mentioned in Section~\ref{sec-intro}, there is a need of tracking a moving object in other areas, \eg an infinite or finite beam.}
%Note that, though it is possible to track a moving object in a rectangular area, it is really hard to design such a shape without introducing a new distance metric other than the well-known metrics \sed and \ped, and at the same time develop an efficient algorithm for that (\todo{appendix}).
%Alternatively, this section defines a rectangle-like area that subtly makes use of \ped and \sed, and then develops such an effective and efficient trajectory tracking algorithm.
%
%, and (2) a circular or strip area is indeed a special case of a rectangle-like area.
%as in \cite{Lin:Dual}, 

\subsection{Building infinite and finite beams}

Just as a \emph{disc} is defined by \sed, a \emph{infinite beam} is defined by \ped. More specifically, if a moving object has a \ped not larger than a threshold $\epsilon_{ped}$ to a line $\mathcal{L}$, then its potential area of activity is inside an \emph{infinite beam} (also called ``parallel strip'' \cite{Chen:Space,Daescu:metric} in the field of computational geometry. If the line is replaced by a line segment, then this area is called ``tolerance-zone'' \cite{Imai:Optimal}) around the line.

For a \emph{finite beam}, things are getting a little more complicated as it cannot be defined by any single distance metric of \ped or \sed. 
Thus, we develop a new distance metric for it, \ie the binary Euclidean distances of \sed and \ped.

\eat{%%%%%%%%%%%%%%%%%%%%
\stitle{Rectangular area and binary Euclidean distances of \red and \ped (\bed (\red, \ped) in short)}. \bed (\red, \ped) is the combination of distance metrics \red and \ped, where the \red (Radial Euclidean Distance) of point $P$ \wrt line segment $\mathcal{L}$, $red(P, \mathcal{L})$, is the Euclidean distance from the perpendicular point $P^*$ of point $P$ \wrt~$\mathcal{L}$ to the synchronized point $P'$ of point $P$ \wrt~$\mathcal{L}$. For example, in Figure \ref{fig:concepts}, $red(P_4, \overline{P_0P_8})$ = $|P^*_4P'_4|$. Indeed, a \bed (\red, \ped) with error bounds of ($\epsilon_{red}$, $\epsilon_{ped}$) forms a rectangle whose width and height are $2\epsilon_{red}$ and $2\epsilon_{ped}$, respectively.
}%%%%%%%%%%%%%%%%%%%%%
%their error bounds are set separately to $\epsilon_{red}$ and $\epsilon_{ped}$;

\stitle{Binary Euclidean distances of \sed and \ped}, shortly \bed (\sed, \ped), is the combination of distance metrics \sed and \ped where their error bounds are separately set to $\epsilon_{sed}$ and $\epsilon_{ped}$, satisfying $\epsilon_{sed} > \epsilon_{ped}$. 

%, whose (1) end sides are circular arcs taking the synchronized point as it center and $\epsilon_{sed}$ as its radius, and (2) . \todo{ped, floating, line segment}

%(1) if $\epsilon_{ped} \ge \epsilon_{sed}$, then it falls back to the \sed, \ie tracking in a circular area, as shown in {Figure~\ref{fig:areas}-(1)};
%
%(2) if $\epsilon_{ped} << \epsilon_{sed}$, then its effect is actually approximate to a \ped, which is familiar in trajectory simplification, but not seemed in position and trajectory tracking, that always ensures that the original data points live in strip-like areas built around the simplified trajectory, as shown in {Figure~\ref{fig:areas}-(2). Otherwise, 
%
%(3) it is the double effects of \ped and \sed, and forms a rectangle-like shape whose short sides are replaced by circular arcs of \sed as shown in {Figure~\ref{fig:areas}-(3)}. 

As shown in {Figure~\ref{fig:areas}-(3)}, given a start point $P_s$ and a velocity $\vv{v}$, the \emph{finite beam} at time $t$ ($t>P_s.t$) is a shape built from an \emph{infinite beam} defined by \ped coupling with a floating \emph{disc} defined by \sed.
Because \ped is a special case of \bed (where $\epsilon_{sed}=\infty$), the \emph{infinite beam} is also a special case of the \emph{finite beam}. Moreover, the shape of a \emph{finite beam} is controlled by two independent parameters, $\epsilon_{ped}$ and $\epsilon_{sed}$, thus, by carefully setting them, we not only build such areas that satisfy the needs of varied applications but also balance two performance metrics, the compression ratios and compression errors, of trajectory simplification/tracking. 
%For the same example of ``the school boy on his way home'' mentioned in Section \ref{sec-intro}, if we only use \ped with a fixed ``$\epsilon$'', then we get a simplified trajectory having a good compression ratio and a unbounded query error; if we only use \sed with the same ``$\epsilon$'', then we get a poorer compression ratio and a bounded query error. However, if we combine them, \ie~use the \bed, then we could get a medium compression ratio and a bounded query error.



\eat{ %%%%%%%%%%%%%%
Given a sequence of points $[P_{s}, P_{s+1}, \ldots, P_{s+k}]$ and an error bound $\epsilon$,
for the start data point $P_s$, any point $P_{s+i}$ and $|\overline{P_sP_{s+i}}|>\epsilon$ ($i\in[1, k]$), there are two directed lines $\overline{P_sP^u_{s+i}}$ and $\overline{P_sP^l_{s+i}}$ such that $ped(P_{s+i}, \overline{P_sP^u_{s+i}})$ $=$ $ped(P_{s+i}, \overline{P_sP^l_{s+i}}) = \epsilon$ and either ($\overline{P_sP^l_{s+i}}.\theta < \overline{P_sP^u_{s+i}}.\theta ~and~\overline{P_sP^u_{s+i}}.\theta - \overline{P_sP^l_{s+i}}.\theta <\pi$) or ($\overline{P_sP^l_{s+i}}.\theta > \overline{P_sP^u_{s+i}}.\theta ~and~ \overline{P_sP^u_{s+i}}.\theta - \overline{P_sP^l_{s+i}}.\theta < -\pi)$. Indeed, they form a \emph{sector} \sector{(P_s, P_{s+i}, \epsilon)} that takes $P_s$ as the center point and $\overline{P_sP^u_{s+i}}$ and $\overline{P_sP^l_{s+i}}$ as the borderlines (Figure~\ref{fig:sleeve}-(1)).
%
Then, like the checking of the common intersection of spatio-temporal cones in \cised, these sector-based algorithms check the common intersection of sectors, \ie $\bigsqcap_{i=1}^{k}$\sector{(P_s, P_{s+i}, \epsilon)} $\ne \{P_s\}$ \cite{Williams:Longest, Sklansky:Cone,Zhao:Sleeve}, to find out whether the sub-trajectory can be represented by a line segment $\overline{P_sQ}$ \wrt error bound $\epsilon$, where $Q$ is a point that may not belong to the sub-trajectory. However, if $Q$ must be $P_{s+k}$, the last point of the sub-trajectory, \eg $P_3$ in Figure~\ref{fig:sleeve}-(2), then (1) the full-$\epsilon$ sectors should be replaced by half-$\epsilon$ sectors, and (2) $P_{s+k}$ should be one of the points having the furthest distances to $P_s$, \ie $|P_sP_{s+k}| \ge l_{m}$, where $l_{m}=max\{|P_sP_{s+i}|\}$~for each $i \in (0, k)$. 
}%%%%%%%%%%%%%%%%%%%%%%eaten



\subsection{Basic tracking approaches}
By using \bed, we are able to track the position of a moving object and simplify its trajectory in a infinite or finite beam, where the position tracking algorithm is the modified \ldr~\cite{Wolfson:PositionTracking} (mLDR in short) and the trajectory simplification algorithm is the modified open/sliding window approach \cite{Meratnia:Spatiotemporal} (mOPW in short), in which the \sed metric used in these algorithms is replaced by \ped or \bed (\sed, \ped). 
%
The mLDR and mOPW could also be integrated into \grts~\cite{Lange:GRTS}, \ie~\grts using \ped or \bed (mGRTS in short), such that we can track a trajectory in a infinite or finite beam in one framework.
%
%The algorithms of mLDR, mOPW and mGRTS please refer to \cite{Wolfson:PositionTracking, Meratnia:Spatiotemporal, Lange:GRTS}.
%
Due to the effectiveness beneficial of mOPW, the mGRTS should also be effective in terms of compression ratio. At the same time, it is still not efficient or space-saving enough. 


\begin{figure}[tb!]
	\centering
	\includegraphics[scale=1.0]{figures/Fig-Sleeve.png}
	\vspace{-2ex}
	\caption{\small Examples of sectors and their intersection.}
	\vspace{-2ex}
	\label{fig:sleeve}
\end{figure}

\subsection{Efficient tracking approaches}
\label{sec:track_cone_sector}

%\ped is familiar in trajectory simplification, where an error-bounded algorithm using \ped always ensures that the original data points live in strip-like areas built from the simplified trajectory. 
%Though \ped is not seemed in position and trajectory tracking, indeed, the sector built from \ped is applicable to track a moving object in a strip area, like the spatio-temporal cone built from \sed that is used to track trajectory in a circular area.

We next explore effective and efficient trajectory tracking algorithms for infinite and finite beams.
%
Similar as the spatio-temporal cones built from \sed are used in efficiently simplifying trajectory in circular areas, an efficient mechanism based on sectors built from \ped is applicable to simplify trajectory in \emph{infinite beams}. As shown in Figure \ref{fig:sleeve}, a \emph{sector} \sector{(P_s, P_{s+i}, \epsilon)} is largely a simplified version of a spatio-temporal cone \cone{(P_s, P_{s+i}, \epsilon)} projected on an $x$--$y$ 2D space that the temporal information is ignored, that converts the \ped distance tolerance into an angle tolerance for efficiently checking the successive data points. 
Then \textit{sector intersection} \cite{Williams:Longest, Sklansky:Cone, Dunham:Cone, Zhao:Sleeve}, originally developed in fields of computational geometry and cartography to efficient simplifies the border lines of a geometric or cartographic shape in digital format, is an efficient and effective way to simplify trajectories as pointed out in \cite{Lin:Cised}.
We next demonstrate that it is also applicable to track trajectory in \emph{infinite beams}.


\begin{proposition}
	\label{theo-half-sector}
	Given a sub-trajectory $[P_s,...,P_{s+k}]$ and an error bound $\epsilon$, the trajectory can be tracked in an infinite beam by an approach based on the intersection of sectors.
\end{proposition}

\begin{proof}
Trajectory tracking is the combination of trajectory simplification and position tracking.
%
(1) Trajectory simplification: It can be simplified in infinite beams as illustrated above;
%
(2) Position tracking: If it can be represented by a line segment by the intersection of sectors, then there is sure a $\vv{v}$ living in the common intersection of sectors, \eg $\vv{v}$ on $P_sP_{s+i}$, such that it is applicable to track the position in infinite beams.
%
Combine (1) and (2), we have the conclusion.
\end{proof}

Because the sector intersection approach of \cite{Williams:Longest, Sklansky:Cone,Zhao:Sleeve} uses a half-$\epsilon$ sector (Figure \ref{fig:sleeve}-(2)) that may limit the performance of compression ratio, we also extend it to the full-$\epsilon$ sector (Figure \ref{fig:sleeve}-(1)) for better performance.

\begin{proposition}
	\label{theo-full-sector}
	Given a sub-trajectory $[P_s,...,P_{s+k}]$ and an error bound $\epsilon$, $ped(P_{s+i}, \overline{P_sP_{s+k}})\le \epsilon$ for each $i \in [1,k]$ if line segment $\overline{P_sP_{s+k}}$ passes through $\bigsqcap_{i=1}^{k-1}$\sector{(P_s, P_{s+i}, \epsilon)} - \{$P_s$\} ~and~ $|P_sP_{s+k}| \ge \sqrt{l_{m}^2 - \epsilon^2}$, where $l_{m} = max\{|P_sP_{s+i}|\}$ for each $i \in (0, k)$.
\end{proposition}

\begin{proof}
First, let $|P_sP_{s+k}| = \sqrt{l_m^2 - \epsilon^2}$. We draw an arc $\widehat{BD}$ taking $P_s$ as its center and $l_m$ as its radius, and two lines $\overline{AB}$ and $\overline{CD}$ paralleling and having a distance of $\epsilon$ to $\overline{P_sP_{s+k}}$, as shown in Figure \ref{fig:sectorinter}.
Because line segment $\overline{P_sP_{s+k}}$ passes through $\bigsqcap_{i=1}^{k-1}$\sector{(P_s, P_{s+j}, \epsilon)}- $\{P_s\}$ and $l_{m} = max\{|P_sP_{s+i}|\}$ for each $i \in (0, k)$, point $P_{s+i}$ must live in the area between lines $\overline{AB}$ and $\overline{CD}$, and in the inner side of arc $\widehat{BD}$.
(1) If $P_{s+i}$ is between $\overline{BD}$ and $\widehat{BD}$, then $ped(P_{s+i}, \overline{P_sP_{s+k}}) = |{P_{s+i}P_{s+k}}| < |{BP_{s+k}}| = \epsilon$; 
(2) If $P_{s+i}$ is in the left side of $\overline{BD}$, then  $ped(P_{s+i}, \overline{P_sP_{s+k}})$ is the perpendicular Euclidean distance from $P_{s+i}$ to the line segment $\overline{P_sP_{s+k}}$ that is also less than $\epsilon$. Combining (1) and (2) we have $ped(P_{s+i}, \overline{P_sP_{s+k}})< \epsilon$ for $|P_sP_{s+k}| = \sqrt{l_m^2 - \epsilon^2}$.
Next, it is easy to find that the conclusion remains the same if $|P_sP_{s+k}| > \sqrt{l_m^2 - \epsilon^2}$.
%then for each point $P_{s+i}$, $i\in (0,k)$, it has a Euclidean distance $d_i$ less than $\epsilon$ to the line that $\overline{P_sP_{s+k}}$ lives on. We also know that $\overline{P_sP_{s+k}}$ is not shorter than any other line segment, thus, the distance $d_i$ is indeed the \ped of $P_{s+i}$ to the line segment $\overline{P_sP_{s+k}}$. That is, $ped(P_{s+i}, P_sP_{s+k}) <\epsilon$ for each $i\in (0,k)$.
\end{proof}

\begin{figure}[tb!]
	\centering
	\includegraphics[scale=1.0]{figures/Fig-SectorInter.png}
	\vspace{-2ex}
	\caption{\small The intersection of full-$\epsilon$ sectors.  }
	\vspace{-1ex}
	\label{fig:sectorinter}
\end{figure}

\eat{%%%%%%%%%%%%%%%%%%%%%%%%%%%%%%%%%%
	\stitle{\textcolor{red}{Proof of Theorem~\ref{theo-full-sector}}}:
	If line segment $\overline{P_sP_{s+k}}$ passes through $\bigsqcap_{i=1}^{k-1}$\sector{(P_s, P_{s+j}, \epsilon)}- $\{P_s\}$, \ie it lives in all the preview sectors, then for each point $P_{s+i}$, $i\in (0,k)$, it has a Euclidean distance $d_i$ less than $\epsilon$ to the line that $\overline{P_sP_{s+k}}$ lives on. We also know that $\overline{P_sP_{s+k}}$ is not shorter than any other line segment, thus, the distance $d_i$ is indeed the \ped of $P_{s+i}$ to the line segment $\overline{P_sP_{s+k}}$. That is, $ped(P_{s+i}, P_sP_{s+k}) <\epsilon$ for each $i\in (0,k)$.
	\eop
	
	\stitle{Proof of Theorem~\ref{theo-sector-vs}}:
	(1) If $\bigsqcap_{i=1}^{k}$\sector{(P_s, P_{s+i}, \epsilon/2)} $\ne \{P_s\}$, then $\overline{P_sP_{s+k}}$ lives in the common intersection of the preview half-$\epsilon$ sectors. 
	Hence, it is sure living in the common intersection of the preview full-$\epsilon$ sectors, \ie it passes through $\bigsqcap_{i=1}^{k-1}$\sector{(P_s, P_{s+i}, \epsilon)} $- \{P_s\}$.
	%
	(2) If line segment $\overline{P_sP_{s+k}}$ passes through the common intersection $\bigsqcap_{i=1}^{k-1}$\sector{(P_s, P_{s+i}, \epsilon)} - \{$P_s$\}, then, given $i$ and $j$ ($i<j<k$), it is possible that $ped(P_{s+i}, \overline{P_sP_{s+k}})> \epsilon/2$ and $ped(P_{s+j}, \overline{P_sP_{s+k}})> \epsilon/2$ , meaning the intersection of \sector{(P_s,P_{s+i},\epsilon/2)} and \sector{(P_s,P_{s+j},\epsilon/2)} is $\{P_s\}$, and thus, $\bigsqcap_{i=1}^{k}$\sector{(P_s, P_{s+i}, \epsilon/2)} is also $\{P_s\}$.
	%
	Combine (1) and (2) we have the conclusion.
	\eop
}%%%%%%%%%%%%%%%%%%%%%%%%%%%%%%%%%%%%%

Proposition \ref{theo-full-sector} tells that the full-$\epsilon$ sector approach with constrains that (1) $\overline{P_sP_{s+k}}$ lives in the common intersection of the preview full-$\epsilon$ sectors and (2) $|P_sP_{s+k}|$ is longer than $\sqrt{l_{m}^2 -\epsilon^2}$, is applicable to simplify and track a moving object in an infinite beam. The full-$\epsilon$ sector and $|P_sP_{s+k}| \ge \sqrt{l_{m}^2 - \epsilon^2}$ are looser constraints than the half-$\epsilon$ sector and $|P_sP_{s+k}| \ge l_{m}$ used in \cite{Williams:Longest, Sklansky:Cone,Zhao:Sleeve}, hence, this new approach is sure of bringing a better compression ratio.
%\myblue{Is there an effective and efficient trajectory algorithm implementing \bed that tracks a moving object in a rectangle-like area? Theorem \ref{theo-binary} is the answer to this question.}
Next, by combining cones and sectors, we are able to efficiently track trajectory in \emph{finite beams}.

\eat{%%%%%%%%%%%%%%%%%%%
	\begin{theorem}
		\label{theo-sector-vs}
		Given a sub-trajectory $[P_s,...,P_{s+k}]$ and an error bound $\epsilon$, if $\bigsqcap_{i=1}^{k}$\sector{(P_s, P_{s+i}, \epsilon/2)} $\ne \{P_s\}$, then $\overline{P_sP_{s+k}}$ passes through $\bigsqcap_{i=1}^{k-1}$\sector{(P_s, P_{s+i}, \epsilon)}-$\{P_s\}$; and the opposite is not necessarily true.
	\end{theorem}
	
	
	Theorem \ref{theo-sector-vs} tells that the full-$\epsilon$ sector approach also brings a better effectiveness than the half-$\epsilon$ sector, hence it is the dominant way to develop trajectory simplification/tracking algorithms.
}%%%%%%%%%%%%%%%%%%%%%%%%

\begin{proposition}
	\label{theo-binary}
	Given a sub-trajectory $[P_s,...,P_{s+k}]$ and two error bounds $\epsilon_{sed}$ and $\epsilon_{ped}$ satisfying $\epsilon_{sed} > \epsilon_{ped}$, it can be tracked in finite beams by combining sectors and spatio-temporal cones.
\end{proposition}

\begin{proof}
Given a sub-trajectory $[P_s,...,P_{s+k}]$ and two error bounds $\epsilon_{sed}$ and $\epsilon_{ped}$ satisfying $\epsilon_{sed} > \epsilon_{ped}$, from Propositions~\ref{theo-ldrh-cised} and \ref{theo-half-sector}, we know this sub-trajectory can be tracked in a floating disc and an infinite beam, respectively.  
%
If we let $P_s$ be the same start point and $\vec{v}$ be the same velocity for each technique of sectors, spatio-temporal cones, and position tracking \wrt~\ped and \sed, such that the \emph{infinite beam} and the floating \emph{disc} \wrt the start point $P_s$ and the velocity $\vec{v}$, exactly form a floating \emph{finite beam}, then
(1) trajectory simplification: the sub-trajectory $[P_s,...,P_{s+k}]$ can be represented by $\overline{P_sP_{s+k}}$ as long as $\overline{P_sP_{s+k}}$ passes through the common intersection of $\bigsqcap_{i=1}^{k-1}$\cone{(P_s, P_{s+i}, \epsilon)} - \{$P_s$\} and the common intersection of $\bigsqcap_{i=1}^{k-1}$\sector{(P_s, P_{s+i}, \epsilon)} - \{$P_s$\}, and satisfying~ $|P_sP_{s+k}| \ge \sqrt{l_{m}^2 - \epsilon^2}$,
\ie this sub-trajectory can be simplified \wrt a floating finite beam by combining sectors and spatio-temporal cones.
(2) position tracking: if the sub-trajectory $[P_s,...,P_{s+k}]$ can be represented by line segment $\overline{P_sP_{s+k}}$ by the intersections of sectors and cones, then there is sure a velocity $\vv{v}$ living in the common intersections of cones and sectors, such that it is applicable to track the position of the object in the floating finite beam \wrt the $P_s$ and $\vec{v}$. 
Combine (1) and (2) we have the conclusion.
\end{proof}

%(1) trajectory simplification: Let $\overline{P_sP_{s+i}}, i\in (0,k],$ be the line segment representing sub-trajectory $[P_s,...,P_{s+i}]$, because $\epsilon_{sed} > \epsilon_{ped}$, these areas indeed form a finite beam, \ie this sub-trajectory can be simplified in finite beams by combining sectors and spatio-temporal cones.
%
%(2) position tracking: If the sub-trajectory $[P_s,...,P_{s+i}]$ can be represented by line segment $\overline{P_sP_{s+i}}$ by the intersections of sectors and cones, then there is sure a velocity $\vv{v}$ living in the common intersections of cones and sectors, such that it is applicable to track the position of the moving object in a finite beam \wrt the $P_s$ and $\vec{v}$. 


Note that, to track the positions and at the same time simplify the trajectory in a \emph{finite beam}, it is important to make sure that during the processing of a sub-trajectory $[P_s,...,P_{s+k}]$, there are the same start point $P_s$ and the same velocity $\vec{v}$ for each technique of spatio-temporal cones, sectors, and position tracking of \ped and \sed, such that an \emph{infinite beam} and a floating \emph{disc} \wrt start point $P_s$ and velocity $\vec{v}$, exactly form a floating \emph{finite beam}. This is the guideline to develop such a trajectory tracking algorithm. 



%%%%%%%%%%%%%%%%%%%%%%%%%%%%%%%%%%%%%%%%%%%%%%%%%%%%%%%%%%%%%%%%%%%%%%%%%%
% Algorithm: Traj tracking based on section intersection using full sectors.
\begin{figure}[tb!]   
	\begin{center}
		{\small
			\begin{minipage}{3.3in}
				\myhrule
				%\vspace{-1ex}
				\mat{0ex}{
					{\bf Algorithm}~\bitt $(\dddot{\mathcal{T}}[P_0,\ldots,P_n], ~\epsilon_{sed}, m, ~\epsilon_{ped})$\\
					%	\sstab
					\bcc \hspace{1ex}\= $P_s := P_0$; ~~~~$\mathcal{R}^*$ := \kw{getRPolygon}($P_s$, $P_{s+1}$, $\epsilon_{sed}$, $m$, $P_{s+1}.t$); \\
					\icc \hspace{1ex}\= $\mathcal{S}^*$ := \kw{getSector}($P_s$, $P_{s+1}$, $\epsilon_{ped}$); \\
					\icc \hspace{1ex}\= $|\vv{v}|:=\frac{|P_{s}P_{s+1}|}{P_{s+1}.t-P_s.t}$; ~~~~$\vv{v}.\theta:=\overline{P_{s}P_{s+1}}.\theta$;  \\
					\icc \hspace{1ex}\= update ($P_{s}, \vv{v}$); 	\\
					\icc \hspace{1ex}\= $l_{m} := |P_sP_{s+1}|$; ~~~~$i:= 2$;  	\\
					\icc \hspace{1ex}\= while $i \le n$ do \\
					\icc \>\hspace{3ex} if $\overline{P_sP_{i}}$ ~does not pass~ $\mathcal{R}^*~or~\mathcal{S}^*$, or $|P_sP_{i}| < \sqrt{l_{m}^2 - \epsilon^2}$ ~then \\ % // updates velocity and location \\
					\icc \>\hspace{7ex}    $P_s := P_{i-1}$; ~~~~$\mathcal{R}^*$ := $\emptyset$;~~~~$\mathcal{S}^*$ := $\emptyset$; ~~~~$l_{m} := 0$;\\
					\icc \>\hspace{7ex}    $|\vv{v}|:=\frac{|P_sP_{i}|}{P_{i}.t-P_s.t}$; ~~~~$\vv{v}.\theta:=\overline{P_{s}P_{i}}.\theta$;  \\
					\icc \>\hspace{7ex}    update ($P_{s}, \vv{v}$); 	\\
					\icc \>\hspace{3ex} else if $sed(P_i, \vv{v}) \ge \epsilon_{sed}$ ~or~ $ped(P_i, \vv{v}) \ge \epsilon_{ped}$ ~then  \\ %$\overline{P_sP_{i}}$ ~passes ~ $\mathcal{R}^*$ and $\mathcal{S}^*$, $|P_sP_{i}| > l_{m} - \epsilon$ \\ \hspace{9ex} ~and~
					\icc \>\hspace{7ex}    $|\vv{v}|:=\frac{|P_sP_{i}|}{P_{i}.t-P_s.t}$; ~~~~$\vv{v}.\theta:=\overline{P_sP_{i}}.\theta$; \\
					\icc \>\hspace{7ex}    update ($\vv{v}$); \\
					\icc \>\hspace{3ex} if $\mathcal{S}^*=\emptyset$ ~then~ $\mathcal{S}^*:=$ \kw{getSector}($P_s$, $P_{i}$, $\epsilon_{ped}$);\\
					\icc \>\hspace{7ex}     $\mathcal{R}^*:=$ \kw{getRPolygon}($P_s$, $P_{i}$, $\epsilon_{sed}$, $m$, $P_{s+1}.t$); \\
					\icc \>\hspace{3ex} else $\mathcal{S}^*$ := $\mathcal{S}^*\bigsqcap$ \kw{getSector}($P_s$, $P_{i}$, $\epsilon_{ped}$); \\
					\icc \>\hspace{7ex}     $\mathcal{R}^*:=\mathcal{R}^*\bigsqcap$ \kw{getRPolygon}($P_s$, $P_{i}$, $\epsilon_{sed}$, $m$, $P_{s+1}.t$);\\
					\icc \>\hspace{3ex} $l_{m} := \max\{|P_sP_{i}|, l_{m}\}$;  ~~~~$i$ := $i +1$;\\
					\icc \>\hspace{0ex} update ($P_{n}$); 
				}
				\vspace{-2ex}
				\myhrule
			\end{minipage}
		}
	\end{center}
	\vspace{-2ex}
	\caption{\small Trajectory tracking based on sector and cone.}
	\label{alg:bitt}
	\vspace{-2ex}
\end{figure}
%%%%%%%%%%%%%%%%%%%%%%%%%%%%%%%%%%%%%

\subsection{Algorithm}
We now present the algorithm of \underline{B}inary \underline{I}ntersection for \underline{T}rajectory \underline{T}racking (BITT) that tracks an object in a finite beam (see Figure~\ref{alg:bitt}). 
%
\bitt is the double checks of cone intersection and sector intersection for each point for the purpose of trajectory simplification, and the double checks of \sed and \ped distance deviations for position tracking. In addition to that, it uses a uniform velocity $\vv{v}$ for position tracking of both \sed and \ped such that either deviation of \ped or \sed distance will cause an update of velocity $\vv{v}$. 
%
\bitt is a position tracking algorithm as well as a trajectory simplification algorithm, and it ensures that any removed point is located in a finite beam around its expected position \wrt a velocity of position tracking or a line segment connecting two neighboring points of the simplified trajectory. 
%\eg $P_2$ is in the rectangle-like area around its synchronized point $P'_2$ \wrt line segment $\overline{P_0P_4}$
%
\bitt like \citt also reports two kinds of messages, \ie \emph{velocity-messages} $\vv{v}$ and \emph{position-velocity-messages} ($P_s$, $\vv{v}$), to the MOD server. 
Indeed, \bitt is the super version of \citt, \ie it falls back to \citt when $\epsilon_{sed} \le \epsilon_{ped}$.

\begin{example}
	Figure~\ref{fig:bitt} is a running example of \bitt. It takes as inputs the same trajectory as the above, the same $\epsilon_{sed}$ as Figure~\ref{fig:citt} and an $\epsilon_{ped}$ of half of $\epsilon_{sed}$. Because its $\epsilon_{sed}$ is the same as Figure~\ref{fig:citt}, its effectiveness is also the same as Figure~\ref{fig:citt}. For the purpose of clearness, we do not show those cones in the figure.
	%
	\bitt uses full-$\epsilon_{sed}$ cones and full-$\epsilon_{ped}$ sectors to simplify the trajectory. Initially, it sets the same start point and initial velocity as Figure~\ref{fig:citt}, 
	Then, (1) $P_3$ lives in the common intersection of the preview cones and sectors, and it has both \sed and \ped distances larger than $\epsilon_{sed}$ and $\epsilon_{ped}$, respectively, \ie $|P_3P'_3| \ge \epsilon_{sed}$ and $|P_3P^*_3| \ge \epsilon_{ped}$, thus, \bitt updates the velocity from $\vec{v_1}$ to $\vec{v_3}$ and the process goes on, (2) $P_5$ is outside of the common intersections of the preview cones and sectors, thus, $P_4$ serves as the new start point, and an update of $(P_4, \vec{v_5})$ is triggered, and (3) $P_7$ lives in the common intersections of the preview cones and sectors, and it has a \ped distance larger than $\epsilon_{ped}$, thus, \bitt updates the velocity from $\vec{v_5}$ to $\vec{v_7}$ (not shown) and the process goes on. Finally, it sends three points, $P_0, P_4$ and $P_8$, and four velocities, $\vec{v_1}$, $\vec{v_3}$, $\vec{v_5}$ and $\vec{v_7}$, to the MOD server. 
\end{example}

\begin{figure}[tb!]
	\centering
	\includegraphics[scale=0.88]{figures/Fig-BITT.png}
	\vspace{-2ex}
	\caption{\small A running example of trajectory tracking by \bitt. In this case, the spatio-temporal cones and their intersections are the same as Figure~\ref{fig:citt}, thus they are not shown here for clearness.  }
	\vspace{-2ex}
	\label{fig:bitt}
\end{figure}



\eat{%%%%%%%%%sitt
\begin{example}
	Figure~\ref{fig:sitt} is a running example of \sitt taking the same input as Figure~\ref{fig:citt}. It uses full-$\epsilon$ sectors, (1) $P_4$ lives in the common intersection of \sector{_{1}}, \sector{_{2}} and \sector{_{3}} and it has a \ped distance larger than $\epsilon$ to $\vec{v_1}$, thus, \sitt updates the velocity from $\vec{v_1}$ to $\vec{v_4}$ and the process goes on, and (2) $P_5$ is outside of the common intersection of the preview sectors, thus, $P_4$ serves as the new start point, and an update is triggered. Finally, \sitt sends three points, $P_0, P_4$ and $P_8$, and three velocities, $\vec{v_1}$, $\vec{v_4}$ and $\vec{v_5}$, to the MOD server. 
\end{example}

\begin{figure}[tb!]
	\centering
	\includegraphics[scale=1.0]{figures/Fig-SITT.png}
	\vspace{-2ex}
	\caption{\small A running example of trajectory tracking by \sitt.  }
	\vspace{-1ex}
	\label{fig:sitt}
\end{figure}

}%%%%%%%%%%%%%%eat sitt




\stitle{Correctness and complexity.} 
{The correctness of algorithm \bitt follows from Propositions \ref{theo-full-cone}, \ref{theo-full-sector} and \ref{theo-binary}.
It is also easy to find that it has a linear time complexity like \citt.}


\stitle{Remark}. If $\epsilon_{sed} >> \epsilon_{ped}$, then ``$\overline{P_sP_{i}}$ does not pass $\mathcal{R}^*$'' of line 7 and ``$sed(P_i, \vv{v}) \ge \epsilon_{sed}$'' of line 11 are always false, thus \bitt degenerates to \underline{S}ector \underline{I}ntersection for \underline{T}rajectory \underline{T}racking (\sitt in short) that tracks trajectory in \emph{infinite beams}.
