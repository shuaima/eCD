\section*{{Appendix:  Proofs}}





\stitle{Proof of Theorem~\ref{theo-ldrh-cised}}:
If $\dddot{\mathcal{T}}_s$ can be represented by line segment $\overline{P_sP_{s+k}}$ by algorithm \ldrh, then we have $|P_{s+i}P'_{s+i}| < \epsilon/2$ for each $i \in (0, k)$, where $P'_{s+i}$ is the expected position (synchronized point) of $P_{s+i}$ \wrt the initial velocity $\vv{v}$ of \ldrh.
From the view of the ``x-y-t" 3D space, $\vv{v}$ must be in the common intersection of  $\bigsqcap_{i=1}^{k}$\cone{(P_s, P_{s+i}, \epsilon/2)}, in other words, we have $\bigsqcap_{i=1}^{k}$\cone{(P_s, P_{s+i}, \epsilon/2)} $\ne \{P_s\}$, meaning this sub-trajectory can be represented by approaches based on the spatio-temporal cones.
\eop

\stitle{Proof of Theorem~\ref{theo-full-cone}}:
Let $P'_{s+i}$ be the intersection point of line segment $\overline{P_sP_{s+k}}$ and plane $t = P_{s+i}.t,~i\in (0,k)$, indeed, $P_{s+i}$ is the synchronized point of $P_{s+i}$ \wrt line segment $\overline{P_sP_{s+k}}$. 
Because $\overline{P_sP_{s+k}}$ passes through $\bigsqcap_{i=1}^{k-1}$\cone{(P_s, P_{s+i}, \epsilon)} - \{$P_s$\}, $P'_{s+i}$ must be inside of the synchronous circle of $P_{s+i}$ on the plane. Thus we have $|P_{s+i}P'_{s+i}|<\epsilon$, \ie $sed(P_{s+i}, \overline{P_sP_{s+k}})|<\epsilon$.
\eop

\stitle{Proof of Theorem~\ref{theo-cone-vs}}:
(1) If $\bigsqcap_{i=1}^{k}$\cone{(P_s, P_{s+i}, \epsilon/2)} $\ne \{P_s\}$, then $sed(P_{s+i}, P_sP_{s+k}) <\epsilon$ for each $i \in [1,k]$. 
The intersection point  $P'_{s+i}$ of line segment $P_s P_{s+k}$ and plane $t = P_{s+i}.t$ is the synchronized point of $P_{s+i}$ \wrt $P_s P_{s+k}$, such that $|P_{s+i}P'_{s+i]}| < \epsilon$. Hence for each $i \in [1,k]$, $P'_{s+i}$ falls in the synchronous circle of $P_{s+i}$ on the plane, meaning $\overline{P_sP_{s+k}}$ passes through the common intersection of the preview cones $\bigsqcap_{i=1}^{k-1}$\cone{(P_s, P_{s+i}, \epsilon)}-$\{P_s\}$.
%
(2) \todo.
\eop

\stitle{Proof of Theorem~\ref{theo-half-sector}}:
Trajectory tracking is the combination of trajectory simplification and position tracking.
%
(1) Trajectory simplification: It can be simplified in strip-like areas as shown in Section \ref{sec:sector-in-simp};
%
(2) Position tracking: If it can be represented by a line segment by the intersection of sectors, then there is sure a $\vv{v}$ living in the common intersection of sectors, \eg $\vv{v}$ on $P_sP_{s+i}$, such that it is applicable to track the position in strip areas.
%
Combine (1) and (2), we have the conclusion.
\eop

\stitle{Proof of Theorem~\ref{theo-full-sector}}:
\todo.
\eop

\stitle{Proof of Theorem~\ref{theo-sector-vs}}:
\todo.
\eop

\stitle{Proof of Theorem~\ref{theo-binary}}:
\todo.
1. traj simplification by cones and sectors.
2. position tracking: one velocity, from the intersection of cones.
\eop


