\section*{{Appendix:  Proofs}}





\stitle{Proof of Theorem~\ref{theo-ldrh-cised}}:
\todo{1. $\vv{v}$ in 3D space; 2...}

If $|P_{s+i}P'_{s+i}|\le \epsilon/2$ for each $i \in [1,k]$, then $\vv{v}$ must live in the common intersection of half-$\epsilon$ cones $\bigsqcap_{i=1}^{k}$\cone{(P_s, P_{s+i}, \epsilon/2)}, where $P'_{s+i}$ is the synchronized point of $P_{s+i}$ \wrt velocity $\vv{v}$.
\eop

\stitle{Proof of Theorem~\ref{theo-full-cone}}:
\todo.
\eop

\stitle{Proof of Theorem~\ref{theo-cone-vs}}:
\todo.
\eop

\stitle{Proof of Theorem~\ref{theo-half-sector}}:
Trajectory tracking is the combination of trajectory simplification and position tracking.
%
(1) Trajectory simplification: It can be simplified in strip-like areas as shown in Section \ref{sec:sector-in-simp};
%
(2) Position tracking: If it can be represented by a line segment by the intersection of sectors, then there is sure a $\vv{v}$ living in the common intersection of sectors, \eg $\vv{v}$ on $P_sP_{s+i}$, such that it is applicable to track the position in strip areas.
%
Combine (1) and (2), we have the conclusion.
\eop

\stitle{Proof of Theorem~\ref{theo-full-sector}}:
\todo.
\eop

\stitle{Proof of Theorem~\ref{theo-sector-vs}}:
\todo.
\eop

\stitle{Proof of Theorem~\ref{theo-binary}}:
\todo.
1. traj simplification by cones and sectors.
2. position tracking: one velocity, from the intersection of cones.
\eop


