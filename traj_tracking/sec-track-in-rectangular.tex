
%%% Local Variables:
%%% mode: latex
%%% TeX-master: "www2019"
%%% End:

\section{Tracking in a rectangle-like area}
\label{sec:rectangle}

This section develops an effective and efficient way to track a moving object in a rectangle-like area. Though it is possible to do a tracking in a rectangular area, however, it is really hard to design such a shape and at the same time develop such an efficient algorithm. Thus, we alternatively define a rectangle-like area.


\subsection{Building the rectangle-like areas}

Like the circular area related to \sed and the strip area related to \ped, a rectangle-like area is also related to a Euclidean distance metric, namely the binary Euclidean distances of \sed and \ped.

\stitle{Binary Euclidean distances of \sed and \ped}, shortly \bed (\sed, \ped), is the combination of distance metrics \sed and \ped where their error bounds are set separately to $\epsilon_{sed}$ and $\epsilon_{ped}$, such that (1) if $\epsilon_{ped} \ge \epsilon_{sed}$, then it falls back to the \sed, \ie tracking in a circular area, otherwise, (2) it is the double effects of \ped and \sed, and forms a rectangle-like shape whose short sides are replaced by circular arcs of \sed as shown in {Figure~\ref{fig:areas}-(3)}. Obviously, if $\epsilon_{ped} << \epsilon_{sed}$, then its effect is actually approximate to a \ped.

%By combining \sed and \ped, we get a 
The shape of a rectangle-like area is controlled by two independent parameters, $\epsilon_{ped}$ and $\epsilon_{sed}$, thus, by carefully setting them, we not only build such rectangle-line areas that satisfy the needs of varied applications but also balance two performance metrics, the compression ratios and the errors of spatio-temporal queries, of trajectory simplification/tracking algorithms. 
For the same example of ``the school boy on his way home" mentioned in Section \ref{sec-intro}, if we only use \ped, then we get a simplified trajectory having a good compression ratio and a unbounded query error; if we only use \sed, then we get a poorer compression ratio and a bounded query error. However, if we combine them, \ie~use the \bed, then we could get a medium compression ratio and a bounded query error.



\subsection{Tracking by cone and sector}

Is there an effective and efficient trajectory algorithm implementing \bed that tracks a moving object in a rectangle-like area? Theorem \ref{theo-binary} is the answer to this question.
\begin{theorem}
	\label{theo-binary}
	Given a sub-trajectory $[P_s,...,P_{s+k}]$ and two error bounds $\epsilon_{sed}$ and $\epsilon_{ped}$, it can be tracked in rectangle-like areas by combining sectors and spatio-temporal cones.
\end{theorem}

To track the positions and simplify the trajectory at the same time in a rectangle-like area, it is important to make sure that during the processing of a sub-trajectory $[P_s,...,P_{s+k}]$, there are the same start point $P_s$ and the same velocity $\vec{v}$ for each technique of spatio-temporal cones, sectors, and position tracking of \ped and \sed, such that a strip and a circle \wrt a velocity $\vec{v}$ or a line segment $\overline{P_sP_{s+i}}$, $0<i\le k$, exactly form a rectangle-like area. This is the guideline to develop such a trajectory tracking algorithm. 



\todo{Could ``$|P_sP_{i}| \le l_{m} $" be replaced by a more relaxed constrain, by benefiting from \sed?}





%%%%%%%%%%%%%%%%%%%%%%%%%%%%%%%%%%%%%%%%%%%%%%%%%%%%%%%%%%%%%%%%%%%%%%%%%%
% Algorithm: Traj tracking based on section intersection using full sectors.
\begin{figure}[tb!]   
	\begin{center}
		{\small
			\begin{minipage}{3.3in}
				\myhrule
				%\vspace{-1ex}
				\mat{0ex}{
					{\bf Algorithm}~\bitt $(\dddot{\mathcal{T}}[P_0,\ldots,P_n], ~\epsilon_{sed}, m, ~\epsilon_{ped})$\\
					%	\sstab
					\bcc \hspace{1ex}\= $P_s := P_0$; ~~~~$\mathcal{R}^*$ := \kw{getRPolygon}($P_s$, $P_{s+1}$, $\epsilon_{sed}$, $m$, $P_{s+1}.t$); \\
					\icc \hspace{1ex}\= $\mathcal{S}^*$ := \kw{getSector}($P_s$, $P_{s+1}$, $\epsilon_{ped}$); ~~~~$l_{m} = |P_sP_{s+1}|$;\\
					\icc \hspace{1ex}\= $|\vv{v}|:=\frac{|P_{s}P_{s+1}|}{P_{s+1}.t-P_s.t}$; ~~~~$\vv{v}.\theta:=\overline{P_{s}P_{s+1}}.\theta$;  \\
					\icc \hspace{1ex}\= update ($P_{s}, \vv{v}$); 	\\
					\icc \hspace{1ex}\= $i:= 2$;  	\\
					\icc \hspace{1ex}\= while $i \le n$ do \\
					\icc \>\hspace{3ex} if $\overline{P_sP_{i}}$ ~does not pass~ $\mathcal{R}^*~or~\mathcal{S}^*$, or $|P_sP_{i}| < l_{m}$ then \\ % // updates velocity and location \\
					\icc \>\hspace{7ex}    $P_s := P_{i-1}$; ~~~~$\mathcal{R}^*$ := $\emptyset$;~~~~$\mathcal{S}^*$ := $\emptyset$; ~~~~$l_{m} = |P_sP_{i}|$;\\
					\icc \>\hspace{7ex}    $|\vv{v}|:=\frac{|P_sP_{i}|}{P_{i}.t-P_s.t}$; ~~~~$\vv{v}.\theta:=\overline{P_{s}P_{i}}.\theta$;  \\
					\icc \>\hspace{7ex}    update ($P_{s}, \vv{v}$); 	\\
					\icc \>\hspace{3ex} else if $sed(P_i, \vv{v}) \ge \epsilon_{sed}$ ~or~ $ped(P_i, \vv{v}) \ge \epsilon_{ped}$ then  \\ %$\overline{P_sP_{i}}$ ~passes ~ $\mathcal{R}^*$ and $\mathcal{S}^*$, $|P_sP_{i}| > l_{m} - \epsilon$ \\ \hspace{9ex} ~and~
					\icc \>\hspace{7ex}    $|\vv{v}|:=\frac{|P_sP_{i}|}{P_{i}.t-P_s.t}$; ~~~~$\vv{v}.\theta:=\overline{P_sP_{i}}.\theta$; \\
					\icc \>\hspace{7ex}    update ($\vv{v}$); \\
					\icc \>\hspace{3ex} if $\mathcal{S}^*=\emptyset$ then $\mathcal{R}^*:=$ \kw{getRPolygon}($P_s$, $P_{i}$, $\epsilon_{sed}$, $m$, $P_{s+1}.t$); \\
					\icc \>\hspace{7ex}    $\mathcal{S}^*:=$ \kw{getSector}($P_s$, $P_{i}$, $\epsilon_{ped}$); \\
					\icc \>\hspace{3ex} else $\mathcal{R}^*:=$ \kw{getRPolygon}($P_s$, $P_{i}$, $\epsilon_{sed}$, $m$, $P_{s+1}.t$); \\
					\icc \>\hspace{7ex}    $\mathcal{S}^*$ := $\mathcal{S}^*\bigsqcap$ \kw{getSector}($P_s$, $P_{i}$, $\epsilon_{ped}$); $l_{m} = \max\{|P_sP_{i}|, l_{m}\}$;\\
					\icc \>\hspace{3ex} $i$ := $i +1$;\\
					\icc \>\hspace{0ex} update ($P_{n}$); 
				}
				\vspace{-2ex}
				\myhrule
			\end{minipage}
		}
	\end{center}
	\vspace{-2ex}
	\caption{\small Trajectory tracking based on sector and cone.}
	\label{alg:bitt}
	\vspace{-2ex}
\end{figure}
%%%%%%%%%%%%%%%%%%%%%%%%%%%%%%%%%%%%%

\subsection{Implementation.}
We now present the algorithm of \underline{B}inary \underline{I}ntersection for \underline{T}rajectory \underline{T}racking (BITT) that tracks moving objects in a rectangle-like area of \bed, as shown in Figure~\ref{alg:bitt}. 
%
Indeed, it is a double check of cone intersection and sector intersection for each point for the purpose of trajectory simplification, and a double check of \sed and \ped distance deviations for position tracking. In addition to that, it uses a uniform velocity $\vv{v}$ for both position tracks of \sed and \ped such that either deviation of \ped or \sed distance will cause an update of velocity $\vv{v}$. 
%
\bitt is the super version of \citt and \sitt, that is, if $\epsilon_{sed} >> \epsilon_{ped}$, then ``$\overline{P_sP_{i}}$ does not pass $\mathcal{R}^*$" of line 7 and ``$sed(P_i, \vv{v}) \ge \epsilon_{sed}$" of line 11 are always false, thus \bitt falls back to \sitt. {Similarly, if $\epsilon_{sed} \le \epsilon_{ped}$, then it falls back to \citt.}




\begin{example}
	Figure~\ref{fig:bitt} is a running example of \bitt. It takes as inputs the same trajectory as the above, the same $\epsilon_{sed}$ as Figure~\ref{fig:citt} and an $\epsilon_{ped}$ of half that of Figure~\ref{fig:sitt}. Because, its $\epsilon_{sed}$ is the same as Figure~\ref{fig:citt}, its effectiveness is also the same as Figure~\ref{fig:citt}. For the purpose of clearness, we do not show those cones in the figure.
	%
	\bitt uses full-$\epsilon_{sed}$ cones and full-$\epsilon_{ped}$ sectors to simplify the trajectory. Initially, it sets the same start point and initial velocity as Figure~\ref{fig:citt}, 
	Then, (1) $P_3$ lives in the common intersection of the preview cones and sectors, and it has both \sed and \ped distances larger than $\epsilon_{sed}$ and $\epsilon_{ped}$, respectively, \ie $|P_3P'_3| \ge \epsilon_{sed}$ and $|P_3P^*_3| \ge \epsilon_{ped}$, thus, \bitt updates the velocity from $\vec{v_1}$ to $\vec{v_3}$ and the process goes on, (2) $P_5$ is outside of the common intersections of the preview cones and sectors, thus, $P_4$ serves as the new start point, and an update is triggered, and (3) $P_7$ lives in the common intersections of the preview cones and sectors, and it has a \ped distance larger than $\epsilon_{ped}$, thus, \bitt updates the velocity from $\vec{v_5}$ to $\vec{v_7}$ (not shown) and the process goes on. Finally, this algorithm sends three points, $P_0, P_4$ and $P_8$, and four velocities, $\vec{v_1}$, $\vec{v_3}$, $\vec{v_5}$ and $\vec{v_7}$, to the MOD server. 
	\bitt ensures that any removed point is located in a rectangle-like area around its expected position \wrt a velocity of trajectory tracking or a line segment connecting two neighboring data points of the simplified trajectory, \eg $P_2$ is in the rectangle-like area around its synchronized point $P'_2$ \wrt line segment $\overline{P_0P_4}$. 
\end{example}

\begin{figure}[tb!]
	\centering
	\includegraphics[scale=1.0]{figures/Fig-BITT.png}
	\vspace{-2ex}
	\caption{\small A running example of trajectory tracking by \bitt. In this case, the spatio-temporal cones and their intersections are the same as Figure~\ref{fig:citt}, thus they are not shown here for clearness.  }
	\vspace{-2ex}
	\label{fig:bitt}
\end{figure}

\stitle{Correctness and complexity.} 
The correctness of algorithm \bitt follows from Theorems \ref{theo-full-cone}, \ref{theo-full-sector} and \ref{theo-binary}.
It is also easy to find that it has a linear time complexity like \citt and \sitt.


%\stitle{Discuss.} Relations to position tracking and traj simplification. 
