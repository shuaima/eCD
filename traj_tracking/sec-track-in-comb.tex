
%%% Local Variables:
%%% mode: latex
%%% TeX-master: "www2019"
%%% End:

\section{Tracking in combined areas}
\label{sec:combine}

In this section, we combine \sed and \ped to track a moving object in an alternative area that balances two key metrics, \ie the compression ratios and the query errors, of trajectory simplification and tracking algorithms, and to satisfy the needs of varied applications.

%\subsection{Combining circular and strip areas}

\stitle{Binary Euclidean distances of \sed and \ped}, shortly \bed (\sed, \ped), is the combination of distance metrics \sed and \ped where their error bounds are set seperately to $\epsilon_s$ and $\epsilon_p$, such that (1) if $\epsilon_p \ge \epsilon_s$, then it falls back to the \sed, otherwise, (2) it is the combination of a circular and a strip areas, and forms a rectangle whose short sides are replaced by circular arcs as shown in \todo{Figure}. Note that if $\epsilon_p << \epsilon_s$, then its effect is actually close to a \ped.

%\subsection{Position tracking in combined areas}
%based on ldr, double check.


\subsection{Tracking by cone and sector}

\begin{theorem}
	\label{theo-full-sector}
	Given a sub-trajectory $[P_s,...,P_{s+k}]$ and two error bounds $\epsilon_{sed}$ and $\epsilon_{ped}$, it can be tracked in circular and strip areas by spatio-temporal cones and sectors.
\end{theorem}

\begin{proof}
	\todo.
	1. traj simplification by cones and sectors.
	2. position tracking: one velocity, from cones.
\end{proof}


-- principle, adjust velocity...
	checking \textcolor{blue}{separately} of cone and sector ($P_sP_{s+k-1}$ is the choice), couple with ldr.
	step 1: check metrics: (if all intersections of one metric are true, then there must be a line segment satisfying all).
	if any intersection of one metric is false, then new line segment.
	step 2:	(check ldr) if any ldr is false, then find the common intersection of all metric and adjust velocity; select a point from the common area (or simply $P_{s+k-1}$).

\subsection{Implementation}
-- algorithm. intersection based trajectory tracking (BITT)

\begin{example}
	\todo.
\end{example}

\stitle{Correctness and complexity.} \todo.

\eat{%%%%%%%%%%%%
weak tracking

-- principle, adjust velocity...
	must checking \textcolor{blue}{consistently} of cone, sector and interval, couple with ldr.
	if the common intersection of all metrics is false, then new line segment.
	otherwise, checks ldr, 	if any ldr is false, then adjust velocity, point select from the common area.
	
-- algorithm. intersection based trajectory tracking (ITT-W)

-- example

-- correctness and complexity
}%%%%%%%%%%%%eat

\stitle{Discuss.} Relations to position tracking and traj simplification.
