
%%% Local Variables:
%%% mode: latex
%%% TeX-master: "www2019"
%%% End:

\section{Tracking in combined areas}
\label{sec:combine}

In this section, we combine \sed and \ped to track a moving object in an alternative area that balances two key metrics, \ie the compression ratios and the query errors, of trajectory simplification and tracking algorithms, and to satisfy the needs of varied applications.


%\subsection{Combining circular and strip areas}
\eat{
the distance metric used in tracking ``the distance between the expecting position and the actual position" ... it is SED in trajectory simplification. 

scope: It is  a CIRCLE around the expecting position. fig.

Limitations.

we can make these regions more flexible as soon as we include another distance metric, PED, in trajectory simplification with a little modification.

scope: fig.

advantages and applications.

}

\stitle{Binary Euclidean distances of \sed and \ped}, shortly \bed (\sed, \ped), is the combination of distance metrics \sed and \ped where their error bounds are set seperately to $\epsilon_{sed}$ and $\epsilon_{ped}$, such that (1) if $\epsilon_{ped} \ge \epsilon_{sed}$, then it falls back to the \sed, otherwise, (2) it is the combination of a circular and a strip areas, and forms a rectangle whose short sides are replaced by circular arcs as shown in \todo{Figure}. Note that if $\epsilon_{ped} << \epsilon_{sed}$, then its effect is actually close to a \ped.

%\subsection{Position tracking in combined areas}
%based on ldr, double check.


%\subsection{Tracking by cone and sector}

\begin{theorem}
	\label{theo-binary}
	Given a sub-trajectory $[P_s,...,P_{s+k}]$ and two error bounds $\epsilon_{sed}$ and $\epsilon_{ped}$, it can be tracked in the combined area of a circular and a strip.
\end{theorem}

\begin{proof}
	\todo.
	1. traj simplification by cones and sectors.
	2. position tracking: one velocity, from the intersection of cones.
\end{proof}




\stitle{Implementation.}
We now present the algorithm of \underline{B}inary \underline{I}ntersection for \underline{T}rajectory \underline{T}racking (BITT) that tracks moving objects in combined areas of circular and strip by \sed and \ped, as shown in Figure~\ref{alg:bitt}. 
%
Indeed, it is a double check of cone intersection and sector intersection for each point for the purpose of trajectory simplification, and a double check of distance deviations for position tracking. And for position tracking, it uses a uniform velocity $\vv{v}$ for both checks of \sed and \ped. %If either \sed or \ped deviation breaks the threshold, then an update is triggered.

%%%%%%%%%%%%%%%%%%%%%%%%%%%%%%%%%%%%%%%%%%%%%%%%%%%%%%%%%%%%%%%%%%%%%%%%%%
% Algorithm: Traj tracking based on section intersection using full sectors.
\begin{figure}[tb!]   
	\begin{center}
		{\small
			\begin{minipage}{3.3in}
				\myhrule
				%\vspace{-1ex}
				\mat{0ex}{
					{\bf Algorithm}~\bitt $(\dddot{\mathcal{T}}[P_0,\ldots,P_n], ~\epsilon_{sed}, m, ~\epsilon_{ped})$\\
					%	\sstab
					\bcc \hspace{1ex}\= $P_s := P_0$; ~~~~$\mathcal{R}^*$ := \kw{getRPolygon}($P_s$, $P_{s+1}$, $\epsilon_{sed}$, $m$, $P_{s+1}.t$); \\
					\icc \hspace{1ex}\= $\mathcal{S}^*$ := \kw{getSector}($P_s$, $P_{s+1}$, $\epsilon_{ped}$); ~~~~$l_{m} = |P_sP_{s+1}|$;\\
					\icc \hspace{1ex}\= $|\vv{v}|:=\frac{|P_{s}P_{s+1}|}{P_{s+1}.t-P_s.t}$; ~~~~$\vv{v}.\theta:=\vv{P_{s}P_{s+1}}.\theta$;  \\
					\icc \hspace{1ex}\= update ($P_{s}, \vv{v}$); 	\\
					\icc \hspace{1ex}\= $i:= 2$;  	\\
					\icc \hspace{1ex}\= while $i \le n$ do \\
					\icc \>\hspace{3ex} if $\vv{P_sP_{i}}$ ~does not pass~ $\mathcal{S}^*~or~\mathcal{R}^*$, or $|P_sP_{i}| \le l_{m} - \epsilon$ then \\ % // updates velocity and location \\
					\icc \>\hspace{7ex}    $P_s := P_{i-1}$; ~~~~$\mathcal{R}^*$ := $\emptyset$;~~~~$\mathcal{S}^*$ := $\emptyset$; ~~~~$l_{m} = |P_sP_{i}|$;\\
					\icc \>\hspace{7ex}    $|\vv{v}|:=\frac{|P_sP_{i}|}{P_{i}.t-P_s.t}$; ~~~~$\vv{v}.\theta:=\vv{P_{s}P_{i}}.\theta$;  \\
					\icc \>\hspace{7ex}    update ($P_{s}, \vv{v}$); 	\\
					\icc \>\hspace{3ex} else if $\vv{P_sP_{i}}$ ~passes ~ $\mathcal{R}^*$ and $\mathcal{S}^*$, $|P_sP_{i}| > l_{m} - \epsilon$ \\ \hspace{9ex} ~and~ ($ped(P_i, \vv{v}) \ge \epsilon_{ped} ~or~ sed(P_i, \vv{v}) \ge \epsilon_{sed}) $ then  \\ %~// updates velocity only 
					\icc \>\hspace{7ex}    $|\vv{v}|:=\frac{|P_sP_{i}|}{P_{i}.t-P_s.t}$; ~~~~$\vv{v}.\theta:=\vv{P_sP_{i}}.\theta$; \\
					\icc \>\hspace{7ex}    update ($\vv{v}$); \\
					\icc \>\hspace{3ex} if $\mathcal{S}^*=\emptyset$ then $\mathcal{R}^*:=$ \kw{getRPolygon}($P_s$, $P_{i}$, $\epsilon_{sed}$, $m$, $P_{s+1}.t$); \\
					\icc \>\hspace{7ex}    $\mathcal{S}^*:=$ \kw{getSector}($P_s$, $P_{i}$, $\epsilon_{ped}$); \\
					\icc \>\hspace{3ex} else $\mathcal{R}^*:=$ \kw{getRPolygon}($P_s$, $P_{i}$, $\epsilon_{sed}$, $m$, $P_{s+1}.t$); \\
					\icc \>\hspace{7ex}    $\mathcal{S}^*$ := $\mathcal{S}^*\bigsqcap$ \kw{getSector}($P_s$, $P_{i}$, $\epsilon_{ped}$); $l_{m} = \max\{|P_sP_{i}|, l_{m}\}$;\\
					\icc \>\hspace{3ex} $i$ := $i +1$;\\
					\icc \>\hspace{0ex} update ($P_{n}$); 
				}
				\vspace{-2ex}
				\myhrule
			\end{minipage}
		}
	\end{center}
	\vspace{-1ex}
	\caption{\small Trajectory tracking based on sector and cone.}
	\label{alg:bitt}
	\vspace{-1ex}
\end{figure}
%%%%%%%%%%%%%%%%%%%%%%%%%%%%%%%%%%%%%



\begin{example}
	\todo.
\end{example}

\stitle{Correctness and complexity.} \todo.

\eat{%%%%%%%%%%%%
weak tracking

-- principle, adjust velocity...
	must checking \textcolor{blue}{consistently} of cone, sector and interval, couple with ldr.
	if the common intersection of all metrics is false, then new line segment.
	otherwise, checks ldr, 	if any ldr is false, then adjust velocity, point select from the common area.
	
-- algorithm. intersection based trajectory tracking (ITT-W)

-- example

-- correctness and complexity
}%%%%%%%%%%%%eat

\stitle{Discuss.} Relations to position tracking and traj simplification.
