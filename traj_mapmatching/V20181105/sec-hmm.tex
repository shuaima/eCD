
%%% Local Variables:
%%% mode: latex
%%% TeX-master: "gis18"
%%% End:
\subsection{Global Route Decoding}
\label{sec:hmm}

We use hidden markov model to find the global optimal route given the local
optimal paths produced by the local path recovery component.
The key of HMM modeling is to define two probabilities, \ie the \emph{emission probabilities} and the \emph{transition probabilities}.
%
A emission probability gives the likelihood that an observation is resulted from
a given state. We adopt the widely used {emission probability estimation} proposed by Newson and Krumm in \cite{Newson2009Hidden}. Note the candidate road segments are selected under the distance constraint, which means that a point would not be matched to a road segment having a distance large than the error bound.

\eat{
\subsubsection{Emission Probabilities}
Emission probabilities give the likelihood that an observation is resulted from a given state. For map-matching, it is likely that the vehicle with a GPS point $P_i$ is on a specific road segment $r_i^j$.
For those candidate road segments close to the GPS point, the emission probability is
dependent on the distance between the GPS point and the road segment, and is estimated by Equation~\ref{equ:emi-prob} proposed by Newson and Krumm \cite{Newson2009Hidden}:
%, using a Gaussian kernel. Specifically, emission probability is modeled as:
\begin{equation}
  \label{equ:emi-prob}
  E(r_i^k| P_i) = \frac{1}{\sqrt{2\pi}\sigma} \exp \frac{d(r_i^k, P_i)^2}{2\sigma^2}
\end{equation}
where, $d(r_i^k, P_i)$ is the great circle distance on the surface of the earth between the observed location $p_i$ and the candidate road segment $r_i^k$. And
$\sigma$ is the standard length of GPS measurements, which can be estimated
from data.
}



%\subsubsection{Transition Probabilities}
A transition probability is the probability of an object moving from
one road segment to another. The appropriate definition of transition
probability is the key of HMM modeling.
In Newson and Krumm's modeling, transition probabilities are modeled based on
the difference of great circle distance of two observed locations and their
corresponding road segments.
% The intuition is that transitions whose driving
% distance is about the same as the great circle distance is more likely to be the
% actual route traveled.
This modeling is reasonable when trajectory is relatively dense, i.e.
distance between two neighboring points is small. However, when a trajectory is sparse,
it is possible that more than one route have similar driving distances as the great
circle distance, thus, roughly choosing the one with the smallest difference may lead
to a circuitous route.
%
To correctly handle this problem, we make use of information from
simplified lines, and propose a model of transition probability based on
similarity.
%
Specifically, we define the transition probability $T$ of moving from road segment
$r_{i-1}^j$ to $r_i^k$ as :
\begin{equation}
  \label{equ:trans-prob}
  T(r_i^k| r_{i-1}^j) = \lambda_De^{-\lambda_D\delta_D}\lambda_Re^{-\lambda_R\delta_R}
\end{equation}
%
\begin{equation}
  \delta_R = |\phi_T -\phi_R|
\end{equation}
%
\begin{equation}
  \delta_D = d_T(P_i,P_{i-1}) - d_R(P_i,P_{i-1})
\end{equation}
%
\begin{equation}
  d_T(P_i,P_{i-1}) = \overline{\mathcal{T}}[i].L_L + \overline{\mathcal{T}}[i].L_R
\end{equation}
%
\begin{equation}
  d_R(P_i,P_{i-1}) =  \sum_{r_j \in R_{i-1, i}}{(r_j.L_L + r_j.L_R)}
\end{equation}
%
where $d_T(P_i,P_{i-1})$ is the length of the raw
sub-trajectory, $d_R(P_i,P_{i-1})$ is the length of the local optimal path $R_{i-1, i}$ on road network detected by the local path recovery component, $\delta_D $ is the
difference of $d_T$ and $d_R$, and $\delta_R $ is the difference of the
ratio of trajectory length residing in the left side of the
simplified line segment and that of the path.

\eat{
%After we estimated the emission probabilities and transition probabilities by Equations \ref{equ:emi-prob} and \ref{equ:trans-prob},
We then use the Viterbi algorithm to search for the optimal route. The Viterbi algorithm is a dynamic programming
algorithm that can quickly detect a sequence of states that maximizes the joint
probability, which is the product of the emission probabilities and transition
probabilities of all the states in the sequence.
The detected sequence is the route with maximum likelihood and thus the global optimal route.
}

