
%%% Local Variables:
%%% mode: latex
%%% TeX-master: "gis18"
%%% End:



\section{Preliminaries}
\label{sec-pre}




In this section, we first introduce the concepts on simplified trajectories and map-matching, then we introduce the basic HMM-based map-matching method that serves as the fundamental of the work.
%, followed by statement the problem of map-matching on simplified trajectories.

\subsection{Notations}


\stitle{Points ($P$)}. A GPS point is defined as a triple $P(x, y, t)$,
which represents that a moving object is located at {\em longitude} $x$ and {\em
  latitude} $y$ at {\em time} $t$.

\stitle{Trajectories ($\dddot{\mathcal{T}}$)}. A trajectory
$\dddot{\mathcal{T}}[P_0, \ldots, P_n]$ is a sequence of data points in a
monotonically increasing order of their associated time values ($P_i.t <
P_j.t$ for any $0\le i<j\le n$). Intuitively, a trajectory is the path (or
track) that a moving object follows through space as a function of time~\cite{physics-trajectory}.


\eat{
\stitle{Simplified line segments ($\mathcal{L}$)}. A Simplified line segment (or
line segment for simplicity) $\mathcal{L}$ is  defined as $\vv{P_{s}P_{e}}$,
which represents the  closed line segment that connects the start point $P_s$ and the end point $P_e$.
There are also two attributes $\mathcal{L}.L_p$ and $\mathcal{L}.L_n$
representing the length of raw trajectory on each side of the simplified line
segment respectively.
}

\stitle{Simplified trajectories ($\overline{\mathcal{T}}$)}. A simplified trajectory $\overline{\mathcal{T}}[\mathcal{L}_0, \ldots , \mathcal{L}_m]$ ($0< m \le n$) of a trajectory $\dddot{\mathcal{T}}[P_0, \ldots, P_n]$ is a sequence of continuous directed line segments $\mathcal{L}_{i}$ = $\vv{P_{s_i}P_{e_i}}$ ($i\in[0,m]$) of $\dddot{\mathcal{T}}$  such that $\mathcal{L}_{0}.P_{s_0} = P_0$, $\mathcal{L}_{m}.P_{e_m} = P_n$ and  $\mathcal{L}_{i}.P_{e_i}$ = $\mathcal{L}_{i+1}.P_{s_{i+1}}$ for all $i\in[0, m-1]$.
Note that (1) each directed line segment in $\overline{\mathcal{T}}$ essentially represents a continuous sequence of data points in $\dddot{\mathcal{T}}$, and
(2) the simplified trajectories are referred to {as} error bounded if for each point $P$ in \trajec{T}, there exist points $P_j$ and $P_{j+1}$ in $\overline{\mathcal{T}}$ such that the distance from $P$ to $\mathcal{L}(P_j,P_{j+1}))$ is less than $\epsilon$.
%error bounded by $\epsilon$ if

\eat{
\stitle{Error bounded trajectory simplification}. Given a trajectory \trajec{T}, an error bound $\epsilon$ and a simplification algorithm $\mathcal{A}$ that produces another trajectory \trajec{T'},
we say that algorithm $\mathcal{A}$ is error bounded by $\epsilon$ if  for each point $P$ in \trajec{T}, there exist points $P_j$ and $P_{j+1}$ in \trajec{T'} such that the distance from $P$ to $\mathcal{L}(P_j,P_{j+1}))$ is less than $\epsilon$.
}



%\subsection{Terms on map-matching}



\stitle{Road segments ($r$)}. A road segment is defined as $r = (v_s,v_e)$, representing an edge directly connecting two ending
points in the map.



\eat{
\stitle{Candidate Road Sets ($C$)}. A candidate road set (candidate set in short) $C_i = \{r_i^1,r_i^2,\ldots,r_i^k\}$ of a GPS point $P_i$
is a set of road segments that are close to the point. The final
matched road segment is selected from the candidate set.
%In this paper, we set the search range as a circle centered at point $P_i$ with radius as 200 meters.
}

\stitle{Routes ($R$)}. A route $R = {[r_0, \ldots,r_m]}$ is a continuous sequence
of road segment such that $r_i.v_e = r_{i+1}.v_s$, $0\le i<m$.

\stitle{Road network ($G$)}. A road network is a directed graph $G(V,E)$ where $V$ is the set of junction points of roads and $E$
is the set of road segments between two junction points.

\stitle{Map-matching}. Given a (simplified) trajectory of a user and a road network, the goal of (trajectory simplification aware) map-matching is to find the most likely route in the road network that has been traveled by the user.




\subsection{HMM-based Map-matching}
In recent years, map-matching is always modeled as a sequence labeling problem and tackled using sequence models such as HMM.
The authors of \cite{Lamb1999Avoiding} first introduce HMM for map-matching, then a number of works \cite{Newson2009Hidden, Wang:eddy, Osogami:2013:IRL, yin:feature-based} follow this idea.
%
In the modeling of HMM-based map-matching, road segments are \emph{hidden states} and GPS points are \emph{observations}.
For example, in \myfig{fig:hmm-model-a}, GPS points $P_1,P_2,P_3$ are observations of the moving object at timestamps $T_1,T_2,T_3$, respectively,
and $r_1^1$ and $r_1^2$, two {candidate road segments} of point $P_1$, are the hidden states of the moving object at timestamp $T_1$.
Moreover, the likelihood of the GPS point residing in a road segment is described by \emph{emission probability} ($E$). For instance, in \myfig{fig:hmm-model-b}, the emission probability of point $P_1$ on road segment {$r_1^2$ is $E_1^2$}.

Then, the map-matching of a sub-trajectory to a road network is
modeled as a weighted directed graph (\myfig{fig:hmm-model-b}), where a vertex is a hidden state (candidate road segment), an edge is the transition from the previous hidden state to the next hidden state, and the weight of an edge, named \emph{transition probability} ($T$), is the probability that the moving object transitions from one road segment to another. For example, {$T_{2}^3$} is the transition probability from {$r_1^2$ to $r_2^3$}.

Finally, the probability of a sub-trajectory \trajec{T}$[P_s, \ldots, P_{s+u}]$ matched to a route $R$ is defined as the joint probability $J(\dddot{\mathcal{T}}, R) = \prod_{i=1}^u{T(r_{s+i}|r_{s+i-1})\cdot E(P_{s+i}|r_{s+i})}$, $P\in \dddot{\mathcal{T}}$ and $r\in R$, and a path in the graph with the highest joint probability is the matched route of the sub-trajectory.
Note that most HMM-based methods share the same model except that they have respective definitions of transition probabilities.
Our \stmm also follows this common model and has specific definition of
transition probability for simplified {trajectories}.

%\begin{equation}
%  \label{equ:joint-prob}
%  P(R,T) = \prod_{i=1}^n{P(r_i|r_{i-1})\cdot P(P_i|r_i)}
%\end{equation}


\begin{figure}[tb!]
  \centering
  \begin{subfigure}{0.4\textwidth}
    \includegraphics[width = \textwidth]{Figures/Fig-HMM-model-road.pdf}
    \caption{finding the candidate road segments.}\label{fig:hmm-model-a}
    \vspace{1ex}
  \end{subfigure}
  \begin{subfigure}{0.42\textwidth}
    \includegraphics[width = \textwidth, height = 0.6\textwidth]{Figures/Fig-HMM-model.pdf}
    \caption{finding the optimal route. }\label{fig:hmm-model-b}
  \end{subfigure}
  \vspace{-1ex}
  \caption{HMM-based map-matching.}
  \label{fig:hmm-model}
 \vspace{-4ex}
\end{figure}




%\subsection{Problem statement}
%Given a simplified trajectory and a road network($G(V,E)$), the goal of map-matching on simplified trajectories is to find the most likely route ($R$) in the road network that has been traveled by the user.



