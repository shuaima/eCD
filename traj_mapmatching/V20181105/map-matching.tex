
%%% Local Variables:
%%% mode: latex
% TeX-master: "gis18"
%%% End:

\documentclass[sigconf, review]{acmart}

% \documentclass[sigconf,edbt]{acmart-edbt2019}
% \usepackage{polyglossia}
\usepackage{booktabs} % For formal tables


\usepackage{graphicx}
\usepackage{enumerate}
\usepackage{amsfonts}
\usepackage{amsmath}
\usepackage{amssymb}
\usepackage{color}
\usepackage{colortbl}
\usepackage{epsfig}
\usepackage{xspace}
\usepackage{esvect} % for arrows
\usepackage{subcaption}
% \usepackage{subfigure}
\usepackage{balance}
% \usepackage{cite}
\usepackage[english]{babel}

% algorithms
\usepackage{algorithm}
\usepackage[noend]{algpseudocode}

\DeclareMathOperator*{\argmax}{argmax} % no space, limits underneath in displays

%%%%%%%%%%%%%%%%%%%%%%%%%%%%%%%%%%%%%
%% DO NOT DELETE!!
%%%%%%%%%%%%%%%%%%%%%%%%%%%%%%%%%%%%%
%\usepackage{tikz}
%\usetikzlibrary{trees}

\usepackage{multirow}
\usepackage{url}

\newcommand{\imp}{\vdash_{\cal I}}


%%%%%%%%%%%%%%%%%%%%%%%%%%%%%%%%%%%%%%%%%%
% Enumerate and Itemize modifications
%\usepackage{enumitem}
%\setlist{topsep=0pt,noitemsep} \setitemize[1]{label=$\circ$}
%%%%%%%%%%%%%%%%%%%%%%%%%%%%%%%%%%%%%%%%%%%

\sloppy
\newcommand{\rtable}[1]{\ensuremath{\mathsf{#1}}}
\newcommand{\ratt}[1]{\ensuremath{\mathit{#1}}}
\newcommand{\at}[1]{\protect\ensuremath{\mathsf{#1}}\xspace}
\newcommand{\myhrule}{\rule[.5pt]{\hsize}{.5pt}}
\newcommand{\oneurl}[1]{\texttt{#1}}
\newcommand{\eat}[1]{}
\newcommand{\stab}{\rule{0pt}{8pt}\\[-1.6ex]}
\newcommand{\sttab}{\rule{0pt}{8pt}\\[-2ex]}
%\newcommand{\sstab}{\rule{0pt}{8pt}\\[-2.4ex]}
\newcommand{\tabstrut}{\rule{0pt}{4pt}\vspace{-0.07in}}
\newcommand{\vs}{\vspace{1ex}}
\newcommand{\exa}[2]{{\tt\begin{tabbing}\hspace{#1}\=\+\kill #2\end{tabbing}}}
\newcommand{\ra}{\rightarrow}
\newcommand{\la}{\leftarrow}
\newcommand{\bi}{\begin{itemize}}
\newcommand{\ei}{\end{itemize}}
\newenvironment{tbi}{\begin{itemize}
        \setlength{\topsep}{1.5ex}\setlength{\itemsep}{0ex}\vspace{-0.5ex}}
        {\end{itemize}\vspace{-0.5ex}}
\newenvironment{tbe}{\begin{enumerate}
        \setlength{\topsep}{0ex}\setlength{\itemsep}{-0.7ex}\vspace{-1ex}}
        {\end{itemize}\vspace{-1ex}}

\newcommand{\mat}[2]{{\begin{tabbing}\hspace{#1}\=\+\kill #2\end{tabbing}}}
\newcommand{\m}{\hspace{0.05in}}
\newcommand{\ls}{\hspace{0.1in}}
\newcommand{\be}{\begin{enumerate}}
\newcommand{\ee}{\end{enumerate}}
\newcommand{\beqn}{\begin{eqnarray*}}
\newcommand{\eeqn}{\end{eqnarray*}}
\newcommand{\card}[1]{\mid\! #1\!\mid}
\newcommand{\fth}{\hfill $\Box$}
\newcommand{\AND}{\displaystyle{\bigwedge_{i=1}^{n}}}
%\newcommand{\U}[1]{\displaystyle{\bigcup_{#1}}}
\newcommand{\Sm}[1]{\displaystyle{\sum_{#1}}}
\newcommand{\stitle}[1]{\vspace{1ex}\noindent{\bf #1}}
\newcommand{\etitle}[1]{\vspace{0.5ex}\noindent{\em \underline{#1}}}
\renewcommand{\t}{\tau}
\newcommand{\Inh}[1]{\$#1}
\renewcommand{\r}[1]{{\it rule}(#1)}
\newcommand{\pa}{\parallel}
\newcommand{\LHS}{\kw{{\small LHS}}}
\newcommand{\RHS}{\kw{RHS}}
\newcommand{\len}{\kw{len}}
\newcommand{\kop}{\kw{op}}
%\newcommand{\st}{\emph{s.t.}\xspace}
\newcommand{\ie}{\emph{i.e.,}\xspace}
\newcommand{\eg}{\emph{e.g.,}\xspace}
\newcommand{\wrt}{\emph{w.r.t.}\xspace}
\newcommand{\aka}{\emph{a.k.a.}\xspace}
\newcommand{\kwlog}{\emph{w.l.o.g.}\xspace}

\newcommand{\VNM}{\kw{VNM}}
\newcommand{\VNMs}{\kw{VNM}}
\newcommand{\VN}{\kw{VN}}
\newcommand{\SN}{\kw{SN}}

%%%%%%%%%%%%%%%%%%%%%%%%%%%%%%%%%%%%%%%%%%%%%%%%%%%%%%%%%%%%%%%%
%                  Relation Algebra operators
%%%%%%%%%%%%%%%%%%%%%%%%%%%%%%%%%%%%%%%%%%%%%%%%%%%%%%%%%%%%%%%%

\newcommand{\RS}{{\small S}\xspace}
\newcommand{\RP}{{\small P}\xspace}
\newcommand{\RJ}{{\sc j}\xspace}
\newcommand{\RC}{{\small C}\xspace}
\newcommand{\RSJ}{{\small SJ}\xspace}
\newcommand{\RSC}{{\small SC}\xspace}
\newcommand{\RSP}{{\small SP}\xspace}
\newcommand{\RPJ}{{\small PJ}\xspace}
\newcommand{\RPC}{{\small PC}\xspace}
\newcommand{\RSPJ}{{\sc spj}\xspace}
\newcommand{\RSPC}{{\small SPC}\xspace}
\newcommand{\RSPJU}{{\sc spju}\xspace}
\newcommand{\RSPCU}{{\small SPCU}\xspace}
\newcommand{\RSPJUN}{{\small SPJU$^N$}\xspace}
\newcommand{\RSPCUN}{{\small SPCU$^N$}\xspace}
%%%%%%%%%%%%%%%%%%%%%%%%%%%%%%%%%%%%%%%%%%%%%%%%%%%%%%%%%%%%%%%%%%%%%%%%%%%%%%
% ALGORITHMS
%%%%%%%%%%%%%%%%%%%%%%%%%%%%%%%%%%%%%%%%%%%%%%%%%%%%%%%%%%%%%%%%%%%%%%%%%%%%%%%

\newcommand{\kw}[1]{{\ensuremath {\mathsf{#1}}}\xspace}

\newcounter{ccc}
\newcommand{\bcc}{\setcounter{ccc}{1}\theccc.}
\newcommand{\icc}{\addtocounter{ccc}{1}\theccc.}
\newcommand{\checking}{{\mbox{\small\sf Checking}\xspace}}
\newcommand{\preProcessing}{{\mbox{\small\sf preProcessing}\xspace}}
\newcommand{\CFDconsistency}{{\mbox{\small\sf CFD\_Checking}\xspace}}
\newcommand{\MCS} {\kw{MCS}}
\newcommand{\templateDB}{{\mbox{\small\sf templateDB}\xspace}}
\newcommand{\ChaseChecking}{{\mbox{\small\sf RandomChecking}\xspace}}
\newcommand{\chase}{{\mbox{\small\sf Chase}\xspace}}
\newcommand{\SAT}{{\mbox{\small\sf SAT}\xspace}}
\newcommand{\kSAT}{{\mbox{\small 3SAT}\xspace}}
\newcommand{\PropCFDSPC}{\kw{Prop{\small CFD\_SPC}}}
\newcommand{\PropCFDSPCU}{\kw{Prop{\small CFD\_SPCU}}}
\newcommand{\UnionEQs}{\kw{UnionEQs}}
\newcommand{\UnionCFDs}{\kw{UnionCFDs}}
\newcommand{\EQ}{\kw{EQ}}
\newcommand{\eq}{\kw{eq}}
\newcommand{\key}{\kw{key}}
\newcommand{\rep}{\kw{rep}}
\newcommand{\PEQ}{\kw{EQ2CFD}}
\newcommand{\Drop}{\kw{Drop}}
%\newcommand{\Res}{\kw{Res}}
\newcommand{\CFD}{{\small CFD}\xspace}
\newcommand{\CFDs}{{\small CFD}{\small s}\xspace}
\newcommand{\CIND}{{\sc cind}\xspace}
\newcommand{\cind}{{\small \sf CIND}}
\newcommand{\cfd}{{\small \sf CFD}}
\newcommand{\CINDp}{{\sc cind}$^+$\xspace}
\newcommand{\CINDn}{{\sc cind}$^-$\xspace}
\newcommand{\CINDs}{{\sc cind}{\small s}\xspace}
\newcommand{\FD}{{\small FD}\xspace}
\newcommand{\FDs}{{\small FD}{\small s}\xspace}
\newcommand{\IND}{{\sc ind}\xspace}
\newcommand{\INDs}{{\sc ind}{\small s}\xspace}
\newcommand{\TGDs}{{\sc tgd}{\small s}\xspace}
\newcommand{\NP}{{\small NP}\xspace}
\newcommand{\DTIME}{{\small DTIME}\xspace}
\newcommand{\NPO}{{\small NPO}\xspace}
\newcommand{\APX}{{\small APX}\xspace}
\newcommand{\DAGs}{{\sc dag}s\xspace}
\newcommand{\NC}{{\sc nc}\xspace}
\newcommand{\coNP}{co{\small NP}\xspace}
\newcommand{\PTIME}{{\small PTIME}\xspace}
\newcommand{\PSPACE}{{\sc pspace}\xspace}
\newcommand{\EXPTIME}{{\sc exptime}\xspace}
\newcommand{\NPSPACE}{{\sc npspace}\xspace}
\newcommand{\dom}{\protect\ensuremath{\mathsf{dom}}\xspace}
\newcommand{\atset}{\protect\ensuremath{\mathsf{attr}}\xspace}
\newcommand{\attr}[1]{\protect\ensuremath{\mathsf{#1}}\xspace}
\newcommand{\attrset}{\protect\ensuremath{\mathsf{attr}}\xspace}
\newcommand{\finatset}{\protect\ensuremath{\mathsf{finattr}}\xspace}
\newcommand{\pvar}{\protect\ensuremath{\mathsf{var\%}}\xspace}
\newcommand{\lLHS}{\protect\ensuremath{\mathsf{{\small LHS}}}\xspace}
\newcommand{\RA}{{\small RA}\xspace}
\newcommand{\RBR}{\kw{RBR}}
\newcommand{\SQL}{{\sc sql}\xspace}
\newcommand{\XSLT}{{\sc xslt}\xspace}
\newcommand{\DBMS}{{\sc dbms}\xspace}
\newcommand{\ATG}{{\sc atg}\xspace}
\newcommand{\ATGs}{{\sc atg}{\small s}\xspace}
\newcommand{\EBI}{{\sc ebi}\xspace}
\newcommand{\GO}{{\sc go}\xspace}
\newcommand{\VEC}[1]{{\sc vec}(#1)}
\newcommand{\DAG}{{\sc dag}\xspace}
\newcommand{\XQ}{{\sc xq}\xspace}
\newcommand{\XQwc}{{\sc xq}$^{\scriptscriptstyle[*]}$\xspace}
\newcommand{\XQdes}{{\sc xq}$^{\scriptscriptstyle[//]}$\xspace}
\newcommand{\XQfull}{{\sc xq}$^{\scriptscriptstyle[*,//]}$\xspace}
\newcommand{\vect}[1]{$\langle$ #1 $\rangle$}
\newcommand{\sem}[1]{[\![#1]\!]}
\newcommand{\NN}[2]{#1\sem{#2}}
\newcommand{\e}[2]{{\mathit (#1,#2)}}
\newcommand{\ep}[2]{{\mathit (#1,#2)+}}
\newcommand{\brname}{\ensuremath{{\mathsf{N}}}}
\newcommand{\budrel}[1]{\ensuremath{{\brname_{#1}}}}
\newcommand{\budgen}[2]{\ensuremath{Q^\brname_\e{#1}{#2}}}
\newcommand{\budcut}[2]{\ensuremath{Q_\e{#1}{#2}}}
\newcommand{\eop}{\hspace*{\fill}\mbox{$\Box$}}     % End of proof
\newcounter{example}%[section]
%\newcommand{\theexample}{\arabic{example}}
\newenvironment{example}{
         \vspace{1.5ex}
         \refstepcounter{example}
         {\noindent\bf Example \theexample:}}{
         \eop\vspace{1.5ex}}
\def\copyrightspace{}
\renewcommand{\ni}{\noindent}
\newcommand{\comlore}[1]{\begin{minipage}{3in}\fbox{\fbox{\parbox[t]{3in}{{\vspace{2mm}\noindent \bf COMM(LORE):~
{ #1}\hfill  END.}}}}\end{minipage}\\}
\newcommand{\comwenfei}[1]{\begin{minipage}{3in}\fbox{\fbox{\parbox[t]{3in}{{\vspace{2mm}\noindent \bf COMM(WENFEI):~
{ #1}\hfill  END.}}}}\end{minipage}\\}
\newcommand{\comshuai}[1]{\begin{minipage}{3in}\fbox{\fbox{\parbox[t]{3in}{{\vspace{2mm}\noindent \bf COMM(SHUAI):~
{ #1}\hfill  END.}}}}\end{minipage}\\}
\newcommand{\nthesection}{\arabic{section}}
%\newcounter{problem}
%\newenvironment{problem}{\begin{em}
%        \refstepcounter{problem}
%        {\vspace{1.5ex} \noindent\bf Problem \theproblem:}}{
%        \end{em}\eop\vspace{1.5ex}}
\newcounter{prop}[section]
%\renewcommand{\theprop}{\arabic{theorem}}
%\newcounter{lemma}[section]
%\renewcommand{\thelemma}{\arabic{theorem}}
%\newcounter{cor}[section]
%\renewcommand{\thecor}{\arabic{theorem}}
\newenvironment{ttheorem}{\begin{em}
         \refstepcounter{theorem}
         {\vspace{1.5ex} \noindent\bf  Theorem  \thetheorem:}}{
        \end{em}\eop\vspace{1.5ex}} %\hspace*{\fill}\vspace*{1ex}}
\newenvironment{pprop}{\begin{em}
        \refstepcounter{theorem}
        {\vspace{1.5ex}\noindent \bf Proposition \thetheorem:}}{
        \end{em}\eop\vspace{1.5ex}}%\hspace*{\fill}\vspace*{1ex}}
\newenvironment{llemma}{\begin{em}
         \refstepcounter{theorem}
        {\vspace{1.5ex}\noindent\bf Lemma \thetheorem:}}{
         \end{em}\eop\vspace{1.5ex}} %\hspace*{\fill}\vspace*{1ex}}
\newenvironment{cor}{\begin{em}
        \refstepcounter{theorem}
        {\vspace{1.5ex}\noindent\bf Corollary \thetheorem:}}{
        \end{em}\eop\vspace{1.5ex}} %\hspace*{\fill}\vspace*{1ex}}

%\newcounter{definition}
%\renewcommand{\thedefinition}{\arabic{definition}}
%\newenvironment{definition}{
%        \vspace{1.5ex}
%        \refstepcounter{definition}
%        {\noindent\bf Definition {\bf \thedefinition}:}}{\eop\vspace{1.5ex}
%}
\newcounter{alg}[section]
\renewcommand{\thealg}{\nthesection.\arabic{alg}}
\newenvironment{alg}[1]{
        \refstepcounter{alg}
        {\vspace{1ex}\noindent\bf Algorithm \thealg:\, #1}}{
        \vspace*{1ex}}
\newcounter{arule}
\renewcommand{\thearule}{\arabic{arule}}
\newenvironment{arule}{
        \vspace{0.6ex}
        \refstepcounter{arule}
        {\noindent \em Rule \thearule:}}{
        }
\newcounter{claim}
\renewcommand{\theclaim}{\arabic{claim}}
\newenvironment{claim}{
        \vspace{0.6ex}
        \refstepcounter{claim}
        {\noindent\em Claim \theclaim:}}{%--{ Wenfei Fan}\\
        }
\renewenvironment{proof}{
%\newenvironment{proof}{
        \vspace{0ex}
        {\noindent\bf Proof:}}{\eop\vspace{1ex}}
\newenvironment{proofS}{
        \vspace{1ex}
        {\noindent\bf Proof sketch:\ }}{\eop\vspace{1ex}}

\newcommand{\dist}{\kw{ldist}}
\newcommand{\pSim}{\kw{JoinMatch}}
\newcommand{\spSim}{\kw{SplitMatch}}
\newcommand{\gpq}{\kw{PQ}}
\newcommand{\gpqs}{\kw{PQs}}
\newcommand{\rrq}{\kw{RQ}}
\newcommand{\rrqs}{\kw{RQs}}
\newcommand{\rpe}{\kw{RPE}}
\newcommand{\rpes}{\kw{RPEs}}

\newcommand{\eps}{\trianglelefteq}
\newcommand{\neps}{\ntrianglelefteq}
\newcommand{\ees}{\preceq_{(e,e)}}
\newcommand{\nees}{\not\preceq_{e,e}}
\newcommand{\Reps}{S}

\newcommand{\added}[1]{\textcolor{blue}{#1}}
\newcommand{\changed}[1]{\textcolor{red}{#1}}
\newcommand{\removed}[1]{\textcolor{gray}{#1}}

\newcommand{\ret}{\kw{ret}}
\newcommand{\remv}{\kw{premv}}
\newcommand{\presim}{\kw{amat}}
\newcommand{\prev}{\kw{prev}}
\newcommand{\subiso}{\kw{SubIso}}

\newcommand{\ssim}{\kw{mat}}
\newcommand{\join}{\kw{Join}}
\newcommand{\nor}{\kw{Normalize}}
\renewcommand{\split}{\kw{Split}}
\newcommand{\sccg}{\kw{Sccgraph}}
\newcommand{\rmv}{\kw{rmv}}
\newcommand{\block}{{\cal B}}
\newcommand{\rel}{\kw{rel}}
\newcommand{\partition}{\kw{par}}
\newcommand{\cpath}{{\em c}-path\xspace}
\newcommand{\cpaths}{{\em c}-paths\xspace}
\newcommand{\psimset}{\kw{Psim}}



\newcommand{\vn}{\kw{VN}}
\newcommand{\vns}{\kw{VNs}}
\newcommand{\sns}{\kw{SNs}}
\newcommand{\vm}{\kw{VM}}
\newcommand{\vms}{\kw{VMs}}
\newcommand{\vmp}{\kw{VMP}}
\newcommand{\sn}{\kw{SN}}
\newcommand{\vne}{\kw{VNE}}

\newcommand{\buildAug}{\kw{compAuxGraph}}
\newcommand{\minVN}{\kw{minVN}}
\newcommand{\compMap}{\kw{compVNM}}
\newcommand{\compMapNS}{\kw{compVNM_{NS}}}
\newcommand{\PTAS}{{\small PTAS}\xspace}
\newcommand{\APTAS}{{\small APTAS}\xspace}
\newcommand{\VM}{\kw{VM}}
\newcommand{\vine}{\kw{ViNE}}
\newcommand{\vineNS}{\kw{ViNE_{NS}}}
\newcommand{\rwsp}{\kw{RW}-\kw{SP}}
\newcommand{\lvb}{\{\!|}
\newcommand{\rvb}{|\!\}}
%% APPENDIX

\newcommand{\gap}{\kw{GAP}}
\newcommand{\rgap}{\kw{RGAP}}
\newcommand{\subgIso}{\kw{Subgraph} \kw{Isomorphism}}
\newcommand{\xtc}{\kw{X3C}}
\newcommand{\binpack}{\kw{Bin} \kw{Packing}}
\newcommand{\parti}{\kw{PARTITION}}
\newcommand{\mwsat}{\kw{Minimum} \kw{Weight} \kw{3SAT}}
\newcommand{\edp}{\kw{EDP}}
\newcommand{\att}{\SIM}
\newcommand{\swsf}{\kw{SWSF\_FP}}

\newcommand{\warn}[1]{\textcolor{red}{#1}}
\newcommand{\revise}[1]{\textcolor{blue}{#1}}
\newcommand{\marked}[1]{\revise{#1}}


%%%%%%%%%%%%%%%%%%%%%%%%%%%%%%%Data sets%%%%%%%%%%%%%%%%%
\newcommand{\taxi}{\kw{Taxi}}
\newcommand{\sercar}{\kw{ServiceCar}}
\newcommand{\pricar}{\kw{PrivateCar}}
\newcommand{\geolife}{\kw{GeoLife}}
\newcommand{\didi}{\kw{Didi}}
\newcommand{\pubdata}{\kw{Public Data}}

%%%%%%%%%%%%%%%%%%%%%%%%%%%%%%% algorithms %%%%%%%%%%%%%%%%%
\newcommand{\gfbased}{\kw{GF}-\kw{MM}}
\newcommand{\hmmbased}{\kw{Basic}-\kw{HMM}}
\newcommand{\stmm}{\kw{TSA}-\kw{MM}}
\newcommand{\cised}{\kw{CISED}}
\newcommand{\siped}{\kw{SIPED}}
\newcommand{\ped}{\kw{PED}} %perpendicular Euclidean distance (PED).
\newcommand{\sed}{\kw{SED}} %synchronous Euclidean distance (SED).

\newcommand{\trajec}[1]{$\dddot{\mathcal{#1}}$}
\newcommand{\ffunc}[1]{{\mathbb{#1}}}
\newcommand{\sstab}{\vspace{0.5ex}\noindent}

\newcommand{\myfig}[1]{\textcolor{blue}{Figure~\ref{#1}}}
\newcommand{\todo}[1]{\textcolor{red}{Todo...#1}}
\newcommand{\myred}[1]{\textcolor{red}{#1}}
\newcommand{\myblue}[1]{\textcolor{blue}{#1}}


% Copyright
%\setcopyright{none}
\setcopyright{acmcopyright}
%\setcopyright{acmlicensed}
%\setcopyright{rightsretained}
%\setcopyright{usgov}
%\setcopyright{usgovmixed}
%\setcopyright{cagov}
%\setcopyright{cagovmixed}





% Copyright
\setcopyright{rightsretained}

% DOI
\acmDOI{10.475/123_4}

% ISBN
\acmISBN{XXX-X-XXXXX-XXX-X}

%Conference
\acmConference[WWW 2019]{International World Wide Web Conference}{May 13-17, 2019}{San Francisco, California, USA}
\acmYear{2019}

\settopmatter{printacmref=false, printccs=false, printfolios=false}

\pagestyle{empty} % removes running headers



\begin{document}
%\title{Map Matching On Simplified Trajectories}
\title{Trajectory Simplification Aware Map-matching}
%\titlenote{Produces the permission block, and copyright information}
% \subtitle{Extended Abstract}
%\subtitlenote{The full version of the author's guide is available as  \texttt{acmart.pdf} document}


% \author{Jiahao Jiang,  Xuelian Lin, tianyu Wo and Shuai Ma}
% \affiliation{%
%   \institution{Beijing Advanced Innovation Center for Big Data and Brain
% Computing, Beihang University}
%   \streetaddress{37th XueYuan Road}
%   \city{Beijing}
%   \country{China}
%   \postcode{100191}
% }
% \email{{jiangjh, linxl, woty, mashuai}@buaa.edu.cn}

\author{Anonymous Author(s)}
\affiliation{%
  \institution{Institution}
  \streetaddress{Street address}
  \city{City}
  \country{Country}
  \postcode{postcode}
}
\email{email}





% The default list of authors is too long for headers.
% \renewcommand{\shortauthors}{J. Jiang et al.}
\renewcommand{\shortauthors}{XXX et al.}


\begin{abstract}
Map-matching is a {widely} used method that matches trajectory
data points to road segments of a road network. It is also an important
preprocessing step for many location-based services. However, the previous
map-matching methods are designed for raw trajectories and overlook the
{characteristics} of simplified trajectories which are normally small subsets of the raw trajectories produced by trajectory simplification algorithms. Thus, they have limited effectiveness and efficiency when they match simplified trajectories to roads.
%Recently, various mobile devices have been used to collect and upload tremendous trajectories, and it is known that raw trajectories waste the storage, network bandwidth and computing resources. Hence, it is a tendency that these raw trajectories are simplified by trajectory simplification methods to reduce the size of trajectories and save resources.
To tackle this problem, in this paper, we proposed a novel trajectory
simplification aware map-matching method that seriously considers the
{characteristics} of simplified trajectories, such as error bound and the distribution of raw trajectory points, so as to achieve better accuracy of map-matching.
We implemented the method based on Hidden Markov model and compared the accuracy and running time of the method with two state-of-the-art map-matching algorithms on two real datasets. The experimental results demonstrate the effectiveness and efficiency of the proposed method.
\end{abstract}





%
% The code below should be generated by the tool at
% http://dl.acm.org/ccs.cfm
% Please copy and paste the code instead of the example below.
%
% \begin{CCSXML}
% <ccs2012>
%  <concept>
%   <concept_id>10010520.10010553.10010562</concept_id>
%   <concept_desc>Computer systems organization~Embedded systems</concept_desc>
%   <concept_significance>500</concept_significance>
%  </concept>
%  <concept>
%   <concept_id>10010520.10010575.10010755</concept_id>
%   <concept_desc>Computer systems organization~Redundancy</concept_desc>
%   <concept_significance>300</concept_significance>
%  </concept>
%  <concept>
%   <concept_id>10010520.10010553.10010554</concept_id>
%   <concept_desc>Computer systems organization~Robotics</concept_desc>
%   <concept_significance>100</concept_significance>
%  </concept>
%  <concept>
%   <concept_id>10003033.10003083.10003095</concept_id>
%   <concept_desc>Networks~Network reliability</concept_desc>
%   <concept_significance>100</concept_significance>
%  </concept>
% </ccs2012>
% \end{CCSXML}

% \ccsdesc[500]{Computer systems organization~Embedded systems}
% \ccsdesc[300]{Computer systems organization~Redundancy}
% \ccsdesc{Computer systems organization~Robotics}
% \ccsdesc[100]{Networks~Network reliability}


% \keywords{Map matching, trajectory compression, HMM}

\maketitle


%%% Local Variables:
%%% mode: latex
%%% TeX-master: "gis18"
%%% End:

\section{introduction}
\label{sec-intro}


\textit{Trajectory tracking} \cite{Lange:Tracking} is a combination of \textit{position tracking} \cite{Wolfson:PositionTracking,Leonhardi:Comparison} and \textit{trajectory simplification} \cite{Lin:Cised,Zhang:Evaluation} in one routine, where \textit{position tracking} is an approach that lets the moving objects database (MOD) server know the current position of a moving object effectively and efficiently, that is, it achieves the desired accuracy of the location information on the server by transmitting as few messages as possible \cite{Leonhardi:Comparison}. Linear dead reckoning (\ldr) \cite{Wolfson:PositionTracking} is such a widely used position tracking method, which is essentially an agreement between a given moving object and a MOD server such that the server could infer the current, excepted position of the moving object whose distance to the actual position of the object is bounded by a user specified threshold;
%
and \textit{trajectory simplification} \cite{Lin:Cised,Zhang:Evaluation} is to approximate a fine trajectory with a coarse one (whose corresponding data points are a subset of the original one), such that the size of the trajectory is reduced under a constrain that the maximum distance of the former to the latter is bounded by a user specified threshold. 
%Linear simplification \cite{Lin:Cised,Zhang:Evaluation} is such an effective and efficient approach that is also widely used in practice.
%
Position tracking and trajectory simplification both are the fundamental technologies of trajectory management and they also share some common target and strategy, \ie, reduce the number of messages or the size of trajectory data by discarding some location information that seems not that important, hence, researchers are trying to combine them in one routine and make it be suitable to run in resource constraint devices.

The authors of \cite{Trajcevski:LDRH} find that the position tracking algorithm \ldr with some tiny modifications is applicable to both track the positions of a moving object and simplify the trajectory built out of these positions. The modified \ldr,  called \ldrh in \cite{Lange:Tracking}, is the first trajectory tracking algorithm that combines position tracking and trajectory simplification into one consistent process. It is concise and efficient, and is suitable for mobile devices. However, it suffers in effectiveness in terms of compression ratio and communication cost, due to the nature of \ldr. 
%
Then, a framework, named the generic remote trajectory simplification (GRTS) \cite{Lange:GRTS,Lange:Tracking}, is developed to improve the effectiveness of trajectory tracking by separate position tracking and trajectory simplification into two sub-processes, where the positions of a moving object is also tracked by \ldr, and these positions are temporarily saved in a buffer and then simplified by some third-party line simplification algorithm. Indeed, it is more effective than \ldrh at a cost of weakening the conciseness and efficiency of \ldrh.
%



\stitle{\todo{Motivations}.}

\ni(1) Trajectory track algorithms are supposed to run in resource-constraint mobile devices, thus, besides good performance of efficiency and effectiveness, they should also be simple and light, \ie having low time and space complexities, otherwise, they are not suitable to run in those mobile devices. In response to these requirements, \ldrh is light, simple and efficient, but not effective; and \grts is effective, but not efficient and light enough. That is, neither of them is the ideal solution for trajectory tracking.
%The emerging of one pass trajectory simplification algorithms. These algorithms can be integrated into grts, however, it is not a natural way to implement a one-pass trajectory tracking algorithm like this way. Acutually, one pass position tracking + one pass trajectory simplification = one pass and effective trajectory tracking algorithm......co-design, like LDRH, yet more effective.


\ni(2) The current works, \ie~\ldrh and \grts, only compress a trajectory or track a moving object in circular areas, \ie the moving object is supposed to locate in a circular taking the expected position of the object as the center. However, in practical, there is a need to track moving objects in other areas, such as strip or rectangular-like areas. \todo{examples and figures of areas,}





\stitle{\todo{Contributions}.}
To the end, we design ways for trajectory tracking in varied areas, including strip and combined areas, and provide three novel one-pass algorithms tracking moving objects effectively and efficiently. 

1. one-pass tracking moving object in circular, citt, effectively and efficiently.

2. one-pass tracking in strips using ped. sitt.
a way that customize region by sed and ped. and implement it in position tracking LDR and trajectory tracking framework GRTS. advantage...

3. one-pass tracking in combined areas using sed and ped. bitt.  
A one-pass trajectory tracking algorithm supporting sed and ped, by a combination cone intersection and sector intersection, \ie co-design of position tracking and trajectory simplification, effective and low time and space complexity, suitable running in resource constraint devices.

4. experiments

\stitle{{Organization}}.
The remainder of the paper is organized as follows:
Section \ref{sec-pre} introduces the basic concepts and the basic HMM method,
Section \ref{sec-method} presents our trajectory simplification aware map-matching method,
Section \ref{sec-exp} reports the experimental results of these methods, followed by related works in Section \ref{sec-related} and conclusion in Section \ref{sec-conclusion}.





%%% Local Variables:
%%% mode: latex
%%% TeX-master: "gis18"
%%% End:



\section{Preliminaries}
\label{sec-pre}




In this section, we first introduce the concepts on simplified trajectories and map-matching, then we introduce the basic HMM-based map-matching method that serves as the fundamental of the work.
%, followed by statement the problem of map-matching on simplified trajectories.

\subsection{Notations}


\stitle{Points ($P$)}. A GPS point is defined as a triple $P(x, y, t)$,
which represents that a moving object is located at {\em longitude} $x$ and {\em
  latitude} $y$ at {\em time} $t$.

\stitle{Trajectories ($\dddot{\mathcal{T}}$)}. A trajectory
$\dddot{\mathcal{T}}[P_0, \ldots, P_n]$ is a sequence of data points in a
monotonically increasing order of their associated time values ($P_i.t <
P_j.t$ for any $0\le i<j\le n$). Intuitively, a trajectory is the path (or
track) that a moving object follows through space as a function of time~\cite{physics-trajectory}.


\eat{
\stitle{Simplified line segments ($\mathcal{L}$)}. A Simplified line segment (or
line segment for simplicity) $\mathcal{L}$ is  defined as $\vv{P_{s}P_{e}}$,
which represents the  closed line segment that connects the start point $P_s$ and the end point $P_e$.
There are also two attributes $\mathcal{L}.L_p$ and $\mathcal{L}.L_n$
representing the length of raw trajectory on each side of the simplified line
segment respectively.
}

\stitle{Simplified trajectories ($\overline{\mathcal{T}}$)}. A simplified trajectory $\overline{\mathcal{T}}[\mathcal{L}_0, \ldots , \mathcal{L}_m]$ ($0< m \le n$) of a trajectory $\dddot{\mathcal{T}}[P_0, \ldots, P_n]$ is a sequence of continuous directed line segments $\mathcal{L}_{i}$ = $\vv{P_{s_i}P_{e_i}}$ ($i\in[0,m]$) of $\dddot{\mathcal{T}}$  such that $\mathcal{L}_{0}.P_{s_0} = P_0$, $\mathcal{L}_{m}.P_{e_m} = P_n$ and  $\mathcal{L}_{i}.P_{e_i}$ = $\mathcal{L}_{i+1}.P_{s_{i+1}}$ for all $i\in[0, m-1]$.
Note that (1) each directed line segment in $\overline{\mathcal{T}}$ essentially represents a continuous sequence of data points in $\dddot{\mathcal{T}}$, and
(2) the simplified trajectories are referred to {as} error bounded if for each point $P$ in \trajec{T}, there exist points $P_j$ and $P_{j+1}$ in $\overline{\mathcal{T}}$ such that the distance from $P$ to $\mathcal{L}(P_j,P_{j+1}))$ is less than $\epsilon$.
%error bounded by $\epsilon$ if

\eat{
\stitle{Error bounded trajectory simplification}. Given a trajectory \trajec{T}, an error bound $\epsilon$ and a simplification algorithm $\mathcal{A}$ that produces another trajectory \trajec{T'},
we say that algorithm $\mathcal{A}$ is error bounded by $\epsilon$ if  for each point $P$ in \trajec{T}, there exist points $P_j$ and $P_{j+1}$ in \trajec{T'} such that the distance from $P$ to $\mathcal{L}(P_j,P_{j+1}))$ is less than $\epsilon$.
}



%\subsection{Terms on map-matching}



\stitle{Road segments ($r$)}. A road segment is defined as $r = (v_s,v_e)$, representing an edge directly connecting two ending
points in the map.



\eat{
\stitle{Candidate Road Sets ($C$)}. A candidate road set (candidate set in short) $C_i = \{r_i^1,r_i^2,\ldots,r_i^k\}$ of a GPS point $P_i$
is a set of road segments that are close to the point. The final
matched road segment is selected from the candidate set.
%In this paper, we set the search range as a circle centered at point $P_i$ with radius as 200 meters.
}

\stitle{Routes ($R$)}. A route $R = {[r_0, \ldots,r_m]}$ is a continuous sequence
of road segment such that $r_i.v_e = r_{i+1}.v_s$, $0\le i<m$.

\stitle{Road network ($G$)}. A road network is a directed graph $G(V,E)$ where $V$ is the set of junction points of roads and $E$
is the set of road segments between two junction points.

\stitle{Map-matching}. Given a (simplified) trajectory of a user and a road network, the goal of (trajectory simplification aware) map-matching is to find the most likely route in the road network that has been traveled by the user.




\subsection{HMM-based Map-matching}
In recent years, map-matching is always modeled as a sequence labeling problem and tackled using sequence models such as HMM.
The authors of \cite{Lamb1999Avoiding} first introduce HMM for map-matching, then a number of works \cite{Newson2009Hidden, Wang:eddy, Osogami:2013:IRL, yin:feature-based} follow this idea.
%
In the modeling of HMM-based map-matching, road segments are \emph{hidden states} and GPS points are \emph{observations}.
For example, in \myfig{fig:hmm-model-a}, GPS points $P_1,P_2,P_3$ are observations of the moving object at timestamps $T_1,T_2,T_3$, respectively,
and $r_1^1$ and $r_1^2$, two {candidate road segments} of point $P_1$, are the hidden states of the moving object at timestamp $T_1$.
Moreover, the likelihood of the GPS point residing in a road segment is described by \emph{emission probability} ($E$). For instance, in \myfig{fig:hmm-model-b}, the emission probability of point $P_1$ on road segment {$r_1^2$ is $E_1^2$}.

Then, the map-matching of a sub-trajectory to a road network is
modeled as a weighted directed graph (\myfig{fig:hmm-model-b}), where a vertex is a hidden state (candidate road segment), an edge is the transition from the previous hidden state to the next hidden state, and the weight of an edge, named \emph{transition probability} ($T$), is the probability that the moving object transitions from one road segment to another. For example, {$T_{2}^3$} is the transition probability from {$r_1^2$ to $r_2^3$}.

Finally, the probability of a sub-trajectory \trajec{T}$[P_s, \ldots, P_{s+u}]$ matched to a route $R$ is defined as the joint probability $J(\dddot{\mathcal{T}}, R) = \prod_{i=1}^u{T(r_{s+i}|r_{s+i-1})\cdot E(P_{s+i}|r_{s+i})}$, $P\in \dddot{\mathcal{T}}$ and $r\in R$, and a path in the graph with the highest joint probability is the matched route of the sub-trajectory.
Note that most HMM-based methods share the same model except that they have respective definitions of transition probabilities.
Our \stmm also follows this common model and has specific definition of
transition probability for simplified {trajectories}.

%\begin{equation}
%  \label{equ:joint-prob}
%  P(R,T) = \prod_{i=1}^n{P(r_i|r_{i-1})\cdot P(P_i|r_i)}
%\end{equation}


\begin{figure}[tb!]
  \centering
  \begin{subfigure}{0.4\textwidth}
    \includegraphics[width = \textwidth]{Figures/Fig-HMM-model-road.pdf}
    \caption{finding the candidate road segments.}\label{fig:hmm-model-a}
    \vspace{1ex}
  \end{subfigure}
  \begin{subfigure}{0.42\textwidth}
    \includegraphics[width = \textwidth, height = 0.6\textwidth]{Figures/Fig-HMM-model.pdf}
    \caption{finding the optimal route. }\label{fig:hmm-model-b}
  \end{subfigure}
  \vspace{-1ex}
  \caption{HMM-based map-matching.}
  \label{fig:hmm-model}
 \vspace{-4ex}
\end{figure}




%\subsection{Problem statement}
%Given a simplified trajectory and a road network($G(V,E)$), the goal of map-matching on simplified trajectories is to find the most likely route ($R$) in the road network that has been traveled by the user.





\section{Method}
\label{sec-method}

In this section, we present our {trajectory simplification} aware map-matching method.


\begin{figure*}[htb!]
	\centering
  \includegraphics[width=0.89\textwidth, height  = 0.22\textwidth]{Figures/Fig-Architecture-en.pdf}\hspace{1ex}
	\vspace{-1ex}
	\caption{The framework of trajectory simplification aware map matching.}
	\label{fig:sys-arc}
	\vspace{-1ex}
\end{figure*}


\subsection{Overview of the Method}
The framework of {trajectory simplification} aware map-matching (\stmm) is illustrated in \myfig{fig:sys-arc}.
\stmm takes as input a raw trajectory and simplifies the raw trajectory by the trajectory simplification component. The output of the component is then injected into the HMM-based map-matching method, including two components, \emph{local path recovery} and \emph{global route decoding}, each utilizes the unique characteristics of the simplified trajectory to improve the accuracies and running time of map-matching.

\stitle{(1) Trajectory simplification.}
It simplifies the input stream of raw trajectory data points by dropping redundant points and keeping significant ones with an error bound. The output of the algorithm is a sequence of directed line segments associated with auxiliary information that is used in the consequent map-matching.

\stitle{(2) Local path recovery.}
This component is used to generate the local optimal path given any two neighboring points of a simplified trajectory.
It operates on an action graph, which is a weighted graph extracted from road network incorporating information from the raw trajectory, and is used to describe the actions of a user.  The optimal paths can be detected through the shortest path searching in the graph. Note when estimating the edge weights of an action graph, we consider the geometry similarities between the raw/simplified trajectories and the paths to better recover the local paths.

%When estimating the weights of edges of an action graph, we consider the distribution of the raw trajectory points in the simplified trajectories to better recover the local path.

\stitle{(3) Global route decoding.}
It is another core component of the HMM-based method. It computes the probability of each candidate route after the local optimal paths recovery, and finds the global optimal route through dynamic programming.
Also, we take the geometry similarity between the raw/simplified trajectories and the routes into considerations in the computation of transition probabilities.
%Based on the local optimal paths generated by Path recovery component,


\subsection{Trajectory Simplification}
\label{sec:simp}


\todo{what and why}

We choose two one-pass algorithms, \ie~\siped \cite{Zhao:Sleeve} using \ped and \cised\cite{Lin:Cised} using \sed, to carry out trajectory
simplification. \ped and \sed are two distance metrics widely used in trajectory simplification algorithms, more details about them please refer to \cite{Lin:Cised, Zhang:Evaluation}.

\todo{full epsilon of siped and cised, plus auxiliary information}

For a sub-trajectory $\dddot{\mathcal{T}}[P_s, ..., P_{s+u}]$, $u\ge 1$, simplified to a line segment $\vv{P_{s}P_{s+u}}$, \stmm also computes the auxiliary information of $\vv{P_{s}P_{s+u}}$, including $L_L$ and $L_R$, the length of raw sub-trajectory $\dddot{\mathcal{T}}[P_s, ..., P_{s+u}]$ on the left and right sides of the simplified line segment $\vv{P_{s}P_{s+u}}$, respectively. 
%
%The distribution of the raw sub-trajectory points according to a simplified line segment can be calculated and saved in advance during trajectory simplification.
Indeed, the distribution of raw trajectory points \wrt the simplified line segments is quite uneven.
%\myfig{fig:traj-sides} is a typical example of the uneven distribution phenomenon, where green lines are simplified line segments and blue
%points are raw trajectory points, and
\myfig{fig:traj-side-stat} shows the percentage of points on one side of the simplified line segments, where more than $85\%$ of raw trajectory points are located on one side of the simplified line segments.
This information is a hint to select the right path/route, hence, it is used in the local path recovery and the global route decoding.

\begin{figure}
  \centering
  \begin{subfigure}{0.34\textwidth}
    \centering
    \includegraphics[width = \textwidth, height = 0.66\textwidth]{Figures/Exp-statistic-side-ratio.png}
  \end{subfigure}
  \vspace{-2ex}
  \caption{\small Distribution of trajectory points.} \vspace{-3ex}
  \label{fig:traj-side-stat}
  \vspace{1ex}
\end{figure}



%%%%%%%%%%%%%%%%%%%%%%%%%%%%%%%%%%%%%%%%%%%%%%%%%%%%%%%%%%%%%%%%%%%%%%%%%%%%%%%%%%%

\subsection{Local Path Recovery}
\label{sec:route}

%The local path recovery module is used to select a most possible path between two neighbouring points of a simplified trajectory.
The simplified trajectories are sparser than the original ones, which leads to higher uncertainties of map-matching.
On the other hand, as pointed out in Section \ref{sec-intro}, the simplified trajectories have unique attributes that can be utilized to improve the accuracy of map-matching.
%\subsubsection{Subgraph Extraction}

First of all, the simplified trajectories are error-bounded. As illustrated in
\myfig{fig:subgraph}, the error bound of the trajectory simplification algorithm defines a range in which the raw
trajectory points may reside. We can extract a small part of graph $G_S(V_S,E_S)$
from the original road network $G(V,E)$ and execute the matching process in the subgraph. This strategy shrinks the searching range and
improves efficiency.
More specifically, we set the range of subgraph as a rectangle range
with the simplified line segment $\mathcal{L}$ as the axis of symmetry, width as $w =
2\times(\epsilon + r_S)$ and length as $l = \mathcal{L}.L_L + \mathcal{L}.L_R +
2\times r_S$, in which $\epsilon$ is the error bound used in trajectory
simplification and $r_S$ is the searching radius used for candidate paths selection.




%\subsubsection{Action Graph Construction and Weight Estimation}
% Traditional map matching algorithms carry out local route recovery through
% shortest path searching directly in road network($G(V,E)$). However, because the simplified
% trajectories are sparse, the results achieved from shortest path
% searching will not lead to a reasonable route. Hence, other factors considering
% the rationality of routes should be taken into consideration.



Then, we extract an \emph{\emph{action graph}} from the road network incorporating the information summarized from the raw and simplified trajectories,  and use it to describe the actions of the user.
%The optimal pathes can be detected through the shortest path searching in the graph.
%An action graph is a graph extracted from road network, Osogami and Raymond first proposed it in \cite{Osogami:2013:IRL}.
The action graph is first proposed in \cite{Osogami:2013:IRL}, as shown in \myfig{fig:action-graph}, a node in the action graph represents  a
road segment, and an edge describes an action of travelling from one road segment to a neighboring one. An edge is also
associated with a weight representing the possibility of taking this action, which can be estimated from the features of the
trajectory and the path.
%
In the estimation of action weight, besides the length of road segment and the turning angle between two road segments, we further consider the similarity between the raw sub-trajectory and the path between two neighbouring points of a simplified trajectory, by computing both the distribution of the raw sub-trajectory data points and the ratio of path located on both sides of the simplified line segment.
%
Obviously, if we select a path having similar distribution with the raw sub-trajectory \wrt a simplified line segment, then the accuracy of map-matching should be improved.
%
To sum up, we estimate the weight of an action by the sum of three terms (Equation~\ref{equ:cost}): %according to the distribution of raw trajectory points by
\begin{equation}
    %\vspace{-2ex}
    \omega = \omega_{L} + \alpha \times \omega_{T} + \beta \times \omega_{\phi}
    \label{equ:cost}
\end{equation}
where $\omega_{L}$ is the length of the ending road segment of the action,
$\omega_{T}$ is the cost of turning from the starting road segment to the ending
one in the action, $\omega_{\phi}$ is the similarly
between two distributions, \ie the distributions of the path and the raw sub-trajectory on the two sides of a simplified
line segment, and $\alpha$ and $\beta$ are two user defined parameters.

\begin{figure}
%\vspace{1ex}
  \begin{subfigure}{0.36\textwidth}
  \centering
  \includegraphics[width = \textwidth, height = 0.6\textwidth]{Figures/Fig-subgraph.png}
  \end{subfigure}
  \vspace{-2ex}
  \caption{\small {A subgraph of a road network.}}
  \label{fig:subgraph}
\vspace{-2ex}
\end{figure}



%%%%%%%%%%%%%%%%%%%%%%%%%%%%%%%%%%%%%%%%%%%%%%%%%
\eat{
\begin{figure}
  \begin{subfigure}{0.36\textwidth}
    \includegraphics[width = \textwidth, height = 0.55\textwidth]{Figures/Fig-traj-side.png}
  \end{subfigure}
  \vspace{-2ex}
  \caption{\small {A sub trajectory is simplified to a line segment, which splits the plane into two parts and most points of the sub trajectory concentrate on one side.}}\vspace{-2ex}
  \label{fig:traj-sides}
\end{figure}
}

More specifically, the cost of turning $\omega_{T}$ is estimated by Equation \ref{equ:turning} defined in \cite{Osogami:2013:IRL}:
\begin{equation}
  \omega_{T} = \left\{
    \begin{aligned}
      0 & \ \ & \theta_{s,e} < \pi / 4 \\
      1 & \ \ & \pi / 4 \le \theta_{s,e} < 3\pi /4 \\
      2 & \ \ & 3\pi / 4 \le \theta_{s,e} \le \pi  \\
    \end{aligned}
  \right.
  \label{equ:turning}
\end{equation}
where, $\theta_{s,e}$ is the turning angle from the starting road segment of the action to the ending one.


The similarity estimation $\omega_{\phi}$ is modeled using a
piecewise function (Equation \ref{equ:sim}). The intuition behind this is that the correct path should have similar ratio of travelling distance on each side of the simplified line segment as the raw trajectory.
%We compute the distance of road segments residing on {one} side of the simplified trajectory and {compare it with the distribution of the corresponding raw sub-trajectory}.

\begin{equation}
  \omega_{\phi} = \left\{
    \begin{aligned}
      1 & \ \ & (\phi_T < 0.25 \wedge \phi_{R} < 0.25) \\
      1 & \ \ & (\phi_T > 0.75 \wedge \phi_{R} > 0.75) \\
      10 & \ \ & (\phi_T < 0.25 \wedge \phi_{R} > 0.75) \\
      10 & \ \ & (\phi_T > 0.75 \wedge \phi_{R} < 0.25) \\
      5 & \ \ & otherwise \\
    \end{aligned}
  \right.
  \label{equ:sim}
\end{equation}
where $\phi_T= \frac{\mathcal{L}.L_L}{\mathcal{L}.L_L + \mathcal{L}.L_R}$ is the ratio of trajectory length residing in the left side of the
simplified line segments, and $\phi_R= \frac{\sum_{r_j \in R}r_j.L_L}{\sum_{r_j \in R}r_j.L_L + r_j.L_R}$ is that of the {roads/path}.

%\begin{equation}
%\phi_L = \frac{\mathcal{L}.L_P}{\mathcal{L}.L_P + \mathcal{L}.L_N}
%\end{equation}

%\begin{equation}
%\phi_R = \frac{\sum_{r_j \in R}r_j.L_P}{\sum_{r_j \in R}r_j.L_P + r_j.L_N}
%\end{equation}

Finally, {a local optimal path can be detected by shortest path search on the action graph.}
%by minimizes the action weight .

\begin{figure}
    \begin{subfigure}{0.36\textwidth}
        \centering
        \includegraphics[width = \textwidth, height = 0.6\textwidth]{Figures/Fig-action-graph-en.pdf}
    \end{subfigure}
    \vspace{-2ex}
    \caption{\small {An example of action graph.}}\label{fig:action-graph}
    \vspace{-3ex}
\end{figure}



%%% Local Variables:
%%% mode: latex
%%% TeX-master: "gis18"
%%% End:
\subsection{Global Route Decoding}
\label{sec:hmm}

We use hidden markov model to find the global optimal route given the local
optimal paths produced by the local path recovery component.
The key of HMM modeling is to define two probabilities, \ie the \emph{emission probabilities} and the \emph{transition probabilities}.
%
A emission probability gives the likelihood that an observation is resulted from
a given state. We adopt the widely used {emission probability estimation} proposed by Newson and Krumm in \cite{Newson2009Hidden}. Note the candidate road segments are selected under the distance constraint, which means that a point would not be matched to a road segment having a distance large than the error bound.

\eat{
\subsubsection{Emission Probabilities}
Emission probabilities give the likelihood that an observation is resulted from a given state. For map-matching, it is likely that the vehicle with a GPS point $P_i$ is on a specific road segment $r_i^j$.
For those candidate road segments close to the GPS point, the emission probability is
dependent on the distance between the GPS point and the road segment, and is estimated by Equation~\ref{equ:emi-prob} proposed by Newson and Krumm \cite{Newson2009Hidden}:
%, using a Gaussian kernel. Specifically, emission probability is modeled as:
\begin{equation}
  \label{equ:emi-prob}
  E(r_i^k| P_i) = \frac{1}{\sqrt{2\pi}\sigma} \exp \frac{d(r_i^k, P_i)^2}{2\sigma^2}
\end{equation}
where, $d(r_i^k, P_i)$ is the great circle distance on the surface of the earth between the observed location $p_i$ and the candidate road segment $r_i^k$. And
$\sigma$ is the standard length of GPS measurements, which can be estimated
from data.
}



%\subsubsection{Transition Probabilities}
A transition probability is the probability of an object moving from
one road segment to another. The appropriate definition of transition
probability is the key of HMM modeling.
In Newson and Krumm's modeling, transition probabilities are modeled based on
the difference of great circle distance of two observed locations and their
corresponding road segments.
% The intuition is that transitions whose driving
% distance is about the same as the great circle distance is more likely to be the
% actual route traveled.
This modeling is reasonable when trajectory is relatively dense, i.e.
distance between two neighboring points is small. However, when a trajectory is sparse,
it is possible that more than one route have similar driving distances as the great
circle distance, thus, roughly choosing the one with the smallest difference may lead
to a circuitous route.
%
To correctly handle this problem, we make use of information from
simplified lines, and propose a model of transition probability based on
similarity.
%
Specifically, we define the transition probability $T$ of moving from road segment
$r_{i-1}^j$ to $r_i^k$ as :
\begin{equation}
  \label{equ:trans-prob}
  T(r_i^k| r_{i-1}^j) = \lambda_De^{-\lambda_D\delta_D}\lambda_Re^{-\lambda_R\delta_R}
\end{equation}
%
\begin{equation}
  \delta_R = |\phi_T -\phi_R|
\end{equation}
%
\begin{equation}
  \delta_D = d_T(P_i,P_{i-1}) - d_R(P_i,P_{i-1})
\end{equation}
%
\begin{equation}
  d_T(P_i,P_{i-1}) = \overline{\mathcal{T}}[i].L_L + \overline{\mathcal{T}}[i].L_R
\end{equation}
%
\begin{equation}
  d_R(P_i,P_{i-1}) =  \sum_{r_j \in R_{i-1, i}}{(r_j.L_L + r_j.L_R)}
\end{equation}
%
where $d_T(P_i,P_{i-1})$ is the length of the raw
sub-trajectory, $d_R(P_i,P_{i-1})$ is the length of the local optimal path $R_{i-1, i}$ on road network detected by the local path recovery component, $\delta_D $ is the
difference of $d_T$ and $d_R$, and $\delta_R $ is the difference of the
ratio of trajectory length residing in the left side of the
simplified line segment and that of the path.

\eat{
%After we estimated the emission probabilities and transition probabilities by Equations \ref{equ:emi-prob} and \ref{equ:trans-prob},
We then use the Viterbi algorithm to search for the optimal route. The Viterbi algorithm is a dynamic programming
algorithm that can quickly detect a sequence of states that maximizes the joint
probability, which is the product of the emission probabilities and transition
probabilities of all the states in the sequence.
The detected sequence is the route with maximum likelihood and thus the global optimal route.
}



% \subsection{Algorithm}

With emission probabilities and transition probabilities estimated from Equations
\ref{equ:emi-prob} and \ref{equ:trans-prob}, we can use the Viterbi algorithm to
compute the optimal path. The Viterbi algorithm is a dynamic programming
algorithm that can quickly detect a sequence of states that maximizes the joint
probability, which is the product of the emission probabilities and transition
probabilities of all the states in the sequence. The detected sequence is the
path with maximum likelihood and thus the global optimal path.



\begin{large}
\begin{algorithm}
\caption{The CT-MM Algorithm}\label{alg:viterbi}
\small
\begin{algorithmic}[1]
 \State  \textbf{Input}: $\overline{\mathcal{T}}$,$\epsilon$,G(V,E)
 \State  \textbf{Output}: map matching result R

 \State $E_1 \gets findCand(G(V,E),p_1)$.

 \State \textbf{Initialize} $f[r_1^k] = p(r_{1}^k|p_{1}), k = 1,2,\cdots,10$.

 % \For{each line segments in $\overline{\mathcal{T}}$}
 \For{$i = 2 \to n$}
   % \State extract candidate set of that GPS point from road network.
   \State $P_i \gets \overline{\mathcal{T}}[i].e$
   \State $P_{i-1} \gets \overline{\mathcal{T}}[i].s$
   \State $E_i \gets findCand(G(V,E),P_i)$.\Comment{find candidate set}
   % \State extract subgraph between the previous and current GPS point.
   \State $G_s \gets extract(G(V,E),\overline{\mathcal{T}}[i],\epsilon)$.
   \Comment{extract subgraph}
   \For{$r_i^j \in E_i$}
      \State compute $p(r_{i}^j|P_{i})$ by equation (6)
      \State $f[r_{i}^j] = -\infty$
      \For{$r_{i-1}^k \in E_{i-1}$}
      % \State compute shortest path from $r_{i-1}$ to $r_{i})$ in the sub action graph.
      % \State $sp_i^j\gets shortestPath(r_{i-1},r_{i},sbGraph)$
      \State $R_{j,k} \gets PathRec(G_s,r_i^j,r_{i-1}^k)$.\Comment{path recovery}
      \State compute $p(r_{i-1}^k,r_{i}^j)$ by equation (7)
      \State $Conj\gets f[r_{i-1}^j] * p(r_{i-1}^j,r_{i}^k) * p(r_{i}^j|P_{i})$
      \If{$Conj \ge f[r_{i}^k]$}
        \State $f[r_{i}^k] = Conj$
        \State $Pre[r_{i}^k] = r_{i-1}^j$
      \EndIf
    \EndFor
  \EndFor
  \EndFor
  \State $R = \argmax_{r_1^{k_1}\rightarrow r_2^{k_2}\rightarrow \cdots \rightarrow r_n^{k_n}}f[r_{n}^{k_n}]$
\State \textbf{return} $R$\Comment{The matched result is R}

% \Procedure{getSubGraph}{$p_{i-1},p_i,length,width$}
% \State get bounding box.
% \EndProcedure
\end{algorithmic}
\end{algorithm}
\end{large}



%%% Local Variables:
%%% mode: latex
%%% TeX-master: "gis18"
%%% End:
\section{Experimental Study}
\label{sec-exp}

\eat{
\begin{table*}[!ht]
	\renewcommand{\arraystretch}{1.20}
	\caption{\small Real-life Trajectory Datasets}
	\vspace{-1.5ex}
	\centering
	\footnotesize
	%\scriptsize
	\begin{tabular}{|l|c|c|c|r|}
		\hline
		\bf{ Data Sets}& \bf{Number\ of Trajectories}     &\bf {Sampling Rates\ (s)}   &\bf{Points Per Trajectory\ (K)}    &\bf {Total points} \\
		\hline
		\sercar	&1,000	    &3-5	    &$\sim114.0$   &114M\\
		\hline
		\geolife &182	    &1-5	    &$\sim131.4$   &24.2M\\
		\hline
		\mopsi &51	    	&2	    &$\sim153.9$     &7.9M\\
		\hline
	\end{tabular}
	\label{tab:datasets}
	\vspace{-2ex}
\end{table*}
}

\begin{table}[tb!]
	\renewcommand{\arraystretch}{1.20}
	\caption{\small Real-life Trajectory Datasets}
	\vspace{-1.5ex}
	\centering
	\footnotesize
	%\scriptsize
	\begin{tabular}{|l|c|c|r|}
		\hline
		\bf{ Properties of Data Sets} & \sercar      &\geolife   &\mopsi \\
		\hline
		{Number\ of Trajectories}	&1,000	    &182	    & 51  \\
		\hline
		 {Sampling Rates\ (s)} &3-5  & 1-5 & 2 \\
		\hline
		{Points Per Trajectory\ (K)}  &	$\sim114.0$    &$\sim131.4$	    & $\sim153.9$ \\
		\hline
		 {Total points (M)} &114   	    	&24.2    &7.9\\
		\hline
	\end{tabular}
	\label{tab:datasets}
	\vspace{-2ex}
\end{table}


In this section, we present an extensive experimental study of our one-pass trajectory tracking algorithms (\citt, \sitt and \bitt) compared with the
existing algorithms of \ldrh and \grts on trajectory datasets. Using three real-life trajectory datasets, we conducted sets of experiments to evaluate:
(1) the compression ratios,
(2) the number of messages (including data points and velocities),
(3) the max and average errors, and
(4) the running time of algorithms \citt, \sitt and \bitt vs. \ldrh and \grts. 
Among them, the impacts of error bounds and distance metrics on messages, errors and running time of these algorithms are evaluated. 




\begin{figure*}[tb!]
	\centering
	\includegraphics[scale = 0.56]{figures/Fig-BITT-mopsi-compression-ratio.png}\hspace{1ex}
	\includegraphics[scale = 0.56]{figures/Fig-BITT-sercar-compression-ratio.png}\hspace{1ex}
	\includegraphics[scale = 0.56]{figures/Fig-BITT-geolife-compression-ratio.png}\hspace{1ex}
	\vspace{-2ex}
	\caption{\small Evaluation of the compression ratios of \bitt: varying error bounds $\epsilon_{sed}$ and $\epsilon_{ped}$.}
	\label{fig:bitt-compression-ratio}
	\vspace{-1ex}
\end{figure*}


\begin{figure*}[tb!]
	\centering
	\includegraphics[scale = 0.565]{figures/Fig-BITT-mopsi-total-messages.png}\hspace{1ex}
	\includegraphics[scale = 0.565]{figures/Fig-BITT-sercar-total-messages.png}\hspace{1ex}
	\includegraphics[scale = 0.565]{figures/Fig-BITT-geolife-total-messages.png}\hspace{1ex}
	\vspace{-2ex}
	\caption{\small Evaluation of the total messages of \bitt: varying error bounds $\epsilon_{sed}$ and $\epsilon_{ped}$.}
	\label{fig:bitt-total-message}
	\vspace{-1ex}
\end{figure*}




\begin{figure*}[tb!]
	\centering
	\includegraphics[scale = 0.56]{figures/Fig-BITT-mopsi-sed-error.png}\hspace{1ex}
	\includegraphics[scale = 0.56]{figures/Fig-BITT-sercar-sed-error.png}\hspace{1ex}
	\includegraphics[scale = 0.56]{figures/Fig-BITT-geolife-sed-error.png}\hspace{1ex}
	\vspace{-2ex}
	\caption{\small Evaluation of the \sed errors of \bitt: varying error bounds $\epsilon_{sed}$ and $\epsilon_{ped}$.}
	\label{fig:bitt-sed-error}
	\vspace{-1ex}
\end{figure*}



\begin{figure*}[tb!]
	\centering
	\includegraphics[scale = 0.56]{figures/Fig-BITT-mopsi-ped-error.png}\hspace{1ex}
	\includegraphics[scale = 0.56]{figures/Fig-BITT-sercar-ped-error.png}\hspace{1ex}
	\includegraphics[scale = 0.56]{figures/Fig-BITT-geolife-ped-error.png}\hspace{1ex}
	\vspace{-2ex}
	\caption{\small Evaluation of the \ped errors of \bitt: varying error bounds $\epsilon_{sed}$ and $\epsilon_{ped}$.}
	\label{fig:bitt-ped-error}
	\vspace{-1ex}
\end{figure*}

\subsection{Experimental setting}

\stitle{Real-life Trajectory Datasets}. We use three reallife datasets ServiceCar, GeoLife and Mopsi shown in Table \ref{tab:datasets} to test our solutions.

\vspace{0.5ex}
\ni \emph{(1) Service car trajectory data} (\sercar) is the GPS trajectories collected by a Chinese car rental company during Apr. 2015 to Nov. 2015. The sampling rate was one point per $3$--$5$ seconds, and
each trajectory has around $114.1K$ points.

\vspace{0.5ex}
\ni \emph{(2) GeoLife trajectory data} (\geolife) is the GPS trajectories collected in GeoLife project by 182 users in a period from Apr. 2007 to Oct. 2011. These trajectories have a variety of sampling rates, among which 91\% are logged in each 1-5 seconds per point. %or each 5-10 meters

\vspace{0.5ex}
\ni \emph{(3) Mopsi trajectory data} (\mopsi) is the GPS trajectories collected in Mopsi project by 51 users in a period from 2008 to 2014. Most routes are in Joensuu region, Finland.
The sampling rate was one point per $2$ seconds, and each trajectory has around $153.9K$ points.

\stitle{Algorithms and implementation}.
We implement five tracking algorithms, \ie our \citt, \sitt and \bitt, \ldrh \cite{Trajcevski:LDRH} (the first and the most efficient trajectory tracking algorithm) and \grts~\cite{Lange:GRTS,Lange:Tracking} (the most effective tracking algorithm).
All algorithms were implemented with Java.
All tests were run on an {x64-based  PC with 8 Intel(R) Core(TM) i5-6500 CPU @ 3.20GHz and 8GB of memory.
	%, and each test was repeated over 3 times and the average is reported here.
	
\stitle{Metrics.}
Following the main stream \cite{Trajcevski:LDRH, Lange:GRTS, Lange:Tracking, Lin:Cised, Zhang:Evaluation}, we use \emph{compression ratio}, \emph{message ratio}, \emph{error} and \emph{running time} to evaluate algorithms.

 \ni \emph{(1) Compression ratio}. {It is defined as follows: Given a set of trajectories $\{\dddot{\mathcal{T}_1}, \ldots, \dddot{\mathcal{T}_M}\}$ and their piece-wise line representations $\{\overline{\mathcal{T}_1}, \ldots, \overline{\mathcal{T}_M}\}$, the compression ratio of an algorithm is $(\sum_{j=1}^{M} |\overline{\mathcal{T}}_j |)/(\sum_{j=1}^{M} |\dddot{\mathcal{T}}_j |)$.
	By the definition, \emph{algorithms with lower compression ratios are better}.}

 \ni \emph{(2) Message ratio}. It is ``the total number of messages'' divided by ``the total number of the original trajectory points''. Note there are totally three kinds of messages, \ie a) \emph{position-message} $P_s$ for \grts, b) \emph{velocity-message} $\vv{v}$ for \bitt (including \citt and \sitt) and c) \emph{position-velocity-message} ($P_s$, $\vv{v}$) for \ldrh, \grts and \bitt. 
 %For fair comparison, a \emph{position-velocity-message}
 

 \ni \emph{(3) Max and average errors}. Max (average) error is the maximal (average) value of the distances from every point of the original trajectories to its representing line segment of the simplified trajectories.
 
 \ni \emph{(4) Running time}. It is the essential execution time of an algorithm in processing a dataset.
 %It is the efficiency  of the algorithms.
 



\begin{figure*}[tb!]
	\centering
	\includegraphics[scale = 0.580]{figures/Fig-mopsi-compression-ratio.png}\hspace{-1ex}
	\includegraphics[scale = 0.580]{figures/Fig-sercar-compression-ratio.png}\hspace{-1ex}
	\includegraphics[scale = 0.580]{figures/Fig-geolife-compression-ratio.png}\hspace{0ex}
	\vspace{-1ex}
	\caption{\small Evaluation of compression ratios: varying error bounds $\epsilon_{sed}$ and $\epsilon_{ped}$.}
	\label{fig:compression-ratio}
	\vspace{-1ex}
\end{figure*}



\begin{figure*}[tb!]
	\centering
	\includegraphics[scale = 0.580]{figures/Fig-mopsi-total-messages.png}\hspace{-1ex}
	\includegraphics[scale = 0.580]{figures/Fig-sercar-total-messages.png}\hspace{-1ex}
	\includegraphics[scale = 0.580]{figures/Fig-geolife-total-messages.png}\hspace{0ex}
	\vspace{-1ex}
	\caption{\small Evaluation of total messages: varying error bounds $\epsilon_{sed}$ and $\epsilon_{ped}$.}
	\label{fig:total-message}
	\vspace{-1ex}
\end{figure*}

\eat{%%%%%%%%%%%%%%%%%%%%velocity messages
\begin{figure*}[tb!]
	\centering
	\includegraphics[scale = 0.580]{figures/Fig-mopsi-speed-messages.png}\hspace{-1ex}
	\includegraphics[scale = 0.580]{figures/Fig-sercar-speed-messages.png}\hspace{-1ex}
	\includegraphics[scale = 0.580]{figures/Fig-geolife-speed-messages.png}\hspace{0ex}
	\vspace{-1ex}
	\caption{\small Evaluation of velocity messages: varying error bounds $\epsilon_{sed}$ and $\epsilon_{ped}$.}
	\label{fig:speed-message}
	\vspace{-1ex}
\end{figure*}
}%%%%%%%%%%%%%%%%%%%%%%velocity messages



\subsection{Experimental Results}

\subsubsection{Evaluation of~algorithm \bitt}
%%%%%%%%%%%% messages
This section test the impacts of \ped and \sed (\ie the shapes of finite beams) on algorithm \bitt. We varied error bounds $\epsilon_{sed}$ and $\epsilon_{ped}$ from $10$ meters to $200$ meters on the entire three datasets, respectively. The results are reported in Figures~\ref{fig:bitt-compression-ratio}, \ref{fig:bitt-total-message}, \ref{fig:bitt-sed-error} and \ref{fig:bitt-ped-error}.

%\stitle{Message and compression ratios}. Figures~\ref{fig:bitt-total-message} and \ref{fig:bitt-compression-ratio} tell 

\ni (1) Both message and compression ratios decrease with the increase of $\epsilon_{sed}$ and $\epsilon_{ped}$, respectively. It is clear that when the tracking area becomes larger, \bitt is more tolerant of the distance deviation of a moving object, hence, fewer messages are transmitted and fewer data points are saved.

\ni (2) The velocity  $\vv{v}$ (see Figures~\ref{alg:citt-s-full} and ~\ref{alg:bitt}) is not frequently updated during the process of a sub-trajectory for all $\epsilon$ in all datasets. More specifically, a) when $\epsilon_{ped} \ge \epsilon_{sed}$, \ie~\bitt falls back to \citt, the \emph{position-velocity-messages} are on average $(39.59\%, 42.69\%, 47.20\%)$ of the total messages \wrt datasets (\mopsi, \sercar, \geolife), respectively, and b) when $\epsilon_{ped} << \epsilon_{sed}$, \ie~\bitt falls back to \sitt, the \emph{position-velocity-messages} are on average $(66.09\%, 71.29\%, 70.07\%)$ of the total messages \wrt datasets (\mopsi, \sercar, \geolife), respectively. Otherwise, \bitt has the number of \emph{position-velocity-messages} between \citt and \sitt.
%More than \myred{half} messages are \emph{position-velocity-messages}  for all $\epsilon$ in all datasets, meaning that, given a start point $P_s$ and an initial velocity $\vv{v}$ as the way shown in Figures~\ref{alg:citt-s-full} and ~\ref{alg:bitt}, 

%\ni (3) Dataset \sercar has the highest message and compression ratios, compared with datasets \mopsi and \geolife, due to its lowest sampling rate. 

\ni (3) Both average \ped and \sed errors increase with the increase of $\epsilon_{sed}$ and $\epsilon_{ped}$, respectively.

\ni (4) Given a $\epsilon_{sed}$, both the compression and message ratios, and the average \sed and \ped errors are constant for all $\epsilon_{ped}$ that are greater than $\epsilon_{sed}$, \eg $\epsilon_{sed}=10$ and $\epsilon_{ped} \ge 10$, showing that \bitt falls back to \citt in these cases.

%\ni (6) When $\epsilon_{sed} > \epsilon_{ped}$, the performance of algorithm \bitt will be similar to that of \sitt, as a result, the average \ped error will be much smaller than $\epsilon_{sed}$,
%the $\epsilon_{ped}$ is mainly in effect,

\begin{figure*}[tb!]
	\centering
	\includegraphics[scale = 0.580]{figures/Fig-mopsi-sed-error.png}\hspace{-1ex}
	\includegraphics[scale = 0.580]{figures/Fig-sercar-sed-error.png}\hspace{-1ex}
	\includegraphics[scale = 0.580]{figures/Fig-geolife-sed-error.png}\hspace{0ex}
	\vspace{-1ex}
	\caption{\small Evaluation of \sed errors: varying error bounds $\epsilon_{sed}$ and $\epsilon_{ped}$.}
	\label{fig:sed-error}
	\vspace{-1ex}
\end{figure*}

\begin{figure*}[tb!]
	\centering
	\includegraphics[scale = 0.580]{figures/Fig-mopsi-ped-error.png}\hspace{-1ex}
	\includegraphics[scale = 0.580]{figures/Fig-sercar-ped-error.png}\hspace{-1ex}
	\includegraphics[scale = 0.580]{figures/Fig-geolife-ped-error.png}\hspace{0ex}
	\vspace{-1ex}
	\caption{\small Evaluation of \ped errors: varying error bounds $\epsilon_{sed}$ and $\epsilon_{ped}$.}
	\label{fig:ped-error}
	\vspace{-1ex}
\end{figure*}

\begin{figure*}[tb!]
	\centering
	\includegraphics[scale = 0.580]{figures/Fig-mopsi-running-time.png}\hspace{-1ex}
	\includegraphics[scale = 0.580]{figures/Fig-sercar-running-time.png}\hspace{-1ex}
	\includegraphics[scale = 0.580]{figures/Fig-geolife-running-time.png}\hspace{0ex}
	\vspace{-1ex}
	\caption{\small Evaluation of running time: varying error bounds $\epsilon_{sed}$ and $\epsilon_{ped}$.}
	\label{fig:running-time}
	\vspace{-1ex}
\end{figure*}


\subsubsection{Comparing algorithms \bitt, \sitt and \citt with \ldrh and \grts.}
This section compares our algorithms \citt, \sitt and \bitt with algorithms \ldrh and \grts.
We varied the error bound (either $\epsilon_{sed}$ or $\epsilon_{ped}$) of \citt, \sitt, \ldrh and \grts from $10$ meters to $200$ meters on the entire three datasets, respectively. 
{For \bitt, its performance depends on the shape of the finite beam, \ie given the same area, it varies \wrt the ratio between $\epsilon_{ped}$ and $\epsilon_{sed}$. {Without losing generality}, we set its $\epsilon_{ped}$ to {$0.5$} times the $\epsilon_{ped}$ of \sitt and its $\epsilon_{sed}$ to {$1.6$} times the $\epsilon_{sed}$ of \citt, \grts and \ldrh, such that the area of the finite beam of \bitt is $3.147\times\epsilon_{sed}^2$, which is approximate to $3.142\times\epsilon_{sed}^2$, the area of the circular of algorithms \citt, \ldrh and \grts~\wrt the given $\epsilon_{sed}$.}
%
The results are reported in Figures~\ref{fig:compression-ratio}, \ref{fig:total-message}, \ref{fig:sed-error}, \ref{fig:ped-error} and \ref{fig:running-time}.


\stitle{Compression ratios.} We first report and analyze the compression ratios from Figure~\ref{fig:compression-ratio}.

\ni (1) When increasing $\epsilon_{sed}$ and $\epsilon_{ped}$, the compression ratios of all these algorithms decrease on all datasets.

\ni (2) \sitt has the best compression ratios, \ldrh is the worst, and \grts, \citt and \bitt are comparable on all datasets and for all $\epsilon$.
The compression ratios of \grts, \bitt,  \citt and \sitt are on average {($27.5\%$, $38.6\%$, $32.2\%$), ($34.9\%$, $37.3\%$, $36.7\%$), ($27.7\%$, $38.9\%$, $32.8\%$) and ($20.2\%$, $19.6\%$, $19.4\%$)} of \ldrh on datasets (\mopsi, \sercar, \geolife), respectively.
For example, when $\epsilon = 40$ meters, \ie~$\epsilon_{sed} = 40$ meters for \ldrh, \grts and \citt, $\epsilon_{ped} = 40$ meters for \sitt and {$(\epsilon_{ped}, \epsilon_{sed}) = (0.5\times 40, 1.6\times 40)=(20, 64)$} meters for \bitt, the compression ratios of \ldrh, \grts, \bitt, \citt and \sitt are
{($10.9\%$, $33.7\%$, $13.6\%$), ($3.0\%$, $13.3\%$, $4.5\%$), {($3.7\%$, $12.7\%$, $5.0\%$)}, ($3.0\%$, $13.2\%$, $4.4\%$) and ($2.1\%$, $6.1\%$, $2.6\%$)} on  {datasets (\mopsi, \sercar, \geolife)}, respectively. 

\ni (3) Datasets have impacts on compression ratios, \ie~datasets with higher sampling rates usually have better performance in terms of compression ratio.
	


\stitle{Message ratios.} We then report and and analyze the message ratios from Figure~\ref{fig:total-message}.

%To evaluate the impacts of distance metrics and error bounds on messages of \citt, \sitt and \bitt vs. \ldrh and \grts, we varied the error bound (either $\epsilon_{sed}$ or $\epsilon_{ped}$) from $10$ meters to $200$ meters on the entire three datasets, respectively. 

\ni (1) When increasing $\epsilon_{sed}$ and $\epsilon_{ped}$, the number of messages of all these algorithms decrease on all datasets.

\ni (2) The message numbers from the largest to the smallest are \ldrh, \grts, \citt, \bitt and \sitt. Among them, \ldrh has the largest messages because it has the worst compression ratios (thus it produces many \emph{position-velocity-messages}), and \sitt has the least messages because it has the best compression ratios (corresponding to the least \emph{position-velocity-messages}) and at the same time it seldom updates velocities during the process of a sub-trajectory (thus it only sends a small amount of \emph{velocity-messages}).

%(all of them are \emph{position-velocity-messages})
%\ni (2) Since \ldrh updates the position information every time the velocity is updated, the percentage of \emph{velocity-messages} is always $50\%$ of all messages.

\ni (3) \citt has a medium amount of messages (including \emph{position-velocity-messages} and \emph{velocity-messages}) and \bitt is between \citt and \sitt in this test.


\ni (4) \grts has more messages than \citt. Recall that \grts has \emph{position-messages} and \emph{position-velocity-messages}, while \citt has \emph{velocity-messages} and \emph{position-velocity-messages}. From Figure \ref{fig:compression-ratio} we know that \grts has a similar compression ratio as \citt, meaning \grts has the similar number of \emph{position-messages} as the \emph{position-velocity-messages} of \citt. Besides, the  number of \emph{position-velocity-messages} of \grts ({on average $(61.81\%, 59.28\%, 61.85\%)$} of the total messages \wrt datasets (\mopsi, \sercar, \geolife), respectively) is usually a bit larger than the \emph{velocity-messages} of \citt (on average $(60.41\%, 57.31\%, 52.80\%)$ of the total messages \wrt datasets (\mopsi, \sercar, \geolife), respectively). As a result, \grts has more total messages than \citt.

%This is because \grts only updates the velocities messages when the buffer is cleared, and mainly transmits position information.




\stitle{Errors.} We next report and analyze the max and average errors from Figures~\ref{fig:sed-error} and \ref{fig:ped-error} and Table \todo{X}.

%The results are reported in Figures~\ref{fig:sed-error} and Figure~\ref{fig:ped-error}.
\ni (1) Average errors increase with the increase of $\epsilon_{sed}$ and $\epsilon_{ped}$.

\ni (2) The average \sed errors of these algorithms from the largest to the smallest are \sitt, \grts (\citt) and \ldrh. Among them, the average \sed error of \citt is very close to \grts. \todo{\bitt}
The average \sed errors of algorithms \citt and \sitt are on average
($412.7\%$, $222.8\%$, $350.1\%$)
and ($842.0\%$, $857.5\%$, $1452.5\%$)
of \ldrh and ($103.1\%$, $98.4\%$, $100.5\%$) and
($209.9\%$, $442.9\%$, $414.0\%$)
of \grts on datasets (\mopsi, \sercar, \geolife), respectively.

\ni (3) The average \ped errors of these algorithms from the largest to the smallest are \sitt, \grts (\citt) and \ldrh. Among them, the average \ped errors of \citt is very close to \grts. \todo{\bitt}
The average \ped errors of algorithms \citt and \sitt are on average
($452.2\%$, $278.1\%$, $388.9\%$)
and ($550.5\%$, $517.0\%$, $589.0\%$)
of \ldrh and ($102.0\%$, $98.3\%$, $99.7\%$) and
($123.9\%$, $187.0\%$, $150.9\%$)
of \grts on datasets (\mopsi, \sercar, \geolife), respectively.




%%%%%%%%%%%%%%%%% running time
%In this part of experiments, we compare the running time of our algorithms \citt, \sitt and \bitt with \ldrh and \grts.
%The results are reported in Figure~\ref{fig:running-time}. 
\stitle{Running time.} We finally report and analyze the running time.
Since the running time of \grts is hundreds of times slower than other algorithms, it is not shown in Figure~\ref{fig:running-time}.

\ni (1) The error bound $\epsilon$ has few impacts on running time of \citt, \sitt, \bitt and \ldrh.

\ni (2) The running time of \bitt is approximately the sum of \citt and \sitt, because it combines the logic of \citt and \sitt.

\ni (3) The running time of these algorithms from the largest to the smallest are \grts, \bitt, \citt, \sitt and \ldrh on all datasets.
The average running time of algorithms \citt and \sitt is on average
($378.8\%$, $363.2\%$, $296.4\%$)
and ($331.5\%$, $294.1\%$, $256.4\%$)
of \ldrh and ($0.607\%$, $21.7\%$, $0.704\%$) and
($0.529\%$, $18.1\%$, $0.623\%$)
of \grts on datasets (\mopsi, \sercar, \geolife), respectively.



\subsubsection{Summary.} % and discuss
From these tests we find the followings.

\sstab\emph{(1) Compression ratios}. The optimal \sitt algorithm has the best compression ratios among all the algorithms. Algorithm \bitt and \citt are comparable with \grts.
They are all better than \ldrh.

\sstab\emph{(2) Message ratios}. The message numbers from the largest to the smallest are \ldrh, \grts, \citt, \bitt and \sitt.

\sstab\emph{(3) Average errors}. The average errors of these algorithms from the largest to the smallest are \sitt, \grts, \citt and \ldrh. Among them, the average error of \citt is very close to \grts.

\sstab\emph{(4) Running time}. Algorithm \ldrh is the fastest and \grts is the slowest. Moreover, the running time of \bitt is approximately the sum of \citt and \sitt.

In a conclusion, \ldrh runs the fastest and has the lowest average errors at a price of the poorest compression and message ratio. \sitt outperforms \grts in every metrics except average errors. \citt outperforms \grts in message ratio and running time, and is comparable with \grts in compression ratio and average errors. \bitt is comparable with \citt except that it has smaller average errors and longer running time.
\stitle{Related work}. We summarize related work as follows.
%
%Scholarly article ranking

Scholarly article ranking has shifted from citation count analysis~\cite{Garfield471,Hirsch15112005} to graph analysis~\cite{ChenXMR07,Zhou07-CoRank,Jiang12-MRank,Liang16AAAI,Li08TSRanking,Wang13AAAI,WalkerXKM07,sayyadi09,
Wang16TIST,Ng11KDD}.
Based on the information used, these methods are divided into four categories: (a) using the citation information only~\cite{Garfield471,Hirsch15112005,ChenXMR07,Ng11KDD}, (b) using the citation and temporal information~\cite{Li08TSRanking,WalkerXKM07}, (c) using the citation information and other heterogeneous information, \eg authors and venues of articles~\cite{Zhou07-CoRank,Jiang12-MRank,Liang16AAAI}, and (d) combining the citation, temporal and other heterogeneous information~\cite{sayyadi09,Wang16TIST,Wang13AAAI}.
Our work belongs to the last category aiming at fully employing information available for scholarly article ranking.


%\stitle{PageRank\&weighted PageRank algorithms}.

%PageRank \cite{Brin98:PageRank} and its extensions have been extensively used for citation analyses \cite{Waltman2014}. While PageRank equally propagates scores along outlinks, Weighted PageRank \cite{Xing04:WPR} extends PageRank by distributing scores based on the popularity of pages. Different from previous work, the Time-Weighted PageRank proposed in this work discriminately propagates scores in terms of citation statistics.

PageRank \cite{Brin98:PageRank} and its extensions have been extensively used for citation analyses \cite{Waltman2014}. While PageRank equally propagates scores along outlinks, Weighted PageRank extends PageRank by distributing scores based on certain criteria such as popularity of pages~\cite{Xing04:WPR} or authority of authors~\cite{Ding11}. Different from previous work, the Time-Weighted PageRank proposed in this work discriminately propagates scores in terms of citation statistics.






%\stitle{Dynamic algorithms}.

Dynamic algorithms have proven useful for various tasks by avoiding computing from scratch~\cite{RamalingamR93}.
% and only recomputing those affected by updates
%Dynamic algorithms have proven useful for graph analysis tasks, \eg incremental graph pattern matching~\cite{FanWW13} and  incremental simrank computation~\cite{YuLZ14}.
To our knowledge, little concern has been paid to dynamic scholarly article ranking except that~\cite{GhoshKHLL11} uses PageRank in dynamic citation networks. However, its solution is based on a strong and impractical assumption that there are no citations between articles in the same years.
Further, although there exist several studies on incremental PageRank computation~\cite{DesikanPSK05,AbiteboulPC03,WuR09} and on incremental PageRank approximation \cite{BahmaniCG10,BahmaniKMU12}, they are not designed for scholarly article ranking.
%
Different from previous work, we study scholarly article ranking in a dynamic environment in terms of
the citation characteristics of scholarly articles, which has never been exploited before.

%Our approach only makes the assumption that there are no mutual references within the citation network, which, we admit, violates xx\% of total citations on \magdata, and is significantly different (yy\% on \magdata) from~\cite{GhoshKHLL11}.  - move to Section 3

Ensemble methods use multiple learners to obtain better performance than could be obtained from a constituent learner alone~\cite{zhihua-book}.
%In this work, we leverage ensembles to produce better and robust results for scholarly article ranking~\cite{zhihua-book,wsdmcup,DuanAMHH16}.
In this work, we leverage  importance assembling  to produce better and robust results for scholarly article ranking~\cite{zhihua-book,wsdmcup,DuanAMHH16}.

\vspace{-1ex}
%%%%%%%%%%%%%%%%%%%%%%%%%%%%%%%%%%%%%%%%%%%%%%%%%%%%%%%%%%%%%%%%%%%%%%%%%%%%%%
\section{Conclusions}
%%%%%%%%%%%%%%%%%%%%%%%%%%%%%%%%%%%%%%%%%%%%%%%%%%%%%%%%%%%%%%%%%%%%%%%%%%%%%%

We have evaluated the state-of-the-art \lsa algorithms for trajectory compression, including \emph{both the optimal and the sub-optimal methods that use either \ped or \sed}. 
Using a variety of real trajectory datasets, we evaluated the performance of each technique.% in terms of its processing time, compression ratio and average error.
Our experimental results show that 
(1) the output sizes of algorithms using \sed are approximate $2$ times of using \ped, 
(2) the output sizes of sub-optimal algorithms are $130\%$--$160\%$ of the optimal algorithms, and 
(3) the one-pass algorithms \siped and \operb and \cised are tens of times faster than the batch algorithms and \textcolor{red}{$xxx$} times faster than online algorithms, while they still have comparable compression ratios with batch algorithms. Hence, they are more suitable for resource constraint mobile devices.

\eat{
\section*{Acknowledgments}
This work is supported in part by NSFC (U1636210), NSFC ({61421003}) and Beijing Advanced Innovation Center for Big Data and Brain
Computing.
}

\balance
\bibliographystyle{ACM-Reference-Format}
%\bibliography{ref-map-matching, ref-traj-simp}
\bibliography{ref}


%\section*{Appendix: Additional Experiments}


%%%%%%%%%%%%%%%%%%%%%%%%%%%%%%%%%%%%%%%%%%%%%%%%%%%%%%%%%%%%%%%%%%%%%%%%%%%%%%
{Evaluation of different compression algorithms}
%%%%%%%%%%%%%%%%%%%%%%%%%%%%%%%%%%%%%%%%%%%%%%%%%%%%%%%%%%%%%%%%%%%%%%%%%%%%%%

\begin{figure*}[tb!]
	\centering
  \includegraphics[height=0.3\textwidth]{Figures/Exp-epsilon-rmf-cmp_matcher_CISED-RPI-Public.png}\hspace{5ex}
  \includegraphics[height=0.3\textwidth]{Figures/Exp-epsilon-rmf-cmp_matcher_CISED-RPI-SerCar.png}\hspace{5ex}
	\vspace{-2.5ex}
  \caption{\small Evaluation of route mismatch fraction : varying the error bound $\epsilon$.}
	\label{fig:rmf-epsilon}
	\vspace{-2ex}
\end{figure*}
\begin{figure*}[tb!]
	\centering
  \includegraphics[height=0.3\textwidth]{Figures/Exp-epsilon-rmf-cmp_matcher_SQUISH-E-Public.png}\hspace{5ex}
  \includegraphics[height=0.3\textwidth]{Figures/Exp-epsilon-rmf-cmp_matcher_SQUISH-E-SerCar.png}\hspace{5ex}
	\vspace{-2.5ex}
  \caption{\small Evaluation of route mismatch fraction : varying the error bound $\epsilon$.}
	\label{fig:rmf-epsilon}
	\vspace{-2ex}
\end{figure*}

\begin{figure*}[tb!]
	\centering
  \includegraphics[height=0.3\textwidth]{Figures/Exp-epsilon-rmf-cmp_matcher_sampling-Public.png}\hspace{5ex}
  \includegraphics[height=0.3\textwidth]{Figures/Exp-epsilon-rmf-cmp_matcher_sampling-SerCar.png}\hspace{5ex}
	\vspace{-2.5ex}
  \caption{\small Evaluation of route mismatch fraction : varying the error bound $\epsilon$.}
	\label{fig:rmf-epsilon}
	\vspace{-2ex}
\end{figure*}

\begin{figure*}[tb!]
	\centering
  \includegraphics[height=0.3\textwidth]{Figures/Exp-epsilon-rmf-cmp_tc_HMM-based-Public.png}\hspace{5ex}
  \includegraphics[height=0.3\textwidth]{Figures/Exp-epsilon-rmf-cmp_tc_HMM-based-SerCar.png}\hspace{5ex}
	\vspace{-2.5ex}
  \caption{\small Evaluation of route mismatch fraction : varying the error bound $\epsilon$.}
	\label{fig:rmf-epsilon}
	\vspace{-2ex}
\end{figure*}
\begin{figure*}[tb!]
	\centering
  \includegraphics[height=0.3\textwidth]{Figures/Exp-epsilon-rmf-cmp_tc_GF-MM-Public.png}\hspace{5ex}
  \includegraphics[height=0.3\textwidth]{Figures/Exp-epsilon-rmf-cmp_tc_GF-MM-SerCar.png}\hspace{5ex}
	\vspace{-2.5ex}
  \caption{\small Evaluation of route mismatch fraction : varying the error bound $\epsilon$.}
	\label{fig:rmf-epsilon}
	\vspace{-2ex}
\end{figure*}
\begin{figure*}[tb!]
	\centering
  \includegraphics[height=0.3\textwidth]{Figures/Exp-epsilon-rmf-cmp_tc_CT-MM-Public.png}\hspace{5ex}
  \includegraphics[height=0.3\textwidth]{Figures/Exp-epsilon-rmf-cmp_tc_CT-MM-SerCar.png}\hspace{5ex}
	\vspace{-2.5ex}
  \caption{\small Evaluation of route mismatch fraction : varying the error bound $\epsilon$.}
	\label{fig:rmf-epsilon}
	\vspace{-2ex}
\end{figure*}



%%%%%%%%%%%%%%%%%%%%%%%%%%%%%%% f-score %%%%%%%%%%%%%%%%%%%%%%%%%%%%%%%


\begin{figure*}[tb!]
	\centering
  \includegraphics[height=0.3\textwidth]{Figures/Exp-epsilon-f-score-cmp_matcher_CISED-RPI-Public.png}\hspace{5ex}
  \includegraphics[height=0.3\textwidth]{Figures/Exp-epsilon-f-score-cmp_matcher_CISED-RPI-SerCar.png}\hspace{5ex}
	\vspace{-2.5ex}
  \caption{\small Evaluation of route mismatch fraction : varying the error bound $\epsilon$.}
	\label{fig:rmf-epsilon}
	\vspace{-2ex}
\end{figure*}
\begin{figure*}[tb!]
	\centering
  \includegraphics[height=0.3\textwidth]{Figures/Exp-epsilon-f-score-cmp_matcher_SQUISH-E-Public.png}\hspace{5ex}
  \includegraphics[height=0.3\textwidth]{Figures/Exp-epsilon-f-score-cmp_matcher_SQUISH-E-SerCar.png}\hspace{5ex}
	\vspace{-2.5ex}
  \caption{\small Evaluation of route mismatch fraction : varying the error bound $\epsilon$.}
	\label{fig:rmf-epsilon}
	\vspace{-2ex}
\end{figure*}


\begin{figure*}[tb!]
	\centering
  \includegraphics[height=0.3\textwidth]{Figures/Exp-epsilon-f-score-cmp_matcher_sampling-Public.png}\hspace{5ex}
  \includegraphics[height=0.3\textwidth]{Figures/Exp-epsilon-f-score-cmp_matcher_sampling-SerCar.png}\hspace{5ex}
	\vspace{-2.5ex}
  \caption{\small Evaluation of route mismatch fraction : varying the error bound $\epsilon$.}
	\label{fig:rmf-epsilon}
	\vspace{-2ex}
\end{figure*}

\begin{figure*}[tb!]
	\centering
  \includegraphics[height=0.3\textwidth]{Figures/Exp-epsilon-f-score-cmp_tc_HMM-based-Public.png}\hspace{5ex}
  \includegraphics[height=0.3\textwidth]{Figures/Exp-epsilon-f-score-cmp_tc_HMM-based-SerCar.png}\hspace{5ex}
	\vspace{-2.5ex}
  \caption{\small Evaluation of route mismatch fraction : varying the error bound $\epsilon$.}
	\label{fig:rmf-epsilon}
	\vspace{-2ex}
\end{figure*}
\begin{figure*}[tb!]
	\centering
  \includegraphics[height=0.3\textwidth]{Figures/Exp-epsilon-f-score-cmp_tc_GF-MM-Public.png}\hspace{5ex}
  \includegraphics[height=0.3\textwidth]{Figures/Exp-epsilon-f-score-cmp_tc_GF-MM-SerCar.png}\hspace{5ex}
	\vspace{-2.5ex}
  \caption{\small Evaluation of route mismatch fraction : varying the error bound $\epsilon$.}
	\label{fig:rmf-epsilon}
	\vspace{-2ex}
\end{figure*}
\begin{figure*}[tb!]
	\centering
  \includegraphics[height=0.3\textwidth]{Figures/Exp-epsilon-f-score-cmp_tc_CT-MM-Public.png}\hspace{5ex}
  \includegraphics[height=0.3\textwidth]{Figures/Exp-epsilon-f-score-cmp_tc_CT-MM-SerCar.png}\hspace{5ex}
	\vspace{-2.5ex}
  \caption{\small Evaluation of route mismatch fraction : varying the error bound $\epsilon$.}
	\label{fig:rmf-epsilon}
	\vspace{-2ex}
\end{figure*}



\end{document}
