
%%% Local Variables:
%%% mode: latex
%%% TeX-master: "gis18"
%%% End:
\section{related work}
\label{sec-related}

The map matching problem has attracted extensive research attentions over the
past decades, we next introduce the state-of-the-art solutions, especially those for sparse trajectories.


\stitle{General map-matching methods.}
As introduced by Quddus et al in \cite{Quddus2003}, the traditional map matching methods are four categories:
geometric methods, topological methods, probabilistic methods and advanced
methods.
Geometric methods only consider the shape of the road
segments without considering the constraints induced by map topology. These
methods suffer from one significant drawback of being sensitive to measurement
errors.
Topological methods \cite{Greenfeld2002b,White2000b,948625} make use of not only the
geometric information of the GPS measurements and the road segments but also the
connectivity and contiguity of the road segments. However, such algorithms are
still vulnerable to sensor noise and unsuitable to highly noisy and sparse data.
The probabilistic {algorithms \cite{zhao1997,Ochieng2009}} require the definition of an elliptical or rectangular confidence region around a position fix obtained from a navigation sensor.
To pursue improved matching accuracy, advanced map matching algorithms \cite{Newson2009Hidden,Hunter2013,liu:st-crf,Wang:eddy} have been proposed in order to take advantage of statistical models such as Kalman Filter and Hidden Markov Models.


\stitle{Map-matching on sparse trajectories.}
In recent years, several papers have addressed the problem of map-matching for
sparse GPS trajectories. These algorithms are basically probabilistic methods
utilizing some kind of sequence labeling models, among which Hidden Markov Model (HMM)
and Conditional Random Field (CRF) are the most popular.
Lamb and Thiebaux of \cite{Lamb1999Avoiding} were the first to use HMM for map-matching algorithms.
In \cite{Newson2009Hidden}, Newson and Krumm proposed an HMM based framework
which takes into consideration the travel distance to define the transition
probabilities and produces decent results against noise and sparseness.
%
Wang et al. proposed a novel HMM-based online map-matching algorithm named Eddy
\cite{Wang:eddy} with a solid error and delay bound analysis.
%
Osogami and Raymond proposed an algorithm {\cite{Osogami:2013:IRL}} using action graph which can
incorporate turnings in the process of local optimal paths search and utilizing inverse reinforcement
learning to automatically learn weights from ground truth data.
%
To further improve the accuracy for matching on sparse data, Yin et al. proposed
a general feature-based framework \cite{yin:feature-based} that utilizing a new segment-based
probabilistic map matching strategy that searches for shortest path between
candidate roads of only the key points detected by trajectory simplification techniques.

Besides HMM-based algorithms, a few papers addressed this problem utilizing the
CRF. Hunter et al. proposed a map matching framework named the Path Inference
Filter(PIF)\cite{Hunter2013}, which considers 10 influencing factors, including the length of the
path, the number of stop signs along the path and minimum average travel time
and so on. However, in general situations, most of these factors are difficult to
obtain, which restricts the generalization of it.
Liu et at. proposed a CRF-based map matching method called ST-CRF \cite{liu:st-crf}for
sparse trajectories. Their algorithm utilizes only the basic information from
GPS trajectory and considers five factors in designing of the feature function.
However, like all other CRF-based methods, their algorithm needs large amount of
historic data to learn the weights in feature function.
%
All these map-matching methods focus on raw trajectories and
overlook the attributes of simplified trajectories.








