
\section{Method}
\label{sec-method}

In this section, we present our {trajectory simplification} aware map-matching method.


\begin{figure*}[htb!]
	\centering
  \includegraphics[width=0.89\textwidth, height  = 0.22\textwidth]{Figures/Fig-Architecture-en.pdf}\hspace{1ex}
	\vspace{-1ex}
	\caption{The framework of trajectory simplification aware map matching.}
	\label{fig:sys-arc}
	\vspace{-1ex}
\end{figure*}


\subsection{Overview of the Method}
The framework of {trajectory simplification} aware map-matching (\stmm) is illustrated in \myfig{fig:sys-arc}.
\stmm takes as input a raw trajectory and simplifies the raw trajectory by the trajectory simplification component. The output of the component is then injected into the HMM-based map-matching method, including two components, \emph{local path recovery} and \emph{global route decoding}, each utilizes the unique characteristics of the simplified trajectory to improve the accuracies and running time of map-matching.

\stitle{(1) Trajectory simplification.}
It simplifies the input stream of raw trajectory data points by dropping redundant points and keeping significant ones with an error bound. The output of the algorithm is a sequence of directed line segments associated with auxiliary information that is used in the consequent map-matching.

\stitle{(2) Local path recovery.}
This component is used to generate the local optimal path given any two neighboring points of a simplified trajectory.
It operates on an action graph, which is a weighted graph extracted from road network incorporating information from the raw trajectory, and is used to describe the actions of a user.  The optimal paths can be detected through the shortest path searching in the graph. Note when estimating the edge weights of an action graph, we consider the geometry similarities between the raw/simplified trajectories and the paths to better recover the local paths.

%When estimating the weights of edges of an action graph, we consider the distribution of the raw trajectory points in the simplified trajectories to better recover the local path.

\stitle{(3) Global route decoding.}
It is another core component of the HMM-based method. It computes the probability of each candidate route after the local optimal paths recovery, and finds the global optimal route through dynamic programming.
Also, we take the geometry similarity between the raw/simplified trajectories and the routes into considerations in the computation of transition probabilities.
%Based on the local optimal paths generated by Path recovery component,


\subsection{Trajectory Simplification}
\label{sec:simp}


\todo{what and why}

We choose two one-pass algorithms, \ie~\siped \cite{Zhao:Sleeve} using \ped and \cised\cite{Lin:Cised} using \sed, to carry out trajectory
simplification. \ped and \sed are two distance metrics widely used in trajectory simplification algorithms, more details about them please refer to \cite{Lin:Cised, Zhang:Evaluation}.

\todo{full epsilon of siped and cised, plus auxiliary information}

For a sub-trajectory $\dddot{\mathcal{T}}[P_s, ..., P_{s+u}]$, $u\ge 1$, simplified to a line segment $\vv{P_{s}P_{s+u}}$, \stmm also computes the auxiliary information of $\vv{P_{s}P_{s+u}}$, including $L_L$ and $L_R$, the length of raw sub-trajectory $\dddot{\mathcal{T}}[P_s, ..., P_{s+u}]$ on the left and right sides of the simplified line segment $\vv{P_{s}P_{s+u}}$, respectively. 
%
%The distribution of the raw sub-trajectory points according to a simplified line segment can be calculated and saved in advance during trajectory simplification.
Indeed, the distribution of raw trajectory points \wrt the simplified line segments is quite uneven.
%\myfig{fig:traj-sides} is a typical example of the uneven distribution phenomenon, where green lines are simplified line segments and blue
%points are raw trajectory points, and
\myfig{fig:traj-side-stat} shows the percentage of points on one side of the simplified line segments, where more than $85\%$ of raw trajectory points are located on one side of the simplified line segments.
This information is a hint to select the right path/route, hence, it is used in the local path recovery and the global route decoding.

\begin{figure}
  \centering
  \begin{subfigure}{0.34\textwidth}
    \centering
    \includegraphics[width = \textwidth, height = 0.66\textwidth]{Figures/Exp-statistic-side-ratio.png}
  \end{subfigure}
  \vspace{-2ex}
  \caption{\small Distribution of trajectory points.} \vspace{-3ex}
  \label{fig:traj-side-stat}
  \vspace{1ex}
\end{figure}



%%%%%%%%%%%%%%%%%%%%%%%%%%%%%%%%%%%%%%%%%%%%%%%%%%%%%%%%%%%%%%%%%%%%%%%%%%%%%%%%%%%

\subsection{Local Path Recovery}
\label{sec:route}

%The local path recovery module is used to select a most possible path between two neighbouring points of a simplified trajectory.
The simplified trajectories are sparser than the original ones, which leads to higher uncertainties of map-matching.
On the other hand, as pointed out in Section \ref{sec-intro}, the simplified trajectories have unique attributes that can be utilized to improve the accuracy of map-matching.
%\subsubsection{Subgraph Extraction}

First of all, the simplified trajectories are error-bounded. As illustrated in
\myfig{fig:subgraph}, the error bound of the trajectory simplification algorithm defines a range in which the raw
trajectory points may reside. We can extract a small part of graph $G_S(V_S,E_S)$
from the original road network $G(V,E)$ and execute the matching process in the subgraph. This strategy shrinks the searching range and
improves efficiency.
More specifically, we set the range of subgraph as a rectangle range
with the simplified line segment $\mathcal{L}$ as the axis of symmetry, width as $w =
2\times(\epsilon + r_S)$ and length as $l = \mathcal{L}.L_L + \mathcal{L}.L_R +
2\times r_S$, in which $\epsilon$ is the error bound used in trajectory
simplification and $r_S$ is the searching radius used for candidate paths selection.




%\subsubsection{Action Graph Construction and Weight Estimation}
% Traditional map matching algorithms carry out local route recovery through
% shortest path searching directly in road network($G(V,E)$). However, because the simplified
% trajectories are sparse, the results achieved from shortest path
% searching will not lead to a reasonable route. Hence, other factors considering
% the rationality of routes should be taken into consideration.



Then, we extract an \emph{\emph{action graph}} from the road network incorporating the information summarized from the raw and simplified trajectories,  and use it to describe the actions of the user.
%The optimal pathes can be detected through the shortest path searching in the graph.
%An action graph is a graph extracted from road network, Osogami and Raymond first proposed it in \cite{Osogami:2013:IRL}.
The action graph is first proposed in \cite{Osogami:2013:IRL}, as shown in \myfig{fig:action-graph}, a node in the action graph represents  a
road segment, and an edge describes an action of travelling from one road segment to a neighboring one. An edge is also
associated with a weight representing the possibility of taking this action, which can be estimated from the features of the
trajectory and the path.
%
In the estimation of action weight, besides the length of road segment and the turning angle between two road segments, we further consider the similarity between the raw sub-trajectory and the path between two neighbouring points of a simplified trajectory, by computing both the distribution of the raw sub-trajectory data points and the ratio of path located on both sides of the simplified line segment.
%
Obviously, if we select a path having similar distribution with the raw sub-trajectory \wrt a simplified line segment, then the accuracy of map-matching should be improved.
%
To sum up, we estimate the weight of an action by the sum of three terms (Equation~\ref{equ:cost}): %according to the distribution of raw trajectory points by
\begin{equation}
    %\vspace{-2ex}
    \omega = \omega_{L} + \alpha \times \omega_{T} + \beta \times \omega_{\phi}
    \label{equ:cost}
\end{equation}
where $\omega_{L}$ is the length of the ending road segment of the action,
$\omega_{T}$ is the cost of turning from the starting road segment to the ending
one in the action, $\omega_{\phi}$ is the similarly
between two distributions, \ie the distributions of the path and the raw sub-trajectory on the two sides of a simplified
line segment, and $\alpha$ and $\beta$ are two user defined parameters.

\begin{figure}
%\vspace{1ex}
  \begin{subfigure}{0.36\textwidth}
  \centering
  \includegraphics[width = \textwidth, height = 0.6\textwidth]{Figures/Fig-subgraph.png}
  \end{subfigure}
  \vspace{-2ex}
  \caption{\small {A subgraph of a road network.}}
  \label{fig:subgraph}
\vspace{-2ex}
\end{figure}



%%%%%%%%%%%%%%%%%%%%%%%%%%%%%%%%%%%%%%%%%%%%%%%%%
\eat{
\begin{figure}
  \begin{subfigure}{0.36\textwidth}
    \includegraphics[width = \textwidth, height = 0.55\textwidth]{Figures/Fig-traj-side.png}
  \end{subfigure}
  \vspace{-2ex}
  \caption{\small {A sub trajectory is simplified to a line segment, which splits the plane into two parts and most points of the sub trajectory concentrate on one side.}}\vspace{-2ex}
  \label{fig:traj-sides}
\end{figure}
}

More specifically, the cost of turning $\omega_{T}$ is estimated by Equation \ref{equ:turning} defined in \cite{Osogami:2013:IRL}:
\begin{equation}
  \omega_{T} = \left\{
    \begin{aligned}
      0 & \ \ & \theta_{s,e} < \pi / 4 \\
      1 & \ \ & \pi / 4 \le \theta_{s,e} < 3\pi /4 \\
      2 & \ \ & 3\pi / 4 \le \theta_{s,e} \le \pi  \\
    \end{aligned}
  \right.
  \label{equ:turning}
\end{equation}
where, $\theta_{s,e}$ is the turning angle from the starting road segment of the action to the ending one.


The similarity estimation $\omega_{\phi}$ is modeled using a
piecewise function (Equation \ref{equ:sim}). The intuition behind this is that the correct path should have similar ratio of travelling distance on each side of the simplified line segment as the raw trajectory.
%We compute the distance of road segments residing on {one} side of the simplified trajectory and {compare it with the distribution of the corresponding raw sub-trajectory}.

\begin{equation}
  \omega_{\phi} = \left\{
    \begin{aligned}
      1 & \ \ & (\phi_T < 0.25 \wedge \phi_{R} < 0.25) \\
      1 & \ \ & (\phi_T > 0.75 \wedge \phi_{R} > 0.75) \\
      10 & \ \ & (\phi_T < 0.25 \wedge \phi_{R} > 0.75) \\
      10 & \ \ & (\phi_T > 0.75 \wedge \phi_{R} < 0.25) \\
      5 & \ \ & otherwise \\
    \end{aligned}
  \right.
  \label{equ:sim}
\end{equation}
where $\phi_T= \frac{\mathcal{L}.L_L}{\mathcal{L}.L_L + \mathcal{L}.L_R}$ is the ratio of trajectory length residing in the left side of the
simplified line segments, and $\phi_R= \frac{\sum_{r_j \in R}r_j.L_L}{\sum_{r_j \in R}r_j.L_L + r_j.L_R}$ is that of the {roads/path}.

%\begin{equation}
%\phi_L = \frac{\mathcal{L}.L_P}{\mathcal{L}.L_P + \mathcal{L}.L_N}
%\end{equation}

%\begin{equation}
%\phi_R = \frac{\sum_{r_j \in R}r_j.L_P}{\sum_{r_j \in R}r_j.L_P + r_j.L_N}
%\end{equation}

Finally, {a local optimal path can be detected by shortest path search on the action graph.}
%by minimizes the action weight .

\begin{figure}
    \begin{subfigure}{0.36\textwidth}
        \centering
        \includegraphics[width = \textwidth, height = 0.6\textwidth]{Figures/Fig-action-graph-en.pdf}
    \end{subfigure}
    \vspace{-2ex}
    \caption{\small {An example of action graph.}}\label{fig:action-graph}
    \vspace{-3ex}
\end{figure}



%%% Local Variables:
%%% mode: latex
%%% TeX-master: "gis18"
%%% End:
\subsection{Global Route Decoding}
\label{sec:hmm}

We use hidden markov model to find the global optimal route given the local
optimal paths produced by the local path recovery component.
The key of HMM modeling is to define two probabilities, \ie the \emph{emission probabilities} and the \emph{transition probabilities}.
%
A emission probability gives the likelihood that an observation is resulted from
a given state. We adopt the widely used {emission probability estimation} proposed by Newson and Krumm in \cite{Newson2009Hidden}. Note the candidate road segments are selected under the distance constraint, which means that a point would not be matched to a road segment having a distance large than the error bound.

\eat{
\subsubsection{Emission Probabilities}
Emission probabilities give the likelihood that an observation is resulted from a given state. For map-matching, it is likely that the vehicle with a GPS point $P_i$ is on a specific road segment $r_i^j$.
For those candidate road segments close to the GPS point, the emission probability is
dependent on the distance between the GPS point and the road segment, and is estimated by Equation~\ref{equ:emi-prob} proposed by Newson and Krumm \cite{Newson2009Hidden}:
%, using a Gaussian kernel. Specifically, emission probability is modeled as:
\begin{equation}
  \label{equ:emi-prob}
  E(r_i^k| P_i) = \frac{1}{\sqrt{2\pi}\sigma} \exp \frac{d(r_i^k, P_i)^2}{2\sigma^2}
\end{equation}
where, $d(r_i^k, P_i)$ is the great circle distance on the surface of the earth between the observed location $p_i$ and the candidate road segment $r_i^k$. And
$\sigma$ is the standard length of GPS measurements, which can be estimated
from data.
}



%\subsubsection{Transition Probabilities}
A transition probability is the probability of an object moving from
one road segment to another. The appropriate definition of transition
probability is the key of HMM modeling.
In Newson and Krumm's modeling, transition probabilities are modeled based on
the difference of great circle distance of two observed locations and their
corresponding road segments.
% The intuition is that transitions whose driving
% distance is about the same as the great circle distance is more likely to be the
% actual route traveled.
This modeling is reasonable when trajectory is relatively dense, i.e.
distance between two neighboring points is small. However, when a trajectory is sparse,
it is possible that more than one route have similar driving distances as the great
circle distance, thus, roughly choosing the one with the smallest difference may lead
to a circuitous route.
%
To correctly handle this problem, we make use of information from
simplified lines, and propose a model of transition probability based on
similarity.
%
Specifically, we define the transition probability $T$ of moving from road segment
$r_{i-1}^j$ to $r_i^k$ as :
\begin{equation}
  \label{equ:trans-prob}
  T(r_i^k| r_{i-1}^j) = \lambda_De^{-\lambda_D\delta_D}\lambda_Re^{-\lambda_R\delta_R}
\end{equation}
%
\begin{equation}
  \delta_R = |\phi_T -\phi_R|
\end{equation}
%
\begin{equation}
  \delta_D = d_T(P_i,P_{i-1}) - d_R(P_i,P_{i-1})
\end{equation}
%
\begin{equation}
  d_T(P_i,P_{i-1}) = \overline{\mathcal{T}}[i].L_L + \overline{\mathcal{T}}[i].L_R
\end{equation}
%
\begin{equation}
  d_R(P_i,P_{i-1}) =  \sum_{r_j \in R_{i-1, i}}{(r_j.L_L + r_j.L_R)}
\end{equation}
%
where $d_T(P_i,P_{i-1})$ is the length of the raw
sub-trajectory, $d_R(P_i,P_{i-1})$ is the length of the local optimal path $R_{i-1, i}$ on road network detected by the local path recovery component, $\delta_D $ is the
difference of $d_T$ and $d_R$, and $\delta_R $ is the difference of the
ratio of trajectory length residing in the left side of the
simplified line segment and that of the path.

\eat{
%After we estimated the emission probabilities and transition probabilities by Equations \ref{equ:emi-prob} and \ref{equ:trans-prob},
We then use the Viterbi algorithm to search for the optimal route. The Viterbi algorithm is a dynamic programming
algorithm that can quickly detect a sequence of states that maximizes the joint
probability, which is the product of the emission probabilities and transition
probabilities of all the states in the sequence.
The detected sequence is the route with maximum likelihood and thus the global optimal route.
}



% \subsection{Algorithm}

With emission probabilities and transition probabilities estimated from Equations
\ref{equ:emi-prob} and \ref{equ:trans-prob}, we can use the Viterbi algorithm to
compute the optimal path. The Viterbi algorithm is a dynamic programming
algorithm that can quickly detect a sequence of states that maximizes the joint
probability, which is the product of the emission probabilities and transition
probabilities of all the states in the sequence. The detected sequence is the
path with maximum likelihood and thus the global optimal path.



\begin{large}
\begin{algorithm}
\caption{The CT-MM Algorithm}\label{alg:viterbi}
\small
\begin{algorithmic}[1]
 \State  \textbf{Input}: $\overline{\mathcal{T}}$,$\epsilon$,G(V,E)
 \State  \textbf{Output}: map matching result R

 \State $E_1 \gets findCand(G(V,E),p_1)$.

 \State \textbf{Initialize} $f[r_1^k] = p(r_{1}^k|p_{1}), k = 1,2,\cdots,10$.

 % \For{each line segments in $\overline{\mathcal{T}}$}
 \For{$i = 2 \to n$}
   % \State extract candidate set of that GPS point from road network.
   \State $P_i \gets \overline{\mathcal{T}}[i].e$
   \State $P_{i-1} \gets \overline{\mathcal{T}}[i].s$
   \State $E_i \gets findCand(G(V,E),P_i)$.\Comment{find candidate set}
   % \State extract subgraph between the previous and current GPS point.
   \State $G_s \gets extract(G(V,E),\overline{\mathcal{T}}[i],\epsilon)$.
   \Comment{extract subgraph}
   \For{$r_i^j \in E_i$}
      \State compute $p(r_{i}^j|P_{i})$ by equation (6)
      \State $f[r_{i}^j] = -\infty$
      \For{$r_{i-1}^k \in E_{i-1}$}
      % \State compute shortest path from $r_{i-1}$ to $r_{i})$ in the sub action graph.
      % \State $sp_i^j\gets shortestPath(r_{i-1},r_{i},sbGraph)$
      \State $R_{j,k} \gets PathRec(G_s,r_i^j,r_{i-1}^k)$.\Comment{path recovery}
      \State compute $p(r_{i-1}^k,r_{i}^j)$ by equation (7)
      \State $Conj\gets f[r_{i-1}^j] * p(r_{i-1}^j,r_{i}^k) * p(r_{i}^j|P_{i})$
      \If{$Conj \ge f[r_{i}^k]$}
        \State $f[r_{i}^k] = Conj$
        \State $Pre[r_{i}^k] = r_{i-1}^j$
      \EndIf
    \EndFor
  \EndFor
  \EndFor
  \State $R = \argmax_{r_1^{k_1}\rightarrow r_2^{k_2}\rightarrow \cdots \rightarrow r_n^{k_n}}f[r_{n}^{k_n}]$
\State \textbf{return} $R$\Comment{The matched result is R}

% \Procedure{getSubGraph}{$p_{i-1},p_i,length,width$}
% \State get bounding box.
% \EndProcedure
\end{algorithmic}
\end{algorithm}
\end{large}


