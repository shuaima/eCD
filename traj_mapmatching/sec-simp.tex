
%%% Local Variables:
%%% mode: latex
%%% TeX-master: "www2019"
%%% End:

\subsection{Trajectory Simplification}
\label{sec:simp}

\todo{how to simp a trajectory, what is PED and SED,  and use which algorithm}
\myblue{
As illustrated in section \ref{sec-pre},\ped and \sed
are two distance metrics used in trajectory simplification algorithms.
We choose one pass algorithms Sleeve\cite{Zhao:Sleeve} and CISED\cite{Lin:Cised} to carry out trajectory
simplification using \ped and \sed, respectively.
}

\todo{how to pick out the features of a traj}
\myblue{
As the simplification procedure progress, we also count the travelling distance on each side of the simplified
line segment, i.e., $\mathcal{L}.L_P} = \sum_j $ and $\mathcal{L}.L_P}$.
}


The distribution of the raw sub-trajectory points according to a simplified line segment can be calculated and saved in advance during trajectory simplification.
Indeed, the distribution of raw trajectory points is quite uneven under certain error bound.
%\myfig{fig:traj-sides} is a typical example of the uneven distribution phenomenon, where green lines are simplified line segments and blue
%points are raw trajectory points, and
\myfig{fig:traj-side-stat} shows the percentage of points on one side of the simplified line segments, where more than $85\%$ of raw trajectory points are located on one side of the corresponding simplified line segments.


\begin{figure}
  \centering
  \begin{subfigure}{0.34\textwidth}
    \centering
    \includegraphics[width = \textwidth, height = 0.62\textwidth]{Figures/Exp-statistic-side-ratio.png}
  \end{subfigure}
  \vspace{-2ex}
  \caption{\small Distribution of trajectory points.} \vspace{-2ex}
  \label{fig:traj-side-stat}
\end{figure}