%%% Local Variables:
%%% mode: latex
%%% TeX-master: "gis18"
%%% End:
\section{evaluation}
\label{sec-exp}

In this section, we present an extensive experimental study of our \stmm compared with two state-of-the-art algorithms designed for sparse trajectories. %designed for sparse trajectories
Using two real-life trajectory datasets, we conducted two sets of experiments
to evaluate the accuracies and running time of these methods.
%
%In the rest of this section, we first describe the setup of our experiments and estimating
%the parameters of models. Subsequently,

\subsection{Datasets and Experimental Setting}
We first introduce the settings of our experimental study.

\subsubsection{Real-life trajectory datasets}
We use two real-life datasets \sercar and \pubdata to test algorithms.
These raw trajectories are first simplified by algorithms, \ie {Sleeve \cite{Zhao:Sleeve}} using PED and CISED \cite{Lin:Cised} using SED, respectively, and then taken as input to map-matching algorithms.
%shown in Table~\ref{tab:datasets}

\vspace{0.5ex}
\ni \emph{(1) Service car trajectory data} (\sercar) is the GPS trajectories
collected by a Chinese car rental company during Apr. 2015 to Nov. 2015. The
sampling rate was one point per $3$--$5$ seconds, and
each trajectory has around $65.8K$ points. We manually labeled 100 trajectories
for the experiments.
%.We randomly chose $1,000$ cars from them

\vspace{0.5ex}
\ni \emph{(2) Public available data} (\pubdata) \cite{pubdata} is a set of GPS trajectories
consisting of 100 trajectories all over the world, and each track is associated
with  a map and a correctly map matched route.%, which will be used as the label in evaluation.
The tracks are collected with a sampling rate at one point per second.


\eat{
\begin{table}
	\vspace{-1ex}
	\caption{\small Real-life trajectory datasets}
	\centering
	\small
	\begin{tabular}{|l|c|c|c|r|}
		\hline
		\kw{Data}& \kw{Number\ of}     &\kw{Sampling}   &\kw{Points~Per}    &\kw{Total} \\
		\kw{Sets} & \kw{Trajectories}   &\kw{Rates (s)}  &\kw{Trajectory (K)}&\kw{points}\\	\hline
		\sercar	&{100}	    &5	        &{$\sim42.8$}      &{21.4M} \\	\hline
		\pubdata	&{100}	    &1	        &{$\sim42.8$}      &{21.4M} \\	\hline
	\end{tabular}
	\label{tab:datasets}
	\vspace{-3ex}
\end{table}
}



\subsubsection{Map-matching algorithms}
We compare our method (\stmm) with two state-of-the-art HMM-based methods designed for sparse trajectories.
% and report route mismatch fraction.

\ni \emph{(1) HMM-based map matching (\hmmbased)}\cite{Newson2009Hidden}.
The HMM-based map matching algorithm that considers noise and sparseness of trajectories.
%which measures the transition probabilities using difference between great circle distance and road network distance.

\ni \emph{(2) General feature-based map matching (\gfbased)}\cite{yin:feature-based} .
It extends the HMM-based map matching by using the action graph for route
recovery and gets better accuracy than the HMM-based method on sparse trajectories.
% and estimates the transition probability between two road segments by a segment-based strategy.

% \ni \emph{(3) map matching on simplified trajectories (\proposed)}. The map matching algorithm proposed in this paper.
%It considers the properties of compressed trajectories in local path recovery and the computation of transition probabilities.

Throughout the experiments, for all methods, we set the standard deviation of GPS measurements (in estimating the emission probabilities),
to 10 meters; for \hmmbased, we set the scaling factor (in estimating the transition probabilities) to 2; and for \gfbased, we set
the balancing factor $\omega$ to 200.

\eat{
Setting of alpha and beta of equation 1.
\begin{table}
	\vspace{-1ex}
	\caption{\small Estimated parameters (What's the meaning?}
	\centering
	\small
	\begin{tabular}{|l|c|c|c|c|c|r|}
		\hline
    {$\epsilon$}  &10	&40 &200 &400 &800 &1000 \\ \hline
		{$\alpha$} &{100}	    &100	   &100 &200 &400 &400        \\	\hline
		{$\beta$} &{0}	    &0.5	   &1.5 &7.0 &3.75 &4.75        \\	\hline
	\end{tabular}
	\label{tab:params}
	\vspace{-3ex}
\end{table}
}


% All algorithms were implemented with Java.
% All tests were run on an x64-based  PC with 8 Intel(R) Core(TM) i7-6700 CPU @ 3.40GHz and 32GB of memory.
%, and each test was repeated over 3 times and the average is reported here.


%\subsection{Evaluation metrics}

%We study the accuracies and running time of each method.

%$L_R$ is the length of matched result,
% As illustrated in \myfig{fig:RMF}
% \begin{figure}
%  \centering
%  \begin{subfigure}{0.4\textwidth}
%    \includegraphics[width=\textwidth]{Figures/Fig-RMF-EN.pdf}
%  \end{subfigure}
%  \caption{Route Mismatch Fraction.}
%  \label{fig:RMF}
%  \vspace{-5ex}
% \end{figure}

% \ni (2) F-score. We also use F-score, usually used to evaluate the accuracy of a
% classification algorithm, as a measure of matching accuracy. F-score, $F = 2
% \times \frac{precision \times recall}{precision + recall}$, is the harmonic mean
% of \emph{precision} and \emph{recall}, where \textcolor{red}{$precision = (L_R -
%   L_+) / L_R$ of a matched route $R$},
% is defined by the ratio of the number of the road segments along R that are also
% contained in ground truth against the total number of road segments in R.
% $recall = (L_R - L_+) / L_T$ is defined as the ratio of the total distance traversed on
% the matched route against that of the ground truth route.

% Besides, we use the Running time to show the speed of map-matching.

% %%%%%%%%%%%%%%%%%%%%%%%%%%%%%%%%%%%%%%%%%%%%%%%%%%%%%%%%%%%%%%%%%%%%%%%%%%%%%%
% \subsection{Experimental Results}
% %%%%%%%%%%%%%%%%%%%%%%%%%%%%%%%%%%%%%%%%%%%%%%%%%%%%%%%%%%%%%%%%%%%%%%%%%%%%%%
% We next present our findings.

\subsection{Experimental Results}

%We next present our findings.

% \begin{table}
% 	\vspace{-1ex}
% 	\caption{\small Estimated parameters (What's the meaning?}
% 	\centering
% 	\small
% 	\begin{tabular}{|l|c|c|c|c|c|r|}
% 		\hline
%     {$\epsilon$}  &10	&40 &200 &400 &800 &1000 \\ \hline
% 		{$\alpha$} &{100}	    &100	   &100 &200 &400 &400        \\	\hline
% 		{$\beta$} &{0}	    &0.5	   &1.5 &7.0 &3.75 &4.75        \\	\hline
% 	\end{tabular}
% 	\label{tab:params}
% 	\vspace{-3ex}
% \end{table}

% \begin{figure*}[tb!]
% 	\centering
%   \includegraphics[width=\textwidth]{Figures/Exp-statistics-all.png}
% 	\vspace{-4ex}
% 	\caption{\small Evaluation of route mismatch fraction : varying the error bound $\epsilon$.}
% 	\label{fig:stas-dis}
% 	\vspace{-3ex}
% \end{figure*}

% \stitle{Exp 1.1: Impacts of Parameters of path recovery on accuracy.}
% we use grid search to decide the path recovery parameters.
% The estimated parameters are shown in Table \ref{tab:params}.
% We can see that when threshold is set small, the estimated parameters are also
% small. As threshold increasing, the estimated parameters increases as well.
% This is because when threshold is small, trajectory is relatively dense, and
% shortest path is more reasonable.

% \stitle{Exp 1.2: Impacts of Parameters of transition probability on accuracy.}
% We first estimate the parameters for transition probability using
% statistics-based estimation.
% Specifically, we use the statistics from the ground truth routes.

% As can be seen from \myfig{fig:stas-dis}, the difference of distances $\delta_D$ and that of
% length ratio $\delta_R$ follow an exponential distribution. We estimate the rate parameter $\lambda_D$
% of the distribution as the inverse of the mean of such values.

% Similar results are achieved using measure of ratio difference as shown in \myfig{fig:stas-ratio}.

% The estimated values of {$\lambda_D$ and $\lambda_R$} are {0.34, 3.46} in
% \pubdata and {0.49, 2.84} in \sercar, respectively.

% \stitle{Exp 1.1: Impacts of error bounds on accuracy and also the comparison of map-matching algorithms.}

\subsubsection {Evaluation of accuracy}
To evaluate the accuracy of our map
matching algorithm compared with the two state-of-the-art algorithms, we set the error bound $\epsilon$
of simplification algorithm to 40m, 200m, 400m, 600m, 800m and 1000m.
The accuracy of map-matching method is evaluated using the
Route Mismatch Fraction(RMF) proposed by Newson and Krumm~\cite{Newson2009Hidden}.
RMF represents the fraction of the total length of mismatched
route with regard to that of ground truth route, {$RMF = (L_- + L_+) / L_T$}, where $L_T$ is the length of ground truth
route, $L_-$ is the total length of the road segments that are in the ground truth route but not in the matched result, while $L_+$ is that of the
road segments that are not in the ground truth route but in the matched result.



%%%%%%%%%%%%%RMF
\begin{figure}[tb!]
	\centering
  \includegraphics[scale=0.3]{Figures/Exp-epsilon-rmf-CISED-RPI-Public.png}
  \includegraphics[scale=0.3]{Figures/Exp-epsilon-rmf-CISED-RPI-SerCar.png}
  \vspace{-2ex}
  \caption{\small Evaluation of accuracy (SED).}
	\label{fig:rmf-epsilon-acc-sed}
 \vspace{-3ex}
\end{figure}

\begin{figure}[tb!]
	\centering
  \includegraphics[scale=0.3]{Figures/Exp-epsilon-rmf-SIPED-Public.png}
  \includegraphics[scale=0.3]{Figures/Exp-epsilon-rmf-SIPED-SerCar.png}
  \vspace{-2ex}
  \caption{\small Evaluation of accuracy (PED).} \label{fig:rmf-epsilon-acc-ped}
  \vspace{-3ex}
\end{figure}

%%%%%%%%%%%%%%%%% F-score %%%%%%%%%%%%
% \begin{figure*}[tb!]
% 	\centering
%   \includegraphics[height=0.3\textwidth]{Figures/Exp-epsilon-f-score-Public.png}\hspace{5ex}
%   \includegraphics[height=0.3\textwidth]{Figures/Exp-epsilon-f-score-SerCar.png}
% 	\vspace{-2.5ex}
% 	\caption{\small Evaluation of route mismatch fraction : varying the error bound $\epsilon$.}
% 	\label{fig:f-epsilon}
% 	\vspace{-2ex}
% \end{figure*}


In this test, the raw trajectories are simplified by algorithms \siped using \ped and \cised using \sed, respectively.
The experimental results are reported in \myfig{fig:rmf-epsilon-acc-sed} and \myfig{fig:rmf-epsilon-acc-ped}.
From these figures, we can find that:

\ni (1) The accuracies decrease with the increase of $\epsilon$, as a larger $\epsilon$ leads to sparser simplified trajectories. %, vice versa.

\ni (2) When error bound is large ($\epsilon \ge 200m$), algorithm \stmm has the best accuracy on both datasets.
For example, when using \sed and $\epsilon=1000$ meters, the RMF of algorithms \hmmbased, \gfbased and \stmm are {$(0.364, 0.282, 0.237)$} and
 {$(0.178, 0.135, 0.131)$} on datasets \sercar and \pubdata, respectively.
When using \ped and $\epsilon=1000$ meters, the RMF of them are {$(0.423, 0.292, 0.231)$} and
 {$(0.211, 0.167, 0.157)$} on these datasets, respectively.

\ni (3) When error bound is relatively small ($\epsilon \le 40m$), algorithm \hmmbased has the best
accuracy, \gfbased is the worst, and \stmm is in the middle.
For example, when using \sed and $\epsilon=40$ meters, the RMFs of algorithms \hmmbased, \gfbased and \stmm are {$(0.056, 0.131, 0.091)$} and
{$(0.040, 0.059, 0.029)$} on datasets \sercar and \pubdata, respectively. When using \ped and $\epsilon=40$ meters, the RMFs of them are {$(0.131, 0.174, 0.114)$} and
{$(0.045, 0.051, 0.034)$} on these datasets, respectively;

\ni (4) Algorithm \stmm has the best accuracy over all distance thresholds on both datasets.
When using \sed, the RMF of algorithm  \stmm is on average ${(74.1\%, 71.8\%)}$ of \hmmbased and ${(80.0\%, 89.4\%)}$ of \gfbased on datasets \sercar and \pubdata, respectively.
When using \ped, the RMF of algorithm  \stmm is on average ${(61.5\%, 73.4\%)}$
of \hmmbased and ${(76.1\%, 91.7\%)}$ of \gfbased on datasets \sercar and \pubdata, respectively.
% And the F-Score is on average ${(103.6\%, 101.8\%)}$ of algorithm \hmmbased and ${(102.5\%, 100.6\%)}$ of \gfbased on datasets \sercar and \pubdata, respectively.




\subsubsection {Evaluation of running time}
We then report the running time of each algorithm, as shown in \myfig{fig:rmf-epsilon-time-sed} and \myfig{fig:rmf-epsilon-time-ped}. These figures show that algorithm \stmm is the fastest on both datasets over all error bounds.
When using \sed, algorithm \stmm is on average {$(1.89, 1.98)$} times faster than \hmmbased
and \gfbased on dataset \pubdata and {$(2.23, 2.00)$} times on dataset \sercar, respectively.
When using \ped, algorithm \stmm is on average {$(1.45, 1.63)$} times faster than \hmmbased
and \gfbased on dataset \pubdata and {$(1.90, 1.87)$} times on dataset \sercar, respectively.
%
The efficiency of algorithm \stmm originates from the fact that the error
bound of trajectory simplification provides a guidance to shrink the searching
space.
However, as error bound increases, the distance between two neighboring points
becomes larger which leads to a larger searching space and makes the algorithm slower to some extent.


% \begin{figure*}[tb!]
% 	\centering
%   \includegraphics[height=0.3\textwidth]{Figures/Exp-epsilon-time-Public.png}\hspace{5ex}
%   \includegraphics[height=0.3\textwidth]{Figures/Exp-epsilon-time-SerCar.png}
% 	\vspace{-2.5ex}
% 	\caption{\small Evaluation of efficiency: varying the error bound $\epsilon$.}
% 	\label{fig:time-epsilon}
% 	\vspace{-2ex}
% \end{figure*}



\begin{figure}[tb!]
	\centering
  \includegraphics[scale=0.3]{Figures/Exp-epsilon-time-CISED-RPI-Public.png}
  \includegraphics[scale=0.3]{Figures/Exp-epsilon-time-CISED-RPI-SerCar.png}
	\vspace{-2ex}
  \caption{\small Evaluation of running time(SED).}
	\label{fig:rmf-epsilon-time-sed}
	\vspace{-3ex}
\end{figure}

\begin{figure}[tb!]
	\centering
  \includegraphics[scale=0.3]{Figures/Exp-epsilon-time-SIPED-Public.png}
  \includegraphics[scale=0.3]{Figures/Exp-epsilon-time-SIPED-SerCar.png}
	\vspace{-2ex}
  \caption{\small Evaluation of running time(PED).}
	\label{fig:rmf-epsilon-time-ped}
	\vspace{-3ex}
\end{figure}

