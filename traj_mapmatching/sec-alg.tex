\subsection{Algorithm}

With emission probabilities and transition probabilities estimated from Equations
\ref{equ:emi-prob} and \ref{equ:trans-prob}, we can use the Viterbi algorithm to
compute the optimal path. The Viterbi algorithm is a dynamic programming
algorithm that can quickly detect a sequence of states that maximizes the joint
probability, which is the product of the emission probabilities and transition
probabilities of all the states in the sequence. The detected sequence is the
path with maximum likelihood and thus the global optimal path.



\begin{large}
\begin{algorithm}
\caption{The CT-MM Algorithm}\label{alg:viterbi}
\small
\begin{algorithmic}[1]
 \State  \textbf{Input}: $\overline{\mathcal{T}}$,$\epsilon$,G(V,E)
 \State  \textbf{Output}: map matching result R

 \State $E_1 \gets findCand(G(V,E),p_1)$.

 \State \textbf{Initialize} $f[r_1^k] = p(r_{1}^k|p_{1}), k = 1,2,\cdots,10$.

 % \For{each line segments in $\overline{\mathcal{T}}$}
 \For{$i = 2 \to n$}
   % \State extract candidate set of that GPS point from road network.
   \State $P_i \gets \overline{\mathcal{T}}[i].e$
   \State $P_{i-1} \gets \overline{\mathcal{T}}[i].s$
   \State $E_i \gets findCand(G(V,E),P_i)$.\Comment{find candidate set}
   % \State extract subgraph between the previous and current GPS point.
   \State $G_s \gets extract(G(V,E),\overline{\mathcal{T}}[i],\epsilon)$.
   \Comment{extract subgraph}
   \For{$r_i^j \in E_i$}
      \State compute $p(r_{i}^j|P_{i})$ by equation (6)
      \State $f[r_{i}^j] = -\infty$
      \For{$r_{i-1}^k \in E_{i-1}$}
      % \State compute shortest path from $r_{i-1}$ to $r_{i})$ in the sub action graph.
      % \State $sp_i^j\gets shortestPath(r_{i-1},r_{i},sbGraph)$
      \State $R_{j,k} \gets PathRec(G_s,r_i^j,r_{i-1}^k)$.\Comment{path recovery}
      \State compute $p(r_{i-1}^k,r_{i}^j)$ by equation (7)
      \State $Conj\gets f[r_{i-1}^j] * p(r_{i-1}^j,r_{i}^k) * p(r_{i}^j|P_{i})$
      \If{$Conj \ge f[r_{i}^k]$}
        \State $f[r_{i}^k] = Conj$
        \State $Pre[r_{i}^k] = r_{i-1}^j$
      \EndIf
    \EndFor
  \EndFor
  \EndFor
  \State $R = \argmax_{r_1^{k_1}\rightarrow r_2^{k_2}\rightarrow \cdots \rightarrow r_n^{k_n}}f[r_{n}^{k_n}]$
\State \textbf{return} $R$\Comment{The matched result is R}

% \Procedure{getSubGraph}{$p_{i-1},p_i,length,width$}
% \State get bounding box.
% \EndProcedure
\end{algorithmic}
\end{algorithm}
\end{large}

