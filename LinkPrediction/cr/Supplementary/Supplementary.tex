\documentclass[10pt,journal,compsoc]{IEEEtran}

\special{papersize=8.5in,11in}
% *** CITATION PACKAGES ***
%
\ifCLASSOPTIONcompsoc
  % IEEE Computer Society needs nocompress option
  % requires cite.sty v4.0 or later (November 2003)
  \usepackage[nocompress]{cite}
\else
  % normal IEEE
  \usepackage{cite}
\fi
% cite.sty was written by Donald Arseneau
% V1.6 and later of IEEEtran pre-defines the format of the cite.sty package
% \cite{} output to follow that of the IEEE. Loading the cite package will
% result in citation numbers being automatically sorted and properly
% "compressed/ranged". e.g., [1], [9], [2], [7], [5], [6] without using
% cite.sty will become [1], [2], [5]--[7], [9] using cite.sty. cite.sty's
% \cite will automatically add leading space, if needed. Use cite.sty's
% noadjust option (cite.sty V3.8 and later) if you want to turn this off
% such as if a citation ever needs to be enclosed in parenthesis.
% cite.sty is already installed on most LaTeX systems. Be sure and use
% version 5.0 (2009-03-20) and later if using hyperref.sty.
% The latest version can be obtained at:
% http://www.ctan.org/pkg/cite
% The documentation is contained in the cite.sty file itself.
%
% Note that some packages require special options to format as the Computer
% Society requires. In particular, Computer Society  papers do not use
% compressed citation ranges as is done in typical IEEE papers
% (e.g., [1]-[4]). Instead, they list every citation separately in order
% (e.g., [1], [2], [3], [4]). To get the latter we need to load the cite
% package with the nocompress option which is supported by cite.sty v4.0
% and later. Note also the use of a CLASSOPTION conditional provided by
% IEEEtran.cls V1.7 and later.





% *** GRAPHICS RELATED PACKAGES ***
%
\ifCLASSINFOpdf
  % \usepackage[pdftex]{graphicx}
  % declare the path(s) where your graphic files are
  % \graphicspath{{../pdf/}{../jpeg/}}
  % and their extensions so you won't have to specify these with
  % every instance of \includegraphics
  % \DeclareGraphicsExtensions{.pdf,.jpeg,.png}
\else
  % or other class option (dvipsone, dvipdf, if not using dvips). graphicx
  % will default to the driver specified in the system graphics.cfg if no
  % driver is specified.
  % \usepackage[dvips]{graphicx}
  % declare the path(s) where your graphic files are
  % \graphicspath{{../eps/}}
  % and their extensions so you won't have to specify these with
  % every instance of \includegraphics
  % \DeclareGraphicsExtensions{.eps}
\fi
% graphicx was written by David Carlisle and Sebastian Rahtz. It is
% required if you want graphics, photos, etc. graphicx.sty is already
% installed on most LaTeX systems. The latest version and documentation
% can be obtained at:
% http://www.ctan.org/pkg/graphicx
% Another good source of documentation is "Using Imported Graphics in
% LaTeX2e" by Keith Reckdahl which can be found at:
% http://www.ctan.org/pkg/epslatex
%
% latex, and pdflatex in dvi mode, support graphics in encapsulated
% postscript (.eps) format. pdflatex in pdf mode supports graphics
% in .pdf, .jpeg, .png and .mps (metapost) formats. Users should ensure
% that all non-photo figures use a vector format (.eps, .pdf, .mps) and
% not a bitmapped formats (.jpeg, .png). The IEEE frowns on bitmapped formats
% which can result in "jaggedy"/blurry rendering of lines and letters as
% well as large increases in file sizes.
%
% You can find documentation about the pdfTeX application at:
% http://www.tug.org/applications/pdftex






% *** MATH PACKAGES ***
%
%\usepackage{amsmath}
% A popular package from the American Mathematical Society that provides
% many useful and powerful commands for dealing with mathematics.
%
% Note that the amsmath package sets \interdisplaylinepenalty to 10000
% thus preventing page breaks from occurring within multiline equations. Use:
%\interdisplaylinepenalty=2500
% after loading amsmath to restore such page breaks as IEEEtran.cls normally
% does. amsmath.sty is already installed on most LaTeX systems. The latest
% version and documentation can be obtained at:
% http://www.ctan.org/pkg/amsmath





% *** SPECIALIZED LIST PACKAGES ***
%
%\usepackage{algorithmic}
% algorithmic.sty was written by Peter Williams and Rogerio Brito.
% This package provides an algorithmic environment fo describing algorithms.
% You can use the algorithmic environment in-text or within a figure
% environment to provide for a floating algorithm. Do NOT use the algorithm
% floating environment provided by algorithm.sty (by the same authors) or
% algorithm2e.sty (by Christophe Fiorio) as the IEEE does not use dedicated
% algorithm float types and packages that provide these will not provide
% correct IEEE style captions. The latest version and documentation of
% algorithmic.sty can be obtained at:
% http://www.ctan.org/pkg/algorithms
% Also of interest may be the (relatively newer and more customizable)
% algorithmicx.sty package by Szasz Janos:
% http://www.ctan.org/pkg/algorithmicx




% *** ALIGNMENT PACKAGES ***
%
%\usepackage{array}
% Frank Mittelbach's and David Carlisle's array.sty patches and improves
% the standard LaTeX2e array and tabular environments to provide better
% appearance and additional user controls. As the default LaTeX2e table
% generation code is lacking to the point of almost being broken with
% respect to the quality of the end results, all users are strongly
% advised to use an enhanced (at the very least that provided by array.sty)
% set of table tools. array.sty is already installed on most systems. The
% latest version and documentation can be obtained at:
% http://www.ctan.org/pkg/array


% IEEEtran contains the IEEEeqnarray family of commands that can be used to
% generate multiline equations as well as matrices, tables, etc., of high
% quality.




% *** SUBFIGURE PACKAGES ***
%\ifCLASSOPTIONcompsoc
%  \usepackage[caption=false,font=footnotesize,labelfont=sf,textfont=sf]{subfig}
%\else
%  \usepackage[caption=false,font=footnotesize]{subfig}
%\fi
% subfig.sty, written by Steven Douglas Cochran, is the modern replacement
% for subfigure.sty, the latter of which is no longer maintained and is
% incompatible with some LaTeX packages including fixltx2e. However,
% subfig.sty requires and automatically loads Axel Sommerfeldt's caption.sty
% which will override IEEEtran.cls' handling of captions and this will result
% in non-IEEE style figure/table captions. To prevent this problem, be sure
% and invoke subfig.sty's "caption=false" package option (available since
% subfig.sty version 1.3, 2005/06/28) as this is will preserve IEEEtran.cls
% handling of captions.
% Note that the Computer Society format requires a sans serif font rather
% than the serif font used in traditional IEEE formatting and thus the need
% to invoke different subfig.sty package options depending on whether
% compsoc mode has been enabled.
%
% The latest version and documentation of subfig.sty can be obtained at:
% http://www.ctan.org/pkg/subfig




% *** FLOAT PACKAGES ***
%
%\usepackage{fixltx2e}
% fixltx2e, the successor to the earlier fix2col.sty, was written by
% Frank Mittelbach and David Carlisle. This package corrects a few problems
% in the LaTeX2e kernel, the most notable of which is that in current
% LaTeX2e releases, the ordering of single and double column floats is not
% guaranteed to be preserved. Thus, an unpatched LaTeX2e can allow a
% single column figure to be placed prior to an earlier double column
% figure.
% Be aware that LaTeX2e kernels dated 2015 and later have fixltx2e.sty's
% corrections already built into the system in which case a warning will
% be issued if an attempt is made to load fixltx2e.sty as it is no longer
% needed.
% The latest version and documentation can be found at:
% http://www.ctan.org/pkg/fixltx2e


%\usepackage{stfloats}
% stfloats.sty was written by Sigitas Tolusis. This package gives LaTeX2e
% the ability to do double column floats at the bottom of the page as well
% as the top. (e.g., "\begin{figure*}[!b]" is not normally possible in
% LaTeX2e). It also provides a command:
%\fnbelowfloat
% to enable the placement of footnotes below bottom floats (the standard
% LaTeX2e kernel puts them above bottom floats). This is an invasive package
% which rewrites many portions of the LaTeX2e float routines. It may not work
% with other packages that modify the LaTeX2e float routines. The latest
% version and documentation can be obtained at:
% http://www.ctan.org/pkg/stfloats
% Do not use the stfloats baselinefloat ability as the IEEE does not allow
% \baselineskip to stretch. Authors submitting work to the IEEE should note
% that the IEEE rarely uses double column equations and that authors should try
% to avoid such use. Do not be tempted to use the cuted.sty or midfloat.sty
% packages (also by Sigitas Tolusis) as the IEEE does not format its papers in
% such ways.
% Do not attempt to use stfloats with fixltx2e as they are incompatible.
% Instead, use Morten Hogholm'a dblfloatfix which combines the features
% of both fixltx2e and stfloats:
%
% \usepackage{dblfloatfix}
% The latest version can be found at:
% http://www.ctan.org/pkg/dblfloatfix




%\ifCLASSOPTIONcaptionsoff
%  \usepackage[nomarkers]{endfloat}
% \let\MYoriglatexcaption\caption
% \renewcommand{\caption}[2][\relax]{\MYoriglatexcaption[#2]{#2}}
%\fi
% endfloat.sty was written by James Darrell McCauley, Jeff Goldberg and
% Axel Sommerfeldt. This package may be useful when used in conjunction with
% IEEEtran.cls'  captionsoff option. Some IEEE journals/societies require that
% submissions have lists of figures/tables at the end of the paper and that
% figures/tables without any captions are placed on a page by themselves at
% the end of the document. If needed, the draftcls IEEEtran class option or
% \CLASSINPUTbaselinestretch interface can be used to increase the line
% spacing as well. Be sure and use the nomarkers option of endfloat to
% prevent endfloat from "marking" where the figures would have been placed
% in the text. The two hack lines of code above are a slight modification of
% that suggested by in the endfloat docs (section 8.4.1) to ensure that
% the full captions always appear in the list of figures/tables - even if
% the user used the short optional argument of \caption[]{}.
% IEEE papers do not typically make use of \caption[]'s optional argument,
% so this should not be an issue. A similar trick can be used to disable
% captions of packages such as subfig.sty that lack options to turn off
% the subcaptions:
% For subfig.sty:
% \let\MYorigsubfloat\subfloat
% \renewcommand{\subfloat}[2][\relax]{\MYorigsubfloat[]{#2}}
% However, the above trick will not work if both optional arguments of
% the \subfloat command are used. Furthermore, there needs to be a
% description of each subfigure *somewhere* and endfloat does not add
% subfigure captions to its list of figures. Thus, the best approach is to
% avoid the use of subfigure captions (many IEEE journals avoid them anyway)
% and instead reference/explain all the subfigures within the main caption.
% The latest version of endfloat.sty and its documentation can obtained at:
% http://www.ctan.org/pkg/endfloat
%
% The IEEEtran \ifCLASSOPTIONcaptionsoff conditional can also be used
% later in the document, say, to conditionally put the References on a
% page by themselves.




% *** PDF, URL AND HYPERLINK PACKAGES ***
%
%\usepackage{url}
% url.sty was written by Donald Arseneau. It provides better support for
% handling and breaking URLs. url.sty is already installed on most LaTeX
% systems. The latest version and documentation can be obtained at:
% http://www.ctan.org/pkg/url
% Basically, \url{my_url_here}.





% *** Do not adjust lengths that control margins, column widths, etc. ***
% *** Do not use packages that alter fonts (such as pslatex).         ***
% There should be no need to do such things with IEEEtran.cls V1.6 and later.
% (Unless specifically asked to do so by the journal or conference you plan
% to submit to, of course. )


% correct bad hyphenation here
\hyphenation{op-tical net-works semi-conduc-tor}

\usepackage{graphicx} % This is used to load the crest in the title page
\usepackage{subfigure}
%%%\usepackage{subfig}
\usepackage{algorithmic}
\usepackage{algorithm}
\usepackage{url}
\usepackage{times}
\usepackage{subfigure}
\usepackage{graphicx,epstopdf}
%\usepackage{latexsym}
\usepackage{amsthm,amssymb}
\usepackage{amsmath}
%\usepackage{showkeys}
\usepackage{xcolor}
\usepackage{balance}
\usepackage{cite}
\usepackage[english]{babel}

% duan
\usepackage{xspace}
\usepackage{multirow}

\newcommand{\spara}[1]{\smallskip\noindent{\bf #1}}
\newcommand{\eat}[1]{}
\newcommand{\squishlist}{
 \begin{list}{$\bullet$}
  {  \setlength{\itemsep}{0pt}
     \setlength{\parsep}{3pt}
     \setlength{\topsep}{3pt}
     \setlength{\partopsep}{0pt}
     \setlength{\leftmargin}{2em}
     \setlength{\labelwidth}{1.5em}
     \setlength{\labelsep}{0.5em}
} }
\newcommand{\squishlisttight}{
 \begin{list}{$\bullet$}
  { \setlength{\itemsep}{0pt}
    \setlength{\parsep}{0pt}
    \setlength{\topsep}{0pt}
    \setlength{\partopsep}{0pt}
    \setlength{\leftmargin}{2em}
    \setlength{\labelwidth}{1.5em}
    \setlength{\labelsep}{0.5em}
} }

\newcommand{\squishdesc}{
 \begin{list}{}
  {  \setlength{\itemsep}{0pt}
     \setlength{\parsep}{3pt}
     \setlength{\topsep}{3pt}
     \setlength{\partopsep}{0pt}
     \setlength{\leftmargin}{1em}
     \setlength{\labelwidth}{1.5em}
     \setlength{\labelsep}{0.5em}
} }

\newcommand{\squishend}{
  \end{list}
}
\newcommand{\sttab}{\rule{0pt}{8pt}\\[-3ex]}
\newcounter{ccc}
\newcommand{\bcc}{\setcounter{ccc}{1}\theccc.}
\newcommand{\icc}{\addtocounter{ccc}{1}\theccc.}
\newcommand{\myhrule}{\rule[.5pt]{\hsize}{.5pt}}
\newcommand{\mat}[2]{{\begin{tabbing}\hspace{#1}\=\+\kill #2\end{tabbing}}}
\newcommand{\stitle}[1]{\vspace{0.5ex}\noindent{\bf #1}}
\newcommand{\etitle}[1]{\vspace{0.5ex}\noindent{\em \underline{#1}}}
\newcommand{\marked}[1]{\textcolor{red}{#1}}
\newcommand{\markedb}[1]{\textcolor{blue}{#1}}

\newcommand{\eop}{\hspace*{\fill}\mbox{$\Box$}}     % End of proof
\newcounter{example}
\renewcommand{\theexample}{\arabic{example}}
\newenvironment{example}{
        \vspace{0ex}
        \refstepcounter{example}
        {\noindent\bf Example \theexample:}}{
        \eop\vspace{0ex}}
\def\copyrightspace{}


%%%%%%%%%% symbols of methods and datasets  by duan 2015-07-13
\newcommand{\DBLP}{{\sf DBLP}\xspace}
\newcommand{\SVD}{{\sf SVD}\xspace}
\newcommand{\NMF}{{\sf NMF}\xspace }
\newcommand{\Node}{{\sf NMF(Node)}\xspace}
\newcommand{\Edge}{{\sf NMF(Edge)}\xspace}
\newcommand{\Biased}{{\sf NMF(Biased)}\xspace}
\newcommand{\Aa}{{\sf AA}\xspace }
\newcommand{\Adamic}{{\sf Adamic/Adar (AA)}\xspace}
\newcommand{\RA}{{\sf RA}\xspace }
\newcommand{\Resource}{{\sf Resource Allocation (RA)}\xspace}
\newcommand{\Katz}{{\sf Katz}\xspace}
\newcommand{\BIGCLAM}{{\sf BIGCLAM}\xspace}
\newcommand{\CAMBN}{{\sf Cluster Affiliation Model for Big Networks (BIGCLAM)}\xspace}
\newcommand{\Digg}{{\sf Digg}\xspace}
\newcommand{\YouTube}{{\sf YouTube}\xspace}
\newcommand{\Flickr}{{\sf Flickr}\xspace}
\newcommand{\Wikipedia}{{\sf Wikipedia}\xspace}
\newcommand{\Twitter}{{\sf Twitter}\xspace}
\newcommand{\Friendster}{{\sf Friendster}\xspace}
\newcommand{\Nodep}{{\sf NMF(Node+)}\xspace}
\newcommand{\Edgep}{{\sf NMF(Edge+)}\xspace}
\newcommand{\Biasedp}{{\sf NMF(Biased+)}\xspace}
\newcommand{\AABiased}{{\sf AA(Biased)}\xspace}
\newcommand{\AABiasedp}{{\sf AA(Biased+)}\xspace}
\newcommand{\RABiased}{{\sf RA(Biased)}\xspace}
\newcommand{\RABiasedp}{{\sf RA(Biased+)}\xspace}

\newcommand{\ie}{\emph{i.e.,}\xspace}
\newcommand{\eg}{\emph{e.g.,}\xspace}
\newcommand{\wrt}{\emph{w.r.t.}\xspace}
\newcommand{\aka}{\emph{a.k.a.}\xspace}
\newcommand{\kwlog}{\emph{w.l.o.g.}\xspace}
\newcommand{\etal}{\emph{et al.}\xspace}
\newcommand{\resp}{\emph{resp.}\xspace}
\newcommand{\sstab}{\rule{0pt}{8pt}\\[-2.4ex]}

\newtheorem{definition}{Definition}

\newtheorem{observation}{Observation}
\newtheorem{lemma}{Lemma}
\newtheorem{prop}{Proposition}
\newtheorem{theorem}{Theorem}
\newtheorem{problem}{Problem}
\newtheorem{corollary}{Corollary}
\newtheorem{property}{Property}
\newcommand{\lemmachar}{{\unskip\nobreak\hfil\penalty50\hskip1em\hbox{}%
\nobreak\hfil\rule{1.2ex}{1.4ex}\hfil%
\parfillskip=0pt \finalhyphendemerits=0 \par}}



%\newenvironment{proof}{{\bf Proof:}}{\lemmachar\par}


\newfont{\mycrnotice}{ptmr8t at 7pt}
\newfont{\myconfname}{ptmri8t at 7pt}
\let\crnotice\mycrnotice%
\let\confname\myconfname%


\begin{document}

\title{An Ensemble Approach to Link Prediction}

\author{Liang~Duan,
        Shuai~Ma$^*$,
        Charu~Aggarwal,
        Tiejun~Ma,
        Jinpeng~Huai
\IEEEcompsocitemizethanks{\IEEEcompsocthanksitem L. Duan, S. Ma (correspondence) and J. Huai are
with SKLSDE lab, School of Computer Science and Engineering, Beihang University, China.\hfil\break
% note need leading \protect in front of \\ to get a newline within \thanks as
% \\ is fragile and will error, could use \hfil\break instead.
E-mail: \{duanliang, mashuai, huaijp\}@buaa.edu.cn
\IEEEcompsocthanksitem C. Aggarwal is with IBM Thomas J. Watson Research Center, USA.\hfil\break
E-mail: charu@us.ibm.com
\IEEEcompsocthanksitem T. Ma is with Centre for Risk Research, Department of Decision Analytics and Risk, University of Southampton, UK.\hfil\break
E-mail: tiejun.ma@soton.ac.uk}% <-this % stops an unwanted space
\thanks{Manuscript received XXX, 2016; revised XXX, 2017.}}


\markboth{IEEE Transactions on Knowledge and Data Engineering,~Vol.~X, No.~X, May~2017}%
{Shell \MakeLowercase{\textit{et al.}}: Bare Demo of IEEEtran.cls for Computer Society Journals}

\IEEEspecialpapernotice{(Supplementary Material)}

\maketitle

\begin{figure*}[tb!]
  \centering
  %\vspace{-2ex}
  % Requires \usepackage{graphicx}
  \subfigure[Digg]{\label{fig_exp_4_1_k_acc_digg}\includegraphics[width= 1.8in, height=1.2in]{eps-script/4-1-a.eps} }
  \quad\quad
  \subfigure[YouTube]{\label{fig_exp_4_1_k_acc_youtube}\includegraphics[width= 1.8in, height=1.2in]{eps-script/4-1-b.eps} }
  \quad\quad
  \subfigure[Wikipedia]{\label{fig_exp_4_1_k_acc_wikipedia}\includegraphics[width= 1.8in, height=1.2in]{eps-script/4-1-c.eps} }
  \subfigure[Digg]{\label{fig_exp_4_1_k_time_digg}\includegraphics[width= 1.8in, height=1.2in]{eps-script/4-1-e.eps} }
  \quad\quad
  \subfigure[YouTube]{\label{fig_exp_4_1_k_time_youtube}\includegraphics[width= 1.8in, height=1.2in]{eps-script/4-1-f.eps} }
  \quad\quad
  \subfigure[Wikipedia]{\label{fig_exp_4_1_k_time_wikipedia}\includegraphics[width= 1.8in, height=1.2in]{eps-script/4-1-g.eps}}
  \vspace{-1ex}
  \caption{Bagging+ vs. Bagging on accuracy and efficiency: with respect to the number $k$ of predicted links.}\label{fig_exp_4_1}
  \vspace{-1ex}
\end{figure*}
\begin{figure*}[tb!]
  \centering
  %\vspace{-2ex}
  % Requires \usepackage{graphicx}
  \subfigure[Digg]{\label{fig_exp_4_2_size_acc_digg}\includegraphics[width= 1.8in, height=1.2in]{eps-script/4-2-a.eps} }
  \quad\quad
  \subfigure[YouTube]{\label{fig_exp_4_2_size_acc_youtube}\includegraphics[width= 1.8in, height=1.2in]{eps-script/4-2-b.eps} }
  \quad\quad
  \subfigure[Wikipedia]{\label{fig_exp_4_2_size_acc_wikipedia}\includegraphics[width= 1.8in, height=1.2in]{eps-script/4-2-c.eps} }
  \subfigure[Digg]{\label{fig_exp_4_2_size_time_digg}\includegraphics[width= 1.8in, height=1.2in]{eps-script/4-2-e.eps} }
  \quad\quad
  \subfigure[YouTube]{\label{fig_exp_4_2_size_time_youtube}\includegraphics[width= 1.8in, height=1.2in]{eps-script/4-2-f.eps} }
  \quad\quad
  \subfigure[Wikipedia]{\label{fig_exp_4_2_size_time_wikipedia}\includegraphics[width= 1.8in, height=1.2in]{eps-script/4-2-g.eps}}
  \vspace{-1ex}
  \caption{Bagging+ vs. Bagging on accuracy and efficiency: with respect to the network sizes.}\label{fig_exp_4_2}
  \vspace{-1ex}
\end{figure*}


\section{Appendix A: The Choice of Values for K}

 As pointed out in \cite{yang2015}, a small
difference in $k$ will lead to different conclusions on link prediction results
and $k$ should be the number of links in the ground truth data for a fair
comparison. Here, we have further conducted three sets of experiments when
$k$ is varied from hundreds to the number of links in the ground truth data to evaluate
the impacts of $k$ on our ensemble-enabled approach compared with conventional methods
\Aa, \RA and \BIGCLAM. These are useful for choosing a proper value for $k$.



\subsection{Bagging+ vs. Bagging}
In the first set of tests, we evaluated the impacts of $k$ on the
effectiveness and efficiency of our bagging+ methods compared with bagging methods.

\stitle{Exp-4.1: Impacts of $k$}. We varied $k$ from hundreds
to the number of links in the ground truth data \cite{yang2015}.
We fixed $r = 30$ on \Digg and \YouTube, $r = 20$ on \Wikipedia according to Exp-6
and fixed other parameters to their default values.
The results of accuracy and efficiency are reported
in Figures \ref{fig_exp_4_1_k_acc_digg}--\ref{fig_exp_4_1_k_acc_wikipedia} and
Figures \ref{fig_exp_4_1_k_time_digg}--\ref{fig_exp_4_1_k_time_wikipedia}, respectively.

The accuracy results tell us that
(a) \Biasedp is the best method on all datasets,
(b) the accuracy of all methods decreases with the increase of $k$,
and (c) the accuracy of the bagging+ methods is higher than that of
their counterparts bagging methods when $k$ is small and very
close to that of their counterparts when $k$ is large.
This means that the bagging+ methods maintain the high accuracy while some of the
edges have been removed compared with their counterparts bagging methods.
This justifies the effectiveness of the bagging+ methods.


The  efficiency  results tell us that (a) the \Biasedp is the fastest, (b) the three bagging+ methods are faster
than their counterparts bagging methods, and (c) the running time of all
methods increases slightly with the increase of $k$. For instance,
\Biasedp is $(1.2, 1.1, 1.2)$ times faster than
\Biased on \Digg, \YouTube and \Wikipedia, respectively.
This justifies the efficiency of the bagging+ methods.


\stitle{Exp-4.2: Impacts of network sizes}. We evaluated the impacts of $k$ with
different network sizes when $k$ is fixed to the number of links in the ground truth data.
We used the same networks with different sizes as Exp-1.3,
fixed $k$ to the number of links in the ground truth data
and used the same setting of other parameters as Exp-4.1.
The results of accuracy and efficiency are reported
in Figures \ref{fig_exp_4_2_size_acc_digg}--\ref{fig_exp_4_2_size_acc_wikipedia} and
Figures \ref{fig_exp_4_2_size_time_digg}--\ref{fig_exp_4_2_size_time_wikipedia}, respectively.
Since we focus on the accuracy comparison in
this section, we did not report the results on \Twitter and \Friendster
as they do not have the ground truth data for comparison.

The accuracy results tell us that (a) the accuracy of bagging+ methods is very close to
that of bagging methods, and (b) the accuracy of all methods decreases with the
increase of network sizes because $k$ is smaller on the small networks, which is consistent
with the conclusion in Exp-4.1 that the accuracy is higher when $k$ is smaller. It is hard
for \Biased to provide a significant accuracy improvement when $k$ is very large, but it still provides
an efficiency advantage while keeping its accuracy close to \Node and \Edge. Similar trends have been found for \Biasedp.
This means that our bagging+ methods are effective on the networks with different sizes.

The efficiency results tell us that (a) \Biasedp is the fastest, (b) bagging+ methods are faster than their counterparts bagging
methods, and (c) the running time of all methods increases nearly linearly with the
increase of network sizes. This justifies the efficiency of the bagging+ methods.








\begin{figure*}[tb!]
  \centering
  %\vspace{-2ex}
  % Requires \usepackage{graphicx}
  \subfigure[Digg]{\label{fig_exp_5_1_k_acc_digg}\includegraphics[width= 1.8in, height=1.2in]{eps-script/1-2-a.eps} }
  \quad\quad
  \subfigure[YouTube]{\label{fig_exp_5_1_k_acc_youtube}\includegraphics[width= 1.8in, height=1.2in]{eps-script/1-2-b.eps} }
  \quad\quad
  \subfigure[Wikipedia]{\label{fig_exp_5_1_k_acc_wikipedia}\includegraphics[width= 1.8in, height=1.2in]{eps-script/1-2-c.eps} }
  \subfigure[Digg]{\label{fig_exp_5_1_k_time_digg}\includegraphics[width= 1.8in, height=1.2in]{eps-script/1-2-e.eps} }
  \quad\quad
  \subfigure[YouTube]{\label{fig_exp_5_1_k_time_youtube}\includegraphics[width= 1.8in, height=1.2in]{eps-script/1-2-f.eps} }
  \quad\quad
  \subfigure[Wikipedia]{\label{fig_exp_5_1_k_time_wikipedia}\includegraphics[width= 1.8in, height=1.2in]{eps-script/1-2-g.eps}}
  \vspace{-1ex}
  \caption{Accuracy and efficiency comparison with \Aa, \RA and \BIGCLAM: with respect to the number $k$ of predicted links.}\label{fig_exp_5_1}
  \vspace{-1ex}
\end{figure*}

\begin{figure*}[tb!]
  \centering
  %\vspace{-2ex}
  % Requires \usepackage{graphicx}
  \subfigure[Digg]{\label{fig_exp_5_2_size_acc_digg}\includegraphics[width= 1.8in, height=1.2in]{eps-script/5-2-a.eps} }
  \quad\quad
  \subfigure[YouTube]{\label{fig_exp_5_2_size_acc_youtube}\includegraphics[width= 1.8in, height=1.2in]{eps-script/5-2-b.eps} }
  \quad\quad
  \subfigure[Wikipedia]{\label{fig_exp_5_2_size_acc_wikipedia}\includegraphics[width= 1.8in, height=1.2in]{eps-script/5-2-c.eps} }
  \subfigure[Digg]{\label{fig_exp_5_2_size_time_digg}\includegraphics[width= 1.8in, height=1.2in]{eps-script/5-2-e.eps} }
  \quad\quad
  \subfigure[YouTube]{\label{fig_exp_5_2_size_time_youtube}\includegraphics[width= 1.8in, height=1.2in]{eps-script/5-2-f.eps} }
  \quad\quad
  \subfigure[Wikipedia]{\label{fig_exp_5_2_size_time_wikipedia}\includegraphics[width= 1.8in, height=1.2in]{eps-script/5-2-g.eps}}
  \vspace{-1ex}
  \caption{Accuracy and efficiency comparison with \Aa, \RA and \BIGCLAM: with respect to the network sizes.}\label{fig_exp_5_2}
  \vspace{-1ex}
\end{figure*}



\begin{figure*}[tb!]
  \centering
  \vspace{1ex}
  % Requires \usepackage{graphicx}
  \subfigure[Digg]{\label{fig_exp_6_r_acc_digg}\includegraphics[width= 1.8in, height=1.2in]{eps-script/3-4-a.eps} }
  \quad\quad
  \subfigure[YouTube]{\label{fig_exp_6_r_acc_youtube}\includegraphics[width= 1.8in, height=1.2in]{eps-script/3-4-b.eps} }
  \quad\quad
  \subfigure[Wikipedia]{\label{fig_exp_6_r_acc_wikipedia}\includegraphics[width= 1.8in, height=1.2in]{eps-script/3-4-c.eps} }
  \subfigure[Digg]{\label{fig_exp_6_r_time_digg}\includegraphics[width= 1.8in, height=1.2in]{eps-script/3-4-e.eps} }
  \quad\quad
  \subfigure[YouTube]{\label{fig_exp_6_r_time_youtube}\includegraphics[width= 1.8in, height=1.2in]{eps-script/3-4-f.eps} }
  \quad\quad
  \subfigure[Wikipedia]{\label{fig_exp_6_r_time_wikipedia}\includegraphics[width= 1.8in, height=1.2in]{eps-script/3-4-g.eps}}
  \vspace{-1ex}
  \caption{Accuracy and efficiency comparison: with respect to the number $r$ of latent factors.}\label{fig_exp_6_r}
  \vspace{-2ex}
\end{figure*}



\subsection{Comparison with \Aa, \RA and \BIGCLAM}
In the second set of tests, we evaluated the impacts of $k$ on the
effectiveness and efficiency of
our methods compared with \Aa, \RA and \BIGCLAM. We chose \Biased and \Biasedp
for the comparison since they are the best bagging methods according to Exp-4.1.



\stitle{Exp-5.1: Impacts of $k$}. Using the same setting as
Exp-4.1, we evaluated the impacts of the number $k$ of predicted links.
The results of accuracy and  efficiency  are reported
in Figures \ref{fig_exp_5_1_k_acc_digg}--\ref{fig_exp_5_1_k_acc_wikipedia} and
Figures \ref{fig_exp_5_1_k_time_digg}--\ref{fig_exp_5_1_k_time_wikipedia}, respectively.

The accuracy results tell us that (a) the bagging+ and bagging methods outperform the other methods on
most datasets, (b) both \Biased and \Biasedp have a higher accuracy than
\NMF, \Aa, \RA and \BIGCLAM, except for \Wikipedia,
and (c) \NMF is more accurate than \Aa, \RA and \BIGCLAM on most datasets.
Moreover, \Biased and \Biasedp perform consistently well on all networks (\ie more robust),
unlike \RA which works well on \Wikipedia, but poorly on the other datasets.
This means that our methods are more accurate and robust.

%the accuracy of \Biasedp is very close to that of \Biased, (c) both
%\Biased and \Biasedp have a higher accuracy than \NMF, \Aa, \RA and \BIGCLAM,
%except for \Aa and \RA on \Wikipedia when $k$ is large,
%and (d) \NMF is more accurate than \Aa, \RA and \BIGCLAM, except for \Aa and \RA on \Wikipedia.
%Indeed, \Biased improves the accuracy by $(2.8\%, 24.3\%, 70.8\%, 11.9\%)$ (\resp $(1.5\%, 34.0\%, 34.0\%, 22.9\%)$
%and $(11.0\%, -16.0\%, -51.3\%, 12.9\%)$) over \NMF, \Aa, \RA and \BIGCLAM on \Digg, \YouTube and \Wikipedia,
%respectively. Moreover, \Biased and \Biasedp perform consistently well on all networks (\ie more robust), unlike \RA which works well
%on \Wikipedia but poorly on other datasets. This verifies the robustness of our bagging methods.


The  efficiency results tell us that (a) \Biasedp is the fastest compared with
\Biased, \NMF and \BIGCLAM, (b) the two bagging methods are much faster
than \NMF and \BIGCLAM, and (c) the running time of \Aa and \RA increases
rapidly with the increase of the network degree since their complexities
are $O(nd^2\log(k))$. Indeed, the two bagging methods finished in 421 seconds on the three datasets.
%This justifies the efficiency of our methods.


\stitle{Exp-5.2: Impacts of network sizes}. Using the same setting as
Exp-4.2, we evaluated the impacts of $k$ with different network sizes.
The results of accuracy and  efficiency  are reported in Figures \ref{fig_exp_5_2_size_acc_digg} -- \ref{fig_exp_5_2_size_acc_wikipedia}
and Figures \ref{fig_exp_5_2_size_time_digg} -- \ref{fig_exp_5_2_size_time_wikipedia} , respectively.

The accuracy results tell us that (a) \Biased is more accurate
than the other methods on most datasets, (b) \Biasedp performs
as well as it counterpart \Biased, and (c) \NMF has a higher accuracy
then \Aa, \RA and \BIGCLAM on most datasets. This justifies  the
effectiveness of our methods.

The running time results tell us that (a) \Biasedp is the fastest
method compared with \Biased, \NMF and \BIGCLAM, (b) \Aa and \RA
run fast on \Digg and \YouTube with small network degree but slow
down on \Wikipedia with larger network degree, and (c) the running
time of all methods increase with the increases of network sizes.
This is consistent with the conclusion in Exp-2.2 and also justifies
the efficiency of our methods.






%\begin{table*}
%\caption{Accuracy (\%) comparison with 95\% confidence intervals
%when $k$ is the number of links in the ground truth data and fixed to the default value.}
%\label{tab_accuracy_1}
%\vspace{-2ex}
%\centering
%\newcommand{\tabincell}[2]{\begin{tabular}{@{}#1@{}}#2\end{tabular}}
%\begin{tabular}{l|c|c|c|c|c|c}
%\hline \hline Algorithm  & Digg & YouTube & Wikipedia  & Digg & YouTube & Wikipedia \\
%\hline \hline
% & \multicolumn{3}{|c}{$k$ is the number of links in the ground truth data} & \multicolumn{3}{|c}{$k$ is fixed to the default value} \\
%\hline
%\Aa & 2.35 & 1.00 &	1.56 & 3.50 &	1.57 &	2.14   \\
%\RA & 1.71 & 1.00 &	2.69 & 2.32 &	1.65 &	3.09  \\
%\BIGCLAM & 2.61 $\pm$ (0.101) &	1.09 $\pm$ (0.107) &	1.16 $\pm$ (0.018) & 3.70 $\pm$ (0.256) &	1.56 $\pm$ (0.190) &	1.82 $\pm$ (0.100)   \\
%\hline
%\NMF  & 2.84 $\pm$ (0.056)	& 1.32 $\pm$ (0.059) 	& 1.18 $\pm$ (0.070) & 4.83 $\pm$ (0.072) 	&2.12 $\pm$ (0.090) 	&2.05 $\pm$ (0.109) \\
%\hline
%\Node & 2.91 $\pm$ (0.027)	& 1.34 $\pm$ (0.083)	& 1.29 $\pm$ (0.026) & 4.91 $\pm$ (0.059)	& 2.26 $\pm$ (0.124)	& 2.28 $\pm$ (0.056) \\
%\Edge & 2.92 $\pm$ (0.025)	& 1.34 $\pm$ (0.038)	& 1.31 $\pm$ (0.023) & 4.94 $\pm$ (0.115)	& 2.27 $\pm$ (0.034)	& 2.28 $\pm$ (0.034) \\
%\Biased & 2.94 $\pm$ (0.041) & 1.34 $\pm$ (0.059) & 1.31 $\pm$ (0.018) & 4.95 $\pm$ (0.076)	& 2.28 $\pm$ (0.159)	& 2.31 $\pm$ (0.038) \\
%\hline
%\Nodep & 2.86 $\pm$ (0.020)	& 1.34 $\pm$ (0.043)	& 1.30 $\pm$ (0.031) & 4.92 $\pm$ (0.058)	& 2.15 $\pm$ (0.074)	& 2.25 $\pm$ (0.062) \\
%\Edgep & 2.89 $\pm$ (0.024)	& 1.38 $\pm$ (0.041)	& 1.30 $\pm$ (0.012) & 4.94 $\pm$ (0.047)	& 2.17 $\pm$ (0.078)	& 2.25 $\pm$ (0.014) \\
%\Biasedp & 2.89 $\pm$ (0.041)	& 1.37 $\pm$ (0.062)	& 1.32 $\pm$ (0.022) & 4.95 $\pm$ (0.096)	& 2.25 $\pm$ (0.112)	& 2.33 $\pm$ (0.045) \\
%\hline \hline
%\end{tabular}
%\vspace{-2ex}
%\end{table*}


\begin{table*}
\caption{Accuracy (\%) comparison with 95\% confidence intervals in Exp-1.2.}
\label{tab_accuracy_1}
\vspace{-2ex}
\centering
\newcommand{\tabincell}[2]{\begin{tabular}{@{}#1@{}}#2\end{tabular}}
\begin{tabular}{c|c|c|c|c|c|c|c}
\hline \hline Dataset  & $k$ & \Node & \Edge & \Biased & \Nodep & \Edgep & \Biasedp  \\
\hline
\multirow{10}{*}{\Digg}
 & $1 \times 10^4 $ & 8.94 $\pm$ (0.147) & 8.86 $\pm$ (0.186) & 9.33 $\pm$ (0.343) & 9.21 $\pm$ (0.153) & 9.38 $\pm$ (0.120) & 9.59 $\pm$ (0.197) \\
 & $2 \times 10^4 $ & 7.54 $\pm$ (0.156) & 7.62 $\pm$ (0.136) & 7.88 $\pm$ (0.304) & 7.74 $\pm$ (0.086) & 7.82 $\pm$ (0.111) & 8.00 $\pm$ (0.141)  \\
 & $3 \times 10^4 $ & 6.84 $\pm$ (0.118) & 6.93 $\pm$ (0.111) & 7.07 $\pm$ (0.190) & 6.93 $\pm$ (0.114) & 7.05 $\pm$ (0.157) & 7.13 $\pm$ (0.108)  \\
 & $4 \times 10^4 $ & 6.39 $\pm$ (0.107) & 6.43 $\pm$ (0.055) & 6.52 $\pm$ (0.162) & 6.46 $\pm$ (0.087) & 6.57 $\pm$ (0.119) & 6.59 $\pm$ (0.064) \\
 & $5 \times 10^4 $ & 6.03 $\pm$ (0.089) & 6.07 $\pm$ (0.077) & 6.13 $\pm$ (0.083) & 6.07 $\pm$ (0.059) & 6.16 $\pm$ (0.055) & 6.17 $\pm$ (0.075)  \\
 & $6 \times 10^4 $ & 5.72 $\pm$ (0.063) & 5.78 $\pm$ (0.066) & 5.79 $\pm$ (0.082) & 5.75 $\pm$ (0.034) & 5.81 $\pm$ (0.021) & 5.81 $\pm$ (0.082)  \\
 & $7 \times 10^4 $ & 5.45 $\pm$ (0.070) & 5.51 $\pm$ (0.093) & 5.53 $\pm$ (0.101) & 5.47 $\pm$ (0.034) & 5.51 $\pm$ (0.041) & 5.55 $\pm$ (0.108)  \\
 & $8 \times 10^4 $ & 5.23 $\pm$ (0.064) & 5.30 $\pm$ (0.139) & 5.32 $\pm$ (0.093) & 5.26 $\pm$ (0.059) & 5.28 $\pm$ (0.057) & 5.31 $\pm$ (0.097)  \\
 & $9 \times 10^4 $ & 5.06 $\pm$ (0.065) & 5.11 $\pm$ (0.113) & 5.12 $\pm$ (0.062) & 5.08 $\pm$ (0.067) & 5.10 $\pm$ (0.036) & 5.11 $\pm$ (0.090)  \\
 & $10 \times 10^4 $ & 4.91 $\pm$ (0.059) & 4.94 $\pm$ (0.115) & 4.95 $\pm$ (0.076) & 4.92 $\pm$ (0.058) & 4.94 $\pm$ (0.047) & 4.95 $\pm$ (0.096)  \\
\hline
\multirow{10}{*}{\YouTube}
 & $1 \times 10^4 $ & 3.75 $\pm$ (0.162) & 3.71 $\pm$ (0.191) & 3.77 $\pm$ (0.373) & 3.59 $\pm$ (0.312) & 3.43 $\pm$ (0.169) & 3.90 $\pm$ (0.171)  \\
 & $2 \times 10^4 $ & 3.16 $\pm$ (0.193) & 3.17 $\pm$ (0.105) & 3.24 $\pm$ (0.255) & 3.06 $\pm$ (0.188) & 2.94 $\pm$ (0.136) & 3.32 $\pm$ (0.107)  \\
 & $3 \times 10^4 $ & 2.95 $\pm$ (0.204) & 2.93 $\pm$ (0.089) & 2.95 $\pm$ (0.226) & 2.82 $\pm$ (0.181) & 2.74 $\pm$ (0.102) & 3.01 $\pm$ (0.137)  \\
 & $4 \times 10^4 $ & 2.78 $\pm$ (0.153) & 2.74 $\pm$ (0.078) & 2.77 $\pm$ (0.222) & 2.66 $\pm$ (0.133) & 2.62 $\pm$ (0.106) & 2.82 $\pm$ (0.126) \\
 & $5 \times 10^4 $ & 2.64 $\pm$ (0.154) & 2.63 $\pm$ (0.059) & 2.65 $\pm$ (0.220) & 2.54 $\pm$ (0.123) & 2.53 $\pm$ (0.112) & 2.65 $\pm$ (0.100)  \\
 & $6 \times 10^4 $ & 2.55 $\pm$ (0.144) & 2.54 $\pm$ (0.058) & 2.57 $\pm$ (0.206) & 2.43 $\pm$ (0.109) & 2.44 $\pm$ (0.096) & 2.55 $\pm$ (0.081)  \\
 & $7 \times 10^4 $ & 2.45 $\pm$ (0.125) & 2.45 $\pm$ (0.062) & 2.48 $\pm$ (0.181) & 2.35 $\pm$ (0.091) & 2.36 $\pm$ (0.095) & 2.46 $\pm$ (0.090)  \\
 & $8 \times 10^4 $ & 2.38 $\pm$ (0.129) & 2.38 $\pm$ (0.055) & 2.40 $\pm$ (0.168) & 2.29 $\pm$ (0.077) & 2.29 $\pm$ (0.087) & 2.38 $\pm$ (0.088)  \\
 & $9 \times 10^4 $ & 2.32 $\pm$ (0.126) & 2.33 $\pm$ (0.034) & 2.35 $\pm$ (0.159) & 2.22 $\pm$ (0.081) & 2.23 $\pm$ (0.083) & 2.31 $\pm$ (0.098)  \\
 & $10 \times 10^4 $ & 2.26 $\pm$ (0.124) & 2.27 $\pm$ (0.034) & 2.28 $\pm$ (0.159) & 2.15 $\pm$ (0.074) & 2.17 $\pm$ (0.078) & 2.25 $\pm$ (0.112)  \\
\hline
\multirow{10}{*}{\Wikipedia}
 & $1 \times 10^5 $ & 4.19 $\pm$ (0.174) & 4.14 $\pm$ (0.201) & 4.35 $\pm$ (0.150) & 3.98 $\pm$ (0.190) & 4.02 $\pm$ (0.134) & 4.27 $\pm$ (0.149)  \\
 & $2 \times 10^5 $ & 3.61 $\pm$ (0.149) & 3.56 $\pm$ (0.161) & 3.72 $\pm$ (0.101) & 3.51 $\pm$ (0.187) & 3.48 $\pm$ (0.070) & 3.73 $\pm$ (0.073)  \\
 & $3 \times 10^5 $ & 3.22 $\pm$ (0.131) & 3.18 $\pm$ (0.155) & 3.26 $\pm$ (0.092) & 3.14 $\pm$ (0.095) & 3.11 $\pm$ (0.087) & 3.32 $\pm$ (0.053)  \\
 & $4 \times 10^5 $ & 2.93 $\pm$ (0.110) & 2.92 $\pm$ (0.102) & 2.97 $\pm$ (0.083) & 2.89 $\pm$ (0.056) & 2.87 $\pm$ (0.081) & 3.04 $\pm$ (0.044) \\
 & $5 \times 10^5 $ & 2.75 $\pm$ (0.082) & 2.75 $\pm$ (0.070) & 2.80 $\pm$ (0.064) & 2.71 $\pm$ (0.049) & 2.70 $\pm$ (0.056) & 2.84 $\pm$ (0.033)  \\
 & $6 \times 10^5 $ & 2.63 $\pm$ (0.067) & 2.62 $\pm$ (0.052) & 2.66 $\pm$ (0.053) & 2.59 $\pm$ (0.056) & 2.58 $\pm$ (0.038) & 2.71 $\pm$ (0.020)  \\
 & $7 \times 10^5 $ & 2.52 $\pm$ (0.060) & 2.52 $\pm$ (0.046) & 2.54 $\pm$ (0.044) & 2.48 $\pm$ (0.060) & 2.48 $\pm$ (0.029) & 2.60 $\pm$ (0.027)  \\
 & $8 \times 10^5 $ & 2.43 $\pm$ (0.058) & 2.43 $\pm$ (0.046) & 2.45 $\pm$ (0.037) & 2.40 $\pm$ (0.064) & 2.40 $\pm$ (0.023) & 2.50 $\pm$ (0.040)  \\
 & $9 \times 10^5 $ & 2.35 $\pm$ (0.057) & 2.35 $\pm$ (0.034) & 2.38 $\pm$ (0.037) & 2.32 $\pm$ (0.062) & 2.33 $\pm$ (0.023) & 2.41 $\pm$ (0.042)  \\
 & $10 \times 10^5 $ & 2.28 $\pm$ (0.056) & 2.28 $\pm$ (0.034) & 2.31 $\pm$ (0.038) & 2.25 $\pm$ (0.062) & 2.26 $\pm$ (0.014) & 2.33 $\pm$ (0.045)  \\
\hline \hline
\end{tabular}
\vspace{-2ex}
\end{table*}

\begin{table*}
\caption{Accuracy (\%) comparison with 95\% confidence intervals in Exp-1.3.}
\label{tab_accuracy_2}
\vspace{-2ex}
\centering
\newcommand{\tabincell}[2]{\begin{tabular}{@{}#1@{}}#2\end{tabular}}
\begin{tabular}{c|r|c|c|c|c|c|c}
\hline \hline Dataset  & Network Size &  \Node & \Edge & \Biased & \Nodep & \Edgep & \Biasedp   \\
\hline
\multirow{5}{*}{\Digg}
 & $3 \times 10^4 $ & 4.19 $\pm$ (0.110) & 4.22 $\pm$ (0.138) & 4.07 $\pm$ (0.095) & 4.13 $\pm$ (0.069) & 4.14 $\pm$ (0.095) & 4.00 $\pm$ (0.069) \\
 & $6 \times 10^4 $ & 4.85 $\pm$ (0.125) & 4.86 $\pm$ (0.104) & 4.81 $\pm$ (0.107) & 4.77 $\pm$ (0.073) & 4.78 $\pm$ (0.153) & 4.71 $\pm$ (0.101)  \\
 & $9 \times 10^4 $ & 4.89 $\pm$ (0.076) & 4.88 $\pm$ (0.081) & 4.90 $\pm$ (0.087) & 4.86 $\pm$ (0.071) & 4.88 $\pm$ (0.124) & 4.83 $\pm$ (0.126)  \\
 & $12 \times 10^4 $ & 4.89 $\pm$ (0.077) & 4.91 $\pm$ (0.107) & 4.87 $\pm$ (0.141) & 4.90 $\pm$ (0.060) & 4.93 $\pm$ (0.043) & 4.90 $\pm$ (0.056) \\
 & $15 \times 10^4 $ & 4.95 $\pm$ (0.063) & 4.92 $\pm$ (0.077) & 4.96 $\pm$ (0.128) & 4.93 $\pm$ (0.063) & 4.95 $\pm$ (0.072) & 4.88 $\pm$ (0.076)  \\
\hline
\multirow{5}{*}{\YouTube}
 & $3 \times 10^5 $ & 2.01 $\pm$ (0.038) & 2.07 $\pm$ (0.103) & 2.01 $\pm$ (0.058) & 2.04 $\pm$ (0.072) & 2.03 $\pm$ (0.044) & 1.99 $\pm$ (0.091)  \\
 & $6 \times 10^5 $ & 2.17 $\pm$ (0.016) & 2.19 $\pm$ (0.084) & 2.22 $\pm$ (0.046) & 2.13 $\pm$ (0.104) & 2.13 $\pm$ (0.070) & 2.10 $\pm$ (0.069)  \\
 & $9 \times 10^5 $ & 2.21 $\pm$ (0.147) & 2.25 $\pm$ (0.141) & 2.23 $\pm$ (0.110) & 2.13 $\pm$ (0.089) & 2.16 $\pm$ (0.059) & 2.17 $\pm$ (0.030)  \\
 & $12 \times 10^5 $ & 2.31 $\pm$ (0.098) & 2.27 $\pm$ (0.087) & 2.32 $\pm$ (0.155) & 2.13 $\pm$ (0.066) & 2.15 $\pm$ (0.078) & 2.18 $\pm$ (0.059)  \\
 & $15 \times 10^5 $ & 2.28 $\pm$ (0.186) & 2.29 $\pm$ (0.123) & 2.33 $\pm$ (0.093) & 2.17 $\pm$ (0.150) & 2.18 $\pm$ (0.050) & 2.22 $\pm$ (0.123)  \\
\hline
\multirow{5}{*}{\Wikipedia}
 & $3 \times 10^5 $ & 2.07 $\pm$ (0.030) & 2.06 $\pm$ (0.037) & 2.12 $\pm$ (0.024) & 2.03 $\pm$ (0.037) & 2.03 $\pm$ (0.046) & 2.10 $\pm$ (0.014)  \\
 & $6 \times 10^5 $ & 2.19 $\pm$ (0.041) & 2.21 $\pm$ (0.038) & 2.26 $\pm$ (0.041) & 2.17 $\pm$ (0.030) & 2.14 $\pm$ (0.041) & 2.18 $\pm$ (0.042)  \\
 & $9 \times 10^5 $ & 2.23 $\pm$ (0.029) & 2.25 $\pm$ (0.027) & 2.27 $\pm$ (0.029) & 2.21 $\pm$ (0.042) & 2.20 $\pm$ (0.023) & 2.27 $\pm$ (0.020)  \\
 & $12 \times 10^5 $ & 2.24 $\pm$ (0.036) & 2.26 $\pm$ (0.028) & 2.25 $\pm$ (0.043) & 2.17 $\pm$ (0.042) & 2.21 $\pm$ (0.044) & 2.28 $\pm$ (0.042)  \\
 & $15 \times 10^5 $ & 2.29 $\pm$ (0.037) & 2.26 $\pm$ (0.057) & 2.30 $\pm$ (0.020) & 2.23 $\pm$ (0.030) & 2.24 $\pm$ (0.061) & 2.28 $\pm$ (0.024) \\
\hline \hline
\end{tabular}
\vspace{-2ex}
\end{table*}

\begin{table*}
\caption{Accuracy (\%) comparison with 95\% confidence intervals in Exp-2.1.}
\label{tab_accuracy_3}
\vspace{-2ex}
\centering
\newcommand{\tabincell}[2]{\begin{tabular}{@{}#1@{}}#2\end{tabular}}
\begin{tabular}{c|c|c|c|c|c|c|c}
\hline \hline Dataset  & $k$ & \Aa & \RA & \BIGCLAM & \NMF & \Biased & \Biasedp  \\
\hline
\multirow{10}{*}{\Digg}
 & $1 \times 10^4 $ & 7.58 & 3.63 & 5.64 $\pm$ (0.836) & 8.37 $\pm$ (0.361) & 9.33 $\pm$ (0.343) & 9.59 $\pm$ (0.197) \\
 & $2 \times 10^4 $ & 6.19 & 3.18 & 5.08 $\pm$ (0.701) & 7.31 $\pm$ (0.194) & 7.88 $\pm$ (0.304) & 8.00 $\pm$ (0.141)  \\
 & $3 \times 10^4 $ & 5.40 & 3.02 & 4.72 $\pm$ (0.507) & 6.61 $\pm$ (0.163) & 7.07 $\pm$ (0.190) & 7.13 $\pm$ (0.108)  \\
 & $4 \times 10^4 $ & 4.87 & 2.86 & 4.46 $\pm$ (0.454) & 6.17 $\pm$ (0.152) & 6.52 $\pm$ (0.162) & 6.59 $\pm$ (0.064) \\
 & $5 \times 10^4 $ & 4.50 & 2.69 & 4.26 $\pm$ (0.396) & 5.81 $\pm$ (0.104) & 6.13 $\pm$ (0.083) & 6.17 $\pm$ (0.075)  \\
 & $6 \times 10^4 $ & 4.17 & 2.34 & 4.09 $\pm$ (0.368) & 5.53 $\pm$ (0.094) & 5.79 $\pm$ (0.082) & 5.81 $\pm$ (0.082)  \\
 & $7 \times 10^4 $ & 3.94 & 2.30 & 3.98 $\pm$ (0.341) & 5.32 $\pm$ (0.079) & 5.53 $\pm$ (0.101) & 5.55 $\pm$ (0.108)  \\
 & $8 \times 10^4 $ & 3.77 & 2.33 & 3.86 $\pm$ (0.304) & 5.13 $\pm$ (0.106) & 5.32 $\pm$ (0.093) & 5.31 $\pm$ (0.097)  \\
 & $9 \times 10^4 $ & 3.62 & 2.32 & 3.77 $\pm$ (0.272) & 4.97 $\pm$ (0.079) & 5.12 $\pm$ (0.062) & 5.11 $\pm$ (0.090)  \\
 & $10 \times 10^4 $ & 3.50 & 2.32 & 3.70 $\pm$ (0.256) & 4.83 $\pm$ (0.072) & 4.95 $\pm$ (0.076) & 4.95 $\pm$ (0.096)  \\
\hline
\multirow{10}{*}{\YouTube}
 & $1 \times 10^4 $ & 2.10 & 2.16 & 1.93 $\pm$ (0.203) & 2.50 $\pm$ (0.233) & 3.77 $\pm$ (0.373) & 3.90 $\pm$ (0.171)  \\
 & $2 \times 10^4 $ & 2.00 & 1.92 & 1.80 $\pm$ (0.242) & 2.51 $\pm$ (0.188) & 3.24 $\pm$ (0.255) & 3.32 $\pm$ (0.107)  \\
 & $3 \times 10^4 $ & 1.88 & 1.75 & 1.78 $\pm$ (0.254) & 2.44 $\pm$ (0.097) & 2.95 $\pm$ (0.226) & 3.01 $\pm$ (0.137)  \\
 & $4 \times 10^4 $ & 1.81 & 1.71 & 1.73 $\pm$ (0.250) & 2.41 $\pm$ (0.104) & 2.77 $\pm$ (0.222) & 2.82 $\pm$ (0.126) \\
 & $5 \times 10^4 $ & 1.73 & 1.65 & 1.68 $\pm$ (0.215) & 2.38 $\pm$ (0.103) & 2.65 $\pm$ (0.220) & 2.65 $\pm$ (0.100)  \\
 & $6 \times 10^4 $ & 1.73 & 1.68 & 1.64 $\pm$ (0.210) & 2.32 $\pm$ (0.097) & 2.57 $\pm$ (0.206) & 2.55 $\pm$ (0.081)  \\
 & $7 \times 10^4 $ & 1.68 & 1.66 & 1.63 $\pm$ (0.213) & 2.26 $\pm$ (0.072) & 2.48 $\pm$ (0.181) & 2.46 $\pm$ (0.090)  \\
 & $8 \times 10^4 $ & 1.65 & 1.65 & 1.60 $\pm$ (0.197) & 2.22 $\pm$ (0.062) & 2.40 $\pm$ (0.168) & 2.38 $\pm$ (0.088)  \\
 & $9 \times 10^4 $ & 1.62 & 1.65 & 1.58 $\pm$ (0.193) & 2.18 $\pm$ (0.089) & 2.35 $\pm$ (0.159) & 2.31 $\pm$ (0.098)  \\
 & $10 \times 10^4 $ & 1.57 & 1.65 & 1.56 $\pm$ (0.190) & 2.12 $\pm$ (0.090) & 2.28 $\pm$ (0.159) & 2.25 $\pm$ (0.112)  \\
\hline
\multirow{10}{*}{\Wikipedia}
 & $1 \times 10^5 $ & 2.13 & 3.61 & 2.60 $\pm$ (0.146) & 3.67 $\pm$ (0.136) & 4.35 $\pm$ (0.150) & 4.27 $\pm$ (0.149)  \\
 & $2 \times 10^5 $ & 2.24 & 3.40 & 2.42 $\pm$ (0.151) & 2.80 $\pm$ (0.071) & 3.72 $\pm$ (0.101) & 3.73 $\pm$ (0.073)  \\
 & $3 \times 10^5 $ & 2.39 & 3.21 & 2.28 $\pm$ (0.190) & 2.53 $\pm$ (0.062) & 3.26 $\pm$ (0.092) & 3.32 $\pm$ (0.053)  \\
 & $4 \times 10^5 $ & 2.38 & 3.10 & 2.17 $\pm$ (0.155) & 2.43 $\pm$ (0.114) & 2.97 $\pm$ (0.083) & 3.04 $\pm$ (0.044) \\
 & $5 \times 10^5 $ & 2.33 & 3.10 & 2.08 $\pm$ (0.127) & 2.38 $\pm$ (0.152) & 2.80 $\pm$ (0.064) & 2.84 $\pm$ (0.033)  \\
 & $6 \times 10^5 $ & 2.29 & 3.10 & 2.01 $\pm$ (0.127) & 2.30 $\pm$ (0.136) & 2.66 $\pm$ (0.053) & 2.71 $\pm$ (0.020)  \\
 & $7 \times 10^5 $ & 2.25 & 3.10 & 1.95 $\pm$ (0.114) & 2.22 $\pm$ (0.125) & 2.54 $\pm$ (0.044) & 2.60 $\pm$ (0.027)  \\
 & $8 \times 10^5 $ & 2.22 & 3.10 & 1.90 $\pm$ (0.090) & 2.16 $\pm$ (0.116) & 2.45 $\pm$ (0.037) & 2.50 $\pm$ (0.040)  \\
 & $9 \times 10^5 $ & 2.18 & 3.10 & 1.85 $\pm$ (0.114) & 2.10 $\pm$ (0.115) & 2.38 $\pm$ (0.037) & 2.41 $\pm$ (0.042)  \\
 & $10 \times 10^5 $ & 2.14 & 3.09 & 1.82 $\pm$ (0.100) & 2.05 $\pm$ (0.109) & 2.31 $\pm$ (0.038) & 2.33 $\pm$ (0.045)  \\
\hline \hline
\end{tabular}
\vspace{-2ex}
\end{table*}

\begin{table*}
\caption{Accuracy (\%) comparison with 95\% confidence intervals in Exp-2.2.}
\label{tab_accuracy_4}
\vspace{-2ex}
\centering
\newcommand{\tabincell}[2]{\begin{tabular}{@{}#1@{}}#2\end{tabular}}
\begin{tabular}{c|r|c|c|c|c|c|c}
\hline \hline Dataset  & Network Size & \Aa & \RA & \BIGCLAM & \NMF & \Biased & \Biasedp  \\
\hline
\multirow{5}{*}{\Digg}
 & $3 \times 10^4 $ & 3.39 & 2.84 & 3.87 $\pm$ (0.240) & 4.40 $\pm$ (0.095) & 4.07 $\pm$ (0.095) & 4.00 $\pm$ (0.069) \\
 & $6 \times 10^4 $ & 3.50 & 2.54 & 3.08 $\pm$ (0.133) & 4.89 $\pm$ (0.154) & 4.81 $\pm$ (0.107) & 4.71 $\pm$ (0.101)  \\
 & $9 \times 10^4 $ & 3.50 & 2.40 & 2.98 $\pm$ (0.219) & 4.81 $\pm$ (0.083) & 4.90 $\pm$ (0.087) & 4.83 $\pm$ (0.126)  \\
 & $12 \times 10^4 $ & 3.50 & 2.37 & 3.08 $\pm$ (0.308) & 4.85 $\pm$ (0.132) & 4.87 $\pm$ (0.141) & 4.90 $\pm$ (0.056) \\
 & $15 \times 10^4 $ & 3.50 & 2.31 & 2.82 $\pm$ (0.161) & 4.84 $\pm$ (0.278) & 4.96 $\pm$ (0.128) & 4.88 $\pm$ (0.076)  \\
\hline
\multirow{5}{*}{\YouTube}
 & $3 \times 10^5 $ & 1.53 & 1.22 & 1.43 $\pm$ (0.052) & 1.87 $\pm$ (0.402) & 2.01 $\pm$ (0.058) & 1.99 $\pm$ (0.091)  \\
 & $6 \times 10^5 $ & 1.58 & 1.53 & 1.67 $\pm$ (0.103) & 2.10 $\pm$ (0.725) & 2.22 $\pm$ (0.046) & 2.10 $\pm$ (0.069)  \\
 & $9 \times 10^5 $ & 1.58 & 1.53 & 1.56 $\pm$ (0.080) & 1.91 $\pm$ (0.305) & 2.23 $\pm$ (0.110) & 2.17 $\pm$ (0.030)  \\
 & $12 \times 10^5 $ & 1.57 & 1.55 & 1.35 $\pm$ (0.186) & 2.02 $\pm$ (0.390) & 2.32 $\pm$ (0.155) & 2.18 $\pm$ (0.059)  \\
 & $15 \times 10^5 $ & 1.57 & 1.65 & 1.56 $\pm$ (0.076) & 2.18 $\pm$ (0.388) & 2.33 $\pm$ (0.093) & 2.22 $\pm$ (0.123)  \\
\hline
\multirow{5}{*}{\Wikipedia}
 & $3 \times 10^5 $ & 1.71 & 2.48 & 1.01 $\pm$ (0.029) & 2.13 $\pm$ (0.065) & 2.12 $\pm$ (0.024) & 2.10 $\pm$ (0.014)  \\
 & $6 \times 10^5 $ & 1.87 & 2.67 & 1.94 $\pm$ (0.063) & 2.05 $\pm$ (0.063) & 2.26 $\pm$ (0.041) & 2.18 $\pm$ (0.042)  \\
 & $9 \times 10^5 $ & 2.04 & 2.77 & 1.92 $\pm$ (0.063) & 2.06 $\pm$ (0.059) & 2.27 $\pm$ (0.029) & 2.27 $\pm$ (0.020)  \\
 & $12 \times 10^5 $ & 2.12 & 2.93 & 1.86 $\pm$ (0.043) & 2.15 $\pm$ (0.039) & 2.25 $\pm$ (0.043) & 2.28 $\pm$ (0.042)  \\
 & $15 \times 10^5 $ & 2.14 & 3.06 & 1.79 $\pm$ (0.123) & 2.13 $\pm$ (0.122) & 2.30 $\pm$ (0.020) & 2.28 $\pm$ (0.024) \\
\hline \hline
\end{tabular}
\vspace{-2ex}
\end{table*}


\begin{table*}
\caption{Accuracy (\%) comparison with 95\% confidence intervals in Exp-4.1.}
\label{tab_accuracy_5}
\vspace{-2ex}
\centering
\newcommand{\tabincell}[2]{\begin{tabular}{@{}#1@{}}#2\end{tabular}}
\begin{tabular}{c|r|r|r|r|r|r|r}
\hline \hline Dataset  & \hspace*{\stretch{1}} $k$ \hspace*{\stretch{1}} & \Node & \Edge & \Biased & \Nodep & \Edgep & \Biasedp  \\
\hline
\multirow{4}{*}{\Digg}
 & 467     & 15.46 $\pm$ (1.503) & 14.18 $\pm$ (1.003) & 16.75 $\pm$ (0.927) & 18.33 $\pm$ (0.933) & 18.20 $\pm$ (1.176) & 18.76 $\pm$ (1.040) \\
 & 4,678   & 10.69 $\pm$ (0.266) & 10.38 $\pm$ (0.414) & 11.55 $\pm$ (0.468) & 11.36 $\pm$ (0.322) & 11.03 $\pm$ (0.266) & 11.66 $\pm$ (0.415)  \\
 & 46,781  & 6.19 $\pm$ (0.100) & 6.11 $\pm$ (0.133) & 6.32 $\pm$ (0.185) & 6.24 $\pm$ (0.111) & 6.22 $\pm$ (0.096) & 6.28 $\pm$ (0.112)  \\
 & 467,816 & 2.91 $\pm$ (0.027) & 2.92 $\pm$ (0.025) & 2.94 $\pm$ (0.041) & 2.86 $\pm$ (0.044) & 2.89 $\pm$ (0.024) & 2.89 $\pm$ (0.041) \\
\hline
\multirow{4}{*}{\YouTube}
 & 806     & 4.86 $\pm$ (0.295) & 4.76 $\pm$ (1.025) & 5.16 $\pm$ (0.929) & 5.33 $\pm$ (1.074) & 5.31 $\pm$ (0.482) & 5.68 $\pm$ (0.300)  \\
 & 8,062   & 3.31 $\pm$ (0.229) & 3.34 $\pm$ (0.286) & 3.50 $\pm$ (0.181) & 3.51 $\pm$ (0.296) & 3.55 $\pm$ (0.086) & 3.66 $\pm$ (0.138)  \\
 & 80,621  & 2.17 $\pm$ (0.121) & 2.22 $\pm$ (0.108) & 2.26 $\pm$ (0.122) & 2.35 $\pm$ (0.099) & 2.37 $\pm$ (0.111) & 2.33 $\pm$ (0.072)  \\
 & 806,213 & 1.31 $\pm$ (0.043) & 1.34 $\pm$ (0.038) & 1.34 $\pm$ (0.059) & 1.34 $\pm$ (0.043) & 1.38 $\pm$ (0.052) & 1.37 $\pm$ (0.062) \\
\hline
\multirow{5}{*}{\Wikipedia}
 & 585       & 3.59 $\pm$ (0.334) & 3.56 $\pm$ (0.690) & 3.76 $\pm$ (0.716) & 3.56 $\pm$ (0.406) & 3.69 $\pm$ (0.608) & 3.42 $\pm$ (0.395)  \\
 & 5,856     & 4.70 $\pm$ (0.334) & 4.59 $\pm$ (0.305) & 4.47 $\pm$ (0.118) & 4.39 $\pm$ (0.236) & 4.38 $\pm$ (0.180) & 4.59 $\pm$ (0.277)  \\
 & 58,565    & 4.05 $\pm$ (0.122) & 3.92 $\pm$ (0.142) & 3.99 $\pm$ (0.169) & 3.97 $\pm$ (0.271) & 3.84 $\pm$ (0.141) & 4.04 $\pm$ (0.162)  \\
 & 585,659   & 2.74 $\pm$ (0.030) & 2.69 $\pm$ (0.044) & 2.74 $\pm$ (0.075) & 2.73 $\pm$ (0.057) & 2.70 $\pm$ (0.061) & 2.77 $\pm$ (0.056) \\
 & 5,856,596 & 1.30 $\pm$ (0.012) & 1.30 $\pm$ (0.020) & 1.30 $\pm$ (0.029) & 1.30 $\pm$ (0.011) & 1.29 $\pm$ (0.021) & 1.32 $\pm$ (0.014)  \\
\hline \hline
\end{tabular}
\vspace{-2ex}
\end{table*}

\begin{table*}
\caption{Accuracy (\%) comparison with 95\% confidence intervals in Exp-4.2.}
\label{tab_accuracy_6}
\vspace{-2ex}
\centering
\newcommand{\tabincell}[2]{\begin{tabular}{@{}#1@{}}#2\end{tabular}}
\begin{tabular}{c|r|c|c|c|c|c|c}
\hline \hline Dataset  & Network Size &  \Node & \Edge & \Biased & \Nodep & \Edgep & \Biasedp   \\
\hline
\multirow{5}{*}{\Digg}
 & $3 \times 10^4 $  & 4.28 $\pm$ (0.099) & 4.27 $\pm$ (0.070) & 4.15 $\pm$ (0.126) & 4.22 $\pm$ (0.099) & 4.13 $\pm$ (0.112) & 3.98 $\pm$ (0.091) \\
 & $6 \times 10^4 $  & 3.78 $\pm$ (0.075) & 3.73 $\pm$ (0.103) & 3.69 $\pm$ (0.060) & 3.68 $\pm$ (0.050) & 3.57 $\pm$ (0.117) & 3.53 $\pm$ (0.040)  \\
 & $9 \times 10^4 $  & 3.49 $\pm$ (0.027) & 3.49 $\pm$ (0.067) & 3.43 $\pm$ (0.089) & 3.39 $\pm$ (0.025) & 3.39 $\pm$ (0.043) & 3.36 $\pm$ (0.063)  \\
 & $12 \times 10^4 $ & 3.42 $\pm$ (0.044) & 3.36 $\pm$ (0.049) & 3.34 $\pm$ (0.066) & 3.33 $\pm$ (0.029) & 3.30 $\pm$ (0.030) & 3.25 $\pm$ (0.026) \\
 & $15 \times 10^4 $ & 3.22 $\pm$ (0.026) & 3.15 $\pm$ (0.050) & 3.16 $\pm$ (0.047) & 3.12 $\pm$ (0.018) & 3.12 $\pm$ (0.019) & 3.04 $\pm$ (0.048)  \\
\hline
\multirow{5}{*}{\YouTube}
 & $3 \times 10^5 $  & 1.72 $\pm$ (0.058) & 1.76 $\pm$ (0.034) & 1.67 $\pm$ (0.014) & 1.71 $\pm$ (0.037) & 1.73 $\pm$ (0.044) & 1.65 $\pm$ (0.026)  \\
 & $6 \times 10^5 $  & 1.64 $\pm$ (0.084) & 1.58 $\pm$ (0.069) & 1.52 $\pm$ (0.097) & 1.54 $\pm$ (0.062) & 1.54 $\pm$ (0.070) & 1.49 $\pm$ (0.045)  \\
 & $9 \times 10^5 $  & 1.45 $\pm$ (0.064) & 1.49 $\pm$ (0.079) & 1.40 $\pm$ (0.078) & 1.43 $\pm$ (0.022) & 1.46 $\pm$ (0.059) & 1.42 $\pm$ (0.044)  \\
 & $12 \times 10^5 $ & 1.39 $\pm$ (0.022) & 1.49 $\pm$ (0.103) & 1.32 $\pm$ (0.035) & 1.40 $\pm$ (0.055) & 1.40 $\pm$ (0.078) & 1.36 $\pm$ (0.053)  \\
 & $15 \times 10^5 $ & 1.34 $\pm$ (0.051) & 1.37 $\pm$ (0.051) & 1.34 $\pm$ (0.071) & 1.35 $\pm$ (0.042) & 1.38 $\pm$ (0.050) & 1.31 $\pm$ (0.040)  \\
\hline
\multirow{5}{*}{\Wikipedia}
 & $3 \times 10^5 $  & 2.38 $\pm$ (0.042) & 2.35 $\pm$ (0.052) & 2.43 $\pm$ (0.023) & 2.37 $\pm$ (0.026) & 2.34 $\pm$ (0.032) & 2.43 $\pm$ (0.039)  \\
 & $6 \times 10^5 $  & 1.94 $\pm$ (0.045) & 1.90 $\pm$ (0.050) & 1.93 $\pm$ (0.027) & 1.90 $\pm$ (0.029) & 1.90 $\pm$ (0.054) & 1.93 $\pm$ (0.034)  \\
 & $9 \times 10^5 $  & 1.70 $\pm$ (0.032) & 1.70 $\pm$ (0.045) & 1.73 $\pm$ (0.020) & 1.70 $\pm$ (0.026) & 1.69 $\pm$ (0.024) & 1.73 $\pm$ (0.025)  \\
 & $12 \times 10^5 $ & 1.53 $\pm$ (0.026) & 1.55 $\pm$ (0.020) & 1.56 $\pm$ (0.024) & 1.52 $\pm$ (0.022) & 1.52 $\pm$ (0.032) & 1.56 $\pm$ (0.018)  \\
 & $15 \times 10^5 $ & 1.37 $\pm$ (0.018) & 1.38 $\pm$ (0.035) & 1.38 $\pm$ (0.021) & 1.38 $\pm$ (0.032) & 1.38 $\pm$ (0.010) & 1.39 $\pm$ (0.031) \\
\hline \hline
\end{tabular}
\vspace{-2ex}
\end{table*}


\begin{table*}
\caption{Accuracy (\%) comparison with 95\% confidence intervals in Exp-5.1.}
\label{tab_accuracy_7}
\vspace{-2ex}
\centering
\newcommand{\tabincell}[2]{\begin{tabular}{@{}#1@{}}#2\end{tabular}}
\begin{tabular}{c|r|c|c|r|r|r|r}
\hline \hline Dataset  & \hspace*{\stretch{1}} $k$ \hspace*{\stretch{1}} & \Aa & \RA & \BIGCLAM & \hspace*{\stretch{1}} \NMF \hspace*{\stretch{1}}  & \Biased & \Biasedp  \\
\hline
\multirow{4}{*}{\Digg}
 & 467     & 15.42 & 3.85 & 4.37 $\pm$ (1.393) & 12.16 $\pm$ (1.178) & 16.75 $\pm$ (0.927) & 18.76 $\pm$ (1.040) \\
 & 4,678   & 9.66 & 4.10 & 6.40 $\pm$ (1.487) & 9.27 $\pm$ (0.631) & 11.55 $\pm$ (0.468) & 11.66 $\pm$ (0.415)  \\
 & 46,781  & 4.59 & 2.77 & 4.30 $\pm$ (0.403) & 5.80 $\pm$ (0.312) & 6.32 $\pm$ (0.185) & 6.28 $\pm$ (0.112)  \\
 & 467,816 & 2.35 & 1.71 & 2.61 $\pm$ (0.101) & 2.82 $\pm$ (0.121) & 2.94 $\pm$ (0.041) & 2.89 $\pm$ (0.041) \\
\hline
\multirow{4}{*}{\YouTube}
 & 806     & 2.85  & 2.73 & 1.99 $\pm$ (0.391) & 2.66 $\pm$ (1.074) & 5.16 $\pm$ (0.929) & 5.68 $\pm$ (0.300)  \\
 & 8,062   & 2.28  & 2.27 & 1.95 $\pm$ (0.241) & 2.46 $\pm$ (0.296) & 3.50 $\pm$ (0.181) & 3.66 $\pm$ (0.138)  \\
 & 80,621  & 1.64  & 1.66 & 1.61 $\pm$ (0.201) & 2.21 $\pm$ (0.099) & 2.26 $\pm$ (0.122) & 2.33 $\pm$ (0.072)  \\
 & 806,213 & 1.00  & 1.00 & 1.09 $\pm$ (0.107) & 1.32 $\pm$ (0.059) & 1.34 $\pm$ (0.059) & 1.37 $\pm$ (0.062) \\
\hline
\multirow{5}{*}{\Wikipedia}
 & 585       & 3.76 & 3.93 & 4.03 $\pm$ (0.593) & 3.15 $\pm$ (0.411) & 3.76 $\pm$ (0.716) & 3.42 $\pm$ (0.395)  \\
 & 5,856     & 3.50 & 3.40 & 3.60 $\pm$ (0.405) & 3.57 $\pm$ (0.383) & 4.47 $\pm$ (0.118) & 4.59 $\pm$ (0.277)  \\
 & 58,565    & 2.70 & 3.63 & 2.82 $\pm$ (0.126) & 4.03 $\pm$ (0.275) & 3.99 $\pm$ (0.169) & 4.04 $\pm$ (0.162)  \\
 & 585,659   & 2.29 & 3.10 & 2.02 $\pm$ (0.046) & 2.40 $\pm$ (0.082) & 2.74 $\pm$ (0.075) & 2.77 $\pm$ (0.056) \\
 & 5,856,596 & 1.56 & 2.69 & 1.16 $\pm$ (0.018) & 1.20 $\pm$ (0.031) & 1.30 $\pm$ (0.029) & 1.32 $\pm$ (0.014)  \\
\hline \hline
\end{tabular}
\vspace{0ex}
\end{table*}

\begin{table*}
\caption{Accuracy (\%) comparison with 95\% confidence intervals in Exp-5.2.}
\label{tab_accuracy_8}
\vspace{-2ex}
\centering
\newcommand{\tabincell}[2]{\begin{tabular}{@{}#1@{}}#2\end{tabular}}
\begin{tabular}{c|r|c|c|c|c|c|c}
\hline \hline Dataset  & Network Size & \Aa & \RA & \BIGCLAM & \NMF & \Biased & \Biasedp  \\
\hline
\multirow{5}{*}{\Digg}
 & $3 \times 10^4 $  & 3.45 & 2.89 & 3.94 $\pm$ (0.238) & 4.39 $\pm$ (0.089) & 4.15 $\pm$ (0.126) & 3.98 $\pm$ (0.091) \\
 & $6 \times 10^4 $  & 2.82 & 2.13 & 2.62 $\pm$ (0.112) & 3.73 $\pm$ (0.103) & 3.69 $\pm$ (0.060) & 3.53 $\pm$ (0.040)  \\
 & $9 \times 10^4 $  & 2.67 & 1.97 & 2.43 $\pm$ (0.100) & 3.41 $\pm$ (0.128) & 3.43 $\pm$ (0.089) & 3.36 $\pm$ (0.063)  \\
 & $12 \times 10^4 $ & 2.59 & 1.90 & 2.45 $\pm$ (0.178) & 3.33 $\pm$ (0.064) & 3.34 $\pm$ (0.066) & 3.25 $\pm$ (0.026) \\
 & $15 \times 10^4 $ & 2.49 & 1.76 & 2.15 $\pm$ (0.103) & 3.10 $\pm$ (0.104) & 3.16 $\pm$ (0.047) & 3.04 $\pm$ (0.048)  \\
\hline
\multirow{5}{*}{\YouTube}
 & $3 \times 10^5 $  & 1.32 & 1.01 & 1.23 $\pm$ (0.073) & 1.57 $\pm$ (0.098) & 1.67 $\pm$ (0.014) & 1.65 $\pm$ (0.026)  \\
 & $6 \times 10^5 $  & 1.15 & 1.00 & 1.28 $\pm$ (0.021) & 1.37 $\pm$ (0.301) & 1.52 $\pm$ (0.097) & 1.49 $\pm$ (0.045)  \\
 & $9 \times 10^5 $  & 1.08 & 0.89 & 1.14 $\pm$ (0.052) & 1.38 $\pm$ (0.151) & 1.40 $\pm$ (0.078) & 1.42 $\pm$ (0.044)  \\
 & $12 \times 10^5 $ & 1.02 & 0.87 & 1.04 $\pm$ (0.072) & 1.39 $\pm$ (0.138) & 1.32 $\pm$ (0.035) & 1.36 $\pm$ (0.053)  \\
 & $15 \times 10^5 $ & 1.00 & 1.00 & 1.06 $\pm$ (0.025) & 1.38 $\pm$ (0.086) & 1.34 $\pm$ (0.071) & 1.31 $\pm$ (0.040)  \\
\hline
\multirow{5}{*}{\Wikipedia}
 & $3 \times 10^5 $  & 1.78 & 2.54 & 1.05 $\pm$ (0.014) & 2.25 $\pm$ (0.050) & 2.43 $\pm$ (0.023) & 2.43 $\pm$ (0.039)  \\
 & $6 \times 10^5 $  & 1.68 & 2.50 & 1.64 $\pm$ (0.034) & 1.77 $\pm$ (0.074) & 1.93 $\pm$ (0.027) & 1.93 $\pm$ (0.034)  \\
 & $9 \times 10^5 $  & 1.69 & 2.54 & 1.50 $\pm$ (0.027) & 1.49 $\pm$ (0.088) & 1.73 $\pm$ (0.020) & 1.73 $\pm$ (0.025)  \\
 & $12 \times 10^5 $ & 1.69 & 2.64 & 1.34 $\pm$ (0.012) & 1.40 $\pm$ (0.053) & 1.56 $\pm$ (0.024) & 1.56 $\pm$ (0.018)  \\
 & $15 \times 10^5 $ & 1.61 & 2.73 & 1.21 $\pm$ (0.029) & 1.25 $\pm$ (0.083) & 1.38 $\pm$ (0.021) & 1.39 $\pm$ (0.031) \\
\hline \hline
\end{tabular}
\vspace{-2ex}
\end{table*}


\subsection{Impacts of Parameters}
In the third set of tests, we evaluated the impacts of the parameter when $k$
is the number of links in the ground truth data.
Since the parameters $\mu$, $f$, $\rho$ and $\epsilon$ have similar impacts as
shown in Section 4.2.3, we keep the default values for these parameters and
report only the impacts of the parameter $r$ on
the accuracy and  efficiency  of bagging+, bagging, \NMF and \BIGCLAM.

\stitle{Exp-6: Impacts of $r$}. To evaluate the impacts of $r$, we varied
$r$ from 10 to 50, fixed $k$ to the number of links in the ground truth data
and fixed other parameters to their values. The accuracy and  efficiency  results
are reported in Figures \ref{fig_exp_6_r_acc_digg}--\ref{fig_exp_6_r_acc_wikipedia}
and Figures \ref{fig_exp_6_r_time_digg}--\ref{fig_exp_6_r_time_wikipedia}, respectively.

The results tell us that (a) bagging+ and bagging methods are more accurate than
the other methods, (b) the accuracy of bagging+ methods is very close to that of bagging methods,
(c) the accuracy of the bagging+, bagging and \NMF increases with the increase of $r$,
and (d) the running time of all methods increase with the increases of $r$ and \Biasedp
is still the fastest one. To obtain the highest accuracy, we keep the default value of
$r$ for \NMF and \BIGCLAM, and fixed $r = 30$ on \Digg and \YouTube and $r = 20$ on
\Wikipedia for bagging+ and bagging methods.

\stitle{Remarks}. From these experimental results, we find that $k$ has great
impacts on predicting links:

\sstab (1) The bagging+ and bagging methods are more accurate than the other
methods when $k$ is varied from hundreds to the number of links in the ground
truth data, except for \RA on \Wikipedia. Particularly, the bagging+ and bagging
methods have significant improvements when $k$ is small, and also more accurate than
the others when $k$ is even the number of links in the ground truth data.

\sstab (2) \Biased is more accurate than \Node and \Edge when $k$ is small, and
very close to the \Node and \Edge when $k$ is large, \eg $k$ is fixed to the
number of links in the ground truth data. In this case, however, it also provides an efficiency
advantage. Similar trends have been found for \Biasedp.

\sstab (3) The accuracy of all methods decreases with the increase of $k$, and
becomes very low when $k$ is very large. For instance, when $k$ is the number of
links in the ground truth data, the accuracy of \Aa and \RA is only 1\% on \YouTube.
Although our methods are more accurate than the other methods under this condition, we still
keep the default value for $k$ since a higher accuracy would be more useful in
practice.


\section{Appendix B: Accuracy Results with Confidence Intervals}

\begin{figure*}[tb!]
  \centering
  %\vspace{-2ex}
  % Requires \usepackage{graphicx}
  \subfigure[Digg]{\label{fig_exp_7_1_k_acc_digg}\includegraphics[width= 1.8in, height=1.2in]{eps-script/2-1-a.eps} }
  \quad\quad
  \subfigure[YouTube]{\label{fig_exp_7_1_k_acc_youtube}\includegraphics[width= 1.8in, height=1.2in]{eps-script/2-1-b.eps} }
  \quad\quad
  \subfigure[Wikipedia]{\label{fig_exp_7_1_k_acc_wikipedia}\includegraphics[width= 1.8in, height=1.2in]{eps-script/2-1-c.eps} }
  \subfigure[Digg]{\label{fig_exp_7_1_k_time_digg}\includegraphics[width= 1.8in, height=1.2in]{eps-script/2-1-e.eps} }
  \quad\quad
  \subfigure[YouTube]{\label{fig_exp_7_1_k_time_youtube}\includegraphics[width= 1.8in, height=1.2in]{eps-script/2-1-f.eps} }
  \quad\quad
  \subfigure[Wikipedia]{\label{fig_exp_7_1_k_time_wikipedia}\includegraphics[width= 1.8in, height=1.2in]{eps-script/2-1-g.eps}}
  \vspace{-1ex}
  \caption{Ensemble-Enabled AA \& RA on accuracy and efficiency: with respect to the number $k$ of predicted links.}\label{fig_exp_7_1}
  \vspace{-1ex}
\end{figure*}


\begin{figure*}[tb!]
  \centering
  %\vspace{-2ex}
  % Requires \usepackage{graphicx}
  \subfigure[Digg]{\label{fig_exp_7_2_size_acc_digg}\includegraphics[width= 1.8in, height=1.2in]{eps-script/2-2-a.eps} }
  \quad\quad
  \subfigure[YouTube]{\label{fig_exp_7_2_size_acc_youtube}\includegraphics[width= 1.8in, height=1.2in]{eps-script/2-2-b.eps} }
  \quad\quad
  \subfigure[Wikipedia]{\label{fig_exp_7_2_size_acc_wikipedia}\includegraphics[width= 1.8in, height=1.2in]{eps-script/2-2-c.eps} }
  \subfigure[Digg]{\label{fig_exp_7_2_size_time_digg}\includegraphics[width= 1.8in, height=1.2in]{eps-script/2-2-e.eps} }
  \quad\quad
  \subfigure[YouTube]{\label{fig_exp_7_2_size_time_youtube}\includegraphics[width= 1.8in, height=1.2in]{eps-script/2-2-f.eps} }
  \quad\quad
  \subfigure[Wikipedia]{\label{fig_exp_7_2_size_time_wikipedia}\includegraphics[width= 1.8in, height=1.2in]{eps-script/2-2-g.eps}}
  \vspace{-1ex}
  \caption{Ensemble-Enabled AA \& RA on accuracy and efficiency: with respect to the network sizes.}\label{fig_exp_7_2}
  \vspace{-1ex}
\end{figure*}






We report the average of accuracy for comparison in Section 4, and here
we further report the accuracy with 95\% confidence
intervals.

The confidence interval of accuracy is given by
\[ (\overline{x} - t_{[n-1,\alpha/2]}\frac{s}{\sqrt{n}}, \overline{x} + t_{[n-1,\alpha/2]}\frac{s}{\sqrt{n}})  \]
where $\overline{x}$ is the average of the accuracy, $\alpha$ is the significance level
(we set $\alpha$ to 0.05 for the $1 - \alpha = 95\%$ confidence level),
$s$ is the sample standard deviation, $n$ is the number of repeated
times of our experiments and $t_{[n-1,\alpha/2]}$ is the ($\alpha/2$)-quantile
of Student's $t$ distribution with $n - 1$ degrees of freedom \cite{stati}.
The confidence interval gives an indication of how much uncertainty there is in
the estimate of the accuracy. The narrower the interval, the more precise is the estimate.

The accuracy results in Exp-1.2, Exp-1.3, Exp-2.1 and Exp-2.2 are shown in
Table \ref{tab_accuracy_1} -- Table \ref{tab_accuracy_4}, respectively.
Further, the accuracy results in Exp-4.1, Exp-4.2, Exp-5.1 and Exp-5.2 are shown in
Table \ref{tab_accuracy_5} -- Table \ref{tab_accuracy_8}, respectively.


\stitle{Remarks}. From the results of accuracy comparison with confidence intervals,
we find the following.

\sstab (1) \Biased and \Biasedp are more accurate than the other methods
on most datasets. Even when $k$ is the number of the ground truth links (See in Table \ref{tab_accuracy_7}), \Biased improves
the accuracy by $(2.8\%, 24.3\%, 70.8\%, 11.9\%)$ (\resp $(1.5\%, 34.0\%, 34.0\%, 22.9\%)$)
over \NMF, \Aa, \RA and \BIGCLAM on \Digg and \YouTube, respectively.
Moreover, bagging+ and bagging methods perform better than the other methods on most datasets.

\sstab (2) The accuracy of all methods decreases with the increase of $k$,
which is consistent with the previous experimental analysis.

\sstab (3) The confidence intervals of all methods are narrow, which means that the accuracy estimate is reasonable.
Thus, we only report the average accuracy for clarity.






\section{Appendix C: Ensemble-Enabled AA \& RA}

One interesting thing should be noted that our ensemble-enabled
method is in principle a general method for decomposing the large network
link prediction problem into smaller subproblems. It can not only
be applied for \NMF, and may be applied to other prediction methods.
Therefore, we implemented \AABiased and \AABiasedp
(\resp \RABiased and \RABiasedp) by replacing \NMF with \Aa (\resp \RA) in
\Biased and \Biasedp. Furthermore, we evaluated the effectiveness and efficiency
of the ensemble-enabled approach with \Aa and \RA.


\stitle{Exp-7.1: Impacts of $k$}. We varied $k$ from hundreds to the number of links in the ground truth data
and fixed other parameters to their default values. The results of accuracy
and  efficiency  are reported in Figures \ref{fig_exp_7_1_k_acc_digg}--\ref{fig_exp_7_1_k_acc_wikipedia}
and Figures \ref{fig_exp_7_1_k_time_digg}--\ref{fig_exp_7_1_k_time_wikipedia}, respectively.


The accuracy results tell us that (a) \AABiased and \AABiasedp (\resp \RABiased and \RABiasedp)
are more accurate than \Aa (\resp \RA) on most datasets,
(b) the accuracy of \AABiasedp (\resp \RABiasedp) is close to that of \AABiased (\resp \RABiased),
and (c) the improvements of the ensemble-enabled methods are significant when $k$
is small.

Note that \AABiased and \AABiasedp (\resp \RABiasedp)
perform worse than their counterpart \Aa (\resp \RA) on \Wikipedia because their diversities
on this dataset are poor. For instance, the pairwise overlapping of predicted links of \AABiased
(\resp \AABiasedp and \RABiasedp) is 0.62 (\resp 0.63 and 0.52). One reason for the decreasing
of diversity may be that the dataset contains some extreme high degree nodes,
\ie nodes with degree between $5 \times 10^4$ to $1.8 \times 10^5$ while the network contains
only $1.6 \times 10^6$ nodes.

The  efficiency  results tell us that \AABiased and \AABiasedp (\resp \RABiased and \RABiasedp)
are faster than \Aa (\resp \RA) on \YouTube and \Wikipedia but
slower on \Digg. This is consistent with the complexity analysis
that the ensemble-enabled \Aa and \RA require $O(nd_{1}^{2}\log(k)\mu/f)$ time,
where $d_1$ is the average degree of each ensemble component. It is means
that the ensemble-enabled \Aa and \RA would be faster when $d_1$ is small.
Indeed, $d_1$ of \AABiased is 9 and 56 on \YouTube and \Wikipedia while the average
degree of these datasets is 5 and 33 respectively. However, $d_1$ of \AABiased is 30
on \Digg  while the average degree of this datasets is 10.



\stitle{Exp-7.2: Impacts of network sizes}. To evaluate the impacts of
network sizes, we used networks with different sizes that used in Exp-1.3,
fixed $k$ to the number of links in the ground truth data
and fixed other parameters to their default values.
The results of accuracy and  efficiency are reported in \ref{fig_exp_7_2_size_acc_digg}--\ref{fig_exp_7_2_size_acc_wikipedia}
and Figures \ref{fig_exp_7_2_size_time_digg}--\ref{fig_exp_7_2_size_time_wikipedia}, respectively.

The accuracy results tell us that (a) the ensemble-enabled methods are
more accuracy than their counterparts \Aa and \RA  on most of the datasets,
and (b) the improvements of \RABiased and \RABiasedp on \Digg and \YouTube are significant.
This means that the ensemble-enabled methods are accurate and robust
with the increase of network sizes.

The  efficiency  results tell us that (a) the ensemble-enabled methods are
faster than their counterparts \Aa and \RA  on most of the datasets,
(b) the bagging+ methods are faster than their counterparts bagging methods,
and (c) the running time of all methods increase nearly linearly with the
increase of network sizes. Note that we also tested the ensemble-enabled approach
with \Aa and \RA on the large networks \Twitter and \Friendster, and the average
speedup of \AABiased and \AABiasedp (\resp \RABiased and \RABiasedp)
are (5, 4) and (14, 5) (\resp (7, 5) and (20, 6)) on (\Twitter and \Friendster),
which justifies the efficiency of our ensemble-enabled approach.

\stitle{Remarks}. From these results, we find the following.

\sstab (1) These results show that  ensemble-enabled \Aa and \RA  are very promising. The ensemble-enabled \Aa and \RA improve the accuracy
compared with their counterparts \Aa and \RA on some networks; The ensemble-enabled \Aa and \RA have an efficiency
advantage when the average degree of each ensemble component is small,
\eg on \YouTube and \Wikipedia.

\sstab (2)
Although our ensemble-enabled approach is a general framework that may be applied to any link prediction
methods, the  sampling techniques for generating ensembles should be redesigned for different link prediction
methods, such as \Aa and \RA to fully employ the advantage of the framework.




\begin{footnotesize}
\bibliographystyle{abbrv}
\bibliography{paper}
\end{footnotesize}


%\begin{table*}
%\caption{Accuracy (\%) comparison for ensemble-enabled \Aa and \RA.}
%\label{tab_accuracy_3}
%\vspace{-2ex}
%\centering
%\newcommand{\tabincell}[2]{\begin{tabular}{@{}#1@{}}#2\end{tabular}}
%\begin{tabular}{l|c|c|c|c}
%\hline \hline Algorithm  & Digg & YouTube & Wikipedia & Flickr  \\
%\hline \hline
%\Aa & 2.35 &	1.00 &	1.56 &  \\
%\AABiased & 2.38 $\pm$ (0.020) &	1.09 $\pm$ (0.107) &	1.16 $\pm$ (0.018) & \\
%\AABiasedp & 2.39 $\pm$ (0.007)	& 1.32 $\pm$ (0.059) 	& 1.18 $\pm$ (0.070) & \\
%\hline
%\RA & 1.71 &	1.00 &	\textbf{2.69} &  \\
%\RABiased & 2.03 $\pm$ (0.027)	& 1.34 $\pm$ (0.059)	& 1.31 $\pm$ (0.018) & \\
%\RABiasedp & 2.46 $\pm$ (0.034)	& \textbf{1.34 $\pm$ (0.044)}	& 1.30 $\pm$ (0.016) & \\
%\hline \hline
%\end{tabular}
%\vspace{-2ex}
%\end{table*}




%\begin{table}
%\caption{Accuracy (\%) comparison with \Aa, \RA and \BIGCLAM.}
%\label{tab_accuracy}
%\vspace{-2ex}
%\centering
%\newcommand{\tabincell}[2]{\begin{tabular}{@{}#1@{}}#2\end{tabular}}
%\begin{tabular}{l|c|c|c}
%\hline \hline Algorithm  & Digg & YouTube & Wikipedia  \\
%\hline \hline
%\Aa & 2.35 &	1.00 &	1.56  \\
%\RA & 1.71 &	1.00 &	\textbf{2.69}  \\
%\BIGCLAM & 2.61 $\pm$ (0.101) &	1.09 $\pm$ (0.107) &	1.16 $\pm$ (0.018)  \\
%\NMF & 2.84 $\pm$ (0.056)	& 1.32 $\pm$ (0.059) 	& 1.18 $\pm$ (0.070) \\
%\Biased & \textbf{2.92 $\pm$ (0.020)}	& 1.34 $\pm$ (0.059)	& 1.31 $\pm$ (0.018)\\
%\Biasedp & 2.88 $\pm$ (0.054)	& \textbf{1.34 $\pm$ (0.044)}	& 1.30 $\pm$ (0.016) \\
%\hline \hline
%\end{tabular}
%\vspace{-2ex}
%\end{table}


%
%\begin{table}
%\caption{Running time (sec.) comparison with \Aa, \RA and \BIGCLAM.}
%\label{tab_time2}
%\vspace{-2ex}
%\centering
%\newcommand{\tabincell}[2]{\begin{tabular}{@{}#1@{}}#2\end{tabular}}
%\begin{tabular}{l|r|r|r}
%\hline \hline Algorithm  & Digg & YouTube & Wikipedia  \\
%\hline \hline
%\Aa & 10.46 &	56.78 &	2557.63  \\
%\RA & 7.74 &	39.12 &	2136.36  \\
%\BIGCLAM & 70.34 &	598.06 &	5642.84  \\
%\NMF & 417.33 	& 3695.67 	& 4690.71 \\
%\hline
%\Node & 74.62	& 431.52	& 403.51 \\
%\Edge & 78.76	& 446.22	& 410.86 \\
%\Biased & 64.58	& 420.21	& 393.26 \\
%\hline
%\Nodep & 65.95	& 405.29	& 356.77 \\
%\Edgep & 68.26	& 415.26	& 374.73 \\
%\Biasedp & 58.91	& 399.84	& 353.12 \\
%\hline \hline
%\end{tabular}
%\vspace{-2ex}
%\end{table}


%\begin{table}
%\caption{Running time (sec.) comparison with \Aa, \RA and \BIGCLAM.}
%\label{tab_time}
%\vspace{-2ex}
%\centering
%\newcommand{\tabincell}[2]{\begin{tabular}{@{}#1@{}}#2\end{tabular}}
%\begin{tabular}{l|r|r|r}
%\hline \hline Algorithm  & Digg & YouTube & Wikipedia  \\
%\hline \hline
%\Aa & 10.46 &	56.78 &	2557.63  \\
%\RA & \textbf{7.74} &	\textbf{39.12} &	2136.36  \\
%\BIGCLAM & 70.34 &	598.06 &	5642.84  \\
%\NMF & 441.81 	& 3500.68 	& 4038.13 \\
%\Biased & 64.58	& 420.21	& 393.26 \\
%\Biasedp & 58.91	& 399.84	& \textbf{352.76} \\
%\hline \hline
%\end{tabular}
%\vspace{-2ex}
%\end{table}


\end{document}
