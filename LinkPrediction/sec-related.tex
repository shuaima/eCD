\section{Related work}
\label{sec-related}

% extension
This paper is an extension of our earlier work \cite{liang2016} by adding (a) top-$k$ and (b) experiments.

\stitle{Link Prediction}. The link prediction problem has been studied extensively in the data 
mining and machine learning community \cite{kleinberg,linyuan-2011}, which falls into unsupervised 
and supervised methods \cite{propflow}.

\stitle{NMF}.

\stitle{Ensemble}.

\stitle{Graph Sampling}.



% link prediction: survey, unsupervised vs. supervised, neighborhood-based, path-based, low-rank approximate, nmf, svd,
% refer to missing values, negative link prediction, explanation, cross networks, community detection
% bagging, graph sampling methods,


The link prediction problem has been studied extensively in the data mining and machine learning community
\cite{kleinberg,linyuan-2011}, which falls into unsupervised and supervised methods \cite{propflow}. Unsupervised methods
often assign scores to potential links based on the topology of the given graphs:
(a) Adamic/Adar \cite{adamic} is a common neighbor based method; (b) Katz \cite{katz-1953} is a
path based method which sums over all paths between two nodes, and there are also other path based
methods, such as Local Path and Random Walk with Restart \cite{linyuan-2011}; And (c)
\cite{kunegis,kleinberg} investigates the low rank approximation methods by generating a
small rank matrix to approximate the initial adjacency matrix. Supervised methods \cite{Link09,propflow,lichen3} typically treat link prediction as a classification problem, e.g., supervised matrix factorization and random walk based approaches \cite{menon,back}


Recently, several models for link prediction have been proposed,
such as  community affiliation models \cite{Yang09,yang-wsdm2013}, stochastic topic models \cite{barbieri2014},
negative link prediction models \cite{tang2015}, statistical relational models \cite{bilgic,Getoor01,Getoor02,Taskar03,yu}
and Markov models \cite{zhu}. Moreover, link prediction has also been studied for mining
missing hyperlinks \cite{adafre,west2015}.
While some recent work has focused on the heterogeneous \cite{qi,sun11,sun12,tang,yang} and temporal \cite{back,dwang} scenarios, these methods are not essentially
designed to search the entire space of $O(n^2)$ possibilities. Indeed, they are often not able to prune the search space of possibilities, and are mostly designed to evaluate the link prediction propensities of a subset of node pairs.



Our method is related to \NMF proposed in \cite{NMF-nature99}, which
has been successfully used for collaborative
filtering \cite{web}. Since the adjacency matrix in our approach is symmetric, we adopt
the symmetric \NMF method \cite{ding}. Our work is also related
to bagging predictor \cite{Breiman96b-1996} that generates an aggregated
predictor based on multiple bootstrap samples. Different from the bootstrap
sampling methods, we focus on sampling subgraphs from large networks. Although a
variety of graph sampling techniques were introduced in \cite{ahmed2014tkdd}, our approach combines link
prediction characteristics \cite{leskovec-2008} with graph sampling
methods to achieve high link prediction accuracy.



%
%The problem of link prediction has been studied extensively in the
%data mining and machine learning community \cite{chancc,lichen2,propflow}. Much of
%the work on this problem is based on defining proximity-based
%measures on the nodes in the underlying network
%\cite{adamic,kleinberg}. The work in \cite{kleinberg} studied the
%usefulness of different topological features for link prediction. A
%second approach is to study the problem in the context of
%statistical relational models \cite{bilgic,Getoor01,Getoor02,yu}.
%However, these methods are restricted to relational models, and  are
%not designed for dynamic networks, or cannot handle attributes of a
%relational nature. Recently, the problem of link prediction has also
%been studied in the context of wikipedia and web data
%\cite{adafre,zhu}. Methods for using supervised random walks for
%performing link prediction are proposed in \cite{back}. The negative
%link prediction has been investigated in \cite{tang2015} with only
%positive links and content-centric interactions in social media. A
%stochastic topic model WTFW (Who to Follow and Why) has been proposed for
%predicting links together with their types of explanation over directed
%and nodes-attributed graphs in \cite{barbieri2014}.
%
%
% The link
%prediction problem has also been studied more generally in the
%context of the classification problem \cite{propflow}, since the
%link prediction problem can be considered as a classification
%problem in which features and class labels (corresponding to
%existence or absence  of links) can be associated with links to be
%predicted. While some work has focussed recently on some aspects of
%the heterogeneous scenario \cite{sun11,sun12,yang}. Some of the
%methods have also been proposed for the  temporal scenario
%\cite{back,dwang}.  Recently, the use of supervised
%methods~\cite{propflow} and  latent factor models~\cite{menon} for
%link prediction has gained increasing importance. However, these
%methods are not essentially designed for searching the space of
%$O(n^2)$ possibilities. Most methods in the literature are not
%naturally designed to prune the space of possibilities for the link
%prediction problem and are generally designed to {\em evaluate} the
%link prediction propensities of a subset of node pairs.
%

