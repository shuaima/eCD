\section{Structural Bagging Methods}
\label{sec-bagging}
Since the link prediction problem scales worse than
linearly with the network size, it is generally more efficient to
solve smaller problems multiple times rather than solve a single
large problem.
The structural bagging approach provides an effective method to decompose
the link prediction problem into smaller pieces that are solved
independently.
Furthermore, the aggregated results from multiple models often
provide a robustness to the decomposition process  \cite{Breiman96b-1996}. In the
following, we introduce three different ways for the bagging decomposition.
We consider a network $G(V, A)$.


\subsection{Random Node Bagging}

Random node bagging is the simplest form of structural bagging, and its basic
idea is to iteratively apply the following three steps:


\begin{enumerate}
%\item  Select a random set of nodes $N_r$ from the network $G$ corresponding to
%a fraction $f$ of the nodes in the network. Determine the node set
%$N_s \supseteq N_r$, corresponding to all nodes adjacent to $N_r$.
%Construct a reduced adjacency matrix $W_s$ from the node set $N_s$
%by using the subgraph induced on $N_s$ to select the relevant rows
%and columns of $W$.

%\vspace{-0.5ex}
\item[(1)] Select a random set of nodes $N_r$ in the netwrok $G$ corresponding to
a fraction $f$ of the nodes in the network. Determine the node set
$N_s \supseteq N_r$, corresponding to all nodes adjacent to $N_r$.


\item[(2)]  Construct a reduced adjacency matrix $W_s$ from the node set $N_s$, by using the subgraph
induced on $N_s$ of $G$ (referred to as an ensemble component or simply an ensemble) to select the relevant $|N_s|$ rows and columns of the matrix $W$.

\item[(3)]  Apply the symmetric \NMF method in Section \ref{sec-NMF} to the reduced matrix $W_s$, and use the  pruning search
process in Section \ref{sec-NMF-topk-optimization} to determine the predictions of all pairs of nodes of $N_s$.
\end{enumerate}



The main efficiency advantage of this approach is because of the
smaller sizes of the matrices in the factorization. Furthermore,
because of the smaller size of the induced subgraph in each ensemble
component, the number of latent factors $r$, which is required, is
also smaller. This will generally translate into efficiency
advantages. In many cases, when the number of nodes is very large,
it may be impractical to solve the entire problem in main memory. In
such cases, the use of ensemble approach decomposes the problem into
smaller memory-resident components.

The main problem  with random node sampling is that it does not
attempt to sample more {\em relevant} regions of the network which are
more likely to contain possible links. Other forms of sampling are likely to
be more effective in this context.

\subsection{Edge Bagging}


Edge bagging is designed to sample more {\em relevant} regions of
the graph. After all, real-world networks are sparse and most of the
$O(n^2)$  possibilities for edges are usually not populated. By
sampling densely populated regions of the network, many node pairs
will not be considered at all, but these node pairs are often not
relevant to begin with.

The edge bagging approach proceeds as follows:

%\vspace{-1ex}
\begin{enumerate}
%\item  Let $N_s$ be a node set containing a single randomly chosen
%node. Nodes which are adjacent to $N_s$ are randomly selected and
%added to $N_s$. In the event that no node is adjacent to $N_s$, a
%randomly chosen node from a different connected component is added
%to $N_s$. The procedure is repeated until $N_s$ contains at least a
%fraction $f$ of the total number of nodes. Construct a reduced
%adjacency matrix $W_s$ from the node set $N_s$ by using the subgraph
%induced on $N_s$ to select the relevant rows and columns of $W$.
\item[(1)]Let $N_s$ be a node set containing a single randomly chosen
node. Nodes which are adjacent to $N_s$ are randomly selected and
added to $N_s$. In the event that no node is adjacent to $N_s$, a
randomly chosen node from a different connected component is added
to $N_s$. The procedure is repeated until $N_s$ contains at least a
fraction $f$ of the total number of nodes.

\item[(2)]  Construct a reduced
adjacency matrix $W_s$ from the node set $N_s$ by using the subgraph
induced on $N_s$ of $G$, i.e., an ensemble, to select the relevant $|N_s|$ rows and columns of the matrix $W$.

\item[(3)] Apply the symmetric \NMF method in Section \ref{sec-NMF}
to the reduced matrix $W_s$, and use the pruning search process in Section \ref{sec-NMF-topk-optimization} to determine the predictions of all pairs of nodes  of $N_s$.
%, except those that are less than $\epsilon$.
\end{enumerate}
%\vspace{-1ex}


This method of growing the sampled node set with edge sampling is
likely to select dense components from the network. Such dense
components are more likely to contain random node pairs. Unlike the
previous case where each node pair is considered with a high
probability, many node pairs will not be considered at all. However,
such node pairs are typically not present in the same dense
component. Therefore, such node pairs are likely to be irrelevant,
and in this way the approach already prunes unimportant node pairs
during the process of ensemble construction.



\subsection{Biased Edge Bagging}

While the edge bagging procedure is effective at discovering dense
components, it does have a drawback. Its main drawback is that  it
selectively includes nodes with  high degrees within the resulting
components.  Therefore, the same high-degree nodes are very likely to be included in
all the ensemble components.  As a result, it often becomes more
difficult to make robust predictions between low-degree nodes.

In biased edge bagging, exactly the same procedure is used as the case
of edge bagging.  The only difference is that when the node set
$N_s$ is grown, a random adjacent node is not selected. Rather, an
adjacent node with the least number of selected times
in previous ensemble components, is used. Ties are broken randomly.

This approach ensures  that each node is selected with an approximately
similar number of times across various ensemble components, and it
prevents the repeated selection of high-degree nodes.
%
Note that the
bias in the edge bagging process makes that the vast majority node
pairs will not be considered.
However, such node pairs will usually be in components that are not
as well connected. Therefore, such nodes  are far less likely to
form links. In most practical applications, one only needs to
recommend a small number of node pairs for prediction.  Therefore,
it is reasonable to ignore such node pairs in the prediction
process.



\subsection{Incorporating Link Prediction Characteristics}
\label{subsec-lp-char}

\eat{
Different from existing sampling methods \cite{ahmed2014tkdd}, our bagging methods should be designed in particular for link prediction.
Motivated by the observation that most of all new links in social networks span within very short distances, typically closing triangles,
which has been justified in \cite{leskovec-2008}, we develop three structural bagging approaches such that {\em a node is always sampled together
with all its neighbors, which guarantees the possibility of forming triangles}. To achieve this, we revise the previous bagging methods as follows.

\vspace{-1ex}
\begin{enumerate}
%\vspace{-0.5ex}
\item[(1)]
For random node bagging, when a node is selected uniformly at random from the network $G$, the node together with all of
its neighbors are put into the node set $N_s$.
%\vspace{-0.5ex}
\item[(2)]
For edge and biased edge bagging, when a node adjacent to $N_s$ is
 selected and added to $N_s$, all of its neighbors are put into the node set $N_s$ together.
\end{enumerate}
\vspace{-1ex}
}
%%%%%%%%%%%%%%%%%%%%%%%%%%%%%%%% 2016-06-21 duan  begin


Different from existing graph sampling methods \cite{ahmed2014tkdd, leskovec2006},
we employ the characteristics of link prediction, as our bagging methods are designed in particular for link prediction.
We develop a novel graph sampling method accompanied with two strategies:
(1) node uptake, to choose those nodes with a high possibility of forming links  and (2) edge filter, to eliminate edges without affecting the prediction accuracy.


\stitle{Node uptake}. Motivated by the observation that most of all new links in social networks span within very short distances, typically closing triangles, which has been justified in \cite{leskovec-2008}.  Inspired by this, we develop a node uptake strategy such that {\em a node is always sampled together
with all its neighbors, which guarantees the possibility of forming triangles}. To achieve this, we revise the previous three bagging methods as follows.

%\vspace{-1ex}
\begin{enumerate}
%\vspace{-0.5ex}
\item[(1)]
For random node bagging, when a node is selected uniformly at random from the network $G$, the node together with all of
its neighbors are put into the node set $N_s$.
%\vspace{-0.5ex}
\item[(2)]
For edge and biased edge bagging, when a node adjacent to $N_s$ is
 selected and added to $N_s$, all of its neighbors are put into the node set $N_s$ together.
\end{enumerate}
%\vspace{-1ex}


\stitle{Edge filter}. Inspired by graph sparsification \cite{chen2015, satuluri2011}
which has been successfully applied to clustering without sacrificing the quality,
we develop an edge filter strategy to choose a portion of edges only, rather than all the edges of the subgraph induced on $N_s$ of $G$,
which significantly improves the efficiency of the bagging methods.
The challenge here is how to remove edges as many as possible while maintaining the high prediction accuracy.
To do this, we make use of the following link prediction characteristic.


Preferential attachment (PA) \cite{albert1999,leskovec-2008} is one of the well-known
link prediction characteristics, which says that {\em the probability
that a new link is connected to a node $i$ is proportional to its degree}.
In other words, nodes with small degree tend to have fewer links in the future.
Therefore, we remove those edges that has at least one endpoint  with a smaller degree.
Let $d_i$ and $d_j$ be the degrees of the endpoints $i$  and $j$ of edge $(i, j)$, respectively. Given a sampling ratio $\rho$,
we sort edges in the induced subgraph by $\min(d_i, d_j)$ in a
descending order and only select the top $m\rho$ edges, where $m$ is the total number of edges.
%
To achieve this, we further revise the previous three bagging methods as follows.

\eat{Common neighbor (CN) \cite{kleinberg,linyuan-2011} is another significant characteristic for link prediction,
which says that {\em two nodes $i$ and $j$ are more likely to have a link in the future if they
have many common neighbors}. Ignoring the edge $(i, j)$ may cause as many as $d_i + d_j - 2$ nodes to lose
their common neighbors. We adopt the method for filtering out edges in order to minimize the
loss of common neighbors  \cite{chen2015}, which sorts edges by $d_i + d_j$ in a descending order, and keeps the top $ms$ edges, where $m$ is the total number of edges.}



%\vspace{-1ex}
\begin{enumerate}
%\vspace{-0.5ex}
\item[(1)]
At the second step of the three bagging methods, construct a reduced adjacency matrix $W_s$
from the reduced subgraph obtained by applying the edge filter
on the subgraph induced on the node set $N_s$ of the network $G$.

%\item[(1)]
%For random node bagging, when a node is selected uniformly at random from the network $G$, the node together with all of
%its neighbors are put into the node set $N_s$.
%%\vspace{-0.5ex}
%\item[(2)]
%For edge and biased edge bagging, when a node adjacent to $N_s$ is
% selected and added to $N_s$, all of its neighbors are put into the node set $N_s$ together.
\end{enumerate}
\vspace{-1ex}

%%%%%%%%%%%%%%%%%%%%%%%%%%%%%%%% 2016-06-21 duan  end

\subsection{Bound of Node Bagging Ensembles}

Observe that even each ensemble component is much smaller,
multiple samples are required. To meaningfully rank the various node
pairs, each node pair needs to be included in the ensemble components with performance guarantees.
What is the required number of samples to
ensure that each node pair is included  at least $\mu$ times? Clearly,
this number depends on the sampling fraction $f$. We next present a
probabilistic bound on the expected number of times that a node pair is included as follows.

%\begin{lemma}
%Each node pair is included at least $\alpha$ times in an ensemble
%component with probability at least $1 - e^{-\alpha}$ in $O( m/f^2)$
%ensemble components.
%\end{lemma}
%\begin{proof}
%Since each ensemble component includes a node with probability at
%least $f$, it follows that each node pair is included with
%probability $f^2$. Therefore, in $1/f^2$, the probability of not
%including a node pair is given by $( 1- f^2)^{(1/f^2)} \leq 1/e$.
%Therefore, the expected of  times a  particular  node pair not
%included  in $9 m$ samples is $9 \alpha /e \leq 4 \alpha$. However,
%it would need to be not included more than $8 \alpha$ times for the
%node pair to be included less than $\alpha$ times. The upper-tail
%Chernoff bound can be used to show that the probability of this is
%less than $e^{-\alpha}$.
%\end{proof}


\begin{prop}
The expected times of each node pair included in $\mu / f^{2}$ ensemble
components is at least $\mu$.
\end{prop}

\begin{IEEEproof}
Since each ensemble component includes a node with probability at
least $f$, it follows that each node pair is included with
probability $f^2$. Furthermore, all ensemble component are independent
of each other. Let $X$ be the times of each node pair is included in
all ensemble components, and the expected value $E(X)$ of $X$ is equal
to $b \times f^2$, where $b$ is the number of ensemble components. For
$E(X) \geq \mu$, we have $b \geq \mu / f^2 $.
\end{IEEEproof}

\eat{
\marked{}
\begin{lemma}
Each node pair is include at least $\mu$ times in an ensemble
component with probability at least $1 - \alpha$ in
$( 2\mu + z_{\alpha} ) / f^{2}$ ensemble components, where $z_{\alpha}$
is the cutoff point of a standard normal distribution.
\end{lemma}
\begin{proof}
Since each ensemble component includes a node with probability at
least $f$, it follows that each node pair is included with
probability $f^2$. Furthermore, all ensemble component are independent
of each other. Let $X$ be the times of each node pair is included in
an ensemble component. From de Moivre-Laplace theorem, we have that
$( X - bf^2 ) / \sqrt{bf^2 ( 1- f^2) }$ is approximately a standard
normal distribution, if the number of ensemble components $b$ is very
large. For $X \geq \mu$, we have $ b \geq ( 2\mu + z_{\alpha} ) / f^{2}$.
\end{proof} }

Note that the above bound holds only for the original random node bagging method, and it provides a theoretical guarantee.
Indeed, we could do much better in practice. For instance, the setting of $\mu = 0.1$ and $f = 0.1$ already achieves a pretty good result, as shown by our experimental study in Section~\ref{sec-exp}.

\subsection{Ensemble Enabled Top-$K$ Predictions}


\eat{
%%%%%%%%%%%%%%%%%%%%%Baseline Algorithm
\begin{figure}[tb!]
%\vspace{-2ex}
\begin{center}
{\small
\begin{minipage}{3.36in}
\myhrule \vspace{-2ex}
\mat{0ex}{
\sttab {\sl Input:\/} \= Network $G(V, A)$, $\mu$ and $f$.\\
{\sl Output:\/} Top-$k$ node pairs, not included in $A$. \\
 \hspace{0ex}\= $\Gamma$ := $\emptyset$;\\
\> {\bf Repeat} $\mu/f^2$  times {\bf do} \\
\>\hspace{2ex}\= $N_s$ is a sampled ensemble of $G$ with at least $f\cdot|V|$ nodes; \\
\>\>$W_s \approx F_s\cdot F_s^T$ using \NMF; \\
\>\> $\Gamma'$ :=  top-$k$ largest value node pairs $(i,j)$ in $\overline{F_{s,i}} \cdot \overline{F_{s,j}}$;\\
\>\> $\Gamma$ \ :=  top-$k$ largest value node pairs $(i,j)$ in $\Gamma'\cup\Gamma$;\\
\> {\bf return} $\Gamma$.
}
\vspace{-2ex} \myhrule
\end{minipage}
}
\end{center}
\vspace{-3ex}
\caption{Top-$K$ Predictions using Ensembles} \label{alg-topk}
\end{figure}
%%%%%%%%%%%%%%%%%%%%%%%%%%%%%%%%%%%%%
}


We now explain the complete framework for top-$k$ predictions enabled with ensembles, shown as follows.

%\vspace{-1ex}
\begin{tabbing}1.\hspace{1ex}\=
 {\bf given} network $G(N, A)$ and parameters $\mu$ and $f$.\\
2.\> {\bf let} $\Gamma$ be empty;\\
3.\> {\bf repeat} $\mu/f^2$  times {\bf do} \\
4.\>\hspace{2ex}\= {\bf let} $N_s$ be a sampled ensemble of $G$ with at least $f\cdot n$ nodes; \\
5.\>\>Compute $W_s \approx F_s\cdot F_s^T$ using \NMF; \\
6.\>\> {\bf let} $\Gamma'$ be top-$k$ largest value node pairs $(i,j)$ in $\{\overline{F_{s,i}} \cdot \overline{F_{s,j}}\}$;\\
7.\>\> {\bf let} $\Gamma$ be top-$k$ largest value node pairs $(i,j)$ in $\Gamma'\cup\Gamma$;\\
8.\> {\bf return} the top-$k$ node pairs $\Gamma$ not included in $A$.
\end{tabbing}
%\vspace{-1ex}


To ensure that a node pair appears in the ensemble components at least $\mu$ expected times,  $\mu / f^{2}$ ensemble components are considered in total.
For each time, an ensemble component $N_s$ is sampled by one of the above
node, edge and biased edge bagging methods, the symmetric \NMF method in Section \ref{sec-NMF} is used on the matrix $W_s$ obtained
by filtering edges on the reduced matrix, and the aforementioned
pruning search process in Section \ref{sec-NMF-topk-optimization} is used to determine the predictions of
all pairs of nodes. If a node pair appears in multiple ensemble components and has multiple
prediction values, the maximum prediction value is considered, which means that we can use the
minimum value in the previous predicted links to speed up the top-$k$ predictions.
And, hence, only the top-$k$ predicted links are maintained for each ensemble component.
At the end of the process, the top-$k$ predictions in all $\mu / f^{2}$ ensemble components are returned.


\eat{Note that the top-$k$ nodes are
not computed at this stage, but the  predicted values are simply
added to any node-pair predictions from previous ensemble
components. The number of times a node-pair has been encountered in
one or more ensembles is also separately maintained.

As in the case of node bagging, the predictions for any pair of
nodes encountered in the search are averaged over the total number
of times they were encountered. The top-$k$ predictions are then
returned.


%, except those that are less than $\epsilon$.
Note that the top-$k$ nodes are
not computed at this stage, but the  predicted values are simply
added to any node-pair predictions from previous ensemble
components. In addition, the number of times a node-pair has been encountered in
one or more ensembles is also separately maintained.}


