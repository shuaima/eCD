%\vspace{-0.5ex}
%\section{Incorporating Built-in Predicates into CFDs}


\section{Extending CFDs with Predicates}
\label{sec-cfd}
%\vspace{-1ex}
%%%%%%%%%%%%%%%%%%%%%%%%%%%%%%%%%



We now define {\em conditional functional dependencies with predicates}, denoted by \pCFDs,
by extending \CFDs~\cite{CFDs} with built-in
predicates ($\ne$, $<$, $\le$, $>$, $\ge$) in addition to equality ($=$).

Consider a relational schema $R$ defined over a finite set of
attributes, denoted by $\attrset(R)$. For each attribute $A \in
\attrset(R)$, its domain is specified in $R$, denoted as $\dom(A)$,
which is either finite (\eg~\at{bool})
or infinite (\eg~\at{string}).
%We use $\fattrset(R)$ to denote
%the set of attributes of $R$ that have a finite domain.
We assume \kwlog that a domain on which $<$, $\le$, $>$ or $\ge$ is defined is totally ordered.


%%%%
\stitle{Syntax}. A \pCFD $\varphi$ on $R$ is a pair
$R(X\ra Y,\ T_p)$, where (1)~$X, Y$ are sets of attributes in
$\attrset(R)$; (2)~$X \ra Y$ is a standard \FD, referred to as the
\FD~{\em embedded in} $\varphi$; and (3)~$T_p$ is a tableau with
attributes in $X$ and $Y$, referred to as the {\em pattern tableau}
of $\varphi$, where for each $A$ in $X \cup Y$ and each tuple $t_p
\in T_p$, $t_p[A]$ is either an unnamed
variable `\_' that draws values from $\dom(A)$, or
`$\op a$', where \op is one of $\{=$, $\ne$,
$<$, $\le$, $>$, $\ge\}$,  and `$a$' is a constant in $\dom(A)$.

If attribute $A$
occurs in both $X$ and $Y$, we use $A_L$ and $A_R$ to indicate the
occurrence of $A$ in $X$ and $Y$, respectively, and we separate the $X$
and $Y$ attributes in a pattern tuple with `$\pa$'.
We simply write $\varphi$ as $(X \ra Y,\ T_p)$ when $R$ is clear from the
context, and denote $X$ as $\mbox{\LHS}(\varphi)$ and $Y$ as
$\mbox{\RHS}(\varphi)$, respectively.



%\vspace{-1ex}
\begin{example}
The dependencies \kw{cfd_1}--\kw{cfd_3} and \kw{pfd_1}--\kw{pfd_4}
that we have seen in Example~\ref{exa-motivation} can all be expressed as \pCFDs.  Some of these \pCFDs
are illustrated in Fig.~\ref{fig-pcfd}, in which $\varphi_1$ is for \FD\ \kw{cfd_2},
$\varphi_2$ is for \CFD\ \kw{cfd_3}, $\varphi_3$ is for \kw{pfd_2},
and $\varphi_4$ is for \kw{pfd_4}, respectively.
\end{example}
\vspace{-1ex}


\stitle{Semantics}. Consider \pCFD $\varphi$ =
$R(X\ra Y$, $T_p)$, where $T_p = \{t_{p_1}, \ldots, t_{p_k}\}$.

A data tuple $t$ of $R$ is said to {\em match} $\LHS(\varphi)$,
denoted by $t[X] \asymp T_p[X]$, if {\em for each tuple
$t_{p_i}$ ($i\in[1, k]$) in $T_p$} and {\em each attribute $A$} in $X$, either
(a)  $t_{p_i}[A]$ is the wildcard `\_' (which matches any value
in $\dom(A)$),  or
(b) $t[A]\ \op a$ if $t_{p_i}[A]$ is `$\op a$', where
the operator $\op$ ($=$, $\ne$,
$<$, $\le$, $>$ or $\ge$) is interpreted by its standard
semantics. Similarly, the notion that $t$ {\em matches} $\RHS(\varphi)$
is defined, denoted by $t[Y] \asymp T_p[Y]$.


Intuitively, each pattern tuple $t_{p_i}$ ($i\in[1, k]$) specifies a condition via
$t_{p_i}[X]$, and $t[X] \asymp T_p[X]$ if
 $t[X]$ satisfies the {\em conjunction} of all these conditions.
Similarly, $t[Y] \asymp T_p[Y]$ if $t[Y]$ matches all the patterns
specified by $t_{p_i}[Y]$ for all pattern tuples $t_{p_i}$ in $T_p$.

An instance $I$ of $R$ {\em satisfies} the \pCFD $\varphi$, denoted
by $I \models \varphi$, if  for {\em each pair} of tuples $t_1, t_2$
in $I$, if $t_1[X] = t_2[X] \asymp T_p[X]$, then
$t_1[Y] = t_2[Y] \asymp T_p[Y]$.  That is, if $t_1[X]$ and $t_2[X]$
are equal and in addition, they both match the pattern tableau
$T_p[X]$, then $t_1[Y]$ and $t_2[Y]$ must also be equal to each
other and they both match the pattern tableau $T_p[Y]$.

Observe that $\varphi$ is imposed only on the subset of tuples in
$I$ that match $\LHS(\varphi)$, rather than on the entire $I$. For
all tuples $t_1, t_2$ in this subset, if $t_1[X] = t_2[X]$, then
(a)~$t_1[Y] = t_2[Y]$, \ie~the semantics of the embedded \FDs is
enforced; and (b)~$t_1[Y] \asymp T_p[Y]$, which assures that the
{\em constants} in $t_1[Y]$ match the {\em constants} in $t_{p_i}[Y]$
for all $t_{p_i}$ in $T_p$. Note that here tuples $t_1$ and $t_2$ can
be the same.



An instance $I$ of $R$ satisfies a set $\Sigma$ of \pCFDs,
denoted by $I\models\Sigma$, if $I\models\varphi$ for {\em each}
\pCFD $\varphi$ in $\Sigma$.


%%%%%%%%%%%%%%%%%%%%%pCIND exmaple%%%%%%%%%%%%
\begin{figure*}[tbh!]
\vspace{1ex}
\begin{center}
(1) $\psi_1$ = (\at{item}\lbrack\at{state}; $\at{type}\rbrack$
$\subseteq$ \at{tax}\lbrack\at{state}; $\NIL\rbrack$, \ $T_{1}$), \\
\sstab (2) $\psi_2$ = (\at{item}\lbrack\at{state}; $\at{type,
state}\rbrack$ $\subseteq$ \at{tax}\lbrack\at{state};
$\at{rate}\rbrack$, \ $T_{2}$)\\ \sstab $T_1$: \begin{tabular}{c||c}
\at{type} &  \kw{ nil} \\ \hline
 $\ne$ art  &  \\
\end{tabular}
\hspace{10ex}
$T_2$:
\begin{tabular}{c | c || c}
\at{type} & \at{state} & \at{rate} \\ \hline
 $\ne$ art  & $=$ DL & $=$ 0 \\
\end{tabular}
\end{center}
\vspace{-2ex} \caption{Example \pCINDs} \label{fig-pcind}
\vspace{-3ex}
\end{figure*}
%%%%%%%%%%%%%%%%%%%%%%%%%%%%%


\vspace{-1ex}
\begin{example}
The instance $D_0$ of
Fig.~\ref{fig-instance} satisfies $\varphi_1$ and $\varphi_2$ of
Fig.~\ref{fig-pcfd}, but neither $\varphi_3$ nor
$\varphi_4$. Indeed, tuple $t_3$ violates (\ie~does not satisfy)
$\varphi_3$, since
$t_3[\at{sale}]$ = `F' {\em and} $20 < t_3[\at{price}] \le 40$,
but $t_3[\at{shipping}]$ is 20 instead of $6$.
Note that $t_3$ matches $\LHS(\varphi_3)$ since it satisfies
the condition specified by the {\em conjunction}
of the pattern tuples in $T_3$.
Similarly, $t_1$ violates $\varphi_4$, since
$t_1[\at{sale}]$ = `T' but $t_1[\at{price}] > 9.99$.
Observe that while it takes two tuples
to violate a standard \FD, a single tuple may violate a
\pCFD.
\end{example}
\vspace{-1.5ex}






\stitle{Special cases.}~(1) A standard \FD $X\ra Y$~\cite{AbHuVi1995} can
be expressed as a \CFD $(X\ra Y,\ T_p)$ in which $T_p$ contains a
single tuple consisting of `\_' only, without constants. (2) A
\CFD $(X\ra Y,\ T_p)$~\cite{CFDs} with
$T_p = \{t_{p1},\ldots, t_{pk}\}$ can be
expressed as a set $\{\varphi_1,\ldots,\varphi_k\}$ of
\pCFDs such that for each $i\in [1, k]$, $\varphi_i$ = $(X\ra Y,\
T_{p_i})$, where $T_{p_i}$ contains the pattern tuple $t_{p_i}$ of
$T_p$ only, defined with equality $(=)$ only.
For example, $\varphi_1$ and $\varphi_2$ in
Fig.~\ref{fig-pcfd} are \pCFDs representing \FD $\kw{cfd_2}$
and \CFD $\kw{cfd_3}$ in Example~\ref{exa-motivation}, respectively.
Note that all data quality
rules in~\cite{CM08,divesh08} can be expressed as
\pCFDs.
