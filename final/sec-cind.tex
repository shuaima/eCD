

\section{Extending CINDs with Predicates}
\label{sec-cind}
%\vspace{-1ex}



Similar to \pCFDs, we define {\em conditional inclusion dependencies with predicates}, denoted by \pCINDs,
by extending \CINDs~\cite{tcs-CINDs} with built-in predicates ($\ne$, $<$, $\le$, $>$, $\ge$) in addition to equality ($=$).
%
Consider two relational schemas $R_1$ and $R_2$.

%\vspace{-0.5ex}
\stitle{Syntax.} A \pCIND $\psi$ is a pair
$(R_1[X;\ X_p] \subseteq R_2[Y;\ Y_p],\ T_p)$, where (1)~$X, X_p$
and  $Y, Y_p$ are lists of attributes in $\attrset(R_1)$ and
$\attrset(R_2)$, respectively; (2)~$R_1[X] \subseteq R_2[Y]$ is a
standard \IND, referred to as the \IND\ {\em embedded} in $\psi$;
and (3)~$T_p$ is a tableau, called the {\em pattern tableau} of
$\psi$ defined over attributes $X_p\cup Y_p$, and for each $A$ in
$X_p$ or $Y_p$ and each pattern tuple $t_p \in T_p$, $t_p[A]$ is either
an unnamed variable `\_' that draws values from $\dom(A)$, or
`\op a', where \op is one of $=, \ne, <, \le, >, \ge$ and `a' is a
constant in $\dom(A)$.

 We denote $X \cup X_p$ as $\LHS(\psi)$, $Y \cup Y_p$ as $\RHS(\psi)$, and
separate the $X_p$ and $Y_p$ attributes in a pattern tuple with
`$\pa$'. We also use \kw{nil} to denote an {\em empty} list. \eat{Relation
$R_1$ (resp. $R_2$) is called the \LHS (resp. \RHS) relation of
\pCIND $\psi$.}


%In addition, we also assume a finite domain contains at least two elements.

\vspace{-0.5ex}
\begin{example}
\label{exam-pcind} Figure~\ref{fig-pcind} shows two example \pCINDs:
$\psi_1$ expresses the $\kw{pind}_1$ in Example~\ref{exa-motivation},
and $\psi_2$ refines $\psi_1$ by stating that for any \at{item}
tuple $t_1$, if its \at{type} is not art and its \at{state} is DL,
then there must be a \at{tax} tuple $t_2$ such that its \at{state}
is DL and \at{rate} is $0$, \ie~$\psi_2$ assures that the sale tax
rate in Delaware is 0.
\end{example}
\vspace{-1ex}






\stitle{Semantics.}~Consider \pCIND $\psi$ =
$(R_1[X;\ X_p] \subseteq R_2[Y;\ Y_p],$ $\ T_p)$.
An instance  $(I_1, I_2)$ of $(R_1, R_2)$ {\em satisfies} the \pCIND
$\psi$, denoted by $(I_1, I_2) \models \psi$, iff for {\em each}
tuple $t_1\in I_1$, if $t_1[X_p] \asymp T_p[X_p]$, then there {\em
exists} a tuple $t_2\in I_2$ such that $t_1[X]$ = $t_2[Y]$ and $t_2[Y_p] \asymp T_p[Y_p]$.


That is, if $t_1[X_p]$ matches
the pattern tableau $T_p[X_p]$, then $\psi$ assures the
existence of $t_2$ such that (1)~$t_1[X] = t_2[Y]$ as
needed by the standard \IND embedded in $\psi$;
and, moreover, (2)~$t_2[Y_p]$
must match the pattern tableau $T_p[Y_p]$. In other words,
$\psi$ is ``conditional'' since its embedded \IND is applied only to the
subset of tuples in $I_1$ that match $T_p[X_p]$, and
$T_p[Y_p]$ is enforced on the tuples in $I_2$
that match those tuples in $I_1$. As remarked in Section~\ref{sec-cfd},
the pattern tableau $T_p$ specifies the {\em conjunction} of
all the pattern tuples in $T_p$.

\vspace{-0.5ex}
\begin{example}
The instance $D_0$ of \at{item} and \at{tax} in
Fig.~\ref{fig-instance} violates \pCIND $\psi_1$. Indeed, tuple
$t_1$ in \at{item} {\em matches} $\LHS(\psi_1)$ since
$t_1[\at{type}] \ne $ `art', but there is no tuple $t$ in \at{tax}
such that $t[\at{state}]$ = $t_1[\at{state}]$ = `WA'. In contrast,
$D_0$ satisfies $\psi_2$.
\end{example}
%\vspace{-0.5ex}


We say that a database
$D$ satisfies a set $\Sigma$ of \pCINDs, denoted by $D \models
\Sigma$, if $D \models \psi$ for each $\psi \in \Sigma$.


\stitle{Safe CIND$^p$s}.
%Consider \pCIND $\psi$ = $(R_1[X;\ X_p] \subseteq R_2[Y;\ Y_p],\ T_p)$.
We say a \pCIND $(R_1[X;\ X_p] \subseteq R_2[Y;$$\ Y_p],\ T_p)$ is
{\em unsafe} if  there exist pattern tuples $t_p, t_p'$ in $T_p$
such that
either (a) there exists $B\in Y_p$, such that $t_p[B]$ and $t_p'[B]$
are not satisfiable when taken together,
or (b) there exist $C \in Y$, $A \in X$ such that $A$ corresponds to $C$ in the
embedded \IND and $t_p[C]$ and $t_p'[A]$ are not satisfiable
when taken together;
\eg~$t_p[\at{price}] = 9.99$ and $t_p'[\at{price}] \ge 19.99$.

Obviously unsafe \pCINDs do not make sense: no nonempty
databases satisfy unsafe \pCINDs. It takes $O(|T_p|^2)$ time
in the size $|T_p|$ of $T_p$ to decide whether a \pCIND is unsafe.
Thus in the sequel we consider safe \pCIND only.


\eat{ We say a \pCIND is {\em unsafe} if there exist an attribute
$A$ and pattern tuples $t_p, t_p'$ in $T_p$ such that $t_p[A]$ and
$t_p'[A]$ are not satisfiable when taken together,
\eg~$t_p[\at{price}] = 9.99$ and $t_p'[\at{price}] \ge 19.99$.
Obviously unsafe \pCINDs do not make sense: there exist no nonempty
database that satisfies unsafe \pCINDs.

Note that patterns on attributes $X\cap X_p$ should be propagated to
their corresponding attributes in $Y$.}




\stitle{Special cases}. (1) A standard
\IND $(R_1[X] \subseteq R_2[Y])$ can be expressed as a \pCIND
$(R_1[X;\ \NIL] \subseteq R_2[Y;\ \NIL],\ T_p)$ such that $T_p$ is
simply a {\em empty} set. (2) A \CIND $(R_1[X;\ X_p]
\subseteq R_2[Y;\ Y_p],\ T_p)$ with $T_p = \{t_{p1},\ldots,
t_{pk}\}$ can be expressed as a set
$\{\psi_1,\ldots,\psi_k\}$ of \pCINDs, where for each $i\in [1, k]$,
$\psi_i$ = $(R_1[X;\ X_p] \subseteq R_2[Y;\ Y_p],\ T_{pi})$ such
that $T_{pi}$ consists of the pattern tuple $t_{pi}$ of $T_p$,
defined with equality ($=$) only.
