


\section{Introduction}
\label{sec-intro}



Extensions of traditional functional dependencies (\FDs) and inclusion dependencies
(\INDs), known as {\em conditional functional dependencies} (\CFDs~\cite{CFDs})
and {\em conditional inclusion dependencies} (\CINDs~\cite{tcs-CINDs}),
respectively, have recently been proposed for improving data quality.
These extensions enforce patterns of semantically
related data values, and
detect errors as violations of the dependencies.
It has been shown that conditional dependencies are able to capture more inconsistencies
than \FDs and \INDs~\cite{repair,tcs-CINDs,book-Fan}.

Conditional dependencies specify constant patterns in terms of
equality $(=)$. In practice, however, the semantics of data often
need to be specified %in terms of other predicates such as
with other predicates such as
$\ne, <, \le, >$ and $\ge$, as illustrated by the following example.



%%%%%%%%%%%%%%%%%%%%%
\begin{figure*}[tb!]
\vspace{1ex}
\begin{center}
\begin{small}
\begin{minipage}[bt]{3.4in}
\subfigure[{\small An \at{item} relation}]{
    \label{fig-sales}
%%(a)pku
\begin{tabular}{c | c | c | c | c | c | c | c |}
\cline{2-8} \multicolumn{1}{c|}{}&\ $\attr{id}$ & \
$\attr{name}$ &\ $\attr{type}$ &\ $\attr{price}$\ &\ $\attr{shipping}$
&\ $\attr{sale}$ &\ $\attr{state}$\\
\cline{2-8}
$t_1$: & b1  & Harry Potter   & book & 25.99\ & 0 & T & WA \\
$t_2$: & c1  & Snow White   & CD & 9.99  & 2 & F & NY\\
$t_3$: & b2  & Catch-22   & book & 34.99 & 20 & F & DL\\
$t_4$: & a1  & Sunflowers  & art & 5m  & 500 & F & DL \\
\cline{2-8}
\end{tabular}
}
\end{minipage}
\quad\quad\quad\quad\quad\quad
\begin{minipage}[bt]{1.2in}
\subfigure[{\small A \at{tax} rate relation}]{
\label{fig-rd}
\begin{tabular}{ c | c | c |}
\cline{2-3} \multicolumn{1}{c|}{}&\ $\attr{state}$ &\
$\attr{rate}$ \\
\cline{2-3}
$t_5$: & PA & 6\\
$t_6$: & NY & 4\\
$t_7$: & DL & 0\\
$t_8$: & NJ & 3.5\\
\cline{2-3}
\end{tabular}
}
\end{minipage}
\end{small}
\vspace{-2ex}
\end{center}
%\vspace{-1ex}
\caption{Example instance $D_0$ of \at{item} and \at{tax}}
\label{fig-instance}
\vspace{-1ex}
\end{figure*}
%%%%%%%%%%%%%%%%%%%%%%%

%\vspace{-1ex}
\begin{example}
\label{exa-motivation}
An online store maintains a database of
two relations: (a) \at{item} for items sold
by the store, and (b) \at{tax} for
the sale tax rates for the items, except artwork,
in various states.
The relations are specified by the following schemas:

\vspace{-.5ex}
\mat{3ex}{
\at{item}(\=\at{id}:~string,
\at{name}:~string, \at{type}:~string, \at{price}:~float,  \\
 \> \at{shipping}:~float, \at{sale}:~bool, \at{state}:~string)\\
\at{tax}(\at{state}:~string, \at{rate}:~float)
} \vspace{-.5ex}




\noindent where
each item is specified by its \at{id}, \at{name}, \at{type}
(\eg~book, {\sc cd}), \at{price}, \at{shipping} fee,
the \at{state} to which it is shipped, and whether it
is on \at{sale}. A \at{tax} tuple specifies the sale
tax rate in a state. An instance $D_0$ of \at{item} and \at{tax} is shown
in Fig.~\ref{fig-instance}.

One wants to specify dependencies on the relations as data quality rules
to detect errors in the data, such that inconsistencies emerge as violations of
the dependencies. Traditional dependencies (\FDs, \INDs; see, \eg~\cite{AbHuVi1995})
and conditional dependencies (\CFDs, \CINDs~\cite{CFDs,tcs-CINDs})
on the data include the following:

\vspace{-0.5ex} \mat{3ex}{ $\kw{cfd}_1$: \ \=
\at{item}(\at{id} $\ra$ \at{name}, \at{type}, \at{price}, \at{shipping},
\at{sale})\\
$\kw{cfd}_2$:  \> \at{tax}(\at{state} $\ra$ \at{rate})\\
$\kw{cfd}_3$: \>
 \at{item}(\at{sale} = `T' $\ra$ \at{shipping} = 0)
} \vspace{-.5ex}

\noindent These are \CFDs: (a) $\kw{cfd}_1$ assures that the \at{id}
of an \at{item} uniquely determines its \at{name}, \at{type},
\at{price}, \at{shipping} and \at{sale}; (b) $\kw{cfd}_2$
states that \at{state} is a key for \at{tax}, \ie~for each state
there is a unique sale tax rate; and (c) $\kw{cfd}_3$ ensures
that for any \at{item} tuple $t$, if $t[\at{sale}]$ = `T' then
$t[\at{shipping}]$ must be $0$; \ie~free shipping is provided
for items on sale. Here $\kw{cfd}_3$ is specified in terms of
patterns of semantically related data values, namely, \at{sale} =
`T' and \at{shipping} = 0. It is to hold only on \at{item} tuples that
match the pattern \at{sale} = `T'. In contrast, $\kw{cfd}_1$
and $\kw{cfd}_2$ are traditional \FDs without constant patterns, a
special case of \CFDs. One  can verify that no sensible \INDs or
\CINDs can be defined across \at{item} and \at{tax}.

Note that $D_0$ of Fig.~\ref{fig-instance} satisfies $\kw{cfd}_1$,
$\kw{cfd}_2$ and $\kw{cfd}_3$. That is, when these dependencies are
used as data quality rules, no errors are found in $D_0$.


In practice, the shipment fee of an item is typically determined
by the price of the item. Moreover, when an item is on sale, the
price of the item is often in a certain range.
Furthermore, for any item sold by the store to
a customer in a state, if the item is {\em not} artwork, then one expects to
find the sale tax rate in the state from the \at{tax} table.
These semantic relations cannot be expressed
as \CFDs of \cite{CFDs} or \CINDs of \cite{tcs-CINDs},
but can be expressed as the following
dependencies:

%\begin{small}
\vspace{-.5ex} \mat{3ex}{
$\kw{pfd}_1$: \ \=
 \at{item}(\at{sale}$=$ `F' $\And$ \at{price}$\le$ 20 $\ra$ \at{shipping} = 3) \\
 $\kw{pfd}_2$:  \> \at{item}(\at{sale}$=$ `F' $\And$ \at{price}$>$ 20 $\And$
                \at{price}$\le$ 40 \\
 \>$\ra$ \at{shipping} = 6) \\
$\kw{pfd}_3$: \>  \at{item}(\at{sale}$=$ `F' $\And$ \at{price}$>$ 40 $\ra$ \at{shipping} = 10) \\
$\kw{pfd}_4$:\> \at{item}(\at{sale}$=$ `T' $\ra$ \at{price}$\ge$
2.99 $\And$
\at{price}$<$ 9.99) \\
$\kw{pind}_1$: \>  \at{item}(\at{state}; \at{type} $\ne$ `art')
$\subseteq$ \at{tax}(\at{state}; $\kw{nil}$) }
\vspace{-.5ex}
%\end{small}

\noindent Here $\kw{pfd}_2$ states that for any \at{item} tuple, if
it is not on sale and
its \at{price} is in the range $(20, 40]$, then its shipment fee
must be $6$; similarly for $\kw{pfd}_1$ and $\kw{pfd}_3$. These
dependencies extend \CFDs~\cite{CFDs} by specifying patterns of
semantically related data values in terms of predicates $<$,
$\le$, $>$ and $\ge$. Similarly, $\kw{pfd}_4$ assures that for any
\at{item} tuple, if it is on sale, then its price must be in the
range $[2.99, 9.99)$. Finally, $\kw{pind}_1$ extends
\CINDs~\cite{tcs-CINDs} by specifying patterns with $\ne$: for any
\at{item} tuple $t$, if $t[\at{type}]$ is {\em not} artwork, then
there must exist a \at{tax} tuple $t'$ such that $t[\at{state}] =
t'[\at{state}]$, \ie~the sale tax of the item can be found from the
\at{tax} relation.

Using dependencies $\kw{pfd}_1$--$\kw{pfd}_4$ and $\kw{pind}_1$ as data quality
rules, we find that $D_0$ of  Fig.~\ref{fig-instance} is {\em not}
clean. Indeed, (a) $t_2$ violates $\kw{pfd}_1$: its price is less
than 20, but its shipping fee is $2$ rather than $3$; similarly,
 $t_3$ violates $\kw{pfd}_2$, and $t_4$ violates $\kw{pfd}_3$. (b) Tuple $t_1$
violates $\kw{pfd}_4$: it is on sale but its price is not in the
range $[2.99, 9.99)$. (c) The database $D_0$ also violates
$\kw{pind}_1$: $t_1$ is not artwork, but its state cannot find a
match in the \at{tax} relation, \ie~no tax rate for WA is found in
$D_0$.
\end{example}
\vspace{-0.5ex}

None of $\kw{pfd}_1$--$\kw{pfd}_4$ and $\kw{pind}_1$
can be expressed as
\FDs or \INDs~\cite{AbHuVi1995}, which do not allows constants,
or as \CFDs~\cite{CFDs} or \CINDs~\cite{tcs-CINDs}, which specify
patterns with  equality $(=)$ only. While there have been
extensions of \CFDs~\cite{icde08,divesh08,ChenFM09}, none of these allows
dependencies to be specified with patterns on data values in terms of
built-in predicates $\ne, <, \le, >$ or $\ge$.  To the best of our
knowledge, the earlier conference version~\cite{dexaChenFM09} of this paper is the first to study these constraints.



These highlight the
need for extending \CFDs and \CINDs to capture errors
commonly found in real-life data.
While one can consider arbitrary extensions,
it is necessary to strike a balance between their expressive power and their complexity.
In particular, we want to be able to reason about data quality rules
expressed as extended \CFDs and \CINDs. Furthermore, we want to
have effective algorithms to detect inconsistencies
based on these extensions.


\stitle{Contributions \& Roadmap}.
To this end we introduce an extension of  \CFDs and \CINDs, investigate the static analyses of these constraints, and develop
effective \SQL-based techniques for detecting errors based on these constraints.

\sstab
\noindent(1) We propose two classes of dependencies, denoted by
\pCFDs and \pCINDs, which respectively extend \CFDs and \CINDs by supporting
$\ne$, $<$, $\le$, $>$, $\ge$ predicates (Sections~\ref{sec-cfd} and \ref{sec-cind}). For example, all
the dependencies we have encountered so far can be expressed as
\pCFDs or \pCINDs. These dependencies are capable of capturing
errors
%in a single relation (\pCFDs) or across different relations (\pCINDs)
in real-world data that cannot be detected by \CFDs or \CINDs.

\sstab
\noindent(2) We establish the complexity bounds for  the satisfiability and
implication problems for \pCFDs and \pCINDs,
taken separately or together (Section~\ref{sec-reasoning}). The
satisfiability problem is to determine whether a set $\Sigma$ of
dependencies has a nonempty model, \ie~whether the rules in $\Sigma$
are consistent themselves. The implication problem is to decide whether
a set $\Sigma$ of dependencies entails another dependency $\varphi$,
\ie~whether the rule $\varphi$ is redundant in the presence of
the rules in $\Sigma$. These are the central technical problems
associated with any dependency language.

We show that despite the increased expressive power,
\pCFDs and \pCINDs do not increase the complexity for
reasoning about them. In particular, we show that
the satisfiability and implication problems remain
(a) \NP-complete and \coNP-complete for \pCFDs, respectively,
(b) in $O(1)$-time (constant-time) and \EXPTIME-complete
for \pCINDs, respectively,
and (c) are undecidable when \pCFDs and \pCINDs are taken together.
These are {\em the same as} their \CFDs and \CINDs counterparts~\cite{tcs-CINDs}.
In contrast, data with linearly ordered domains often makes our lives harder \cite{Mayden97}.

%While data with linearly ordered domains often makes our lives harder (see, \eg~\cite{Mayden97}), \pCFDs and \pCINDs
%do not complicate their static analyses.

\sstab
\noindent(3) We provide \SQL-based techniques to detect errors
based on \pCFDs and \pCINDs (Section~\ref{sec-detect}). Given a set $\Sigma$ of {\pCFDs}
and \pCINDs on a database $D$, we automatically generate
a set of \SQL queries that, when evaluated on $D$, find all tuples in $D$ that
violate some dependencies in $\Sigma$. Further,
the \SQL queries are independent of the size and cardinality
of $\Sigma$.
%No previous work has been studied for error detection based on \CINDs,
%not to mention \pCFDs and \pCINDs taken together.
These provide the capability of detecting errors in a single
relation (\pCFDs) and across different relations (\pCINDs)
within the immediate reach of commercial \rdms.


\sstab
\noindent(4) Using real-life data (\hosp and \dblp), we finally conduct
an extensive experimental study (Section \ref{sec-exp}). We show that (a) the running time of \pCFDs and \pCINDs is comparable to their \CFDs and \CINDs counterparts, which is consistent with the static analyses in Section~\ref{sec-reasoning}, and (b) \pCFDs and \pCINDs are able to capture more errors  than their \CFDs and \CINDs counterparts (22\% on \hosp and  75\% on \dblp), due to the increased expressive power.



\stitle{Related work}. This paper is an extension of our earlier work~\cite{dexaChenFM09} by adding  %BFM07
(a) the proofs for the complexity bounds for the satisfiability and
implication analyses of \pCFDs and \pCINDs, separately and taken together (Section~\ref{sec-reasoning}),
%(b) the details of \SQL queries for \pCFDs (in the supplementary material),
and (b) an extensive experimental study of \pCFDs and \pCINDs (Section~\ref{sec-exp}), which was not investigated in~\cite{dexaChenFM09}.



%Extensions of \FDs and \INDs, known as \CFDs and \CINDs, were proposed and investigated in~\cite{CFDs} and~\cite{tcs-CINDs}, respectively.
%Recently, data dependencies have generated renewed interests for improving data
%quality (\cite{leo99,Chomicki07,CFDs,CM08,divesh08,FanGJLM11-md,FanLMTY-interaction,FanLTY12,FanLMTY12-vldbj,SahaS14}).



%%%%%%%%%%%%%%%%%%pCFD example
\begin{figure*}[tb!]
\vspace{1ex}
\begin{center}
(1)~$\varphi_1$ = \at{tax}(\at{state} $\ra$ \at{rate},
$T_{1}$) \hspace{5ex} (2)~$\varphi_2$ = \at{item}(\at{sale}
$\ra$ \at{shipping}, $T_{2}$)\\
\begin{tabular}{cc||c}
&\at{state} & \at{rate}\\
\cline{2-3} $T_1$: \ \ \ &\_& \_
\end{tabular}
\hspace{30ex}
\begin{tabular}{cc||c}
&\at{sale} & \at{shipping}\\
\cline{2-3} $T_2$: \ \ \  &= T &  = 0
\end{tabular}
\end{center}
\vspace{-0ex}

\begin{center}
(3)~$\varphi_3$ = \at{item}(\at{sale}, \at{price} $\ra$ \at{shipping},
$T_{3}$) \hspace{2ex} (4)~\pCFD $\varphi_4$ =
\at{item}(\at{sale} $\ra$ \at{price}, $T_{4}$)\\
\begin{tabular}{c c | c||c}
& \at{sale} & \at{price} & \at{shipping}\\
\cline{2-4}
& $=$ F &$>$ 20 & = 6   \\
$T_3$: \ \ \ & $=$ F &$\le$  40 & = 6
\end{tabular}
\hspace{20ex}
\begin{tabular}{cc||c}
&\at{sale} & \at{price}\\
\cline{2-3}
 &$=$ T &\ $\ge$ 2.99 \\
 $T_4$: \ \ \ & $=$ T &\ $<$ 9.99
\end{tabular}
\end{center}
\vspace{-1ex} \caption{Example \pCFDs} \label{fig-pcfd}
\vspace{-3ex}
\end{figure*}
%%%%%%%%%%%%%%%%%%%%%%%%%%%%%

Recently, data dependencies have generated renewed interests for improving data quality~\cite{Chomicki07,CFDs,tcs-CINDs,icde08,CM08,divesh08,ZanziT14,BauckmannALMN12,SahaS14}.
 Constraint-based data cleaning was introduced in~\cite{leo99}, which
proposed to use dependencies, \eg~\FDs, \INDs and denial
constraints, to detect and repair errors in real-life data (see,
\eg~\cite{AbHuVi1995,Chomicki07,SahaS14} for details).
Data dependencies
have been studied for relational databases
since the introduction of \FDs by Codd~\cite{Codd72} in 1972
(see,~\eg~\cite{AbHuVi1995} for details), and the theory of \INDs was established in \cite{CFP84}, which developed
a sound and complete inference system and the \PSPACE-completeness for the implication analysis
of \INDs.  As an extension
of traditional \FDs, \CFDs were developed in~\cite{CFDs}, for
improving the quality of data. It was shown in~\cite{CFDs} that the
satisfiability and implication problems for \CFDs are \NP-complete
and \coNP-complete, respectively. Along the same lines, \CINDs~\cite{tcs-CINDs} were proposed to extend \INDs, and it was shown~\cite{tcs-CINDs}
that the satisfiability and implication problems for \CINDs are in
constant time and \EXPTIME-complete, respectively. \SQL techniques
were developed in~\cite{CFDs} to detect errors by using \CFDs, but
have not been studied for \CINDs. This work extends the static
analyses of conditional dependencies of~\cite{CFDs,tcs-CINDs}, and has
established several new complexity results, notably in the absence
of finite-domain attributes (\eg~Theorems~\ref{thm-sat-pcfd-infin}, \ref{thm-imp-pcind-infin} and Proposition~\ref{thm-imp-pcfd-infin}). In addition,
it is the first work to develop \SQL techniques for checking
violations of \CINDs and violations of \pCFDs and \pCINDs taken
together.





Extensions of \CFDs have been proposed to support disjunction and
negation~\cite{icde08}, cardinality constraints and synonym
rules~\cite{ChenFM09}, and to specify patterns in terms of value
ranges~\cite{divesh08}. While \pCFDs are more powerful than the
extension of~\cite{divesh08}, they cannot express
disjunctions~\cite{icde08}, cardinality constraints and synonym
rules~\cite{ChenFM09}. To our knowledge no extensions of \CINDs have
been studied. This work is the first full treatment of extensions of
\CFDs and \CINDs by incorporating built-in predicates ($\ne, <, \le,
>, \ge$), from static analyses to error detection.



Methods have been developed for discovering
\CFDs~\cite{CM08,divesh08}, \pCFDs~\cite{ZanziT14} and \CINDs~\cite{BauckmannALMN12} and for repairing data based on
either \CFDs~\cite{repair}, traditional \FDs and \INDs taken
together~\cite{sigmod05}, \CFDs and \CINDs taken
together~\cite{jdiq-Cure12}, denial
constraints~\cite{BertossiBFL08}, aggregate
constraints~\cite{FlescaFP05}, matching dependencies~\cite{FanGJLM11-md}, matching dependencies and \CFDs~\cite{FanLMTY-interaction}, or editing rules and master data \cite{FanLMTY12-vldbj}. We defer the treatment of these topics for \pCFDs and \pCINDs to future work.




A variety of extensions of \FDs and \INDs have been studied for
specifying constraint databases and constraint logic
programs~\cite{ODGinsburgH83,BCW99,BP83,Maher97,MS96}.
While the languages
of~\cite{ODGinsburgH83,BCW99,Maher97} cannot express \pCFDs,
constraint-generating dependencies (\CGDs)
of~\cite{BCW99} and constrained tuple-generating dependencies
(\CTGDs) of~\cite{MS96} can express \pCFDs,
and \CTGDs can also express \pCINDs. The increased expressive
power of \CTGDs comes at the price of a higher complexity:
both their satisfiability and implication problems are undecidable.
Built-in predicates and arbitrary constraints are supported
by \CGDs, for which it is not clear whether effective \SQL
queries can be developed to detect errors. It is worth mentioning
that Theorems~\ref{thm-sat-pcfd-infin} and \ref{thm-imp-pcfd-infin}
of this work provide lower bounds for the consistency and implication analyses
of \CGDs, by using patterns with built-in predicates only.

Observe that constraints specifying semantics with orderings have long been recognized, such as order dependencies~\cite{ODGinsburgH83} supporting the comparison of attributes with $=, <, \le, >, \ge$, matching dependencies \cite{FanGJLM11-md} and differential dependencies \cite{DDSong011} that support the comparison of attributes with $=, \ne, <, \le, >, \ge$ for record matching. However, different from \CFDs and \pCFDs, these constraints do not specify conditions on those tuples such that the embedded \FDs hold. Further, it is also possible that other existing constraints could be improved by incorporating these built-in predicates, such as metric functional dependencies \cite{MDKoudasSSV09}.
