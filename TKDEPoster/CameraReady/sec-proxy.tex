\section{Routing Proxies}
\label{sec-proxy}



We first propose {\em routing  proxies} and {\em deterministic routing areas} (\dras) to capture the idea of representatives and the set of
nodes being represented, respectively.




\stitle{Proxies}. Given a node $u$ in graph $G(V, E)$, we say that $u$ is a {\em routing proxy} (or simply {\em proxy}) of a set of nodes, denoted by $A_{u}$, if and only if:

\sstab(1) node $u\in A_{u}$ is reachable to any node of $A_u$ in $G$,

\sstab(2) all neighbors of any node $v\in A_u\setminus \{u\}$ are in $A_u$,  and

\sstab(3) the size $|A_u|$ of $A_u$ is equal to or less than $c\cdot\lfloor\sqrt{|V|}\rfloor$, where $c$ is a small constant number, such as $2$ or $3$.


Here condition (1) guarantees the connectivity of subgraph $G[A_u]$,  condition (2) implies that not all neighbors of proxy $u$ are necessarily in $A_u$;
and condition (3), called {\em size restriction}, limits the size of $A_u$ of proxy $u$.
Intuitively, one simply checks the graph by removing $u$ from $G$ and its newly created
connected components (\ccs), and a proxy of $u$ is a union of such \ccs whose total number of nodes is at most $c\cdot\lfloor\sqrt{|V|}\rfloor - 1$.




\stitle{Deterministic routing areas}. A node $u$ may be a proxy of multiple sets of nodes $A^1_u, \ldots, A^k_u$.
We denote as $A^{+}_u$ the union of all the sets of nodes whose proxy is $u$ , \ie  $A^{+}_u$ = $A^1_u\cup\ldots\cup$ $A^k_u$,
and $A^i_u$ ($i\in[1,k]$) is said a component of $A^{+}_u$.

We refer to the {\em subgraph} $G[A^+_u]$ as a deterministic routing area (\dra) of proxy $u$.
%
Intuitively, \dra $G[A^+_u]$ is a {\em maximal} connected subgraph, union of a set of \ccs, that connects to the rest of graph $G$ through proxy $u$ only.
That is, for any node $v$ in $G[A^+_u]$ and any node $v$ in the rest of graph $G$, $u$ must appear on the shortest path $\path(v, v')$.

\stitle{Maximal proxies}.  We say that a proxy $u$ is {\em maximal} if there exist no other proxies $u'$ such that $u'\ne u$ and $A^+_{u} \subset A^+_{u'}$.

\stitle{Trivial proxies}. We say that a maximal proxy $u$ is {\em trivial} if $A^+_u$ contains itself only, \ie $A^+_{u}$ = $\{u\}$.


\stitle{Equivalent proxies}. We say that two proxies $u$ and $u'$ are {\em equivalent}, denoted by $u\equiv u'$, if $A^+_{u} = A^+_{u'}$.


 Trivial proxies only represent themselves. Hence, we only focus on non-trivial maximal proxies (or simply proxies).

We then give an analysis of the properties of  \dras and their routing proxies, and show that they indeed hold the desired properties of representatives discussed in Introduction.




\begin{prop}
\label{pro-proxy-path} Given any two nodes $v, v'$ in the \dra $G[A^+_u]$ of proxy $u$ in graph $G$,
(1) the shortest path in $G[A^+_u]$ is exactly the one in the entire graph $G$, and
(2) it can be computed in linear time in the size of $G$.
\end{prop}


\begin{cor}
\label{cor-proxy-distance} Given any two nodes $v, v'$ in the \dra $G[A^+_u]$ of proxy $u$ in graph $G$,
(1) the shortest distance $\dist(v, v')$ in $G[A^+_u]$ is exactly the one in the entire graph $G$, and
(2) it can be computed in linear time in the size of $G$.
\end{cor}


\begin{prop}
\label{pro-proxy-path-global} Given two nodes $v$ and $u$ with two distinct proxies $x$ and $y$, respectively, in graph $G$, the shortest path from $v$ to $u$ is the concatenation of three paths, \ie $\path(v, x)/\path(x, y)/\path(y, u)$.
\end{prop}

\begin{cor}
\label{cor-proxy-distance-global} Given two nodes $v$ and $u$ with two distinct proxies $x$ and $y$, respectively, in graph $G$, the shortest distance $\dist(v, u)$ = $\dist(v, x)$ $+$ $\dist(x, y)$  $+$ $\dist(y, u)$.
\end{cor}

\begin{theorem}
\label{thm-compute-dras} Finding all \dras, each associated with one maximal proxy, in a graph can be done in linear time.
\end{theorem}






