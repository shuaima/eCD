\section{Experimental Study}
\label{sec-expt}


Using real-life road networks and social graphs, we next conduct an extensive experimental study. 
(1) {\em Co-authorship networks} extracted a co-authorship graph from \dblp (SNAP, \url{http://snap.stanford.edu/data}),
 and (2) {\em Road networks} (DIMACS, \url{http://www.dis.uniroma1.it/challenge9/download.shtml}).
Note that \tnr is designed for road networks and it is very inefficient for \tnr to preprocess dense graphs such as DBLP (it took more than 1 week to finish the preprocessing). Hence, we remove all nodes whose degrees are higher than 14, and choose the largest connected component in the remaining graph, referred to as \dblpone. Also \ah requires the coordinate information to answer shortest path or distance queries, not available in both \dblp and \dblpone.

\stitle{Experimental results}.
(1) In sparse graphs whose average degree is less than 4, about 1/3 nodes in the graph are captured by proxies, leaving the reduced graph about 2/3 of the input graph. In some cases such as \dblpone, about 2/3 nodes in the graph are captured by proxies, leaving the reduced graph about only 1/3 of the input graph.

\sstab(2) The performance of proxies and \dras is sensitive to the density and degree distribution of graphs, and they perform well on graphs following the power law distribution. Meanwhile, for a given degree distribution, \dras tend to capture less nodes when the average degree is higher. 

\sstab(3) Proxies and their \dras benefit existing shortest path and distance algorithms in terms of time cost. They reduce (20\%, 1\%) time for (\arcflag, \ah), and have comparable running time for \tnr on road networks; They reduce (4\%, 49\%) time for (\arcflag, \tnr) on the co-authorship network \dblpone.

\sstab(4) Existing shortest path and distance algorithms also benefit from using proxies in terms of space overhead. Proxy+\tnr can handle the road network C-US while \tnr cannot. Moreover, (Proxy+\arcflag, Proxy+\tnr, Proxy+\ah) incur less space overhead than their counterparts, and are about (38\%--68\%, 72\%--92\%, 82\%) of (\arcflag, \tnr, \ah), respectively.

