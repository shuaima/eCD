\section{Experimental Study}
\label{sec-expt}


Using real-life road networks and social graphs, we next conduct an extensive experimental study. 
We use two types of datasets, (1) {\em co-authorship networks} extracted a co-authorship graph from \dblp (SNAP Datasets, \url{http://snap.stanford.edu/data}),
 and (2){\em road networks} from the Ninth DIMACS
Implementation Challenge ({\url{http://www.dis.uniroma1.it/challenge9/download.shtml}}).
\tnr is designed for road networks and it is very inefficient for \tnr to preprocess dense graphs such as DBLP (it took more than 1 week to finish the preprocessing). To guarantee that we can evaluate the improvement of \tnr with proxies on general graphs, we remove all nodes whose degrees are higher than 14, and choose the largest connected component in the remaining graph, referred to as \dblpone.

\stitle{Experimental results}.
We find the following. (1) According to our experiments, in sparse graphs whose average degree is less than 4, about 1/3 nodes in the graph are captured by proxies, leaving the reduced graph about 2/3 of the input graph. In some special cases (like \dblpone), about 2/3 nodes in the graph are captured by proxies, leaving the reduced graph about only 1/3 of the input graph. 
(2) The performance of proxies and \dras is sensitive to the density and degree distribution of graphs, and they perform well on graphs following the power law distribution. Meanwhile, for a given degree distribution, \dras tend to capture less nodes when the average degree is higher. (3) Proxies and their \dras benefit existing shortest path and distance algorithms in terms of time cost. They reduce (20\%, 1\%) time for (\arcflag, \ah) on road networks, respectively. They also have comparable time cost for \tnr on road networks; They reduce (4\%, 49\%) time for (\arcflag, \tnr) on the co-authorship network \dblpone, respectively. (4) Existing shortest path and distance algorithms also benefit from using proxies in terms of space overhead. Proxy+\tnr can handle the road network C-US while \tnr cannot. Moreover, Proxy+\arcflag incurs less space overhead than \arcflag (from 38\% to 68\%), Proxy+\tnr is about from 72\% to 92\% of its counterpart without proxies, and Proxy+\ah is about 82\% of its counterpart without proxies.

