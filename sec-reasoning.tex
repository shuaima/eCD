%\vspace{-0.5ex}
\section{Reasoning about CFD$^p$s and CIND$^p$s}
\label{sec-reasoning}
%\vspace{-1ex}

The satisfiability problem and the implication problem are the two
central technical questions associated with any dependency languages.
In this section we investigate these
problems for \pCFDs and \pCINDs, separately and taken together.

%%%%%%%%%%%%%%%%%
\vspace{-1ex}
\subsection{The Satisfiability Analysis}

The satisfiability problem is to determine, given a set $\Sigma$ of
constraints, whether there exists a {\em nonempty} database that
satisfies $\Sigma$.

The satisfiability analysis of conditional dependencies is not only
of theoretical interest, but is also important in practice. Indeed,
when \pCFDs and \pCINDs are used as data quality rules, this
analysis helps one check whether the rules make sense themselves.
The need for this is particularly evident when the rules are
manually designed or discovered from various
datasets~\cite{CM08,divesh08,icde09}.

\stitle{The satisfiability analysis of CFD$^p$s}.~Given any
\FDs, one does not need to worry about their satisfiability
since any set of \FDs is always satisfiable. However, as observed
in~\cite{CFDs}, for a set $\Sigma$ of \CFDs on a relational schema
$R$, there may not exist a {\em nonempty} instance $I$ of $R$ such
that $I\models\Sigma$. As \CFDs are a special case of \pCFDs, the
same problem exists when it comes to \pCFDs.

\vspace{-0.5ex}
\begin{example}
\label{exam-sat-cfd} Consider \pCFD $\varphi$ = $(R: A\ra
B,\ T_p)$ such that $T_p$ = $\{(\_\pa = a), (\_ \pa \ne a)\}$.
Then there exists no {\em nonempty} instance $I$ of $R$ that
satisfies $\varphi$. Indeed, for any tuple $t$ of $R$,
$\varphi$ requires that both $t[B] = a$ and
$t[B]\ne a$.
\eop
\end{example}
\vspace{-0.5ex}

This problem is already \NP-complete for \CFDs~\cite{CFDs}.
Below we show that it has the
same complexity for \pCFDs despite their increased
expressive power.

%\vspace{-0.5ex}
\begin{prop}
\label{thm-sat-pcfd-fin} The satisfiability problem for \pCFDs is
\NP-complete. \eop
\end{prop}
\vspace{-0.5ex}

\proofs The lower bound follows from the \NP-hardness of their \CFDs
counterparts~\cite{CFDs}, since \CFDs are a special case of \pCFDs.
The upper bound is verified by presenting an \NP algorithm that,
given a set $\Sigma$ of \pCFDs defined on a relation schema $R$,
determines whether $\Sigma$ is satisfiable. \eop \vspace{1ex}

\eat{This requires the establishment of a small model property.}

It is known~\cite{CFDs} that the satisfiability problem for \CFDs is
in \PTIME when the \CFDs considered are defined over attributes that
have an infinite domain, \ie in the absence of finite domain
attributes. However, this is no longer the case for \pCFDs. This
tells us that the increased expressive power of \pCFDs does take a
toll in this special case. It should be remarked that while the
proof of Proposition~\ref{thm-sat-pcfd-fin} is an extension of its
counterpart in~\cite{CFDs}, the result below is new.


\begin{theorem}
\label{thm-sat-pcfd-infin} In the absence of finite domain
attributes, the satisfiability problem for \pCFDs remains
\NP-complete. \eop
\end{theorem}

\vspace{-0.5ex} \proofs The problem is in \NP by
Proposition~\ref{thm-sat-pcfd-fin}. Its \NP-hardness is shown by
reduction from the \kSAT\ problem, which is \NP-complete
(cf.~\cite{GaJo79}).\eop
\vspace{0.5ex}


\stitle{The satisfiability analysis  of CIND$^p$s}. Like \FDs,
one can specify arbitrary \INDs or \CINDs without worrying about
their satisfiability. Below we show that \pCINDs also have this
property, by extending the proof of its counterpart in~\cite{CINDs}.

\begin{prop}
\label{thm-sat-pcind} Any set $\Sigma$ of \pCINDs is always
satisfiable. \eop
\end{prop}
\vspace{-1ex}

\proofs Given a set $\Sigma$ of \pCINDs over a
database schema $\cal R$, one can always construct a
{\em nonempty} instance $D$ of $\cal R$ such that $D \models
\Sigma$. \eop
\vspace{0.5ex}


\stitle{The satisfiability  analysis of CFD$^p$s and
CIND$^p$s}. The satisfiability problem for \CFDs
and \CINDs taken together is undecidable~\cite{CINDs}. Since \pCFDs and
\pCINDs subsume \CFDs and \CINDs, respectively, from these we
immediately have:

\vspace{-0.5ex}
\begin{cor}
\label{thm-sat-pcfd-pcind}The satisfiability problem for \pCFDs and
\pCINDs is undecidable.\eop
\end{cor}


%%%%%%%%%%%%%%
\vspace{-3ex}
\subsection{The Implication Analysis}

The implication problem is to determine, given a set $\Sigma$ of
dependencies and another dependency $\phi$, whether or not
$\Sigma$ entails $\phi$, denoted by $\Sigma\models\phi$. That is,
whether or not for all databases $D$, if $D\models\Sigma$ then
$D\models\phi$.

The implication analysis helps us
remove redundant data quality rules, and thus improve the
performance of error detection and repairing based on
the rules.

\begin{example}
\label{exa-implication} The \pCFDs of Fig.~\ref{fig-pcfd} imply
\pCFDs $\varphi =$ \at{item}(\at{sale}, \at{price} $\ra$
\at{shipping}, $T$), where $T$ consists of a single pattern tuple
$(\at{sale} = $`F', $\at{price} = 30 \pa \at{shipping} = 6)$. Thus
in the presence of the \pCFDs of  Fig.~\ref{fig-pcfd}, $\varphi$ is
redundant. \eop
\end{example}
\vspace{-1ex}

\stitle{The implication analysis of CFD$^p$s}.~We first show that
the implication problem for \pCFDs retains
the same complexity as their \CFDs counterpart. The result
below is verified by
extending the proof of its counterpart in~\cite{CFDs}.

\begin{prop}
\label{thm-imp-pcfd-fin}The implication problem for \pCFDs is
\coNP-complete. \eop
\end{prop}
\vspace{-1ex}

\proofs The lower bound follows from the
\coNP-hardness of their \CFDs counterpart~\cite{CFDs}, since \CFDs
are a special case of \pCFDs. The \coNP upper bound is verified
by presenting an \NP algorithm for its complement problem, \ie
the problem for determining whether $\Sigma\not\models\varphi$. \eop
\vspace{0.5ex}

Similar to the satisfiability analysis,
it is known~\cite{CFDs} that the implication analysis of
\CFDs is in \PTIME when the \CFDs are defined only with attributes that
have an infinite domain. Analogous to Theorem~\ref{thm-sat-pcfd-infin},
the result below shows that this is no longer the case for \pCFDs,
which does not find a counterpart in~\cite{CFDs}.

\begin{theorem}
\label{thm-imp-pcfd-infin} In the absence of finite domain
attributes, the implication problem for \pCFDs remains
\coNP-complete. \eop
\end{theorem}
\vspace{-1ex}

\proofs It is in \coNP by
Proposition~\ref{thm-imp-pcfd-fin}. The \coNP-hardness is shown
by reduction from the \kSAT\ problem to its complement problem, \ie
the problem for determining whether $\Sigma\not\models\varphi$.
\eop
\vspace{0.5ex}


\stitle{The implication analysis of CIND$^p$s}.~We next show that \pCINDs
do not make their implication
analysis harder. This is verified by extending the proof of
their \CINDs counterpart given in~\cite{CINDs}.

\begin{prop}
\label{thm-imp-pcind-fin}The implication problem for \pCINDs is
\EXPTIME-complete. \eop
\end{prop}
\vspace{-1ex}

\proofs The implication problem for \CINDs is
\EXPTIME-hard~\cite{CINDs}. The lower bound carries over to \pCINDs
since \pCINDs subsume \CINDs. The \EXPTIME upper bound is shown
by presenting an \EXPTIME algorithm that, given a set
$\Sigma\cup\{\psi\}$ of \pCINDs over a database schema $\cal R$,
determines whether $\Sigma\models\psi$. \eop
\vspace{0.5ex}

It is known~\cite{CINDs} that the implication problem is
\PSPACE-complete for \CINDs defined with infinite-domain
attributes. Similar to
Theorem~\ref{thm-imp-pcfd-infin}, below we present a new result
showing that this no longer holds for \pCINDs.


\begin{theorem}
\label{thm-imp-pcind-infin} In the absence of finite domain
attributes, the implication problem for \pCINDs remains
\EXPTIME-complete. \eop
\end{theorem}
\vspace{-1ex}

\proofs The \EXPTIME upper bound follows from
Proposition~\ref{thm-imp-pcind-fin}. The \EXPTIME-hardness is
shown by reduction from the implication problem for \CINDs in the
general setting, in which finite-domain attributes may be present;
the latter is known to be \EXPTIME-complete~\cite{CINDs}. \eop


%%%%%%%%%%%%%%%%%%%%%%%%Complexity Summary%%%%%%%%%%%%%%%%%%%%%%%%%%%%%%
\begin{table*}[tb!]
\vspace{-1ex}
 \caption{Summary of Complexity Results\label{tab-complexity}}
\begin{center}
\begin{small}
\begin{tabular}{|c|c|c||c|c|} \hline
&  \multicolumn{2}{|c||}{\at{General\ setting}} & \multicolumn{2}{|c|}{\at{Infinite\ domain\ only}}\\
\cline{2-5}
\raisebox{1.5ex}[0pt]{$\Sigma$}  & \at{Satisfiability} & \at{Implication} & \at{Satisfiability} & \at{Implication}\\
\hline\hline \CFDs~\cite{CFDs} & \NP-complete& \coNP-complete & \PTIME & \PTIME \\
\hline
\pCFDs   & \NP-complete& \coNP-complete &\NP-complete& \coNP-complete  \\
\hline
\CINDs~\cite{CINDs} & $O(1)$ & \EXPTIME-complete &  $O(1)$ & \PSPACE-complete\\
\hline
\pCINDs  &  $O(1)$  & \EXPTIME-complete & $O(1)$  & \EXPTIME-complete\\
\hline
\CFDs+ \CINDs~\cite{CINDs} & undecidable& undecidable & undecidable& undecidable\\
\hline
\pCFDs+ \pCINDs & undecidable& undecidable & undecidable& undecidable\\
\hline
\end{tabular}
\end{small} 
\end{center}
\vspace{-5ex}
\end{table*}



\stitle{The implication analysis of  CFD$^p$s and
CIND$^p$s}. When \pCFDs and \pCINDs are taken together,
their  implication analysis is beyond reach in practice.
This is not surprising since
the implication problem for \FDs and
\INDs is already undecidable~\cite{AbHuVi1995}. Since
\pCFDs and \pCINDs subsume \FDs and \INDs, respectively,
from the undecidability result for \FDs and
\INDs,  the corollary below follows immediately.

\begin{cor}
\label{thm-IM-pcfd-pcind} The implication problem for \pCFDs and
\pCINDs is undecidable.\eop
\end{cor}



%\vspace{-0.5ex}
\stitle{Summary}. The complexity bounds for reasoning about
\pCFDs and \pCINDs are summarized in Table~\ref{tab-complexity}.
To give a complete picture we also include
in Table~\ref{tab-complexity} the complexity bounds for
the static analyses of
\CFDs and \CINDs, taken from~\cite{CFDs,CINDs}. The results shown in
Table~\ref{tab-complexity} tell us the following.

\sstab
(a) Despite the increased expressive
power, \pCFDs and \pCINDs do not complicate the static analyses: the
satisfiability and implication problems for \pCFDs and \pCINDs have
the same complexity bounds as their
counterparts for \CFDs and \CINDs, taken separately or together.

\sstab
(b) In the special case when \pCFDs and \pCINDs are defined
with infinite-domain attributes only, however, the
static analyses of \pCFDs and \pCINDs
do not get simpler, as opposed to their counterparts for
\CFDs and \CINDs. That is, in this special case the increased
expressive power of \pCFDs and \pCINDs comes at a price.
