\documentclass[11pt]{letter}

\usepackage{times,,xspace,amsmath,amssymb,color}
\usepackage[english]{babel}
\parindent=0cm
\parskip=0.28cm
\setlength{\textwidth}{17cm}
\setlength{\oddsidemargin}{-0.24cm}
\setlength{\evensidemargin}{-0.24cm}
\setlength{\topmargin}{0cm}
\setlength{\headheight}{0cm}
\setlength{\textheight}{24cm}
\setlength{\headsep}{0cm}
\newcommand{\eat}[1]{}
\date{}

\newcommand{\sstab}{\rule{0pt}{8pt}\\[-3.4ex]}
\newcommand{\stab}{\rule{0pt}{8pt}\\[-2.4ex]}
\newcommand{\vs}{\vspace{1ex}}
\newcommand{\svs}{\vspace{0.36ex}}
\newcommand{\kw}[1]{{\ensuremath {\mathsf{#1}}}\xspace}
\newcommand{\dist}{\kw{dist}}
\newcommand{\pred}{\kw{pred}}
\newcommand{\desc}{\kw{desc}}
\newcommand{\pSim}{\kw{Match}}
\newcommand{\at}[1]{\protect\ensuremath{\mathsf{#1}}\xspace}

\newcommand{\NP}{\kw{NP}}
\newcommand{\DAGs}{{\sc dag}s\xspace}
\newcommand{\NC}{\kw{NC}\xspace}
\newcommand{\coNP}{co\kw{NP}\xspace}
\newcommand{\PTIME}{\kw{PTIME}}
\newcommand{\PSPACE}{\kw{PSPACE}}
\newcommand{\EXPTIME}{\kw{EXPTIME}\xspace}
\newcommand{\NPSPACE}{\kw{NPSPACE}\xspace}


\newcommand{\bi}{\begin{itemize}}
\newcommand{\ei}{\end{itemize}}
\newcommand{\be}{\begin{enumerate}}
\newcommand{\ee}{\end{enumerate}}
\newcommand{\im}{\item}
\newenvironment{tbi}{\begin{itemize}
        \setlength{\topsep}{1.5ex}\setlength{\itemsep}{0ex}\vspace{-0.5ex}}
        {\end{itemize}\vspace{-0.5ex}}
\newenvironment{tbe}{\begin{enumerate}
        \setlength{\topsep}{0ex}\setlength{\itemsep}{-0.7ex}\vspace{-1ex}}
        {\end{itemize}\vspace{-1ex}}

\newcommand{\eps}{\prec}
\newcommand{\deps}{\prec_{D}}
\newcommand{\leps}{\prec_L}
\newcommand{\dleps}{\prec_{D}^{L}}
\newcommand{\iso}{\lhd}
\newcommand{\bieps}{\sim}
\newcommand{\embed}{\lessdot}
\newcommand{\neps}{\ntrianglelefteq}
\newcommand{\ees}{\preceq_{(e,e)}}
\newcommand{\nees}{\not\preceq_{e,e}}
\newcommand{\Reps}{S}
\newcommand{\bcp}{{\sc bcp}\xspace}
\newcommand{\ie}{\emph{i.e.,}\xspace}
\newcommand{\eg}{\emph{e.g.,}\xspace}
\newcommand{\wrt}{\emph{w.r.t.}\xspace}
\newcommand{\aka}{\emph{a.k.a.}\xspace}
\newcommand{\kwlog}{\emph{w.l.o.g.}\xspace}


\definecolor{gray}{rgb}{0.5,0.5,0.5}
\newcommand{\added}[1]{\textcolor{blue}{#1}}
\newcommand{\changed}[1]{\textcolor{red}{#1}}
\newcommand{\removed}[1]{\textcolor{gray}{#1}}

\newcommand{\ball}[1]{\hat{G}[#1]}
\newcommand{\amazon}{\kw{Amazon}}
\newcommand{\Amazon}{\kw{Amazon}}
\newcommand{\youtube}{\kw{YouTube}}
\newcommand{\YouTube}{\kw{YouTube}}


\newcommand{\match}{\kw{Match}}
\newcommand{\optmatch}{\kw{Match^+}}
\newcommand{\dismatch}{\kw{dMatch}}
\newcommand{\optdismatch}{\kw{dMatch^+}}
\newcommand{\minq}{\kw{minQ}}
\newcommand{\graphsim}{\kw{Sim}}
%\newcommand{\subiso}{\kw{SubIso}}
%\newcommand{\dissubiso}{\kw{dSubIso}}
\newcommand{\metis}{{\sc Metis}\xspace}
\newcommand{\vf}{\kw{VF2}}
\newcommand{\tale}{\kw{TALE}}
\newcommand{\mcs}{\kw{MCS}}
\newcommand{\dsim}{\kw{dSim}}
\newcommand{\dissubiso}{\kw{dVF2}}
\newcommand{\dvf}{\kw{dVF2}}

\newcommand{\stitle}[1]{\vspace{0.5ex} \noindent{\bf #1}}
\newcommand{\etitle}[1]{\vspace{1ex}\noindent{\underline{\em #1}}}






\begin{document}



\noindent
Prof. Jian Pei,\\
Editor-in-Chief,\\
IEEE Transactions on Knowledge and Data Engineering

\vspace{0.3cm}
\noindent
Dear Prof. Pei,


Attached please find a revised version of our submission to IEEE Transactions on Knowledge and Data Engineering,
{\em Extending Conditional Dependencies with Built-in Predicates}.

The paper has been substantially revised according to the referees'
comments. In particular,
%
(1) we have added discussions on the axiomatizations of $CFD^ps$ and $CIND^ps$, 
%
(2) we have added the discussions of three new references in the related work, and
%
(3) we have also taken this opportunity to rewrite several parts of the paper to improve the presentation.



We would like to thank all the referees for their thorough reading of our
paper and for their valuable comments.

Below please find our responses to the comments by the referees.

%%%%%%%%%%%%%%%%%%%%%
\vspace{3.6ex}
\hrule
\vspace{0.6ex}

\vspace{2ex} \stitle{Response to the comments of Referee 1}.

{\em
{\bf [R1C1]} An important problem in dependency theory is implication axioms. The paper does not mention this. I guess the reason could be that the axioms are the same as those for CFDs. Even so, I would like to see a discussion on this somewhere around Sec4.2.}
\svs

Yes, we agree with the reviewer, and we have added remarks to explain this in the end of Section 4.2 (page 8). Thanks for the suggestion!

\noindent
{\em
{\bf [R1C2]}  Please check $t_1, t_2$, and t before Example 8 on Page 9.}
\svs

We have changed all $t$ to $t_1$ in bullet (1).  Thanks for spotting this out!

\vspace{2.8ex}
\hrule
\vspace{0.6ex}
{\bf Response to the comments of Referee 2.}



\vs
\noindent
{\em
{\bf [R2C1]}
1. The techniques in Sec 5 are not easy to follow. A paragraph before Sec 5.1 explaining the basic idea would help.
}
\svs

We have added a paragraph to explain the basic idea before Section 5.1. Thanks for the suggestion!

\vs
\noindent
{\em
{\bf [R2C2]}
2. Figure 6(a): Why does the running time of $CFD^pS$ fluctuate when $20 <= |I_1| <= 40$?}
\svs

This is indeed caused by the changing of the query plans of SQL Server. In our experiments, we used the default settings of SQL Server, and it happened that when $|I_1| > 20k$, SQL Server used ``Parallelism Distribute Streams'' for optimizing the join operations, but not when $|I_1| <= 20k$. Hence, there is a drop when $I_1 = 30k$ in Figure 6(a). However, the entire running time trend is clear, and we choose not to explain this to avoid introducing more confusions to readers.





\vs
\noindent
{\em
{\bf [R2C3]}
3. Figures 9, 10, and 11: The gaps between $CFD^pS$ ($CIND^pS and CFD^pS + CIND^pS$) and their counterparts are more significant on DBLP than on HOSP. Could you please explain the reason?}
\svs

This is caused by the different ratios of $CFD^ps$ and $CIND^ps$  and their approximate $CFDs$ and $CINDs$ in DBLP and HOSP, and the more $CFD^ps$  and $CIND^ps$ that can bee expressed by their corresponding approximate $CFDs$ and $CINDs$, the small the gap is for the accuracy tests.
% 
There are about 66.7\% $CFD^ps$ and 66.7\% $CIND^ps$ in HOSP that can be expressed by CFDs and CINDs, respectively. However, there are only about 41.8\% $CFD^ps$ and 51.3\% $CIND^ps$ in DBLP that can be expressed by CFDs and CINDs, respectively. Hence, the gaps are more significant on DBLP than on HOSP.


We have added bullet (4) to explain this in the last paragraph of Section 6.2.2.


\vs
\noindent
{\em
{\bf [R2Minor1]}
Line 48, col 2, page 3: (b) t[A] op a if tpi [A] is 'op a' -> 'op a'?}
\svs


Indeed this is not a typo, and here $t$ refers to a tuple in a relation.


\vs
\noindent
{\em
{\bf [R2Minor2]}
Para 3, col 2, page 11: All the "$CFD^p$" in this paragraph should be "$CIND^p$"?}
\svs

Yes, we have fixed this. Thanks!


\vspace{2.8ex}
\hrule
\vspace{0.6ex}
{\bf Response to the comments of Referee 3.}
\vs
\noindent
{\em
{\bf [R3.W2]}
 Proofs of theorems are rather simple and not that much involving (also not surprising). As authors stated lower bound �C NP-hardness �C for new dependencies follows from the fact that they subsume CFDs and CINDs. The remaining part that the problem is in NP is rather simple. (Proofs are already sketched in the conference paper.)}
\svs

As we pointed out in the related work, this work indeed extends the static analyses of CFDs [22] and CINDs [30], respectively, and has established several new and interesting complexity results, notably in the absence of finite-domain attributes (e.g., Theorems 2, 8 and Proposition 6).  The results of Proposition 6 and Theorem  8  are beyond NP-complete, e.g., coNP-complete and EXPTIME-complete.



\vs
\noindent
{\em
{\bf [R3.W2]}
 This paper would be a more interesting extension of the conference version of the paper if authors would provide some new theoretical results. I would recommend adding a sound and complete axiomatization, which is missing.  I believe the proof of completeness of axiomatization would be more involved and provide more background how reason about these new dependencies. (See axiomatization for Order Dependencies as it is related work, Ginsburg et al.)}
\svs

Yes, we agree with the reviewer that a sound and complete axiomatization is interesting. However, the proofs and analyses of this work does not count on the axiomatization system. Moreover, the paper is already very dense, and we have indeed moved the part of generating SQLs for $CFD^ps$ to supplementary.
Hence, we simply added a discussion on the axiomatization in the end of Section 4.2, as reviewer 1 suggested. Thanks for the suggestion!


\vs
\noindent
{\em
{\bf [R3.W3\&W4]}
 Authors only discuss relationship to (conditional) functional dependencies and inclusion dependencies. However, no context is provided about other integrity constraints that are directly related. Author should discuss relationship to other constraints such as, Differential Dependencies (Song et al.), Order Dependencies (Ginsburg et al.) and Denial Constraints. For instance, Order Dependencies and Differential Dependencies allow to express $<, \le, >, \ge$ and $!=$ predicates. (For instance, inference problem for Differential Dependencies and Order Dependencies is also co-NP-complete).\\
 
 This paper focuses on CFDs and CINDs but does not include discussion on other constraint. Authors could add some discussion (e.g., one or two paragraphs) if other types of FDs, such as recent Metric Functional Dependencies could be improved with expanding the list of predicates allowed.}
 \svs


We have added the following three new references, \eg order dependencies, metric functional dependencies and differential dependencies.

{\small [27] S. Ginsburg and R. Hull. Order dependency in the relational model. Theor. Comput. Sci., 26:149�C195, 1983.

[29]  N. Koudas, A. Saha, D. Srivastava, and S. Venkatasubramanian.
Metric functional dependencies. In ICDE, 2009. 
 
[34] S. Song and L. Chen. Differential dependencies: Reasoning and
discovery. ACM Trans. Database Syst., 36(3):16, 2011.
}

We have also added a discussion in the end of the related work part. Thanks for the suggestion!  Note that, to save space, we also removed some references, mostly our own previous work



\vs
\noindent
{\em
{\bf [R3.W5]}
Authors could add some discussion if there are other types of operators that would be worth including into the list of predicates allowed (e.g., in future work).}
\svs

We indeed have already mentioned this in the future work, e.g., the first sentence in the second paragraph of Section 7. Thanks for the suggestion!


\vs
\noindent
{\em
{\bf [R3Minor]}
Defining a ��wild car�� symbol as the underscore (��\_��) was a little confusing. A more logical choice for a ��wild card�� would be one that is often used in regular expressions, such as an asterisk (��*��)}
\svs

This indeed follows the convention of previous work, such as CFDs [22] and CINDs [30], and hence we believe that it would be better to keep this convention for the convenience of readers.

\vspace{3.6ex}
\hrule
\vspace{3.6ex}
\closing{Your sincerely,}

\vspace{-8ex}
Shuai Ma, Liang Duan, Wenfei Fan, Chunming Hu, and Wenguang Chen
\end{document}
