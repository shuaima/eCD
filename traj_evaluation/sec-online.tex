\subsection{Online Algorithms}

Online \lsa algorithms adopt local checking policies by restricting the distance checking within a sliding or opening window such that there is no need to have the entire trajectory ready before compressing. That is, online algorithms essentially combine {\em batch algorithms} with {\em sliding or opening windows}, \eg\
\opwa \cite{Meratnia:Spatiotemporal} is a combination of top-down algorithm \dpa and opening windows while \kw{SWAB} \cite{Keogh:online} is a combination of bottom-up algorithm \tpa and \textit{sliding windows}.
Though these algorithms support the three distance metrics \ped, \sed and \dad, they still have high time and/or space complexities \cite{Liu:BQS}.
To design more efficient online algorithms, techniques typically need to be designed closely coupled with distance metrics.
Indeed, \bqsa \cite{Liu:BQS} and \squishe \cite{Muckell:Compression} propose to utilize convex hulls and priority queues, respectively, and they speed up trajectory simplification using \ped and \sed, respectively. To our knowledge, no specific techniques have been developed for \dad.
Hence, {\em we choose algorithms \bqsa, \squishe and \opwa as the representatives of online algorithms using \ped, \sed and \dad, respectively}.


%, which significantly hinders their utility in resource-constrained mobile devices \cite{Liu:BQS}.
%\bqsa \cite{Liu:BQS} and \squishe\cite{Muckell:Compression} further optimize these online algorithms, respectively.
%Where
%
\eat{
Recently, \bqsa \cite{Liu:BQS} has been proposed, using a new distance checking method by picking out at most eight special points from an open window based on a convex hull, \eg a rectangular bounding box with two bounding lines, so that when a new point is added to a window, it only needs to calculate the distances of the  special points to a line, instead of all data points in the window, in many cases.
The time complexity of \bqsa remains $O(n^2)$ in the worst case, as \bqsa falls back to \dpa when the eight special points cannot be used. However, its simplified version, \fbqsa directly outputs a line segment, and starts a new window when the eight special points cannot bound all the points considered so far. Indeed, \fbqsa has a linear time complexity, and is the fastest \lsa based solution for trajectory compression.
}
%
%We next review the optimized online algorithms \bqsa and \squishe evaluated in out experiments.

%\vspace{-0.5ex}
%\subsubsection{Bounded Quadrant System Using \ped}

\eat{%%%% opening windows
\stitle{{Algorithm \opwa \cite{Meratnia:Spatiotemporal}}.}
{The \opwa algorithm~\cite{Meratnia:Spatiotemporal} combines the Top-down and opening window strategies, and enforces the constrained global checking in the window.}

Given a trajectory $\dddot{\mathcal{T}}[P_0, \ldots, P_n]$ and an error bound $\epsilon$, algorithm \opwa~\cite{Meratnia:Spatiotemporal} maintains a window $W[P_s, \ldots, P_k]$, where $P_s$ and $P_k$ are the start and end points, respectively. Initially, $P_s$ = $P_0$ and $P_k$ = $P_1$, and the window $W$ is gradually expanded by adding new points one by one. \opwa tries to compress all points in $W[P_s, \ldots, P_k]$ to a single line segment $\mathcal{L}(P_{s}, P_{k})$. If the distances $ped(P_i, {\mathcal{L}})\le \epsilon$ for all points $P_i$ ($i\in[s, k]$), it simply expands $W$ to $[P_s, \ldots, P_k, P_{k+1}]$ $(k+1\le n)$ by adding a new point $P_{k+1}$. Otherwise, it produces a new line segment $\mathcal{L}(P_{s}, P_{k-1})$, and replaces $W$ with a new window $[P_{k-1},\ldots,P_{k+1}]$. The above process repeats until all points in $\dddot{\mathcal{T}}$ have been considered.
%
%\textcolor[rgb]{0.00,0.07,1.00}{According to the different methods of selecting the end points of a line segment, Open Window can further be divided into Normal Penning Window and Before Opening Window~\cite{Meratnia:Spatiotemporal}. When the distance of the point to compressed trajectory exceeds a certain threshold, Normal Opening Window algorithm select that point as the end point, while Before Opening Window select the last point within the window as the end point of the current trajectory.}
%
Algorithm \opwa is not efficient enough for compressing long trajectories as it remains in $O(n^2)$ time, the same as the \dpa algorithm.
%Also, \ped and \sed are both supported in \opwa as the algorithm \dpa does.
}%%%%%Opening windows


\stitle{Algorithm BQS Using \ped \cite{Liu:BQS}}.
It is essentially an efficiency optimized \opwa algorithm \cite{Meratnia:Spatiotemporal}, and reduces the running time by introducing convex hulls to pick out a certain number of points, which makes it specific for \ped.

For a buffer $W$ with sub-trajectory $[P_s, \ldots, P_k]$, it splits the space into four quadrants. A buffer here is similar to a window in \opwa \cite{Meratnia:Spatiotemporal}. For each quadrant, a rectangular bounding box is firstly created using the least and highest $x$ and $y$ values among points $\{P_s,\ldots,P_k\}$, respectively. Then another two bounding lines connecting points $P_s$ and $P_{h}$ and points $P_s$ and $P_{l}$ are created such that lines $\vv{P_sP_{h}}$ and $\vv{P_sP_{l}}$ have the largest and smallest angles with the $x$-axis, respectively.
Here $P_{h},P_{l} \in\{P_s,\ldots,P_k\}$. The bounding box and the two lines together form a convex hull.
Each time a new point $P_k$ is added to buffer $W$, \bqsa first picks out at most eight significant points from the convex hull in a quadrant. It calculates the distances of the significant points to line $\vv{P_sP_k}$, among which the largest distance $d_{u}$ and the smallest distance $d_l$ are an upper bound and  a lower bound of the distances of all points in $[P_s, \ldots, P_k]$ to line $\vv{P_sP_k}$.
(1) If $d_l\ge \epsilon$, it produces a new line segment $\mathcal{L}(P_{s}, P_{k-1})$, and produces a new window $[P_{k-1},\ldots,P_{k}]$ to replace $W$.
(2) If $d_u < \epsilon$, it simply expands buffer $W$ to $[P_s, \ldots, P_k, P_{k+1}]$ $(k+1\le n)$ by adding a new point $P_{k+1}$.
(3) Otherwise, it computes all distances $d(P_i, {\mathcal{L}(P_s,P_k)})$ ($i\in[s, k]$) as algorithm \dpa does.
%
The time complexity of \bqsa remains $O(n^2)$. However, its simplified version \fbqsa has a linear time complexity by essentially avoiding case (3) to speed up the process.
%The performance of \fbqsa has already been evaluated in work \cite{Lin:Operb}. %our preview work


\begin{example}
	\label{exm-alg-bqs}
	Figure~\ref{fig:bqs} is an example of \bqsa. The bounding box $c_1c_2c_3c_4$ and the two lines $\vv{P_sP_{h}} = \vv{P_0P_1}$ and $\vv{P_sP_{l}} = \vv{P_0P_2}$ form a convex hull $u_1u_2c_2l_2l_1c_4$. \bqsa computes the distances of $u_1,u_2,c_2,l_2,l_1$ and $c_4$ to line $\vv{P_0P_6}$ when $k=6$ or to line $\vv{P_0P_7}$ when $k=7$.
	%
	When $k=6$, all these distances to $\vv{P_0P_6}$  are less than $\epsilon$, hence \bqsa goes on to the next point (case 2); When $k=7$,
	the max and min distances to $\vv{P_0P_7}$ are larger and less than $\epsilon$, respectively, and \bqsa needs to compress sub-trajectory $[P_0, \ldots, P_7]$ along the same line as \dpa (case 3).
\end{example}

\begin{figure}[tb!]
	%\vspace{-1ex}
	\centering
	\includegraphics[scale = 0.66]{Figures/Fig-BQS.png}
	\vspace{-1ex}
	\caption{{\small Examples for algorithm \bqsa.}}
	\label{fig:bqs}
	\vspace{-2ex}
\end{figure}


\stitle{Algorithm SQUISH-E Using \sed~\cite{Muckell:Compression}}.
It is a bottom-up algorithm with a buffer, and has two forms: \squishe($\lambda$) ensuring the compression ratio $\lambda$, and \squishe($\epsilon$) ensuring the \sed error bound $\epsilon$. In this study, we  use \squishe($\epsilon$), as we focus on error bounded trajectory simplification.

\eat{%%%%%%%%%%%%%%%%
taking as input a trajectory \trajec{T} and two additional parameters $\lambda$ and $\epsilon$.
It first compresses trajectory \trajec{T} while striving to minimize \sed error and achieving the compression ratio of $\lambda$. Then, it further compresses \trajec{T} as long as this compression will not increase the max \sed error beyond $\epsilon$.

Meanwhile, \squishe($\lambda$) is the case where $\epsilon$ is set to $0$ and therefore it minimizes \sed error ensuring the compression ratio of $\lambda$, and
\squishe($\epsilon$) denotes another case, \ie the \emph{min-$\#$ problem}, where $\lambda$ is set to $1$ and therefore it maximizes compression ratio while keeping \sed error under $\epsilon$.
In this paper, we only discuss \squishe($\epsilon$).
}%%%%%%%%%%%%%%%%%%%%

Algorithm \squishe  optimizes algorithm \tpa with a doubly linked list $Q$. Each node in the list is a tuple $P(pre, suc, prio, mnprio)$, where $P$ is a trajectory data point, $pre$ and $suc$ are the predecessive  and successive points of $P$, respectively,  $prio$ is the priority of $P$ defined as an upper bound of the \sed error that the removal of $P$ introduces, and $mnprio$ is the max priority of its predecessive and successive points removed from the list.
%
Initially, trajectory points are loaded to $Q$ one by one.
At the same time, $mnprio$ of each point is set to $zero$ as no node has been removed from the list.
Moreover, the priorities of points $P_0$ and $P_{|Q|-1}$ are set to $\infty$, and the priority of point $P_i$ ($0<i<|Q|-1$) is set to $sed(P_i, \vv{pre(P_{i})suc(P_{i})})$.
%
Then, \squishe finds and removes a point $P_j$ from $Q$ that has the lowest priority $prio(P_j)<\epsilon$, and the properties $mnprio$ of predecessor $pre(P_j)$ and successor $suc(P_j)$ are updated to $\max(mnprio(pre(P_j)), prio(P_j))$ and $\max(mnprio(suc(P_j)), prio(P_j))$, respectively.
Next, the properties $prio$ of $pre(P_j)$ and $suc(P_j)$ are further updated to $mnprio(pre(P_j))$ + $sed(pre(P_j), \vv{pre(pre(P_{j}))suc(P_{j})})$ and $mnprio(suc(P_j))$ + $sed(suc(P_j),\vv{pre(P_{j})suc(suc(P_{j}))})$, respectively.
%
After that, a new point is read to the list and the information of its predecessor in the list is updated.
%
The above process repeated until that no points have a priority smaller than $\epsilon$. % \ie  the \sed up bound
%
\squishe finds and removes a point from $Q$ that has the lowest priority in $O(\log |Q|)$ time, where $|Q|$ denotes the number of points stored in $Q$.
Thus, \squishe runs in $O(n\log |Q|)$ time and $O(|Q|)$ space.
%And \squishe only supports \sed.


\begin{figure*}[tb!]
	\centering
	\includegraphics[scale=0.40]{Figures/Fig-Squishe.png}
	\vspace{-3ex}
	\caption{\small The trajectory $\dddot{\mathcal{T}}[P_0, \ldots, P_{10}]$ is compressed by the \squishe algorithm using \sed to five line segments. The size of Q is 6, and the data structure after point $P$ is a tuple $(pre, suc, mmprio, prio)$. }
	\vspace{-1ex}
	\label{fig:squishe}
\end{figure*}





\begin{example}
	\label{exm-alg-squishe}
	Figure~\ref{fig:squishe} is an example of \squishe.
	%
	(1) Initially, $|Q| = 6$ points are read to the list. The tuple $(pre, suc, mmprio, prio)$ for each point is initialized. For example, the tuple of $P_1$ is set to $(0, 2, 0, 0.42\epsilon)$, where $0.42\epsilon$ is the \sed from $P_1$ to $\vv{P_0P_2}$.
	%
	(2) The priority of $P_3$ has the minimal value, thus, it is removed from the list.
	The $mnprio$ properties of $P_2$ and $P_4$ are updated to $max\{mnprio(pre(P_3)), prio(P_3)\}$ = $max\{mnprio(P_2), prio(P_3)\}$ = $max\{0, 0.39\epsilon\}$ = $0.39\epsilon$, and $max\{mnprio(P_4), ~prio(P_3)\}$ = $0.39\epsilon$, respectively.
	Furthermore, the $prio$ property of $P_4$ is updated to $mnprio(suc(P_j)) + sed(suc(P_j),\vv{pre(P_{j})suc(suc(P_{j}))})$ = $mnprio(P_4) + sed(P_4,\vv{P_2P_5})$ = $0.39\epsilon + 2.50\epsilon$ = $2.89\epsilon$, and the $prio$ property of $P_2$ is updated to $mnprio(P_2) + sed(P_2,\vv{P_1P_4})$ = $0.39\epsilon + 2.12\epsilon$ = $2.51\epsilon$.
	Then, $P_6$ is read, and the information of $P_5$ is updated.
	%
	(3) $P_5$ is removed and $P_7$ is read to the list.
	%
	(4) Finally, the algorithm outputs 5 line segments $\vv{P_0P_2},\vv{P_2P_4},\vv{P_4P_7},\vv{P_7P_9}$ and $\vv{P_9P_{10}}$.
\end{example}


%%%%%%%%%%%%%%%%%%%%%%%%%%%%%%%%%%%%%%%%%%%%%%%%%%%%%%%%%%%%%%%%%%%%% END %%%%%%%%%%%%%%%%%%%%%%%%%%%%%%%%%%%%%%%%%%%%%%%%%%%%%%%%%%%%%%%%%%%%%%%%%%



\eat{%%%%%%%%%%%%%%%%%%%%%%%%%%%%%%%%%%%%%%%%%%%%%%%%%%%%%%%

\subsubsection{Opening Window and Top-down}

The \opwa algorithm~\cite{Meratnia:Spatiotemporal} combines the Top-down and opening window strategies, and enforces the constrained global checking in the window.

Given a trajectory $\dddot{\mathcal{T}}[P_0, \ldots, P_n]$ and an error bound $\epsilon$, algorithm \opwa~\cite{Meratnia:Spatiotemporal} maintains a window $W[P_s, \ldots, P_k]$, where $P_s$ and $P_k$ are the start and end points, respectively. Initially, $P_s$ = $P_0$ and $P_k$ = $P_1$, and the window $W$ is gradually expanded by adding new points one by one. \opwa tries to compress all points in $W[P_s, \ldots, P_k]$ to a single line segment $\mathcal{L}(P_{s}, P_{k})$. If the distances $ped(P_i, {\mathcal{L}})\le \epsilon$ for all points $P_i$ ($i\in[s, k]$), it simply expands $W$ to $[P_s, \ldots, P_k, P_{k+1}]$ $(k+1\le n)$ by adding a new point $P_{k+1}$. Otherwise, it produces a new line segment $\mathcal{L}(P_{s}, P_{k-1})$, and replaces $W$ with a new window $[P_{k-1},\ldots,P_{k+1}]$. The above process repeats until all points in $\dddot{\mathcal{T}}$ have been considered.
%
%\textcolor[rgb]{0.00,0.07,1.00}{According to the different methods of selecting the end points of a line segment, Open Window can further be divided into Normal Penning Window and Before Opening Window~\cite{Meratnia:Spatiotemporal}. When the distance of the point to compressed trajectory exceeds a certain threshold, Normal Opening Window algorithm select that point as the end point, while Before Opening Window select the last point within the window as the end point of the current trajectory.}
%
Algorithm \opwa is not efficient enough for compressing long trajectories as it remains in $O(n^2)$ time, the same as the \dpa algorithm.
Also, \ped and \sed are both supported in \opwa as the algorithm \dpa does.


\subsubsection{Sliding Window and Bottom-up}

The \swab algorithm~\cite{Keogh:online} is essentially the combination of the Sliding Window mechanism and the Bottom-up algorithm.
It keeps a window, $w[P_s, \ldots, P_{s+k-1}]$, of a fixed size of $k$.
The window size $k$ should be carefully chosen so that there are enough data points in the window to create about 5 or 6 line segments \cite{Keogh:online}.
Initially, $P_s=P_0$.
Next, the Bottom-Up algorithm, \eg \pavlidis algorithm, is applied to the points in the window, which merges the points into segments with the left-most segment being $\vv{P_sP_{s+i}}$, $i<k$.
Then $\vv{P_sP_{s+i}}$ is output, the window slides to right taking $P_{s+i+1}$ as the new start point of the window, and the Bottom-Up algorithm is applied again.
This process repeated until all points have been merged to segments.

The time complexity of \swab is a small constant factor worse than that of the standard Bottom-Up algorithm~\cite{Keogh:online}.
Also, it supports \sed. % as the standard Bottom-Up algorithm does.

\textcolor[rgb]{1.00,0.00,0.00}{Todo...dis-continuous line segments.}

} %%%%%%%%%%%%%%%%%%%%%%%%%%%%%%%%%%%%%% End of eat

