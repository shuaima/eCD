
\textcolor{blue}{\section{Aging of trajectories}}

In this section, we discuss the data aging problem of \lsa algorithms and distance metrics.
%
Suppose we first compressed a trajectory $\dddot{\mathcal{T}_0}$ using an \lsa algorithm $\mathcal{A}$ with error bound $\epsilon_1$ to $\overline{\mathcal{T}}_1$, then we need a coarser trajectory produced from $\overline{\mathcal{T}}_1$  having error bound $\epsilon_1 + \epsilon_2$ corresponding to $\dddot{\mathcal{T}}_0$, can we get that by compressing $\overline{\mathcal{T}}_1$ with error bound $\epsilon_2$?  \emph{Aging friendly \cite{Cao:Spatio}} and \emph{Aging feasible} are concepts to answer this question. Note that \emph{aging friendly} was introduced in \cite{Cao:Spatio} and seldom discussed in other works, while \emph{aging feasible} is a new concept.


\subsection{Aging Friendly}
	
	
\stitle{Aging friendly \cite{Cao:Spatio}}. {A \lsa algorithm $\mathcal{A}$ is aging friendly with respect to a distance metric $D$ if for every $\epsilon_1$ and every $\epsilon_2$ such that $\epsilon_1 < \epsilon_2$, and for every trajectory $\dddot{\mathcal{T}}$, $\mathcal{A}(\dddot{\mathcal{T}}, \epsilon_2, D)= \mathcal{A}(\mathcal{A}(\dddot{\mathcal{T}}, \epsilon_1, D), \epsilon_2, D)$.}

	
Authors in \cite{Cao:Spatio} also discussed the aging friendliness of algorithms \opt and \dpa with respect to distance metrics \ped and \sed. It proved that algorithm \dpa is aging friendly \wrt~\ped and \sed, but algorithm \opt is not. 
Though \cite{Cao:Spatio} did not discuss \dad, it is easy to find that algorithm \dpa is aging friendly \wrt~\dad and algorithm \opt is still not. Next we discuss the aging friendliness of other algorithms.


\begin{figure}[tb!]
	\centering
	\includegraphics[scale=0.66]{Figures/Fig-aging-pavlidis.png}

	\caption{\small A counter example of the aging friendliness of algorithm \tpa, where (1) the original trajectory is compressed using $\epsilon_2=6$ to three points $\{P_0, P_3, P_4\}$; (2) the original trajectory is first compressed using $\epsilon_1=3$ to four points $\{P_0, P_1, P_3, P_4\}$, then compressed using $\epsilon_2=6$ to two points $\{P_0, P_4\}$. }
	\vspace{-1ex}
	\label{fig:aging-pavlidis}
\end{figure}


\begin{theorem}
\label{theo-aging-tp}
The bottom-up algorithm \tpa is not aging friendly \wrt~\ped, \sed or \dad.
\end{theorem}

\begin{proof}
We will prove the theorem by construction.
As shown in Figure~\ref{fig:aging-pavlidis}, we let error bounds $\epsilon_1 =3$ and $\epsilon_2=6$, and without losing generality, we use \ped as the distance metric.
%The proof is easy to be extended to \sed and \dad.

\underline{(1) ${\tpa}(\dddot{\mathcal{T}}, 6, \ped)$}. In the first step, it merges $\overline{P_1P_2}$ and $\overline{P_2P_3}$ to $\overline{P_1P_3}$ as the merging of them has the lowest cost of $2$, the distance from point $P_2$ to line segment $\overline{P_1P_3}$; then it merges $\overline{P_0P_1}$ and $\overline{P_1P_3}$ to $\overline{P_0P_3}$ with the lowest cost of $4.5$, the distance from point $P_1$ to line segment $\overline{P_0P_3}$; at last, because the merging of $\overline{P_0P_3}$ and $\overline{P_3P_4}$ has a cost of $7$, the distance from point $P_2$ to line segment $\overline{P_0P_4}$, which is larger than the error bound of $6$, it outputs three points $\{P_0, P_3, P_4\}$.

\underline{(2) ${\tpa}(\tpa(\dddot{\mathcal{T}}, 3, \ped), 6, \ped)$}. In the first round ($\epsilon_1=3$), the original trajectory is compressed to four points $\{P_0, P_1, P_3, P_4\}$, and in the second round ($\epsilon_1=6$), because all points in the result trajectory $\{P_0, P_1, P_3, P_4\}$ have distances to line segment $P_0P_4$ less than $6$, it is finally compressed to two points $\{P_0, P_4\}$.

In this case, ${\tpa}(\dddot{\mathcal{T}}, 6, \ped) \ne {\tpa}(\tpa(\dddot{\mathcal{T}}, 3, \ped), 6, \ped)$. Thus, the \tpa algorithm is not aging friendly \wrt~\ped.
\textcolor{blue}{Similarly, we can prove that \tpa is not aging friendly \wrt~ \sed or \dad. Hence, we have the conclusion.}
\end{proof}

\begin{lemma}
	\label{theo-aging-tp}
	The online algorithm \squishe is not aging friendly \wrt~\sed.
\end{lemma}

\begin{proof}
	\textcolor{blue}{(In brief) The \squishe algorithm runs in a bottom-up manner that is a slight variation of algorithm \tpa. As \tpa, it is not aging friendly \wrt~\sed.}
\end{proof}

	
\begin{figure}[tb!]
	\centering
	\includegraphics[scale=0.66]{Figures/Fig-aging-incre.png}
	
	\caption{\small Counter examples of the aging friendliness of incremental algorithms (either online or one-pass).}
	\vspace{-2ex}
	\label{fig:aging-incre}
\end{figure}

\begin{theorem}
	\label{theo-aging-online}
	The online and one-pass algorithms are not aging friendly \wrt~\ped, \sed or \dad.

\end{theorem}

\begin{proof}
	For online algorithm \squishe, it only supports \sed and it is not aging friendly \wrt~\sed.
	For other online and one-pass algorithms, though they apply different distance checking approaches, they run in a common incremental manner, \ie they incrementally read data points until they can not represent those read points by one line segment, then they output the simplified sub-trajectory and continue to process the rest data points. We next construct counter examples to show an incremental algorithm $\mathcal{A}$ is not aging friendly.
	
	\underline{(1) ${\mathcal{A}}(\dddot{\mathcal{T}}, \epsilon_2, D)$}. As shown in Figure~\ref{fig:aging-incre}-(1)(3)(5), the algorithm $\mathcal{A}$ incrementally reads $\{P_0, P_1,\dddot, P_5\}$ and finds they can be represented by line segment $\overline{P_0P_4}$, thus the process progresses. Then, after point $P_6$ is read, it finds that these points can not be represented by any line segment, hence $\overline{P_0P_5}$ is output. Finally, the algorithm outputs $\{P_0, P_5, P_6\}$.
	
	\underline{(2) ${\mathcal{A}}(\mathcal{A}(\dddot{\mathcal{T}}, \epsilon_1, D), \epsilon_2, D)$}. When using $\epsilon_1=4$ or $\epsilon_1=30^o $, the algorithm also outputs $\{P_0, P_5, P_6\}$. Then $\{P_0, P_5, P_6\}$ is compressed using $\epsilon_2=6$ or $\epsilon_2=45^o$ to $\{P_0, P_6\}$, as shown in Figure~\ref{fig:aging-incre}-(2)(4)(6).
	
	Combining (1) and (2), it is clear that the incremental algorithm is not aging friendly \wrt~distance metric \ped, \sed or \dad.
\end{proof}

From the above we can find that only algorithm \dpa is aging friendly, others are not. Indeed, {aging friendly} is the result of the \textcolor{blue}{specific nature}, \ie top-down and batch, of algorithm \dpa.
% pretty and rare feature of


\subsection{Aging Feasible}
In this section, we define and discuss a widely existing, maybe more practical, feature of \lsa algorithms and distance metrics.

\stitle{Aging feasible}. {A \lsa algorithm $\mathcal{A}$ is aging feasible with respect to a distance metric $D$ if for every $\epsilon_1$ and every $\epsilon_2$, and for every trajectory $\dddot{\mathcal{T}}$, the max error between $\dddot{\mathcal{T}}$ and $\overline{\mathcal{T}}=\mathcal{A}(\mathcal{A}(\dddot{\mathcal{T}}, \epsilon_1, D), \epsilon_2, D)$ is not more than $\epsilon_1+ \epsilon_2$.}


