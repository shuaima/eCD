
\textcolor{blue}{\section{Aging of trajectories}}
\label{sec-aging}

Suppose we first compressed a trajectory $\dddot{\mathcal{T}_0}$ using an \lsa algorithm $\mathcal{A}$ with error bound $\epsilon_1$ to $\overline{\mathcal{T}}_1$, then as time progresses, we need a coarser trajectory $\overline{\mathcal{T}}_2$ derived from $\overline{\mathcal{T}}_1$. 
What is the relationship among $\overline{\mathcal{T}}_2$, $\overline{\mathcal{T}}_1$ and $\dddot{\mathcal{T}_0}$? And what is the right way to get the coarser trajectory $\overline{\mathcal{T}}_2$?
%
This section is to answer these questions from the views of friendliness \cite{Cao:Spatio} and error.
Note that if we get $\overline{\mathcal{T}}_1$ and $\overline{\mathcal{T}}_2$ by optimal algorithms \wrt error bounds $\epsilon_1$ and $\epsilon_2$ in turn, then $\overline{\mathcal{T}}_2$ may not be optimal of $\dddot{\mathcal{T}_0}$ \wrt any error bound \cite{Cao:Spatio}. Thus, in this section, we let alone the optimal algorithms.

\subsection{Friendliness of Data Aging}
We first discuss the relationship between $\overline{\mathcal{T}}_1$ and $\overline{\mathcal{T}}_2$ from a view of friendliness which was defined in \cite{Cao:Spatio} but seldom discussed in other works.
	
\stitle{Aging friendly \cite{Cao:Spatio}}. {A \lsa algorithm $\mathcal{A}$ is aging friendly with respect to a distance metric $\mathcal{D}$ if for every $\epsilon_1$ and every $\epsilon_2$ such that $\epsilon_1 < \epsilon_2$, and for every trajectory $\dddot{\mathcal{T}}$, $\mathcal{A}(\dddot{\mathcal{T}}, \epsilon_2, \mathcal{D})= \mathcal{A}(\mathcal{A}(\dddot{\mathcal{T}}, \epsilon_1, \mathcal{D}), \epsilon_2, \mathcal{D})$.}

Cao et al. proved in \cite{Cao:Spatio} that ``the top-down algorithm \dpa is aging friendly \wrt~\ped and \sed" on the premise that the second run of \dpa takes as input the whole simplified trajectory produced by the first run. 
In spite of that, algorithms other than \dpa and distance metrics other than \ped and \sed are not discussed in \cite{Cao:Spatio}, thus, their effectiveness in data aging remains an open problem. 
In the rest of the section, we will fully discuss this problem, start from \dpa coupling with \dad.
%And, when \dpa is coupling with \dad, things are changed.


\begin{proposition}
	\label{theo-aging-dp-dad}
	The top-down algorithm \dpa is not aging friendly \wrt~\dad.
\end{proposition}

\begin{proof}
	We will prove the proposition by construction.
	As shown in Figure~\ref{fig:aging-dp-dad}, we let error bounds $\epsilon_1 =30^o$ and $\epsilon_2=45^o$.
	
	\underline{(1) ${\dpa}(\dddot{\mathcal{T}}, 45^o, \dad)$}. It finds that line segment $\overline{P_1P_2}$ has the largest angular deviation to line segment $\overline{P_0P_4}$ which is also larger than the error bound $45^o$, hence it uses point $P_2$ as the splitting point and splits the original trajectory to $\{P_0, P_1, P_2\}$ and $\{P_2, P_3, P_4\}$. At last, it outputs three points $\{P_0, P_2, P_4\}$.
	
	\underline{(2) ${\dpa}(\dpa(\dddot{\mathcal{T}}, 30^o, \dad), 45^o, \dad)$}. In the first round ($\epsilon_1=30^o$), the original trajectory is compressed to three points $\{P_0, P_2, P_4\}$, and in the second round ($\epsilon_2=45^o$), because line segments  $\overline{P_0P_2}$ and $\overline{P_2P_4}$ both have angular deviations to line segment $\overline{P_0P_4}$ less than $45^o$, it is finally compressed to two points $\{P_0, P_4\}$.
	
	In this case, ${\dpa}(\dddot{\mathcal{T}}, 45^o, \dad) \ne {\dpa}(\dpa(\dddot{\mathcal{T}}, 30^o, \dad), 45^o, \dad)$. Thus, the \dpa algorithm is not aging friendly \wrt~\dad.
\end{proof}

\begin{figure}
	\centering
	\includegraphics[scale=0.66]{Figures/Fig-aging-dp.png}
	
	\caption{\small A counter example of the aging friendliness of \dpa using \dad, where (1) the original trajectory is compressed using $\epsilon_2=45^o$ to three points $\{P_0, P_2, P_4\}$, and (2) the original trajectory is first compressed using $\epsilon_1=30^o$ to three points $\{P_0, P_2, P_4\}$, then compressed using $\epsilon_2=45^o$ to two points $\{P_0, P_4\}$. }
	\vspace{-1ex}
	\label{fig:aging-dp-dad}
\end{figure}

The \dpa using \dad is different with \ped and \sed in that \dad is the deviation between two line segments rather than the deviation between a point and a line segment. For example, in Figure \ref{fig:aging-dp-dad}-(2), the angular deviations of line segments $\overline{P_1P_2}$ and $\overline{P_0P_2}$ to line segment $\overline{P_0P_4}$ are certainly different though they pass through the same point $P_2$, thus, in the first round of run ($\epsilon_1=30^o$), the point $P_2$ serves as a splitting point while in the second round of run ($\epsilon_2=45^o$), it is no more a splitting point. This difference is the key that lets \dpa using \dad not aging friendly.
%
We next discuss the aging friendliness of other algorithms.

\begin{figure}
	\centering
	\includegraphics[scale=0.66]{Figures/Fig-aging-pavlidis.png}

	\caption{\small A counter example of the aging friendliness of algorithm \tpa, where (1) the original trajectory is compressed using $\epsilon_2=6$ to three points $\{P_0, P_3, P_4\}$, and (2) the original trajectory is first compressed using $\epsilon_1=3$ to four points $\{P_0, P_1, P_3, P_4\}$, then compressed using $\epsilon_2=6$ to two points $\{P_0, P_4\}$. }
	\vspace{-1ex}
	\label{fig:aging-pavlidis}
\end{figure}


\begin{proposition}
\label{theo-aging-tp}
The bottom-up algorithm \tpa is not aging friendly \wrt~\ped, \sed or \dad.
\end{proposition}

\begin{proof}
We will prove the proposition by construction.
As shown in Figure~\ref{fig:aging-pavlidis}, we let error bounds $\epsilon_1 =3$ and $\epsilon_2=6$, and without losing generality, we use \ped as the distance metric.

\underline{(1) ${\tpa}(\dddot{\mathcal{T}}, 6, \ped)$}. It first merges $\overline{P_1P_2}$ and $\overline{P_2P_3}$ to $\overline{P_1P_3}$ as the merging of them has the lowest cost of $2$, the distance from point $P_2$ to line segment $\overline{P_1P_3}$; then it merges $\overline{P_0P_1}$ and $\overline{P_1P_3}$ to $\overline{P_0P_3}$ with the lowest cost of $4.5$, the distance from point $P_1$ to line segment $\overline{P_0P_3}$; at last, because the merging of $\overline{P_0P_3}$ and $\overline{P_3P_4}$ has a cost of $7$, the distance from point $P_2$ to line segment $\overline{P_0P_4}$, which is larger than the error bound of $6$, it outputs three points $\{P_0, P_3, P_4\}$.

\underline{(2) ${\tpa}(\tpa(\dddot{\mathcal{T}}, 3, \ped), 6, \ped)$}. In the first round ($\epsilon_1=3$), the original trajectory is compressed to four points $\{P_0, P_1, P_3, P_4\}$, and in the second round ($\epsilon_2=6$), because all points in the result trajectory $\{P_0, P_1, P_3, P_4\}$ have distances to line segment $P_0P_4$ less than $6$, it is finally compressed to two points $\{P_0, P_4\}$.

In this case, ${\tpa}(\dddot{\mathcal{T}}, 6, \ped) \ne {\tpa}(\tpa(\dddot{\mathcal{T}}, 3, \ped), 6, \ped)$. Thus, \tpa is not aging friendly \wrt~\ped.
\textcolor{blue}{Similarly, \tpa is not aging friendly \wrt~ \sed or \dad. Hence, we have the conclusion.}
\end{proof}

\begin{proposition}
	\label{theo-aging-squishe}
	The online algorithm \squishe is not aging friendly \wrt~\sed.
\end{proposition}

\begin{proof}
	\textcolor{blue}{(In brief) The \squishe algorithm runs in a bottom-up manner that is a slight variation of algorithm \tpa. As \tpa, it is not aging friendly \wrt~\sed.}
\end{proof}

	
\begin{figure}[tb!]
	\centering
	\includegraphics[scale=0.66]{Figures/Fig-aging-incre.png}
	\vspace{-1ex}
	\caption{\small Counter examples of the aging friendliness of incremental algorithms (either online or one-pass).}
	\vspace{-1ex}
	\label{fig:aging-incre}
\end{figure}

\begin{proposition}
	\label{theo-aging-online}
	The online and one-pass algorithms are not aging friendly \wrt~\ped, \sed or \dad.

\end{proposition}

\begin{proof}
	For online algorithm \squishe, it only supports \sed and it is not aging friendly \wrt~\sed.
	For other online and one-pass algorithms, though they apply different distance checking approaches, they run in a common incremental manner, \ie they incrementally read data points until they can not represent those read points by one line segment, then they output the simplified sub-trajectory and continue to process the rest data points. We next construct counter examples to show that an incremental algorithm $\mathcal{A}$ is not aging friendly.
	
	\underline{(1) ${\mathcal{A}}(\dddot{\mathcal{T}}, \epsilon_2, \mathcal{D})$}. As shown in Figure~\ref{fig:aging-incre}-(1)(3)(5), the algorithm $\mathcal{A}$ incrementally reads $\{P_0, P_1,\dddot, P_5\}$ and finds they can be represented by line segment $\overline{P_0P_4}$, thus the process progresses. Then, after point $P_6$ is read, it finds that these points can not be represented by any line segment, hence $\overline{P_0P_5}$ is output. Finally, the algorithm outputs $\{P_0, P_5, P_6\}$.
	
	\underline{(2) ${\mathcal{A}}(\mathcal{A}(\dddot{\mathcal{T}}, \epsilon_1, \mathcal{D}), \epsilon_2, \mathcal{D})$}. When using $\epsilon_1=4$ or $\epsilon_1=30^o $, the algorithm also outputs $\{P_0, P_5, P_6\}$. Then $\{P_0, P_5, P_6\}$ is compressed using $\epsilon_2=6$ or $\epsilon_2=45^o$ to $\{P_0, P_6\}$, as shown in Figure~\ref{fig:aging-incre}-(2)(4)(6).
	
	Combining (1) and (2), it is clear that the incremental algorithms are not aging friendly \wrt~distance metric \ped, \sed or \dad.
\end{proof}



\subsection{Error of Data Aging}
As the most algorithms are not aging friendly, are they error bounded in data aging? 
if so, what are the bounds?
This section focuses on these problems and discusses the error between the simplified trajectory $\overline{\mathcal{T}}_2$ and the original trajectory $\dddot{\mathcal{T}_0}$.

\begin{proposition}
	\label{theo-aging-error-dp}
	Given error bounds $\epsilon_1>0$ and $\epsilon_2>0$, for every distance metric $\mathcal{D}$ of \ped and \sed, the error bound between original trajectory $\dddot{\mathcal{T}}$ and simplified trajectory $\overline{\mathcal{T}}=DP(DP(\dddot{\mathcal{T}}, \epsilon_1, \mathcal{D}), \epsilon_2, \mathcal{D})$ is $max\{\epsilon_1, \epsilon_2\}$.
\end{proposition}

\begin{proof}
	For $\epsilon_2 \ge \epsilon_1$, as proved in \cite{Cao:Spatio}, $DP(DP(\dddot{\mathcal{T}}, \epsilon_1, \mathcal{D}), \epsilon_2, \mathcal{D}) = DP(\dddot{\mathcal{T}}, \epsilon_2, \mathcal{D})$, which has the max error of $\epsilon_2$ to the original trajectory $\dddot{\mathcal{T}}$.
	
	For $\epsilon_1 > \epsilon_2$, we will first prove $DP(DP(\dddot{\mathcal{T}}, \epsilon_1, \mathcal{D}), \epsilon_2, \mathcal{D}) = DP(\dddot{\mathcal{T}}, \epsilon_1, \mathcal{D})$ by total induction on the number of points of \trajec{T}.
	
	(1)  For a trajectory \trajec{T} with one or two points ($n=1$ or $n=2$), the simplified trajectories with any $\epsilon$ are sure identical to the original trajectory. 
	Consider a trajectory \trajec{T} =	$[P_0, P_1, P_2]$ ($n = 3$),
	if the distance from $P_1$ to $\overline{P_0P_2}$ is less than $\epsilon_1$, then $DP(\dddot{\mathcal{T}}, \epsilon_1, \mathcal{D}) = [P_0, P_2]$. Obviously $DP(DP(\dddot{\mathcal{T}}, \epsilon_1, \mathcal{D}), \epsilon_2, \mathcal{D}))$ is $[P_0, P_2]$ too;	 
	if the distance from $P_1$ to $\overline{P_0P_2}$ is larger than $\epsilon_1$, then $DP(\dddot{\mathcal{T}}, \epsilon_1, \mathcal{D})=\dddot{\mathcal{T}}$, and $DP(DP(\dddot{\mathcal{T}}, \epsilon_1, \mathcal{D}), \epsilon_2, \mathcal{D}) = DP(\dddot{\mathcal{T}}, \epsilon_2, \mathcal{D})=\dddot{\mathcal{T}}$.
	
	(2) Assume it is true for every trajectory \trajec{T} having $n$, $(n \ge 3)$, points.
	Consider a trajectory with $n+1$ points. Let $d_{max}$ denote the maximum distance between point $P_i$, $i \in [0,n]$, and the line segment $\overline{P_0P_{n}}$. 
	If $d_{max}<\epsilon_1$, then $DP(\dddot{\mathcal{T}}, \epsilon_1, \mathcal{D})=[P_0, P_{n}]$, and $DP(DP(\dddot{\mathcal{T}}, \epsilon_1, \mathcal{D}), \epsilon_2, \mathcal{D}) = DP([P_0, P_{n}], \epsilon_2, \mathcal{D})=[P_0, P_{n}]$.
	If $d_{max} > \epsilon_1$, then in $DP(\dddot{\mathcal{T}}, \epsilon_1, \mathcal{D})$, point $P_i$ will split the trajectory \trajec{T} into two sub-trajectories, \ie $[P_0, ..., P_i]$ and $[P_{i}, ..., P_{n}]$, and continue to simplify each sub-trajectories. Hence, the result of $DP(\dddot{\mathcal{T}}, \epsilon_1, \mathcal{D})$ is the union of $DP([P_0, ..., P_i], \epsilon_1, \mathcal{D})$ and $DP([P_i, ..., P_n], \epsilon_1, \mathcal{D})$.
	Obviously, the points $P_0$, $P_i$ and $P_n$ are in the simplified trajectory of $DP(\dddot{\mathcal{T}}, \epsilon_1, \mathcal{D})$, and $P_i$ is still the first spiting point of the \dpa algorithm taking the simplified trajectory and $\epsilon_2$ as input. Hence, $DP(DP(\dddot{\mathcal{T}}, \epsilon_1, \mathcal{D}), \epsilon_2, \mathcal{D})$ is the union of $DP(DP([P_0, ..., P_i], \epsilon_1, \mathcal{D}), \epsilon_2, \mathcal{D})$ and $DP(DP([P_i, ..., P_n], \epsilon_1, \mathcal{D}), \epsilon_2, \mathcal{D})$. By the assumption, we have $DP(DP([P_0, ..., P_i], \epsilon_1, \mathcal{D}), \epsilon_2, \mathcal{D}) = DP([P_0, ..., P_i], \epsilon_1, \mathcal{D})$ and $DP(DP([P_i, ..., P_n], \epsilon_1, \mathcal{D}), \epsilon_2, \mathcal{D}) = DP([P_i, ..., P_n], \epsilon_1, \mathcal{D})$. Thus, $DP(DP(\dddot{\mathcal{T}}, \epsilon_1, \mathcal{D}), \epsilon_2, \mathcal{D}) = DP(\dddot{\mathcal{T}}, \epsilon_1, \mathcal{D})$.
	
	Combining (1) and (2), we have $DP(DP(\dddot{\mathcal{T}}, \epsilon_1, \mathcal{D}), \epsilon_2, \mathcal{D}) = DP(\dddot{\mathcal{T}}, \epsilon_1, \mathcal{D})$, whose max error to the original trajectory $\dddot{\mathcal{T}}$ is $\epsilon_1$.
\end{proof}


\begin{proposition}
	\label{theo-aging-distance}
	Given error bounds $\epsilon_1>0$ and $\epsilon_2>0$, for any non-optimal \lsa algorithm $\mathcal{A}$ and distance metric $\mathcal{D}$ of \ped, \sed and \dad other than \dpa using \ped and \sed, the error bound between original trajectory $\dddot{\mathcal{T}}$ and simplified trajectory $\overline{\mathcal{T}}=\mathcal{A}(\mathcal{A}(\dddot{\mathcal{T}}, \epsilon_1, \mathcal{D}), \epsilon_2, \mathcal{D})$ is $\epsilon_1+ \epsilon_2$.
\end{proposition}

\begin{proof} 	
	(1) We first prove that the error bound between them is not more than $\epsilon_1+ \epsilon_2$. Suppose a point $P_k$ is represented by line segment $\overline{P_iP_j}$ with error bound $\epsilon_1$, and points $P_i$ and $P_j$ are further represented by line segment $\overline{P_sP_t}$ with error bound $\epsilon_2$ (Figure~\ref{fig:aging-error}).
	If the distance metric is \ped, then the distance from $P'_k$ to $\overline{P_sP_t}$ is less than $\epsilon_2$, hence, the distance from $P_k$ to $\overline{P_sP_t}$ is less than $\epsilon_1 + \epsilon_2$.
	If it is \sed, then $|P_iP''_i|<\epsilon_2$, $|P_jP''_j|<\epsilon_2$, and $\frac{|P_iP'_k|}{|P'_kP_j|} = \frac{|P''_iP''_k|}{|P''_kP''_j|}$, hence, $|P'_kP''_k|<\epsilon_2$, and the distance from $P_k$ to $\overline{P_sP_t}$, \ie $|P_kP''_k|$, is less than $\epsilon_1 + \epsilon_2$.
	If it is \dad, then obviously the error between $\overline{P_kP_{k+1}}$ and $\overline{P_sP_t}$ is not more than $\epsilon_1+ \epsilon_2$.
	
	(2) We then prove that the  error bound between them is not less than $\epsilon_1+ \epsilon_2$.
	If $\mathcal{A}$ is a top-down algorithm using \dad, then from Figures~\ref{fig:aging-dp-dad}-(2) we can find that the error from $\overline{P_3P_4}$ to $\overline{P_0P_4}$ is $\angle{P_3P_4P_0} = \angle{P_3P_4P_2} + \angle{P_2P_4P_0}$, whose bound is not less than $\epsilon_1 + \epsilon_2$, thus the error bound between original trajectory $\dddot{\mathcal{T}}$ and simplified trajectory $\overline{\mathcal{T}}=\mathcal{A}(\mathcal{A}(\dddot{\mathcal{T}}, \epsilon_1, \mathcal{D}), \epsilon_2, \mathcal{D})$ is not less than $\epsilon_1+ \epsilon_2$.
	If $\mathcal{A}$ is a bottom-up algorithm or an incremental algorithm, either online or one-pass, then from Figure~\ref{fig:aging-pavlidis}-(2) or Figure~\ref{fig:aging-incre} we also have the conclusion.
	
	
	Combining (1) and (2), we have the conclusion.
\end{proof}

\begin{figure}[tb!]
	\centering
	\includegraphics[scale=0.6]{Figures/Fig-aging-error.png}
	
	\caption{\small Examples of aging error.}
	\vspace{-2ex}
	\label{fig:aging-error}
\end{figure}