\section{Optimal Algorithms}
\label{sec-optimal}

This section reviews the optimal \lsa algorithms.
For the ``min-\#" problem that finds out the minimal number of points or segments to represent the original polygonal lines \wrt an error bound $\epsilon$, Imai and Iri \cite{Imai:Optimal} first formulated it as a graph problem, and showed that it could be solved in  $O(n^3)$ time, where $n$ is the number of the original points.
%
Toussaint of \cite{Toussaint:Optimal} and Melkman and O'Rourke of \cite{Melkman:Optimal} improved the time complexity to $O(n^2 \log n)$ by using either \textit{convex hull} or \textit{sector intersection} methods.
%
The authors of \cite{Chan:Optimal} presented an optimal algorithm using \ped (\optp) that achieves $O(n^2)$ time by using the \textit{sector intersection} mechanism (see Section~\ref{sec-siped}).
Because the \textit{sector intersection} and the \textit{convex hull} mechanisms can not work with \sed, hence, currently the time complexity of the optimal algorithm using \sed remains $O(n^3)$.
However, we argue that an optimal and a near optimal \lsa algorithms using \sed can achieve $O(n^2 \log n)$ time and $O(n^2)$ time, respectively, by applying the spatio-temporal cone intersection mechanism shown in Section~\ref{sec-cised}. % developed in our preview work \cised~\cite{Lin:Cised}. 



In the rest of this section, we first introduce the naive optimal (\opt) algorithm \cite{Imai:Optimal} that runs in $O(n^3)$ time. We then review a \ped specific optimal algorithm \optp \cite{Chan:Optimal} that achieves $O(n^2)$ time. Finally, we describe a \sed enabled near optimal algorithm \nopts that achieves $O(n^2)$ time and $O(n)$ space. 
Algorithms \optp and \nopts are evaluated in our experiments.

\subsection{The Naive Optimal Algorithm}
Given a trajectory \trajec{T}${[P_0, \ldots, P_n]}$ and an error bound $\epsilon$, the optimal trajectory simplification problem, as formulated by Imai and Iri in \cite{Imai:Optimal}, can be solved in two steps: (1) construct a reachability graph $G$ of \trajec{T} and (2) search a shortest path from $P_0$ to $P_{n}$ in graph $G$.

The reachability graph of a trajectory \trajec{T}${[P_0, \ldots, P_n]}$ \wrt an error bound $\epsilon$ is $G$ = ($V$, $E)$, where (1) $V = \{P_0, \ldots, P_n\}$, and (2) for any nodes $P_s$ and $P_{s+k} \in V$ ($s\ge 0, k>0, s+k\le n$), edge $(P_s, P_{s+k}) \in E$ if and only if the distance of each point $P_{s+i} (0<i<k)$ to line segment $\vv{P_sP_{s+k}}$ is no greater than $\epsilon$.
%
Observe that in the graph $G$, (1) a path from nodes $P_0$ to $P_{n}$ is a representation of trajectory \trajec{T}. The path also reveals the subset of points of \trajec{T} used in the approximate trajectory, (2) the path length corresponds to the number of line segments in the approximation trajectory, and 
(3) a shortest path is an optimal representation of trajectory \trajec{T}.

A straightforward way of constructing the reachability graph $G$ needs to check for all pair of points $P_s$ and $P_{s+k}$ whether the distances of all points ($P_{s+i}$, $0<i<k$) to the line segment $\vv{P_sP_{s+k}}$ are less than $\epsilon$.
There are $O(n^2)$ pairs of points in the trajectory and checking the error of all points $P_{s+i}$ to a line segment $\vv{P_sP_{s+k}}$ takes $O(n)$ time.
Thus, the construction step takes $O(n^3)$ time.
Finding a shortest path takes no more than $O(n^2)$ time. Hence, the straightforward algorithm, \ie~\opt, takes $O(n^3)$ time in total.
For space complexity, it needs $O(n^2)$ space.

\begin{figure}[tb!]
	\centering
	\includegraphics[scale=0.75]{Figures/Fig-Optimal.png}
	\vspace{-2ex}
	\caption{\small Example of reachability graph of trajectory \trajec{T}$[P_0, \ldots, P_n]$ whose shortest path is $(P_0, P_4, P_{10})$.}
	\vspace{-3ex}
	\label{fig:optimal}
\end{figure}

\begin{example}
	\label{exm-alg-optimal}
	Figure~\ref{fig:optimal} is an example of the naive optimal algorithm using \ped taking as input the trajectory \trajec{T} shown in Figure~\ref{fig:notations}. The reachability graph of \trajec{T} is constructed and a shortest path with 2 edges is founded. 
	At last, the algorithm outputs two line segments $\vv{P_0P_4}$ and $\vv{P_4P_{10}}$.	
\end{example}
\vspace{-1ex}
%Though the brute-force algorithm was initially developed using \ped, it can be used for \sed.
%We next introduce two improved optimal algorithms using \ped and \sed that run in $O(n^2)$ and $O(n^2 \log n)$ time, respectively, by fasting the construction of reachability graph $G$.

\subsection{The Optimal Algorithm using \ped}
Authors of \cite{Chan:Optimal} provide an optimal algorithm using \ped (\optp) that constructs the graph $G$ in $O(n^2)$ time, by using the \textit{sector intersection} mechanism (See Section~\ref{sec-siped}). 

Given a point $P_s$, $0 \le s < n$, \optp builds a sector taking $P_s$ as the start point and checks for all points $P_{s+k}$, $k>0 ~and~ s+k \le n$, whether $(P_s, P_k)$ can be included in graph $G$. It can be added to the graph if and only if $(\bigsqcap_{i=1}^{k-1}\mathcal{S}(P_s, P_{s+i}, \epsilon)) \bigsqcap \vv{P_sP_{s+k}} \ne \{P_s\}$, \ie line segment $\vv{P_sP_{s+k}}$ ~passing through the common intersection area of sectors $\mathcal{S}(P_s, P_{s+i}, \epsilon))$, $0<i<k$. 

There are $n$ points of $P_s$ in the trajectory and the checking of $(\bigsqcap_{i=1}^{k-1}\mathcal{S}(P_s, P_{s+i}, \epsilon)) \bigsqcap \vv{P_sP_{s+k}} \ne \{P_s\}$ for one $P_s$ takes $O(n)$ time.
Thus, the construction step takes $O(n^2)$ time and the algorithm takes $O(n^2)$ time in total.
For space complexity, this algorithm takes $O(n^2)$ space. However, it could be optimized to $O(n)$ space by the approach presented in \cite{Chen:Space}, whose fundamental idea is that it computes a shortest path in graph $G$ without having to maintain $G$ explicitly.




\eat{%%%%%%%%%%%%%%%%%%%%%%%%%%%%%%%%%%%%%%%
\subsection{The Optimal Algorithm using \sed}
The constructing of the reachability graph $G$ using \sed can be optimized, by using the \textit{spatio-temporal cone intersection} mechanism (See Section~\ref{sec-siped}). 

Given a point $P_s$, $0 \le s < n$, the optimal builds a cone taking $P_s$ as the start point and checks for all points $P_{s+k}$, $k>0 ~and~ s+k \le n$, whether $(P_s, P_k)$ can be included in graph $G$ by checking $(\bigsqcap_{i=1}^{k-1}\mathcal{C}(P_s, P_{s+i}, \epsilon)) \bigsqcap \vv{P_sP_{s+k}} \ne \{P_s\}$, \ie line segment $\vv{P_sP_{s+k}}$ ~passing through the common intersection area of cones $\mathcal{C}(P_s, P_{s+i}, \epsilon))$, $0<i<k$. 

The checking of intersection of cones is equal to check the intersection of circles on a plane \cite{Lin:Cised}, which is claimed of $O(n \log n)$ time for $n$ circles~\cite{Shamos:Circle}. 
There are $n$ points of $P_s$ in the trajectory and the checking for one $P_s$ takes $O(n \log n)$ time.
Thus, the construction step takes $O(n^2 \log n)$ time and the algorithm takes $O(n^2 \log n)$ time in total.

\stitle{Remark}.
If we approxite each circle by its $m$-edges inscribed regular polygon and approxite the intersection of $n$ circles on a plane by the intersection of their inscribed regular polygons, the way \cised does, then we get a near optimal graph $G'$ such that $\lim_{m \to \infty}{G'=G}$, and a \textit{near optimal} algorithm using \sed, \ie ~\nopts, that achieves $O(n^2)$ time. The space complexity of the near optimal algorithm also achieves $O(n)$ by combining the approach of \cite{Chen:Space}.
} 