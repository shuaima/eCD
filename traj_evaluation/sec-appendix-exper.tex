\vspace{-1ex}
\section*{Appendix B: Additional Experiments}
\label{sec-app-exp}


%\subsection*{B.1 The effectiveness of near optimal algorithm}
%\label{sec-app-exp-cr}
In this section, we compare the performance of the near optimal algorithm using \sed (\nopts) and the naive optimal algorithm (\opt) using \sed.
As the naive optimal algorithm has both high time and space complexities, \ie $O(n^3)$ time and $O(n^2)$ space, it is impossible to compress the entire datasets (too slow and out of memory). Hence, we build four \textit{small datasets} for this test, and each dataset includes $10$ middle-size ($10K$ points per trajectory) trajectories selected from \ucar, \geolife, \mopsi and \act, respectively.
The results are reported in Figure~\ref{fig:cr-nopts}. It shows that the
compression ratios of algorithm \nopts are on average ($100.22\%$, $100.22\%$,
$100.55\%$, $100.22\%$) of \opt using \sed on datasets (\ucar, \geolife, \mopsi, \act), respectively.

\begin{figure}[h]
	\vspace{-1.5ex}
	\centering
	\hspace{-2ex}
	\includegraphics[scale=0.2]{Figures/Exp-Append-NearOpt-CR-epsilon-service.png}	
	\includegraphics[scale=0.2]{Figures/Exp-Append-NearOpt-CR-epsilon-geolife.png}		
	\includegraphics[scale=0.2]{Figures/Exp-Append-NearOpt-CR-epsilon-mopsi.png}		
	\includegraphics[scale=0.2]{Figures/Exp-Append-NearOpt-CR-epsilon-zh.png}		\hspace{-2ex}
	\vspace{-2.5ex}
	\caption{\small Evaluation of compression ratios of the near optimal algorithm: varying the error bound $\epsilon$.}
	\label{fig:cr-nopts}
	\vspace{-4ex}
\end{figure}



\eat{%%%%%%%%%%%%%%%%%%%%%%%%%%%%%%%%%%%%%%%%%%%%%%%%%%%
\subsection*{B.2 The efficiencies of optimal algorithms}
\label{sec-app-exp-time}
We next test the efficiencies of the optimal algorithm using \ped (\optp) and the near optimal algorithm using \sed (\nopts).
The results are reported in Figure~\ref{fig:time-opt}.
The running time of algorithms \optp and \nopts are on average \textcolor{red}{(100, 100, 100, 100)} and \textcolor{red}{(100, 100, 100, 100)} times slower than one-pass algorithms \siped and \cised on datasets (\ucar, \geolife, \mopsi, \act), respectively. 
%For example, the running time of \optp and \nopts are \textcolor{red}{(100, 100, 100, 100)} and \textcolor{red}{(100, 100, 100, 100)} on datasets (\ucar, \geolife, \mopsi, \act), respectively, when $\epsilon=40m$.

\begin{figure}[h]
	\centering
	\includegraphics[scale=0.15]{Figures/Exp-SED-CR-epsilon-service.png}	
	\includegraphics[scale=0.15]{Figures/Exp-SED-CR-epsilon-geolife.png}	
	\includegraphics[scale=0.15]{Figures/Exp-SED-CR-epsilon-mopsi.png}	
	\includegraphics[scale=0.15]{Figures/Exp-SED-CR-epsilon-zh.png} 
	\vspace{-2.5ex}
	\caption{\small Evaluation of running time of the (near) optimal algorithms: varying the error bound $\epsilon$.}
	\label{fig:time-opt}
	\vspace{-3ex}
\end{figure}
}%%%%%%%%%%%%%%%%