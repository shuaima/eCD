\section{Preliminary}	%\section{Overview of algorithms}
\label{sec-problem}


In this section, we introduce some basic concepts for trajectory simplification.
Notations used are summarized in Table \ref{tab:notations}.



%We first introduce basic notations.
\stitle{Trajectory}. A \textit{trajectory} $\dddot{\mathcal{T}}\left[P_0, \ldots, P_n\right]$ is a sequence of points in a monotonically increasing order of their associated time values (\ie $P_i.t < P_j.t$ for any $0\le i<j\le n$), where a data \textit{point} is defined as a triple $P\left(x, y, t\right)$, which represents that a moving object is located at {\em longitude} $x$ and {\em latitude} $y$ at {\em time} $t$. Note that data points can be viewed as points in a three-dimension Euclidean space.


\eat{
A \textit{directed line segment} (or line segment for simplicity) $\mathcal{L}$ is defined as $\vv{P_{s}P_{e}}$, which represents the closed line segment that connects the start point $P_s$ and the end point $P_e$.
	We also use $|\mathcal{L}|$ and $\mathcal{L}.\theta\in [0, 2\pi)$ to denote the length of a directed line segment $\mathcal{L}$, and its angle with the $x$-axis of the coordinate system $(x, y)$, where $x$ and $y$ are the longitude and latitude, respectively.
	That is, a directed line segment $\mathcal{L}$ = $\vv{P_{s}P_{e}}$ can be treated as a triple $(P_s, |\mathcal{L}|, \mathcal{L}.\theta)$.
}

%Intuitively, a trajectory can also be represented by a continuous $n$-pieces directed line segments (or line segment for simplicity) $\mathcal{L}_i$, $0\le i < n$, where  $\mathcal{L}_i = \vv{P_{i}P_{i+1}}$, represents the closed line segment that connects the start point $P_{i}$ and the end point $P_{i+1}$.

\stitle{Piece-wise line representation}. A \textit{piece-wise line representation} $\overline{\mathcal{T}}\left[\mathcal{L}_0, \ldots, \mathcal{L}_m\right]$ ($0< m \le n$) of a trajectory $\dddot{\mathcal{T}}\left[P_0, \ldots, P_n\right]$ is a sequence of continuous \textit{directed line segments} (or line segment for simplicity) $\mathcal{L}_{i}$ = $\vv{P_{s_i}P_{e_i}}$ ($i\in\left[0,m\right]$) of $\dddot{\mathcal{T}}$ such that $\mathcal{L}_{0}.P_{s_0} = P_0$, $\mathcal{L}_{m}.P_{e_m} = P_n$ and  $\mathcal{L}_{i}.P_{e_i}$ = $\mathcal{L}_{i+1}.P_{s_{i+1}}$ for all $i\in\left[0, m-1\right]$.
Note that each directed line segment in $\overline{\mathcal{T}}$ essentially represents a continuous sequence of data points in trajectory $\dddot{\mathcal{T}}$.


	\begin{table}
	\renewcommand{\arraystretch}{1.20}
	\vspace{-1ex}
	\caption{\small Summary of notations}
	\centering
	\footnotesize
	%\scriptsize
	\begin{tabular}{|c|l|}
		\hline
		{\bf Notations}& {\bf Semantics}   \\		\hline %\hline
		$P$ & a data point \\		\hline
		$\dddot{\mathcal{T}}$ & a trajectory $\dddot{\mathcal{T}}$ is a sequence of data points\\		\hline
		$\overline{\mathcal{T}}$&  {a piece-wise line representation of a trajectory $\dddot{\mathcal{T}}$}	\\		\hline
		$\mathcal{L}$ & a directed line segment  \\		\hline
		$ped\left(P, \mathcal{L}\right)$ &  {the perpendicular Euclidean distance of point $P$ to line segment $\mathcal{L}$}	\\	\hline
		$sed\left(P, \mathcal{L}\right)$ & {the synchronous Euclidean distance of point $P$ to line segment $\mathcal{L}$} 	\\		\hline
		$dad\left(\mathcal{L}_1, \mathcal{L}_2\right)$ & {the direction-aware distance of line segment $\mathcal{L}_1$ to line segment $\mathcal{L}_2$} 	\\		\hline
		$\epsilon$ & the error bound \\		\hline
		%\sector{} & a sector \\		\hline
		%		$\vv{A} \times \vv{B}$ & the cross product of (vectors) $\vv{A}$ and $\vv{B}$\\		\hline
		%		$\mathcal{H}(\mathcal{L})$ & The open half-plane to the left of $\mathcal{L}$ \\		\hline
		%		$\mathcal{R}$& a convex polygon \\		\hline
		%		$\mathcal{R}^*$ & the intersection of convex polygons \\		\hline
		%		$m$ & the maximum number of edges of a polygon\\		\hline
		%		$E^j$ & a group of edges labeled with $j$\\		\hline
		%		$g(e)$ & the label of an edge $e$ of polygons \\		\hline
		%		\circle{} & a synchronous circle\\		\hline
		%\cone{} & a spatio-temporal cone \\		\hline
		%		\pcircle{} & a cone projection circle \\		\hline
		%$\bigsqcap$ & intersection of geometries\\		\hline
		%$G$ &	the reachability graph of a trajectory\\		\hline
	\end{tabular}
	\label{tab:notations}
	\vspace{-1ex}
\end{table}


 For trajectory simplification, three distance metrics are commonly used, namely, the \emph{perpendicular Euclidean distance} (\ped), the \emph{synchronous Euclidean distance} \cite{Meratnia:Spatiotemporal} (\sed) and the \emph{direction-aware distance}\cite{Long:Direction, Zhang:Evaluation} (\dad).
%
Consider a data point $P$ and a directed line segment $\mathcal{L}$ = $\vv{P_{s}P_{e}}$.

\stitle{Perpendicular Euclidean distance}. The perpendicular Euclidean distance $ped\left(P, \mathcal{L}\right)$ of point $P$ to line segment $\mathcal{L}$ is $\min\{|PQ|\}$ for any point $Q$ on $\vv{P_{s}P_{e}}$.

\stitle{Synchronous Euclidean distance}. The synchronous Euclidean distance $sed\left(P, \mathcal{L}\right)$ of point $P$ to line segment $\mathcal{L}$ is $|\vv{PP'}|$ that is the Euclidean distance from $P$ to its \textit{synchronized data point} $P'$ \wrt $\mathcal{L}$, where the synchronized data point $P'$ \wrt $\mathcal{L}$ is defined as follows:
(a) $P'.x$ = $P_s.x +  c\cdot\left(P_e.x - P_s.x\right)$,
(b) $P'.y$ = $P_s.y +  c\cdot\left(P_e.y - P_s.y\right)$ and
(c) $P'.t$ = $P.t$, where $c= \frac{P.t-P_s.t}{P_e.t-P_s.t}$.

\textcolor{blue}{Synchronized points are essentially virtual points with the assumption that an object moved along a straight line segment from $P_s$ to $P_e$ with a uniform speed, \ie the average speed $\frac{|\vv{P_sP_e}|}{P_e.t-P_s.t}$ between points $P_s$ and $P_e$ \cite{Cao:Spatio,Lin:Cised}. Then the \emph{synchronized point} $P'$ of a point $P$ \wrt the line segment $\vv{P_sP_e}$ is the expected position of the moving object on $\vv{P_sP_e}$ at time $P.t$, obtained by a linear interpolation \cite{Cao:Spatio}. More specifically, a synchronized point $P'_i$ of $P_i$ ($s\le i < e$) \wrt the line segment $\vv{P_sP_e}$ is a point on $\vv{P_sP_{e}}$ satisfying ${|\vv{P_sP'_i}|} = \frac{P_i.t - P_s.t}{P_e.t - P_s.t}\cdot {|\vv{P_sP_e}|}$, which means that the object moves from $P_s$ to $P_e$ at an average speed $\frac{|\vv{P_sP_e}|}{P_e.t-P_s.t}$, and its position at time $P_i.t$ is the point $P'_i$ on $\overrightarrow{P_sP_{e}}$ having a distance of $\frac{P_i.t - P_s.t}{P_e.t - P_s.t}\cdot|\vv{P_sP_e}|$ to $P_s$~\cite{Cao:Spatio, Lin:Cised,Meratnia:Spatiotemporal, Chen:Fast, Zhang:Evaluation}.}

\stitle{Direction-aware distance distance}.The direction-aware distance $dad\left(\mathcal{L}_1, \mathcal{L}_2\right)$ is the direction deviation from $\mathcal{L}_1$ to $\mathcal{L}_2$, \ie $\Delta\left(\mathcal{L}_1.\theta, \mathcal{L}_2.\theta\right) = \min\{|\mathcal{L}_1.\theta - \mathcal{L}_2.\theta|, 2\pi - |\mathcal{L}_1.\theta - \mathcal{L}_2.\theta|\}$, where $\theta \in \left[0, 2\pi\right)$ is the angular of $\mathcal{L}$.
\textcolor{blue}{Note \dad differs from \ped and \sed in that it is a measure of angle, rather than Euclidean distances.}


%algorithm $\mathcal{A}$
% an extension of the famous \emph{min-$\#$} problem of piece-wise line simplification\cite{Chan:Optimal, Imai:Optimal,Pavlidis:Segment},


%\lsa algorithms using \sed may produce more line segments, however, the use of \sed ensures that the Euclidean distance between a data point and its synchronized point \wrt the corresponding line segment is bounded.
%Hence, the above spatio-temporal query over the trajectories compressed by \sed enabled approaches returns the synchronized point $P'$ of a data point $P$ within a bounded distance.

%In addition, \dad\cite{Long:Direction, Zhang:Evaluation} was introduced to preserve the direction information of a moving object. The \dad of $\vv{P_{i-1}P_{i}}$ ($s<i\le e$) to $\vv{P_{s}P_{e}}$ is the direction deviation among them. For instance, in Figure~\ref{fig:notations}, $\theta_{56}$ is the \dad of line segment $\vv{P_5P_6}$ to $\vv{P_0P_{10}}$. It is claimed that \dad is important in applications of trajectory clustering and direction-based query processing\cite{Long:Direction}. Note that the temporal information is also lost when using \dad.






\stitle{Min-$\#$ problem}. Given a trajectory \trajec{T}$\left[P_0, \dots, P_n\right]$ and a pre-specified constant $\epsilon$, the \emph{min-$\#$} problem of trajectory simplification is to approximate the trajectory \trajec{T} with $\overline{\mathcal{T}}\left[\mathcal{L}_0, \ldots , \mathcal{L}_m\right]$ ($0< m \le n$), such that
(1) on each of them the points $\left[P_{s_i}, \dots, P_{e_i}\right]$ are approximated by a line segment $\mathcal{L}_i = \vv{P_{s_i}P_{e_i}}$ with the maximum \ped or \sed \emph{error} of point $P_j$ (or \dad \emph{error} of line segment $|\vv{P_jP_{j+1}}|$) to line segment $\mathcal{L}_i$, $s_i \le j<e_i$,  less than $\epsilon$, and
(2) $P_{s_i}$ and $P_{e_i} \in$ \trajec{T}.


We illustrate the problem and notations with examples.




\begin{example}
	\label{exm-notations}
	Consider {Figure}~\ref{fig:dp}, in which
	%
	(1) $\dddot{\mathcal{T}}\left[P_0, \ldots, P_{10}\right]$ is a trajectory having 11 data points,
	%
    (2) the set of two continuous line segments $\{\vv{P_0P_4}$, $\vv{P_4P_{10}}$\}, the set of four continuous line segments $\{\vv{P_0P_2}$, $\vv{P_2P_4}$, $\vv{P_4P_7}$, $\vv{P_7P_{10}}$\} and the set of three continuous line segments $\{\vv{P_0P_4}$, $\vv{P_4P_5}$, $\vv{P_5P_{10}}$\} are three piecewise line representations of trajectory $\dddot{\mathcal{T}}$,
	%
	(3) $ped\left(P_4, \vv{P_0P_{10}}\right)=|\vv{P_4P^*_4}|$, where $P^*_4$ is the perpendicular point of $P_4$ \wrt line segment $\vv{P_0P_{10}}$,
	%
	(4) For $P_4$, its synchronized point $P'_4$ \wrt $\vv{P_0P_{10}}$ satisfies $\frac{|\vv{P_0P'_4}|}{|\vv{P_0P_{10}}|} = \frac{P_4.t - P_0.t}{P_{10}.t-P_0.t} = \frac{4-0}{10-0}= \frac{2}{5}$,
	%
	(5) $sed\left(P_4, \vv{P_0P_{10}}\right)= |\vv{P_4P'_4}|$, $sed\left(P_2, \vv{P_0P_{4}}\right)= |\vv{P_2P'_2}|$ and $sed\left(P_7, \vv{P_4P_{10}}\right)$ $=$ $|\vv{P_7P'_7}|$,
	where points $P'_4$, $P'_2$ and $P'_7$ are the synchronized points of $P_4$, $P_2$ and $P_7$ \wrt line segments $\vv{P_0P_{10}}$, $\vv{P_0P_{4}}$ and $\vv{P_4P_{10}}$, respectively.  and
    %
    (6) $dad\left(\vv{P_5P_6}, \vv{P_0P_{10}}\right)=\theta_{56}$ is the \dad of line segment $\vv{P_5P_6}$ to $\vv{P_0P_{10}}$.
\end{example}


\begin{figure}[tb!]
	\centering
	\includegraphics[scale=0.46]{Figures/Fig-DP.png}\vspace{-1ex}
	%\caption{\small A trajectory is simplified by algorithm \dpa using distance metrics \ped, \sed and \dad, respectively.}
	\caption{\small  A trajectory $\dddot{\mathcal{T}}[P_0, \ldots, P_{10}]$ with 11 points is compressed by the Douglas--Peucker algorithm \cite{Douglas:Peucker} using distance metrics \ped, \sed and \dad, respectively.}
		\vspace{-2ex}
	\label{fig:dp}
\end{figure}
