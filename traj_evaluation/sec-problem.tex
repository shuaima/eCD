\section{Line Simplification Problem}
%In this section, we introduce some basic concepts and existing line simplification based algorithms for trajectory compression.
%\textcolor[rgb]{1.00,0.00,0.00}{A discussion and summary is presented in the last.}

\begin{figure*}[tb!]
\centering
\vspace{-1ex}
\includegraphics[scale=0.66]{figures/Fig-DP.png}
\vspace{-1ex}
\caption{\small A trajectory $\dddot{\mathcal{T}}[P_0, \ldots, P_{8}]$  with nine points is represented by two (left) and four (right) continuous line segments (solid blue), compressed by the Douglas--Peucker algorithm \cite{Douglas:Peucker} with error metrics \ped and \sed, respectively.}
\vspace{-2ex}
\label{fig:notations}
\end{figure*}

\subsection{Notations}

We first introduce basic notations.

\stitle{Points ($P$)}. A data point is defined as a triple $P(x, y, t)$, which represents that a moving object is located at {\em longitude} $x$ and {\em latitude} $y$ at {\em time} $t$. Note that data points can be viewed as points in a three-dimension Euclidean space.

\stitle{Trajectories ($\dddot{\mathcal{T}}$)}. A trajectory $\dddot{\mathcal{T}}[P_0, \ldots, P_n]$ is a sequence of data points in a monotonically increasing order of their associated time values (\ie $P_i.t < P_j.t$ for any $0\le i<j\le n$). Intuitively, a trajectory is the path (or track) that a moving object follows through space as a function of time~\cite{physics-trajectory}.


\stitle{Directed line segments ($\mathcal{L}$)}. A directed line segment (or line segment for simplicity) $\mathcal{L}$ is defined as $\vv{P_{s}P_{e}}$, which represents the closed line segment that connects the start point $P_s$ and the end point $P_e$.
Note that here $P_s$ or $P_e$ may not be a point in a trajectory $\dddot{\mathcal{T}}$, and hence, we also use notation $\mathcal{R}$ instead of $\mathcal{L}$ when both $P_s$ and $P_e$ belong to $\dddot{\mathcal{T}}$.

We also use $|\mathcal{L}|$ and $\mathcal{L}.\theta\in [0, 2\pi)$ to denote the length of a directed line segment $\mathcal{L}$, and its angle with the $x$-axis of the coordinate system $(x, y)$, where $x$ and $y$ are the longitude and latitude, respectively.
That is, a directed line segment $\mathcal{L}$ = $\vv{P_{s}P_{e}}$ can be treated as a triple $(P_s, |\mathcal{L}|, \mathcal{L}.\theta)$.

\stitle{Piecewise line representation ($\overline{\mathcal{T}}$)}. A piece-wise line representation of a trajectory $\dddot{\mathcal{T}}[P_0, \ldots, P_n]$ is $\overline{\mathcal{T}}[\mathcal{L}_0, \ldots , \mathcal{L}_m]$ ($0< m \le n$), a sequence of continuous directed line segments $\mathcal{L}_{i}$ = $\vv{P_{s_i}P_{e_i}}$ of $\dddot{\mathcal{T}}$ ($i\in[0,m]$)  such that $\mathcal{L}_{0}.P_{s_0} = P_0$, $\mathcal{L}_{m}.P_{e_m} = P_n$ and  $\mathcal{L}_{i}.P_{e_i}$ = $\mathcal{L}_{i+1}.P_{s_{i+1}}$ for all $i\in[0, m-1]$. Note that each directed line segment in $\overline{\mathcal{T}}$ essentially represents a continuous sequence of data points in $\dddot{\mathcal{T}}$.


\subsubsection{Notations of error metrics}

{For line simplification, there are distance based and shape based error metrics\cite{Shi:Survey} that measure the errors between the original trajectory and the simplified trajectory.
However, for trajectory simplification, the distance based metrics are definitely the distinct metrics.}

\stitle{Included angles ($\angle$)}. Given two directed line segments $\mathcal{L}_1$ = $\vv{P_{s}P_{e_1}}$ and $\mathcal{L}_2$ = $\vv{P_{s}P_{e_2}}$ with the same start point $P_s$, the included angle from $\mathcal{L}_1$ to $\mathcal{L}_2$, denoted as $\angle(\mathcal{L}_1, \mathcal{L}_2)$,  is $\mathcal{L}_2.\theta - \mathcal{L}_1.\theta$. For convenience, we also represent the included angle  $\angle(\mathcal{L}_1, \mathcal{L}_2)$ as $\angle{P_{e_1}P_sP_{e_2}}$.

\stitle{Perpendicular points ($P^*$)}. Given a sub trajectory $\dddot{\mathcal{T}}_s[P_s, \ldots, P_e]$ and a point $P \in \dddot{\mathcal{T}}_s$, the perpendicular point $P^* = (x^*, y^*, t)$ of point $P$ \wrt directed line segment $\vv{P_sP_e}$ is the intersection point of line $\vv{P_sP_e}$ and its perpendicular line pass the point $P$. %Note that $|PP^*|=|P_sP|*\sin\angle{P_eP_sP}$.

\stitle{Perpendicular Euclidean Distances ($\ped$)}. Given a point $P_i$ and a directed line segment $\mathcal{L}$ = $\vv{P_{s}P_{e}}$, the distance of $P_i$ to $\mathcal{L}$, denoted as $ped(P_i, \mathcal{L})$ ={$|P_iP_i^*|$}, is the Euclidean distance from $P_i$ to $\vv{P_sP_e}$, commonly adopted by most existing trajectory simplification methods, \eg~\cite{Douglas:Peucker, Hershberger:Speeding, Keogh:online, Chen:Fast, Liu:BQS}.


\stitle{Synchronized points ($P'$)}. Given a sub trajectory $\dddot{\mathcal{T}}_s[P_s, \ldots, P_e]$, the synchronized point $P'$ of point $P$, $P \in \dddot{\mathcal{T}}_s$, ~\wrt line $\vv{P_sP_e}$ is $(x', y', t)$, satisfying $x' = P_s.x +  \frac{P.t-P_s.t}{P_e.t-P_s.t}(P_e.x - P_s.x)$ and $y' = P_s.y +  \frac{P.t-P_s.t}{P_e.t-P_s.t}(P_e.y - P_s.y)$.


\stitle{Synchronized Euclidean Distances ($\sed$)}. The \sed of point $P_i$ to directed line segment $\mathcal{L} = \vv{P_{s}P_{e}}$, denoted as $sed(P_i, \mathcal{L})$, is $|P_iP_i'|$, the distance from $P_i$ to the synchronized data point $P_i'$ \wrt $\mathcal{L}$, adopted by recent trajectory simplification methods~\cite{Meratnia:Spatiotemporal, Chen:Fast, Muckell:Compression, Popa:Spatio}.

%\stitle{Radial Euclidean Distances ($\red$)}. The \red of $P$ \wrt $\mathcal{L}$, denoted as $red(P, \mathcal{L})$, is the distance from the perpendicular point $P^*$ of $P$ to the synchronized point $P'$ of $P$.


%\stitle{Dual Euclidean distances $(\ped, ~\red)$}. A \ded is a two-tuples $(\ped, ~\red)$, whereas, \ped and \red either has its error bound ($\epsilon_p$ of \ped and $\epsilon_r$ of \red), and is checked separately. Note that the distance checking approach of the original \lsa algorithms, whose \red error bound $\epsilon_r$ is fixed with $\infty$, is a special case of \ded.


\begin{example}
\label{exm-notations}
Figure~\ref{fig:notations} is examples of these notations:
(1) a trajectory $\dddot{\mathcal{T}}[P_0$, $\ldots, P_{10}]$ having eleven data points;
(2) the set of two continuous line segments $\{\vv{P_0P_4}$, $\vv{P_4P_{10}}$\} (Left) and the set of four continuous line segments $\{\vv{P_0P_2}$, $\vv{P_2P_4}$, $\vv{P_4P_7}$, $\vv{P_7P_{10}}$\} (Right) in the figure are two piecewise line representations of the trajectory $\dddot{\mathcal{T}}$;
(3) $P^*_4$ is the perpendicular point of $P_4$ \wrt line segment $\vv{P_0P_{10}}$;
(4) $ped(P_4, \vv{P_0P_{10}})=|P_4P^*_4|$;
(5) $P'_4$, $P'_2$ and $P'_7$ are the synchronized points of points $P_4$, $P_2$ and $P_7$ \wrt line segments $\vv{P_0P_{10}}$, $\vv{P_0P_{4}}$ and $\vv{P_4P_{10}}$, respectively;
(6) $sed(P_4, \vv{P_0P_{10}})= |P_4P'_4|$, $sed(P_2, \vv{P_0P_{4}})= |P_2P'_2|$ and $sed(P_7, \vv{P_4P_{10}}) = |P_7P'_7|$;
%(7) $red(P_7, \vv{P_4P_8})= |P^*_7P'_7|$.
\end{example}


%%%%%%%%%%%%%%%%%%%%%%%%%%%%%%%%%%%%%%%%%%%%%%%%%%%%%%%%%%%%%%%%%%%%%%%%%%%%%%%%%%%%%%%%%%%
%\vspace{-3ex}
\subsection{Problem Definition}
This paper focus on the \emph{min-$\#$ problem} \cite{Chan:Optimal, Imai:Optimal,Pavlidis:Segment} of trajectory simplification.
Given a trajectory \trajec{T}$[P_0, \dots, P_n]$ and a pre-specified constant $\epsilon$, a trajectory simplification algorithm $\mathcal{A}$ approximates \trajec{T} by $\overline{\mathcal{T}}[\mathcal{L}_0, \ldots , \mathcal{L}_m]$ ($0< m \le n$), where on
each of them the data points $[P_{s_i}, \dots, P_{e_i}]$ are approximated by a line segment $\mathcal{L}_i = \vv{P_{s_i}P_{e_i}}$ with the maximum error of  $ped(P_j, \mathcal{L}_i)$ or $sed(P_j, \mathcal{L}_i)$, $s_i < j<e_i$,  less than $\epsilon$.
The optimal methods that find the minimal $m$, having the time complexity of $O(n^2)$\cite{Chan:Optimal},
making it impractical for large inputs\cite{Heckbert:Survey}.
Hence, this paper evaluates the distinct sub-optimal methods.

\eat{%%%%%%%%%%%%%%%%%%%%%%%%%%%%%%%%%

\begin{figure}[tb!]
\label{fig:scope}
\centering
\includegraphics[scale=0.8]{figures/Fig-scope.png}
\vspace{-1ex}
\caption{Example \emph{scopes} of synchronized points. The synchronized point $P'$ has (1) a circle scope  when using \sed, and (2) a rectangle scope when using \ded.}

\vspace{-2ex}
\end{figure}

%Given a line $\overline{P_sP_e}$, a synchronized point $P'$ on line $\overline{P_sP_e}$ and a error bound $\epsilon_s$ of ~\sed (or error bounds $\epsilon_p$ and $\epsilon_r$ of \ded), all points potentially be compressed to a synchronized point $P'$ forms a \emph{scope} of $P'$ (Fig.\ref{fig:scope}). The \emph{scope} of $P'$ in \sed is a circle around $P'$ whose radius is less than the \sed error bound $\epsilon_s$, and the \emph{scope} of $P'$ in \ded is a rectangle, paralleling to the line $\overline{P_sP_e}$, taking $P'$ as the central point and whose height and width are less than the \ded error bounds $\epsilon_p$ and $\epsilon_r$ respectively. With he help of \ded, one can set the value of $\epsilon_p$ and $\epsilon_r$ separately according to varied application requirements, \eg a smaller $\epsilon_p$ to limit the perpendicular deviation of the car to the road while a bigger $\epsilon_r$ to ensure a better compression ratio.
}%%%%%%%%%%%%%%%%%%%%%%%%%%%%%%%%%
