\section{Problem}
In this section, we first introduce basic concepts, then state the piece-wise line simplification problem of trajectory.

Notations used are summarized in Table \ref{tab:notations}.

\begin{figure*}[tb!]
\centering
\vspace{-1ex}
\includegraphics[scale=0.66]{figures/Fig-DP.png}
\vspace{-1ex}
\caption{\small A trajectory $\dddot{\mathcal{T}}[P_0, \ldots, P_{10}]$  with 11 points is represented by two (left) and four (right) continuous line segments (solid blue), compressed by the Douglas--Peucker algorithm \cite{Douglas:Peucker} with error metrics \ped and \sed, respectively.}
\vspace{-2ex}
\label{fig:notations}
\end{figure*}


\subsection{Basic Notations}
\label{subsec-notation}

We first introduce basic notations.

\begin{table}
	\renewcommand{\arraystretch}{1.20}
	\vspace{-1ex}
	\caption{\small Summary of notations}
	\centering
%	\footnotesize
	\small
	\begin{tabular}{|c|l|}
		\hline
		{\bf Notations}& {\bf Semantics}   \\		\hline %\hline
		$P$ & a data point \\		\hline
		$\dddot{\mathcal{T}}$ & a trajectory $\dddot{\mathcal{T}}$ is a sequence of data points\\		\hline
		$\overline{\mathcal{T}}$&  {a piece-wise line representation of a trajectory $\dddot{\mathcal{T}}$}	\\		\hline
		$\mathcal{L}$ & a directed line segment  \\		\hline
		$ped(P, \mathcal{L})$ &  {the perpendicular Euclidean distance of point $P$}	\\
				& to line segment $\mathcal{L}$	\\		\hline
		$sed(P, \mathcal{L})$ & {the synchronous Euclidean distance of point $P$}	\\
				& to line segment $\mathcal{L}$	\\		\hline
		$\epsilon$ & the error bound \\		\hline
		\sector{} & a sector \\		\hline
%		$\vv{A} \times \vv{B}$ & the cross product of (vectors) $\vv{A}$ and $\vv{B}$\\		\hline
%		$\mathcal{H}(\mathcal{L})$ & The open half-plane to the left of $\mathcal{L}$ \\		\hline
%		$\mathcal{R}$& a convex polygon \\		\hline
%		$\mathcal{R}^*$ & the intersection of convex polygons \\		\hline
%		$m$ & the maximum number of edges of a polygon\\		\hline
%		$E^j$ & a group of edges labeled with $j$\\		\hline
%		$g(e)$ & the label of an edge $e$ of polygons \\		\hline
%		\circle{} & a synchronous circle\\		\hline
		\cone{} & a spatio-temporal cone \\		\hline
%		\pcircle{} & a cone projection circle \\		\hline
		$\bigsqcap$ & intersection of geometries\\		\hline
		$G$ &	the reachability graph of a trajectory\\		\hline
	\end{tabular}
	\label{tab:notations}
	\vspace{-1ex}
\end{table}


\stitle{Points ($P$)}. A data point is defined as a triple $P(x, y, t)$, which represents that a moving object is located at {\em longitude} $x$ and {\em latitude} $y$ at {\em time} $t$. Note that data points can be viewed as points in a three-dimension Euclidean space.

\stitle{Trajectories ($\dddot{\mathcal{T}}$)}. A trajectory $\dddot{\mathcal{T}}[P_0, \ldots, P_n]$ is a sequence of data points in a monotonically increasing order of their associated time values (\ie $P_i.t < P_j.t$ for any $0\le i<j\le n$). Intuitively, a trajectory is the path (or track) that a moving object follows through space as a function of time~\cite{physics-trajectory}.

\stitle{Directed line segments ($\mathcal{L}$)}. A directed line segment (or line segment for simplicity) $\mathcal{L}$ is defined as $\vv{P_{s}P_{e}}$, which represents the closed line segment that connects the start point $P_s$ and the end point $P_e$.
Note that here $P_s$ or $P_e$ may not be a point in a trajectory $\dddot{\mathcal{T}}$.

%, and hence, we also use notation $\mathcal{R}$ instead of $\mathcal{L}$ when both $P_s$ and $P_e$ belong to $\dddot{\mathcal{T}}$.

We also use $|\mathcal{L}|$ and $\mathcal{L}.\theta\in [0, 2\pi)$ to denote the length of a directed line segment $\mathcal{L}$, and its angle with the $x$-axis of the coordinate system $(x, y)$, where $x$ and $y$ are the longitude and latitude, respectively.
That is, a directed line segment $\mathcal{L}$ = $\vv{P_{s}P_{e}}$ can be treated as a triple $(P_s, |\mathcal{L}|, \mathcal{L}.\theta)$.

\stitle{Piecewise line representation ($\overline{\mathcal{T}}$)}. A piece-wise line representation $\overline{\mathcal{T}}[\mathcal{L}_0, \ldots , \mathcal{L}_m]$ ($0< m \le n$) of a trajectory $\dddot{\mathcal{T}}[P_0, \ldots, P_n]$ is a sequence of continuous directed line segments $\mathcal{L}_{i}$ = $\vv{P_{s_i}P_{e_i}}$ ($i\in[0,m]$) of $\dddot{\mathcal{T}}$  such that $\mathcal{L}_{0}.P_{s_0} = P_0$, $\mathcal{L}_{m}.P_{e_m} = P_n$ and  $\mathcal{L}_{i}.P_{e_i}$ = $\mathcal{L}_{i+1}.P_{s_{i+1}}$ for all $i\in[0, m-1]$. Note that each directed line segment in $\overline{\mathcal{T}}$ essentially represents a continuous sequence of data points in $\dddot{\mathcal{T}}$.

%comments: 20171126
%\stitle{Perpendicular Euclidean Distance (\ped)}. Given a data point $P$ and a directed line segment $\mathcal{L}$ = $\vv{P_{s}P_{e}}$, the perpendicular Euclidean distance (or simply perpendicular distance) $ped(P, \mathcal{L})$ of $P$ to $\mathcal{L}$ is the Euclidean distance of $P$ to line $\overline{P_{s}P_{e}}$, adopted by many trajectory simplification methods, \eg~\cite{Douglas:Peucker, Hershberger:Speeding, Liu:BQS, Williams:Longest, Sklansky:Cone, Dunham:Cone, Zhao:Sleeve, Lin:Operb}.


\stitle{Perpendicular Euclidean Distance (\ped)}. Given a data point $P$ and a directed line segment $\mathcal{L}$ = $\vv{P_{s}P_{e}}$, the perpendicular Euclidean distance (or simply perpendicular distance) $ped(P, \mathcal{L})$ of point $P$ to line segment $\mathcal{L}$ is $\min\{|PQ|\}$ for any point $Q$ on $\vv{P_{s}P_{e}}$.

\stitle{Synchronized points \cite{Meratnia:Spatiotemporal}}. Given a sub-trajectory $\dddot{\mathcal{T}}_s[P_s$, $\ldots, P_e]$, the synchronized point $P'$ of a data point  $P \in \dddot{\mathcal{T}}_s$,~\wrt line segment $\vv{P_sP_e}$ is defined as follows:
(1) $P'.x$ = $P_s.x +  c\cdot(P_e.x - P_s.x)$,
(2) $P'.y$ = $P_s.y +  c\cdot(P_e.y - P_s.y)$ and
(3) $P'.t$ = $P.t$, where $c= \frac{P.t-P_s.t}{P_e.t-P_s.t}$.

\stitle{Synchronous Euclidean Distance (\sed) \cite{Meratnia:Spatiotemporal}}. Given a data point $P$ and a directed line segment $\mathcal{L}$ = $\vv{P_{s}P_{e}}$, the synchronous Euclidean distance (or simply synchronous distance) $sed(P, \mathcal{L})$ of $P$ to $\mathcal{L}$ is $|\vv{PP'}|$ that is the Euclidean distance from $P$ to its synchronized data point $P'$ \wrt $\mathcal{L}$. %, adopted by recent trajectory simplification methods~\cite{Meratnia:Spatiotemporal, Potamias:Sampling, Chen:Fast, Muckell:Compression, Popa:Spatio}.

We illustrate these notations with examples.


\begin{example}
	\label{exm-notations}
	Consider Figure~\ref{fig:notations}, in which
	
	\sstab(1) $\dddot{\mathcal{T}}[P_0$, $\ldots, P_{10}]$ is a trajectory having 11 data points,
	
	\sstab (2) the set of two continuous line segments $\{\vv{P_0P_4}$, $\vv{P_4P_{10}}$\} (Left) and the set of four continuous line segments $\{\vv{P_0P_2}$, $\vv{P_2P_4}$, $\vv{P_4P_7}$, $\vv{P_7P_{10}}$\} (Right) are two piecewise line representations of trajectory $\dddot{\mathcal{T}}$,
	
	\sstab(3) $ped(P_4, \vv{P_0P_{10}})=|\vv{P_4P^*_4}|$, where $P^*_4$ is the perpendicular point of $P_4$ \wrt line segment $\vv{P_0P_{10}}$, and
	
	\sstab (4) $sed(P_4, \vv{P_0P_{10}})= |\vv{P_4P'_4}|$, $sed(P_2, \vv{P_0P_{4}})= |\vv{P_2P'_2}|$ and $sed(P_7, \vv{P_4P_{10}})$ $=$ $|\vv{P_7P'_7}|$,
	where points $P'_4$, $P'_2$ and $P'_7$ are the synchronized points of $P_4$, $P_2$ and $P_7$ \wrt line segments $\vv{P_0P_{10}}$, $\vv{P_0P_{4}}$ and $\vv{P_4P_{10}}$, respectively.
\end{example}


\eat{%%%%%%%%%%
\stitle{Error bounded algorithms}. Given a trajectory \trajec{T} and its compression  algorithm $\mathcal{A}$ using \sed (respectively \ped) that produces another trajectory \trajec{T'},
we say that algorithm $\mathcal{A}$ is error bounded by $\epsilon$ if  for each point $P$ in \trajec{T}, there exist points $P_j$ and $P_{j+1}$ in \trajec{T'} such that $sed(P, \mathcal{L}(P_j,P_{j+1}))\le \epsilon$ (respectively $ped(P, \mathcal{L}(P_j,P_{j+1}))\le \epsilon$).
}%%%%%%%%%%%%%%


\eat{%%%%%%%%%%%%%%%%%%%%%%%%
\begin{table*}
	\renewcommand{\arraystretch}{1.20}
	\vspace{1ex}
	\caption{\small Summary of piece-wise line based algorithms for trajectory simplification}
	\label{tab:summary-lsa}
	\centering
	\small
	\begin{tabular}{|l|c|c|c|c|c|c|c|}
		\hline
		\kw{Name}  & \kw{Type}    & \kw{Key~Ideas}  &\multicolumn{2}{|c|}{\kw{Error~Metrics}} &\multicolumn{2}{|c|}{ \kw{Time~Complexity}} & \kw{Space} \\		\cline{4-7}
				   &  			&            &\kw{\ped} &\kw{\sed} 		& \kw{Best}& \kw{Worst}		& \kw{Complexity}\\		\hline
		\dpa~\cite{Douglas:Peucker, Meratnia:Spatiotemporal}	&batch &Top-down     &Y &Y   & $\Omega(n)$ & $O(n^2)$ & O(n)  \\		\hline
		\tpa~\cite{Pavlidis:Segment}	&batch	&Bottom-up       &Y &Y   & $\Omega(n\log n)$ & $O(n^2/K)$ & O(n)  \\		\hline
		\bqsa~\cite{Liu:BQS}	&online	&Top-down and Convex hull    &Y   & N &$\Omega(n)$ & $O(n^2)$  & $O(|Q|)$   \\		\hline
		\squishe~\cite{Muckell:SQUISH}	&{online}	& Bottom-up and Priority queue       & N &Y   & $\Omega(n\log|Q|)$ & $O(n\log|Q|)$ & $O(|Q|)$ \\		\hline
		%swab	&   &online   &Y &Y  & $\Omega(n)$ & $O(n^2)$ & O(w) &\textcolor[rgb]{1.00,0.00,0.00}{} \\		\hline
		%owtd	&   &online    &Y &Y   & $\Omega(n)$ & $O(n^2)$ & O(n) &\textcolor[rgb]{1.00,0.00,0.00}{} \\ 	\hline
		%ddr	    &       &one-pass    &Y  &Y  & $\Omega(n)$ & $O(n)$ & O(n) &\textcolor[rgb]{1.00,0.00,0.00}{} \\  \hline
		\operb~\cite{Lin:Operb}	& one-pass	& Fitting function    &Y & N &   $\Omega(n)$ & $O(n)$ & O(1)  \\		\hline
		\sia~\cite{Williams:Longest,Sklansky:Cone,Dunham:Cone, Zhao:Sleeve}	&one-pass	& Sector         &Y & N &   $\Omega(n)$ & $O(n)$ & O(1) \\		\hline
		\cia~\cite{Lin:Cised}	&one-pass	& Spatio-temporal Cone         &N & Y &   $\Omega(n)$ & $O(n)$ & O(1) \\		\hline
		\oped~\cite{Chan:Optimal}	&optimal	& Graph and Sector     &Y & N &   $\Omega(n)$ & $O(n^2)$ & \textcolor{red}{$O(n^2)$} \\		\hline
		\osed~	&optimal	& Graph and Spatio-temporal Cone         &N & Y &   $\Omega(n)$ & $O(n^2 \log n)$ & \textcolor{red}{$O(n)$} \\		\hline
	\end{tabular}
	{\\ \small Note: $K$ is the number of the final segments and $|Q|$ is the buffer size.}
	%\vspace{-3ex}
\end{table*}
}%%%%%%%%%%%%%%%%%%EAT

\begin{table*}
	\renewcommand{\arraystretch}{1.20}
	\vspace{1ex}
	\caption{\small Summary of piece-wise line based algorithms for trajectory simplification}
	\label{tab:summary-lsa}
	\centering
	\small
	\begin{tabular}{|l|c|c|c|c|c|c|c|}
		\hline
		\kw{Name}  & \kw{Type}    & \kw{Key~Ideas}  &\kw{\ped} &\kw{\sed}  &  \kw{Time~Complexity} & \kw{Space~Complexity} \\		\hline
		\dpa~\cite{Douglas:Peucker, Meratnia:Spatiotemporal}	&batch &Top-down     &Y &Y    & $O(n^2)$ & O(n)  \\		\hline
		\tpa~\cite{Pavlidis:Segment}	&batch	&Bottom-up       &Y &Y   & $O(n^2/K)$ & O(n)  \\		\hline
		\bqsa~\cite{Liu:BQS}	&online	&Top-down and convex hull    &Y   & N& $O(n^2)$  & $O(|Q|)$   \\		\hline
		\squishe~\cite{Muckell:SQUISH}	&{online}	& Bottom-up and priority queue       & N &Y    & $O(n\log|Q|)$ & $O(|Q|)$ \\		\hline
		%swab	&   &online   &Y &Y  & $\Omega(n)$ & $O(n^2)$ & O(w) &\textcolor[rgb]{1.00,0.00,0.00}{} \\		\hline
		%owtd	&   &online    &Y &Y   & $\Omega(n)$ & $O(n^2)$ & O(n) &\textcolor[rgb]{1.00,0.00,0.00}{} \\ 	\hline
		%ddr	    &       &one-pass    &Y  &Y  & $\Omega(n)$ & $O(n)$ & O(n) &\textcolor[rgb]{1.00,0.00,0.00}{} \\  \hline
		\operb~\cite{Lin:Operb}	& one-pass	& Fitting function    &Y & N  & $O(n)$ & O(1)  \\		\hline
		\siped~\cite{Williams:Longest,Sklansky:Cone,Dunham:Cone, Zhao:Sleeve}	&one-pass	& Sector intersection        &Y & N  & $O(n)$ & O(1) \\		\hline
		\cised~\cite{Lin:Cised}	&one-pass	& Spatio-temporal Cone intersection        &N & Y & $O(n)$ & O(1) \\		\hline
		\oped~\cite{Chan:Optimal}	&optimal	& Graph and sector intersection    &Y & N & $O(n^2)$	& {$O(n)$} \\		\hline
		\osed~	&optimal	&Graph and spatio-temporal cone intersection&N & Y  & $O(n^2 \log n)$	& \textcolor{red}{$O(n)$} \\		\hline
	\end{tabular}
	{\\  Note: $K$ is the number of the final segments of a trajectory and $|Q|$ is the size of buffer/window.}
	%\vspace{-3ex}
\end{table*}


%%%%%%%%%%%%%%%%%%%%%%%%%%%%%%%%%%%%%%%%%%%%%%%%%%%%%%%%%%%%%%%%%%%%%%%%%%%%%%%%%%%%%%%%%%%
%\vspace{-3ex}
\subsection{Problem Statement}
This paper focus on the \emph{min-$\#$ problem} \cite{Chan:Optimal, Imai:Optimal,Pavlidis:Segment} of the piece-wise line based trajectory simplification.
%
Given a trajectory \trajec{T}$[P_0, \dots, P_n]$ and a pre-specified constant $\epsilon$, a trajectory simplification algorithm $\mathcal{A}$ approximates \trajec{T} by $\overline{\mathcal{T}}[\mathcal{L}_0, \ldots , \mathcal{L}_m]$ ($0< m \le n$), where 
(1) on each of them the data points $[P_{s_i}, \dots, P_{e_i}]$ are approximated by a line segment $\mathcal{L}_i = \vv{P_{s_i}P_{e_i}}$ with the maximum error of  $ped(P_j, \mathcal{L}_i)$ or $sed(P_j, \mathcal{L}_i)$, $s_i < j<e_i$,  less than $\epsilon$, and 	%\ie error bounded by ped or sed
(2) $P_{s_i}$ and $P_{e_i} \in$ \trajec{T}. 	%\ie Strong simplification

This paper evaluates both the optimal and the sub-optimal methods using \ped and/or \sed for the \emph{min-$\#$ problem} of trajectory simplification.
They are nine distinct \lsa algorithms evaluated in our experiments, namely,
(i) Douglas-Peucker (\dpa)\cite{Douglas:Peucker,Meratnia:Spatiotemporal} and \pavlidis (\tpa)~\cite{Pavlidis:Segment}, two distinct batch algorithms,
(ii) \bqsa\cite{Liu:BQS} and \squishe~\cite{Muckell:SQUISH}, two online algorithms, %\textcolor[rgb]{1.00,0.00,0.00}{\opwa, \swab  and}
(iii) \operb\cite{Lin:Operb}, sector intersection (\siped)~\cite{Williams:Longest,Sklansky:Cone,Dunham:Cone, Zhao:Sleeve} and spatio-temporal cone intersection using \sed~(\cised)~\cite{Lin:Cised}, three one-pass algorithms, and 
(iv) \oped\cite{Chan:Optimal} and \osed, two optimal \lsa algorithms.
%
Table~\ref{tab:summary-lsa} is a summary of these algorithms.
Interested readers are referred to Appendix A for a summary of other existing work on trajectory compression.



