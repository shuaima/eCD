\section{Related work}
Apart from the techniques evaluated in this paper, other approaches that satisfies different requirements, have been proposed or could be applied for trajectory compression.
We next summarized other representative \lsa algorithms. Interested readers refer to \cite{Shi:Survey}, \cite{Muckell:Compression} and \cite{Lin:Cised} for more information. %related work.

%\subsection*{A.1 Other Line simplification algorithms}

\stitle {Weak simplification algorithms.}
The weak simplification allows that the output data points may not belong to the original data sets \cite{Trajcevski:DDR} (otherwise, the strong simplification). 
That is, weak simplification allows data interpolation. Algorithms \sleeve \cite{Zhao:Sleeve}, \operb\cite{Lin:Operb} and \cised \cite{Lin:Cised} are both strong and weak \lsa algorithms whose weak simplification versions have better compression ratios than their strong simplification versions, and even comparable with the optimal algorithms \cite{Lin:Cised}.

\stitle {Algorithms using other error metric.} %on the min-$\#$ problem
A number of algorithms \cite{Agarwal:Metric, Chen:Fast, Wu:Graph} have been developed to solve the ``min-\#" problem under alternative error metrics.
\cite{Agarwal:Metric} studied the problem under the $L_1$ and uniform (also known as Chebyshev) metric. % while this study focuses on the $L_2$ metric.
\cite{Chen:Fast} defined \eat{the local integral square synchronous Euclidean distance (LISSED) and} the integral square synchronous Euclidean distance (ISSED), and used it to fast the construction of reachability graphs, and \cite{Wu:Graph} followed the ideas of \cite{Chen:Fast}.
ISSED is a cumulative error that leads to varied effectiveness compared with \sed.



\stitle {Algorithms on the min-$\epsilon$ problem.}
Some work focuses on the minimal $\epsilon$ problem \cite{Chan:Optimal} that, given $m$, constructs an approximate curve consisting of at most $m$ line segments with minimum error. SQUISH($\lambda$) \cite{Muckell:SQUISH} and SQUISH-E($\lambda$) \cite{Muckell:Compression} are such algorithms that compress a trajectory of length $n$ into a trajectory of length at most $n/\lambda$. 
These methods lack the capability of ensuring that SED error is within a user-specified bound\cite{Muckell:Compression}.
%It is based on the priority queue data structure, where the prioritization is determined based on estimating the amount of \sed introduced into the compression if that point was removed from the compression\cite{Muckell:SQUISH}.

\eat{
\subsection*{A.2 Semantic based algorithms}
The trajectories of certain moving objects such as cars and trucks are constrained by road networks. These moving objects typically travel along road networks, instead of the line segment between two points. Trajectory compression methods based on road networks \cite{Chen:Trajectory, Popa:Spatio,Civilis:Techniques,Hung:Clustering, Gotsman:Compaction, Song:PRESS, Han:Compress}  project trajectory points onto roads (also known as Map-Matching). Moreover, \cite{Gotsman:Compaction, Song:PRESS, Han:Compress} mine and use high frequency patterns of compressed trajectories, instead of roads, to further improve compression effectiveness.
%
Some methods \cite{Schmid:Semantic, Richter:Semantic} compress trajectories beyond the use of road networks, and further make use
of other user specified domain knowledge, such as places of interests along the trajectories \cite{Richter:Semantic}.
%
There are also compression algorithms preserving the direction of the trajectory \cite{Long:Direction}. %As it mentions that the trajectory of a moving object contains a large amount of semantic information that could be used to compress the trajectory. For example, the driving direction of a moving object implies the information of the next direction.

These  semantics based approaches are orthogonal to line simplification based methods, and may be combined with each other to  improve the effectiveness of trajectory compression.
}


