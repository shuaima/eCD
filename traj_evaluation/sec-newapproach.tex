\section{New Approach}

In this section, we first present the Spatio-temporal simplification (with \sed) problem in 3D space, then we solve this problem by an extended cone intersection method.
\subsection{Spatio-temporal simplification in 3D space}

Recall in the definition of the Synchronized Euclidean Distances ($\sed$), the \sed of point $P_i$ to directed line segment $\mathcal{L} = \vv{P_{s}P_{e}}$, denoted as $sed(P_i, \mathcal{L})$, is $|P_iP_i'|$, the distance from $P_i$ to the synchronized data point $P_i' (x_i', y_i', t_i)$ \wrt $\mathcal{L}$.
The \sed is demonstrated in 2D space in the previews works~\cite{Meratnia:Spatiotemporal, Chen:Fast, Muckell:Compression, Popa:Spatio}. However,  the trajectory and the \sed are also capable presented in 3D space.

\begin{llemma}
\label{prop-3d-syn-point}
Given a sub trajectory $\{P_s, \ldots, P_e\}$ and a point $P_i$, $s<i<e$, the intersection point $P'_i$ of the plane $t=t_i$ and the directed line segment $\mathcal{L} = \vv{P_sP_e}$ is the synchronized data point of $P_i$ \wrt $\mathcal{L}$ in 3D space.
\end{llemma}

\begin{proof}
As shown in Figure~\ref{fig:sed3d}, the point $P'_i (x'_i, y'_i, t'_i)$ is the intersection point of the plane $t=t_i$ and the directed line segment $\mathcal{L} = \vv{P_sP_e}$, thus, $t'_i = t_i$ and $\frac{t'_i - t_s}{t_e - t_s}$ = $\frac{t_i - t_s}{t_e - t_s}$  = $\frac{|P_sP'_i|}{|P_sP_e|}$ = $\frac{x'_i - x_s}{x_e - x_s}$ = $\frac{y'_i - y_s}{y_e - y_s}$. Hence, $x'_i = x_s +  \frac{t_i-t_s}{t_e - t_s}(x_e - x_s)$ and $y'_i = y_s +  \frac{t_i - t_s}{t_e - t_s}(y_e - y_s)$, which satisfies the definition of the synchronized data point.
\end{proof}

With Lemma~\ref{prop-3d-syn-point}, and from the geometrical perspective, we have Theorem~\ref{prop-3d-sed-sim}.

\begin{ttheorem}
\label{prop-3d-sed-sim}
Given a sub trajectory $\{P_s, \ldots, P_e\}$ and a constant $\epsilon$, the sub trajectory can be represented by the line segment $\vv{P_sP_e}$ if and only if the line segment passes through all circles $\mathcal{O}(P_i, \epsilon)$ for $i \in (s, e)$, where $\mathcal{O}(P_i, \epsilon)$ is a circle on the plane $t=t_i$, around point $P_i$ and having radius of $\epsilon$.
\end{ttheorem}

\begin{proof}
As shown in Figure~\ref{fig:sed3d}, let $P'_i$ be the intersection point of the line segment $\vv{P_sP_e}$ and the plane $t=t_i$.
By Lemma~\ref{prop-3d-syn-point}, the point $P'_i$ is the synchronized data point of $P_i$.
Because line segment $\vv{P_sP_e}$ passes through the circle which is on the plane $t=t_i$ and around point $P_i$, and has a radius of $\epsilon$, thus, $|P'_iP_i| <\epsilon$. Furthermore, the property is hold for each $i \in (s, e)$, hence, the sub trajectory $\{P_s, \ldots, P_e\}$ can be represented by the line segment $\vv{P_sP_e}$.
Vice versa.
\end{proof}


\begin{figure}[tb!]
\centering
\includegraphics[scale=0.5]{figures/Fig-SEDin3D.png}
\vspace{-1ex}
\caption{\small Spatio-temporal simplification in 3D space. }
\vspace{-2ex}
\label{fig:sed3d}
\end{figure}

%A sub trajectory $\{P_s, \ldots, P_e\}$ is represented by a line segment $\overline{P_sP_e}$
%which passes through the circle on the plane $t=t_i$ and around point $P_i$, for each $s<i<e$.}


\subsection{Cone intersection in 3D space}

In this section, we expand the elegant \emph{cone intersection} method from 2D space to 3D space.
%, so as to present a new efficient and effective spatio-temporal simplification approach.
We first define the notion of \emph{oblique circular cone}.

\stitle{Oblique circular cone ($\mathcal{C}$)}. Given a sub trajectory $\{P_s,...,P_e\}$ and a constant $\epsilon$, the oblique circular cone of point $P_i$ \wrt point $P_s$, denoted as $\mathcal{C}(P_s, P_i, \epsilon)$, has a vertex $P_s$ and a bottom circle around $P_i$ with radius $\epsilon$.

By the oblique circular cone, we have the follows.

\begin{ttheorem}
\label{prop-3d-ci}
Given a sub trajectory $\{P_s,...,P_e\}$ and a constant $\epsilon$, there exists a point $Q$, $Q.t \ge P_e.t$, such that $sed(P_i, \vv{P_sQ})<\epsilon$ for all $P_i$, $i \in [s,e]$, if and only if $\bigcap_{i=1}^{e}{\mathcal{C}(P_s, P_i, \epsilon)} \ne \{P_s\}$.
\end{ttheorem}

\eat{ %%%%%%%%%%%%%%%%%%%%%%%%%%%%%%%%%%%%%%%%%%%%%
\begin{proof}
\textcolor[rgb]{0.00,0.07,1.00}{if $\bigcap_{i=1}^{e}{\mathcal{C}(P_s, P_i, \epsilon)} \ne \{P_s\}$,then there exits a point $Q = (x_e,y_e,t_e)$ within the circle of $P_e$ such that $\vv{P_sQ}$ passes through all the circles $\mathcal{O}(P_i, \epsilon)$ for $i \in (s, e)$.Let $P'_i$ be the intersection point of the line segment $\vv{P_sP_e}$ and the plane $t=t_i$.By Lemma~\ref{prop-3d-syn-point}, the point $P'_i$ is the synchronized data point of $P_i$.Because $\vv{P_sQ}$ passes through all the circles $\mathcal{O}(P_i, \epsilon)$ ,$P'_i$is within the corresponding circle $\mathcal{O}(P_i, \epsilon)$ of $P_i$,we have $|P'_iP_i| < \zeta$,for all $P_i$,$i \in [s,e-1]$.And $Q$ is within $\mathcal{O}(P_e, \epsilon)$ ,thus,$|QP_e| < \zeta$}
\end{proof}
} %%%%%%%%%%%%%%%%%%%%%%%%%%%%%%%%%%%%%%%%%%%%%%%%

\begin{proof}
Omitted!
\end{proof}


\begin{figure}[tb!]
\centering
\includegraphics[scale=0.5]{figures/Fig-cis.png}
\vspace{-1ex}
\caption{\small The trajectory $\dddot{\mathcal{T}}[P_0, \ldots, P_{10}]$ is compressed by the \conei algorithm to two line segments.}
\vspace{-2ex}
\label{fig:cis}
\end{figure}


{Moreover, if we use a narrow cone, \ie the threshold is set to $\epsilon/2$, then the point $Q$ of Theorem~\ref{prop-3d-ci} can be replace by point $P_e$, the end point of the sub trajectory.}

\begin{cor}
\label{prop-3d-ci-half}
Given a sub trajectory $\{P_s, \ldots, P_e\}$ and a constant $\epsilon$, the sub trajectory can be represented by the line segment $\vv{P_sP_e}$  if and only if $\bigcap_{i=1}^{e}{\mathcal{C}(P_s, P_i, \epsilon/2)} \ne \{P_s\}$.
\end{cor}

\begin{proof}
If $\bigcap_{i=1}^{e}{\mathcal{C}(P_s, P_i, \epsilon/2)} \ne \{P_s\}$, then by Theorem~\ref{prop-3d-ci}, there exists a point $Q$, $Q.t = P_e.t$, such that $sed(P_i, \vv{P_sQ})<\epsilon/2$ for all $P_i$, $i \in [s,e]$. We also know that $|P_eQ| < \epsilon/2$, hence, $sed(P_i, \vv{P_sP_e})<\epsilon$.
\end{proof}


Furthermore, the checking of circular cones intersection could be enforced by a more simple way, \ie the checking of circles intersection on the same plane.

\begin{cor}
\label{prop-circle-intersection}
 Let circle $\mathcal{O}(P^c_i, r^c)$ be the projection of the bottom circle of cone $\mathcal{C}(P_s, P_i, \epsilon)$ on the plane $t=t_c$, where $t_c>t_s$, $P^c_i$ =  $(x^c_i, y^c_i, t_c)$ = $(x_s +  \frac{t_c-t_s}{t_i - t_s}(x_i - x_s),~ y_s +  \frac{t_c - t_s}{t_i - t_s}(y_i - y_s),~t_c)$ and $r^c = \frac{t_c-t_s}{t_i-t_s}{\epsilon}$, then there exists a point $Q$ such that $sed(P_i, \vv{P_sQ})<\epsilon$ for all points $P_i$, $i \in [s,e]$, if and only if $\bigcap_{i=1}^{e}{\mathcal{O}(P^c_i, r^c)} \ne \varnothing$.
\end{cor}

\begin{proof}
For circular cones having the same vertex $P_s$, ``$\bigcap_{i=1}^{e}{\mathcal{O}(P^c_i, r^c)} \ne \varnothing$" is naturally equivalent to ``$\bigcap_{i=1}^{e}{\mathcal{C}(P_s, P_i, \epsilon)} \ne \{P_s\}$". Hence, by Theorem~\ref{prop-3d-ci}, we have the conclusion.
\end{proof}


\subsection{\textcolor[rgb]{0.00,0.07,1.00}{Algorithm}}
We now are ready to present our 3D cone intersection algorithm (\coneid). 
It takes as input a trajectory $\dddot{\mathcal{T}}$ and an error bound $\zeta$, and returns the simplified trajectory $\overline{\mathcal{T}}$.

The algorithm sequentially checks all the points in a trajectory whether the circular cone of current point has any intersection with the intersection of cones formed by the preview points. If the intersection is not empty, the current is feasible to be the end point of a sub-trajectory. Otherwise, the current sub-trajectory is terminated and a new sub-trajectory is started.

As shown in Corollary~\ref{prop-circle-intersection}, the intersection of cones are checked by the intersection of projection circles.
%\stitle{Practical Considerations}.
\textcolor[rgb]{0.00,0.07,1.00}{However, the algorithms to check the intersection of circles have at least $O()$ time.....
Hence, we replace each circle by a regular polygon, \eg a regular octagon, to develop a linear time checking algorithm.}
We first project the bottom circle of circular cone to a \textcolor[rgb]{0.00,0.07,1.00}{standard plane}.
Then we use a regular octagon to approximate the projection circle and check the intersection of regular octagons on the standard plane.

%%%%%%%%%%%%%%%%%%%%%Baseline Algorithm
\begin{figure}[tb!]
	%\vspace{-2ex}
\begin{center}
{\small
\begin{minipage}{3.36in}
\myhrule \vspace{-1ex}
\mat{0ex}{
	{\bf Algorithm}~$\coneid(\dddot{\mathcal{T}}[P_0,\ldots,P_n], \epsilon)$\\
	\sstab
	\bcc \hspace{1ex}\= $\bar{\mathcal{T}} = \varnothing$; $i = 0$; $P_s = P_0$; $P_e = P_1$; \\
         \hspace{2ex}     $intersection = \varnothing$; $radius = \zeta/2$ \\
	\icc \hspace{1ex}\= \While $i < n$ \Do \\
	\icc \>\hspace{3ex} polygon = {\bf getPolygon}($p_i$,$p_s$,$radius$) \\
	\icc \>\hspace{3ex} intersection = {\bf getIntersection}(intersection,polygon) \\
	\icc \>\hspace{3ex} \If $intersection = \varnothing$ \Then \\
	\icc \> \hspace{5ex} $\bar{\mathcal{T}} = \bar{\mathcal{T}}\cup \{\mathcal{L}(P_s,P_e)\}$ \\
	\icc \> \hspace{5ex} $p_s = p_i-1$ \\
	\icc \>\hspace{3ex} \Else\\
	\icc \> \hspace{5ex} $p_e = p_i$ \\
	\icc \> \hspace{5ex} $i = i+1$ \\
	\icc \hspace{1ex}\Return $\bar{\mathcal{T}}$
}
\vspace{-2.5ex}
\myhrule
\end{minipage}
}
\end{center}
\vspace{-3ex}
\caption{\small Spatio-Temporal (3D) Cone Intersection algorithm}
\label{alg:CI3d}
\vspace{-2ex}
\end{figure}
%%%%%%%%%%%%%%%%%%%%%%%%%%%%%%%%%%%%%

\subsection{Correctness and Complexity}
\textcolor[rgb]{0.00,0.07,1.00}{For each data points in a trajectory, the procedures getPolygon and getIntersection are called. Procedure getPolygon completes in O(1) time, and  procedure  getIntersection completes in O(m+n) time,where m and n are the number of vertices in polygons.If the number of vertices of regular octagons does not exceed eight.This algorithm takes as most O(n) time.
}
