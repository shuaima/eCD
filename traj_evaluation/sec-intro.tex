%%%%%%%%%%%%%%%%%%%%%%%%%%%%%%%%%%%%%%%%%%%%%%%%%%%%%%
%\vspace{-0.5ex}
\section{Introduction}
\label{sec-intro}
%%%%%%%%%%%%%%%%%%%%%%%%%%%%%%%%%%%%%%%%%%%%%%%%%%%%%



%A trajectory is a sequence of GPS data points representing the track of a moving object, which is collected at a certain sampling rate by a GPS sensor, transmitted to and saved in cloud servers for subsequent applications, such as location based services, moving object tracking, user behavior analysis, POI recommendation, {traffic analysis} and so on.
%
With the increasing popularity of GPS sensors on various mobile devices, such as smart-phones, on-board diagnostics, personal navigation devices and wearable smart devices, trajectory data has been continuously growing. Further, sampling rates are also improved for acquiring more accurate position information, which leads to longer trajectories as well. Thus, transmitting and storing raw trajectory data consume a large amount of network, storage and computing resources \cite{Chen:Trajectory, Chen:Fast, Meratnia:Spatiotemporal, Keogh:online, Liu:BQS, Muckell:Compression,Cao:Spatio, Popa:Spatio, Schmid:Semantic,Richter:Semantic,Long:Direction,Nibali:Trajic}, and trajectory compression techniques \cite{Douglas:Peucker, Hershberger:Speeding, Meratnia:Spatiotemporal, Liu:BQS, Muckell:Compression, Chen:Trajectory, Chen:Fast, Keogh:online, Cao:Spatio, Shi:Survey, Richter:Semantic, Long:Direction, Song:PRESS, Nibali:Trajic,Cao:Dots,Chen:Compression,Ghica:DTracking,Ke:Interval,Lange:Tracking,Lin:Cised,Lin:Operb,Liu:Amnesic,Muckell:SQUISH,Trajcevski:DDR,Wu:Graph,Gotsman:Compaction,Potamias:Sampling} have been developed to alleviate this situation.
%
%Various mobile devices, such as smart-phones, on-board diagnostics, personal navigation devices, and wearable smart devices, have been using their GPS sensors to collect massive trajectory data of moving objects at a certain sampling rate, and transmit it to cloud servers for location based services, trajectory mining and many other applications.
%
%
%Further, we find that the online transmitting of raw trajectories also seriously aggravates several other issues such as out-of-order and duplicate data points in our experiences when implementing an online vehicle-to-cloud data transmission system.
%Fortunately, these issues can be resolved or greatly alleviated by the trajectory compression techniques \cite{Douglas:Peucker, Hershberger:Speeding, Meratnia:Spatiotemporal, Liu:BQS, Muckell:Compression, Chen:Trajectory, Chen:Fast, Keogh:online, Cao:Spatio, Shi:Survey, Richter:Semantic ,Long:Direction, Song:PRESS, Nibali:Trajic}.


Due to the  limitations (poor compression ratio and data reconstruction overhead) of lossless compression, lossy compression techniques have become the mainstream of trajectory compression \cite{Lin:Operb,Zhang:Evaluation}. Quite a few lossy trajectory compression techniques, most notably the piece-wise line simplification \cite{Keogh:online,Liu:BQS, Muckell:Compression, Chen:Trajectory, Chen:Fast, Cao:Spatio,Shi:Survey,Cao:Dots,Chen:Compression,Ghica:DTracking,Ke:Interval,Lange:Tracking,Lin:Cised,Lin:Operb,Liu:Amnesic,Long:Direction,Meratnia:Spatiotemporal,Muckell:SQUISH,Trajcevski:DDR,Wu:Graph} have been developed. 
%,
%\textcolor{blue}{The idea of piece-wise line simplification (\lsa) comes from computational geometry, whose target is to solve the \emph{min-$\#$} problem \cite{Chan:Optimal, Imai:Optimal,Pavlidis:Segment}, \ie given a curve and an $\epsilon$, construct an approximate curve with error within $\epsilon$ and having the minimum number of line segments. }
%
The idea of piece-wise line simplification (\lsa) comes from graphic and computational geometry, whose target is to approximate a fine piece-wise linear curve with a coarse one \textcolor{blue}{(in computer graphics, it is basic that a planar line or curve is defined by a sequence of data points\cite{Williams:Bounded}. If all data points of the coarse curve are a subset of the original curve, then it's a \emph{strong simplification}; otherwise, it's a \emph{weak simplification} \cite{Trajcevski:DDR}). \lsa has two optimization problems \cite{Chan:Optimal, Imai:Optimal,Pavlidis:Segment}, (1) the \emph{min-$\#$} problem: given an $\epsilon$, construct an approximate curve with error within $\epsilon$ and having the minimum number of line segments \cite{Chan:Optimal, Imai:Optimal}. Most \lsa algorithms for trajectory compression \cite{Douglas:Peucker, Hershberger:Speeding, Keogh:online,Liu:BQS, Muckell:Compression, Chen:Trajectory, Chen:Fast, Cao:Spatio,Cao:Dots,Chen:Compression,Ghica:DTracking,Ke:Interval,Lange:Tracking,Lin:Cised,Lin:Operb,Liu:Amnesic,Long:Direction,Meratnia:Spatiotemporal,Trajcevski:DDR,Wu:Graph} solve this problem, and} 
\textcolor{blue}{(2) the \emph{min-$\epsilon$} problem: given $m$, construct an approximate curve consisting of at most $m$ line segments with the minimum error \cite{Chan:Optimal, Imai:Optimal}. Algorithms like SQUISH ($\lambda$) \cite{Muckell:SQUISH} and SQUISH-E ($\lambda$) \cite{Muckell:Compression} solve this problem and are the complementary ones to those solving the \emph{min-$\#$} problem.}
% that compress a trajectory of size $n$ into a trajectory at most $n/\lambda$ points
%These methods lack the capability of compressing trajectories while ensuring that errors are within a user-specified bound \cite{Muckell:Compression}.
%
Line simplification has wide usage due to its beneficial features: (a) simple and easy to implement, \textcolor{blue}{(b) light and applicable to resource-constrained devices}, and (c) bounded errors (refer to algorithms focusing on the \emph{min-$\#$} problem) with good compression ratios. 
%
\textcolor{blue}{In this paper, we focus on \lsa algorithms solving the \emph{min-$\#$} problem. For algorithms solving the \emph{min-$\epsilon$} problem, please refer to \cite{Muckell:SQUISH,Muckell:Compression,Zhang:Evaluation}.}
%
%but its number is quite small compared with algorithm solving \emph{min-$\epsilon$} problem.
%These techniques may be lossless or lossy \cite{Muckell:Compression}.
%Lossless compression methods enable exact reconstruction of the original data from the compressed data without information loss.
%For example, delta compression \cite{Nibali:Trajic} is a lossless compression technique for trajectory data, which has zero error.
%The limitation of lossless compression lies in that its compression ratio is relatively poor \cite{Nibali:Trajic} and {queries on the compressed data are time consuming because data reconstructions from the compressed data are needed before the queries}.
%
%In contrast, lossy compression methods typically identify important data points and remove statistical redundant data points from the original data, and allow errors or derivations compared with the original trajectory data.
%These techniques focus on good compression ratios with acceptable errors, and are the mainstream techniques in field of trajectory compression.
%, or replace original data points in a trajectory with other places of interests, such as roads and shops.


\stitle{Algorithm taxonomy}. \lsa algorithms fall into two categories: \textit{compression optimal} and \textit{compression sub-optimal} algorithms.
\textit{Compression optimal} methods \cite{Imai:Optimal,Chan:Optimal} are to find the minimum number of points or segments to represent the original polygonal lines \wrt an error bound $\epsilon$, by transforming the problem to search for the shortest path of a graph built from the original trajectory.
The optimal \lsa algorithms have relatively high time/space complexities which make them impractical for large trajectory data.
Hence, \textit{compression sub-optimal} algorithms have been developed and/or introduced for trajectory compression, and they achieve better efficiency at an expense of outputting a little more data points. \textcolor{blue}{Following the taxonomy shown in \cite{Lange:Tracking, Lin:Operb, Lin:Cised}}, compression sub-optimal algorithms can further be classified into batch, online and one-pass algorithms \textcolor{blue}{(in \cite{Lange:Tracking}, they are called offline, online and real-time algorithms, respectively)}.

\sstab (1) {\em Batch algorithms} are divided into top-down methods, \eg Douglas-Peucker (\dpa) \cite{Douglas:Peucker,Meratnia:Spatiotemporal} and bottom-up methods, \eg\ \pavlidis (\tpa)~\cite{Pavlidis:Segment}, and apply global distance checking policies such that all trajectory points need to be loaded before starting compression such that a point may be checked multiple times to compute its distance to the corresponding line segments.

\sstab (2) {\em Online algorithms}, such as \opwa \cite{Meratnia:Spatiotemporal}, \squishe \cite{Muckell:Compression} and \bqsa \cite{Liu:BQS}, apply local distance checking policies, and need not have the entire trajectory ready before compressing. They restrict the checking within a window/buffer, but may still check a point  multiple times during the process.

\sstab (3) {\em One-pass algorithms}, such as \operb\cite{Lin:Operb}, \siped \cite{Williams:Longest,Sklansky:Cone,Dunham:Cone, Zhao:Sleeve}, \cised \cite{Lin:Cised}, \intersec\cite{Long:Direction} and \interval \cite{Ke:Interval}, apply better local distance checking policies, which even do not need a window or buffer for the previously read points, and process each point in a trajectory once and only once.

%By using local distance checking, online and one-pass algorithms are much more efficient than batch algorithms.

%, such as the convex hull in \bqsa~\cite{Liu:BQS}, the priority queue in \squishe \cite{Muckell:Compression}, the {fitting function} in \operb \cite {Lin:Operb} and the spatio-temporal cone intersection in \cised \cite {Lin:Cised}.

%This work focus on the piece-wise line simplification (\lsa) methods of trajectory compression, more specially, the \emph{min-$\#$ problem} \cite{Chan:Optimal, Imai:Optimal,Pavlidis:Segment} of the piece-wise line based trajectory simplification.
%In trajectory compression, the \lsa algorithms commonly use two distance metrics, \ie the \emph{perpendicular Euclidean distances} (\ped) and the \emph{synchronous Euclidean distances} (\sed).
%Originally, \lsa algorithms adopt \ped as a distance metric.
%\eg $|\vv{P_4P^*_4}|$ is the \ped of data point $P_4$ to line segment $\vv{P_0P_{10}}$ in Figure~\ref{fig:notations} (left).
%\lsa algorithms using \ped have good compression ratios~ \cite{Douglas:Peucker, Hershberger:Speeding, Liu:BQS, Muckell:Compression, Chen:Trajectory, Cao:Spatio, Shi:Survey}. However, when using \ped, the temporal information is lost.
%Thus, a spatio-temporal query, \eg ``\emph{the position of a moving object at time $t$}", on the compressed trajectories by line simplification algorithms using \ped may return an approximate point $P'$ whose distance to the actual position $P$ of the moving object at time $t$ is unbounded.
%For example, a query for the position of $P_7$ at time $t_7$ may return an approximate data point $P'_7$ whose distance to $P_7$ is great than the  bound $\epsilon$ in Figure~\ref{fig:notations} (left).

\stitle{Distance metrics}. 
Trajectory simplification algorithms are closely coupled with distance metrics,
and different techniques are typically needed for different distance metrics. We consider three
widely adopted metrics: \emph{perpendicular Euclidean distances} (\ped), \emph{synchronous Euclidean distances} (\sed) and \emph{direction-aware distances}.




%, and different techniques are typically needed for different distance metrics. 
%We consider three widely adopted metrics: \emph{perpendicular Euclidean distances} (\ped), \emph{synchronous Euclidean distances} (\sed) and \emph{direction-aware distances} (\dad).
% aging friendliness and query friendliness of these representative algorithms. 
%The importance of each quality criterion is varied with the passage of time.


\eat{
\begin{figure}[tb!]
	\centering
	%\vspace{-1ex}
	\includegraphics[scale=1.2]{figures/Fig-Distance.png}
	%\vspace{-1ex}
	\caption{\small A sub-trajectory $[P_s, \ldots, P_e]$ is simplified using \ped and \sed, respectively. A spatio-temporal query ``\emph{the position $P$ of the moving object at time $t$}'' on the simplified trajectory returns a point (a) in a large zone consisting of a rectangle and two half-circles, or (b) inside a small circle.}
	\vspace{-3ex}
	\label{fig:distances}
\end{figure}
}

%The distance is originally measured by the \emph{perpendicular Euclidean distances} (\ped), and is constantly enriching over time because of the increasing needs of trajectory applications. 

Originally, \lsa algorithms adopt \ped, {the shortest Euclidean distance from a point to a line segment, as the distance metric, \textcolor{blue}{and most of these algorithms ensure that the maximal distance from the output trajectory to the input trajectory is bounded by a preset \ped error bound. When researchers introduced it to simplify trajectories, they basically treated the input trajectory as a sequence of spatial data points. As the result,} \emph{\lsa algorithms using \ped bring good compression ratios~\cite{Douglas:Peucker, Hershberger:Speeding, Liu:BQS, Muckell:Compression, Chen:Trajectory, Cao:Spatio, Shi:Survey} at a cost of losing temporal information of trajectories}.} \textcolor{blue}{Hence, though it is error bounded by \ped, it is \emph{not spatio-temporal query friendly}, \ie a spatio-temporal query like \emph{where\_at}~\cite{Cao:Spatio} on such a simplified trajectory possibly returns a point that has a distance great larger than the \ped error bound used in the simplification algorithm.}

{ \sed was then introduced to preserve the temporal information \cite{Meratnia:Spatiotemporal,Cao:Spatio}. 
The \sed of a point to a line segment is the Euclidean distance between the point and its \emph{synchronized point} \wrt the line segment, the expected position of the moving object on the line segment at the same time \emph{with an assumption that the object moves straightly along the line segment at a uniform speed} \cite{Cao:Spatio}. \textcolor{blue}{Most of these algorithms ensure that the \emph{maximal} \sed from the output trajectory to the input trajectory is bounded by a preset \sed error bound.} 
They friendly support applications such as spatio-temporal queries, \ie a spatio-temporal query like \emph{where\_at}~\cite{Cao:Spatio} on such a simplified trajectory will return the expected (synchronized) point that has a (\sed) distance less than the error bound used in the simplification algorithm.
\textcolor{blue}{Obviously, given the same error bound, algorithms using \sed will output more points than \ped as they also save temporal information.}
%
%\textcolor{blue}{Besides, in order to speed up the simplification, a few algorithms \cite{Chen:Fast, Wu:Graph,Cao:Dots} alternatively use an accumulation of square \sed, named Local Integral Square Synchronous Euclidean Distance (\lissed) \cite{Chen:Fast}, as the error bound. \bumr~\cite{Chen:Fast} is such a batch algorithm, and \dagots~\cite{Cao:Dots} follows the ideas of \cite{Chen:Fast} and provides an online version. These \lissed-equipped algorithms ensure that any \lissed from data points to their representing line segment is limited by a preset \lissed error bound.}



\dad was introduced to preserve the direction information of moving objects~\cite{Long:Direction, Zhang:Evaluation}, and was initially called the \emph{direction-based measurement} in \cite{Long:Direction}. 
It is important for applications such as trajectory clustering and direction-based query processing \cite{Long:Direction,Long:Mining}.
\textcolor{blue}{Different from \ped and \sed, \dad is the direction deviation of a moving object, measured by angular rather than Euclidean distances.}
\textcolor{blue}{The temporal information is also lost when using \dad, hence, it is not friendly to spatio-temporal queries too.}


%\todo{\cite{Agarwal:Metric} studies the problem under the $L_1$ and uniform (also known as Chebyshev) metric.} % while this study focuses on the $L_2$ metric.

\begin{table*}
	\renewcommand{\arraystretch}{1.20}
	\vspace{-1ex}
	\caption{\small Error bounded trajectory simplification algorithms}
	\label{tab:summary-lsa}
	\centering
	\scriptsize
	\begin{tabular}{|c|c|l|c|c|c|c|c|c|c|c}
		\hline
		\multicolumn{2}{|c|}{\bf{Category}} &\bf{Algorithms} &\bf{\ped} &\bf{\sed}  &\bf{\dad} &  \bf{Time} & \bf{Space} &\bf{Rep} & \bf{Key~Ideas} \\		
        \hline
        \multicolumn{2}{|c|}{\multirow{3}*{\bf{optimal}}} &\opt~\cite{Imai:Optimal}					&\checkmark  & \checkmark & \checkmark & $O(n^3)$	& {$O(n^2)$} & \checkmark & reachability graph \\		
        \cline{3-10}
        \multicolumn{2}{|c|}{}&\optp\cite{Chan:Optimal}             		&\checkmark &$\times$ &$\times$ & $O(n^2)$	& {$O(n^2)$} & & {reachability graph \& sector intersection}  \\		
        \cline{3-10}
        \multicolumn{2}{|c|}{}& \textcolor{blue}{\optss\cite{Chen:Fast} } 		&$\times$ & \checkmark$^{*}$ & $\times$ & $O(n^2)$	& {$O(n^2)$} & & {reachability graph \& \lissed} \\		
        \cline{3-10}
        \multicolumn{2}{|c|}{}&\kw{SP} \cite{Long:Direction}          &$\times$ &$\times$ & \checkmark & $O(n^2)$	& {$O(n^2)$} & & {reachability graph \& range intersection}  \\		
        \hline
        
        %%%%%%%%%%%%%%%%%%%%% batch %%%%%%%%%%
        
        \multirow{14}*{\rotatebox{90}{\bf{sub}-\bf{optimal}}}
        &\multirow{3}*{\rotatebox{90}{\bf{batch}}}  &{\kw{Ramer} \cite{Ramer:Split}}		&\checkmark &\checkmark & \checkmark   & $O(n^2)$ & $O(n)$  & & top-down \\		
        \cline{3-10}
        
		& &\dpa\cite{Douglas:Peucker, Meratnia:Spatiotemporal}	&\checkmark &\checkmark &\checkmark   & $O(n^2)$ & $O(n)$  &\checkmark & top-down \\		
		\cline{3-10}
		
        & &\tpa\cite{Pavlidis:Segment}			&\checkmark &\checkmark  &\checkmark  	& $O(n^2/K)$ & $O(n)$  &\checkmark & bottom-up \\		
        
	    \cline{3-10}
	    & &\textcolor{blue}{\bumr\cite{Chen:Fast}}		&$\times$ &{\checkmark}$^{*}$ &$\times$  	& $O(n)$ & { $O(n)$}  & & \lissed \& bottom-up\\	
		\cline{2-10}
		
        %%%%%%%%%%%%%%%%%%% online %%%%%%%%%%%%
        
        &\multirow{4}*{\rotatebox{90}{\bf{online}}}	&\opwa \cite{Meratnia:Spatiotemporal} 	&\checkmark &\checkmark  &\checkmark   	& $O(n^2)$	& $O(n)$  &\checkmark &top-down in opening window	\\		
        \cline{3-10}
        
		& &\bqsa\cite{Liu:BQS}					&\checkmark &$\times$ &$\times$ 		& $O(n^2)$  & $O(|n|)$   &\checkmark &{top-down, window \& convex hull} \\	
			
        \cline{3-10}
		& &\kw{SWAB} \cite{Keogh:online} 	        &\checkmark &\checkmark  &\checkmark   	& $O(n*|Q|)$	& $O(|Q|)$  & &bottom-up in sliding window	\\		

        \cline{3-10}
	    & &\squishe\cite{Muckell:Compression}		&$\times$ &\checkmark  &$\times$  	& $O(n\log|Q|)$ & $O(|Q|)$  &\checkmark & bottom-up \& priority queue \\	
	    	

	    
        \cline{3-10}
		& &\textcolor{blue}{\dagots~\cite{Cao:Dots}}		&$\times$ &{\checkmark}$^{*}$ &$\times$  	& $O(n^2/K)$ & { $O(|Q|^2)$}  &\checkmark &\lissed \& incremental DAG\\	
		
		\cline{3-10}
		& &\textcolor{blue}{\olts\cite{Wu:Graph}}		&$\times$ &{\checkmark}$^{*}$ &$\times$  	& $O(n^2/K)$ & { $O(|Q|^2)$}  & &\lissed \& incremental DAG\\		
		\cline{2-10}

        %%%%%%%%%%%%%%%%% one pass %%%%%%%%%%%%%%
		
        &\multirow{4}*{\rotatebox{90}{\bf{one}-\bf{pass}~~~ }}&\rwa \cite{Reumann:Strip}              &\checkmark &$\times$ &$\times$ 		& $O(n)$ 	& $O(1)$  & & strip  \\		
        \cline{3-10}
		& &\ldr\cite{Lange:Tracking,Trajcevski:DDR} &$\times$ &\checkmark &$\times$ 		& $O (n)$ 	& $O(1)$  & & linear dead reckoning  \\		
		\cline{3-10}
		& &\operb\cite{Lin:Operb}					&\checkmark &$\times$ &$\times$ 		& $O (n)$ 	& $O(1)$   &\checkmark & fitting function \\		
        \cline{3-10}
		& &\siped\cite{Dunham:Cone, Zhao:Sleeve}	&\checkmark &$\times$ &$\times$ 		& $O (n)$ 	& $O(1)$  &\checkmark & sector intersection\\		 %Williams:Longest, Sklansky:Cone,
        \cline{3-10}
		& &\cised\cite{Lin:Cised}					&$\times$ & \checkmark &$\times$ 		& $O (n)$ 	& $O(1)$  &\checkmark & spatio-temporal cone intersection \\		
        \cline{3-10}
		& &\intersec\cite{Long:Direction}			&$\times$ &$\times$ & \checkmark 		& $O (n)$ 	& $O(1)$  &\checkmark & range intersection with $\frac{\epsilon}{2}$-range\\		
        \cline{3-10}
        & &\interval\cite{Ke:Interval}				&$\times$ &$\times$ & \checkmark 		& $O (n)$ 	& $O(1)$  &\checkmark & range intersection with $\epsilon$-range \\		
        \hline
	\end{tabular}
	{\\ \vspace{2ex} Here, (1) $K$ is the number of the final segments of a trajectory, $|Q|$ is the size of a buffer/window and ``Rep" means ``Representative" \textcolor{blue}{(we comprehensively consider the performance of algorithms, the supporting of distance metrics and the novelty of their key ideas to select the representatives)}, \textcolor{blue}{(2) algorithms \optss, \bumr, \dagots and \olts alternatively use Local Integral Square \sed (\lissed \cite{Chen:Fast}) as error measure instead of directly using \sed, and among them, \dagots is the representative, (3) one-pass algorithms \operb and \cised both have strong and weak versions \cite{Lin:Cised, Lin:Operb}, where the former only outputs data points belonging to the original trajectory and the later allows data interpolations,} and (4) batch and top-down algorithms \dpa\cite{Douglas:Peucker, Meratnia:Spatiotemporal} and \kw{Ramer} \cite{Ramer:Split} are aging friendly, while other algorithms are not.   }
	\vspace{-2ex}
\end{table*}
% and 
% .

\stitle{\textcolor{blue}{Quality criteria}}. 
\textcolor{blue}{Quality criteria are needed to evaluate \lsa algorithms. Roughly speaking, these quality criteria are two levels, the first level comes from \lsa itself, including compression ratio, efficiency and simplification errors, where 
	(1) \emph{compression ratio} is the percentage of data size of the simplified trajectories compared with the original trajectories, 
	(2) \emph{efficiency} is the time needed by an \lsa algorithm to simplify the input trajectories, and
	(3) \emph{simplification errors} are the \emph{maximal} and \emph{average} distances from the input trajectories to the output trajectories.
Most works in this area test their algorithms following a part or all of these criteria, and it's worth pointing out that some works \cite{Zhang:Evaluation,Cao:Dots,Wu:Graph} also test the compression ratios under average errors, so as to advise the performance trade-off between them.}
	
	
\textcolor{blue}{The second level comes from trajectory applications, \eg a ``where\_at" query concerns the errors of its answer, a trajectory clustering method concerns the similarity between the clusters of original and simplified trajectories. Indeed, every trajectory application is applicable to evaluate the simplified trajectories from its point of view.}	
%
\textcolor{blue}{In this paper, we choose \emph{spatio-temporal queries} \cite{Cao:Spatio} as the representatives of trajectory applications, because they are well-known and commonly used in the management of trajectories.}
%
\textcolor{blue}{We use errors, \ie~\emph{query errors}, to denote the qualities, where query errors are the temporal differences (\eg ``when\_at") or spatial distances (\eg ``where\_at") between the original trajectory and the answers to the queries on the simplified trajectories.}




%such as spatio-temporal queries, trajectory clustering, and so on. 


\stitle{Motivations}. Empirical studies of trajectory compression algorithms have been conducted~\cite{Muckell:Compression,MuckellHLR10}. However, they only discuss a small number of algorithms. 
%
The very recent study \cite{Zhang:Evaluation} does evaluate a wide range of trajectory simplification algorithms.
However, it provides {an experimental study} on compression errors and spatio-temporal query analyses only, and important aspects of trajectory simplification (compression ratios, running time, aging friendliness and safety) are not systematically studied. 
\textcolor{blue}{More specifically, }
\textcolor{blue}{(1) the impacts of both error bound and the size of a trajectory on running time are not tested;}
%
\textcolor{blue}{(2) the impact of error bound on compression ratio is not tested; and }
%
\textcolor{blue}{(3) it does not cover the \emph{data aging} problem.} The simplified trajectories are stored in data stores, as time goes by, less precision may become acceptable for old trajectories \cite{Cao:Spatio}, and they need to be further simplified to coarse trajectories to save storage. However, can these simplified trajectories be further compressed by \lsa algorithms to coarse trajectories while still having bounded errors \wrt the original trajectories? This is referred to as the \emph{data aging} problem, which is initially introduced and partially discussed for algorithms \opt and \dpa only in \cite{Cao:Spatio}.

\textcolor{blue}{Besides, some important algorithms are not investigated, \eg optimal algorithms using \ped and \sed, and one-pass algorithms \siped and \cised are not investigated in \cite{Zhang:Evaluation}. Indeed, algorithm \siped is {\em completely overlooked} by existing trajectory compression studies as it is originally developed in fields of computational geometry and pattern recognition~\cite{Williams:Longest,Sklansky:Cone,Dunham:Cone, Zhao:Sleeve}, and we recently find it can be easily adopted for trajectory compression with good performance.} %\cite{Lin:Cised}.
%
That is to say, a systematic experimental evaluation of line simplification algorithms for trajectory compression remains necessary for practitioners to decide the appropriate algorithms and distance metrics for specific applications.


%concerning the compression ratio and error of the result data, and the execution time of these compression algorithms.
% Thus, these issues call for a more comprehensive evaluation of the existing \lsa techniques. % on trajectory data.



\stitle{Contributions}.
In this article, we conduct a systematic experimental evaluation and analyses of the mainstream error bounded trajectory simplification algorithms for large-scale trajectory data.
%, {reveal the intrinsic characteristics of algorithms, and provide guidelines and suggestions on the choices of algorithms and distance metrics for different scenarios.}

\stab (1) We classify the error bounded \lsa algorithms into different categories, review each category of algorithms, and systematically evaluate the representative algorithms of each category.
%
\mytable{tab:summary-lsa} summarizes the algorithms that consist of optimal and sub-optimal algorithms, and the latter are further classified into batch, online and one-pass algorithms.
Note that online and one-pass algorithms are typically designed for a specific distance metric  (\ped, \sed or \dad) only.
%optimal algorithm \opt,
%batch algorithms \douglas and \pavlidis,
%online algorithms \bqsa and \squishe,  and
%one-pass algorithms \operb, \siped, \cised and \interval.

\stab (2) {We study the data aging problem of \lsa algorithms and distance metrics from the views of \emph{aging friendliness} and \emph{aging safety}, and prove that (a) only algorithms running in both batch and top-down manners and using \ped and/or \sed are \emph{aging friendly}, having an error bound of $\max{\{\epsilon_1, \epsilon_2\}}$ \wrt the original trajectory, \textcolor{blue}{where $\epsilon_1$ and $\epsilon_2$ are the error bounds set in the first and second times of simplification, respectively}, and (b) other algorithms \textcolor{blue}{are \emph{aging safe}}, having an error bound of ${\epsilon_1+\epsilon_2}$ for data aging.  These together provide a full picture of the data aging problem.}

\stab (3) Using real-life trajectory datasets, we systematically evaluate and analyze the performance (compression ratio, average error, running time, {aging friendliness and query friendliness}) of error bounded line simplification algorithms with respect to {distance metrics,} trajectory sizes and error bounds.
Our study reveals the characteristics of these algorithms, which lead to guidelines for practitioners to choose appropriate algorithms and distance metrics for specific applications.


Essentially, this study is a necessary complement to existing studies by providing a systematic evaluation and analyses of error bounded trajectory simplification algorithms. 
\textcolor{blue}{In comparison with the recent study \cite{Zhang:Evaluation},
(1) for a fair running time analysis, all algorithms are (re)-implemented in Java, unlike \cite{Zhang:Evaluation} that reports running time of algorithms with different programming languages,
(2) compression ratio analyses are systematically considered,
(3) variations of distance metrics are studied, 
(4) optimal algorithms using \ped and \sed and one-pass algorithms \siped and \cised are investigated, 
(5) weak simplification algorithms \operba~and \cised-W are studied, and
(6) data aging of \lsa algorithms is studied.
Some new findings are summarized in Section \ref{sec-exp-summary}.}


\eat{Essentially, this study is an important complimentary to \cite{Zhang:Evaluation} by providing a thorough and systematic evaluation and analyses of error bounded trajectory simplification algorithms.
Further, though  \cite{Zhang:Evaluation} tests the average errors of algorithms from the aspect of applications, \ie spatio-temporal queries, different applications have different requirements, and it is hardly to enumerate all of them. Hence,}



\stitle{Organization}. Section 2 introduces basic concepts of trajectory simplification,
Section 3 and Section 4 systematically review optimal and sub-optimal \lsa methods, respectively,
{Section 5 analyzes the data aging of \lsa algorithms and}
Section 6 reports and analyzes the experimental results, followed by
conclusions in Section 7.
{Additional \myblue{trajectory compression methods} are discussed in the appendix.}
% and experimental results.% on algorithm \nopts compared with \opt.



