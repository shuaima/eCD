%%%%%%%%%%%%%%%%%%%%%%%%%%%%%%%%%%%%%%%%%%%%%%%%%%%%%%
%\vspace{-0.5ex}
\section{Introduction}
\label{sec-into}
%%%%%%%%%%%%%%%%%%%%%%%%%%%%%%%%%%%%%%%%%%%%%%%%%%%%%

A trajectory is a sequence of GPS data points representing the track of a moving object, which is collected at a certain sampling rate by a GPS sensor, transmitted to and saved in cloud servers for subsequent applications, such as location based services, moving object tracking, user behavior analysis, POI recommendation, {traffic analysis} and so on.
%
With the increasing use of GPS sensors on various mobile devices, such as smart-phones, on-board diagnostics, personal navigation devices and wearable smart devices, the number of trajectories are increasing rapidly.
At the same time, the sampling rate is also improving, \eg from 1 point per minute to currently 1 point per second, in order to get more accurate position information of a moving object, which leads to a larger size of a trajectory than before. Thus, the transmitting and storing raw trajectory data consume increasing size of network bandwidth and storage capacity \cite{Chen:Trajectory,  Chen:Fast, Meratnia:Spatiotemporal, Keogh:online, Liu:BQS, Muckell:Compression,Cao:Spatio, Popa:Spatio, Schmid:Semantic,Richter:Semantic,Long:Direction,Nibali:Trajic}.

Trajectory compression techniques \cite{Douglas:Peucker, Hershberger:Speeding, Meratnia:Spatiotemporal, Liu:BQS, Muckell:Compression, Chen:Trajectory, Chen:Fast, Keogh:online, Cao:Spatio, Shi:Survey, Richter:Semantic ,Long:Direction, Song:PRESS, Nibali:Trajic} have been developed to reduce the size of trajectory data, so as to save the network, storage and computing resources.
%
%Various mobile devices, such as smart-phones, on-board diagnostics, personal navigation devices, and wearable smart devices, have been using their GPS sensors to collect massive trajectory data of moving objects at a certain sampling rate, and transmit it to cloud servers for location based services, trajectory mining and many other applications.
%
%
%Further, we find that the online transmitting of raw trajectories also seriously aggravates several other issues such as out-of-order and duplicate data points in our experiences when implementing an online vehicle-to-cloud data transmission system.
%Fortunately, these issues can be resolved or greatly alleviated by the trajectory compression techniques \cite{Douglas:Peucker, Hershberger:Speeding, Meratnia:Spatiotemporal, Liu:BQS, Muckell:Compression, Chen:Trajectory, Chen:Fast, Keogh:online, Cao:Spatio, Shi:Survey, Richter:Semantic ,Long:Direction, Song:PRESS, Nibali:Trajic}.
%
These techniques may be lossless or lossy \cite{Muckell:Compression}.
Lossless compression methods enable exact reconstruction of the original data from the compressed data without information loss.
%For example, delta compression \cite{Nibali:Trajic} is a lossless compression technique for trajectory data, which has zero error.
The limitation of lossless compression lies in that its compression ratio is relatively poor \cite{Nibali:Trajic} and {queries on the compressed data are time consuming because data reconstructions from the compressed data are needed before the queries}.
%
In contrast, lossy compression methods typically identify important data points and remove statistical redundant data points from the original data, and allow errors or derivations compared with the original trajectory data.
These techniques focus on good compression ratios with acceptable errors, and are the mainstream techniques in field of trajectory compression.
%, or replace original data points in a trajectory with other places of interests, such as roads and shops.

A large number of lossy trajectory compression techniques, most notably the piece-wise line simplification \cite{Douglas:Peucker, Hershberger:Speeding, Keogh:online,Liu:BQS, Muckell:Compression, Chen:Trajectory, Chen:Fast, Cao:Spatio, Shi:Survey} solving the \emph{min-$\#$} problem \cite{Chan:Optimal, Imai:Optimal,Pavlidis:Segment}, have been developed. The idea of piece-wise line simplification (\lsa) comes from computational geometry, whose target is to approximate a given finer piece-wise linear curve by another coarser piece-wise linear curve ({normally} a sub set of the former), such that the maximum distance of the former from the later is bounded by a user specified constant. It is widely used due to its distinct advantages: (a) simple and easy to implement, (b) no need of extra knowledge and suitable for freely  moving  objects \cite{Popa:Spatio}, and (c) bounded errors with good compression ratios.

The \lsa algorithms fall into two categories, \ie the \textit{optimal} and \textit{sub-optimal} algorithms.
The \textit{optimal} methods\cite{Imai:Optimal,Chan:Optimal} are to find the minimal number of points or segments to represent the original polygonal lines \wrt an error bound $\epsilon$.
%An optimal $O(n^3)$ \lsa algorithm using \ped was firstly developed in \cite{Imai:Optimal},  where $n$ is the number of the original points.
%Later, an improved optimal  $O(n^2)$  algorithm using \ped was designed in \cite{Chan:Optimal}, with the help of {sector intersection} mechanism.
%However, the time complexity of the optimal \lsa algorithm using \sed remains in $O(n^3)$, as the optimization mechanisms are \ped specific, and cannot work with \sed.
%	The high time complexities of the optimal \lsa algorithms using \sed make them impractical.
%
%Due to the high time complexities of optimal \lsa algorithms,
The optimal \lsa algorithms have relative high time/space complexities which make them impractical for large trajectory data.
Hence, \textit{sub-optimal} \lsa algorithms have been developed and/or introduced for trajectory compression, including batch algorithms (\eg Douglas-Peucker \cite{Douglas:Peucker, Meratnia:Spatiotemporal, Cao:Spatio} and Theo~Pavlidis \cite{Pavlidis:Segment}), online algorithms (\eg~\bqsa\cite{Liu:BQS} and \squishe \cite{Muckell:Compression}) and most recent, one-pass algorithms (\eg~\operb \cite{Lin:Operb} and \cised \cite{Lin:Cised}). Online algorithms and one-pass algorithms are claimed more efficient by using some optimization strategies, such as the convex hull in \bqsa~\cite{Liu:BQS}, the priority queue in \squishe \cite{Muckell:Compression}, the {fitting function} in \operb \cite {Lin:Operb} and the spatio-temporal cone intersection in \cised \cite {Lin:Cised}.

%This work focus on the piece-wise line simplification (\lsa) methods of trajectory compression, more specially, the \emph{min-$\#$ problem} \cite{Chan:Optimal, Imai:Optimal,Pavlidis:Segment} of the piece-wise line based trajectory simplification.
%In trajectory compression, the \lsa algorithms commonly use two distance metrics, \ie the \emph{perpendicular Euclidean distances} (\ped) and the \emph{synchronous Euclidean distances} (\sed).
%Originally, \lsa algorithms adopt \ped as a distance metric.
%\eg $|\vv{P_4P^*_4}|$ is the \ped of data point $P_4$ to line segment $\vv{P_0P_{10}}$ in Figure~\ref{fig:notations} (left).
%\lsa algorithms using \ped have good compression ratios~ \cite{Douglas:Peucker, Hershberger:Speeding, Liu:BQS, Muckell:Compression, Chen:Trajectory, Cao:Spatio, Shi:Survey}. However, when using \ped, the temporal information is lost.
%Thus, a spatio-temporal query, \eg ``\emph{the position of a moving object at time $t$}", on the compressed trajectories by line simplification algorithms using \ped may return an approximate point $P'$ whose distance to the actual position $P$ of the moving object at time $t$ is unbounded.
%For example, a query for the position of $P_7$ at time $t_7$ may return an approximate data point $P'_7$ whose distance to $P_7$ is great than the  bound $\epsilon$ in Figure~\ref{fig:notations} (left).
%
For trajectory simplification, there are varied distance metrics, most typically, the \emph{perpendicular Euclidean distances} (\ped), the \emph{synchronous Euclidean distances}\cite{Meratnia:Spatiotemporal} (\sed) and the \emph{direction-aware distance}\cite{Zhang:Evaluation}(\dad) (It is first called the \emph{direction-based measurement}\cite{Long:Direction}).
Originally, \lsa algorithms adopt \ped as the distance metric, which brings good compression ratios~ \cite{Douglas:Peucker, Hershberger:Speeding, Liu:BQS, Muckell:Compression, Chen:Trajectory, Cao:Spatio, Shi:Survey}. Nevertheless, when uses it for trajectory compression, the temporal information is lost.
%
\sed was then introduced to save temporal information of trajectories \cite{Meratnia:Spatiotemporal}. \sed is the Euclidean distance of a data point to its \emph{approximate temporally synchronized data point} \cite{Meratnia:Spatiotemporal} on the corresponding line segment.
%For instance, $P'_4$ and $P'_7$ are the \emph{synchronized data points} of points $P_4$ and $P_7$ \wrt line segments $\vv{P_0P_{10}}$ and $\vv{P_4P_{10}}$, respectively, in Figure~\ref{fig:notations} (right).
\lsa algorithms using \sed may produce more line segments, however, the use of \sed ensures that the Euclidean distance between a data point and its synchronized point \wrt the corresponding line segment is bounded.
%Hence, the above spatio-temporal query over the trajectories compressed by \sed enabled approaches returns the synchronized point $P'$ of a data point $P$ within a bounded distance.
%
In addition, \dad\cite{Long:Direction, Zhang:Evaluation} was introduced to preserve the direction information of a moving object, which is claimed important in applications of trajectory clustering and direction-based query processing\cite{Long:Direction}. Note that the temporal information is also lost when using \dad.
%When using \dad, maximum angular difference between original trajectory and simpfiled trajectory

\stitle{Motivations}. A number of techniques are developed or can be applied for trajectory compression, however, these techniques have not been compared systematically under the same experimental framework.
Though recently, Zhang et al in \cite{Zhang:Evaluation} experimental compared a wide range of trajectory simplification algorithms, they focus on the compression quality, \ie the error between the compressed data and the original data. Meanwhile, two important features of those algorithms, \ie the compression ratio and execution time, are not comprehensively studied. %Also some important algorithms are not discussed in this study.
More specifically,

%\begin{enumerate}
\ni (1) Three distance metrics, \ie the \ped and \dad for spatial compression and the \sed \cite{Meratnia:Spatiotemporal} for spatio-temporal compression, which are supported in those algorithms, and what impacts they are on performance, have not been compared systematically under the same experimental platform.

\ni (2) The effectiveness and the room for improvement of sub optimal \lsa algorithms compared with the optimal \lsa algorithms on trajectory data are not comprehensively studied.

\ni (3) The ``sector intersection" algorithms \cite{Williams:Longest, Sklansky:Cone, Dunham:Cone, Zhao:Sleeve} developed in fields of graphic, cartographic and pattern recognition, which run fast as well as have good performance, are still not introduced to the field of trajectory compression.

\ni (4) The recent important progresses in the field of trajectory simplification, \ie online algorithms \squishe~\cite{Muckell:Compression} using \sed and \bqsa~\cite{Liu:BQS} using \ped, and one-pass algorithms \operb~\cite{Lin:Operb} using \ped and \cised~\cite{Lin:Cised} using \sed, are not systematically evaluated for trajectory data.

%\end{enumerate}

As a consequence, it is difficult for a practitioner to decide which algorithm and distance metric should be adopted for a specific application concerning the compression ratio and error of the result data, and the execution time of these compression algorithms.
%
Thus, these issues call for a more comprehensive evaluation of the existing \lsa techniques. % on trajectory data.

\stitle{\textcolor{blue}{Contributions}}.
This paper presents an experimental comparison of the state-of-the-art \lsa algorithms for trajectory compression, including \emph{both the optimal and the sub-optimal methods that use either \ped or \sed or \dad}.

We organize \lsa algorithms into categories, % based on the nature of them,
review algorithms of each category and evaluate the most distinct and/or recent algorithms of each category.
The algorithms reviewed and evaluated in our experiments are summarized in \mytable{tab:summary-lsa}, namely,
(i) \opt\cite{Chan:Optimal, Imai:Optimal}, a naive optimal \lsa algorithm.
(ii) Douglas-Peucker\cite{Douglas:Peucker,Meratnia:Spatiotemporal} and \pavlidis~\cite{Pavlidis:Segment}, two distinct batch \lsa algorithms,
(iii) \bqsa\cite{Liu:BQS} and \squishe~\cite{Muckell:SQUISH}, two famous online \lsa algorithms, and
(iv) \operb\cite{Lin:Operb}, sector intersection (\siped) \cite{Williams:Longest,Sklansky:Cone,Dunham:Cone, Zhao:Sleeve}, \cised \cite{Lin:Cised} and \interval\cite{Ke:Interval}, four one-pass \lsa algorithms.

%
%
Using a variety of real trajectory datasets, we evaluated the performance of each technique in terms of its processing time, compression ratio and average error.
%with up to \textcolor{red}{$2.5$} billion data points
Our experimental results reveal the characteristics of different techniques.
\textcolor{red}{In addition, some of our findings are not reported in preview work and some are the supplements of or in contradiction with the preview works such as \cite{Zhang:Evaluation}.}

Based on these findings, we provide guidelines on selecting appropriate methods and distance metrics for various scenarios.

\stitle{Roadmap}.
The remainder of the paper is organized as follows.
Section 2 introduces notations used in the paper and states the problem of trajectory simplification,
Section 3 and Section 4 review the optimal and sub-optimal \lsa methods, respectively, 
Section 5 reports the experimental results, followed by
Section 6 that concludes this paper.
%, and the appendix covers additional related work and experimental results.% on algorithm \nopts compared with \opt.



