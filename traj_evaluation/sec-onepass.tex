
\vspace{-1ex}
\subsection{One-pass Algorithms}


{One-pass algorithms that enforce the local checking policy}.
The local checking policy \eat{, the key to achieve the \emph{one-pass} processing,} does not need a window to buffer the preview read points. 
Instead, it processes each point in a trajectory once and only once. % when compressing the trajectory.
Obviously, the one-pass algorithms have a linear time complexity.

Reumann-Witkam (\rwa) routine \cite{Reumann:Strip} is a straightforward one-pass algorithm that builds a strip paralleling to the line connecting the first two points, then the points within this strip compose a section of the line. \rwa runs fast, but it is limited in compression ratio.
%
The Linear Dead Reckoning (\ldr) method \cite{Lange:Tracking} for position tracking follows the similar routine as \rwa except that it uses \sed and assumes a velocity ${\vv{v}}$ for each section. In addition, the authors of \cite{Trajcevski:DDR} proved that \ldr is also suitable for online spatio-temporal compression. % as long as the tolerance threshold is set to $\epsilon/2$.
However, its compression ratios are poor because both the value and the direction of velocity ${\vv{v}}$ are pre-defined and fixed between two updates.
%
The authors of this paper improved the \rwa routine by essentially allow the angle of a strip be dynamically adjustable, and developed the One-Pass ERror Bounded (\operb) algorithm~\cite{Lin:Operb} coupled with several detailed optimization techniques.

There are also {sector intersection (SI)} based algorithms, developed in fields of graphic, cartographic and pattern recognition, that can be easily adopted for trajectory compression although \textit{they have been overlooked by existing trajectory simplification studies}.
Williams~\cite{Williams:Longest} and Sklansky and Gonzalez \cite{Sklansky:Cone} proposed linear time algorithms based on the idea of ``sector intersection", Dunham \cite{Dunham:Cone} extend these algorithms, and the Sleeve algorithm \cite{Zhao:Sleeve} in the cartographic discipline
essentially applies the same idea as the SI algorithm.
These SI algorithms are \ped specific, and run fast as well as have comparable compression ratios in comparing with the \dpa algorithm using \ped.
%
%Moreover, {fast \bqsa \cite{Liu:BQS} (\fbqsa in short), the simplified version of \bqsa, has a linear time complexity.}
%

In addition, the authors of this paper recently extends the \textit{sector intersection} method from a 2D space to a Spatio-Temporal 3D space, and develop a new one-pass and \sed enabled algorithm \cised that runs fast as well as has comparable compression ratio with the \dpa algorithm using \sed.

We next review algorithm \operb, sector intersection method and algorithm \cised evaluated in out experiments.



\eat{%%%%%%%%%%%%%%%%%%%%%%%%%%%%%%%%%%%%%%%%%%%%%%%%%%%%%%%
	
The $n^{th}$ point routine and the routine of random-selection of points \cite{Shi:Survey} are two naive one-pass algorithms.
In these routines, for every fixed number of consecutive points along the line, the $n^{th}$ point and one random point among them are retained, respectively.
They run fast, however, they are not error bounded.
	
\subsubsection{Reumann-Witkam and LDR}

\begin{figure*}[tb!]
\centering
\includegraphics[scale=0.66]{Figures/Fig-LDR.png}
\vspace{-1ex}
\caption{\small The trajectory $\dddot{\mathcal{T}}[P_0, \ldots, P_{10}]$ is compressed by the Reumann-Witkam and Linear Dead Reckoning algorithms to four and eight line segments, respectively.}
\vspace{-2ex}
\label{fig:ldr}
\end{figure*}

In Reumann-Witkam\cite{Reumann:Strip}, the input data is divided into sections by strips.
Initially, the first strip, with the width of $2*\epsilon$, takes the line $\vv{P_0P_1}$ connecting the first two points, $P_0$ and $P_1$ as its middle line.
Then the strip is expending over the line into the direction of its initial tangent, covering the succeed points, $P_2, \ldots, P_{j}$, until the strip hits the line $\vv{P_jP_{j+1}}$ (meaning that the next point $P_{j+1}$, $j>1$, is out side of the strip).
The points, $[P_0, \ldots, P_{j}]$, within this strip compose a section. The first and last points of the section, \ie $P_0,P_{j}$, are output, and those points between them are removed.
The last point $P_{j}$ is the initial point of the next strip.
The whole process is repeated until the strip contains the end point $P_n$ of the input data.
The Reumann-Witkam is a one-pass algorithm.

{The Linear Dead Reckoning (LDR)\cite{Lange:Tracking} for position tracking follows the similar routine as the Reumann-Witkam algorithm except that it assumes a velocity ${\vv{v}}$ for each section and uses \sed instead of \ped in distance checking.
Moreover, the authors of \cite{Trajcevski:DDR} proved that LDR is also suitable for online spatio-temporal compression as long as the tolerance threshold of the algorithm is set to $\epsilon/2$.}

\begin{example}
\label{exm-alg-strip}
In Figure~\ref{fig:ldr}, the trajectory $\dddot{\mathcal{T}}[P_0, \ldots, P_{10}]$ is compressed
%
(1) by the Reumann-Witkam to four line segments $\vv{P_0P_2}$, $\vv{P_2P_4}$, $\vv{P_4P_7}$ and $\vv{P_7P_{10}}$. First, a strip with width $2\epsilon$ is built parallel to the line $\vv{P_0P_1}$, then the strip is extended over the line and includes point $P_2$. Because $P_3$ is outside of the strip, $P_2$ becomes the end point of the first section and the start point of the second section.
%
(2) by the Linear Dead Reckoning algorithm to eight line segments $\vv{P_0P_1}$, $\vv{P_1P_2}$, $\vv{P_2P_3}$, $\vv{P_3P_4}$, $\vv{P_4P_5}$, $\vv{P_5P_7}$, $\vv{P_7P_8}$ and $\vv{P_8P_{10}}$. First, an initial velocity ${\vv{v}_0}$ is set to $|P_0P_1|/(t_1-t_0)$. Then the synchronized point $P'_2$ of $P_2$ is estimated based on the velocity ${\vv{v}_0}$ and time of $P_2$, \ie ${v}_0 * (t_2-t_0)$. Because the \sed from $P_2$ to the line $\vv{P_0P'_2}$ , \ie $|P_2P'_2|$, is great than $\epsilon/2$, the algorithm outputs $\vv{P_0P_1}$ and starts the next section.
\end{example}

}%%%%%%%%%%%%%%%%%%%%%%%%%%%%%%%%%%%%%%%%%%%%%%%%%%%%%%%End of Eat

\subsubsection{\operb Using \ped}
\label{sec-operb}


%Algorithm \operb, after initializing, repeatedly processes the data points in $\dddot{\mathcal{T}}[P_0,$ $\ldots, P_{n}]$ one by one until that all data points have been considered.
Consider an error bound $\epsilon$ and a sub-trajectory $\dddot{\mathcal{T}_s}[P_s,$ $\ldots, P_{s+k}]$.
\operb dynamically maintains a directed line segment $\mathcal{L}_i$ ($i\in[1,k]$), whose start point is fixed with $P_s$ and its end point is identified (may not in $\{P_s, \ldots, P_{s+i}\}$) to {\em fit} all the previously processed points $\{P_s, \ldots, P_{s+i}\}$.
The directed line segment $\mathcal{L}_i$ is built by a function named \emph{fitting function $\mathbb{F}$}, such that when a new point $P_{s+i+1}$ is considered, only its distance to the directed line segment $\mathcal{L}_i$ is checked, instead of checking the distances of all or a subset of data points of $\{P_{s}, \ldots, P_{s+i}\}$ to $\mathcal{R}_{i+1}$ = $\vv{P_sP_{s+i+1}}$ as the global distance checking does.
During processing, if the distance of point $P_{s+i}$ to the directed line segment $\mathcal{L}_{i-1}$ is larger than the threshold, then a directed line segment, start from $P_s$, is generated and output;
otherwise, the directed line segment $\mathcal{L}_i$ is updated by the fitting function $\mathbb{F}$, as follows.

\begin{small}
	\vspace{-2ex}
	\begin{equation*}
	\label{equ-function}
	\hspace{-1.5ex}\left\{
	\begin{aligned}
	&\hspace{-1.5ex}\left[
	\begin{aligned}
	% & |\mathcal{L}_{i}| = |\mathcal{R}_{i-1}|    \\
	% & \mathcal{L}_{i}.\theta = \mathcal{R}_{i-1}.\theta\\
	& \mathcal{L}_{i} = \mathcal{L}_{i-1}\\
	\end{aligned}
	\right]\hspace{12.5ex}~when~(|\mathcal{R}_{i}| - |\mathcal{L}_{i-1}|) \le \frac{\epsilon}{4}   \\
	&\hspace{-1.5ex}\left[
	\begin{aligned}
	& |\mathcal{L}_{i}|  = j*{\epsilon}/{2} \\
	& \mathcal{L}_{i}.\theta = \mathcal{R}_{i}.\theta    \\
	\end{aligned}
	\right]\hspace{8.5ex}~when~|\mathcal{R}_{i}| >  \frac{\epsilon}{4}~\And~|\mathcal{L}_{i-1}|=0    \\
	&\hspace{-1.5ex}\left[
	\begin{aligned}
	& |\mathcal{L}_{i}|  = j*{\epsilon}/{2}\\
	& \mathcal{L}_{i}.\theta = \mathcal{L}_{i-1}.\theta + f(\mathcal{R}_i,\mathcal{L}_{i-1})*\arcsin(\frac{ped(P_{s+i}, \mathcal{L}_{i-1})}{j*\epsilon/2})/j \\	
	% & \theta^- = \mathcal{L}_{i-1}.\theta - \arcsin(\frac{d(P_i, \mathcal{L}_{i-1})}{j*\epsilon/2})/j \\	
	% & \mathcal{L}_{i}.\theta = \arg_{\mathcal{L}_{i}.\theta}\min({d(P_{i+1}, \mathcal{L}_{i}}), \mathcal{L}_{i}.\theta \in\{\theta^+,\theta^-\})\\	
	\end{aligned}
	\right]\hspace{0ex}else\\
	\end{aligned}
	\right.
	\end{equation*}
	\vspace{-2ex}
\end{small}


\ni where (a) $1 \le i \le k+1$; (b) $\mathcal{R}_{i-1}$ = $\vv{P_sP_{s+i-1}}$, is the directed line segment whose end point $P_{s+i-1}$ is in $\dddot{\mathcal{T}_s}[P_s, \ldots, P_{s+k}]$; (c) $\mathcal{L}_{i}$ is the directed line segment built by fitting function $\mathbb{F}$ to fit sub-trajectory $\dddot{\mathcal{T}_s}[P_s, \ldots, P_{s+i}]$ and $\mathcal{L}_{0}$ = $\mathcal{R}_{0}$; (d) $j = \lceil(|\mathcal{R}_{i}|*2/\epsilon - 0.5)\rceil$; (e) $f()$ is a sign function such that $ f(\mathcal{R}_i,\mathcal{L}_{i-1})$ = $1$ if the included angle $\angle(\mathcal{R}_{i-1}, \mathcal{R}_{i})$ = $(\mathcal{R}_i.\theta - \mathcal{L}_{i-1}.\theta)$ falls in the range of $(-2\pi, -\frac{3\pi}{2}]$, $[-\pi, -\frac{\pi}{2}]$, $[0, \frac{\pi}{2}]$ and $[\pi, \frac{3\pi}{2})$, and $f(\mathcal{R}_i,\mathcal{L}_{i-1})$ = $-1$, otherwise; (f) $\epsilon/2$ is a step length to control the increment of $|\mathcal{L}|$.

Optimization techniques are equipped to achieve a better compression ratio\cite{Lin:Operb}.
Algorithm \operb runs in $O(n)$ time and takes $O(1)$ space, and it only supports \ped.


\begin{example}
	\label{exm-alg-operb}
	Figure~\ref{fig:operb} is a running example of the \operb algorithm compressing the same trajectory $\dddot{\mathcal{T}}[P_0, \ldots, P_{10}]$.
	(1) It takes $P_0$ as the start point, reads $P_1$ and sets $\mathcal{L}_1$ = $\vv{P_0P_1}$.
	(2) It reads $P_2$. The distance from $P_2$ to $\mathcal{L}_1$ is less than the threshold, thus, it updates $\mathcal{L}_1$  to $\mathcal{L}_2$ by the fitting function $\mathbb{F}$.
	(3) It reads $P_3$ and $P_4$, and updates $\mathcal{L}_2$ to $\mathcal{L}_3$ and $\mathcal{L}_3$ to $\mathcal{L}_4$, respectively.
	(4) It reads $P_5$. The distance from $P_5$ to $\mathcal{L}_4$ is larger than the threshold, thus, it outputs $\vv{P_0P_4}$ and start the next section taking $P_4$ as the new start point.
	(5) The process continues until all points have been processed. At last, the algorithm outputs two continuous line segments $\vv{P_0P_4}$ and $\vv{P_4P_{10}}$.
\end{example}

\begin{figure}[tb!]
	\centering
	\includegraphics[scale=0.66]{Figures/Fig-OPER.png}
	\vspace{-5ex}
	\caption{\small The trajectory $\dddot{\mathcal{T}}[P_0, \ldots, P_{10}]$ is compressed by the \operb algorithm using \ped to two line segments.}
	\vspace{-2ex}
	\label{fig:operb}
\end{figure}


\vspace{-0.5ex}
\subsubsection {Sector Intersection Using \ped}
\label{sec-siped}

%The Sector intersection algorithms using \ped (\siped) \cite{Williams:Longest,Sklansky:Cone,Dunham:Cone}, also named \sleeve in \cite{Zhao:Sleeve}, are {the angle-adjustable strips too}.
%Note that the ``cone intersection" algorithms, developed in fields of graphic, cartographic and pattern recognition, are still not familiar to researchers of trajectory compression.
Given a sequence of points $[P_{s}, P_{s+1}, \ldots, P_{s+k}]$ and an error bound $\epsilon$, the \siped approach processes the input points in order and produces a simplified poly-line. Instead of using the distance tolerance $\epsilon$ directly as the distance threshold, \siped converts the distance tolerance into a variable angle tolerance for testing the points.

For the start data point $P_s$, any point $P_{s+i}$ and $|\vv{P_sP_{s+i}}|>\epsilon$ ($i\in[1, k]$), there are two directed lines $\vv{P_sP^u_{s+i}}$ and $\vv{P_sP^l_{s+i}}$ such that $ped(P_{s+i}, \vv{P_sP^u_{s+i}})$ $=$ $ped(P_{s+i}, \vv{P_sP^l_{s+i}}) = \epsilon$ and either ($\vv{P_sP^l_{s+i}}.\theta < \vv{P_sP^u_{s+i}}.\theta ~and~\vv{P_sP^u_{s+i}}.\theta - \vv{P_sP^l_{s+i}}.\theta <\pi$) or ($\vv{P_sP^l_{s+i}}.\theta > \vv{P_sP^u_{s+i}}.\theta ~and~ \vv{P_sP^u_{s+i}}.\theta - \vv{P_sP^l_{s+i}}.\theta < -\pi)$. Indeed, they forms a \emph{sector} \sector{(P_s, P_{s+i}, \epsilon)} that takes $P_s$ as the center point and $\vv{P_sP^u_{s+i}}$ and $\vv{P_sP^l_{s+i}}$ as the border lines.
%
Then there exists a data point $Q$ such that for any data point $P_{s+i}$ ($i \in [1, ... k]$), its perpendicular Euclidean distance to
directed line $\overline{P_sQ}$ is no greater than the error bound $\epsilon$ if and only if the $k$ sectors \sector{(P_s, P_{s+i}, \epsilon)} ($i\in[1,k]$) share common data points other than $P_s$, \ie $\bigsqcap_{i=1}^{k}$\sector{(P_s, P_{s+i}, \epsilon)} $\ne \{P_s\}$ \cite{Williams:Longest, Sklansky:Cone,Zhao:Sleeve}.
%
The point $Q$ may not belong to $\{P_{s}, P_{s+1},$ $\ldots, P_{s+k}\}$.
However, if $P_{s+i}$ ($1\le i\le k$) is chosen as $Q$, then
for any data point $P_{s+j}$ ($j \in [1, ... i]$), its perpendicular Euclidean distance to
line segment $\overline{P_sP_{s+i}}$ is no greater than the error bound $\epsilon$ if and only if $\bigsqcap_{j=1}^{i}$\sector{(P_s, P_{s+j}, \epsilon/2)} $\ne \{P_s\}$, as pointed out in \cite{Zhao:Sleeve}. 
%That is, {\em these sector intersection based algorithms can be easily adopted for trajectory compression}.

Algorithm \siped runs in $O(n)$ time and takes $O(1)$ space, and \siped also only supports \ped.


\begin{figure}[tb!]
	\centering
	\includegraphics[scale=0.66]{Figures/Fig-sleeve.png}
	\vspace{-5ex}
	\caption{\small The trajectory $\dddot{\mathcal{T}}$ is compressed by the sector intersection algorithm using \ped to two line segments.}
	\vspace{-2ex}
	\label{fig:sleeve}
\end{figure}


\begin{example}
	\label{exm-alg-sleeve}
	Figure~\ref{fig:sleeve} is a running example of the narrow \emph{sector} method taking as input the same trajectory $\dddot{\mathcal{T}}[P_0, \ldots, P_{10}]$. At the beginning, $P_0$ is the first start point, and points $P_1$, $P_2$, $P_3$, etc., each has a narrow \emph{sector}.
	For example, the narrow \emph{sector} $\mathcal{S}$($P_0$, $P_{3}$, $\epsilon/2$) takes $P_0$ as the center point and $\vv{P_0P^u_{3}}$ and $\vv{P_0P^l_{3}}$ as the border lines.
	Because $\bigsqcap_{i=1}^{4}\mathcal{S}(P_0, P_{0+i}, \epsilon/2) \ne \{P_0\}$ and $\bigsqcap_{i=1}^{5}\mathcal{S}(P_0, P_{0+i}, \epsilon/2) = \{P_0\}$, $\vv{P_0P_4}$ is output and $P_4$ becomes the start point of the next section.
	At last, the algorithm outputs two continuous line segments $\vv{P_0P_4}$ and $\vv{P_4P_{10}}$.
\end{example}



\vspace{-0.5ex}
\subsubsection {Cone Intersection Using \sed}
\label{sec-cised}

Given a sub-trajectory $[P_s,...,P_{s+k}]$ and an error bound $\epsilon$, any point $P'_{s+i}$, $0< i \le k$, on the plane $P.t-P_{s+i}.t = 0$ is a synchronized data point of $P_{s+i}$. For all $P'_{s+i}$ in the plane satisfying $|P_{s+i}P'_{s+i}| \le \epsilon$, they form a \textit{synchronous circle $\mathcal{O}(P_{s+i}, \epsilon)$} of $P_{s+i}$ with $P_{s+i}$ as its center and $\epsilon$ as its radius.
%
A spatio-temporal cone (or simply \textit{cone}) of a data point $P_{s+i}$ ($1\le i\le k$) in $\dddot{\mathcal{T}}_s$ \wrt a point $P_s$ and an error bound $\epsilon$, denoted as \cone{(P_s, \mathcal{O}(P_{s+i}, \epsilon))}, or \cone{_{s+i}} in short, is an oblique circular cone such that point $P_s$ is its apex and the synchronous circle $\mathcal{O}(P_{s+i}, \epsilon)$ is its base (Figure~\ref{fig:cis}).
%
Then, there exists a point $Q$ such that $Q.t = P_{s+k}.t$ and $sed(P_{s+i}, \vv{P_sQ})\le \epsilon$ for each $i \in [1,k]$ if and only if $\bigsqcap_{i=1}^{k}$\cone{(P_s, \mathcal{O}(P_{s+i}, \epsilon))} $\ne \{P_s\}$.
%
Algorithm \cised applies the idea of spatio-temporal cone intersection and runs in the similar routine as \siped. And its strong simplification version also applies the half $\epsilon$ in building a cone.

In addition, because these spatio-temporal cones have the same apex $P_s$, the checking of their intersection can be computed by a much simpler way, \ie the checking of intersection of cone projection circles on a plane (Figure~\ref{fig:cis}), and a circle is further approximated with its $m$-edge inscribed regular polygon, whose intersection can be computed more efficiently.
Finally, \cised achieves $O(n)$ time and $O(1)$ space. 

	
\begin{figure}[tb!]
	\centering
	\includegraphics[scale=0.66]{Figures/Fig-CIS.png}
	\vspace{-3ex}
	\caption{\small Examples of spatio-temporal cones.} % in a 3D Cartesian coordinate system
	\vspace{-2ex}
	\label{fig:cis}
\end{figure}


%%%%%%%%%%%%%%% example of Algorithm CISED
\begin{figure}[tb!]
	\centering
	\includegraphics[scale=0.66]{Figures/Fig-Conest.png}
	\vspace{-3ex}
	\caption{\small A running example of the \cised algorithm. The points and the oblique circular cones are projected on an x-y space. }%The trajectory $\dddot{\mathcal{T}}[P_0, \ldots, P_{10}]$ is compressed into four line segments.
	\vspace{-3ex}
	\label{fig:exm-const}
\end{figure}
%%%%%%%%%%%%%%%%


\begin{example}
\label{exm-alg-conest}
	Figure~\ref{fig:exm-const} shows a running example of \cised for compressing the trajectory \trajec{T} in Figure~\ref{fig:notations}. 
	For convenience, we project the points and the oblique circular cones on a x-y space.
%	
	(1) After initialization, the \cised algorithm reads point $P_1$ and builds a narrow \emph{oblique circular cone}~\cone{(P_0, \mathcal{O}(P_{1}, \epsilon/2))}, taking $P_0$ as its apex and \circle{(P_1, \epsilon/2)} as its base (green dash). The \emph{circular cone} is projected on the plane $P.t-P_1.t=0$, and the inscribe regular polygon $\mathcal{R}_1$ of the projection circle is returned. As $\mathcal{R}^*$ is empty, $\mathcal{R}^*$ is set to $\mathcal{R}_1$.
%	
	(2) The algorithm reads $P_2$ and builds \cone{(P_0, \mathcal{O}(P_{2}, \epsilon/2))} (red dash). The \emph{circular cone} is also projected on the plane $P.t-P_1.t=0$ and the inscribe regular polygon $\mathcal{R}_2$ of the projection circle is returned. As $\mathcal{R}^*=\mathcal{R}_1$ is not empty, $\mathcal{R}^*$ is set to the intersection of $\mathcal{R}_2$ and $\mathcal{R}^*$, which is $\mathcal{R}_1 \bigsqcap \mathcal{R}_2 \ne \emptyset$.
%	
	(3) For point $P_3$, the algorithm runs the same routine as $P_2$ until the intersection of $\mathcal{R}_3$ and $\mathcal{R}^*$ is $\emptyset$. Thus, a line segment $\vv{P_0P_2}$ is generated, and the process of a new line segment is started, taking $P_2$ as the new start point and $P.t-P_3.t=0$ as the new projection plane.
%	
	(4) At last, the algorithm outputs four continuous line segments, \ie $\{\vv{P_0P_2}$, $\vv{P_2P_4}$, $\vv{P_4P_{7}}$, $\vv{P_7P_{10}}\}$. %\eop
\end{example} 	
\vspace{-1ex}


