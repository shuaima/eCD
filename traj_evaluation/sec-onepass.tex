

\subsection{One-pass Algorithms}


\emph{one-pass algorithms that enforce the local checking policy}.
The local checking policy, the key to achieve the \emph{one-pass} processing, does not need a window to buffer the preview read points.
%Meanwhile, a trajectory compression algorithm is {\em one-pass} if it processes each point in a trajectory once and only once when compressing the trajectory.
Obviously, the one-pass algorithms have linear time complexity.
%
The $n^{th}$ point routine and the routine of random-selection of points \cite{Shi:Survey} are two naive one-pass algorithms.
In these routines, for every fixed number of consecutive points along the line, the $n^{th}$ point and one random point among them are retained, respectively.
They run fast, however, they are not error bounded.
%
In Reumann-Witkam routine\cite{Reumann:Strip}, it builds a strip paralleling to the line connecting the first two points, then the points within this strip compose one section of the line.
The Reumann-Witkam routine also runs fast but it is limited in compression ratio.
%
Williams~\cite{Williams:Longest} and Sklansky and Gonzalez \cite{Sklansky:Cone} proposed linear time algorithms based one the idea of ``cone intersection", Dunham \cite{Dunham:Cone} and Zhao \cite{Zhao:Sleeve} further extend these algorithms.
These algorithms run fast as well as have comparable compression ratio with the \dpa and \pavlidis algorithms.
%
Moreover, {fast \bqsa \cite{Liu:BQS} (\fbqsa in short), the simplified version of \bqsa, has a linear time complexity.}
%
The authors of this paper have also developed the One-Pass ERror Bounded (\operb) algorithm\cite{Lin:Operb}.




\eat{%%%%%%%%%%%%%%%%%%%%%%%%%%%%%%%%%%%%%%%%%%%%%%%%%%%%%%%
\subsubsection{Reumann-Witkam and LDR}

\begin{figure*}[tb!]
\centering
\includegraphics[scale=0.66]{figures/Fig-LDR.png}
\vspace{-1ex}
\caption{\small The trajectory $\dddot{\mathcal{T}}[P_0, \ldots, P_{10}]$ is compressed by the Reumann-Witkam and Linear Dead Reckoning algorithms to four and eight line segments, respectively.}
\vspace{-2ex}
\label{fig:ldr}
\end{figure*}

In Reumann-Witkam\cite{Reumann:Strip}, the input data is divided into sections by strips.
Initially, the first strip, with the width of $2*\epsilon$, takes the line $\vv{P_0P_1}$ connecting the first two points, $P_0$ and $P_1$ as its middle line.
Then the strip is expending over the line into the direction of its initial tangent, covering the succeed points, $P_2, \ldots, P_{j}$, until the strip hits the line $\vv{P_jP_{j+1}}$ (meaning that the next point $P_{j+1}$, $j>1$, is out side of the strip).
The points, $[P_0, \ldots, P_{j}]$, within this strip compose a section. The first and last points of the section, \ie $P_0,P_{j}$, are output, and those points between them are removed.
The last point $P_{j}$ is the initial point of the next strip.
The whole process is repeated until the strip contains the end point $P_n$ of the input data.
The Reumann-Witkam is a one-pass algorithm.

{The Linear Dead Reckoning (LDR)\cite{Lange:Tracking} for position tracking follows the similar routine as the Reumann-Witkam algorithm except that it assumes a velocity ${\vv{v}}$ for each section and uses \sed instead of \ped in distance checking.
Moreover, the authors of \cite{Trajcevski:DDR} proved that LDR is also suitable for online spatio-temporal compression as long as the tolerance threshold of the algorithm is set to $\epsilon/2$.}

\begin{example}
\label{exm-alg-strip}
In Figure~\ref{fig:ldr}, the trajectory $\dddot{\mathcal{T}}[P_0, \ldots, P_{10}]$ is compressed
%
(1) by the Reumann-Witkam to four line segments $\vv{P_0P_2}$, $\vv{P_2P_4}$, $\vv{P_4P_7}$ and $\vv{P_7P_{10}}$. First, a strip with width $2\epsilon$ is built parallel to the line $\vv{P_0P_1}$, then the strip is extended over the line and includes point $P_2$. Because $P_3$ is outside of the strip, $P_2$ becomes the end point of the first section and the start point of the second section.
%
(2) by the Linear Dead Reckoning algorithm to eight line segments $\vv{P_0P_1}$, $\vv{P_1P_2}$, $\vv{P_2P_3}$, $\vv{P_3P_4}$, $\vv{P_4P_5}$, $\vv{P_5P_7}$, $\vv{P_7P_8}$ and $\vv{P_8P_{10}}$. First, an initial velocity ${\vv{v}_0}$ is set to $|P_0P_1|/(t_1-t_0)$. Then the synchronized point $P'_2$ of $P_2$ is estimated based on the velocity ${\vv{v}_0}$ and time of $P_2$, \ie ${v}_0 * (t_2-t_0)$. Because the \sed from $P_2$ to the line $\vv{P_0P'_2}$ , \ie $|P_2P'_2|$, is great than $\epsilon/2$, the algorithm outputs $\vv{P_0P_1}$ and starts the next section.
\end{example}

}%%%%%%%%%%%%%%%%%%%%%%%%%%%%%%%%%%%%%%%%%%%%%%%%%%%%%%%End of Eat

\subsubsection{\operb Using \ped}
\label{sec-operb}

\operb\cite{Lin:Operb} is essentially the angle-adjustable strip\cite{Reumann:Strip}.
%Algorithm \operb, after initializing, repeatedly processes the data points in $\dddot{\mathcal{T}}[P_0,$ $\ldots, P_{n}]$ one by one until that all data points have been considered.
Consider an error bound $\epsilon$ and a sub-trajectory $\dddot{\mathcal{T}_s}[P_s,$ $\ldots, P_{s+k}]$.
\operb dynamically maintains a directed line segment $\mathcal{L}_i$ ($i\in[1,k]$), whose start point is fixed with $P_s$ and its end point is identified (may not in $\{P_s, \ldots, P_{s+i}\}$) to {\em fit} all the previously processed points $\{P_s, \ldots, P_{s+i}\}$.
The directed line segment $\mathcal{L}_i$ is built by a function named \emph{fitting function $\mathbb{F}$}, such that when a new point $P_{s+i+1}$ is considered, only its distance to the directed line segment $\mathcal{L}_i$ is checked, instead of checking the distances of all or a subset of data points of $\{P_{s}, \ldots, P_{s+i}\}$ to $\mathcal{R}_{i+1}$ = $\vv{P_sP_{s+i+1}}$ as the global distance checking does.
During processing, if the distance of point $P_{s+i}$ to the directed line segment $\mathcal{L}_{i-1}$ is larger than the threshold, then a directed line segment, start from $P_s$, is generated and output;
otherwise, the directed line segment $\mathcal{L}_i$ is updated by the fitting function $\mathbb{F}$, as follows.

\begin{small}
	\vspace{-2ex}
	\begin{equation*}
	\label{equ-function}
	\hspace{-1.5ex}\left\{
	\begin{aligned}
	&\hspace{-1.5ex}\left[
	\begin{aligned}
	% & |\mathcal{L}_{i}| = |\mathcal{R}_{i-1}|    \\
	% & \mathcal{L}_{i}.\theta = \mathcal{R}_{i-1}.\theta\\
	& \mathcal{L}_{i} = \mathcal{L}_{i-1}\\
	\end{aligned}
	\right]\hspace{12.5ex}~when~(|\mathcal{R}_{i}| - |\mathcal{L}_{i-1}|) \le \frac{\epsilon}{4}   \\
	&\hspace{-1.5ex}\left[
	\begin{aligned}
	& |\mathcal{L}_{i}|  = j*{\epsilon}/{2} \\
	& \mathcal{L}_{i}.\theta = \mathcal{R}_{i}.\theta    \\
	\end{aligned}
	\right]\hspace{8.5ex}~when~|\mathcal{R}_{i}| >  \frac{\epsilon}{4}~\And~|\mathcal{L}_{i-1}|=0    \\
	&\hspace{-1.5ex}\left[
	\begin{aligned}
	& |\mathcal{L}_{i}|  = j*{\epsilon}/{2}\\
	& \mathcal{L}_{i}.\theta = \mathcal{L}_{i-1}.\theta + f(\mathcal{R}_i,\mathcal{L}_{i-1})*\arcsin(\frac{ped(P_{s+i}, \mathcal{L}_{i-1})}{j*\epsilon/2})/j \\	
	% & \theta^- = \mathcal{L}_{i-1}.\theta - \arcsin(\frac{d(P_i, \mathcal{L}_{i-1})}{j*\epsilon/2})/j \\	
	% & \mathcal{L}_{i}.\theta = \arg_{\mathcal{L}_{i}.\theta}\min({d(P_{i+1}, \mathcal{L}_{i}}), \mathcal{L}_{i}.\theta \in\{\theta^+,\theta^-\})\\	
	\end{aligned}
	\right]\hspace{0ex}else\\
	\end{aligned}
	\right.
	\end{equation*}
	\vspace{-2ex}
\end{small}


\ni where (a) $1 \le i \le k+1$; (b) $\mathcal{R}_{i-1}$ = $\vv{P_sP_{s+i-1}}$, is the directed line segment whose end point $P_{s+i-1}$ is in $\dddot{\mathcal{T}_s}[P_s, \ldots, P_{s+k}]$; (c) $\mathcal{L}_{i}$ is the directed line segment built by fitting function $\mathbb{F}$ to fit sub-trajectory $\dddot{\mathcal{T}_s}[P_s, \ldots, P_{s+i}]$ and $\mathcal{L}_{0}$ = $\mathcal{R}_{0}$; (d) $j = \lceil(|\mathcal{R}_{i}|*2/\epsilon - 0.5)\rceil$; (e) $f()$ is a sign function such that $ f(\mathcal{R}_i,\mathcal{L}_{i-1})$ = $1$ if the included angle $\angle(\mathcal{R}_{i-1}, \mathcal{R}_{i})$ = $(\mathcal{R}_i.\theta - \mathcal{L}_{i-1}.\theta)$ falls in the range of $(-2\pi, -\frac{3\pi}{2}]$, $[-\pi, -\frac{\pi}{2}]$, $[0, \frac{\pi}{2}]$ and $[\pi, \frac{3\pi}{2})$, and $f(\mathcal{R}_i,\mathcal{L}_{i-1})$ = $-1$, otherwise; (f) $\epsilon/2$ is a step length to control the increment of $|\mathcal{L}|$.

Optimization techniques are equipped to achieve a better compression ratio\cite{Lin:Operb}.
Algorithm \operb runs in $O(n)$ time and takes $O(1)$ space.
Note \operb only supports \ped.


\begin{figure}[tb!]
	\centering
	\includegraphics[scale=0.66]{figures/Fig-oper.png}
	\vspace{-1ex}
	\caption{\small The trajectory $\dddot{\mathcal{T}}[P_0, \ldots, P_{10}]$ is compressed by the \operb algorithm to two line segments.}
	\vspace{-2ex}
	\label{fig:operb}
\end{figure}


\begin{example}
	\label{exm-alg-operb}
	Figure~\ref{fig:operb} is a running example of the \operb algorithm.
	(1) It takes $P_0$ as the start point, reads $P_1$ and sets $\mathcal{L}_1$ = $\vv{P_0P_1}$.
	(2) It reads $P_2$. The distance from $P_2$ to $\mathcal{L}_1$ is less than the threshold, thus, it updates $\mathcal{L}_1$  to $\mathcal{L}_2$ by the fitting function $\mathbb{F}$.
	(3) It reads $P_3$ and $P_4$, and updates $\mathcal{L}_2$ to $\mathcal{L}_3$ and $\mathcal{L}_3$ to $\mathcal{L}_4$, respectively.
	(4) It reads $P_5$. The distance from $P_5$ to $\mathcal{L}_4$ is larger than the threshold, thus, it outputs $\vv{P_0P_4}$ and start the next section taking $P_4$ as the new start point.
	(5) The process continues until all points have been processed. At last, the algorithm outputs two continuous line segments $\vv{P_0P_4}$ and $\vv{P_4P_{10}}$.
\end{example}



\subsubsection {Sector Intersection Using \ped}
\label{sec-siped}

The Sector Intersection algorithms using \ped (\sia) \cite{Williams:Longest,Sklansky:Cone,Dunham:Cone}, also named \sleeve in \cite{Zhao:Sleeve}, are {the angle-adjustable strips too}.
%Note that the ``cone intersection" algorithms, developed in fields of graphic, cartographic and pattern recognition, are still not familiar to researchers of trajectory compression.
Given a sequence of points $[P_{s}, P_{s+1}, \ldots, P_{s+k}]$ and an error bound $\epsilon$, the \cia approach processes the input points in order and produces a simplified poly-line.  Moreover, instead of using the distance tolerance $\epsilon$ directly as the distance threshold, \cia converts the distance tolerance into a variable angle tolerance for testing the points.
%
For the start point $P_s$ and any point $P_{s+i}$, $1\le i\le k$, there are two different lines $\vv{P_sP^u_{s+i}}$ and $\vv{P_sP^l_{s+i}}$, satisfying $ped(P_i, \vv{P_sP^u_{s+i}}) = \epsilon$, $ped(P_i, \vv{P_sP^l_{s+i}}) = \epsilon$ and $\vv{P_sP^l_{s+i}}.\theta < \vv{P_sP^u_{s+i}}.\theta$. Indeed, they forms a \emph{cone} $\mathcal{C}$($P_s$, $P_{s+i}$, $\epsilon$) taking $P_s$ as the center point and $\vv{P_sP^u_{s+i}}$ and $\vv{P_sP^l_{s+i}}$ as the border lines.
% , where $P_s$ is the center point, and $\vv{P_sP^l_{s+i}}$ and $\vv{P_sP^l_{s+i}}$ are two bound lines.
Then there exists a point $Q$ such that all points $P_{s+i}$, $i \in [1, ... k]$, have perpendicular distances to
line $\mathcal{L}(P_s,Q)$ within the given \ped tolerance $\epsilon$ if and only if $\bigcap_{i=1}^{k}\mathcal{C}(P_s, P_{s+i}, \epsilon) \ne \phi$.
Note that in this case, point $Q$ may not belong to $[P_{s}, P_{s+1}, \ldots, P_{s+k}]$.
However, if the last point $P_{s+k}$ is chosen as the end point $Q$ of the approximating line segment, then a narrow \emph{cone} with a distance tolerance $\epsilon/2$ could be employed to ensure the algorithm be \ped error bounded by $\epsilon$\cite{Zhao:Sleeve}.

Algorithm \cia runs in $O(n)$ time and takes $O(1)$ space.
\cia also only supports \ped.


\begin{figure}[tb!]
	\centering
	\includegraphics[scale=0.66]{figures/Fig-sleeve.png}
	\vspace{-1ex}
	\caption{\small The trajectory $\dddot{\mathcal{T}}[P_0, \ldots, P_{10}]$ is compressed by the \conei algorithm to two line segments.}
	\vspace{-2ex}
	\label{fig:sleeve}
\end{figure}


\begin{example}
	\label{exm-alg-sleeve}
	Figure~\ref{fig:sleeve} is a running example of narrow \emph{cone} taking a distance tolerance $\epsilon/2$. At the beginning, $P_0$ is the first start point, $P_1$, $P_2$, $P_3$ and so on, each has a narrow \emph{cone}.
	For example, the narrow \emph{cone} $\mathcal{C}$($P_0$, $P_{3}$, $\epsilon/2$) takes $P_0$ as the center point and $\vv{P_0P^u_{3}}$ and $\vv{P_0P^l_{3}}$ as the border lines.
	Because $\bigcap_{i=1}^{4}\mathcal{C}(P_0, P_{s+i}, \epsilon/2) \ne \phi$ and $\bigcap_{i=1}^{5}\mathcal{C}(P_0, P_{s+i}, \epsilon/2) = \phi$, $\vv{P_0P_4}$ is output and $P_4$ becomes the start point of the next section.
	At last, the algorithm outputs two continuous line segments $\vv{P_0P_4}$ and $\vv{P_4P_{10}}$.
\end{example}




\subsubsection {Cone Intersection Using \sed}
\label{sec-cised}

The \cia extends the \textit{sector intersection} method in Section~\ref{sec-siped} from a 2D space to a Spatio-Temporal 3D space.

\stitle{Synchronous Circles (\circle{})}. The synchronous circle of a data point $P_{s+i}$ ($1\le i\le k$) in $\dddot{\mathcal{T}}_s$ \wrt an error bound $\epsilon$, denoted as \circle{(P_{s+i}, \epsilon)}, or \circle{_{s+i}} in short, is a circle on the plane $P.t-P_{s+i}.t = 0$ such that $P_{s+i}$ is its center and $\epsilon$ is its radius.

Figure~\ref{fig:cis} shows two synchronous circles, \circle{(P_{s+i}, \epsilon)} of point $P_{s+i}$ and \circle{(P_{s+k}, \epsilon)} of point $P_{s+k}$.
It is easy to know that for any point in the area of a circle \circle{(P_{s+i}, \epsilon)}, its distance to $P_{s+i}$ is no greater than $\epsilon$.


\stitle{Spatio-temporal cones (\cone{})}. The spatio-temporal cone (or simply \textit{cone}) of a data point $P_{s+i}$ ($1\le i\le k$) in $\dddot{\mathcal{T}}_s$ \wrt a point $P_s$ and an error bound $\epsilon$, denoted as \cone{(P_s, \mathcal{O}(P_{s+i}, \epsilon))}, or \cone{_{s+i}} in short, is an oblique circular cone such that point $P_s$ is its apex and the synchronous circle $\mathcal{O}(P_{s+i}, \epsilon)$ of point $P_{s+i}$ is its base.

Figure~\ref{fig:cis} also illustrates two example spatio-temporal cones: \cone{(P_s, \mathcal{O}(P_{s+i}, \epsilon))} {(purple)} and \cone{(P_s, \mathcal{O}(P_{s+k}, \epsilon))} (red), with the same apex $P_s$ and error bound $\epsilon$.


\begin{figure}[tb!]
	\centering
	\includegraphics[scale=0.66]{figures/Fig-CIS.png}
	%	\vspace{-1ex}
	\caption{\small Examples of spatio-temporal cones in a 3D Cartesian coordinate system, where (1) $P_s$, $P_{s+i}$ and $P_{s+k}$ are three points, (2) \circle{_{s+i}} and \circle{_{s+k}} are two synchronous circles, (3) \cone{_{s+i}} and \cone{_{s+k}} are two spatio-temporal cones, (4) $Q$ is a point in synchronous circle \circle{_{s+k}}, and (5) $P'_{s+i}$ is the intersection point of line $\overline{P_sQ}$ and synchronous circle \circle{_{s+i}}.}
	%\vspace{-1ex}
	\label{fig:cis}
\end{figure}



	Given a sub-trajectory $[P_s, \ldots, P_{s+k}]$ and a point $Q$ in the area of synchronous circle \circle{(P_{s+k}, \epsilon)}, the intersection point $P'_{s+i}$ of the directed line segment $\vv{P_sQ}$ and the plane $P.t - P_{s+i}.t = 0$ is the synchronized point of $P_{s+i}$ ($1\le i\le k$) \wrt  $\vv{P_sQ}$, and the distance $|\vv{P_{s+i}P'_{s+i}}|$ from $P_{s+i}$ to $P'_{s+i}$ is the synchronous distance of $P_{s+i}$ to $\vv{P_sQ}$.
	%in the spatio-temporal space.

	Given a sub-trajectory $[P_s,...,P_{s+k}]$ and an error bound $\epsilon$, there exists a point $Q$ such that $Q.t = P_{s+k}.t$ and $sed(P_{s+i}, \vv{P_sQ})\le \epsilon$ for each $i \in [1,k]$ if and only if $\bigsqcap_{i=1}^{k}$\cone{(P_s, \mathcal{O}(P_{s+i}, \epsilon))} $\ne \{P_s\}$.


For spatio-temporal cones with the same apex $P_s$, the checking of their intersection can be computed by a much simpler way, \ie the checking of intersection of cone projection circles on a plane.


Finding the common intersection of $n$ circles on a plane has a time complexity of ${O(n\log n)}$~\cite{Shamos:Circle}, which cannot be used for designing one-pass trajectory simplification algorithms using \sed.
%
However, we can approximate a circle with its $m$-edge inscribed regular polygon, whose intersection can be computed more efficiently.

