%\vspace{-1ex}
\section*{{Appendix: Additional Trajectory Compression Algorithms}}
\textcolor{blue}{Apart from the techniques evaluated in this article, there exist other approaches for various requirements of trajectory compression, most notable the semantic-based methods.}

\textcolor{blue}{The trajectories of certain moving objects in urban, such as cars and trucks, are constrained by road networks. Hence, there are trajectory compression methods based on road networks \cite{Chen:Trajectory, Popa:Spatio,Civilis:Techniques,Hung:Clustering, Kellaris:Map, Song:PRESS, Han:Compress, Cao:Road} that first project trajectory points onto roads (known as map-matching \cite{Quddus:MapMatching}), then compress the matched data points by \emph{piece-wise linear approximation} method \cite{Elmeleegy:Stream, Xie:Stream,Luo:Streaming,ORourke:Fitting} (dilution-matching-encoding \cite{Gotsman:Compaction} is an exception of such methods in that it first simplifies the original trajectory points by some line simplification method, then it projects the simplified points onto roads).}
%
\textcolor{blue}{Some methods \cite{Schmid:Semantic, Richter:Semantic} compress trajectories making use of other domain knowledge, such as places of interests (POI) along the trajectories \cite{Richter:Semantic}, and some other works \cite{Gotsman:Compaction, Song:PRESS, Han:Compress,Koide:CiNCT} mine and use high frequency patterns of compressed trajectories instead of roads to further improve compression effectiveness.}
%
\textcolor{blue}{One benefit of these works is that, the location information from sensors is usually imprecise, may be noisy and error prone \cite{Cao:Road}, while map-matching is believed able to correct the error by “snapping” data points onto the road network.
Another important benefit is that they are able to represent a trajectory basing on long paths (a path is a sequence of roads connected one by one) and further mine the frequency patterns, so as to get the overall compression trajectories.} 
%
\textcolor{blue}{However, users should take in mind that the effectiveness of these semantic-based methods highly depends on the quality of semantic information (\eg road network) and sequence labeling (\eg map-matching) algorithms. In some cases, such as the road network is not up to date or there is no road nearby or moving objects actually move paralleling to the roads (but outside the roads), these methods may result in incorrect map-matching, and thus, introduce errors.}
%\textcolor{blue}{Besides, sequence labeling has higher time/space consuming compared with line simplification. }
%These limitations make them be better running in server side rather than in resource constraint end devices that some line simplification algorithms are applicable.

\textcolor{blue}{Anyway, the semantic-based methods and the line simplification methods are not necessary contradictory, indeed, they are orthogonal, and may be combined with each other to improve the effectiveness of trajectory compression.}

%\todo{compression ratio or information saved}


