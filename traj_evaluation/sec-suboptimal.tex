\newdef{definition}{Definition}
\newtheorem{theorem}{Theorem}
\newtheorem{lemma}{Lemma}

%%%%%%%%%%%%%%%%%%%%%%%%%%%%%%%%%%%%%%%%%%%%%%%%%%%%%%%%%%%%%%%%%%%%%%%%
\vspace{-1ex}
\section{Sub-Optimal Algorithms}
\label{sec-subopt}
%%%%%%%%%%%%%%%%%%%%%%%%%%%%%%%%%%%%%%%%%%%%%%%%%%%%%%%%%%%%%%%%%%%%%%%%

%In particular, the state-of-the-art the sub-optimal line simplification approaches can be classified into three categories.


This section reviews the seven techniques evaluated in our experiments, namely,
(i) Douglas-Peucker (\dpa)\cite{Douglas:Peucker,Meratnia:Spatiotemporal} and \pavlidis (\tpa)~\cite{Pavlidis:Segment}, two distinct batch algorithms applying the \emph{global checking policy},
(ii) \bqsa\cite{Liu:BQS} and \squishe~\cite{Muckell:SQUISH}, online algorithms applying the \emph{constrained global checking} policy, %\textcolor[rgb]{1.00,0.00,0.00}{\opwa, \swab  and}
and (iii) \operb\cite{Lin:Operb}, sector intersection (\sia)~\cite{Williams:Longest,Sklansky:Cone,Dunham:Cone, Zhao:Sleeve} and spatio-temporal cone intersection using \sed~(\cia)~\cite{Lin:Cised}, one-pass algorithms applying the \emph{local checking} policy.


\subsection{Batch Algorithms}
Batch algorithms essentially apply global distance checking policies for trajectory simplification, and can be either top-down or bottom-up.
Global checking policies enforce batch algorithms to have an entire trajectory first \cite{Meratnia:Spatiotemporal}.

(1) Top-down algorithms recursively divide a trajectory into sub-trajectories until the stopping condition is met.
\textcolor{blue}{Algorithms Ramer \cite{Ramer:Split}} and Douglas-Peucker (\dpa)  \cite{Douglas:Peucker} can support all the three distances \ped, \sed and \dad.
%Algorithm \dpa is modified to further support \sed \cite{Meratnia:Spatiotemporal},
An improved method of \dpa with a time complexity of $O(n\log n)$, based on \emph{convex hulls} is proposed in \cite{Hershberger:Speeding}, which is the best \dpa based  algorithm in terms of time complexities, and is designed for \ped only, not for \sed and \dad.

(2) Bottom-up algorithms are the natural complement of top-down ones, and they recursively merge adjacent sub-trajectories with the smallest distance, initially $n/2$  sub-trajectories for a trajectory with $n$ points, until the stopping condition is met. Note that the distances of newly generated line segments are recalculated in {each} iteration. To our knowledge, Theo-Pavlidis (\tpa) \cite{Pavlidis:Segment} is the only bottom-up \textcolor{blue}{batch} \lsa algorithm for trajectory simplification.

Note that, compared with top-down algorithms, bottom-up algorithms fit better for trajectories with lower sampling rates, as they typically need more rounds to merge smaller line segments into larger line segments. {\em Batch algorithms basically work for small and medium size trajectories, and we choose \dpa and \tpa that all support \ped, \sed and \dad  as the   representatives of  batch \lsa algorithms}.


%We next review algorithms Douglas-Peucker and Theo Pavlidis evaluated in out experiments.




%%%%%%%%%%%%%%%%%%%%%%%%%%%%%%%%%%%%%%%%%%%%%%%%%%%%%%
%\vspace{-1ex}
%\subsubsection{Douglas-Peucker Algorithm}

\eat{
%%%%%%%%%%%%%%%%%%%%%Baseline Algorithm
\begin{figure}[tb!]
	%\vspace{-2ex}
	\vspace{1ex}
	\begin{center}
		{\small
			\begin{minipage}{3.3in}
				\myhrule \vspace{-1ex}
				\mat{0ex}{
					{\bf Algorithm}~$\dpa(\dddot{\mathcal{T}}[P_0,\ldots,P_n], \epsilon)$\\
					\sstab
					\bcc \hspace{1ex}\=\For each point $P_i$ ($i\in[0,n]$) in $\dddot{\mathcal{T}}[P_0, \ldots, P_n]$ \Do\\
					\icc \>\hspace{3ex}compute $ped(P_i, {\mathcal{L}})$ between $P_i$ and ${\mathcal{L}}(P_0, P_n)$;\\
					\icc \> \Let $ped(P_k, {\mathcal{L}})$ := $\max \{ped(P_0, {\mathcal{L}}), \ldots, ped(P_n, {\mathcal{L}}) \}$;\\
					\icc \> \If $ped(P_k, {\mathcal{L}}) \le\epsilon$ \Then \\
					\icc \> \hspace{3ex}\Return $\{\mathcal{L}(P_0,P_n)\}$.\\
					\icc \> \Else\\
					\icc \> \hspace{3ex}\Return $\dpa($\trajec{T}$[P_0, \ldots, P_k], \epsilon)\cup\dpa($\trajec{T}$[P_{k}, \ldots, P_n], \epsilon)$.
				}
				\vspace{-2.5ex}
				\myhrule
			\end{minipage}
		}
	\end{center}
	\vspace{-3ex}
	\caption{\small Basic Douglas-Peucker algorithm}
	\label{alg:dp}
	\vspace{-3ex}
\end{figure}
}
%%%%%%%%%%%%%%%%%%%%%%%%%%%%%%%%%%%%%


\stitle{Algorithm Douglas-Peucker  (\dpa) \cite{Douglas:Peucker}}. It is invented for reducing the number of points required to represent a digitized line or its caricature in the context of computer graphics and image processing.



Given a trajectory $\dddot{\mathcal{T}}[P_0, \ldots, P_n]$ and an error bound $\epsilon$,  algorithm \dpa uses the first point $P_0$ and the last point $P_n$ of \trajec{T} as the start point $P_s$ and the end point $P_e$ of the first line segment $\mathcal{L}(P_0, P_n)$, then it calculates the distance $ped(P_i, {\mathcal{L}})$ for each point $P_i$ ($i\in[0,n]$). If $ped(P_k, {\mathcal{L}})$ = $\max \{ped(P_0, {\mathcal{L}}), \ldots, ped(P_n, {\mathcal{L}}) \} \le \epsilon$, then it returns $\{\mathcal{L}(P_0,P_n)\}$. Otherwise, it divides \trajec{T} into two sub-trajectories \trajec{T}$[P_0, \ldots, P_k]$ and \trajec{T}$[P_{k}, \ldots, P_n]$, and recursively compresses these sub-trajectories until the entire trajectory has been considered.
%
The time complexity of \dpa is $\Omega(n)$ in the best case, but is $O(n^2)$ in the worst case.
%The basic \dpa uses \ped, however, it also supports \sed \cite{Meratnia:Spatiotemporal} that runs in the same routine as using \ped. %, except that $ped$ in
%Figure~\ref{alg:dp} is replace by $sed$.

%The \dpa algorithm are \textcolor[rgb]{0.00,0.07,1.00}{widely considered have excellent compression ratio and accuracy.}
%However, the batch nature and high time complexity make it not suitable for the online scenarios.



%\vspace{-1ex}
%\subsubsection{Theo Pavlidis Algorithm}

\stitle{Algorithm Theo-Pavlidis  (\tpa) ~\cite{Pavlidis:Segment}}. It initially employs the global checking policy to output disjoint line segments, and we slightly modify it to have continuous line segments.

Given a trajectory $\dddot{\mathcal{T}}[P_0, \ldots, P_n]$ and an error bound $\epsilon$,
algorithm \tpa begins by creating the finest possible  trajectory approximation: $[P_0, P_1]$, $[P_1, P_2], \ldots,[P_{n-1}, P_n]$, so that $n$ segments are used to approximate the original trajectory.
Next, for each pair of adjacent segments $[P_{s}, P_{s+j}]$ and $[P_{s+j}, P_{s+k}]$ ($0\le s<s+j < s+k \le n$),
the  distance $ped(P_{s+i}, \vv{P_sP_{s+k}})$ of each point $P_{s+i}$ ($0<i<k$) to the line segment $\vv{P_sP_{s+k}}$, is calculated, and the max distance is saved and denoted as the \emph{cost} of merging them.
Then \tpa begins to iteratively merge the adjacent segment pair with the lowest cost
until no cost is below $\epsilon$.
After the pair of adjacent segments $[P_{s}, P_{s+j}]$ and $[P_{s+j}, P_{s+k}]$ are merged to a new segment $[P_{s}, P_{s+k}]$, \tpa needs to recalculate the costs of the new segment with its preceding and successive segments.
%
Algorithm \tpa runs in $O(n^2/K)$ time, where $K$ is the number of the final segments.

%Furthermore, it is also easy to implement \sed in the \tpa algorithm.
%The \tpa algorithm like \dpa algorithm originally only supports \ped, but it is also easy to be extended to support \sed.


\subsection{Online Algorithms}

Online \lsa algorithms adopt local checking policies by restricting the distance checking within a sliding or opening window such that there is no need to have the entire trajectory ready before compressing. That is, online algorithms essentially combine {\em batch algorithms} with {\em sliding or opening windows}, \eg\
\opwa \cite{Meratnia:Spatiotemporal} is a combination of top-down algorithm \dpa and opening windows while \kw{SWAB} \cite{Keogh:online} is a combination of bottom-up algorithm \tpa and \textit{sliding windows}.
Though these algorithms support the three distance metrics \ped, \sed and \dad, they still have high time and/or space complexities \cite{Liu:BQS}.
To design more efficient online algorithms, techniques typically need to be designed closely coupled with distance metrics.
Indeed, \bqsa \cite{Liu:BQS} and \squishe \cite{Muckell:Compression} propose to utilize convex hulls and priority queues, respectively, and they speed up trajectory simplification using \ped and \sed, respectively. To our knowledge, no specific techniques have been developed for \dad.
Hence, {\em we choose algorithms \bqsa, \squishe and \opwa as the representatives of online algorithms using \ped, \sed and \dad, respectively}.


%, which significantly hinders their utility in resource-constrained mobile devices \cite{Liu:BQS}.
%\bqsa \cite{Liu:BQS} and \squishe\cite{Muckell:Compression} further optimize these online algorithms, respectively.
%Where
%
\eat{
Recently, \bqsa \cite{Liu:BQS} has been proposed, using a new distance checking method by picking out at most eight special points from an open window based on a convex hull, \eg a rectangular bounding box with two bounding lines, so that when a new point is added to a window, it only needs to calculate the distances of the  special points to a line, instead of all data points in the window, in many cases.
The time complexity of \bqsa remains $O(n^2)$ in the worst case, as \bqsa falls back to \dpa when the eight special points cannot be used. However, its simplified version, \fbqsa directly outputs a line segment, and starts a new window when the eight special points cannot bound all the points considered so far. Indeed, \fbqsa has a linear time complexity, and is the fastest \lsa based solution for trajectory compression.
}
%
%We next review the optimized online algorithms \bqsa and \squishe evaluated in out experiments.

%\vspace{-0.5ex}
%\subsubsection{Bounded Quadrant System Using \ped}

\eat{%%%% opening windows
\stitle{{Algorithm \opwa \cite{Meratnia:Spatiotemporal}}.}
{The \opwa algorithm~\cite{Meratnia:Spatiotemporal} combines the Top-down and opening window strategies, and enforces the constrained global checking in the window.}

Given a trajectory $\dddot{\mathcal{T}}[P_0, \ldots, P_n]$ and an error bound $\epsilon$, algorithm \opwa~\cite{Meratnia:Spatiotemporal} maintains a window $W[P_s, \ldots, P_k]$, where $P_s$ and $P_k$ are the start and end points, respectively. Initially, $P_s$ = $P_0$ and $P_k$ = $P_1$, and the window $W$ is gradually expanded by adding new points one by one. \opwa tries to compress all points in $W[P_s, \ldots, P_k]$ to a single line segment $\mathcal{L}(P_{s}, P_{k})$. If the distances $ped(P_i, {\mathcal{L}})\le \epsilon$ for all points $P_i$ ($i\in[s, k]$), it simply expands $W$ to $[P_s, \ldots, P_k, P_{k+1}]$ $(k+1\le n)$ by adding a new point $P_{k+1}$. Otherwise, it produces a new line segment $\mathcal{L}(P_{s}, P_{k-1})$, and replaces $W$ with a new window $[P_{k-1},\ldots,P_{k+1}]$. The above process repeats until all points in $\dddot{\mathcal{T}}$ have been considered.
%
%\textcolor[rgb]{0.00,0.07,1.00}{According to the different methods of selecting the end points of a line segment, Open Window can further be divided into Normal Penning Window and Before Opening Window~\cite{Meratnia:Spatiotemporal}. When the distance of the point to compressed trajectory exceeds a certain threshold, Normal Opening Window algorithm select that point as the end point, while Before Opening Window select the last point within the window as the end point of the current trajectory.}
%
Algorithm \opwa is not efficient enough for compressing long trajectories as it remains in $O(n^2)$ time, the same as the \dpa algorithm.
%Also, \ped and \sed are both supported in \opwa as the algorithm \dpa does.
}%%%%%Opening windows


\stitle{Algorithm BQS Using \ped \cite{Liu:BQS}}.
It is essentially an efficiency optimized \opwa algorithm \cite{Meratnia:Spatiotemporal}, and reduces the running time by introducing convex hulls to pick out a certain number of points, which makes it specific for \ped.

For a buffer $W$ with sub-trajectory $[P_s, \ldots, P_k]$, it splits the space into four quadrants. A buffer here is similar to a window in \opwa \cite{Meratnia:Spatiotemporal}. For each quadrant, a rectangular bounding box is firstly created using the least and highest $x$ and $y$ values among points $\{P_s,\ldots,P_k\}$, respectively. Then another two bounding lines connecting points $P_s$ and $P_{h}$ and points $P_s$ and $P_{l}$ are created such that lines $\vv{P_sP_{h}}$ and $\vv{P_sP_{l}}$ have the largest and smallest angles with the $x$-axis, respectively.
Here $P_{h},P_{l} \in\{P_s,\ldots,P_k\}$. The bounding box and the two lines together form a convex hull.
Each time a new point $P_k$ is added to buffer $W$, \bqsa first picks out at most eight significant points from the convex hull in a quadrant. It calculates the distances of the significant points to line $\vv{P_sP_k}$, among which the largest distance $d_{u}$ and the smallest distance $d_l$ are an upper bound and  a lower bound of the distances of all points in $[P_s, \ldots, P_k]$ to line $\vv{P_sP_k}$.
(1) If $d_l\ge \epsilon$, it produces a new line segment $\mathcal{L}(P_{s}, P_{k-1})$, and produces a new window $[P_{k-1},\ldots,P_{k}]$ to replace $W$.
(2) If $d_u < \epsilon$, it simply expands buffer $W$ to $[P_s, \ldots, P_k, P_{k+1}]$ $(k+1\le n)$ by adding a new point $P_{k+1}$.
(3) Otherwise, it computes all distances $d(P_i, {\mathcal{L}(P_s,P_k)})$ ($i\in[s, k]$) as algorithm \dpa does.
%
The time complexity of \bqsa remains $O(n^2)$. However, its simplified version \fbqsa has a linear time complexity by essentially avoiding case (3) to speed up the process.
%The performance of \fbqsa has already been evaluated in work \cite{Lin:Operb}. %our preview work


\begin{example}
	\label{exm-alg-bqs}
	Figure~\ref{fig:bqs} is an example of \bqsa. The bounding box $c_1c_2c_3c_4$ and the two lines $\vv{P_sP_{h}} = \vv{P_0P_1}$ and $\vv{P_sP_{l}} = \vv{P_0P_2}$ form a convex hull $u_1u_2c_2l_2l_1c_4$. \bqsa computes the distances of $u_1,u_2,c_2,l_2,l_1$ and $c_4$ to line $\vv{P_0P_6}$ when $k=6$ or to line $\vv{P_0P_7}$ when $k=7$.
	%
	When $k=6$, all these distances to $\vv{P_0P_6}$  are less than $\epsilon$, hence \bqsa goes on to the next point (case 2); When $k=7$,
	the max and min distances to $\vv{P_0P_7}$ are larger and less than $\epsilon$, respectively, and \bqsa needs to compress sub-trajectory $[P_0, \ldots, P_7]$ along the same line as \dpa (case 3).
\end{example}

\begin{figure}[tb!]
	%\vspace{-1ex}
	\centering
	\includegraphics[scale = 0.66]{Figures/Fig-BQS.png}
	\vspace{-1ex}
	\caption{{\small Examples for algorithm \bqsa.}}
	\label{fig:bqs}
	\vspace{-2ex}
\end{figure}


\stitle{Algorithm SQUISH-E Using \sed~\cite{Muckell:Compression}}.
It is a bottom-up algorithm with a buffer, and has two forms: \squishe($\lambda$) ensuring the compression ratio $\lambda$, and \squishe($\epsilon$) ensuring the \sed error bound $\epsilon$. In this study, we  use \squishe($\epsilon$), as we focus on error bounded trajectory simplification.

\eat{%%%%%%%%%%%%%%%%
taking as input a trajectory \trajec{T} and two additional parameters $\lambda$ and $\epsilon$.
It first compresses trajectory \trajec{T} while striving to minimize \sed error and achieving the compression ratio of $\lambda$. Then, it further compresses \trajec{T} as long as this compression will not increase the max \sed error beyond $\epsilon$.

Meanwhile, \squishe($\lambda$) is the case where $\epsilon$ is set to $0$ and therefore it minimizes \sed error ensuring the compression ratio of $\lambda$, and
\squishe($\epsilon$) denotes another case, \ie the \emph{min-$\#$ problem}, where $\lambda$ is set to $1$ and therefore it maximizes compression ratio while keeping \sed error under $\epsilon$.
In this paper, we only discuss \squishe($\epsilon$).
}%%%%%%%%%%%%%%%%%%%%

Algorithm \squishe  optimizes algorithm \tpa with a doubly linked list $Q$. Each node in the list is a tuple $P(pre, suc, prio, mnprio)$, where $P$ is a trajectory data point, $pre$ and $suc$ are the predecessive  and successive points of $P$, respectively,  $prio$ is the priority of $P$ defined as an upper bound of the \sed error that the removal of $P$ introduces, and $mnprio$ is the max priority of its predecessive and successive points removed from the list.
%
Initially, trajectory points are loaded to $Q$ one by one.
At the same time, $mnprio$ of each point is set to $zero$ as no node has been removed from the list.
Moreover, the priorities of points $P_0$ and $P_{|Q|-1}$ are set to $\infty$, and the priority of point $P_i$ ($0<i<|Q|-1$) is set to $sed(P_i, \vv{pre(P_{i})suc(P_{i})})$.
%
Then, \squishe finds and removes a point $P_j$ from $Q$ that has the lowest priority $prio(P_j)<\epsilon$, and the properties $mnprio$ of predecessor $pre(P_j)$ and successor $suc(P_j)$ are updated to $\max(mnprio(pre(P_j)), prio(P_j))$ and $\max(mnprio(suc(P_j)), prio(P_j))$, respectively.
Next, the properties $prio$ of $pre(P_j)$ and $suc(P_j)$ are further updated to $mnprio(pre(P_j))$ + $sed(pre(P_j), \vv{pre(pre(P_{j}))suc(P_{j})})$ and $mnprio(suc(P_j))$ + $sed(suc(P_j),\vv{pre(P_{j})suc(suc(P_{j}))})$, respectively.
%
After that, a new point is read to the list and the information of its predecessor in the list is updated.
%
The above process repeated until that no points have a priority smaller than $\epsilon$. % \ie  the \sed up bound
%
\squishe finds and removes a point from $Q$ that has the lowest priority in $O(\log |Q|)$ time, where $|Q|$ denotes the number of points stored in $Q$.
Thus, \squishe runs in $O(n\log |Q|)$ time and $O(|Q|)$ space.
%And \squishe only supports \sed.


\begin{figure*}[tb!]
	\centering
	\includegraphics[scale=0.40]{Figures/Fig-Squishe.png}
	\vspace{-3ex}
	\caption{\small The trajectory $\dddot{\mathcal{T}}[P_0, \ldots, P_{10}]$ is compressed by the \squishe algorithm using \sed to five line segments. The size of Q is 6, and the data structure after point $P$ is a tuple $(pre, suc, mmprio, prio)$. }
	\vspace{-1ex}
	\label{fig:squishe}
\end{figure*}





\begin{example}
	\label{exm-alg-squishe}
	Figure~\ref{fig:squishe} is an example of \squishe.
	%
	(1) Initially, $|Q| = 6$ points are read to the list. The tuple $(pre, suc, mmprio, prio)$ for each point is initialized. For example, the tuple of $P_1$ is set to $(0, 2, 0, 0.42\epsilon)$, where $0.42\epsilon$ is the \sed from $P_1$ to $\vv{P_0P_2}$.
	%
	(2) The priority of $P_3$ has the minimal value, thus, it is removed from the list.
	The $mnprio$ properties of $P_2$ and $P_4$ are updated to $max\{mnprio(pre(P_3)), prio(P_3)\}$ = $max\{mnprio(P_2), prio(P_3)\}$ = $max\{0, 0.39\epsilon\}$ = $0.39\epsilon$, and $max\{mnprio(P_4), ~prio(P_3)\}$ = $0.39\epsilon$, respectively.
	Furthermore, the $prio$ property of $P_4$ is updated to $mnprio(suc(P_j)) + sed(suc(P_j),\vv{pre(P_{j})suc(suc(P_{j}))})$ = $mnprio(P_4) + sed(P_4,\vv{P_2P_5})$ = $0.39\epsilon + 2.50\epsilon$ = $2.89\epsilon$, and the $prio$ property of $P_2$ is updated to $mnprio(P_2) + sed(P_2,\vv{P_1P_4})$ = $0.39\epsilon + 2.12\epsilon$ = $2.51\epsilon$.
	Then, $P_6$ is read, and the information of $P_5$ is updated.
	%
	(3) $P_5$ is removed and $P_7$ is read to the list.
	%
	(4) Finally, the algorithm outputs 5 line segments $\vv{P_0P_2},\vv{P_2P_4},\vv{P_4P_7},\vv{P_7P_9}$ and $\vv{P_9P_{10}}$.
\end{example}


%%%%%%%%%%%%%%%%%%%%%%%%%%%%%%%%%%%%%%%%%%%%%%%%%%%%%%%%%%%%%%%%%%%%% END %%%%%%%%%%%%%%%%%%%%%%%%%%%%%%%%%%%%%%%%%%%%%%%%%%%%%%%%%%%%%%%%%%%%%%%%%%



\eat{%%%%%%%%%%%%%%%%%%%%%%%%%%%%%%%%%%%%%%%%%%%%%%%%%%%%%%%

\subsubsection{Opening Window and Top-down}

The \opwa algorithm~\cite{Meratnia:Spatiotemporal} combines the Top-down and opening window strategies, and enforces the constrained global checking in the window.

Given a trajectory $\dddot{\mathcal{T}}[P_0, \ldots, P_n]$ and an error bound $\epsilon$, algorithm \opwa~\cite{Meratnia:Spatiotemporal} maintains a window $W[P_s, \ldots, P_k]$, where $P_s$ and $P_k$ are the start and end points, respectively. Initially, $P_s$ = $P_0$ and $P_k$ = $P_1$, and the window $W$ is gradually expanded by adding new points one by one. \opwa tries to compress all points in $W[P_s, \ldots, P_k]$ to a single line segment $\mathcal{L}(P_{s}, P_{k})$. If the distances $ped(P_i, {\mathcal{L}})\le \epsilon$ for all points $P_i$ ($i\in[s, k]$), it simply expands $W$ to $[P_s, \ldots, P_k, P_{k+1}]$ $(k+1\le n)$ by adding a new point $P_{k+1}$. Otherwise, it produces a new line segment $\mathcal{L}(P_{s}, P_{k-1})$, and replaces $W$ with a new window $[P_{k-1},\ldots,P_{k+1}]$. The above process repeats until all points in $\dddot{\mathcal{T}}$ have been considered.
%
%\textcolor[rgb]{0.00,0.07,1.00}{According to the different methods of selecting the end points of a line segment, Open Window can further be divided into Normal Penning Window and Before Opening Window~\cite{Meratnia:Spatiotemporal}. When the distance of the point to compressed trajectory exceeds a certain threshold, Normal Opening Window algorithm select that point as the end point, while Before Opening Window select the last point within the window as the end point of the current trajectory.}
%
Algorithm \opwa is not efficient enough for compressing long trajectories as it remains in $O(n^2)$ time, the same as the \dpa algorithm.
Also, \ped and \sed are both supported in \opwa as the algorithm \dpa does.


\subsubsection{Sliding Window and Bottom-up}

The \swab algorithm~\cite{Keogh:online} is essentially the combination of the Sliding Window mechanism and the Bottom-up algorithm.
It keeps a window, $w[P_s, \ldots, P_{s+k-1}]$, of a fixed size of $k$.
The window size $k$ should be carefully chosen so that there are enough data points in the window to create about 5 or 6 line segments \cite{Keogh:online}.
Initially, $P_s=P_0$.
Next, the Bottom-Up algorithm, \eg \pavlidis algorithm, is applied to the points in the window, which merges the points into segments with the left-most segment being $\vv{P_sP_{s+i}}$, $i<k$.
Then $\vv{P_sP_{s+i}}$ is output, the window slides to right taking $P_{s+i+1}$ as the new start point of the window, and the Bottom-Up algorithm is applied again.
This process repeated until all points have been merged to segments.

The time complexity of \swab is a small constant factor worse than that of the standard Bottom-Up algorithm~\cite{Keogh:online}.
Also, it supports \sed. % as the standard Bottom-Up algorithm does.

\textcolor[rgb]{1.00,0.00,0.00}{Todo...dis-continuous line segments.}

} %%%%%%%%%%%%%%%%%%%%%%%%%%%%%%%%%%%%%% End of eat



\vspace{-1ex}
\subsection{One-pass Algorithms}


{One-pass algorithms that enforce the local checking policy}.
The local checking policy \eat{, the key to achieve the \emph{one-pass} processing,} does not need a window to buffer the preview read points. 
Instead, it processes each point in a trajectory once and only once. % when compressing the trajectory.
Obviously, the one-pass algorithms have a linear time complexity.

Reumann-Witkam (\rwa) routine \cite{Reumann:Strip} is a straightforward one-pass algorithm that builds a strip paralleling to the line connecting the first two points, then the points within this strip compose a section of the line. \rwa runs fast, but it is limited in compression ratio.
%
The Linear Dead Reckoning (\ldr) method \cite{Lange:Tracking} for position tracking follows the similar routine as \rwa except that it uses \sed and assumes a velocity ${\vv{v}}$ for each section. In addition, the authors of \cite{Trajcevski:DDR} proved that \ldr is also suitable for online spatio-temporal compression. % as long as the tolerance threshold is set to $\epsilon/2$.
However, its compression ratios are poor because both the value and the direction of velocity ${\vv{v}}$ are pre-defined and fixed between two updates.
%
The authors of this paper improved the \rwa routine by essentially allow the angle of a strip be dynamically adjustable, and developed the One-Pass ERror Bounded (\operb) algorithm~\cite{Lin:Operb} coupled with several detailed optimization techniques.

There are also {sector intersection (SI)} based algorithms, developed in fields of graphic, cartographic and pattern recognition, that can be easily adopted for trajectory compression although \textit{they have been overlooked by existing trajectory simplification studies}.
Williams~\cite{Williams:Longest} and Sklansky and Gonzalez \cite{Sklansky:Cone} proposed linear time algorithms based on the idea of ``sector intersection", Dunham \cite{Dunham:Cone} extend these algorithms, and the Sleeve algorithm \cite{Zhao:Sleeve} in the cartographic discipline
essentially applies the same idea as the SI algorithm.
These SI algorithms are \ped specific, and run fast as well as have comparable compression ratios in comparing with the \dpa algorithm using \ped.
%
%Moreover, {fast \bqsa \cite{Liu:BQS} (\fbqsa in short), the simplified version of \bqsa, has a linear time complexity.}
%

In addition, the authors of this paper recently extends the \textit{sector intersection} method from a 2D space to a Spatio-Temporal 3D space, and develop a new one-pass and \sed enabled algorithm \cised that runs fast as well as has comparable compression ratio with the \dpa algorithm using \sed.

We next review algorithm \operb, sector intersection method and algorithm \cised evaluated in out experiments.



\eat{%%%%%%%%%%%%%%%%%%%%%%%%%%%%%%%%%%%%%%%%%%%%%%%%%%%%%%%
	
The $n^{th}$ point routine and the routine of random-selection of points \cite{Shi:Survey} are two naive one-pass algorithms.
In these routines, for every fixed number of consecutive points along the line, the $n^{th}$ point and one random point among them are retained, respectively.
They run fast, however, they are not error bounded.
	
\subsubsection{Reumann-Witkam and LDR}

\begin{figure*}[tb!]
\centering
\includegraphics[scale=0.66]{Figures/Fig-LDR.png}
\vspace{-1ex}
\caption{\small The trajectory $\dddot{\mathcal{T}}[P_0, \ldots, P_{10}]$ is compressed by the Reumann-Witkam and Linear Dead Reckoning algorithms to four and eight line segments, respectively.}
\vspace{-2ex}
\label{fig:ldr}
\end{figure*}

In Reumann-Witkam\cite{Reumann:Strip}, the input data is divided into sections by strips.
Initially, the first strip, with the width of $2*\epsilon$, takes the line $\vv{P_0P_1}$ connecting the first two points, $P_0$ and $P_1$ as its middle line.
Then the strip is expending over the line into the direction of its initial tangent, covering the succeed points, $P_2, \ldots, P_{j}$, until the strip hits the line $\vv{P_jP_{j+1}}$ (meaning that the next point $P_{j+1}$, $j>1$, is out side of the strip).
The points, $[P_0, \ldots, P_{j}]$, within this strip compose a section. The first and last points of the section, \ie $P_0,P_{j}$, are output, and those points between them are removed.
The last point $P_{j}$ is the initial point of the next strip.
The whole process is repeated until the strip contains the end point $P_n$ of the input data.
The Reumann-Witkam is a one-pass algorithm.

{The Linear Dead Reckoning (LDR)\cite{Lange:Tracking} for position tracking follows the similar routine as the Reumann-Witkam algorithm except that it assumes a velocity ${\vv{v}}$ for each section and uses \sed instead of \ped in distance checking.
Moreover, the authors of \cite{Trajcevski:DDR} proved that LDR is also suitable for online spatio-temporal compression as long as the tolerance threshold of the algorithm is set to $\epsilon/2$.}

\begin{example}
\label{exm-alg-strip}
In Figure~\ref{fig:ldr}, the trajectory $\dddot{\mathcal{T}}[P_0, \ldots, P_{10}]$ is compressed
%
(1) by the Reumann-Witkam to four line segments $\vv{P_0P_2}$, $\vv{P_2P_4}$, $\vv{P_4P_7}$ and $\vv{P_7P_{10}}$. First, a strip with width $2\epsilon$ is built parallel to the line $\vv{P_0P_1}$, then the strip is extended over the line and includes point $P_2$. Because $P_3$ is outside of the strip, $P_2$ becomes the end point of the first section and the start point of the second section.
%
(2) by the Linear Dead Reckoning algorithm to eight line segments $\vv{P_0P_1}$, $\vv{P_1P_2}$, $\vv{P_2P_3}$, $\vv{P_3P_4}$, $\vv{P_4P_5}$, $\vv{P_5P_7}$, $\vv{P_7P_8}$ and $\vv{P_8P_{10}}$. First, an initial velocity ${\vv{v}_0}$ is set to $|P_0P_1|/(t_1-t_0)$. Then the synchronized point $P'_2$ of $P_2$ is estimated based on the velocity ${\vv{v}_0}$ and time of $P_2$, \ie ${v}_0 * (t_2-t_0)$. Because the \sed from $P_2$ to the line $\vv{P_0P'_2}$ , \ie $|P_2P'_2|$, is great than $\epsilon/2$, the algorithm outputs $\vv{P_0P_1}$ and starts the next section.
\end{example}

}%%%%%%%%%%%%%%%%%%%%%%%%%%%%%%%%%%%%%%%%%%%%%%%%%%%%%%%End of Eat

\subsubsection{\operb Using \ped}
\label{sec-operb}


%Algorithm \operb, after initializing, repeatedly processes the data points in $\dddot{\mathcal{T}}[P_0,$ $\ldots, P_{n}]$ one by one until that all data points have been considered.
Consider an error bound $\epsilon$ and a sub-trajectory $\dddot{\mathcal{T}_s}[P_s,$ $\ldots, P_{s+k}]$.
\operb dynamically maintains a directed line segment $\mathcal{L}_i$ ($i\in[1,k]$), whose start point is fixed with $P_s$ and its end point is identified (may not in $\{P_s, \ldots, P_{s+i}\}$) to {\em fit} all the previously processed points $\{P_s, \ldots, P_{s+i}\}$.
The directed line segment $\mathcal{L}_i$ is built by a function named \emph{fitting function $\mathbb{F}$}, such that when a new point $P_{s+i+1}$ is considered, only its distance to the directed line segment $\mathcal{L}_i$ is checked, instead of checking the distances of all or a subset of data points of $\{P_{s}, \ldots, P_{s+i}\}$ to $\mathcal{R}_{i+1}$ = $\vv{P_sP_{s+i+1}}$ as the global distance checking does.
During processing, if the distance of point $P_{s+i}$ to the directed line segment $\mathcal{L}_{i-1}$ is larger than the threshold, then a directed line segment, start from $P_s$, is generated and output;
otherwise, the directed line segment $\mathcal{L}_i$ is updated by the fitting function $\mathbb{F}$, as follows.

\begin{small}
	\vspace{-2ex}
	\begin{equation*}
	\label{equ-function}
	\hspace{-1.5ex}\left\{
	\begin{aligned}
	&\hspace{-1.5ex}\left[
	\begin{aligned}
	% & |\mathcal{L}_{i}| = |\mathcal{R}_{i-1}|    \\
	% & \mathcal{L}_{i}.\theta = \mathcal{R}_{i-1}.\theta\\
	& \mathcal{L}_{i} = \mathcal{L}_{i-1}\\
	\end{aligned}
	\right]\hspace{12.5ex}~when~(|\mathcal{R}_{i}| - |\mathcal{L}_{i-1}|) \le \frac{\epsilon}{4}   \\
	&\hspace{-1.5ex}\left[
	\begin{aligned}
	& |\mathcal{L}_{i}|  = j*{\epsilon}/{2} \\
	& \mathcal{L}_{i}.\theta = \mathcal{R}_{i}.\theta    \\
	\end{aligned}
	\right]\hspace{8.5ex}~when~|\mathcal{R}_{i}| >  \frac{\epsilon}{4}~\And~|\mathcal{L}_{i-1}|=0    \\
	&\hspace{-1.5ex}\left[
	\begin{aligned}
	& |\mathcal{L}_{i}|  = j*{\epsilon}/{2}\\
	& \mathcal{L}_{i}.\theta = \mathcal{L}_{i-1}.\theta + f(\mathcal{R}_i,\mathcal{L}_{i-1})*\arcsin(\frac{ped(P_{s+i}, \mathcal{L}_{i-1})}{j*\epsilon/2})/j \\	
	% & \theta^- = \mathcal{L}_{i-1}.\theta - \arcsin(\frac{d(P_i, \mathcal{L}_{i-1})}{j*\epsilon/2})/j \\	
	% & \mathcal{L}_{i}.\theta = \arg_{\mathcal{L}_{i}.\theta}\min({d(P_{i+1}, \mathcal{L}_{i}}), \mathcal{L}_{i}.\theta \in\{\theta^+,\theta^-\})\\	
	\end{aligned}
	\right]\hspace{0ex}else\\
	\end{aligned}
	\right.
	\end{equation*}
	\vspace{-2ex}
\end{small}


\ni where (a) $1 \le i \le k+1$; (b) $\mathcal{R}_{i-1}$ = $\vv{P_sP_{s+i-1}}$, is the directed line segment whose end point $P_{s+i-1}$ is in $\dddot{\mathcal{T}_s}[P_s, \ldots, P_{s+k}]$; (c) $\mathcal{L}_{i}$ is the directed line segment built by fitting function $\mathbb{F}$ to fit sub-trajectory $\dddot{\mathcal{T}_s}[P_s, \ldots, P_{s+i}]$ and $\mathcal{L}_{0}$ = $\mathcal{R}_{0}$; (d) $j = \lceil(|\mathcal{R}_{i}|*2/\epsilon - 0.5)\rceil$; (e) $f()$ is a sign function such that $ f(\mathcal{R}_i,\mathcal{L}_{i-1})$ = $1$ if the included angle $\angle(\mathcal{R}_{i-1}, \mathcal{R}_{i})$ = $(\mathcal{R}_i.\theta - \mathcal{L}_{i-1}.\theta)$ falls in the range of $(-2\pi, -\frac{3\pi}{2}]$, $[-\pi, -\frac{\pi}{2}]$, $[0, \frac{\pi}{2}]$ and $[\pi, \frac{3\pi}{2})$, and $f(\mathcal{R}_i,\mathcal{L}_{i-1})$ = $-1$, otherwise; (f) $\epsilon/2$ is a step length to control the increment of $|\mathcal{L}|$.

Optimization techniques are equipped to achieve a better compression ratio\cite{Lin:Operb}.
Algorithm \operb runs in $O(n)$ time and takes $O(1)$ space, and it only supports \ped.


\begin{example}
	\label{exm-alg-operb}
	Figure~\ref{fig:operb} is a running example of the \operb algorithm compressing the same trajectory $\dddot{\mathcal{T}}[P_0, \ldots, P_{10}]$.
	(1) It takes $P_0$ as the start point, reads $P_1$ and sets $\mathcal{L}_1$ = $\vv{P_0P_1}$.
	(2) It reads $P_2$. The distance from $P_2$ to $\mathcal{L}_1$ is less than the threshold, thus, it updates $\mathcal{L}_1$  to $\mathcal{L}_2$ by the fitting function $\mathbb{F}$.
	(3) It reads $P_3$ and $P_4$, and updates $\mathcal{L}_2$ to $\mathcal{L}_3$ and $\mathcal{L}_3$ to $\mathcal{L}_4$, respectively.
	(4) It reads $P_5$. The distance from $P_5$ to $\mathcal{L}_4$ is larger than the threshold, thus, it outputs $\vv{P_0P_4}$ and start the next section taking $P_4$ as the new start point.
	(5) The process continues until all points have been processed. At last, the algorithm outputs two continuous line segments $\vv{P_0P_4}$ and $\vv{P_4P_{10}}$.
\end{example}

\begin{figure}[tb!]
	\centering
	\includegraphics[scale=0.66]{Figures/Fig-OPER.png}
	\vspace{-5ex}
	\caption{\small The trajectory $\dddot{\mathcal{T}}[P_0, \ldots, P_{10}]$ is compressed by the \operb algorithm using \ped to two line segments.}
	\vspace{-2ex}
	\label{fig:operb}
\end{figure}


\vspace{-0.5ex}
\subsubsection {Sector Intersection Using \ped}
\label{sec-siped}

%The Sector intersection algorithms using \ped (\siped) \cite{Williams:Longest,Sklansky:Cone,Dunham:Cone}, also named \sleeve in \cite{Zhao:Sleeve}, are {the angle-adjustable strips too}.
%Note that the ``cone intersection" algorithms, developed in fields of graphic, cartographic and pattern recognition, are still not familiar to researchers of trajectory compression.
Given a sequence of points $[P_{s}, P_{s+1}, \ldots, P_{s+k}]$ and an error bound $\epsilon$, the \siped approach processes the input points in order and produces a simplified poly-line. Instead of using the distance tolerance $\epsilon$ directly as the distance threshold, \siped converts the distance tolerance into a variable angle tolerance for testing the points.

For the start data point $P_s$, any point $P_{s+i}$ and $|\vv{P_sP_{s+i}}|>\epsilon$ ($i\in[1, k]$), there are two directed lines $\vv{P_sP^u_{s+i}}$ and $\vv{P_sP^l_{s+i}}$ such that $ped(P_{s+i}, \vv{P_sP^u_{s+i}})$ $=$ $ped(P_{s+i}, \vv{P_sP^l_{s+i}}) = \epsilon$ and either ($\vv{P_sP^l_{s+i}}.\theta < \vv{P_sP^u_{s+i}}.\theta ~and~\vv{P_sP^u_{s+i}}.\theta - \vv{P_sP^l_{s+i}}.\theta <\pi$) or ($\vv{P_sP^l_{s+i}}.\theta > \vv{P_sP^u_{s+i}}.\theta ~and~ \vv{P_sP^u_{s+i}}.\theta - \vv{P_sP^l_{s+i}}.\theta < -\pi)$. Indeed, they forms a \emph{sector} \sector{(P_s, P_{s+i}, \epsilon)} that takes $P_s$ as the center point and $\vv{P_sP^u_{s+i}}$ and $\vv{P_sP^l_{s+i}}$ as the border lines.
%
Then there exists a data point $Q$ such that for any data point $P_{s+i}$ ($i \in [1, ... k]$), its perpendicular Euclidean distance to
directed line $\overline{P_sQ}$ is no greater than the error bound $\epsilon$ if and only if the $k$ sectors \sector{(P_s, P_{s+i}, \epsilon)} ($i\in[1,k]$) share common data points other than $P_s$, \ie $\bigsqcap_{i=1}^{k}$\sector{(P_s, P_{s+i}, \epsilon)} $\ne \{P_s\}$ \cite{Williams:Longest, Sklansky:Cone,Zhao:Sleeve}.
%
The point $Q$ may not belong to $\{P_{s}, P_{s+1},$ $\ldots, P_{s+k}\}$.
However, if $P_{s+i}$ ($1\le i\le k$) is chosen as $Q$, then
for any data point $P_{s+j}$ ($j \in [1, ... i]$), its perpendicular Euclidean distance to
line segment $\overline{P_sP_{s+i}}$ is no greater than the error bound $\epsilon$ if and only if $\bigsqcap_{j=1}^{i}$\sector{(P_s, P_{s+j}, \epsilon/2)} $\ne \{P_s\}$, as pointed out in \cite{Zhao:Sleeve}. 
%That is, {\em these sector intersection based algorithms can be easily adopted for trajectory compression}.

Algorithm \siped runs in $O(n)$ time and takes $O(1)$ space, and \siped also only supports \ped.


\begin{figure}[tb!]
	\centering
	\includegraphics[scale=0.66]{Figures/Fig-sleeve.png}
	\vspace{-5ex}
	\caption{\small The trajectory $\dddot{\mathcal{T}}$ is compressed by the sector intersection algorithm using \ped to two line segments.}
	\vspace{-2ex}
	\label{fig:sleeve}
\end{figure}


\begin{example}
	\label{exm-alg-sleeve}
	Figure~\ref{fig:sleeve} is a running example of the narrow \emph{sector} method taking as input the same trajectory $\dddot{\mathcal{T}}[P_0, \ldots, P_{10}]$. At the beginning, $P_0$ is the first start point, and points $P_1$, $P_2$, $P_3$, etc., each has a narrow \emph{sector}.
	For example, the narrow \emph{sector} $\mathcal{S}$($P_0$, $P_{3}$, $\epsilon/2$) takes $P_0$ as the center point and $\vv{P_0P^u_{3}}$ and $\vv{P_0P^l_{3}}$ as the border lines.
	Because $\bigsqcap_{i=1}^{4}\mathcal{S}(P_0, P_{0+i}, \epsilon/2) \ne \{P_0\}$ and $\bigsqcap_{i=1}^{5}\mathcal{S}(P_0, P_{0+i}, \epsilon/2) = \{P_0\}$, $\vv{P_0P_4}$ is output and $P_4$ becomes the start point of the next section.
	At last, the algorithm outputs two continuous line segments $\vv{P_0P_4}$ and $\vv{P_4P_{10}}$.
\end{example}



\vspace{-0.5ex}
\subsubsection {Cone Intersection Using \sed}
\label{sec-cised}

Given a sub-trajectory $[P_s,...,P_{s+k}]$ and an error bound $\epsilon$, any point $P'_{s+i}$, $0< i \le k$, on the plane $P.t-P_{s+i}.t = 0$ is a synchronized data point of $P_{s+i}$. For all $P'_{s+i}$ in the plane satisfying $|P_{s+i}P'_{s+i}| \le \epsilon$, they form a \textit{synchronous circle $\mathcal{O}(P_{s+i}, \epsilon)$} of $P_{s+i}$ with $P_{s+i}$ as its center and $\epsilon$ as its radius.
%
A spatio-temporal cone (or simply \textit{cone}) of a data point $P_{s+i}$ ($1\le i\le k$) in $\dddot{\mathcal{T}}_s$ \wrt a point $P_s$ and an error bound $\epsilon$, denoted as \cone{(P_s, \mathcal{O}(P_{s+i}, \epsilon))}, or \cone{_{s+i}} in short, is an oblique circular cone such that point $P_s$ is its apex and the synchronous circle $\mathcal{O}(P_{s+i}, \epsilon)$ is its base (Figure~\ref{fig:cis}).
%
Then, there exists a point $Q$ such that $Q.t = P_{s+k}.t$ and $sed(P_{s+i}, \vv{P_sQ})\le \epsilon$ for each $i \in [1,k]$ if and only if $\bigsqcap_{i=1}^{k}$\cone{(P_s, \mathcal{O}(P_{s+i}, \epsilon))} $\ne \{P_s\}$.
%
Algorithm \cised applies the idea of spatio-temporal cone intersection and runs in the similar routine as \siped. And its strong simplification version also applies the half $\epsilon$ in building a cone.

In addition, because these spatio-temporal cones have the same apex $P_s$, the checking of their intersection can be computed by a much simpler way, \ie the checking of intersection of cone projection circles on a plane (Figure~\ref{fig:cis}), and a circle is further approximated with its $m$-edge inscribed regular polygon, whose intersection can be computed more efficiently.
Finally, \cised achieves $O(n)$ time and $O(1)$ space. 

	
\begin{figure}[tb!]
	\centering
	\includegraphics[scale=0.66]{Figures/Fig-CIS.png}
	\vspace{-3ex}
	\caption{\small Examples of spatio-temporal cones.} % in a 3D Cartesian coordinate system
	\vspace{-2ex}
	\label{fig:cis}
\end{figure}


%%%%%%%%%%%%%%% example of Algorithm CISED
\begin{figure}[tb!]
	\centering
	\includegraphics[scale=0.66]{Figures/Fig-Conest.png}
	\vspace{-3ex}
	\caption{\small A running example of the \cised algorithm. The points and the oblique circular cones are projected on an x-y space. }%The trajectory $\dddot{\mathcal{T}}[P_0, \ldots, P_{10}]$ is compressed into four line segments.
	\vspace{-3ex}
	\label{fig:exm-const}
\end{figure}
%%%%%%%%%%%%%%%%


\begin{example}
\label{exm-alg-conest}
	Figure~\ref{fig:exm-const} shows a running example of \cised for compressing the trajectory \trajec{T} in Figure~\ref{fig:notations}. 
	For convenience, we project the points and the oblique circular cones on a x-y space.
%	
	(1) After initialization, the \cised algorithm reads point $P_1$ and builds a narrow \emph{oblique circular cone}~\cone{(P_0, \mathcal{O}(P_{1}, \epsilon/2))}, taking $P_0$ as its apex and \circle{(P_1, \epsilon/2)} as its base (green dash). The \emph{circular cone} is projected on the plane $P.t-P_1.t=0$, and the inscribe regular polygon $\mathcal{R}_1$ of the projection circle is returned. As $\mathcal{R}^*$ is empty, $\mathcal{R}^*$ is set to $\mathcal{R}_1$.
%	
	(2) The algorithm reads $P_2$ and builds \cone{(P_0, \mathcal{O}(P_{2}, \epsilon/2))} (red dash). The \emph{circular cone} is also projected on the plane $P.t-P_1.t=0$ and the inscribe regular polygon $\mathcal{R}_2$ of the projection circle is returned. As $\mathcal{R}^*=\mathcal{R}_1$ is not empty, $\mathcal{R}^*$ is set to the intersection of $\mathcal{R}_2$ and $\mathcal{R}^*$, which is $\mathcal{R}_1 \bigsqcap \mathcal{R}_2 \ne \emptyset$.
%	
	(3) For point $P_3$, the algorithm runs the same routine as $P_2$ until the intersection of $\mathcal{R}_3$ and $\mathcal{R}^*$ is $\emptyset$. Thus, a line segment $\vv{P_0P_2}$ is generated, and the process of a new line segment is started, taking $P_2$ as the new start point and $P.t-P_3.t=0$ as the new projection plane.
%	
	(4) At last, the algorithm outputs four continuous line segments, \ie $\{\vv{P_0P_2}$, $\vv{P_2P_4}$, $\vv{P_4P_{7}}$, $\vv{P_7P_{10}}\}$. %\eop
\end{example} 	
\vspace{-1ex}








