

\section*{Appendix: Examples}


\begin{figure*}[tb!]
	\centering
	%\vspace{-1ex}
	\includegraphics[scale=0.45]{Figures/Fig-DP.png}
	\vspace{-1ex}
	\caption{\small A trajectory $\dddot{\mathcal{T}}[P_0, \ldots, P_{10}]$  with 11 points is represented by (2) two and (3) four continuous line segments (solid blue), compressed by the Douglas--Peucker algorithm \cite{Douglas:Peucker} with error metrics \ped and \sed, respectively.}
	\vspace{-0ex}
	\label{fig:notations}
\end{figure*}

\begin{figure*}[tb!]
	\centering
	\includegraphics[scale=0.4]{Figures/Fig-Pavlidis.png}
	\vspace{-1ex}
	\caption{\small The trajectory $\dddot{\mathcal{T}}[P_0, \ldots, P_{8}]$ is compressed by the \pavlidis~algorithm using \ped to two line segments. The triple $(i, j, cost)$ is the $cost$ of merging the line segments $\overline{P_iP_t}$ and $\overline{P_tP_j}$.} % The algorithm finally outputs $\vv{P_0P_4}$ and $\vv{P_4P_{10}}$
	\vspace{-0ex}
	\label{fig:pavlidis}
\end{figure*}


\begin{example}
	\label{exm-notations}
	Consider Figure~\ref{fig:notations}, in which
	%
	(1) $\dddot{\mathcal{T}}[P_0$, $\ldots, P_{10}]$ is a trajectory having 11 data points,
	%
    (2) the set of two continuous line segments $\{\vv{P_0P_4}$, $\vv{P_4P_{10}}$\}, the set of four continuous line segments $\{\vv{P_0P_2}$, $\vv{P_2P_4}$, $\vv{P_4P_7}$, $\vv{P_7P_{10}}$\} and the set of three continuous line segments $\{\vv{P_0P_4}$, $\vv{P_4P_5}$, $\vv{P_5P_{10}}$\} are three piecewise line representations of trajectory $\dddot{\mathcal{T}}$,
	%
	(3) $ped(P_4, \vv{P_0P_{10}})=|\vv{P_4P^*_4}|$, where $P^*_4$ is the perpendicular point of $P_4$ \wrt line segment $\vv{P_0P_{10}}$,
	%
	(4) For $P_4$, its synchronized point $P'_4$ \wrt $\vv{P_0P_{10}}$ satisfies $\frac{|\vv{P_0P'_4}|}{|\vv{P_0P_{10}}|} = \frac{P_4.t - P_0.t}{P_{10}.t-P_0.t} = \frac{4-0}{10-0}= \frac{2}{5}$,
	%
	(5) $sed(P_4, \vv{P_0P_{10}})= |\vv{P_4P'_4}|$, $sed(P_2, \vv{P_0P_{4}})= |\vv{P_2P'_2}|$ and $sed(P_7, \vv{P_4P_{10}})$ $=$ $|\vv{P_7P'_7}|$,
	where points $P'_4$, $P'_2$ and $P'_7$ are the synchronized points of $P_4$, $P_2$ and $P_7$ \wrt line segments $\vv{P_0P_{10}}$, $\vv{P_0P_{4}}$ and $\vv{P_4P_{10}}$, respectively.  and
    %
    (6) $dad(\vv{P_5P_6}, \vv{P_0P_{10}})=\theta_{56}$ is the \dad of line segment $\vv{P_5P_6}$ to $\vv{P_0P_{10}}$.
\end{example}






































