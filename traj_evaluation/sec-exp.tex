%%%%%%%%%%%%%%%%%%%%%%%%%%%%%%%%%%%%%%%%%%%%%%%%%%%%%%%%%%%%%%%%%%%%%%%%%%%%%%
\section{Evaluation} %Experimental Study
\label{sec-exp}
%%%%%%%%%%%%%%%%%%%%%%%%%%%%%%%%%%%%%%%%%%%%%%%%%%%%%%%%%%%%%%%%%%%%%%%%%%%%%%
In this section, we present an extensive experimental study of the nine \lsa algorithms listed in \mytable{tab:summary-lsa} on trajectory data sets.
Using four real-life datasets, we conducted three sets of experiments to evaluate:
(1) the compression ratios of these algorithms, and the impacts of distance metrics,\ie \sed and \ped, to the compression ratios, 
(2) the execution time of these algorithms, and
(3) the average errors of these algorithms.
\textcolor{blue}{Note the additional comparison of the near optimal algorithm (\nopts) and the optimal algorithm (\opt) is presented in Appendix B.}


\subsection{Experimental Setting}
We first introduce the settings of our experimental study.

\stitle{Real-life Trajectory Datasets}.
We use four real-life datasets shown in \mytable{tab:datasets}, namely, Service car trajectory data (\ucar) collected by a Chinese car rental company, Geolife trajectory data (\geolife) collected in GeoLife project~\cite{Web:Geolife}, Mopsi trajectory data (\mopsi) collected in Mopsi project \cite{Web:Mopsi} and Act trajectory data (\act) collected by our team members, to evaluate those \lsa algorithms.

\begin{table}
	%\vspace{-2ex}
	\caption{\small Real-life trajectory datasets}
	\centering
	\small
	\begin{tabular}{|l|c|c|c|r|}
		\hline
		\kw{Data}& \kw{Number\ of}     &\kw{Sampling}   &\kw{Points~Per}    &\kw{Total} \\
		\kw{Sets} & \kw{Trajectories}   &\kw{Rates (s)}  &\kw{Trajectory (K)}&\kw{points}\\	\hline
		%\taxi	&12,727	    &60	        &$\sim39.1$      &498M \\	\hline
		%\truck	&10,368	    &1-60	    &$\sim71.9$     &746M \\	\hline
		%\ucar	&11,000	    &3-5	    &$\sim119.1$   &1.31G\\		\hline
		%
		\ucar	&1,00	    &3-5	&$\sim114.0$   &114M 	\\	\hline
		\geolife\cite{Web:Geolife} &182	    &1-5	&$\sim131.4$   &24.2M	\\	\hline
		\mopsi\cite{Web:Mopsi}	&51	    	&2	    &$\sim153.9$   &7.9M	\\	\hline
		\act	& 10	    &1	    &$\sim11.8$    &112.8K	\\	\hline
	\end{tabular}
	\label{tab:datasets}
	\vspace{-3ex}
\end{table}


% \ni \emph{(1) Truck trajectory data} (\truck) is the GPS trajectories collected by \eat{10,368} trucks equipped with GPS sensors in China
% during a period from Mar. 2015 to Oct. 2015. The sampling rate varied from 1s to 60s.
%Trajectories mostly have around $50$ to $90$ thousand data points.

%\vspace{0.5ex}
%\ni \emph{(1) Service car trajectory data} (\ucar) is the GPS trajectories collected by a Chinese car rental company during Apr. 2015 to Nov. 2015. The sampling rate was one point per $3$--$5$ seconds, and each trajectory has around $114.1K$ points.
%.We randomly chose $1,000$ cars from them

%\vspace{0.5ex}
%\ni \emph{(2) GeoLife trajectory data} (\geolife) is the GPS trajectories collected in GeoLife project~\cite{Web:Geolife} by 182 users in a period from Apr. 2007 to Oct. 2011. These trajectories have a variety of sampling rates, among which 91\% are logged in each 1-5 seconds per point. %or each 5-10 meters
%This dataset contains 182 trajectories, one trajectory for each user, with a total distance of about 1.2 million kilometers. 
%The longest trajectory has 2,156,994 points.

%\vspace{0.5ex}
%\ni \emph{(3) Mopsi trajectory data} (\mopsi) is the GPS trajectories collected in Mopsi project~\cite{Web:Mopsi} by 51 users in a period from 2008 to 2014. Most routes are in Joensuu region, Finland. The sampling rate was one point per $2$ seconds, and each trajectory has around $153.9K$ points.
%exist on every continent.

%\vspace{0.5ex}
%\ni \emph{(4) Act trajectory data} (\act) is a small set GPS trajectories collected with a high sampling rate of one point per second by our team members in 2017. There are 10 trajectories and each trajectory has around 11.8K points.

%The details of these datasets are shown in Table~\ref{tab:dataset}.

\stitle{Algorithms and implementation}.
We implement the algorithms listed in \mytable{tab:summary-lsa}.
For algorithm \cised, we fixed parameter $m=16$ as evaluated in \cite{Lin:Cised}, \ie 16-edges inscribe regular polygon.
For algorithm \nopts, we fixed parameter \textcolor{red}{$m=32$}. 
All algorithms were implemented with Java.
All tests were run on an x64-based  PC with 4 Intel(R) Core(TM) i5-4570 CPU @ 3.20GHz  and 16GB of memory, and each test was repeated
over 3 times and the average is reported here.


\eat{%%%%%%%%%%%%%%%%%%%
	\stitle{Real-life Trajectory Datasets}.
	We use four real-life datasets shown in Table~\ref{tab:dataset} to test our solutions.
	
	\sstab{\bf(1) Taxi trajectory data}, referred to as \taxi, is the GPS trajectories collected by $12,727$ taxies equipped with GPS sensors in Beijing during a period
	from Nov. 1, 2010 to Nov. 30, 2010. The sampling rate was one point  per 60s, and \taxi has $39,100$ data points on average per trajectory.
	
	\sstab{\bf(2) Truck trajectory data}, referred to as \truck, is the GPS trajectories collected by 10,368 trucks equipped with GPS sensors in China
	during a period from Mar. 2015 to Oct. 2015. The sampling rate varied from 1s to 60s. Trajectories mostly have around $50$ to $90$ thousand data points.
	
	\sstab{\bf(3) Service car trajectory data}, referred to as \ucar,  is the GPS trajectories collected by a car rental company.
	We chose $11,000$ cars from them, during Apr. 2015 to Nov. 2015. The sampling rate was one point per $3$--$5$ seconds, and
	each trajectory has around $119.1K$ data points.
	
	{\sstab{\bf(4) GeoLife trajectory data}, refered to as \geolife, is the GPS trajectories collected in GeoLife project~\cite{Zheng:GeoLife} by 182 users in a period from Apr. 2007 to Oct. 2011. These trajectories have a variety of sampling rates, among which 91\% are logged in each 1-5 seconds or each 5-10 meters per point. The longest trajectory has 2,156,994 points.}
	%This dataset contains 182 trajectories, one trajectory for each user, with a total distance of about 1.2 million kilometers. 
}%%%%%%%%%%%%%%%%%%%%%%%%%%%%%%%%%%%%%%%%%%%%%%%%%%%%%%%%%%%%%






\begin{figure*}[tb!]
	\centering
	\includegraphics[scale=0.315]{Figures/Exp-PED-CR-epsilon-service.png} 	\hspace{1ex}
	\includegraphics[scale=0.315]{Figures/Exp-PED-CR-epsilon-geolife.png}	\hspace{1ex}
	\includegraphics[scale=0.315]{Figures/Exp-PED-CR-epsilon-mopsi.png}		\hspace{1ex}
	\includegraphics[scale=0.315]{Figures/Exp-PED-CR-epsilon-geolife.png}
	\vspace{-2.5ex}
	\caption{\small Evaluation of compression ratios (\ped): varying the error bound $\epsilon$.}
	\label{fig:cr-ped}
	\vspace{-.5ex}
\end{figure*}

\begin{figure*}[tb!]
	\centering
	\includegraphics[scale=0.315]{Figures/Exp-SED-CR-epsilon-service.png} 	\hspace{1ex}
	\includegraphics[scale=0.315]{Figures/Exp-SED-CR-epsilon-geolife.png}	\hspace{1ex}
	\includegraphics[scale=0.315]{Figures/Exp-SED-CR-epsilon-mopsi.png}		\hspace{1ex}
	\includegraphics[scale=0.315]{Figures/Exp-SED-CR-epsilon-geolife.png}
	\vspace{-2.5ex}
	\caption{\small Evaluation of compression ratios (\sed): varying the error bound $\epsilon$.}
	\label{fig:cr-sed}
	\vspace{-.5ex}
\end{figure*}

\begin{figure*}[tb!]
	\centering
	\includegraphics[scale=0.315]{Figures/Exp-PED-error-epsilon-service.png}	\hspace{1ex}
	\includegraphics[scale=0.315]{Figures/Exp-PED-error-epsilon-geolife.png}	\hspace{1ex}
	\includegraphics[scale=0.315]{Figures/Exp-PED-error-epsilon-mopsi.png}	\hspace{1ex}
	\includegraphics[scale=0.315]{Figures/Exp-PED-error-epsilon-geolife.png} \vspace{-2ex}
	\caption{\small Evaluation of average errors (\ped): varying the error bound $\epsilon$.}
	\label{fig:ae-ped}
	\vspace{-.5ex}
\end{figure*}

\begin{figure*}[tb!]
	\centering
	\includegraphics[scale=0.315]{Figures/Exp-SED-error-epsilon-service.png}	\hspace{1ex}
	\includegraphics[scale=0.315]{Figures/Exp-SED-error-epsilon-geolife.png}	\hspace{1ex}
	\includegraphics[scale=0.315]{Figures/Exp-SED-error-epsilon-mopsi.png}		\hspace{1ex}
	\includegraphics[scale=0.315]{Figures/Exp-SED-error-epsilon-geolife.png}
	\vspace{-2ex}
	\caption{\small Evaluation of average errors (\sed): varying the error bound $\epsilon$.}
	\label{fig:ae-sed}
	\vspace{-2ex}
\end{figure*}


%%%%%%%%%%%%%%%%%%%%%%%%%%%%%%%%%%%%%%%%%%%%%%%%%%%%%%%%%%%%%%%%%%%%%%%%%%%%%%
\subsection{Experimental Results}
%%%%%%%%%%%%%%%%%%%%%%%%%%%%%%%%%%%%%%%%%%%%%%%%%%%%%%%%%%%%%%%%%%%%%%%%%%%%%%
We next present our findings.




%%%%%%%%%%%%%%%%%%%%%%%%%%%%%%%%%%%%%%%%%%%%%%%%%%%%%%%%%%%%%%%%%%%%%%%%%%%%%
\subsubsection{Evaluation of Compression Effectiveness}
%%%%%%%%%%%%%%%%%%%%%%%%%%%%%%%%%%%%%%%%%%%%%%%%%%%%%%%%%%%%%%%%%%%%%%%%

In the first set of tests, we compare the compression ratios of these algorithms using \ped and \sed, respectively.


Given a set of trajectories $\{\dddot{\mathcal{T}_1}, \ldots, \dddot{\mathcal{T}_M}\}$ and their piecewise line representations
$\{\overline{\mathcal{T}_1}, \ldots, \overline{\mathcal{T}_M}\}$,
 the compression ratio is $(\sum_{j=1}^{M} |\overline{\mathcal{T}}_j |)/(\sum_{j=1}^{M} |\dddot{\mathcal{T}}_j |)$.
 Note that by the definition, algorithms with lower compression ratios are better.


\stitle{Exp-1.1: Impacts of the error bound $\epsilon$}.
To evaluate the impacts of $\epsilon$ on compression ratios of these algorithms, we varied $\epsilon$ from $5m$ to $100m$ on
 the entire four datasets, respectively.
The results are reported in Figure~\ref{fig:cr-ped} and Figure~\ref{fig:cr-sed}.

\sstab (1) When increasing $\epsilon$, the compression ratios decrease. For example, in \ucar,
the compression ratios are greater than $28.7\%$ when $\epsilon$ = $5m$, but are less than $6.6\%$ when $\epsilon$ = $100m$.

\sstab (2) using \ped, optimal and sub-optimal.

\sstab (3) using \sed, optimal and sub-optimal.

\sstab (4) \ped vs \sed.
	
\sstab (5) sampling rate and datasets.
\geolife has the lowest compression ratios, compared with \taxi, \truck and \ucar, due to its highest sampling rate, \taxi has the highest compression ratios due to its lowest sampling rate, and \truck and \ucar have the compression ratios in the middle accordingly.

\begin{figure*}[tb!]
	\centering
	\includegraphics[scale=0.315]{Figures/Exp-PED-time-epsilon-service.png}	\hspace{1ex}
	\includegraphics[scale=0.315]{Figures/Exp-PED-time-epsilon-geolife.png}	\hspace{1ex}
	\includegraphics[scale=0.315]{Figures/Exp-PED-time-epsilon-mopsi.png}	\hspace{1ex}
	\includegraphics[scale=0.315]{Figures/Exp-PED-time-epsilon-geolife.png}	\hspace{1ex}
	\vspace{-2.5ex}
	\caption{\small Evaluation of efficiency (\ped): varying the error bound $\epsilon$.}\label{fig:time-epsilon-ped}
	\vspace{-1ex}
\end{figure*}

\begin{figure*}[tb!]
	\centering
	\includegraphics[scale=0.315]{Figures/Exp-SED-time-epsilon-service.png}	\hspace{1ex}
	\includegraphics[scale=0.315]{Figures/Exp-SED-time-epsilon-geolife.png}	\hspace{1ex}
	\includegraphics[scale=0.315]{Figures/Exp-SED-time-epsilon-mopsi.png}	\hspace{1ex}
	\includegraphics[scale=0.315]{Figures/Exp-SED-time-epsilon-geolife.png}	\hspace{1ex}
	\vspace{-2.5ex}
	\caption{\small Evaluation of efficiency (\sed): varying the error bound $\epsilon$.}\label{fig:time-epsilon-sed}
	\vspace{-1ex}
\end{figure*}

\begin{figure*}[tb!]
	\centering
	\includegraphics[scale=0.315]{Figures/Exp-PED-time-epsilon-service.png}	\hspace{1ex}
	\includegraphics[scale=0.315]{Figures/Exp-PED-time-epsilon-geolife.png}	\hspace{1ex}
	\includegraphics[scale=0.315]{Figures/Exp-PED-time-epsilon-mopsi.png}	\hspace{1ex}
	\includegraphics[scale=0.315]{Figures/Exp-PED-time-epsilon-geolife.png}	\hspace{1ex}
	\vspace{-2.5ex}
	\caption{\small Evaluation of efficiency (\ped): varying the size of trajectories.}\label{fig:time-size-ped}
	\vspace{-1ex}
\end{figure*}

\begin{figure*}[tb!]
	\centering
	\includegraphics[scale=0.315]{Figures/Exp-SED-time-epsilon-service.png}	\hspace{1ex}
	\includegraphics[scale=0.315]{Figures/Exp-SED-time-epsilon-geolife.png}	\hspace{1ex}
	\includegraphics[scale=0.315]{Figures/Exp-SED-time-epsilon-mopsi.png}	\hspace{1ex}
	\includegraphics[scale=0.315]{Figures/Exp-SED-time-epsilon-geolife.png}	\hspace{1ex}
	\vspace{-2.5ex}
	\caption{\small Evaluation of efficiency (\sed): varying the size of trajectories.}\label{fig:time-size-sed}
	\vspace{-1ex}
\end{figure*}
	
\eat{%%%%%%%%%%%%%%%%%%%%%%%%
First, algorithm \operb has comparable compression ratios with \fbqsa and \dpa.
For example, when $\epsilon = 40m$, the compression ratios are ($20.9\%$, $20.4\%$, $11.1\%$, $3.1\%$) of \fbqsa, ($20.7\%$, $18.7\%$, $8.9\%$, $2.6\%$) of \dpa and ($22.3\%$, $20.3\%$, $10.0\%$, $2.6\%$) of \operb on (\taxi, \truck, \ucar, \geolife), respectively.
For all $\epsilon$, the compression ratios of \operb are on average ($107.2\%$, $98.3\%$, $92.9\%$, {$85.1\%$}) of \fbqsa and ($107.7\%$, $106.6\%$, $113.5\%$, $99.6\%$) of \dpa on (\taxi, \truck, \ucar, \geolife), respectively.
\operb is better than  \fbqsa on \truck, \ucar and \geolife, while a little worse on \taxi. The results also show that \operb has a better performance than \fbqsa on datasets with high sampling rates.

Second, algorithm \operba achieves the best compression ratios on all datasets and nearly all $\epsilon$ values.
Its compression ratios are on average ($83.7\%$, $79.5\%$, $79.7\%$, $81.0\%$) of \fbqsa and ($84.2\%$, $86.4\%$, $97.1\%$, $94.7\%$) of \dpa on (\taxi, \truck, \ucar, \geolife), respectively.
}%%%%%%%%%%%%%%






%%%%%%%%%%%%%%%%%%%%%%%%%%%%%%%%%%%%%%%%%%%%%%%%%%%%%%%%%%%%%%%%%%%%%%%%%%%%%%


%%%%%%%%%%%%%%%%%%%%%%%%%%%%%%%%%%%%%%%%%%%%%%%%%%%%%%%%%%%%%%%%%%%%%%%%%%%%%%
\vspace{-0.5ex}
\subsubsection{Evaluation of Average Errors}
%%%%%%%%%%%%%%%%%%%%%%%%%%%%%%%%%%%%%%%%%%%%%%%%%%%%%%%%%%%%%%%%%%%%%%%%%%%%%%


In the third set of tests, we evaluate the average errors of these algorithms.
Given a set of trajectories $\{\dddot{\mathcal{T}_1}, \ldots, \dddot{\mathcal{T}_M}\}$ and their piecewise line representations
$\{\overline{\mathcal{T}_1}, \ldots, \overline{\mathcal{T}_M}\}$, and point $P_{j,i}$ denoting
a point in trajectory $\dddot{\mathcal{T}}_j$ contained in a line segment $\mathcal{L}_{l,i}\in\overline{\mathcal{T}_l}$ ($l\in[1,M]$),
then the average error is $\sum_{j=1}^{M}\sum_{i=0}^{M} d(P_{j,i},
\mathcal{L}_{l,i})/\sum_{j=1}^{M}{|\dddot{\mathcal{T}}_j |}$.




\stitle{Exp-2.1: Impacts of the error bound $\epsilon$}.
To evaluate the impacts of $\epsilon$ on errors of these algorithms, we varied $\epsilon$ from $5m$ to $100m$ on the entire \taxi, \truck, \ucar and \geolife, respectively.
The results are reported in Figure~\ref{fig:ae-ped} and Figure~\ref{fig:ae-sed}.

\sstab (1) When increasing $\epsilon$, the average errors increase. For example, in \ucar,
the compression ratios are greater than $28.7\%$ when $\epsilon$ = $5m$, but are less than $6.6\%$ when $\epsilon$ = $100m$.

\sstab (2) using \ped, optimal and sub-optimal.

\sstab (3) using \sed, optimal and sub-optimal.

\sstab (4) \ped vs \sed.

\sstab (5) sampling rate and datasets.



%%%%%%%%%%%%%%%%%%%%%%%%%%%%%%%%%%%%%%%%%%%%%%%%%%%%%%%%%%%%%%%%%%%%%%%%%%%%%%
\subsubsection{Evaluation of Compression Efficiency}
%%%%%%%%%%%%%%%%%%%%%%%%%%%%%%%%%%%%%%%%%%%%%%%%%%%%%%%%%%%%%%%%%%%%%%%%%%%%%%



In the first set of tests, we compare the efficiency (execution time) of our approaches \operb and \operba with algorithms \dpa and \fbqsa
and with algorithms \kw{Raw}-\operb and \kw{Raw}-\operba.
For fairness, we load and compress trajectories one by one, and only count the running time of the compressing process.
%For a small size trajectory, we repeat compress it tens of times and accumulation the total running time so as  to get the average compression time.


\stitle{Exp-3.1: Impacts of the error bound $\epsilon$}.
To evaluate the impacts of $\epsilon$ on efficiency of these algorithms, we varied $\epsilon$ from $5m$ to $100m$ on the entire \taxi, \truck, \ucar and \geolife, respectively.
The results are reported in Figure~\ref{fig:ae-ped} and Figure~\ref{fig:ae-sed}.

\sstab (1) 
All algorithms are not very sensitive to $\epsilon$, but their running time all decreases a little bit with the increase of $\epsilon$,
as the increment of $\epsilon$ decreases the number of directed line segments in the output.
Further, algorithm \dpa is more sensitive to $\epsilon$ than the other three algorithms.

\sstab (2) \ped vs \sed.

\sstab (3) using \ped, optimal and sub-optimal.

\sstab (4) using \sed, optimal and sub-optimal.

Algorithms \operb and \operba are obviously faster than \dpa and \fbqsa in all cases.
\operb is on average ($13.9$, $17.4$, $14.7$, {$20.6$}) times faster than \dpa, and ($4.1$, $4.1$, $5.4$, {$5.2$}) times faster than \fbqsa on (\taxi, \truck, \ucar, {\geolife}), respectively. Algorithm \operba is as fast as \operb because trajectory interpolation is a light weight operation.

\sstab (5) sampling rate and datasets.







\stitle{{Exp-3.2}: Impacts of the sizes of trajectories}.
To evaluate the impacts of the number of data points in a trajectory (\ie the size of a trajectory),
we chose $100$ trajectories from \taxi, \truck, \ucar and \geolife, respectively,
and varied the size \trajec{|T|} of trajectories from $2,000$ to $10,000$, while fixed $\epsilon = 40$ meters (m).
The results are reported in Figure~\ref{fig:time-size}.

\sstab(1) Algorithms \operb, \operba and \fbqsa  scale well with the increase of the size of trajectories on all datasets,
and show a linear running time, while algorithm \dpa does not.
This is consistent with their time complexity analyses.

\sstab(2) Algorithms \operb and \operba are the fastest \lsa algorithms, and are {($3.8$--$5.3$, $3.5$--$4.8$, $4.6$--$7.2$, $6.2$--$8.4$)} times faster than \fbqsa,
and {($9.6$--$17.6$, $8.8$--$15.4$, $8.4$--$16.3$, $9.0$--$14.4$)} times faster than \dpa on (\taxi, \truck, \ucar, \geolife), respectively. The running time of \operb and \operba is similar, and the difference is below 10\%.
\eat{Moreover, \operba is a bit faster than \operb, especially when there are more than $1,000$ data points in  trajectories.}

%%%%%%%%%%%%%%%%%%%%%%%%%%%%%%%%%%%%%%%%%%%%%%%%%%%%%%%%%%%%%%%%%%%%%%%%%%%%%%
\stitle{Summary}.
%%%%%%%%%%%%%%%%%%%%%%%%%%%%%%%%%%%%%%%%%%%%%%%%%%%%%%%%%%%%%%%%%%%%%%%%%%%%%%
From these tests we find the following.

\emph{\sstab{(1) Compression ratios}}. (a) \operb is comparable {with \fbqsa and \dpa}. Its compression ratios are on average $(107.2\%, ~98.3\%, ~92.9\%, ~85.1\%)$ of \fbqsa and ($107.7\%$, $106.6\%$, $113.5\%$, $99.6\%$) of \dpa on (\taxi, \truck, \ucar, \geolife), respectively, and \operb has a better performance on trajectories with higher sampling rates.
(b) \operba has the best compression ratios on all datasets and nearly all $\epsilon$ values.
Its compression ratios are on average {($83.7\%$, $79.5\%$, $79.7\%$, $81.0\%$)} of \fbqsa and {($84.2\%$, $86.4\%$, $97.1\%$, $94.7\%$)} of \dpa on (\taxi, \truck, \ucar, \geolife), respectively.
It shows its advantage on trajectories with both high and low sampling rates.
(c) The optimization techniques are effective for algorithms \operb and \operba on all datasets.

\emph{\sstab{(2) Average errors}}. {Algorithm \operb has similar average errors with \operba. They have lower average errors than the other algorithms on \taxi while higher on \ucar.}


\emph{\sstab{(3) Efficiency}}. \operb and \operba are the fastest algorithms, which are on average $(13.9, ~17.4, ~14.7, {~20.6})$ times faster than \dpa, and $(4.1,~4.1,~5.4, {~5.2})$
times faster than \fbqsa on (\taxi, \truck, \ucar, \geolife), respectively.


%%********************************* The End **********************************


