%%%%%%%%%%%%%%%%%%%%%%%%%%%%%%%%%%%%%%%%%%%%%%%%%%%%%%%%%%%%%%%%%%%%%%%%%%%%%%
%\vspace{-1ex}
\section{Evaluation} %Experimental Study
\label{sec-exp}
%%%%%%%%%%%%%%%%%%%%%%%%%%%%%%%%%%%%%%%%%%%%%%%%%%%%%%%%%%%%%%%%%%%%%%%%%%%%%%
In this section, we present an extensive experimental study of the nine \lsa algorithms listed in \mytable{tab:summary-lsa} on trajectory data sets.
Using four real-life datasets, we conducted three sets of experiments to evaluate the compression ratios, average errors and execution time of these algorithms, and the impacts of distance metrics,\ie \sed and \ped, to them.

%(1) the effectiveness of these algorithms, and
%(2) the efficiency of them.
%{An additional test of the effectiveness of near optimal algorithm (\nopts) is presented in Appendix B.}

\vspace{-1ex}
\subsection{Experimental Setting}
%We first introduce the settings of our experimental study.

\stitle{Real-life Trajectory Datasets}.
We use four real-life datasets shown in \mytable{tab:datasets}, namely, taxi trajectory data (\taxi) collected by a Beijing taxi company, Service car trajectory data (\ucar) collected by a Chinese car rental company, Geolife trajectory data (\geolife) collected in GeoLife project~\cite{Web:Geolife} and Mopsi trajectory data (\mopsi) collected in Mopsi project \cite{Web:Mopsi}, to evaluate those \lsa algorithms. These data sets have varied sampling rates, ranging from one point per minute to one point per second.

\begin{table}
	\vspace{-1ex}
	\caption{\small Real-life trajectory datasets}
	\centering
	\small
	\begin{tabular}{|l|c|c|c|r|}
		\hline
		\kw{Data}& \kw{Number\ of}     &\kw{Sampling}   &\kw{Points~Per}    &\kw{Total} \\
		\kw{Sets} & \kw{Trajectories}   &\kw{Rates (s)}  &\kw{Trajectory (K)}&\kw{points}\\	\hline
		\taxi	&{500}	    &60	        &{$\sim42.8$}      &{21.4M} \\	\hline
		%\truck	&10,368	    &1-60	    &$\sim71.9$     &746M \\	\hline
		%\ucar	&11,000	    &3-5	    &$\sim119.1$   &1.31G\\		\hline
		%
		\ucar	&200	    &3-5	&$\sim114.0$   &22.8M 	\\	\hline
		\geolife\cite{Web:Geolife} &182	    &1-5	&$\sim131.4$   &24.2M	\\	\hline
		\mopsi\cite{Web:Mopsi}	&51	    	&2	    &$\sim153.9$   &7.9M	\\	\hline
		%\act	& 10	    &1	    &$\sim11.8$    &112.8K	\\	\hline
	\end{tabular}
	\label{tab:datasets}
	\vspace{-3ex}
\end{table}


% \ni \emph{(1) Truck trajectory data} (\truck) is the GPS trajectories collected by \eat{10,368} trucks equipped with GPS sensors in China
% during a period from Mar. 2015 to Oct. 2015. The sampling rate varied from 1s to 60s.
%Trajectories mostly have around $50$ to $90$ thousand data points.

%\vspace{0.5ex}
%\ni \emph{(1) Service car trajectory data} (\ucar) is the GPS trajectories collected by a Chinese car rental company during Apr. 2015 to Nov. 2015. The sampling rate was one point per $3$--$5$ seconds, and each trajectory has around $114.1K$ points.
%.We randomly chose $1,000$ cars from them

%\vspace{0.5ex}
%\ni \emph{(2) GeoLife trajectory data} (\geolife) is the GPS trajectories collected in GeoLife project~\cite{Web:Geolife} by 182 users in a period from Apr. 2007 to Oct. 2011. These trajectories have a variety of sampling rates, among which 91\% are logged in each 1-5 seconds per point. %or each 5-10 meters
%This dataset contains 182 trajectories, one trajectory for each user, with a total distance of about 1.2 million kilometers. 
%The longest trajectory has 2,156,994 points.

%\vspace{0.5ex}
%\ni \emph{(3) Mopsi trajectory data} (\mopsi) is the GPS trajectories collected in Mopsi project~\cite{Web:Mopsi} by 51 users in a period from 2008 to 2014. Most routes are in Joensuu region, Finland. The sampling rate was one point per $2$ seconds, and each trajectory has around $153.9K$ points.
%exist on every continent.

%\vspace{0.5ex}
%\ni \emph{(4) Act trajectory data} (\act) is a small set GPS trajectories collected with a high sampling rate of one point per second by our team members in 2017. There are 10 trajectories and each trajectory has around 11.8K points.

%The details of these datasets are shown in Table~\ref{tab:dataset}.

\stitle{Algorithms and implementation}.
We implement the algorithms listed in \mytable{tab:summary-lsa}.
For algorithm \cised, we fixed parameter $m=16$ as evaluated in \cite{Lin:Cised}, \ie 16-edges inscribe regular polygon.
% For algorithm \nopts, we fixed parameter $m=32$. 
All algorithms were implemented with Java.
All tests were run on an x64-based  PC with 4 Intel(R) Core(TM) i7-6700 CPU @
3.40GHz  and 8GB of memory, and {the max heap size of Java VM is 4GB.}
%, and each test was repeated over 3 times and the average is reported here.


\eat{%%%%%%%%%%%%%%%%%%%
	\stitle{Real-life Trajectory Datasets}.
	We use four real-life datasets shown in Table~\ref{tab:dataset} to test our solutions.
	
	\sstab{\bf(1) Taxi trajectory data}, referred to as \taxi, is the GPS trajectories collected by $12,727$ taxies equipped with GPS sensors in Beijing during a period
	from Nov. 1, 2010 to Nov. 30, 2010. The sampling rate was one point  per 60s, and \taxi has $39,100$ data points on average per trajectory.
	
	\sstab{\bf(2) Truck trajectory data}, referred to as \truck, is the GPS trajectories collected by 10,368 trucks equipped with GPS sensors in China
	during a period from Mar. 2015 to Oct. 2015. The sampling rate varied from 1s to 60s. Trajectories mostly have around $50$ to $90$ thousand data points.
	
	\sstab{\bf(3) Service car trajectory data}, referred to as \ucar,  is the GPS trajectories collected by a car rental company.
	We chose $11,000$ cars from them, during Apr. 2015 to Nov. 2015. The sampling rate was one point per $3$--$5$ seconds, and
	each trajectory has around $119.1K$ data points.
	
	{\sstab{\bf(4) GeoLife trajectory data}, refered to as \geolife, is the GPS trajectories collected in GeoLife project~\cite{Zheng:GeoLife} by 182 users in a period from Apr. 2007 to Oct. 2011. These trajectories have a variety of sampling rates, among which 91\% are logged in each 1-5 seconds or each 5-10 meters per point. The longest trajectory has 2,156,994 points.}
	%This dataset contains 182 trajectories, one trajectory for each user, with a total distance of about 1.2 million kilometers. 
}%%%%%%%%%%%%%%%%%%%%%%%%%%%%%%%%%%%%%%%%%%%%%%%%%%%%%%%%%%%%%



\subsection{Evaluation Metrics}
Effectiveness (compression ratios and average errors) and efficiency are used to evaluate these algorithms.

\stitle{Compression ratios.}
Given a set of trajectories $\{\dddot{\mathcal{T}_1}, \ldots, \dddot{\mathcal{T}_M}\}$ and their piecewise line representations $\{\overline{\mathcal{T}_1}, \ldots, \overline{\mathcal{T}_M}\}$,
the compression ratio is $(\sum_{j=1}^{M} |\overline{\mathcal{T}}_j |)/(\sum_{j=1}^{M} |\dddot{\mathcal{T}}_j |)$.
By this definition, algorithms with lower compression ratios are better.

\stitle{Average error.}
Given a set of trajectories $\{\dddot{\mathcal{T}_1}, \ldots, \dddot{\mathcal{T}_M}\}$ and their piecewise line representations
$\{\overline{\mathcal{T}_1}, \ldots, \overline{\mathcal{T}_M}\}$, and $P_{j,i}$ denoting
a point in trajectory $\dddot{\mathcal{T}}_j$ contained in a line segment $\mathcal{L}_{l,i}\in\overline{\mathcal{T}_l}$ ($l\in[1,M]$),
then the average error is $\sum_{j=1}^{M}\sum_{i=0}^{M} d(P_{j,i},
\mathcal{L}_{l,i})/\sum_{j=1}^{M}{|\dddot{\mathcal{T}}_j |}$.

\stitle{Efficiency.}
Efficiency describes the speed of compressing trajectories. We load and compress trajectories one by one, and only count the running time of the compressing process.

%%%%%%%%%%%%%%%%%%%%%%%%%%%%%%%%%%%%%%%%%%%%%%%%%%%%%%%%%%%%%%%%%%%%%%%%%%%%%%
\subsection{Experimental Results}
%%%%%%%%%%%%%%%%%%%%%%%%%%%%%%%%%%%%%%%%%%%%%%%%%%%%%%%%%%%%%%%%%%%%%%%%%%%%%%
We next present our findings.


\begin{figure*}[tb!]
	\centering
	\includegraphics[scale=0.315]{Figures/Exp-PED-CR-epsilon-taxi.png}\hspace{1ex}
	\includegraphics[scale=0.315]{Figures/Exp-PED-CR-epsilon-service.png} 	\hspace{1ex}
	\includegraphics[scale=0.315]{Figures/Exp-PED-CR-epsilon-geolife.png}	\hspace{1ex}
	\includegraphics[scale=0.315]{Figures/Exp-PED-CR-epsilon-mopsi.png}		
	\vspace{-3ex}
	\caption{\small Evaluation of compression ratios (\ped): varying the error bound $\epsilon$.}
	\label{fig:cr-ped-epsilon}
	\vspace{-2ex}
\end{figure*}

\begin{figure*}[tb!]
	\centering
	\includegraphics[scale=0.315]{Figures/Exp-SED-CR-epsilon-taxi.png}\hspace{1ex}
	\includegraphics[scale=0.315]{Figures/Exp-SED-CR-epsilon-service.png} 	\hspace{1ex}
	\includegraphics[scale=0.315]{Figures/Exp-SED-CR-epsilon-geolife.png}	\hspace{1ex}
	\includegraphics[scale=0.315]{Figures/Exp-SED-CR-epsilon-mopsi.png}		
	\vspace{-3ex}
	\caption{\small Evaluation of compression ratios (\sed): varying the error bound $\epsilon$.}
	\label{fig:cr-sed-epsilon}
	\vspace{-2ex}
\end{figure*}

\begin{figure*}[tb!]
	\centering
	\includegraphics[scale=0.315]{Figures/Exp-DAD-CR-epsilon-taxi.png}\hspace{1ex}
	\includegraphics[scale=0.315]{Figures/Exp-DAD-CR-epsilon-service.png} 	\hspace{1ex}
	\includegraphics[scale=0.315]{Figures/Exp-DAD-CR-epsilon-geolife.png}	\hspace{1ex}
	\includegraphics[scale=0.315]{Figures/Exp-DAD-CR-epsilon-mopsi.png}		
	\vspace{-3ex}
	\caption{\small Evaluation of compression ratios (\dad): varying the error bound $\epsilon$.}
	\label{fig:cr-ped-epsilon}
	\vspace{-2ex}
\end{figure*}

\begin{figure*}[tb!]
	\centering
	\includegraphics[scale=0.315]{Figures/Exp-PED-CR-size-taxi.png}\hspace{1ex}
	\includegraphics[scale=0.315]{Figures/Exp-PED-CR-size-service.png} 	\hspace{1ex}
	\includegraphics[scale=0.315]{Figures/Exp-PED-CR-size-geolife.png}	\hspace{1ex}
	\includegraphics[scale=0.315]{Figures/Exp-PED-CR-size-mopsi.png}		
	\vspace{-3ex}
	\caption{\small Evaluation of compression ratios (\ped): varying the size of
    trajectories.}
  \label{fig:cr-ped-size}
	\vspace{-2ex}
\end{figure*}

\begin{figure*}[tb!]
	\centering
	\includegraphics[scale=0.315]{Figures/Exp-SED-CR-size-taxi.png}\hspace{1ex}
	\includegraphics[scale=0.315]{Figures/Exp-SED-CR-size-service.png} 	\hspace{1ex}
	\includegraphics[scale=0.315]{Figures/Exp-SED-CR-size-geolife.png}	\hspace{1ex}
	\includegraphics[scale=0.315]{Figures/Exp-SED-CR-size-mopsi.png}		
	\vspace{-3ex}

	\caption{\small Evaluation of compression ratios (\sed): varying the size of
    trajectories.}
  \label{fig:cr-sed-size}
	\vspace{-2ex}
\end{figure*}

\begin{figure*}[tb!]
	\centering
	\includegraphics[scale=0.315]{Figures/Exp-DAD-CR-size-taxi.png}\hspace{1ex}
	\includegraphics[scale=0.315]{Figures/Exp-DAD-CR-size-service.png} 	\hspace{1ex}
	\includegraphics[scale=0.315]{Figures/Exp-DAD-CR-size-geolife.png}	\hspace{1ex}
	\includegraphics[scale=0.315]{Figures/Exp-DAD-CR-size-mopsi.png}		
	\vspace{-3ex}
	\caption{\small Evaluation of compression ratios (\dad): varying the size of trajectories.}\label{fig:time-size-sed}
	\label{fig:cr-ped-epsilon}
	\vspace{-3ex}
\end{figure*}


%%%%%%%%%%%%%%%%%%%%%%%%%%%%%%%%%%%%%%%%%%%%%%%%%%%%%%%%%%%%%%%%%%%%%%%%%%%%%
\vspace{-1ex}
\subsubsection{Evaluation of Compression Ratios}
%%%%%%%%%%%%%%%%%%%%%%%%%%%%%%%%%%%%%%%%%%%%%%%%%%%%%%%%%%%%%%%%%%%%%%%%

In the first set of tests, we compare the compression ratios of these algorithms, using \ped and \sed, respectively.

\stitle{Exp-1.1: Impacts of the error bound $\epsilon$}.
To compare the compression ratios of these algorithms and evaluate the impacts of $\epsilon$ on them, we varied $\epsilon$ from $10m$ to $100m$ on the entire four datasets, respectively.
The results are reported in Figure~\ref{fig:cr-ped-epsilon} and Figure~\ref{fig:cr-sed-epsilon} (Note that algorithm \opt using \sed is not reported in Figure~\ref{fig:cr-sed-epsilon} because it can not run with the full dataset as input).


\sstab (1) When increasing $\epsilon$, the compression ratios decrease. 
%For example, in \mopsi, the compression ratios are greater than {$4\%$} when $\epsilon$ = $10m$, and less than {$2\%$} when $\epsilon$ = $100m$.

\sstab (2) Dataset \mopsi has the lowest compression ratios, compared with \taxi, \ucar and \geolife, due to its highest sampling rate, \taxi has the highest compression ratios due to its lowest sampling rate, and \ucar and \geolife and  have the compression ratios in the middle accordingly.

\sstab (3) When using \ped, the compression ratios from the best
to the worst are the optimal algorithm \optp, online algorithm \bqsa, batch algorithms \tpa and
\dpa, and one-pass algorithms \siped and \operb. 
The output sizes of algorithm \bqsa are on average
($103.58\%$, $113.32\%$, $120.22\%$, $120.83\%$) of the optimal algorithm \optp
on datasets \dSets, respectively. 
Algorithms \tpa and \dpa are comparable, and their output sizes are on average
($103.17\%$, $125.05\%$, $131.01\%$, $138.01\%$) and ($106.98\%$, $130.03\%$, $140.56\%$, $139.00\%$) of \optp
on datasets \dSets, respectively.
Algorithms \siped and \operb are comparable, and they are on average
($113.09\%$, $136.73\%$, $150.23\%$, $152.29\%$)
and ($119.89\%$, $143.14\%$, $147.80\%$, $152.37\%$) of \optp on datasets \dSets, respectively.
%
For example, in \mopsi, the compression ratios of algorithms
(\optp, \tpa, \dpa, \bqsa, \siped, \operb ) are ($1.6\%$, $2.2\%$, $2.2\%$, $1.9\%$, $2.4\%$, $2.4\%$) when $\epsilon$ = $40m$.

\sstab (4) When using \sed, the compression ratios from the best
to the worst are batch algorithms \tpa and
\dpa, one-pass algorithm \cised, and online algorithm \squishe. 
For example, in \mopsi, the compression ratios of algorithms
(\tpa, \dpa, \squishe, \cised)
are ($3.45\%$, $3.41\%$, $5.75\%$, $3.86\%$), respectively, when $\epsilon$ = $40m$. 

%Algorithms \tpa and \dpa are comparable, and they are on average ($122.69\%$, $129.08\%$, $131.97\%$, $131.01\%$) and ($121.36\%$, $129.27\%$, $130.11\%$, $126.21\%$) of the near optimal algorithm \nopts on datasets \dSets, respectively.
%Algorithms \cised and \squishe are on average ($132.07\%$, $139.67\%$, $146.56\%$, $135.10\%$) and ($164.47\%$, $189.87\%$, $213.30\%$, $186.72\%$) of \nopts on datasets \dSets, respectively.
%
%For example, in \mopsi, the compression ratios of algorithms (\nopts, \tpa, \dpa, \squishe, \cised) are ($2.62\%$, $3.45\%$, $3.41\%$, $5.75\%$, $3.86\%$), respectively, when $\epsilon$ = $40m$. 

\sstab  \todo{(4) When using \dad,}

\sstab (5) The compression ratios of algorithms using \ped are obviously better
than using \sed.
More specifically, the compression ratios of algorithms \tpa and \dpa
using \ped are on average ($55.08\%$, $43.55\%$, $47.49\%$, $63.15\%$) and ($56.75\%$, $45.79\%$,
$50.88\%$, $64.50\%$) of algorithms \tpa and \dpa using \sed on datasets \dSets, respectively.
	


\stitle{{Exp-1.2}: Impacts of the sizes of trajectories}.
To compare the compression ratios of these algorithms and evaluate the impacts of trajectory sizes on them, we chose $10$ trajectories from each dataset \taxi, \ucar, \geolife and \mopsi, respectively,
and varied the size \trajec{|T|} of a trajectory from $1,000$ points to $10,000$ points, while fixed error bound $\epsilon = 60$ meters (\todo{or $\epsilon = 45$ degrees?}).
The results are reported in Figure~\ref{fig:cr-ped-size} and Figure~\ref{fig:cr-sed-size}.


\sstab (1) The sizes of trajectories have few impacts on compression ratios.

\sstab (2) The compression ratios in this test are consistent with test Exp-1.1.

\sstab (3) When using \sed, the compression ratios from the best
to the worst are the near optimal algorithm \opt, batch algorithms \tpa and
\dpa, one-pass algorithm \cised, and online algorithm \squishe. 
%
{Algorithms \tpa and \dpa are comparable, and they are on average
($102.72\%$, $125.23\%$, $143.92\%$, $128.63\%$) and ($103.18\%$, $123.93\%$, $141.46\%$, $121.14\%$)
 of the optimal algorithm \opt on datasets \dSets, respectively.}
%
{Algorithms \cised and \squishe are on average ($108.00\%$,
  $134.35\%$, $159.30\%$, $136.06\%$) and ($110.27\%$, $165.94\%$, $225.68\%$, $206.90\%$)
 of \opt on datasets \dSets, respectively.}

\sstab  \todo{(4) When using \dad,}

\sstab (4) The compression ratios of algorithms using \ped are obviously better
than using \sed.
More specifically, the compression ratios of algorithms \optp, \tpa and \dpa
using \ped are on average {($56.03\%$, $33.42\%$, $33.12\%$, $72.42\%$),
($55.78\%$, $31.88\%$, $34.27\%$, $69.44\%$) and ($58.09\%$, $34.82\%$,
$40.91\%$, $75.06\%$)} of algorithms \opt, \tpa and \dpa using \sed on datasets \dSets, respectively. 



\begin{figure*}[tb!]
	\centering
	\includegraphics[scale=0.315]{Figures/Exp-PED-error-epsilon-taxi.png} \hspace{1ex}
	\includegraphics[scale=0.315]{Figures/Exp-PED-error-epsilon-service.png}	\hspace{1ex}
	\includegraphics[scale=0.315]{Figures/Exp-PED-error-epsilon-geolife.png}	\hspace{1ex}
	\includegraphics[scale=0.315]{Figures/Exp-PED-error-epsilon-mopsi.png}	
	\vspace{-3ex}
	\caption{\small Evaluation of average errors (\ped): varying the error bound $\epsilon$.}
	\label{fig:ae-ped-epsilon}
	\vspace{-2ex}
\end{figure*}

\begin{figure*}[tb!]
	\centering
	\includegraphics[scale=0.315]{Figures/Exp-SED-error-epsilon-taxi.png} \hspace{1ex}
	\includegraphics[scale=0.315]{Figures/Exp-SED-error-epsilon-service.png}	\hspace{1ex}
	\includegraphics[scale=0.315]{Figures/Exp-SED-error-epsilon-geolife.png}	\hspace{1ex}
	\includegraphics[scale=0.315]{Figures/Exp-SED-error-epsilon-mopsi.png}		
	\vspace{-3ex}
	\caption{\small Evaluation of average errors (\sed): varying the error bound $\epsilon$.}
	\label{fig:ae-sed-epsilon}
	\vspace{-2ex}
\end{figure*}

\begin{figure*}[tb!]
	\centering
	\includegraphics[scale=0.315]{Figures/Exp-DAD-error-epsilon-taxi.png}\hspace{1ex}
	\includegraphics[scale=0.315]{Figures/Exp-DAD-error-epsilon-service.png} 	\hspace{1ex}
	\includegraphics[scale=0.315]{Figures/Exp-DAD-error-epsilon-geolife.png}	\hspace{1ex}
	\includegraphics[scale=0.315]{Figures/Exp-DAD-error-epsilon-mopsi.png}		
	\vspace{-3ex}
	\caption{\small Evaluation of average errors (\dad): varying the error bound $\epsilon$.}
	\label{fig:cr-ped-epsilon}
	\vspace{-2ex}
\end{figure*}

\begin{figure*}[tb!]
	\centering
	\includegraphics[scale=0.315]{Figures/Exp-PED-error-size-taxi.png}\hspace{1ex}
	\includegraphics[scale=0.315]{Figures/Exp-PED-error-size-service.png} 	\hspace{1ex}
	\includegraphics[scale=0.315]{Figures/Exp-PED-error-size-geolife.png}	\hspace{1ex}
	\includegraphics[scale=0.315]{Figures/Exp-PED-error-size-mopsi.png}		
	\vspace{-3ex}
	\caption{\small Evaluation of average errors (\ped): varying the size of
    trajectories.}
  \label{fig:ae-ped-size}
	\vspace{-2ex}
\end{figure*}

\begin{figure*}[tb!]
	\centering
	\includegraphics[scale=0.315]{Figures/Exp-SED-error-size-taxi.png}\hspace{1ex}
	\includegraphics[scale=0.315]{Figures/Exp-SED-error-size-service.png} 	\hspace{1ex}
	\includegraphics[scale=0.315]{Figures/Exp-SED-error-size-geolife.png}	\hspace{1ex}
	\includegraphics[scale=0.315]{Figures/Exp-SED-error-size-mopsi.png}		
	\vspace{-3ex}
	\caption{\small Evaluation of average errors (\sed): varying the size of
    trajectories.}
  \label{fig:ae-sed-size}
	\vspace{-2ex}
\end{figure*}

\begin{figure*}[tb!]
	\centering
	\includegraphics[scale=0.315]{Figures/Exp-DAD-error-size-taxi.png}\hspace{1ex}
	\includegraphics[scale=0.315]{Figures/Exp-DAD-error-size-service.png} 	\hspace{1ex}
	\includegraphics[scale=0.315]{Figures/Exp-DAD-error-size-geolife.png}	\hspace{1ex}
	\includegraphics[scale=0.315]{Figures/Exp-DAD-error-size-mopsi.png}		
	\vspace{-3ex}
	\caption{\small Evaluation of average errors (\dad): varying the size of trajectories.}\label{fig:time-size-sed}
	\label{fig:cr-ped-epsilon}
	\vspace{-2ex}
\end{figure*}

%%%%%%%%%%%%%%%%%%%%%%%%%%%%%%%%%%%%%%%%%%%%%%%%%%%%%%%%%%%%%%%%%%%%%%%%%%%%%%%
\vspace{-0.5ex}
\subsubsection{Evaluation of Average Errors}
%%%%%%%%%%%%%%%%%%%%%%%%%%%%%%%%%%%%%%%%%%%%%%%%%%%%%%%%%%%%%%%%%%%%%%%%%%%%%%
%In the second set of tests, we evaluate the average errors of these algorithms.
We then evaluate the average errors of these algorithms.





\stitle{Exp-2.1: Impacts of the error bound $\epsilon$}.
To compare the average errors of these algorithms and evaluate the impacts of $\epsilon$ on them, we varied $\epsilon$ from $10m$ to $100m$ on the entire four datasets, respectively.
The results are reported in Figure~\ref{fig:ae-ped-epsilon} and Figure~\ref{fig:ae-sed-epsilon}.


\sstab (1) When increasing $\epsilon$, the average errors increase linearly. 

\sstab (2) Datasets have few impacts on the average errors.

\sstab (3) When using \ped, the average errors from the smallest
to the largest are batch algorithms \tpa and \dpa, one-pass
algorithms \siped and \operb, the optimal algorithm \optp and online algorithm \bqsa. 
Algorithms \tpa and
\dpa are comparable, and they are on average ($77.43\%$, $58.69\%$, $61.34\%$,
$57.57\%$) and ($88.12\%$, $57.61\%$, $62.66\%$, $60.23\%$) of the optimal algorithm \optp on datasets \dSets, respectively.
Algorithms \siped and \operb are comparable, and they are on average
($73.08\%$, $80.96\%$, $79.12\%$, $79.33\%$), ($73.03\%$, $70.60\%$, $76.64\%$, $78.71\%$) of \optp on datasets \dSets, respectively.
Algorithm \bqsa is on average ($97.69\%$, $104.67\%$, $108.91\%$, $106.92\%$) of \optp on datasets \dSets, respectively.
%
For example, in \mopsi, the average errors of algorithms
(\optp, \tpa, \dpa, \bqsa, \siped, \operb ) are ($16.08$, $9.19$, $9.68$, $17.4$, $12.96$, $12.77$)  metres when $\epsilon$ = $40m$.

\sstab (4) When using \sed, the average errors from the smallest
to the largest are online algorithm \squishe, batch algorithms \tpa and \dpa,
and one-pass algorithm \cised.
For example, in \mopsi, the average errors of algorithms
(\tpa, \dpa, \squishe, \cised) are ($12.17$, $12.20$, $6.76$, $14.71$) metres, respectively, when $\epsilon$ = $40m$.

\sstab  \todo{(5) When using \dad,}


\sstab (5) The average errors of algorithms using \sed are a bit larger than using \ped. 




\stitle{{Exp-2.2}: Impacts of the sizes of trajectories}.
To compare the errors of these algorithms and evaluate the impacts of the sizes of trajectories on them, we chose the same $10$ trajectories from each dataset \taxi, \ucar, \geolife and \mopsi, respectively,
and varied the size \trajec{|T|} of a trajectory from $1,000$ points to $10,000$ points, while fixed error bound $\epsilon = 60$ meters (\todo{or $\epsilon = 45$ degrees?}).
The results are reported in Figure~\ref{fig:ae-ped-size} and Figure~\ref{fig:ae-sed-size}.

\sstab (1) The sizes of trajectories have few impacts on average errors.

\sstab (2) The average errors in this test are consistent with test Exp-2.1.

\sstab (3) When using \sed, the average errors from the smallest
to the largest are online algorithm \squishe, batch algorithms \tpa and \dpa,
one-pass algorithm \cised, and the naive optimal algorithm \opt.
Algorithms \tpa and \dpa are comparable, and they are on average
{($87.22\%$, $60.36\%$, $66.11\%$, $62.43\%$), ($91.04\%$, $62.54\%$, $67.04\%$, $68.64\%$)} of \opt on datasets \dSets, respectively.
Algorithms \cised and \squishe are on average {($73.15\%$, $75.29\%$, $76.03\%$, $81.44\%$), ($46.27\%$, $40.61\%$, $38.15\%$, $34.22\%$)} of \opt on datasets \dSets, respectively.

\sstab  \todo{(4) When using \dad,}



%%%%%%%%%%%%%%%%%%%%%%%%%%%%%%%%%%%%%%%%%%%%%%%%%%%%%%%%%%%%%%%%%%%%%%%%%%%%%%
\vspace{-1ex}
\subsubsection{Evaluation of Compression Efficiency}
%%%%%%%%%%%%%%%%%%%%%%%%%%%%%%%%%%%%%%%%%%%%%%%%%%%%%%%%%%%%%%%%%%%%%%%%%%%%%%

In the third set of tests, we compare the efficiency (execution time) of these algorithms.
%For fairness, we load and compress trajectories one by one, and only count the running time of the compressing process.
%For a small size trajectory, we repeat compress it tens of times and accumulation the total running time so as  to get the average compression time.


\stitle{Exp-3.1: Impacts of the error bound $\epsilon$}.
To evaluate the impacts of $\epsilon$ on efficiencies of these algorithms, we varied $\epsilon$ from $10m$ to $100m$ on the entire four datasets, respectively.
%
The results show that the running time of algorithm \optp is on
average ($261.71$, $334.49$, $403.56$, $202.82$) times slower than one-pass algorithms \siped and \cised on datasets \dSets, respectively. 
For example, in \mopsi, the running time of algorithms
(\optp, \siped, \cised) are  {($3014.2$, $7.5$, $9.8$)} seconds when $\epsilon$ = $40m$.
%
It is not easy to show all these algorithms in a figure, thus, only the results of sub-optimal algorithms are reported in Figure~\ref{fig:time-epsilon-ped} and Figure~\ref{fig:time-epsilon-sed}.

\sstab (1) One-pass algorithms are not very sensitive to error bound $\epsilon$. 
More specifically, the running time of algorithms \dpa and \tpa decreases and increases with the increase of $\epsilon$ due to the top-down and bottom-up approaches they applying, respectively. It also shows that \dpa is more suitable than \tpa in circumstances that compression ratios are less than $\sim 10\%$.

\sstab (2) When using \ped, in most cases, the running time from the smallest to the largest are one-pass algorithms \siped and \operb, batch algorithms \tpa and \dpa, and online algorithm \bqsa. 
Algorithms \siped and \operb are comparable, and algorithms \tpa, \dpa and \bqsa are on average
($19.19$, $26.79$, $28.25$, $29.87$), ($17.90$, $16.32$, $15.40$, $11.02$) and ($15.07$, $37.73$, $62.23$, $61.29$)
times slower than the one-pass algorithm \siped on datasets \dSets, respectively.
% 
For example, in \mopsi, the running time of algorithms
(\tpa, \dpa, \bqsa, \siped, \operb ) are ($232.9$, $124.2$, $469.4$, $7.6$, $8.6$) seconds when $\epsilon$ = $40m$.

\sstab (3) When using \sed, the running time from the smallest to the largest are one-pass algorithm \cised, online algorithm \squishe, and batch algorithms \tpa and \dpa. 
Algorithms \tpa, \dpa and \squishe are on average
($8.58$, $13.33$, $15.81$, $13.09$), ($12.81$, $12.93$, $10.64$, $8.79$) and
($2.63$, $2.75$, $2.78$, $2.57$) times slower than \cised on datasets \dSets, respectively.
%
For example, in \mopsi, the running time of algorithms
(\tpa, \dpa, \squishe, \cised) are  ($156.6$, $104.8$, $27.2$, $9.7$) seconds when $\epsilon$ = $40m$.

\sstab  \todo{(4) When using \dad,}

\sstab (4) Batch algorithms \dpa and \tpa using \sed run a bit faster than using
\ped, while the one-pass algorithm \cised run $1.3$ times slower than \siped and \operb.



\begin{figure*}[tb!]
	\centering
	\includegraphics[scale=0.315]{Figures/Exp-PED-time-epsilon-taxi.png}	\hspace{1ex}
	\includegraphics[scale=0.315]{Figures/Exp-PED-time-epsilon-service.png}	\hspace{1ex}
	\includegraphics[scale=0.315]{Figures/Exp-PED-time-epsilon-geolife.png}	\hspace{1ex}
	\includegraphics[scale=0.315]{Figures/Exp-PED-time-epsilon-mopsi.png}	\hspace{1ex}
	\vspace{-3ex}
	\caption{\small Evaluation of efficiency (\ped): varying the error bound $\epsilon$.}\label{fig:time-epsilon-ped}
	\vspace{-2ex}
\end{figure*}

\begin{figure*}[tb!]
	\centering
	\includegraphics[scale=0.315]{Figures/Exp-SED-time-epsilon-taxi.png}	\hspace{1ex}
	\includegraphics[scale=0.315]{Figures/Exp-SED-time-epsilon-service.png}	\hspace{1ex}
	\includegraphics[scale=0.315]{Figures/Exp-SED-time-epsilon-geolife.png}	\hspace{1ex}
	\includegraphics[scale=0.315]{Figures/Exp-SED-time-epsilon-mopsi.png}	\hspace{1ex}
	\vspace{-3ex}
	\caption{\small Evaluation of efficiency (\sed): varying the error bound $\epsilon$.}\label{fig:time-epsilon-sed}
	\vspace{-2ex}
\end{figure*}

\begin{figure*}[tb!]
	\centering
	\includegraphics[scale=0.315]{Figures/Exp-DAD-time-epsilon-taxi.png}\hspace{1ex}
	\includegraphics[scale=0.315]{Figures/Exp-DAD-time-epsilon-service.png} 	\hspace{1ex}
	\includegraphics[scale=0.315]{Figures/Exp-DAD-time-epsilon-geolife.png}	\hspace{1ex}
	\includegraphics[scale=0.315]{Figures/Exp-DAD-time-epsilon-mopsi.png}		
	\vspace{-3ex}
	\caption{\small Evaluation of efficiency (\dad): varying the error bound $\epsilon$.}
	\label{fig:cr-ped-epsilon}
	\vspace{-2ex}
\end{figure*}

\begin{figure*}[tb!]
	\centering
	\includegraphics[scale=0.315]{Figures/Exp-PED-time-size-taxi.png}\hspace{1ex}
	\includegraphics[scale=0.315]{Figures/Exp-PED-time-size-service.png}	\hspace{1ex}
	\includegraphics[scale=0.315]{Figures/Exp-PED-time-size-geolife.png}	\hspace{1ex}
	\includegraphics[scale=0.315]{Figures/Exp-PED-time-size-mopsi.png}	\hspace{1ex}
	\vspace{-3ex}
	\caption{\small Evaluation of efficiency (\ped): varying the size of trajectories.}\label{fig:time-size-ped}
	\vspace{-2ex}
\end{figure*}

\begin{figure*}[tb!]
	\centering
	\includegraphics[scale=0.315]{Figures/Exp-SED-time-size-taxi.png}\hspace{1ex}
	\includegraphics[scale=0.315]{Figures/Exp-SED-time-size-service.png}	\hspace{1ex}
	\includegraphics[scale=0.315]{Figures/Exp-SED-time-size-geolife.png}	\hspace{1ex}
	\includegraphics[scale=0.315]{Figures/Exp-SED-time-size-mopsi.png}	\hspace{1ex}
	\vspace{-3ex}
	\caption{\small Evaluation of efficiency (\sed): varying the size of trajectories.}\label{fig:time-size-sed}
	\vspace{-2ex}
\end{figure*}

\begin{figure*}[tb!]
	\centering
	\includegraphics[scale=0.315]{Figures/Exp-DAD-time-size-taxi.png}\hspace{1ex}
	\includegraphics[scale=0.315]{Figures/Exp-DAD-time-size-service.png} 	\hspace{1ex}
	\includegraphics[scale=0.315]{Figures/Exp-DAD-time-size-geolife.png}	\hspace{1ex}
	\includegraphics[scale=0.315]{Figures/Exp-DAD-time-size-mopsi.png}		
	\vspace{-3ex}
	\caption{\small Evaluation of efficiency (\dad): varying the size of trajectories.}\label{fig:time-size-sed}
	\label{fig:cr-ped-epsilon}
	\vspace{-2ex}
\end{figure*}





\stitle{{Exp-3.2}: Impacts of the sizes of trajectories}.
To evaluate the impacts of the number of data points in a trajectory (\ie the size of a trajectory),
we chose $10$ trajectories from each dataset \taxi, \ucar, \geolife and \mopsi, respectively,
and varied the size \trajec{|T|} of trajectories from $1,000$ points to $10,000$ points, while fixed error bound $\epsilon = 60$ meters (\todo{or $\epsilon = 45$ degrees?}).
The results are reported in Figure~\ref{fig:time-size-ped} and Figure~\ref{fig:time-size-sed}. Note the running time of algorithm \opt is great slower than \optp, and it is hard to show it in the figure.

\sstab(1) One-pass algorithms \siped, \operb and \cised scale well with the increase of trajectory size \eat{on all datasets} and show a linear running time, while batch and online algorithms do not.
This is consistent with their time complexity analyses.

\sstab(2) The running time from the smallest to the largest of these algorithms is the same as test {Exp-3.1}.

%\sstab (2) When using \ped, the running time from the smallest to the largest are one-pass algorithms \siped and \operb, and batch and online algorithms \tpa, \dpa and \bqsa. Algorithms \siped and \operb are comparable. Algorithms \tpa, \dpa and \bqsa are comparable, and they are on average \textcolor{red}{($3.8$--$5.3$, $3.5$--$4.8$, $4.6$--$7.2$, $6.2$--$8.4$)} times slower than the one-pass algorithms \siped and \operb on datasets \dSets, respectively. 

%\sstab (3) When using \sed, the running time from the smallest to the largest are one-pass algorithm \cised, online algorithm \squishe, and batch algorithms \tpa and \dpa.  Algorithms \squishe, \tpa and \dpa are on average \textcolor{red}{($9.6$--$17.6$, $8.8$--$15.4$, $8.4$--$16.3$, $9.0$--$14.4$)}, \textcolor{red}{($9.6$--$17.6$, $8.8$--$15.4$, $8.4$--$16.3$, $9.0$--$14.4$)} and \textcolor{red}{($9.6$--$17.6$, $8.8$--$15.4$, $8.4$--$16.3$, $9.0$--$14.4$)} times slower than \cised on datasets \dSets, respectively.

%\sstab (4) Batch algorithms \dpa and \tpa using \sed run a bit faster than using \ped, while the one-pass algorithm \cised run \textcolor{red}{$2.0$--$3.0$} times slower than \siped and \operb.


%%%%%%%%%%%%%%%%%%%%%%%%%%%%%%%%%%%%%%%%%%%%%%%%%%%%%%%%%%%%%%%%%%%%%%%%%%%%%%
\stitle{Summary}.
%%%%%%%%%%%%%%%%%%%%%%%%%%%%%%%%%%%%%%%%%%%%%%%%%%%%%%%%%%%%%%%%%%%%%%%%%%%%%%
From these tests we find the following.

\emph{\sstab{(1) Compression ratios}}. 
(a) When using \ped, the output data sizes of sub-optimal algorithms (\tpa,
\dpa, \bqsa, \siped, \operb) are on average ($125.67\%$, $130.24\%$, $115.94\%$, $139.41\%$, $141.15\%$)
of the optimal algorithm \optp, respectively.
%the compression ratios from the best to the worst are the optimal algorithm \optp, batch and online algorithms \tpa, \dpa and \bqsa, and one-pass algorithms \siped and \operb. 
(b) When using \sed, the output data sizes of sub-optimal algorithms (\tpa,
\dpa, \squishe, \cised) are on average ($126.10\%$, $124.31\%$, $180.41\%$, $136.66\%$) of the naive optimal algorithm \opt, respectively.
%the compression ratios from the best to the worst are the near optimal algorithm \nopts, batch algorithms \tpa and \dpa, one-pass algorithm \cised, and online algorithm \squishe.  
(c) The compression ratios of algorithms using \ped are better than
using \sed. The output data sizes of algorithms (\optp, \tpa, \dpa) using \ped
are on average ($46.66\%$, $46.64\%$, $48.96\%$) of (\opt, \tpa, \dpa) using \sed, respectively.

\emph{\sstab{(2) Average errors}}. 
(a) When using \ped, the average errors from the smallest to the largest are batch algorithms \tpa, \dpa, one-pass algorithms \siped and \operb, the optimal algorithm \optp, and online algorithm \bqsa. 
(b) When using \sed, the average errors from the smallest to the largest are online algorithm \squishe, batch algorithms \tpa and \dpa, one-pass algorithm \cised, and the naive optimal algorithm \opt.
%(c) The average errors of algorithms using \sed are a bit larger than using \ped. 

\emph{\sstab{(3) Efficiency}}.
(a) When using \ped, algorithms \siped and \operb are comparable, and algorithms
(\tpa, \dpa, \bqsa) are on average ($23.97$, $15.97$, $37.16$) times slower than the one-pass algorithms \siped, respectively. 
%the running time from the smallest to the largest are one-pass algorithms \siped and \operb, and batch and online algorithms \tpa, \dpa and \bqsa. 
(b) When using \sed, algorithms (\tpa, \dpa, \squishe) are on average ($11.60$, $14.37$, $2.63$) times slower than \cised, respectively.
%the running time from the smallest to the largest are one-pass algorithm \cised, online algorithm \squishe, and batch algorithms \tpa and \dpa. 
(c) Batch algorithms \dpa and \tpa using \sed run a bit faster than using \ped, while the one-pass algorithm \cised runs {$1.30$} times slower than \siped and \operb.


%%********************************* The End **********************************







