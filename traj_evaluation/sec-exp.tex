%%%%%%%%%%%%%%%%%%%%%%%%%%%%%%%%%%%%%%%%%%%%%%%%%%%%%%%%%%%%%%%%%%%%%%%%%%%%%%
\section{Evaluation} %Experimental Study
\label{sec-exp}
%%%%%%%%%%%%%%%%%%%%%%%%%%%%%%%%%%%%%%%%%%%%%%%%%%%%%%%%%%%%%%%%%%%%%%%%%%%%%%
In this section, we present an extensive experimental study of the nine \lsa algorithms listed in \mytable{tab:summary-lsa} on trajectory data sets.
Using four real-life datasets, we conducted two sets of experiments to evaluate:
%(1) the compression ratios and average errors of these algorithms, and the impacts of distance metrics,\ie \sed and \ped, to them
%(2) the execution time of them.
(1) the effectiveness of these algorithms, and
(2) the efficiency of them.
\textcolor{blue}{Note the additional comparison of the effectiveness of the near optimal algorithm (\nopts) and the optimal algorithm (\opt), and the efficiencies of the (near) optimal algorithms are presented in Appendix B.}


\subsection{Experimental Setting}
We first introduce the settings of our experimental study.

\stitle{Real-life Trajectory Datasets}.
We use four real-life datasets shown in \mytable{tab:datasets}, namely, Service car trajectory data (\ucar) collected by a Chinese car rental company, Geolife trajectory data (\geolife) collected in GeoLife project~\cite{Web:Geolife}, Mopsi trajectory data (\mopsi) collected in Mopsi project \cite{Web:Mopsi} and Act trajectory data (\act) collected by our team members, to evaluate those \lsa algorithms.

\begin{table}
	%\vspace{-2ex}
	\caption{\small Real-life trajectory datasets}
	\centering
	\small
	\begin{tabular}{|l|c|c|c|r|}
		\hline
		\kw{Data}& \kw{Number\ of}     &\kw{Sampling}   &\kw{Points~Per}    &\kw{Total} \\
		\kw{Sets} & \kw{Trajectories}   &\kw{Rates (s)}  &\kw{Trajectory (K)}&\kw{points}\\	\hline
		%\taxi	&12,727	    &60	        &$\sim39.1$      &498M \\	\hline
		%\truck	&10,368	    &1-60	    &$\sim71.9$     &746M \\	\hline
		%\ucar	&11,000	    &3-5	    &$\sim119.1$   &1.31G\\		\hline
		%
		\ucar	&1,00	    &3-5	&$\sim114.0$   &114M 	\\	\hline
		\geolife\cite{Web:Geolife} &182	    &1-5	&$\sim131.4$   &24.2M	\\	\hline
		\mopsi\cite{Web:Mopsi}	&51	    	&2	    &$\sim153.9$   &7.9M	\\	\hline
		\act	& 10	    &1	    &$\sim11.8$    &112.8K	\\	\hline
	\end{tabular}
	\label{tab:datasets}
	\vspace{-3ex}
\end{table}


% \ni \emph{(1) Truck trajectory data} (\truck) is the GPS trajectories collected by \eat{10,368} trucks equipped with GPS sensors in China
% during a period from Mar. 2015 to Oct. 2015. The sampling rate varied from 1s to 60s.
%Trajectories mostly have around $50$ to $90$ thousand data points.

%\vspace{0.5ex}
%\ni \emph{(1) Service car trajectory data} (\ucar) is the GPS trajectories collected by a Chinese car rental company during Apr. 2015 to Nov. 2015. The sampling rate was one point per $3$--$5$ seconds, and each trajectory has around $114.1K$ points.
%.We randomly chose $1,000$ cars from them

%\vspace{0.5ex}
%\ni \emph{(2) GeoLife trajectory data} (\geolife) is the GPS trajectories collected in GeoLife project~\cite{Web:Geolife} by 182 users in a period from Apr. 2007 to Oct. 2011. These trajectories have a variety of sampling rates, among which 91\% are logged in each 1-5 seconds per point. %or each 5-10 meters
%This dataset contains 182 trajectories, one trajectory for each user, with a total distance of about 1.2 million kilometers. 
%The longest trajectory has 2,156,994 points.

%\vspace{0.5ex}
%\ni \emph{(3) Mopsi trajectory data} (\mopsi) is the GPS trajectories collected in Mopsi project~\cite{Web:Mopsi} by 51 users in a period from 2008 to 2014. Most routes are in Joensuu region, Finland. The sampling rate was one point per $2$ seconds, and each trajectory has around $153.9K$ points.
%exist on every continent.

%\vspace{0.5ex}
%\ni \emph{(4) Act trajectory data} (\act) is a small set GPS trajectories collected with a high sampling rate of one point per second by our team members in 2017. There are 10 trajectories and each trajectory has around 11.8K points.

%The details of these datasets are shown in Table~\ref{tab:dataset}.

\stitle{Algorithms and implementation}.
We implement the algorithms listed in \mytable{tab:summary-lsa}.
For algorithm \cised, we fixed parameter $m=16$ as evaluated in \cite{Lin:Cised}, \ie 16-edges inscribe regular polygon.
For algorithm \nopts, we fixed parameter \textcolor{red}{$m=32$}. 
All algorithms were implemented with Java.
All tests were run on an x64-based  PC with 4 Intel(R) Core(TM) i5-4570 CPU @ 3.20GHz  and 16GB of memory, and each test was repeated
over 3 times and the average is reported here.


\eat{%%%%%%%%%%%%%%%%%%%
	\stitle{Real-life Trajectory Datasets}.
	We use four real-life datasets shown in Table~\ref{tab:dataset} to test our solutions.
	
	\sstab{\bf(1) Taxi trajectory data}, referred to as \taxi, is the GPS trajectories collected by $12,727$ taxies equipped with GPS sensors in Beijing during a period
	from Nov. 1, 2010 to Nov. 30, 2010. The sampling rate was one point  per 60s, and \taxi has $39,100$ data points on average per trajectory.
	
	\sstab{\bf(2) Truck trajectory data}, referred to as \truck, is the GPS trajectories collected by 10,368 trucks equipped with GPS sensors in China
	during a period from Mar. 2015 to Oct. 2015. The sampling rate varied from 1s to 60s. Trajectories mostly have around $50$ to $90$ thousand data points.
	
	\sstab{\bf(3) Service car trajectory data}, referred to as \ucar,  is the GPS trajectories collected by a car rental company.
	We chose $11,000$ cars from them, during Apr. 2015 to Nov. 2015. The sampling rate was one point per $3$--$5$ seconds, and
	each trajectory has around $119.1K$ data points.
	
	{\sstab{\bf(4) GeoLife trajectory data}, refered to as \geolife, is the GPS trajectories collected in GeoLife project~\cite{Zheng:GeoLife} by 182 users in a period from Apr. 2007 to Oct. 2011. These trajectories have a variety of sampling rates, among which 91\% are logged in each 1-5 seconds or each 5-10 meters per point. The longest trajectory has 2,156,994 points.}
	%This dataset contains 182 trajectories, one trajectory for each user, with a total distance of about 1.2 million kilometers. 
}%%%%%%%%%%%%%%%%%%%%%%%%%%%%%%%%%%%%%%%%%%%%%%%%%%%%%%%%%%%%%




%%%%%%%%%%%%%%%%%%%%%%%%%%%%%%%%%%%%%%%%%%%%%%%%%%%%%%%%%%%%%%%%%%%%%%%%%%%%%%
\subsection{Experimental Results}
%%%%%%%%%%%%%%%%%%%%%%%%%%%%%%%%%%%%%%%%%%%%%%%%%%%%%%%%%%%%%%%%%%%%%%%%%%%%%%
We next present our findings.

\begin{figure*}[tb!]
	\centering
	\includegraphics[scale=0.315]{Figures/Exp-PED-CR-epsilon-service.png} 	\hspace{1ex}
	\includegraphics[scale=0.315]{Figures/Exp-PED-CR-epsilon-geolife.png}	\hspace{1ex}
	\includegraphics[scale=0.315]{Figures/Exp-PED-CR-epsilon-mopsi.png}		\hspace{1ex}
	\includegraphics[scale=0.315]{Figures/Exp-PED-CR-epsilon-geolife.png}
	\vspace{-2.5ex}
	\caption{\small Evaluation of compression ratios (\ped): varying the error bound $\epsilon$.}
	\label{fig:cr-ped}
	\vspace{-2ex}
\end{figure*}

\begin{figure*}[tb!]
	\centering
	\includegraphics[scale=0.315]{Figures/Exp-SED-CR-epsilon-service.png} 	\hspace{1ex}
	\includegraphics[scale=0.315]{Figures/Exp-SED-CR-epsilon-geolife.png}	\hspace{1ex}
	\includegraphics[scale=0.315]{Figures/Exp-SED-CR-epsilon-mopsi.png}		\hspace{1ex}
	\includegraphics[scale=0.315]{Figures/Exp-SED-CR-epsilon-geolife.png}
	\vspace{-2.5ex}
	\caption{\small Evaluation of compression ratios (\sed): varying the error bound $\epsilon$.}
	\label{fig:cr-sed}
	\vspace{-2ex}
\end{figure*}

\begin{figure*}[tb!]
	\centering
	\includegraphics[scale=0.315]{Figures/Exp-PED-error-epsilon-service.png}	\hspace{1ex}
	\includegraphics[scale=0.315]{Figures/Exp-PED-error-epsilon-geolife.png}	\hspace{1ex}
	\includegraphics[scale=0.315]{Figures/Exp-PED-error-epsilon-mopsi.png}	\hspace{1ex}
	\includegraphics[scale=0.315]{Figures/Exp-PED-error-epsilon-geolife.png} 
	\vspace{-2.5ex}
	\caption{\small Evaluation of average errors (\ped): varying the error bound $\epsilon$.}
	\label{fig:ae-ped}
	\vspace{-2ex}
\end{figure*}

\begin{figure*}[tb!]
	\centering
	\includegraphics[scale=0.315]{Figures/Exp-SED-error-epsilon-service.png}	\hspace{1ex}
	\includegraphics[scale=0.315]{Figures/Exp-SED-error-epsilon-geolife.png}	\hspace{1ex}
	\includegraphics[scale=0.315]{Figures/Exp-SED-error-epsilon-mopsi.png}		\hspace{1ex}
	\includegraphics[scale=0.315]{Figures/Exp-SED-error-epsilon-geolife.png}
	\vspace{-2.5ex}
	\caption{\small Evaluation of average errors (\sed): varying the error bound $\epsilon$.}
	\label{fig:ae-sed}
	\vspace{-3ex}
\end{figure*}

%%%%%%%%%%%%%%%%%%%%%%%%%%%%%%%%%%%%%%%%%%%%%%%%%%%%%%%%%%%%%%%%%%%%%%%%%%%%%
\subsubsection{Evaluation of Compression Effectiveness}
%%%%%%%%%%%%%%%%%%%%%%%%%%%%%%%%%%%%%%%%%%%%%%%%%%%%%%%%%%%%%%%%%%%%%%%%

In the first set of tests, we compare the effectiveness (compression ratios and average errors) of these algorithms.% using \ped and \sed, respectively.

\stitle{Exp-1.1: Compression ratios}. Given a set of trajectories $\{\dddot{\mathcal{T}_1}, \ldots, \dddot{\mathcal{T}_M}\}$ and their piecewise line representations $\{\overline{\mathcal{T}_1}, \ldots, \overline{\mathcal{T}_M}\}$,
the compression ratio is $(\sum_{j=1}^{M} |\overline{\mathcal{T}}_j |)/(\sum_{j=1}^{M} |\dddot{\mathcal{T}}_j |)$.
By this definition, algorithms with lower compression ratios are better.

To compare the compression ratios of these algorithms and evaluate the impacts of $\epsilon$ on them, we varied $\epsilon$ from $10m$ to $100m$ on the entire four datasets, respectively.
The results are reported in Figure~\ref{fig:cr-ped} and Figure~\ref{fig:cr-sed}.


\sstab (1) When increasing $\epsilon$, the compression ratios decrease. For example, in \mopsi,
the compression ratios are greater than {$5\%$} when $\epsilon$ = $10m$, but are less than {$2\%$} when $\epsilon$ = $100m$.

\sstab (2) Dataset \act has the lowest compression ratios, compared with \ucar, \geolife and \mopsi, due to its highest sampling rate, \ucar has the highest compression ratios due to its lowest sampling rate, and \geolife and \mopsi have the compression ratios in the middle accordingly.

\sstab (3) When using \ped, in most cases, the compression ratios from the best
to the worst are the optimal algorithm \optp, batch and online algorithms \tpa,
\dpa and \bqsa, and one-pass algorithms \siped and \operb. Algorithms \tpa, \dpa
and \bqsa are comparable, and their output sizes are on average
($58.60\%$, $61.34\%$, $57.57\%$, $61.34\%$), ($57.56\%$, $62.66\%$, $60.23\%$,
$62.66\%$) and \textcolor{red}{($5\%,5\%,5\%,5\%$)} of the optimal algorithm \optp
on datasets (\ucar, \geolife, \mopsi, \act), respectively.
Algorithms \siped and \operb are comparable, and they are on average ($80.88\%$,
$79.12\%$, $79.33\%$, $79.12\%$) and ($70.64\%$, $76.64\%$, $78.71\%$, $76.64\%$) of
the optimal algorithm \optp on datasets (\ucar, \geolife, \mopsi, \act), respectively.
For example, in \mopsi, the compression ratios of algorithms
(\optp,\tpa,\dpa,\bqsa,\siped,\operb ) are ($1.61\%$, $2.21\%$, $2.23\%$,
\textcolor{red}{$2.44\%$},$2.44\%$, $2.44\%$) when $\epsilon$ = $40m$.

\sstab (4) When using \sed, in most cases, the compression ratios from the best
to the worst are the near optimal algorithm \nopts, batch algorithms \tpa and
\dpa, one-pass algorithm \cised, and online algorithm \squishe. Algorithms \tpa
and \dpa are comparable, and they are on average
($122.69\%$, $129.08\%$, $131.97\%$, $129.08\%$) and ($121.36\%$, $129.27\%$, $130.11\%$, $129.27\%$)
 of the near optimal algorithm \nopts on datasets (\ucar, \geolife, \mopsi,
 \act), respectively.
 Algorithms \cised and \squishe are on average ($132.07\%$, $139.67\%$,
 $146.56\%$, $139.67\%$) and ($164.47\%$, $189.87\%$, $213.30\%$, $189.87\%$)
 of the near optimal algorithm \nopts on datasets (\ucar, \geolife, \mopsi, \act), respectively.
For example, in \mopsi, the compression ratios of algorithms
(\nopts,\tpa,\dpa,\squishe,\cised)
are ($2.62\%$, $3.45\%$, $3.41\%$, $5.75\%$, $3.86\%$) when $\epsilon$ = $40m$. 

\sstab (5) The compression ratios of algorithms using \ped are obviously better
than using \sed.
More specifically, the compression ratios of algorithms (\optp, \tpa, \dpa)
using \ped are on average ($42.17\%$, $46.77\%$, $60.37\%$, $46.77\%$),
($42.88\%$, $47.49\%$, $63.15\%$, $47.49\%$) and ($45.17\%$, $50.88\%$, $64.50\%$,
$50.88\%$) of algorithms (\nopts, \tpa, \dpa) using \sed on datasets (\ucar, \geolife, \mopsi, \act), respectively.
	



%%%%%%%%%%%%%%%%%%%%%%%%%%%%%%%%%%%%%%%%%%%%%%%%%%%%%%%%%%%%%%%%%%%%%%%%%%%%%%%
%\vspace{-0.5ex}
%\subsubsection{Evaluation of Average Errors}
%%%%%%%%%%%%%%%%%%%%%%%%%%%%%%%%%%%%%%%%%%%%%%%%%%%%%%%%%%%%%%%%%%%%%%%%%%%%%%
%In the third set of tests, we evaluate the average errors of these algorithms.


\stitle{Exp-1.2: Average errors}.
Given a set of trajectories $\{\dddot{\mathcal{T}_1}, \ldots, \dddot{\mathcal{T}_M}\}$ and their piecewise line representations
$\{\overline{\mathcal{T}_1}, \ldots, \overline{\mathcal{T}_M}\}$, and $P_{j,i}$ denoting
a point in trajectory $\dddot{\mathcal{T}}_j$ contained in a line segment $\mathcal{L}_{l,i}\in\overline{\mathcal{T}_l}$ ($l\in[1,M]$),
then the average error is $\sum_{j=1}^{M}\sum_{i=0}^{M} d(P_{j,i},
\mathcal{L}_{l,i})/\sum_{j=1}^{M}{|\dddot{\mathcal{T}}_j |}$.

To compare the average errors of these algorithms and evaluate the impacts of $\epsilon$ on them, we varied $\epsilon$ from $5m$ to $100m$ on the entire four datasets, respectively.
The results are reported in Figure~\ref{fig:ae-ped} and Figure~\ref{fig:ae-sed}.


\sstab (1) When increasing $\epsilon$, the average errors increase linearly. 

\sstab (2) Datasets have few impacts on the average errors.

\sstab (3) When using \ped, in most cases, the average errors from the smallest
to the largest are batch and online algorithms \tpa, \dpa and \bqsa, one-pass
algorithms \siped and \operb, and the optimal algorithm \optp. Algorithms \tpa,
\dpa and \bqsa are comparable, and they are on average ($58.60\%$, $61.34\%$,
$57.57\%$, $61.34\%$), ($57.56\%$, $62.66\%$, $60.23\%$, $62.66\%$),
\textcolor{red}{($57.56\%$, $62.66\%$, $60.23\%$, $62.66\%$)}
of the optimal algorithm \optp on datasets (\ucar, \geolife, \mopsi, \act), respectively.
Algorithms \siped and \operb are comparable, and they are on average
($80.88\%$, $79.12\%$, $79.33\%$, $79.12\%$), ($70.64\%$, $76.64\%$, $78.71\%$,
$76.64\%$) of the optimal algorithm \optp on datasets (\ucar, \geolife, \mopsi, \act), respectively.
For example, in \mopsi, the average errors of algorithms
(\optp, \tpa, \dpa, \bqsa, \siped, \operb ) are ($16.08$, $9.19$, $9.68$, \textcolor{red}{0.0} $12.96$, $12.77$) when $\epsilon$ = $40m$.

\sstab (4) When using \sed, in most cases, the average errors from the smallest
to the largest are online algorithm \squishe, batch algorithms \tpa and \dpa,
one-pass algorithm \cised, and the near optimal algorithm \nopts.
Algorithms \tpa and \dpa are comparable, and they are on average
($64.62\%$, $64.06\%$, $62.53\%$, $64.06\%$), ($65.03\%$, $63.72\%$, $62.54\%$,
$63.72\%$) of the near optimal algorithm \nopts on datasets (\ucar, \geolife,
\mopsi, \act), respectively.
Algorithms \cised and \squishe are on average ($74.06\%$, $75.77\%$, $75.35\%$,
$75.77\%$), ($41.75\%$, $38.50\%$, $35.19\%$, $38.50\%$) of the near optimal
algorithm \nopts on datasets (\ucar, \geolife, \mopsi, \act), respectively.
For example, in \mopsi, the average errors of algorithms
(\nopts, \tpa, \dpa, \squishe, \cised) are ($19.40\%$, $12.17\%$, $12.20\%$, $6.76\%$, $14.71\%$) when $\epsilon$ = $40m$.

\sstab (5) The average errors of algorithms using \sed are a bit larger than using \ped. 



%%%%%%%%%%%%%%%%%%%%%%%%%%%%%%%%%%%%%%%%%%%%%%%%%%%%%%%%%%%%%%%%%%%%%%%%%%%%%%
\subsubsection{Evaluation of Compression Efficiency}
%%%%%%%%%%%%%%%%%%%%%%%%%%%%%%%%%%%%%%%%%%%%%%%%%%%%%%%%%%%%%%%%%%%%%%%%%%%%%%

In the second set of tests, we compare the efficiency (execution time) of these algorithms.
For fairness, we load and compress trajectories one by one, and only count the running time of the compressing process.
%For a small size trajectory, we repeat compress it tens of times and accumulation the total running time so as  to get the average compression time.


\stitle{Exp-2.1: Impacts of the error bound $\epsilon$}.
To evaluate the impacts of $\epsilon$ on efficiency of these algorithms, we varied $\epsilon$ from $10m$ to $100m$ on the entire four datasets, respectively.
The results, except the optimal algorithms, are reported in Figure~\ref{fig:time-epsilon-ped} and Figure~\ref{fig:time-epsilon-sed}.

\sstab (1) All sub-optimal algorithms, especially one-pass algorithms, are not very sensitive to error bound $\epsilon$. 
More specifically, the running time of algorithms \dpa and \tpa decreases and increases with the increase of $\epsilon$ due to the top-down and bottom-up approaches they applying, respectively. It also shows that \dpa is more suitable than \tpa in circumstances that compression ratios are less than $\sim 10\%$.

\sstab (2) When using \ped, the running time from the smallest to the largest are one-pass algorithms \siped and \operb, and batch and online algorithms \tpa, \dpa and \bqsa. 
Algorithms \siped and \operb are comparable.
Algorithms \tpa, \dpa and \bqsa are comparable, and they are on average
($19.90$, $28.30$, $30.35$, $28.30$), ($18.59$, $17.21$, $16.49$, $17.21$),\textcolor{red}{0.0}
times slower than the one-pass algorithms \siped and \operb on datasets (\ucar, \geolife, \mopsi, \act), respectively. 
For example, in \mopsi, the running time of algorithms
(\tpa, \dpa, \bqsa, \siped, \operb ) are ($232.9$, $124.2$, \textcolor{red}{0.0}, $7.6$, $8.6$) seconds when $\epsilon$ = $40m$.

\sstab (3) When using \sed, the running time from the smallest to the largest are one-pass algorithm \cised, online algorithm \squishe, and batch algorithms \tpa and \dpa. 
Algorithms \squishe, \tpa and \dpa are on average
($8.58$, $13.33$, $15.81$, $13.33$), ($12.81$, $12.93$, $10.64$, $12.93$) and
($2.63$, $2.75$, $2.78$, $2.75$) times slower than \cised on datasets (\ucar, \geolife, \mopsi, \act), respectively.
For example, in \mopsi, the running time of algorithms
(\tpa, \dpa, \squishe, \cised) are  ($156.6$, $104.8$, $27.2$, $9.7$) seconds when $\epsilon$ = $40m$.

\sstab (4) Batch algorithms \dpa and \tpa using \sed run a bit faster than using
\ped, while the one-pass algorithm \cised run $1.3$ times slower than \siped and \operb.


\begin{figure*}[tb!]
	\centering
	\includegraphics[scale=0.315]{Figures/Exp-PED-time-epsilon-service.png}	\hspace{1ex}
	\includegraphics[scale=0.315]{Figures/Exp-PED-time-epsilon-geolife.png}	\hspace{1ex}
	\includegraphics[scale=0.315]{Figures/Exp-PED-time-epsilon-mopsi.png}	\hspace{1ex}
	\includegraphics[scale=0.315]{Figures/Exp-PED-time-epsilon-geolife.png}	\hspace{1ex}
	\vspace{-2.5ex}
	\caption{\small Evaluation of efficiency (\ped): varying the error bound $\epsilon$.}\label{fig:time-epsilon-ped}
	\vspace{-2ex}
\end{figure*}

\begin{figure*}[tb!]
	\centering
	\includegraphics[scale=0.315]{Figures/Exp-SED-time-epsilon-service.png}	\hspace{1ex}
	\includegraphics[scale=0.315]{Figures/Exp-SED-time-epsilon-geolife.png}	\hspace{1ex}
	\includegraphics[scale=0.315]{Figures/Exp-SED-time-epsilon-mopsi.png}	\hspace{1ex}
	\includegraphics[scale=0.315]{Figures/Exp-SED-time-epsilon-geolife.png}	\hspace{1ex}
	\vspace{-2.5ex}
	\caption{\small Evaluation of efficiency (\sed): varying the error bound $\epsilon$.}\label{fig:time-epsilon-sed}
	\vspace{-2ex}
\end{figure*}

\begin{figure*}[tb!]
	\centering
	\includegraphics[scale=0.315]{Figures/Exp-PED-time-size-service.png}	\hspace{1ex}
	\includegraphics[scale=0.315]{Figures/Exp-PED-time-size-geolife.png}	\hspace{1ex}
	\includegraphics[scale=0.315]{Figures/Exp-PED-time-size-mopsi.png}	\hspace{1ex}
	\includegraphics[scale=0.315]{Figures/Exp-PED-time-size-geolife.png}	\hspace{1ex}
	\vspace{-2.5ex}
	\caption{\small Evaluation of efficiency (\ped): varying the size of trajectories.}\label{fig:time-size-ped}
	\vspace{-2ex}
\end{figure*}

\begin{figure*}[tb!]
	\centering
	\includegraphics[scale=0.315]{Figures/Exp-SED-time-size-service.png}	\hspace{1ex}
	\includegraphics[scale=0.315]{Figures/Exp-SED-time-size-geolife.png}	\hspace{1ex}
	\includegraphics[scale=0.315]{Figures/Exp-SED-time-size-mopsi.png}	\hspace{1ex}
	\includegraphics[scale=0.315]{Figures/Exp-SED-time-size-geolife.png}	\hspace{1ex}
	\vspace{-2.5ex}
	\caption{\small Evaluation of efficiency (\sed): varying the size of trajectories.}\label{fig:time-size-sed}
	\vspace{-3ex}
\end{figure*}




\stitle{{Exp-2.2}: Impacts of the sizes of trajectories}.
To evaluate the impacts of the number of data points in a trajectory (\ie the size of a trajectory),
we chose $10$ trajectories from each dataset \ucar, \geolife, \mopsi and \act, respectively,
and varied the size \trajec{|T|} of trajectories from $1,000$ points to $10,000$ points, while fixed error bound $\epsilon = 40$ meters.
The results are reported in Figure~\ref{fig:time-size-ped} and Figure~\ref{fig:time-size-sed}.

\sstab(1) One-pass algorithms \siped, \operb and \cised scale well with the increase of trajectory size \eat{on all datasets},
and show a linear running time, while batch and online algorithms do not.
This is consistent with their time complexity analyses.

\sstab(2) The running time from the smallest to the largest is the same as test {Exp-2.1}.

%\sstab (2) When using \ped, the running time from the smallest to the largest are one-pass algorithms \siped and \operb, and batch and online algorithms \tpa, \dpa and \bqsa. Algorithms \siped and \operb are comparable. Algorithms \tpa, \dpa and \bqsa are comparable, and they are on average \textcolor{red}{($3.8$--$5.3$, $3.5$--$4.8$, $4.6$--$7.2$, $6.2$--$8.4$)} times slower than the one-pass algorithms \siped and \operb on datasets (\ucar, \geolife, \mopsi, \act), respectively. 

%\sstab (3) When using \sed, the running time from the smallest to the largest are one-pass algorithm \cised, online algorithm \squishe, and batch algorithms \tpa and \dpa.  Algorithms \squishe, \tpa and \dpa are on average \textcolor{red}{($9.6$--$17.6$, $8.8$--$15.4$, $8.4$--$16.3$, $9.0$--$14.4$)}, \textcolor{red}{($9.6$--$17.6$, $8.8$--$15.4$, $8.4$--$16.3$, $9.0$--$14.4$)} and \textcolor{red}{($9.6$--$17.6$, $8.8$--$15.4$, $8.4$--$16.3$, $9.0$--$14.4$)} times slower than \cised on datasets (\ucar, \geolife, \mopsi, \act), respectively.

%\sstab (4) Batch algorithms \dpa and \tpa using \sed run a bit faster than using \ped, while the one-pass algorithm \cised run \textcolor{red}{$2.0$--$3.0$} times slower than \siped and \operb.

%%%%%%%%%%%%%%%%%%%%%%%%%%%%%%%%%%%%%%%%%%%%%%%%%%%%%%%%%%%%%%%%%%%%%%%%%%%%%%
\stitle{Summary}.
%%%%%%%%%%%%%%%%%%%%%%%%%%%%%%%%%%%%%%%%%%%%%%%%%%%%%%%%%%%%%%%%%%%%%%%%%%%%%%
From these tests we find the following.

\emph{\sstab{(1) Compression ratios}}. 
(a) When using \ped, the output data sizes of sub-optimal algorithms (\tpa,
\dpa, \bqsa, \siped, \operb) are on average ($131.1\%$, $136.5\%$,
\textcolor{red}{0.0}, $146.6\%$, $147.9\%$)
of the optimal algorithm \optp, respectively.
%the compression ratios from the best to the worst are the optimal algorithm \optp, batch and online algorithms \tpa, \dpa and \bqsa, and one-pass algorithms \siped and \operb. 
(b) When using \sed, the output data sizes of sub-optimal algorithms (\tpa,
\dpa, \squishe, \cised) are on average ($127.91\%$, $126.91\%$, $189.21\%$, $139.43\%$) of the near optimal algorithm \nopts, respectively.
%the compression ratios from the best to the worst are the near optimal algorithm \nopts, batch algorithms \tpa and \dpa, one-pass algorithm \cised, and online algorithm \squishe.  
(c) The compression ratios of algorithms using \ped are obviously better than using \sed. The output data sizes of algorithms (\optp, \tpa, \dpa) using \ped are on average \textcolor{red}{($5\%,5\%,5\%$)} of algorithms (\nopts, \tpa, \dpa) using \sed, respectively.

\emph{\sstab{(2) Average errors}}. 
(a) When using \ped, the average errors from the smallest to the largest are batch and online algorithms \tpa, \dpa and \bqsa, one-pass algorithms \siped and \operb, and the optimal algorithm \optp. 
(b) When using \sed, the average errors from the smallest to the largest are online algorithm \squishe, batch algorithms \tpa and \dpa, one-pass algorithm \cised, and the near optimal algorithm \nopts.
(c) The average errors of algorithms using \sed are a bit larger than using \ped. 

\emph{\sstab{(3) Efficiency}}.
(a) When using \ped, algorithms (\tpa, \dpa, \bqsa) are on average ($24.74$,
$16.54$, \textcolor{red}{$5$)}) times slower than the one-pass algorithms \siped and \operb, respectively. 
%the running time from the smallest to the largest are one-pass algorithms \siped and \operb, and batch and online algorithms \tpa, \dpa and \bqsa. 
(b) When using \sed, algorithms (\tpa, \dpa, \squishe) are on average ($12.57$, $12.13$, $2.72$) times slower than \cised, respectively.
%the running time from the smallest to the largest are one-pass algorithm \cised, online algorithm \squishe, and batch algorithms \tpa and \dpa. 
(c) Batch algorithms \dpa and \tpa using \sed run a bit faster than using \ped, while the one-pass algorithm \cised runs \textcolor{red}{$1.30$} times slower than \siped and \operb.


%%********************************* The End **********************************


