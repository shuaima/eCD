%%%%%%%%%%%%%%%%%%%%%%%%%%%%%%%%%%%%%%%%%%%%%%%%%%%%%%%%%%%%%%%%%%%%%%%%%%%%%%
%\vspace{-1ex}
\section{Evaluation} %Experimental Study
\label{sec-exp}
%%%%%%%%%%%%%%%%%%%%%%%%%%%%%%%%%%%%%%%%%%%%%%%%%%%%%%%%%%%%%%%%%%%%%%%%%%%%%%
In this section, we present an extensive experimental study of the nine \lsa algorithms listed in \mytable{tab:summary-lsa} on trajectory data sets.
Using four real-life datasets, we conducted three sets of experiments to evaluate the compression ratios, errors and execution time of these algorithms, and the impacts of distance metrics,\ie \ped, \sed and \dad, to them.

%(1) the effectiveness of these algorithms, and
%(2) the efficiency of them.
%{An additional test of the effectiveness of near optimal algorithm (\nopts) is presented in Appendix B.}

\vspace{-1ex}
\subsection{Experimental Setting}
%We first introduce the settings of our experimental study.

\stitle{Real-life Trajectory Datasets}.
We use four varied real-life datasets shown in \mytable{tab:datasets}, namely, taxi trajectory data (\taxi) collected by a Beijing taxi company, Service car trajectory data (\ucar) collected by a Chinese car rental company, Geolife trajectory data (\geolife) collected in GeoLife project~\cite{Web:Geolife} and \mopsi trajectory data (\mopsi) collected in Mopsi project \cite{Web:Mopsi}, to evaluate those \lsa algorithms. These data sets have varied sampling rates, ranging from one point per minute to one point per second.
They also come from different sources, where \taxi and \ucar are collected by cars in urban, and \geolife and \mopsi are a mixing of cars and individuals. The data source and sampling rate also affect the performance of \lsa algorithms using certain distance metrics.

\begin{table}
	\vspace{-1ex}
	\caption{\small Real-life trajectory datasets}
	\centering
	\small
	\begin{tabular}{|l|c|c|c|r|}
		\hline
		\kw{Data}& \kw{Number\ of}     &\kw{Sampling}   &\kw{Points~Per}    &\kw{Total} \\
		\kw{Sets} & \kw{Trajectories}   &\kw{Rates (s)}  &\kw{Trajectory (K)}&\kw{points}\\	\hline
		\taxi	&{500}	    &60	        &{$\sim42.8$}      &{21.4M} \\	\hline
		%\truck	&10,368	    &1-60	    &$\sim71.9$     &746M \\	\hline
		%\ucar	&11,000	    &3-5	    &$\sim119.1$   &1.31G\\		\hline
		%
		\ucar	&200	    &3-5	&$\sim114.0$   &22.8M 	\\	\hline
		\geolife\cite{Web:Geolife} &182	    &1-5	&$\sim131.4$   &24.2M	\\	\hline
		\mopsi\cite{Web:Mopsi}	&51	    	&2	    &$\sim153.9$   &7.9M	\\	\hline
		%\act	& 10	    &1	    &$\sim11.8$    &112.8K	\\	\hline
	\end{tabular}
	\label{tab:datasets}
	\vspace{-3ex}
\end{table}


% \ni \emph{(1) Truck trajectory data} (\truck) is the GPS trajectories collected by \eat{10,368} trucks equipped with GPS sensors in China
% during a period from Mar. 2015 to Oct. 2015. The sampling rate varied from 1s to 60s.
%Trajectories mostly have around $50$ to $90$ thousand data points.

%\vspace{0.5ex}
%\ni \emph{(1) Service car trajectory data} (\ucar) is the GPS trajectories collected by a Chinese car rental company during Apr. 2015 to Nov. 2015. The sampling rate was one point per $3$--$5$ seconds, and each trajectory has around $114.1K$ points.
%.We randomly chose $1,000$ cars from them

%\vspace{0.5ex}
%\ni \emph{(2) GeoLife trajectory data} (\geolife) is the GPS trajectories collected in GeoLife project~\cite{Web:Geolife} by 182 users in a period from Apr. 2007 to Oct. 2011. These trajectories have a variety of sampling rates, among which 91\% are logged in each 1-5 seconds per point. %or each 5-10 meters
%This dataset contains 182 trajectories, one trajectory for each user, with a total distance of about 1.2 million kilometers.
%The longest trajectory has 2,156,994 points.

%\vspace{0.5ex}
%\ni \emph{(3) Mopsi trajectory data} (\mopsi) is the GPS trajectories collected in Mopsi project~\cite{Web:Mopsi} by 51 users in a period from 2008 to 2014. Most routes are in Joensuu region, Finland. The sampling rate was one point per $2$ seconds, and each trajectory has around $153.9K$ points.
%exist on every continent.

%\vspace{0.5ex}
%\ni \emph{(4) Act trajectory data} (\act) is a small set GPS trajectories collected with a high sampling rate of one point per second by our team members in 2017. There are 10 trajectories and each trajectory has around 11.8K points.

%The details of these datasets are shown in Table~\ref{tab:dataset}.

\stitle{Algorithms and implementation}.
We implement the algorithms listed in \mytable{tab:summary-lsa}.
For algorithm \cised, we fixed parameter $m=16$ as evaluated in \cite{Lin:Cised}, \ie 16-edges inscribe regular polygon.
% For algorithm \nopts, we fixed parameter $m=32$.
All algorithms were implemented with Java.
All tests were run on an x64-based  PC with 4 Intel(R) Core(TM) i7-6700 CPU @
3.40GHz  and 8GB of memory, and {the max heap size of Java VM is 4GB.}
%, and each test was repeated over 3 times and the average is reported here.


\eat{%%%%%%%%%%%%%%%%%%%
	\stitle{Real-life Trajectory Datasets}.
	We use four real-life datasets shown in Table~\ref{tab:dataset} to test our solutions.
	
	\sstab{\bf(1) Taxi trajectory data}, referred to as \taxi, is the GPS trajectories collected by $12,727$ taxies equipped with GPS sensors in Beijing during a period
	from Nov. 1, 2010 to Nov. 30, 2010. The sampling rate was one point  per 60s, and \taxi has $39,100$ data points on average per trajectory.
	
	\sstab{\bf(2) Truck trajectory data}, referred to as \truck, is the GPS trajectories collected by 10,368 trucks equipped with GPS sensors in China
	during a period from Mar. 2015 to Oct. 2015. The sampling rate varied from 1s to 60s. Trajectories mostly have around $50$ to $90$ thousand data points.
	
	\sstab{\bf(3) Service car trajectory data}, referred to as \ucar,  is the GPS trajectories collected by a car rental company.
	We chose $11,000$ cars from them, during Apr. 2015 to Nov. 2015. The sampling rate was one point per $3$--$5$ seconds, and
	each trajectory has around $119.1K$ data points.
	
	{\sstab{\bf(4) GeoLife trajectory data}, referred to as \geolife, is the GPS trajectories collected in GeoLife project~\cite{Zheng:GeoLife} by 182 users in a period from Apr. 2007 to Oct. 2011. These trajectories have a variety of sampling rates, among which 91\% are logged in each 1-5 seconds or each 5-10 meters per point. The longest trajectory has 2,156,994 points.}
	%This dataset contains 182 trajectories, one trajectory for each user, with a total distance of about 1.2 million kilometers.
}%%%%%%%%%%%%%%%%%%%%%%%%%%%%%%%%%%%%%%%%%%%%%%%%%%%%%%%%%%%%%



\subsection{Evaluation Metrics}
Compression ratios, errors and running time are the most popular metrics to evaluate trajectory simplification algorithms.

\stitle{Compression ratios.}
Given a set of trajectories $\{\dddot{\mathcal{T}_1}, \ldots, \dddot{\mathcal{T}_M}\}$ and their piecewise line representations $\{\overline{\mathcal{T}_1}, \ldots, \overline{\mathcal{T}_M}\}$,
the compression ratio is $(\sum_{j=1}^{M} |\overline{\mathcal{T}}_j |)/(\sum_{j=1}^{M} |\dddot{\mathcal{T}}_j |)$.
By this definition, algorithms with lower compression ratios are better.

\stitle{Errors.}
All these algorithms in \mytable{tab:summary-lsa} are error bounded, \ie the max errors are approximate the values of error bounds. Hence, we only evaluate the average errors here. Given a set of trajectories $\{\dddot{\mathcal{T}_1}, \ldots, \dddot{\mathcal{T}_M}\}$ and their piecewise line representations
$\{\overline{\mathcal{T}_1}, \ldots, \overline{\mathcal{T}_M}\}$, and $P_{j,i}$ denoting
a point in trajectory $\dddot{\mathcal{T}}_j$ contained in a line segment $\mathcal{L}_{l,i}\in\overline{\mathcal{T}_l}$ ($l\in[1,M]$),
then the average error is $\sum_{j=1}^{M}\sum_{i=0}^{M} d(P_{j,i},
\mathcal{L}_{l,i})/\sum_{j=1}^{M}{|\dddot{\mathcal{T}}_j |}$.

\stitle{Running time.}
It is the time of compressing trajectories.
%We load and compress trajectories one by one, and only count the running time of the compressing process.

%%%%%%%%%%%%%%%%%%%%%%%%%%%%%%%%%%%%%%%%%%%%%%%%%%%%%%%%%%%%%%%%%%%%%%%%%%%%%%
\subsection{Experimental Results}
%%%%%%%%%%%%%%%%%%%%%%%%%%%%%%%%%%%%%%%%%%%%%%%%%%%%%%%%%%%%%%%%%%%%%%%%%%%%%%
We next present our findings.


\begin{figure*}[tb!]
	\centering
	\includegraphics[scale=0.315]{Figures/Exp-PED-CR-epsilon-taxi.png}\hspace{1ex}
	\includegraphics[scale=0.315]{Figures/Exp-PED-CR-epsilon-service.png} 	\hspace{1ex}
	\includegraphics[scale=0.315]{Figures/Exp-PED-CR-epsilon-geolife.png}	\hspace{1ex}
	\includegraphics[scale=0.315]{Figures/Exp-PED-CR-epsilon-mopsi.png}		
	\vspace{-3ex}
	\caption{\small Evaluation of compression ratios (\ped): varying the error bound $\epsilon$.}
	\label{fig:cr-ped-epsilon}
	\vspace{-2ex}
\end{figure*}

\begin{figure*}[tb!]
	\centering
	\includegraphics[scale=0.315]{Figures/Exp-SED-CR-epsilon-taxi.png}\hspace{1ex}
	\includegraphics[scale=0.315]{Figures/Exp-SED-CR-epsilon-service.png} 	\hspace{1ex}
	\includegraphics[scale=0.315]{Figures/Exp-SED-CR-epsilon-geolife.png}	\hspace{1ex}
	\includegraphics[scale=0.315]{Figures/Exp-SED-CR-epsilon-mopsi.png}		
	\vspace{-3ex}
	\caption{\small Evaluation of compression ratios (\sed): varying the error bound $\epsilon$.}
	\label{fig:cr-sed-epsilon}
	\vspace{-2ex}
\end{figure*}

\begin{figure*}[tb!]
	\centering
	\includegraphics[scale=0.315]{Figures/Exp-DAD-CR-epsilon-taxi.png}\hspace{1ex}
	\includegraphics[scale=0.315]{Figures/Exp-DAD-CR-epsilon-service.png} 	\hspace{1ex}
	\includegraphics[scale=0.315]{Figures/Exp-DAD-CR-epsilon-geolife.png}	\hspace{1ex}
	\includegraphics[scale=0.315]{Figures/Exp-DAD-CR-epsilon-mopsi.png}		
	\vspace{-3ex}
	\caption{\small Evaluation of compression ratios (\dad): varying the error bound $\epsilon$.}
	\label{fig:cr-dad-epsilon}
	\vspace{-2ex}
\end{figure*}

\begin{figure*}[tb!]
	\centering
	\includegraphics[scale=0.315]{Figures/Exp-PED-CR-size-taxi.png}\hspace{1ex}
	\includegraphics[scale=0.315]{Figures/Exp-PED-CR-size-service.png} 	\hspace{1ex}
	\includegraphics[scale=0.315]{Figures/Exp-PED-CR-size-geolife.png}	\hspace{1ex}
	\includegraphics[scale=0.315]{Figures/Exp-PED-CR-size-mopsi.png}		
	\vspace{-3ex}
	\caption{\small Evaluation of compression ratios (\ped): varying the size of
    trajectories.}
  \label{fig:cr-ped-size}
	\vspace{-2ex}
\end{figure*}

\begin{figure*}[tb!]
	\centering
	\includegraphics[scale=0.315]{Figures/Exp-SED-CR-size-taxi.png}\hspace{1ex}
	\includegraphics[scale=0.315]{Figures/Exp-SED-CR-size-service.png} 	\hspace{1ex}
	\includegraphics[scale=0.315]{Figures/Exp-SED-CR-size-geolife.png}	\hspace{1ex}
	\includegraphics[scale=0.315]{Figures/Exp-SED-CR-size-mopsi.png}		
	\vspace{-3ex}

	\caption{\small Evaluation of compression ratios (\sed): varying the size of
    trajectories.}
  \label{fig:cr-sed-size}
	\vspace{-2ex}
\end{figure*}

\begin{figure*}[tb!]
	\centering
	\includegraphics[scale=0.315]{Figures/Exp-DAD-CR-size-taxi.png}\hspace{1ex}
	\includegraphics[scale=0.315]{Figures/Exp-DAD-CR-size-service.png} 	\hspace{1ex}
	\includegraphics[scale=0.315]{Figures/Exp-DAD-CR-size-geolife.png}	\hspace{1ex}
	\includegraphics[scale=0.315]{Figures/Exp-DAD-CR-size-mopsi.png}		
	\vspace{-3ex}
	\caption{\small Evaluation of compression ratios (\dad): varying the size of trajectories.}
	\label{fig:cr-dad-size}
	\vspace{-3ex}
\end{figure*}



%%%%%%%%%%%%%%%%%%%%%%%%%%%%%%%%%%%%%%%%%%%%%%%%%%%%%%%%%%%%%%%%%%%%%%%%%%%%%
\vspace{-1ex}
\subsubsection{Evaluation and Analysis of Compression Ratios}
%%%%%%%%%%%%%%%%%%%%%%%%%%%%%%%%%%%%%%%%%%%%%%%%%%%%%%%%%%%%%%%%%%%%%%%%

We first test the compression ratios of these algorithms under varied error bounds $\epsilon$ and trajectory sizes, respectively. We varied $\epsilon$ from $10m$ to $100m$ in \ped and \sed (or from $15^o$ to $90^o$ in \dad) on the entire four datasets, respectively. The results are reported in Figure~\ref{fig:cr-ped-epsilon}, Figure~\ref{fig:cr-sed-epsilon} and Figure~\ref{fig:cr-dad-epsilon} ({Note that the naive optimal algorithm using \sed or \dad is not reported here because it can not run with the full dataset as input}).
%
We then chose $10$ trajectories from each dataset \taxi, \ucar, \geolife and \mopsi, respectively, and varied the size \trajec{|T|} of a trajectory from $1,000$ points to $10,000$ points, while fixed error bound $\epsilon = 60$ meters for \ped and \sed ({or $\epsilon = 45$ degrees for \dad}).
The experimental results are reported in Figure~\ref{fig:cr-ped-size}, Figure~\ref{fig:cr-sed-size} and Figure~\ref{fig:cr-dad-size}. 

%%%%%%%%%%%%%%%%%
(1) The compression ratios of algorithms using \ped from the best
to the worst are the optimal algorithm \opt, online algorithm \bqsa, batch algorithms \tpa and
\dpa, and one-pass algorithms \siped and \operb.
The output sizes of algorithm \bqsa are on average
($103.58\%$, $113.32\%$, $120.22\%$, $120.83\%$) of the optimal algorithm \opt
on datasets \dSets, respectively.
Algorithms \tpa and \dpa are comparable, and their output sizes are on average
($103.17\%$, $125.05\%$, $131.01\%$, $138.01\%$) and ($106.98\%$, $130.03\%$, $140.56\%$, $139.00\%$) of \opt
on datasets \dSets, respectively.
Algorithms \siped and \operb are comparable, and they are on average
($113.09\%$, $136.73\%$, $150.23\%$, $152.29\%$)
and ($119.89\%$, $143.14\%$, $147.80\%$, $152.37\%$) of \opt on datasets \dSets, respectively.
%
For example, in \mopsi, the compression ratios of algorithms
(\opt, \tpa, \dpa, \bqsa, \siped, \operb ) are ($1.6\%$, $2.2\%$, $2.2\%$, $1.9\%$, $2.4\%$, $2.4\%$) when $\epsilon$ = $40m$.
%

%%%%%%%%%%%%%%%%%
(2) The compression ratios of algorithms using \sed from the best
to the worst are the optimal algorithm \opt, batch algorithms \tpa and
\dpa, one-pass algorithm \cised, and online algorithm \squishe.
%
{Algorithms \tpa and \dpa are comparable, and they are on average
($102.72\%$, $125.23\%$, $143.92\%$, $128.63\%$) and ($103.18\%$, $123.93\%$, $141.46\%$, $121.14\%$)
 of the optimal algorithm \opt on datasets \dSets, respectively.}
%
{Algorithms \cised and \squishe are on average ($108.00\%$,
  $134.35\%$, $159.30\%$, $136.06\%$) and ($110.27\%$, $165.94\%$, $225.68\%$, $206.90\%$)
 of \opt on datasets \dSets, respectively.}
%
For example, in \mopsi, the compression ratios of algorithms
(\tpa, \dpa, \squishe, \cised)
are ($3.45\%$, $3.41\%$, $5.75\%$, $3.86\%$), respectively, when $\epsilon$ = $40m$.
%
%Algorithms \tpa and \dpa are comparable, and they are on average ($122.69\%$, $129.08\%$, $131.97\%$, $131.01\%$) and ($121.36\%$, $129.27\%$, $130.11\%$, $126.21\%$) of the near optimal algorithm \nopts on datasets \dSets, respectively.
%Algorithms \cised and \squishe are on average ($132.07\%$, $139.67\%$, $146.56\%$, $135.10\%$) and ($164.47\%$, $189.87\%$, $213.30\%$, $186.72\%$) of \nopts on datasets \dSets, respectively.
%
%For example, in \mopsi, the compression ratios of algorithms (\nopts, \tpa, \dpa, \squishe, \cised) are ($2.62\%$, $3.45\%$, $3.41\%$, $5.75\%$, $3.86\%$), respectively, when $\epsilon$ = $40m$.
%

%%%%%%%%%%%%%%%%%
(3) The compression ratios of algorithms using \dad from the best
to the worst are the optimal algorithm \opt, batch algorithm \tpa,
one-pass algorithm \interval and batch algorithm \dpa.
%
{Algorithms \tpa and \interval are comparable, and they are on average
($100.81\%$, $102.91\%$, $102.27\%$, $106.88\%$) and ($102.38\%$, $101.98\%$, $103.52\%$, $103.43\%$)
 of the optimal algorithm \opt on datasets \dSets, respectively.}
%
{Algorithm \dpa is on average ($133.19\%$, $283.93\%$, $143.79\%$, $278.89\%$)
 of the optimal algorithm \opt on datasets \dSets, respectively.}
%
For example, in \mopsi, the compression ratios of algorithms (\tpa, \dpa, \interval)
are ($13.3\%$, $23.1\%$, $13.7\%$), respectively, when $\epsilon$ = $45$ degrees.
%


We then discuss our findings from the views of \lsa algorithms and distance metrics.

\stitle{Analysis of \lsa algorithms}. The optimal algorithm is the best algorithm in term of compression ratio, and one pass algorithms using full $\epsilon$ are the first class among all sub optimal algorithms.

%\todo{Batch: Bottom-up vs Top-down.}
For batch algorithms, Bottom-up algorithm (\tpa) and Top-down algorithm (\dpa) have the similar compression ratios when using either \ped or \sed. However, when using \dad, Bottom-up method has obviously better compression ratios than Top-down method. We note that the experimental results of \cite{Long:Direction} also show this phenomenon (but without discuss). As we know the Top-down algorithm will split a long trajectory $[P_s, ..., P_e]$ into two sub trajectories by finding out a splitting point $P_i (s<i<e)$ who has the max position (or whose line segment $\vv{P_{i-1}P_{i}}$ has the max direction) deviation to line segment $\vv{P_sP_{e}}$. Through this strategy works well with \ped and \sed, a point with the max direction deviation may not be a reasonable splitting point in the direction-aware scenario. Thus it leads to a poorer compression ratio. The Bottom-up method does not have this weakness as it always merges neighbouring points.



%\todo{Online: }
For online algorithms, \bqsa and \opwa are comparable with the best sub-optimal algorithms. This is because \opwa is indeed a combination of \dpa and opening window, and \bqsa is mainly an efficiency optimization of the \opwa.
\squishe has the poorest compression ratio among all algorithms using \sed. This is the result of its nature: \squishe estimates the lowest \sed error and removes the point with is ``predicted to introduce the lowest amount of error into the compression"\cite{Muckell:SQUISH}. Its ``prediction" method is not accurate enough, thus, in order to ensure the error bound, it may ignore too many potential points that should be represented by a line segment.

%\todo{One pass: half $\epsilon$ vs full $\epsilon$.}
For one-pass algorithms, the full sector/cone/range combining with a position/direction constraint always have better compression ratios then the half sector/cone/range versions in all datasets, and they are comparable with the best sub optimal algorithms.
%This is related to the moving habits or patterns of moving objects that engender trajectories.
We can find in our daily lives, a moving object, like an individual or a car, will keep moving forward for quite a long time, engendering a sequence of data points distributing in a narrow strip. Under such circumstance, a new data point is quite possible living in the common intersection of larger sectors/cones/ranges, which leads to a great compression ratio.
%which lets more data points be represented by a result line segment.
\todo{strict proof?}
%combining with a constraint that the current point should live in the common intersection of preview sectors/cones/ranges



\stitle{Analysis of distance metrics}.
Though \ped, \sed and \dad are different distances whose usages are mainly decided by application requirements, we can still compare them from the view of compression ratio, so as to choose an effective distance metric.

First, given the same error bound $\epsilon$, the compression ratios of algorithms using \ped are obviously better
than using \sed. More specifically, \emph{the output sizes of using \sed are approximately twice of \ped.}
%
As shown in Figure~\ref{fig:cr-ped-epsilon} and Figure~\ref{fig:cr-sed-epsilon}, the compression ratios of algorithms \tpa and \dpa
using \ped are on average ($55.08\%$, $43.55\%$, $47.49\%$, $63.15\%$) and ($56.75\%$, $45.79\%$,
$50.88\%$, $64.50\%$) of algorithms \tpa and \dpa using \sed on datasets \dSets, respectively,
%
and in Figure~\ref{fig:cr-ped-size} and Figure~\ref{fig:cr-sed-size}, the compression ratios of algorithms \opt, \tpa and \dpa
using \ped are on average {($56.03\%$, $33.42\%$, $33.12\%$, $72.42\%$),
	($55.78\%$, $31.88\%$, $34.27\%$, $69.44\%$) and ($58.09\%$, $34.82\%$,
	$40.91\%$, $75.06\%$)} of algorithms \opt, \tpa and \dpa using \sed on datasets \dSets, respectively.
%
This result shows ~\sed saves temporal information at a cost of saving more points.


Secondly, in practical (\eg $\epsilon <100$ meters and $\epsilon < 60$ degrees), \sed have obviously better compression ratios than \dad in datasets \geolife and \mopsi, a bit better than \dad in \taxi and a bit poorer than \dad in \ucar.
This is because some \geolife and \mopsi trajectories are collected by individuals that are in transportation modes of walking, running and riding, and moving objects in those modes may change their directions with a considerable range (\eg large than $60$ degrees) more frequently than cars in urban. Moreover, \geolife and \mopsi have higher sampling rates than \taxi and \ucar, which are helpful to capture more direction changes, \ie direction changes in a small time interval.

%It seems that it is a normal phenomenon that a moving object frequently changes its direction with a considerable range (\eg large than $60$ degrees) during a trip.




\eat{

\sstab (1) Trajectory sizes have few impacts on compression ratios.

\sstab (1) When increasing $\epsilon$, the compression ratios decrease.
%For example, in \mopsi, the compression ratios are greater than {$4\%$} when $\epsilon$ = $10m$, and less than {$2\%$} when $\epsilon$ = $100m$.

\sstab (2) Dataset \mopsi usually has the lowest compression ratios, compared with \taxi, \ucar and \geolife, due to its highest sampling rate, \taxi has the highest compression ratios due to its lowest sampling rate, and \ucar and \geolife and  have the compression ratios in the middle accordingly.
}


\begin{figure*}[tb!]
	\centering
	\includegraphics[scale=0.315]{Figures/Exp-PED-error-epsilon-taxi.png} \hspace{1ex}
	\includegraphics[scale=0.315]{Figures/Exp-PED-error-epsilon-service.png}	\hspace{1ex}
	\includegraphics[scale=0.315]{Figures/Exp-PED-error-epsilon-geolife.png}	\hspace{1ex}
	\includegraphics[scale=0.315]{Figures/Exp-PED-error-epsilon-mopsi.png}	
	\vspace{-3ex}
	\caption{\small Evaluation of average errors (\ped): varying the error bound $\epsilon$.}
	\label{fig:ae-ped-epsilon}
	\vspace{-2ex}
\end{figure*}

\begin{figure*}[tb!]
	\centering
	\includegraphics[scale=0.315]{Figures/Exp-SED-error-epsilon-taxi.png} \hspace{1ex}
	\includegraphics[scale=0.315]{Figures/Exp-SED-error-epsilon-service.png}	\hspace{1ex}
	\includegraphics[scale=0.315]{Figures/Exp-SED-error-epsilon-geolife.png}	\hspace{1ex}
	\includegraphics[scale=0.315]{Figures/Exp-SED-error-epsilon-mopsi.png}		
	\vspace{-3ex}
	\caption{\small Evaluation of average errors (\sed): varying the error bound $\epsilon$.}
	\label{fig:ae-sed-epsilon}
	\vspace{-2ex}
\end{figure*}

\begin{figure*}[tb!]
	\centering
	\includegraphics[scale=0.315]{Figures/Exp-DAD-pedAveErr-epsilon-taxi.png} \hspace{1ex}
	\includegraphics[scale=0.315]{Figures/Exp-DAD-pedAveErr-epsilon-service.png}	\hspace{1ex}
	\includegraphics[scale=0.315]{Figures/Exp-DAD-pedAveErr-epsilon-geolife.png}	\hspace{1ex}
	\includegraphics[scale=0.315]{Figures/Exp-DAD-pedAveErr-epsilon-mopsi.png}	
	\vspace{-3ex}
	\caption{\small Evaluation of average errors (\dad): varying the error bound $\epsilon$.}
	\label{fig:ae-dad-ped-epsilon}
	\vspace{-2ex}
\end{figure*}

\begin{figure*}[tb!]
	\centering
	\includegraphics[scale=0.315]{Figures/Exp-PED-error-size-taxi.png}\hspace{1ex}
	\includegraphics[scale=0.315]{Figures/Exp-PED-error-size-service.png} 	\hspace{1ex}
	\includegraphics[scale=0.315]{Figures/Exp-PED-error-size-geolife.png}	\hspace{1ex}
	\includegraphics[scale=0.315]{Figures/Exp-PED-error-size-mopsi.png}		
	\vspace{-3ex}
	\caption{\small Evaluation of average errors (\ped): varying the size of
    trajectories.}
  \label{fig:ae-ped-size}
	\vspace{-2ex}
\end{figure*}

\begin{figure*}[tb!]
	\centering
	\includegraphics[scale=0.315]{Figures/Exp-SED-error-size-taxi.png}\hspace{1ex}
	\includegraphics[scale=0.315]{Figures/Exp-SED-error-size-service.png} 	\hspace{1ex}
	\includegraphics[scale=0.315]{Figures/Exp-SED-error-size-geolife.png}	\hspace{1ex}
	\includegraphics[scale=0.315]{Figures/Exp-SED-error-size-mopsi.png}		
	\vspace{-3ex}
	\caption{\small Evaluation of average errors (\sed): varying the size of
    trajectories.}
  \label{fig:ae-sed-size}
	\vspace{-2ex}
\end{figure*}


\begin{figure*}[tb!]
	\centering
	\includegraphics[scale=0.315]{Figures/Exp-DAD-pedAveErr-size-taxi.png} \hspace{1ex}
	\includegraphics[scale=0.315]{Figures/Exp-DAD-pedAveErr-size-service.png}	\hspace{1ex}
	\includegraphics[scale=0.315]{Figures/Exp-DAD-pedAveErr-size-geolife.png}	\hspace{1ex}
	\includegraphics[scale=0.315]{Figures/Exp-DAD-pedAveErr-size-mopsi.png}	
	\vspace{-3ex}
	\caption{\small Evaluation of average errors (\dad): varying the size of trajectories.}
	\label{fig:ae-dad-ped-size}
	\vspace{-2ex}
\end{figure*}

%%%%%%%%%%%%%%%%%%%%%%%%%%%%%%%%%%%%%%%%%%%%%%%%%%%%%%%%%%%%%%%%%%%%%%%%%%%%%%%
\vspace{-0.5ex}
\subsubsection{Evaluation and Analysis of Average Errors}
%%%%%%%%%%%%%%%%%%%%%%%%%%%%%%%%%%%%%%%%%%%%%%%%%%%%%%%%%%%%%%%%%%%%%%%%%%%%%%
%In the second set of tests, we evaluate the average errors of these algorithms.
We then evaluate the average errors of these algorithms.





\stitle{Exp-2.1: Impacts of the error bound $\epsilon$}.
To compare the average errors of these algorithms and evaluate the impacts of $\epsilon$ on them, we varied $\epsilon$ from $10m$ to $100m$ in \ped and \sed (or from $15^o$ to $90^o$ in \dad) on the entire four datasets, respectively.
For \dad, we computed the average \ped errors of points to their corresponding line segments.
The results are reported in Figure~\ref{fig:ae-ped-epsilon}, Figure~\ref{fig:ae-sed-epsilon} and Figure~\ref{fig:ae-dad-ped-epsilon}.


\sstab (1) When increasing $\epsilon$, the average errors increase linearly.

\sstab (2) Datasets have few impacts on the average errors.

\sstab (3) When using \ped, the average errors from the smallest
to the largest are batch algorithms \tpa and \dpa, one-pass
algorithms \siped and \operb, the optimal algorithm \opt and online algorithm \bqsa.
Algorithms \tpa and
\dpa are comparable, and they are on average ($77.43\%$, $58.69\%$, $61.34\%$,
$57.57\%$) and ($88.12\%$, $57.61\%$, $62.66\%$, $60.23\%$) of the optimal algorithm \opt on datasets \dSets, respectively.
Algorithms \siped and \operb are comparable, and they are on average
($73.08\%$, $80.96\%$, $79.12\%$, $79.33\%$), ($73.03\%$, $70.60\%$, $76.64\%$, $78.71\%$) of \opt on datasets \dSets, respectively.
Algorithm \bqsa is on average ($97.69\%$, $104.67\%$, $108.91\%$, $106.92\%$) of \opt on datasets \dSets, respectively.
%
For example, in \mopsi, the average errors of algorithms
(\opt, \tpa, \dpa, \bqsa, \siped, \operb ) are ($16.08$, $9.19$, $9.68$, $17.4$, $12.96$, $12.77$)  metres when $\epsilon$ = $40m$.

\sstab (4) When using \sed, the average errors from the smallest
to the largest are online algorithm \squishe, batch algorithms \tpa and \dpa,
and one-pass algorithm \cised.
For example, in \mopsi, the average errors of algorithms
(\tpa, \dpa, \squishe, \cised) are ($12.17$, $12.20$, $6.76$, $14.71$) metres, respectively, when $\epsilon$ = $40m$.


\sstab {(5) When using \dad, the average errors from the smallest
to the largest are batch algorithms \dpa and \tpa,
and one-pass algorithm \interval.
For example, in \mopsi, the average errors of algorithms
(\tpa, \dpa, \interval) in terms of \ped are ($65.6$, $75.4$, $68.1$) meters, respectively, when $\epsilon$ = $45$ degrees.
It is also worth pointing out that the max errors of algorithms
(\tpa, \dpa, \interval) using \dad may be very large in terms of \ped, \eg they are ($68247.9$, $66794.4$, $68247.9$) meters, respectively, when $\epsilon$ = $45$ degrees.}

\sstab (6) Given the same error bound $\epsilon$, the average errors of algorithms using \sed are a bit larger than using \ped.
And in practical (\eg $\epsilon <100$ meters and $\epsilon < 60$ degrees), the average errors of algorithms using \dad, in terms of \ped, are obvious larger than algorithms using \ped and \sed.


\stitle{{Exp-2.2}: Impacts of the sizes of trajectories}.
To compare the errors of these algorithms and evaluate the impacts of the sizes of trajectories on them, we chose the same $10$ trajectories from each dataset \taxi, \ucar, \geolife and \mopsi, respectively,
and varied the size \trajec{|T|} of a trajectory from $1,000$ points to $10,000$ points, while fixed error bound $\epsilon = 60$ meters ({or $\epsilon = 45$ degrees}).
The results are reported in Figure~\ref{fig:ae-ped-size}, Figure~\ref{fig:ae-sed-size} and Figure~\ref{fig:ae-dad-ped-size}.

\sstab (1) Trajectory sizes have few impacts on average errors.

\sstab (2) The average errors in this test are consistent with test Exp-2.1.

\sstab (3) {When using \ped,} the average errors from the smallest
to the largest are  batch algorithms \tpa and \dpa,
one-pass algorithms \siped and \operb, the naive optimal algorithm \opt and online algorithm \bqsa.
Algorithms \tpa and \dpa are comparable, and they are on average
{($105.94\%$, $53.16\%$, $66.84\%$, $61.50\%$), ($111.96\%$, $52.04\%$, $78.65\%$, $60.57\%$)} of \opt on datasets \dSets, respectively.
Algorithms \siped and \operb are on average {($85.30\%$, $73.55\%$, $82.01\%$,
  $85.23\%$), ($97.69\%$, $67.20\%$, $82.58\%$, $81.88\%$)}
of \opt on datasets \dSets, respectively.
Algorithm \bqsa is  on average  {($108.51\%$, $102.06\%$, $104.63\%$, $113.80\%$)}
of \opt on datasets \dSets, respectively.




\sstab (4) When using \sed, the average errors from the smallest
to the largest are online algorithm \squishe, batch algorithms \tpa and \dpa,
one-pass algorithm \cised, and the naive optimal algorithm \opt.
Algorithms \tpa and \dpa are comparable, and they are on average
{($87.22\%$, $60.36\%$, $66.11\%$, $62.43\%$), ($91.04\%$, $62.54\%$, $67.04\%$, $68.64\%$)} of \opt on datasets \dSets, respectively.
Algorithms \cised and \squishe are on average {($73.15\%$, $75.29\%$, $76.03\%$, $81.44\%$), ($46.27\%$, $40.61\%$, $38.15\%$, $34.22\%$)} of \opt on datasets \dSets, respectively.


\sstab {(5) When using \dad, the average errors from the smallest
to the largest are batch algorithms \dpa and \tpa, one-pass algorithm \interval, and the optimal algorithm \opt.
Algorithms \tpa and \interval are comparable, and they are on average
{($88.94\%$, $91.35\%$, $61.45\%$, $73.71\%$) and ($93.97\%$, $90.36\%$, $68.23\%$, $163.47\%$)} of algorithm \opt on datasets \dSets, respectively.
Algorithm \dpa is on average ($76.86\%$, $82.45\%$, $96.52\%$, $137.95\%$) of \opt on datasets \dSets, respectively.
}



%%%%%%%%%%%%%%%%%%%%%%%%%%%%%%%%%%%%%%%%%%%%%%%%%%%%%%%%%%%%%%%%%%%%%%%%%%%%%%
\vspace{-1ex}
\subsubsection{Evaluation and Analysis of running time}
%%%%%%%%%%%%%%%%%%%%%%%%%%%%%%%%%%%%%%%%%%%%%%%%%%%%%%%%%%%%%%%%%%%%%%%%%%%%%%

In the third set of tests, we compare the running time of these algorithms.
%For fairness, we load and compress trajectories one by one, and only count the running time of the compressing process.
%For a small size trajectory, we repeat compress it tens of times and accumulation the total running time so as  to get the average compression time.


\stitle{Exp-3.1: Impacts of the error bound $\epsilon$}.
To evaluate the impacts of $\epsilon$ on running time of these algorithms, we varied $\epsilon$ from $10m$ to $100m$ in \ped and \sed (or from $15^o$ to $90^o$ in \dad) on the entire four datasets, respectively.
%
%The results show that the running time of algorithm \optp is on average ($261.71$, $334.49$, $403.56$, $202.82$) times slower than one-pass algorithms \siped and \cised on datasets \dSets, respectively.
%For example, in \mopsi, the running time of algorithms (\opt, \siped, \cised) are  {($3014.2$, $7.5$, $9.8$)} seconds when $\epsilon$ = $40m$.
%
The results show that the running time of algorithm \opt is thousands of times slower than one-pass algorithms on our datasets.
Indeed, it is not easy to show all these algorithms in a figure, thus, only the results of sub-optimal algorithms are reported in Figure~\ref{fig:time-epsilon-ped}, Figure~\ref{fig:time-epsilon-sed} and Figure~\ref{fig:time-epsilon-dad}.

\sstab (1) One-pass algorithms are not very sensitive to error bound $\epsilon$. The running time of batch algorithms \dpa and \tpa decreases and increases with the increase of $\epsilon$, respectively, due to the top-down and bottom-up approaches they applying.
%It also shows that \dpa is more suitable than \tpa in circumstances that compression ratios are less than $\sim 10\%$.

\sstab (2) When using \ped, in most cases, the running time from the smallest to the largest are one-pass algorithms \siped and \operb, batch algorithms \tpa and \dpa, and online algorithm \bqsa.
Algorithms \siped and \operb are comparable, and algorithms \tpa, \dpa and \bqsa are on average
($19.19$, $26.79$, $28.25$, $29.87$), ($17.90$, $16.32$, $15.40$, $11.02$) and ($15.07$, $37.73$, $62.23$, $61.29$)
times slower than the one-pass algorithm \siped on datasets \dSets, respectively.
%
For example, in \mopsi, the running time of algorithms
(\tpa, \dpa, \bqsa, \siped, \operb ) are ($232.9$, $124.2$, $469.4$, $7.6$, $8.6$) seconds when $\epsilon$ = $40m$.

\sstab (3) When using \sed, the running time from the smallest to the largest are one-pass algorithm \cised, online algorithm \squishe, and batch algorithms \tpa and \dpa.
Algorithms \tpa, \dpa and \squishe are on average
($8.58$, $13.33$, $15.81$, $13.09$), ($12.81$, $12.93$, $10.64$, $8.79$) and
($2.63$, $2.75$, $2.78$, $2.57$) times slower than \cised on datasets \dSets, respectively.
%
For example, in \mopsi, the running time of algorithms
(\tpa, \dpa, \squishe, \cised) are  ($156.6$, $104.8$, $27.2$, $9.7$) seconds when $\epsilon$ = $40m$.

\sstab {(4) When using \dad,} the running time from the smallest to the
largest are one-pass algorithm \interval and batch algorithms \tpa and \dpa.
%
Algorithms \tpa and \dpa are on average
($6.66$, $13.55$, $12.73$, $12.73$) and ($11.67$, $14.24$, $16.51$, $17.59$)
times slower than \interval on datasets \dSets, respectively.
%
For example, in \mopsi, the running time of algorithms
(\tpa, \dpa, \interval) are ($105.57$, $152.53$, $8.57$) seconds when
$\epsilon=45$ degrees, respectively.

\sstab (5) For batch algorithms \dpa and \tpa, they run a bit faster using \sed and \dad than using
\ped.

\sstab (6) One pass algorithms \siped, \operb and \interval have the similar running time, and algorithm \cised runs a bit slower than them.






\begin{figure*}[tb!]
	\centering
	\includegraphics[scale=0.315]{Figures/Exp-PED-time-epsilon-taxi.png}	\hspace{1ex}
	\includegraphics[scale=0.315]{Figures/Exp-PED-time-epsilon-service.png}	\hspace{1ex}
	\includegraphics[scale=0.315]{Figures/Exp-PED-time-epsilon-geolife.png}	\hspace{1ex}
	\includegraphics[scale=0.315]{Figures/Exp-PED-time-epsilon-mopsi.png}	\hspace{1ex}
	\vspace{-3ex}
	\caption{\small Evaluation of running time (\ped): varying the error bound $\epsilon$.}\label{fig:time-epsilon-ped}
	\vspace{-2ex}
\end{figure*}

\begin{figure*}[tb!]
	\centering
	\includegraphics[scale=0.315]{Figures/Exp-SED-time-epsilon-taxi.png}	\hspace{1ex}
	\includegraphics[scale=0.315]{Figures/Exp-SED-time-epsilon-service.png}	\hspace{1ex}
	\includegraphics[scale=0.315]{Figures/Exp-SED-time-epsilon-geolife.png}	\hspace{1ex}
	\includegraphics[scale=0.315]{Figures/Exp-SED-time-epsilon-mopsi.png}	\hspace{1ex}
	\vspace{-3ex}
	\caption{\small Evaluation of running time (\sed): varying the error bound $\epsilon$.}\label{fig:time-epsilon-sed}
	\vspace{-2ex}
\end{figure*}

\begin{figure*}[tb!]
	\centering
	\includegraphics[scale=0.315]{Figures/Exp-DAD-time-epsilon-taxi.png}\hspace{1ex}
	\includegraphics[scale=0.315]{Figures/Exp-DAD-time-epsilon-service.png} 	\hspace{1ex}
	\includegraphics[scale=0.315]{Figures/Exp-DAD-time-epsilon-geolife.png}	\hspace{1ex}
	\includegraphics[scale=0.315]{Figures/Exp-DAD-time-epsilon-mopsi.png}		
	\vspace{-3ex}
	\caption{\small Evaluation of running time (\dad): varying the error bound $\epsilon$.}\label{fig:time-epsilon-dad}
	\vspace{-2ex}
\end{figure*}

\begin{figure*}[tb!]
	\centering
	\includegraphics[scale=0.315]{Figures/Exp-PED-time-size-taxi.png}\hspace{1ex}
	\includegraphics[scale=0.315]{Figures/Exp-PED-time-size-service.png}	\hspace{1ex}
	\includegraphics[scale=0.315]{Figures/Exp-PED-time-size-geolife.png}	\hspace{1ex}
	\includegraphics[scale=0.315]{Figures/Exp-PED-time-size-mopsi.png}	\hspace{1ex}
	\vspace{-3ex}
	\caption{\small Evaluation of running time (\ped): varying the size of trajectories.}\label{fig:time-size-ped}
	\vspace{-2ex}
\end{figure*}

\begin{figure*}[tb!]
	\centering
	\includegraphics[scale=0.315]{Figures/Exp-SED-time-size-taxi.png}\hspace{1ex}
	\includegraphics[scale=0.315]{Figures/Exp-SED-time-size-service.png}	\hspace{1ex}
	\includegraphics[scale=0.315]{Figures/Exp-SED-time-size-geolife.png}	\hspace{1ex}
	\includegraphics[scale=0.315]{Figures/Exp-SED-time-size-mopsi.png}	\hspace{1ex}
	\vspace{-3ex}
	\caption{\small Evaluation of running time (\sed): varying the size of trajectories.}\label{fig:time-size-sed}
	\vspace{-2ex}
\end{figure*}

\begin{figure*}[tb!]
	\centering
	\includegraphics[scale=0.315]{Figures/Exp-DAD-time-size-taxi.png}\hspace{1ex}
	\includegraphics[scale=0.315]{Figures/Exp-DAD-time-size-service.png} 	\hspace{1ex}
	\includegraphics[scale=0.315]{Figures/Exp-DAD-time-size-geolife.png}	\hspace{1ex}
	\includegraphics[scale=0.315]{Figures/Exp-DAD-time-size-mopsi.png}		
	\vspace{-3ex}
	\caption{\small Evaluation of running time (\dad): varying the size of trajectories.}\label{fig:time-size-dad}
	\vspace{-2ex}
\end{figure*}





\stitle{{Exp-3.2}: Impacts of the sizes of trajectories}.
To evaluate the impacts of the number of data points in a trajectory (\ie the size of a trajectory) on running time,
we chose $10$ trajectories from each dataset \taxi, \ucar, \geolife and \mopsi, respectively,
and varied the size \trajec{|T|} of trajectories from $1,000$ points to $10,000$ points, while fixed error bound $\epsilon = 60$ meters ({or $\epsilon = 45$ degrees}).
The results are reported in Figure~\ref{fig:time-size-ped}, Figure~\ref{fig:time-size-sed} and Figure~\ref{fig:time-size-dad}.

\sstab(1) One-pass algorithms \siped, \operb, \cised and \interval scale well with the increase of trajectory size \eat{on all datasets} and show a linear running time, while batch and online algorithms do not.
This is consistent with their time complexity analyses.

\sstab(2) The running time from the smallest to the largest of these algorithms and distances is consistent with test {Exp-3.1}.

%\sstab (2) When using \ped, the running time from the smallest to the largest are one-pass algorithms \siped and \operb, and batch and online algorithms \tpa, \dpa and \bqsa. Algorithms \siped and \operb are comparable. Algorithms \tpa, \dpa and \bqsa are comparable, and they are on average \textcolor{red}{($3.8$--$5.3$, $3.5$--$4.8$, $4.6$--$7.2$, $6.2$--$8.4$)} times slower than the one-pass algorithms \siped and \operb on datasets \dSets, respectively.

%\sstab (3) When using \sed, the running time from the smallest to the largest are one-pass algorithm \cised, online algorithm \squishe, and batch algorithms \tpa and \dpa.  Algorithms \squishe, \tpa and \dpa are on average \textcolor{red}{($9.6$--$17.6$, $8.8$--$15.4$, $8.4$--$16.3$, $9.0$--$14.4$)}, \textcolor{red}{($9.6$--$17.6$, $8.8$--$15.4$, $8.4$--$16.3$, $9.0$--$14.4$)} and \textcolor{red}{($9.6$--$17.6$, $8.8$--$15.4$, $8.4$--$16.3$, $9.0$--$14.4$)} times slower than \cised on datasets \dSets, respectively.

%\sstab (4) Batch algorithms \dpa and \tpa using \sed run a bit faster than using \ped, while the one-pass algorithm \cised run \textcolor{red}{$2.0$--$3.0$} times slower than \siped and \operb.


%%%%%%%%%%%%%%%%%%%%%%%%%%%%%%%%%%%%%%%%%%%%%%%%%%%%%%%%%%%%%%%%%%%%%%%%%%%%%%
%\stitle{Summary}.
\subsubsection{Summary}
%%%%%%%%%%%%%%%%%%%%%%%%%%%%%%%%%%%%%%%%%%%%%%%%%%%%%%%%%%%%%%%%%%%%%%%%%%%%%%
From these tests we find the following.

\emph{\sstab{(1) Compression ratios}}.
(a) When using \ped, the output data sizes of sub-optimal algorithms (\tpa,
\dpa, \bqsa, \siped, \operb) are on average ($125.67\%$, $130.24\%$, $115.94\%$, $139.41\%$, $141.15\%$)
of the optimal algorithm \opt, respectively.
%the compression ratios from the best to the worst are the optimal algorithm \opt, batch and online algorithms \tpa, \dpa and \bqsa, and one-pass algorithms \siped and \operb.
(b) When using \sed, the output data sizes of sub-optimal algorithms (\tpa,
\dpa, \squishe, \cised) are on average ($126.10\%$, $124.31\%$, $180.41\%$, $136.66\%$) of the naive optimal algorithm \opt, respectively.
%the compression ratios from the best to the worst are the near optimal algorithm \nopts, batch algorithms \tpa and \dpa, one-pass algorithm \cised, and online algorithm \squishe.
(c) When using \dad, the output data sizes of sub-optimal algorithms (\tpa,
\dpa, \interval) are on average ($103.22\%$, $109.95\%$, $102.83\%$) of the naive optimal algorithm \opt, respectively.
(d) The compression ratios of algorithms using \ped are better than
using \sed. The output data sizes of algorithms (\opt, \tpa, \dpa) using \ped
are on average ($46.66\%$, $46.64\%$, $48.96\%$) of them using \sed, respectively.
(e) In practical (\eg $\epsilon <100$ meters and $\epsilon < 60$ degrees), \ped and \sed usually bring obvious better compression ratios than \dad, especially in high sampling data sets.

\emph{\sstab{(2) Average errors}}.
(a) When using \ped, the average errors from the smallest to the largest are batch algorithms \tpa, \dpa, one-pass algorithms \siped and \operb, the optimal algorithm \opt, and online algorithm \bqsa.
(b) When using \sed, the average errors from the smallest to the largest are online algorithm \squishe, batch algorithms \tpa and \dpa, one-pass algorithm \cised, and the naive optimal algorithm \opt.
(c) When using \dad, the average errors from the smallest
to the largest are batch algorithms \dpa and \tpa, one-pass algorithm \interval, and the naive optimal algorithm \opt.
(d) The average errors of algorithms using \sed are a bit larger than using \ped.
(e) In practical (\eg $\epsilon <100$ meters and $\epsilon < 60$ degrees), the average errors of algorithms using \dad, in terms of \ped, are obvious larger than algorithms using \ped and \sed.

\emph{\sstab{(3) Running time}}.
(a) When using \ped, algorithms \siped and \operb are comparable, and algorithms
(\tpa, \dpa, \bqsa) are on average ($24.0$, $16.0$, $37.2$) times slower than the one-pass algorithms \siped, respectively.
%the running time from the smallest to the largest are one-pass algorithms \siped and \operb, and batch and online algorithms \tpa, \dpa and \bqsa.
(b) When using \sed, algorithms (\tpa, \dpa, \squishe) are on average ($11.6$, $14.4$, $2.6$) times slower than \cised, respectively.
%the running time from the smallest to the largest are one-pass algorithm \cised, online algorithm \squishe, and batch algorithms \tpa and \dpa.
(c) When using \dad, algorithms \tpa and \dpa are on average
$11.4$ and $15.0$ times slower than \interval, respectively.
(d) Batch algorithms \dpa and \tpa using \sed and \dad run a bit faster than using \ped, while one-pass algorithm \cised (\sed) runs a bit slower than \siped and \operb (\ped) and \interval (\dad).


%%********************************* The End **********************************








