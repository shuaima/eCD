%%%%%%%%%%%%%%%%%%%%%%%%%%%%%%%%%%%%%%%%%%%%%%%%%%%%%%%%%%%%%%%%%%%%%%%%
\section{One-Pass Simplification}
\label{sec-operb}
%%%%%%%%%%%%%%%%%%%%%%%%%%%%%%%%%%%%%%%%%%%%%%%%%%%%%%%%%%%%%%%%%%%%%%%%


In this section, we first develop a local distance checking approach that is the key for one-pass trajectory simplification algorithms. We then present a \underline{O}ne-\underline{P}ass \underline{ER}ror \underline{B}ounded trajectory simplification algorithm, referred to as  \operb. Finally, we propose five optimization techniques.


\subsection{Local Distance Checking}
\label{subsec-fittingfunction}




Consider an error bound $\zeta$ and a sub-trajectory $\dddot{\mathcal{T}_s}[P_s,$ $\ldots, P_{s+k}]$. To achieve the local distance checking, \operb first dynamically maintains a directed line segment $\mathcal{L}_i$ ($i\in[1,k]$), whose start point is fixed with $P_s$ and its end point is identified (may not in $\{P_s, \ldots, P_{s+i}\}$) to {\em fit} all the previously processed points $\{P_s, \ldots, P_{s+i}\}$. The directed line segment $\mathcal{L}_i$ is built by a function named \emph{fitting function $\mathbb{F}$}, such that when a new point $P_{s+i+1}$ is considered, only its distance to the directed line segment $\mathcal{L}_i$ is checked, instead of checking the distances of all or a subset of data points of $\{P_{s}, \ldots, P_{s+i}\}$ to $\mathcal{R}_{i+1}$ = $\vv{P_sP_{s+i+1}}$  as the global distance checking does. In this way, a data point is checked only once during the entire process of trajectory simplification.


%%%%%%%%%%%%% Start eat %%%%%%%%%%%%%%%%%%%%%%%%%%%%%%%%%%%%%%%%%%
\eat{
More specifically, the state-of-the-art of the local distance checking approach can be demonstrated as following.

\vspace{1.5ex}

{\em The property ``$d(P_{s+i}, {\vv{P_{s}P_{s+k}})}\le \zeta$ for all points $P_{s+i}$ ($i\in[0, k]$) of $\dddot{\mathcal{T}_s}$", where $\vv{P_{s}P_{s+k}}$ is the line segment connecting the start point $P_{s}$ and the end point $P_{s+k}$, will be hold if

    (1.1) $\mathcal{L}_0$  = $\vv{P_sP_{s}}$ (\ie $|\mathcal{L}_0|$ = 0 and $\mathcal{L}_0.\theta$ = 0),

    (1.2) $\mathcal{L}_{i}$ = $\mathbb{F}(P_{s+i}, \mathcal{L}_{i-1})$ for $i\in[1, k+1]$, and

    (2.1) $d(P_{s+i}, \mathcal{L}_{i-1})\le \zeta/2$ for each $i\in[0, k]$.
}

\vspace{1.5ex}

\ni Here (1.1) and (1.2) are used to maintain the directed line segment $\mathcal{L}_i$ ($i\in[1,k]$), where $\mathbb{F}$ is referred to as a \emph{fitting function} which dynamically adjusts the  directed line segment $\mathcal{L}_i$, and (2.1) is the distance checking.
}
%%%%%%%%%%%%% End of eat %%%%%%%%%%%%%%%%%%%%%%%%%%%%%%%%%%%%%%%%%%


We next present the details of our fitting function   $\mathbb{F}$ that is designed for local distance checking.

\stitle{Fitting function $\mathbb{F}$}. Given an error bound $\zeta$ and a sub-trajectory $\dddot{\mathcal{T}_s}[P_s, \ldots, P_{s+k}]$, $\mathbb{F}$ is as follows.

\begin{small}
\vspace{-2ex}
\begin{equation*}
\label{equ-function}
\hspace{-1.5ex}\left\{
    \begin{aligned}
        &\hspace{-1.5ex}\left[
            \begin{aligned}
           % & |\mathcal{L}_{i}| = |\mathcal{R}_{i-1}|    \\
           % & \mathcal{L}_{i}.\theta = \mathcal{R}_{i-1}.\theta\\
            & \mathcal{L}_{i} = \mathcal{L}_{i-1}\\
            \end{aligned}
        \right]\hspace{12.5ex}~when~(|\mathcal{R}_{i}| - |\mathcal{L}_{i-1}|) \le \frac{\zeta}{4} \hspace{14.5ex}(1)\\
        &\hspace{-1.5ex}\left[
            \begin{aligned}
            & |\mathcal{L}_{i}|  = j*{\zeta}/{2} \\
            & \mathcal{L}_{i}.\theta = \mathcal{R}_{i}.\theta    \\
            \end{aligned}
        \right]\hspace{8.5ex}~when~|\mathcal{R}_{i}| >  \frac{\zeta}{4}~\And~|\mathcal{L}_{i-1}|=0   \hspace{8ex}(2)~ \\
        &\hspace{-1.5ex}\left[
            \begin{aligned}
            & |\mathcal{L}_{i}|  = j*{\zeta}/{2}\\
            & \mathcal{L}_{i}.\theta = \mathcal{L}_{i-1}.\theta + f(\mathcal{R}_i,\mathcal{L}_{i-1})*\arcsin(\frac{d(P_{s+i}, \mathcal{L}_{i-1})}{j*\zeta/2})/j \\	
           % & \theta^- = \mathcal{L}_{i-1}.\theta - \arcsin(\frac{d(P_i, \mathcal{L}_{i-1})}{j*\zeta/2})/j \\	
           % & \mathcal{L}_{i}.\theta = \arg_{\mathcal{L}_{i}.\theta}\min({d(P_{i+1}, \mathcal{L}_{i}}), \mathcal{L}_{i}.\theta \in\{\theta^+,\theta^-\})\\	
            \end{aligned}
        \right]\hspace{0ex}~else\hspace{1ex}(3)\\
    \end{aligned}
    \right.
\end{equation*}
\vspace{-2ex}
\end{small}


\ni where (a) $1 \le i \le k+1$; (b) $\mathcal{R}_{i-1}$ = $\vv{P_sP_{s+i-1}}$, is the directed line segment whose end point $P_{s+i-1}$ is in $\dddot{\mathcal{T}_s}[P_s, \ldots, P_{s+k}]$; (c) $\mathcal{L}_{i}$ is the directed line segment built by fitting function $\mathbb{F}$ to fit sub-trajectory $\dddot{\mathcal{T}_s}[P_s, \ldots, P_{s+i}]$ and $\mathcal{L}_{0}$ = $\mathcal{R}_{0}$; (d) $j = \lceil(|\mathcal{R}_{i}|*2/\zeta - 0.5)\rceil$; (e) $f()$ is a sign function such that $ f(\mathcal{R}_i,\mathcal{L}_{i-1})$ = $1$ if the included angle $\angle(\mathcal{R}_{i-1}, \mathcal{R}_{i})$ = $(\mathcal{R}_i.\theta - \mathcal{L}_{i-1}.\theta)$ falls in the range of $(-2\pi, -\frac{3\pi}{2}]$, $[-\pi, -\frac{\pi}{2}]$, $[0, \frac{\pi}{2}]$ and $[\pi, \frac{3\pi}{2})$, and $f(\mathcal{R}_i,\mathcal{L}_{i-1})$ = $-1$, otherwise; (f) $\zeta/2$ is a step length to control the increment of $|\mathcal{L}|$.

Given any sub-trajectory $\dddot{\mathcal{T}_s}[P_s$, $\ldots$, $P_{s+k}]$ and any error bound $\zeta$, the expression $j = \lceil(|\mathcal{R}_{i}|*2/\zeta - 0.5)\rceil$ in the fitting function $\mathbb{F}$ essentially partitions the space into zones around the center point $P_s$ such that for each $j\ge 0$, zone $Z_j$ = $\{P_j \ |\ j*\zeta/2-\zeta/4 < |\vv{P_sP_j}|\le j*\zeta/2+\zeta/4\}$, \ie the radii of $Z_0, Z_1, Z_2$ and $Z_3$ to $P_s$ are in the ranges of $(-\frac{1}{4}\zeta, \frac{1}{4}\zeta]$, $(\frac{1}{4}\zeta, \frac{3}{4}\zeta]$, $(\frac{3}{4}\zeta, \frac{5}{4}\zeta]$ and $(\frac{5}{4}\zeta, \frac{7}{4}\zeta]$, respectively, and all ranges have a fixed size $\zeta/2$ as shown in Figure~\ref{fig:fitfunction}. Moreover, the angle of the  directed line segment $\mathcal{L}_i$ is adjusted from $\mathcal{L}_{i-1}$ to make it closer to $P_{s+i}$ than $\mathcal{L}_{i-1}$, \ie $d(P_{s+i}, \mathcal{L}_{i}) \le d(P_{s+i}, \mathcal{L}_{i-1})$ for any $i \in (s, s+k]$, and the angle from $\mathcal{L}_{1}$ to $\mathcal{L}_{k}$ is bounded by a constant (Lemma~\ref{lemma-stepwise-angle}).

%This strategy let more points be compressed to a line segment. Moveover, the fitting function guarantees the one-pass as well as the error bounded features, which are proved in the next section.


The fitting function $\mathbb{F}$ also creates a virtual stepwise sub-trajectory $\dddot{\mathcal{T}_v}[V_s$, $\ldots$, $V_{s+l}]$ such that $V_s = P_s$, $|\vv{V_sV_{s+j}}|$ = $\zeta/2*j$ $(j\in[0,l], 0<l\le k)$.
For each point $P_{s+i}$ in the sub trajectory, it is mapped to a virtual point $V_{s+j}$ in $\dddot{\mathcal{T}_v}$ locating in zone $Z_j$.
%
Observe that (a) it is possible that $(|\mathcal{R}_{i}| - |\mathcal{L}_{i-1}|) \le 0$, (b) the fitting function  $\mathbb{F}$ forms a directed line segment $\mathcal{L}_i$, which is closer to $\mathcal{R}_i$ than $\mathcal{L}_{i-1}$, and partitions the points in the sub-trajectory $\dddot{\mathcal{T}}$ into two classes:


\stab (1) {\em Active points}. Points $P_{s}$ and $P_{s+i}$ such that $|\mathcal{R}_{i}| - |\mathcal{L}_{i-1}|$ $>$ $\zeta/4$ are referred to as active points. An active point $P_{s+i}$ is mapped to the virtual point in zone $Z_j$ with $j= \lceil(|\mathcal{R}_{i}|*2/\zeta - 0.5)\rceil$, and each zone has at most one active point.

\stab(2) {\em Inactive points}. Points $P_{s+i}$ such that $|\mathcal{R}_{i}| - |\mathcal{L}_{i-1}| \le \zeta/4$ are referred to as inactivated points. For an inactive point $P_{s+i}$, it is mapped to zone $Z_j$ with $j = |\mathcal{L}_{i-1}|*2/\zeta$. There may exist none or multiple inactive points in a zone.



We next explain the fitting function $\mathbb{F}$ with an example.

\begin{figure}[tb!]
\centering
\includegraphics[scale = 0.62]{figures/Fig-FitFunction.png}
\vspace{-2ex}
\caption{\small An example of the fitting function}
\label{fig:fitfunction}
\vspace{-4ex}
\end{figure}

%%%%%%%%%%%%%%%%%
\begin{example}
\label{exm-fittingfunction}
Consider the sub-trajectory $[P_0,\ldots, P_7]$  in Figure~\ref{fig:fitfunction} whose eight points fall in zones $Z_0, Z_1, Z_2, Z_3$.

\sstab (1) Point $P_0$ is the start point and the first active point, and $\mathcal{L}_0 = \mathcal{R}_0 = \vv{P_0P_0}$.

\sstab (2) Point $P_1$ is inactive in zone $Z_0$, as $|\mathcal{R}_{1}|$ = $|\vv{P_0P_1}|$ $<$ $\frac{\zeta}{4}$ and $(|\mathcal{R}_{1}| - |\mathcal{L}_{0}|) \le \frac{\zeta}{4}$. Hence, $\mathcal{L}_{1} = \mathcal{L}_{0}$  (case (1)).

\sstab (3) Point $P_2$ is active in $Z_1$, as $|\mathcal{R}_{2}| >  \frac{\zeta}{4}$ and $|\mathcal{L}_{1}|=0$ .
Hence, $|\mathcal{L}_{2}|  = \frac{\zeta}{2}$ and $\mathcal{L}_{2}.\theta = \mathcal{R}_{2}.\theta$ (case 2).

\sstab (4) Point $P_3$ is inactive in $Z_1$, as $(|\mathcal{R}_{3}| - |\mathcal{L}_{2}|) \le \frac{\zeta}{4}$. Hence, $\mathcal{L}_{3} = \mathcal{L}_{2}$  (case 1).

\sstab (5) Point $P_4$ is active in $Z_2$, as $|\mathcal{R}_{4}| - |\mathcal{L}_{3}|) > \frac{\zeta}{4}$ and $|\mathcal{L}_{3}|\ne 0$. Hence, $|\mathcal{L}_{4}|  = 2*\frac{\zeta}{2} = \zeta$ and the angle of $\mathcal{L}_4$ is also calculated accordingly (case 3).

\sstab (6) Similarly, point $P_5$ is active in $Z_3$ (case 3), and points $P_6$ and $P_7$ are inactive  (case 1). Here $P_6$ is mapped to $Z_3$ as $|\mathcal{L}_{5}| = \frac{3}{2}\zeta$ though it is physically located in zone $Z_2$.
\end{example}
%\vspace{-1ex}




\subsection {Analyses of the Fitting Function}
\label{susubbsec-fittingfunction-analysis}

We next give an analysis of the fitting function $\mathbb{F}$.

\subsubsection{Correctness of the Fitting Function}

First, by the definition of $\mathbb{F}$, it is easy to have the following.

\begin{pprop}
\label{prop-fitting-func-time}
Given any sub-trajectory $\dddot{\mathcal{T}_s}[P_s, \ldots, P_{s+k}]$ and error bound $\zeta$, the directed line segment $\mathcal{L}_{i}$ $(i\in[1,k])$ can be computed by the fitting function $\mathbb{F}$  in $O(1)$ time.
\end{pprop}

The fitting function $\mathbb{F}$ also enables a local distance checking method, as shown below.

\begin{ttheorem}
\label{prop-fitting-function}
Given any sub-trajectory $\dddot{\mathcal{T}_s}[P_s, \ldots, P_{s+k}]$ with $k\le 4\times 10^5$ and error bound $\zeta$,
then $d(P_{s+i}, {\vv{P_{s}P_{s+k}})}\le \zeta$ for each $i\in[0, k]$ if $P_{s+k}$ is an active point and $d(P_{s+i}, \mathcal{L}_{i-1})\le \zeta/2$ for each $i\in[1, k]$.
\end{ttheorem}


To prove Theorem~\ref{prop-fitting-function}, we first introduce a special class of trajectories, based on which we show that Theorem~\ref{prop-fitting-function} holds.

\stitle{Stepwise trajectories}. We say that a trajectory $\dddot{\mathcal{T}}[P_s$, $\ldots$, $P_{s+k}]$ is stepwise \wrt $\zeta/2$ if and only if $|\mathcal{R}_{i}| = i*\zeta/2$ for each directed line segment $\mathcal{R}_{i}$ = $\vv{P_sP_{s+i}}$ ($i\in[0,k]$).

Observe that $|\mathcal{R}_{i}|$  $-$ $|\mathcal{R}_{i-1}|$ = $\zeta/2$  and $\lceil|\mathcal{R}_{i}|*2/\zeta - 0.5\rceil = i$ ($i\in[1, k]$).
Hence, for stepwise sub-trajectories  $\dddot{\mathcal{T}}[P_s$, $\ldots$, $P_{s+k}]$, the  fitting function can be simplified as  $\mathbb{F}'$ below.

\begin{small}
\vspace{-1ex}
\begin{equation*}
\hspace{-1.5ex} \left\{
    \begin{aligned}
        &\hspace{-1.5ex}\left[
            \begin{aligned}
           % & |\mathcal{L}_{i}| = |\mathcal{R}_{i-1}|    \\
           % & \mathcal{L}_{i}.\theta = \mathcal{R}_{i-1}.\theta\\
            & \mathcal{L}_{1} = {\zeta}/{2}\\
            & \mathcal{L}_{1}.\theta = \mathcal{R}_{1}.\theta\\
             \end{aligned}
        \right]\hspace{46.5ex}~i=1\hspace{1ex}(1)\\
        &\hspace{-1.5ex}\left[
            \begin{aligned}
            & |\mathcal{L}_{i}|  = i*{\zeta}/{2}\\
           % & \mathcal{L}_{i}.\theta = \mathcal{L}_{i-1}.\theta + \arcsin(\frac{d(P_i, \mathcal{L}_{i-1})}{i*\zeta/c})/f(i) \\	
            & \mathcal{L}_{i}.\theta = \mathcal{L}_{i-1}.\theta + f(\mathcal{R}_i,\mathcal{L}_{i-1})*\arcsin(\frac{d(P_{s+i}, \mathcal{L}_{i-1})}{i*\zeta/2})/i \\	
            %& \theta^- = \mathcal{L}_{i-1}.\theta - \arcsin(\frac{d(P_i, \mathcal{L}_{i-1})}{i*\zeta/2})/i \\	
            %& \mathcal{L}_{i}.\theta = \arg_{\mathcal{L}_{i}.\theta}\min({d(P_{i+1}, \mathcal{L}_{i}}), \mathcal{L}_{i}.\theta \in\{\theta^+,\theta^-\})\\	
            \end{aligned}
        \right]\hspace{.5ex}i\ge 2\hspace{1ex}(2)~ \\
    \end{aligned}
    \right.
%    \eqno{\mathbb{F}'}
\end{equation*}
\vspace{-1ex}
\end{small}


Stepwise trajectories have the following properties.


%%%%%%%%%Lemma 1.%%%%%%%%%%%%
\begin{llemma}
\label{lemma-stepwise-angle}
Given any sub-trajectory $\dddot{\mathcal{T}_s}[P_s, \ldots, P_{s+k}]$ and error bound $\zeta$,
if $d(P_{s+i}, \mathcal{L}_{i-1})\le \frac{\zeta}{2}$  for each $i$ $\in[2, k]$, then the angle change between $\mathcal{L}_{1}$ and $\mathcal{L}_{k}$ is bounded by $\Delta\theta$ = $\lim\limits_{k \to \infty }{\sum_{i=2}^k\frac{1}{i} * {\arcsin(\frac{1}{i})}} < 0.8123$ (or $46.54^o$).
\end{llemma}


\begin{proof}
By the revised fitting function $\mathbb{F'}$ for a stepwise sub-trajectory and $d(P_{s+i}, \mathcal{L}_{i-1})\le \zeta/2$ for all $i$ $\in[2, k]$,
we have $\Delta\theta$ $\le {\sum_{i=2}^{k}\frac{1}{i} * {\arcsin(\frac{1}{i})}}$, which is  monotonically increasing with the increment of $k$. %, and it is known that $\lim\limits_{n \to \infty }{\sum_{i=2}^{k-1}\left(\frac{1}{i} * {\arcsin(1/i)}\right)} \approx 0.6602$.


As $x\le \arcsin(x) \le \frac{x}{\sqrt{1-x^2}}$ ($0\le x< 1$), we have $\frac{1}{i} \le \arcsin\frac{1}{i}\le$ $\frac{1}{\sqrt{i^2-1}}$ ($i\ge 2$), from which we have the following.

\sstab(1) $\Delta\theta\ge\lim\limits_{k \to \infty }{\sum_{i=2}^{k}\frac{1}{i^2}} = \pi^2/6 - 1 \approx 0.6449$, and

\sstab (2) $\Delta\theta \le \lim\limits_{k \to \infty }{\sum_{i=2}^{k}(\frac{1}{i}*\frac{1}{\sqrt{i^2-1}})}$\\

\hspace{8ex}$< {\int_{2}^{\infty}(\frac{1}{x}*\frac{1}{\sqrt{x^2-1}})}dx$ + $\frac{1}{2}*\frac{1}{\sqrt{2^2-1}}$\\

\hspace{8ex}$=\frac{\pi}{6}$ + $\frac{1}{2\sqrt{3}} \approx 0.8123$.

From these, we have the conclusion.
\end{proof}


%%%%%%%%%%%%%%%%%%%%%%%%%%%%%%%%%%%%%%%
%%  0.6602$ (or $37.83^o$), when u = 1;
%%  0.6180 (35.40)      when u=0.936
%%%%%%%%%%%%%%%%%%%%%%%%%%%%%%%%%%%%%%%

\begin{figure}[tb!]
\centering
\includegraphics[scale = 0.68]{figures/Fig-MaxDistance.png}
\vspace{-2ex}
\caption{\small Example for the proof of Lemma~\ref{lemma-stepwise-errorbound}}
\label{fig:stepwise-maxdistance}
\vspace{-3ex}
\end{figure}



%%%%%%%%%%%%%%%%%Lemma 2.
\vspace{-1ex}
\begin{llemma}
\label{lemma-stepwise-errorbound}
Given any sub-trajectory $\dddot{\mathcal{T}_s}[P_s, \ldots, P_{s+k}]$ with $k\le 4\times 10^5$ and error bound $\zeta$,
then $d(P_{s+i}, {\vv{P_{s}P_{s+k}})}\le \zeta$ for each $i\in[0, k]$ if $d(P_{s+i}, \mathcal{L}_{i-1})\le \frac{\zeta}{2}$ for each $i\in[1, k]$.
\end{llemma}




\stitle{Proof:}
Consider the four directed line segments  $\mathcal{R}_{i} = \vv{P_sP_{s+i}}$, $\mathcal{R}_{k} = \vv{P_sP_{s+k}}$,  $\mathcal{L}_{i-1}$ and  $\mathcal{L}_{k-1}$
shown in Figure~\ref{fig:stepwise-maxdistance} by Lemma~\ref{lemma-stepwise-angle}.
Further, let $\beta_1$ = $\mathcal{R}_{i}.\theta - \mathcal{L}_{i-1}.\theta$, $\beta_2$ = $\mathcal{L}_{k-1}.\theta - \mathcal{L}_{i}.\theta$, $\beta_3$ = $\mathcal{R}_{k}.\theta - \mathcal{L}_{k-1}.\theta$.
%
We then adjust the included angles $\beta_1$ as follows:

(a) if $\pi/2<|\beta_1|\le \pi$, then $\beta_1$ = $\pi$ - $\beta_1$,

(b) if $\pi<|\beta_1|\le \frac{3}{2}\pi$, then $\beta_1$ = $\beta_1$ - $\pi$,

(c) if $\frac{3}{2}\pi<|\beta_1|\le {2}\pi$, then $\beta_1$ = $2\pi$ - $\beta_1$, and

(d) {$\beta_1$ = $|\beta_1|$, otherwise.

The included angle $\beta_2$ is bounded by Lemma~\ref{lemma-stepwise-angle}, and  the included angle $\beta_3$ is adjusted along the same line as $\beta_1$.

Observe that $d(P_{s+i}, \mathcal{R}_{k})\le i*\frac{\zeta}{2}*\sin(|\beta_1| + |\beta_2| + |\beta_3|)$
%
$\le$ $i*\frac{\zeta}{2}*\sin(\arcsin(\frac{1}{i})+\sum_{j=i+1}^{k-1}{(\frac{1}{j}*\arcsin\frac{1}{j})}+\arcsin(\frac{1}{k}))$.


(1) We first show that $\arcsin(\frac{1}{i})+\sum_{j=i+1}^{k-1}{(\frac{1}{j}*\arcsin\frac{1}{j})}+\arcsin(\frac{1}{k})$ falls in $(0, \frac{\pi}{2})$.


The distance function $d(P_{s+i}, \mathcal{R}_{k})$ is monotonically increasing \wrt $k$ when fixing $i$. % and  \wrt $i$ when fixing $k$.
Let $g(k) = \sum_{j=i+1}^{k-1}{(\frac{1}{j}*\arcsin\frac{1}{j})}+\arcsin(\frac{1}{k})$, and it suffices to show that $g(k+1) - g(k)\ge 0$, which is shown below.

$g(k+1) - g(k)$ $= \frac{1}{k}*\arcsin\frac{1}{k} + \arcsin\frac{1}{k+1} - \arcsin\frac{1}{k}$\\
$\ge \frac{1}{k}*\frac{1}{k} + \frac{1}{k+1} - \frac{1}{\sqrt{k^2-1}} > 0$.


When $i\ge 2$, we have the following.

 $\arcsin\frac{1}{i}+\sum_{j=i+1}^{k-1}{(\frac{1}{j}*\arcsin\frac{1}{j})} + \arcsin\frac{1}{k}$\\
 $\le\arcsin(\frac{1}{i})$ + $\sum_{j=i+1}^{\infty}{(\frac{1}{j}*\arcsin\frac{1}{j})}$\\
 $\le\arcsin(\frac{1}{2})$ + $\sum_{j=3}^{\infty}{(\frac{1}{j}*\arcsin\frac{1}{j})}$\\
 $\approx 1.0741 = 61.54^o$.


(2) For $k\le 4\times 10^5$, we have $i**\sin(|\beta_1| + |\beta_2| + |\beta_3|)<2$, and, hence,
we have $d(P_{s+i}, \mathcal{R}_{k})< \frac{\zeta}{2}*2 = \zeta$.


This completes the proof.
\eop
\vspace{1ex}


By mapping inactive and active points of a trajectory to virtual points of a trajectory stepwise \wrt ${\zeta}/2$, one can readily prove Theorem~\ref{prop-fitting-function} along the lines as Lemma~\ref{lemma-stepwise-errorbound}.
%Detailed proofs and extra analyses can be found in~\cite{OPERB-full}.


\begin{figure}[tbh!]
\centering
\includegraphics[scale = 0.72]{figures/Fig-AnyTraj.png}
\vspace{-0ex}
\caption{Example for the proof of Theorem~\ref{prop-fitting-function}}
\label{fig:appendix-anytraj}
\vspace{-2ex}
\end{figure}


\stitle{Proof:} We only need to consider those points $P_{s+i}$ such that $|\vv{P_sP_{s+i}}|>\zeta$ (otherwise, it is obvious that $d(P_{s+i}, \mathcal{R}_{k})\le \zeta$).
(1) First consider inactive points  $P_{s+i}$ that are mapped to a virtual point $V_{j_i}$ in $\dddot{\mathcal{T}_v}$ ($j_i\ge 2$).
(2) Then consider active points  $P_{s+i}$ that are mapped to a virtual point $V_{j_i}$ in $\dddot{\mathcal{T}_v}$ ($j_i= \lceil(|\mathcal{R}_{i}|*2/\zeta - 0.5)\rceil$).
We assume that $P_{s+k}$ is an active point and $j_k>j_i$.

Then $d(P_{s+i}, \mathcal{R}_{k})$ = $|\mathcal{R}_{i}|*\sin(|\beta_1|+|\beta_2|+|\beta_3|)$, where $\beta_1$, $\beta_2$ and $\beta_3$ are the included angles between
$V_{j_i}$ and $P_{s+i}$, between $V_{j_i}$ and $V_{j_k}$ and between $V_{j_k}$ and $P_{s+k}$ that are adjusted along the same lines as the proof of Lemma~\ref{lemma-stepwise-errorbound}, as shown in Figure~\ref{fig:appendix-anytraj}.
%
Note that here $\beta_1\le \frac{\zeta}{4}/(j_i*\frac{\zeta}{2}) = \arcsin\frac{1}{2j_i}$ and $\beta_3\le \frac{\zeta}{4}/(j_k*\frac{\zeta}{2}) = \arcsin\frac{1}{2j_k}$.
Along the same lines as the proof of Lemma~\ref{lemma-stepwise-errorbound}, we have the following.

$d(P_{s+i}, \mathcal{R}_{k})$ (when $i\ge 2$ and $k\le 4\times 10^5$)\\
$\le \frac{\zeta}{2}*(j_i+\frac{1}{2})*\sin(\arcsin\frac{1}{2j_i}+\arcsin\frac{1}{2j_k}+\arcsin\sum_{i=j_i+1}^{j_k-1} \frac{1}{i}*\arcsin\frac{1}{i})$
%$\le (j_i+1)*\frac{\zeta}{2}*\sin(\arcsin\frac{1}{j_i+1}+\arcsin\frac{1}{j_k+1}+\arcsin\sum_{i=j_i}^{j_k} \frac{1}{i}*\arcsin\frac{1}{i})$\\
%$\le (j_i+1)*\frac{\zeta}{2}*(\sin(\arcsin(\frac{1}{(j_i+1)}))+\sin(\arcsin(\frac{1}{(j_i+1)})))$\\
$\le\zeta$.
\eop









\stitle{Remarks}.
%
(1) Our fitting function achieves local distance checking, as indicated by Proposition~\ref{prop-fitting-func-time} and Theorem~\ref{prop-fitting-function};
(2) For a sub-trajectory $\dddot{\mathcal{T}_s}[P_s, \ldots, P_{s+k-1}]$ represented by a single directed line segment,
 we restrict $k\le 4*10^5$, which suffices  for the need of trajectory simplification in practice.



\begin{ttheorem}
Given any sub trajectory $\dddot{\mathcal{T}}[P_s, \ldots, P_{s+k}]$ and error bound $\zeta$, let $d^+_{max}$ = $\max\{d(P_{s+i}, \mathcal{L}_{i-1}) ~|~ f(\mathcal{R}_i,\mathcal{L}_{i-1})=1 ~and~ i \in [1, k]\}$ and $d^-_{max}$ = $\max \{d(P_{s+i}, \mathcal{L}_{i-1}) ~|~f(\mathcal{R}_i,\mathcal{L}_{i-1})=-1 ~and~ i \in [s+1, s+k]\}$. Then the condition $d(P_{s+i}, \mathcal{L}_{i-1}) \le \zeta/2$ in Theorem~\ref{prop-fitting-function} can be replaced with $d^-_{max} + d^+_{max} \le \zeta$, and algorithm \operb remains error bounded by $\zeta$.
\end{ttheorem}

For example, if $d^+_{max} = 0.3\zeta$ and  $d^-_{max} = 0.6\zeta$, then $d(P_{s+i}, \mathcal{L}_{k})$ for each $i \in [s, s+k]$ is still less than $0.3\zeta + 0.6\zeta= 0.9\zeta < \zeta$.

Note that  $d(P_{s+i}, \mathcal{L}_{i-1}) \le \zeta/2$ implies $d^+_{max} \le \zeta/2$ and $d^-_{max} \le \zeta/2$, and, hence, $d^-_{max} + d^+_{max} \le \zeta$.
Therefore, $d(P_{s+i}, \mathcal{L}_{i-1}) \le \zeta/2$ is a special case of  $d^-_{max} + d^+_{max} \le \zeta$.



\subsubsection{\textcolor[rgb]{1.00,0.00,0.00}{Effectiveness of the Fitting Function}}

\textcolor[rgb]{1.00,0.00,0.00}{(todo...comparing with Sleeve)}


We can elaborately change $\psi(j)$ of $\ffunc{F_B}$ to get other fitting functions, as follows:

\begin{itemize}
  \item {$\psi(j) = \log(j)$. $\log()$ function returns the natural logarithm (base $e$) of a number. If $\psi(j) = \log(j)$, then the deflection of the angle of $\vec{R}'_n$ to $\vec{R}'_1$,  $|\theta'_n - \theta'_1|$, is less than $\sum_{i=2}^n\left( \frac{\mu}{\log j} * {\arcsin(1/j)} \right)$. However, it is increasing with $n$ without an up bound. The fitting function using $\psi= \log(j)$ is \textbf{not} error bounded, so it is not an alternative fitting function.}

  \item  {$\psi(j) = j$. It is the function used in $\ffunc{F_B}$.}

  \item  {$\psi(j) = j*\log(j)$. If we let $\psi(j) = j*\log(j)$, then $\Delta\theta'_{max}$ = $\sum_{j=2}^n\left( \frac{\mu}{\psi(j)} * {\arcsin(1/j)} \right)$ has an up bound which is obviously less than $0.6245$. So it is an alternative fitting function for \operaa. Note that $\psi(j)$ may have some other variations, such as changing "$\log(j)$" to "$\log(1+j)$" or "$1+ \log(j)$".}

  \item  {$\psi(j) = j^2$. It has an even smaller $\Delta\theta'_{max}$. It is another alternative fitting function for \operaa.}

\end{itemize}

\begin{theorem}
\label{ffunction}
Given any trajectory $\dddot{\mathcal{T}}[P_0, \ldots, P_n]$ and any error bound $\zeta$, \operaa using $\psi(j)= j$, $\psi(j)= j*\log(j)$ and $\psi(j)= j^2$ have the same error bound.
\end{theorem}

\begin{proof}
First, the $\Delta\theta'_{max}$ of \operaa using $\psi(j)= j*\log(j)$ or $\psi(j)= j^2$ has a smaller up bound than using $\psi(j)= j$, so its error bound should not larger than using $\psi(j)= j$; Secondly, as shown in Figure~\ref{fig:maxdistance}, the max value of $d(P_k, {\mathcal{L}}(\vec{R}_{n}))$ of \operaa using $\psi(j)= j*\log(j)$ or $\psi(j)= j^2$  is also converging to $2\zeta'$ when $k=n-1$ and $n \to \infty$. So they have the same error bound, and they are all error bounded by $2*\zeta' = \zeta$.
\end{proof}

\stitle{Remark}. \operaa is more preferable to $\psi(j)= j$ than $\psi(j)= j*\log(j)$ or $\psi(j)= j^2$, as they have the same error bound for stepwise trajectories  while \operaa using $\psi(j)= j$ has a larger $\Delta\theta'_{max}$, so it is likely to compress more points into a line segment.





\subsection{Algorithm OPERB}
\label{subsec-algorithm}

We are now ready to present our one-pass error bounded algorithm, which makes use of the local distance checking method based on the fitting function $\mathbb{F}$.

The main result of this section is as follows.


\begin{ttheorem}
\label{thm-operb}
Given any trajectory $\dddot{\mathcal{T}}[P_0, \ldots, P_{n}]$ and error bound $\zeta$, there exists a one-pass trajectory simplification algorithm that is error bounded by $\zeta$.
\end{ttheorem}

We prove Theorem~\ref{thm-operb} by providing such an algorithm for  trajectory simplification, referred to as \operb shown in Figure~\ref{alg:operb}. Given
a trajectory $\dddot{\mathcal{T}}[P_0, \ldots, P_n]$ and an error bound $\zeta$ as input, algorithm \operb outputs a compression trajectory, \ie a piecewise line representation $\overline{\mathcal{T}}$ of $\dddot{\mathcal{T}}$.

\eat{%%%%%%%%%%%%%%%%
\begin{algorithm}[tb!]
\begin{small}
\caption{A One-Pass Error Bounded Compression}\label{alg:operb}
$\operb(\dddot{\mathcal{T}}[P_0,\ldots,P_n], \zeta)$

1   ~~$\overline{\mathcal{T}}$ := $\emptyset$; \ \ $P_a := P_0$;\ \ $P_e := P_0$; \\
2   ~~$\While P_a \ne \kw{nil}\ \Do$ \{\\
3   ~~~~~~$P_s := P_e$; \ \ $\mathcal{L}_{a}$ = $\mathbb{F}(P_a, \vv{P_sP_s})$;\\%\ \ $\mathcal{L}_{s}$ = $\vv{P_sP_s}$;\\
%3   ~~~~~$P_a := P_e := P_s$; \ \ $\mathcal{L}_{a}$ := $\mathcal{L}_{e}$ := $\mathcal{L}_{s}$;\\
4   ~~~~~~$(P_a,flag)$ := $\kw{getNextActivePoint}(\dddot{\mathcal{T}}, P_s, P_a, \mathcal{L}_{a})$;\\
%~ \hspace{10ex} ~~~~   ~~ {\small /*case (1) of $\mathbb{F}$*/}\\
5   ~~~~~~\While $flag=\True$ \Do \{\\
6   ~~~~~~~~~$\mathcal{L}_{a}$ := $\mathbb{F}(P_a, \mathcal{L}_{a})$;\ \   $P_e := P_a$;  \\ %$\mathcal{L}_{e}$ := $\mathcal{L}_{a}$;\\
7   ~~~~~~~~~$(P_a,flag)$ := $\kw{getNextActivePoint}(\dddot{\mathcal{T}}, P_s, P_a, \mathcal{L}_{a})$;\ \}\\
8  ~~~~~~$\overline{\mathcal{T}}$  := $\overline{\mathcal{T}}\cup\{\vv{P_sP_e}\}$;\ \}\\
9   ~~\Return $\overline{\mathcal{T}}$ .\\

\kw{Procedure} $\kw{getNextActivePoint}(\dddot{\mathcal{T}}, P_s, P_a, \mathcal{L}_a)$\\
1   ~~$i := a+1$;\ \ $flag$ := \True; \\
%3   ~~$\mathcal{R}_i$ := $\vv{P_sP_i}$;\\
2  ~~\While($(|\mathcal{R}_i| - |\mathcal{L}_a|) \le \zeta/4$ \& $i\le n$ \&  $(i-s)\le 4\times 10^5$ \Do\{\\
%3   ~~~~\If ${\bf abs}(|\mathcal{R}_i| - |\mathcal{L}|) \le \zeta/4$ \Then\\
3   ~~~~~\If $d(P_i, \mathcal{L}_a) > \zeta/2$ \Or $d(P_i, \mathcal{R}_a) > \zeta$ \Then \\
4   ~~~~~~~ $flag$ := \False; \ \ \Break; \\
5   ~~~~~$i$ := $i + 1$; \}\\  % $\mathcal{R}_i$ := $\vv{P_sP_i}$; \}\\
%3   ~~~~~~\ \ $\mathcal{R}_i$ := $\vv{P_sP_i}$;\\
6   ~~\If $i = n + 1$  \Then $P_i$ :=\kw{nil};\ \ $flag$ := \False; \\
7   ~~\Return $(P_i, flag)$.
\end{small}
\end{algorithm}
}%%%%%%%%%%%%%%%%%%%%%%%%%%%%%


%%%%%%%%%%%%%%%%%%%%%Baseline Algorithm
\begin{figure}[tb!]
%\vspace{1ex}
\begin{center}
{\small
\begin{minipage}{3.36in}
\myhrule \vspace{-2ex}
\mat{0ex}{
{\bf Algorithm}~$\operb(\dddot{\mathcal{T}}[P_0,\ldots,P_n], \zeta)$\\
\sstab
\bcc \hspace{1ex}\=$\overline{\mathcal{T}}$ := $\emptyset$; $P_e := P_0$; $(P_a,flag)$ := $\kw{getActivePoint}(\dddot{\mathcal{T}}, P_0, P_0, \mathcal{L}_{0}, \zeta)$;\\
\icc \>$\While P_a \ne \kw{nil}\ \Do$ \{\\
\icc \>\hspace{4ex}\=$P_s := P_e$; \ \ $\mathcal{L}_{a}$ = $\mathbb{F}(P_a, \vv{P_sP_s})$;\\
\icc \>\>$(P_a,flag)$ := $\kw{getActivePoint}(\dddot{\mathcal{T}}, P_s, P_a, \mathcal{L}_{a}, \zeta)$;\\
\icc \>\>\While $P_a \ne \kw{nil} ~\&~ flag=\True$ \Do \{\\
\icc \>\>\hspace{4ex}\=$\mathcal{L}_{a}$ := $\mathbb{F}(P_a, \mathcal{L}_{a})$;\ \   $P_e := P_a$;  \\
\icc \>\>\>$(P_a,flag)$ := $\kw{getActivePoint}(\dddot{\mathcal{T}}, P_s, P_a, \mathcal{L}_{a}, \zeta)$;\ \}\\
\icc \>\>$\overline{\mathcal{T}}$ := $\overline{\mathcal{T}}\cup\{\vv{P_sP_e}\}$;\ \}\\
\icc \>\Return $\overline{\mathcal{T}}$.\\
\sstab
{\bf Procedure}~$\kw{getActivePoint}(\dddot{\mathcal{T}}, P_s, P_a, \mathcal{L}_a, {\zeta})$\\
\sstab
\bcc \hspace{1ex}\=$i := a+1$;\ \ $flag$ := \True; \\
\icc\> \While($(|\mathcal{R}_i| - |\mathcal{L}_a|) \le \zeta/4$ \& $i\le n$ \&  $(i-s)\le 4\times 10^5$ \Do\{\\
\icc \>\hspace{4ex}\=\If $d(P_i, \mathcal{L}_a) > \zeta/2$ \Or $d(P_i, \mathcal{R}_a) > \zeta$ \Then \\
\icc \>\>\hspace{4ex}\=$flag$ := \False; \ \ \Break; \\
\icc \>\>$i$ := $i + 1$; \}\\
\icc \>\If $d(P_i, \mathcal{L}_a) > \zeta/2$   \&  $|\mathcal{L}_a|>0$  \Then $flag$ := \False;\\
\icc \>\If $i = n + 1$  \Then $P_i$ :=\kw{nil};\\
\icc \>\Return $(P_i, flag)$.
}
\vspace{-2.5ex}
\myhrule
\end{minipage}
}
\end{center}
\vspace{-3.5ex}
\caption{\small Algorithm \operb}\label{alg:operb}
\vspace{-3ex}
\end{figure}
%%%%%%%%%%%%%%%%%%%%%%%%%%%%%%%%%%%%%


We first describe its procedure, and then present \operb.

\stitle{Procedure \kw{getActivePoint}}. It takes as input a trajectory $\dddot{\mathcal{T}}$, a start point $P_s$, the current active point $P_a$, the current directed line segment $\mathcal{L}_a$ and the error bound $\zeta$, and finds the next active point $P_i$. (1) When $P_i = \kw{nil}$, it means that no more active points could be found in the remaining sub-trajectory $\dddot{\mathcal{T}}[P_s,\ldots,P_n]$; (2) When $flag = \True$, it means that the next active point $P_i$ can be combined with the current directed line segment $\mathcal{L}_a$ to form a new directed line segment; Otherwise, (3) a new line segment should be generated, and a new start point is considered.

It first increases $a$ by $1$ as it considers the data points after $P_a$, and sets $flag$ to \True\ (line 1).
Secondly, by the definition of the fitting function $\mathbb{F}$, it finds the next active point $P_i$ (lines 2--{6}).
\eat{More specifically, if $(|\mathcal{R}_i| - |\mathcal{L}_a|) \le \zeta/4$ and $d(P_i, \mathcal{L}_a) > \zeta/2$ (or $d(P_i, \mathcal{R}_a) > \zeta$), then $flag$ is set to \False~(lines 3--4); if $(|\mathcal{R}_i| - |\mathcal{L}_a|) > \zeta/4$ and $d(P_i, \mathcal{L}_a) > \zeta/2$, then $flag$ is also set to \False~(lines 6).}
Thirdly, if $i = n + 1$, then all data points in $\dddot{\mathcal{T}}$ have been considered, hence, $P_i$ is set to \kw{nil}(line 7).
Finally, $(P_i, flag)$ is returned (line 8).


%which means a new directed line segment $\vv{P_sP_i}$ should be formed, and a new start active point $P_i$ should be considered.
%We now  present \operb.

\stitle{Algorithm \operb}.  It takes as input a trajectory $\dddot{\mathcal{T}}$ and an error bound $\zeta$, and returns the simplified trajectory $\overline{\mathcal{T}}$.

%%%%%%%%%Full version of description
\eat{
It first initializes $\overline{\mathcal{T}}$ to be an empty set and the last active point $P_e$ to be $P_0$, and gets the current active point $P_a$ by calling procedure \kw{getActivePoint} (line 1). It then repeatedly processes the data points in $\dddot{\mathcal{T}}$ one by one until all data points have been considered, \ie $P_a = \kw{nil}$ (lines 2--8).
The start point $P_s$ of a new directed line segment is always set to the last active point $P_e$, and directed line segment $\mathcal{L}_a$ is formed by the fitting function $\mathbb{F}$ (line 3). Note that here $\mathcal{L}_{a}$ = $\mathbb{F}(P_a, \vv{P_sP_s})$ is an abbreviation, which needs to recursively call $\mathbb{F}$ to compute $\mathcal{L}_{a}$. It next calls \kw{getActivePoint} to get the next active point $P_a$.
If $P_a$ is not $\kw{nil}$ and $flag$ is \True, it means that $P_a$ can be combined with the current directed line segment $L_a$ (lines 5--7). Hence, $\mathcal{L}_{a}$ is recomputed by $\mathbb{F}$ and $P_e$ is set to $P_a$ (line 6), and \kw{getActivePoint} is called again to recompute $(P_a, flag)$ (line 7). If $flag$ is \False, then a directed line segment $\vv{P_sP_e}$ is generated and added to $\overline{\mathcal{T}}$ (line 8).
Finally, the set $\overline{\mathcal{T}}$ of directed line segments, \ie a piecewise line segmentation of $\dddot{\mathcal{T}}$, is returned (line 9).
}

%%%%%%%%%Short version of description
After initializing (line 1), it then repeatedly processes the data points in $\dddot{\mathcal{T}}$ one by one until all data points have been considered, \ie $P_a = \kw{nil}$ (lines 2--8).
\eat{The start point $P_s$ of a new directed line segment is always set to the last active point $P_e$, and directed line segment $\mathcal{L}_a$ is formed by the fitting function $\mathbb{F}$ (line 3). Note that here $\mathcal{L}_{a}$ = $\mathbb{F}(P_a, \vv{P_sP_s})$ is an abbreviation, which needs to recursively call $\mathbb{F}$ to compute $\mathcal{L}_{a}$. It next calls \kw{getActivePoint} to get the next active point $P_a$. }
If $P_a$ is not $\kw{nil}$ and $flag$ is \True, it means that $P_a$ can be combined with the current directed line segment $L_a$ (lines 5--7).
%Hence, $\mathcal{L}_{a}$ is recomputed by $\mathbb{F}$ and $P_e$ is set to $P_a$ (line 6), and \kw{getActivePoint} is called again to recompute $(P_a, flag)$ (line 7).
If $flag$ is \False, then a directed line segment $\vv{P_sP_e}$ is generated and added to $\overline{\mathcal{T}}$ (line 8).
Finally, the set $\overline{\mathcal{T}}$ of directed line segments, \ie a piecewise line segmentation of $\dddot{\mathcal{T}}$, is returned (line 9).


We next explain algorithm \operb with an example.

\begin{example}
\label{exam-alg-operb}
Algorithm  \operb takes as input the trajectory and $\zeta$ shown in Figure~\ref{fig:dp}, and its output is illustrated in Figure~\ref{exa-fig:operb}.
The process of algorithm  \operb is as follows.

\sstab(1) It initializes $\overline{\mathcal{T}}$  with $\emptyset$, the last active point $P_e$ with $P_0$ and the current active point $P_a$ with $P_1$ (line 1).

\sstab(2) As $P_a \ne \kw{nil}$ (line 2), it then sets $P_s = P_e = P_0$ and $\mathcal{L}_a = \mathbb{F}(P_a, \vv{P_sP_s}) =\mathbb{F}(P_1, \vv{P_0P_0})$ = $\mathcal{L}_1$ (line 3).

\sstab(3) It then calls $\kw{getActivePoint()}$ to get the next active point $P_a = P_3$ and $flag = \True$ (line 4). As $flag = \True$ means that $P_3$ can be combined with the current line segment $\mathcal{L}_a$ = $\mathcal{L}_1$, so it updates $\mathcal{L}_a$  to $\mathcal{L}_3$, and $P_e$ to $P_3$ (line 6).

\sstab(4) Then it continues reading the next active point $P_a$ = $P_5$ with $flag = \True$  (line 7), and updates the current line segment $\mathcal{L}_a$ to $\mathcal{L}_5$, and $P_e$  to $P_5$ (lines 5, 6).

\sstab(5) It gets the next active $P_a = P_6$ and $flag = \False$,  as $d(P_6, \mathcal{L}_5)>\frac{\zeta}{2}$, meaning that $P_6$ should not be compressed to the current directed line segment (line 5). Hence, it adds $\vv{P_sP_e} = \vv{P_0P_5}$ to $\overline{\mathcal{T}}$ (line 8), sets $P_s = P_e = P_5$, and updates $\mathcal{L}_a$ to $\mathbb{F}(P_a, \vv{P_sP_s}) =  \mathbb{F}(P_6, \vv{P_5P_5})$ = $\mathcal{L}_6$ (line 3).

\sstab(6) The process continues until all points have been processed. At last, the algorithm outputs five continuous line segments $\{\vv{P_0P_5}$, $\vv{P_5P_6}$, $\vv{P_6P_8}$, $\vv{P_8P_{10}}$, $\vv{P_{10}P_{14}}\}$.
\end{example}

\begin{figure}[tb!]
\centering
\includegraphics[scale = 0.64]{figures/Fig-oper.png}
\vspace{-2ex}
\caption{\small A running example of algorithm \operb.}\label{exa-fig:operb}
\vspace{-3ex}
\end{figure}

\vspace{-1ex}
\stitle{Correctness \& complexity analysis}. The correctness of algorithm \operb follows from  Theorem~\ref{prop-fitting-function} immediately.
 Observe that for a trajectory $\dddot{\mathcal{T}}$ with $n$ data points, the fitting function $\mathbb{F}$ is called at most $n$ times, and each data point is considered once and only once. By Proposition~\ref{prop-fitting-func-time}, algorithm \operb runs in $O(n)$ time.
It is also easy to verify that algorithm \operb takes $O(1)$ space, as the directed line segment in $\overline{\mathcal{T}}$ can be output immediately once it is generated.

Note that this also completes the proof of Theorem~\ref{thm-operb}.


\subsection{Optimization Techniques}
\label{subsec-optimizations}

We further propose four optimization techniques for \operb to achieve a better compression ratio. The key idea behind this is to (1) compress as many points as possible with a directed line segment, or (2) to let the directed line segment $\mathcal{L}_i$ as close as possible to the current active point $P_i$ so that it has a higher possibility to represent $P_{i+1}$. These optimization techniques are organized by the processing order from the start to the end points of directed line segments.

\stitle{(1) Choosing the first active point after $P_s$}. Algorithm~\operb calls procedure $\kw{getActivePoint}$ to get the first active point $P_a$ in a sub-trajectory $\dddot{\mathcal{T}}[P_{s+1}, \ldots,$ $P_{s+k}]$ such that $|\vv{P_sP_{a}}|>\zeta/4$ (line 4 in Figure~\ref{alg:operb}).
However, as indicated by the fitting function $\mathbb{F}$, we can replace $P_a$ with the first point $P_{b}$ such that $|\vv{P_sP_{b}}|>\zeta$, without affecting the boundness of algorithm~\operb.
This method potentially improves the compression ratio because more points are covered by the directed line segment $\vv{P_sP_b}$ than $\vv{P_sP_a}$,
and $\vv{P_sP_b}$ is also closer to $P_b$ than $\vv{P_sP_a}$.

%start from $P_{s+1}$ which satisfies $|P_sP_i| \ge \zeta'$, and lets it as the end point of the first fitting vector $\vec{R}'$ (Lines 3,10). However, in practical, a $P_i$ satisfying $|P_sP_i| \ge \zeta$ is usually a better choice in term of compression ratio.

\stitle{(2) Making $\mathcal{L}$ more close to the active points}. When \operb calculates the angle $\mathcal{L}_{i}.\theta$ of an active point $P_i$, the factor $d(P_i, \mathcal{L}_{i-1})$ in the fitting function $\mathbb{F}$ can be replaced by a bigger number $d_x$ such that $0 \le d_x \le d^-_{max}$ when $f(\mathcal{R}_i,\mathcal{L}_{i-1})=-1$ or $0 \le d_x \le d^+_{max}$ when $f(\mathcal{R}_i,\mathcal{L}_{i-1})=1$, to let $\mathcal{L}_i$ be more close to $P_i$, under the restriction that $(\arcsin(\frac{d_x}{j*\zeta/2})/j)$ is not larger than $\arcsin(\frac{d(P_i, \mathcal{L}_{i-1})}{j*\zeta/2})$.

\stitle{(3) Incorporating missing active points}. For a sub-trajectory $\dddot{\mathcal{T}}[P_s$, $\ldots, P_a, \ldots, P_{a+i},\ldots, P_{s+k}]$, whereas $P_{a}$ and $P_{a+i}$ are two consecutive active points. Let $j_a$ = $\lceil(|\mathcal{R}_{a}|*2/\zeta - 0.5)\rceil$, $j_{a+1}$ = $\lceil(|\mathcal{R}_{a+i}|*2/\zeta - 0.5)\rceil$ and $\Delta j = j_{a+1} - j_a$. If $\Delta j > 1 $, then there are no active points between zones $Z_{j_a}$ and $Z_{j_{a+1}}$. In this case, we replace $\mathcal{L}_{a+i}.\theta$ with $\mathcal{L}_{a+i-1}.\theta + f(\mathcal{R}_{a+i},\mathcal{L}_{a+i-1})*\arcsin(\frac{d(P_{a+i}, \mathcal{L}_{a+i-1})}{j_{a+1}*\zeta/2})*\frac{\Delta j}{j_{a+1}}$ for the fitting function $\mathbb{F}$, instead of $\mathcal{L}_{a+i-1}.\theta + f(\mathcal{R}_{a+i},\mathcal{L}_{a+i-1})*\arcsin(\frac{d(P_{a+i}, \mathcal{L}_{a+i-1})}{j_{a+1}*\zeta/2})*\frac{1}{j_{a+1}}$ to compensate the side effects of missing active points to make the line $\mathcal{L}_{a+i}$ more closer to $P_{a+i}$. Note that $\mathcal{L}_{a+i-1} = \mathcal{L}_{a}$ and $\Delta j>0$. Moreover, $d(P_{a+i}, \mathcal{L}_{a+i-1})$ could also be replaced by $d_x$ as above.

\stitle{(4) Absorbing data points after $P_{s+k}$}.
Given any sub-trajectory $\dddot{\mathcal{T}_s}[P_s, \ldots, P_{s+k},\ldots,P_{s+t}]$ and error bound $\zeta$, if $[P_s, \ldots, P_{s+k}]$ is compressed to a line segment $\vv{P_{s}P_{s+k}}$, then any point $P_{s+t}$ ($t>k$) can also be compressed to $\vv{P_{s}P_{s+k}}$ as long as $d(P_{s+t}, {\vv{P_{s}P_{s+k}}})\le\zeta$ such that \operb can compress more points into the directed line segment.



%To implement this feature, one could simply insert a line of codes ``$\kw{While}~(d(P_a, \mathcal{L}_e) < \zeta)~ \{~a:=a+1;\}$" before line 8 of Algorithm~\ref{alg:operb}.

\stitle{Remark}. These optimization techniques are seamlessly integrated into \operb, and Theorem~\ref{thm-operb} remains intact.

%%%%%%%%%%%%%%%%%%%%%%%%%%%%%%%%%%%%%%%%%%%%%%%%%%%%%%%%%%%%%% The End %%%%%%%%%%%%%%%%%%%%%%%%%%%%%%%%%%%%%

