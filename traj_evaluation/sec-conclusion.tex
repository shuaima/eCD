\vspace{-1ex}
%%%%%%%%%%%%%%%%%%%%%%%%%%%%%%%%%%%%%%%%%%%%%%%%%%%%%%%%%%%%%%%%%%%%%%%%%%%%%%
\section{Conclusions}
%%%%%%%%%%%%%%%%%%%%%%%%%%%%%%%%%%%%%%%%%%%%%%%%%%%%%%%%%%%%%%%%%%%%%%%%%%%%%%

We have evaluated the state-of-the-art \lsa algorithms for trajectory compression, including \emph{both the optimal and sub-optimal methods that use either \ped or \sed}. 
Using a variety of real trajectory datasets, we evaluated the performance of each technique. % in terms of its processing time, compression ratio and average error.
Our experimental results show that 
(1) the output sizes of algorithms using \sed are approximately $200\%$ of using \ped, 
(2) the output sizes of sub-optimal algorithms, except \squishe, are approximately $120\%$--$150\%$ of the optimal algorithms, and 
(3) the one-pass algorithms \siped and \operb and \cised are tens of times faster than the batch and online algorithms \tpa, \dpa and \bqsa, and more than $2$ times faster than \squishe, while they still have comparable compression ratios with batch algorithms. Hence, they are more suitable for trajectory compression, especially on resource constraint mobile devices.