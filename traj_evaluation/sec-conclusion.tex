\vspace{-1ex}
%%%%%%%%%%%%%%%%%%%%%%%%%%%%%%%%%%%%%%%%%%%%%%%%%%%%%%%%%%%%%%%%%%%%%%%%%%%%%%
\section{Conclusions}
%%%%%%%%%%%%%%%%%%%%%%%%%%%%%%%%%%%%%%%%%%%%%%%%%%%%%%%%%%%%%%%%%%%%%%%%%%%%%%

We have evaluated the state-of-the-art \lsa algorithms for trajectory compression, including \emph{both the optimal and sub-optimal methods that use either \ped or \sed}. 
Using a variety of real trajectory datasets, we evaluated the performance of each technique. % in terms of its processing time, compression ratio and average error.
Our experimental results show that: 

(1) The using of synchronized distance \sed saves temporal information of trajectories with a cost of approximately double-sized (\ie $200\%$) outputs compared with using \ped, in all datasets and all algorithms. Users are freely to choose \sed or \ped, depending on application requirements.

(2) The optimal algorithms bring the best compression ratios, however, their efficciecies are more than tens (hundreds) of times poorer than sub-optimal (one-pass) algorithms. It seems that the optimal algorithms are impractical, especially in case that the input data set is large or computing resources are limited.

(3) The output sizes of sub-optimal algorithms, except \squishe, are approximately $120\%$--$150\%$ of the optimal algorithms, and in sub-optimal algorithms, the one-pass algorithms \siped and \operb and \cised are tens of times faster than the batch and online algorithms \tpa, \dpa and \bqsa, and more than $2$ times faster than \squishe, while they still have comparable compression ratios with batch algorithms. Hence, they are more suitable for trajectory compression, especially on resource constraint mobile devices.

In the future of trajectory simplification, there may be some work worthy of exploration:

(1) The efficiency of the optimal algorithm using \sed is very poor and it is impossible to run it for a a long trajectory. Thus, a fast and resource-optimized optimal algorithm using \sed may be valuable.

(2) The effectiveness of sub-optimal algorithms remains a small room to be improved as their outputs are approximately $120\%$--$150\%$ of the optimal algorithms.

(3) The combination of trajectory simplificatoin algorithms and the self-adaptation of algorithm parameters.

%(3) The combination of trajectory simplification algorithms with other fundamental algorithms, such as location track, event detection, \etc.