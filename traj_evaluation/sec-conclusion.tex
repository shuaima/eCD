\vspace{-1ex}
%%%%%%%%%%%%%%%%%%%%%%%%%%%%%%%%%%%%%%%%%%%%%%%%%%%%%%%%%%%%%%%%%%%%%%%%%%%%%%
\section{Conclusions}
%%%%%%%%%%%%%%%%%%%%%%%%%%%%%%%%%%%%%%%%%%%%%%%%%%%%%%%%%%%%%%%%%%%%%%%%%%%%%%

Using four real-life trajectory datasets, we have systematically evaluated and analyzed error bounded \lsa algorithms for trajectory compression, including \emph{both the optimal and the sub-optimal methods that use \ped, \sed and/or \dad},  in terms of compression ratios, average errors and efficiency.
Our experimental studies and analyses show that:
%First of all,


\stitle{(1) Choice of \lsa algorithms}. Optimal algorithms bring the best compression ratios, however, their efficiency is obvious poorer than sub-optimal algorithms. The optimal simplified trajectory has little meaning from the perspective of applications. Hence, it seems that the optimal algorithms are essentially impractical, especially in cases when the input data set is large or computing resources are limited.

For sub-optimal algorithms, the output sizes of algorithms \bqsa and \siped($\epsilon$) using \ped, \cised($\epsilon$) using \sed, and \tpa and \interval using \dad are approximately $103\%$--$124\%$, $102\%$--$115\%$ and $102\%$--$107\%$ of the optimal algorithms, respectively. Batch algorithms using \ped and \sed also have good compression ratios. Thus, they are the alternates of the optimal algorithms.
%
More specifically, in case compression ratios are the first consideration, then algorithms \bqsa and \siped($\epsilon$) using \ped, \cised($\epsilon$) using \sed, and \tpa and \interval using \dad are good candidates.
%
In case average errors are concerned, then batch algorithms are good candidates as one-pass algorithms and online algorithms \opwa and \bqsa all have large average errors. % and algorithm \squishe has really limited compression ratios
%
In case running time is the important factor or computing resources are limited, then one-pass algorithms are the best candidates.
%
Indeed, one-pass algorithms run fast and require few resources, and have comparable compression ratios compared with batch and online algorithms. Hence, they are prominent trajectory compression algorithms when average errors are not the main concern.

\stitle{(2) Choice of distance metrics}. Users essentially choose a distance metric of \ped, \sed and \dad from the needs of applications. Further, the choice of a distance metric has impacts on the performance.
For compression ratios, the using of synchronized distance \sed saves temporal information of trajectories with a cost of approximately double-sized outputs compared with using \ped in all datasets; \ped has obvious better compression ratios than \dad in all datasets and \sed is also better than \dad in high sampling datasets.
%For average error, algorithms using \dad will bring larger position error than algorithms using \ped and \sed.
For efficiency, the computation time of \ped and \sed are 2.3 and 1.7 times of \dad, respectively.
%\sed and \dad in batch algorithms \dpa and \tpa run a bit faster than \ped, and \sed enabled one-pass algorithm \cised runs a bit slower than \ped (or \dad) enabled one-pass algorithm \siped and \operb (or \interval).


%In the future of trajectory simplification, there are some work worthy of exploration:
%(1) The combination of trajectory simplification algorithms with other algorithms, \eg Map-Matching and Position Tracking, and
%(2) The impacts of simplified trajectory on trajectory data mining and trajectory data services.

%
%(1) The effectiveness of sub-optimal algorithms remains a small room to be improved as their outputs are approximately $120\%$--$150\%$ of the optimal algorithms,
%
%(3) The combination of trajectory simplification algorithms with other fundamental algorithms, such as location track, event detection, \etc.
