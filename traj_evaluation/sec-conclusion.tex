%%%%%%%%%%%%%%%%%%%%%%%%%%%%%%%%%%%%%%%%%%%%%%%%%%%%%%%%%%%%%%%%%%%%%%%%%%%%%%
\section{Conclusions and Future Work}
%%%%%%%%%%%%%%%%%%%%%%%%%%%%%%%%%%%%%%%%%%%%%%%%%%%%%%%%%%%%%%%%%%%%%%%%%%%%%%


\eat{
We have proposed \operb and \operba, two one-pass error bounded trajectory simplification algorithms.
%
First, we have developed a novel local distance checking approach, based on which we then have designed \operb, together with optimization techniques for improving its compression ratio.
%
Second, by allowing interpolating new data points into a trajectory under certain conditions, we have developed an aggressive one-pass error bounded trajectory simplification algorithm \operba, which has significantly improved the compression ratio.
%
Finally, we have experimentally verified that both \operaa and \operab are much faster than \fbqsa, the fastest existing \lsa algorithm,
and in terms of compression ratio, \operaa is comparable with \dpa, and \operab is better than \dpa on average, the existing \lsa algorithm with the best compression ratio.

A couple of issues need further study. We are to incorporate semantic based methods and to study alternative forms of fitting functions to further improve the effectiveness of trajectory compression.
}