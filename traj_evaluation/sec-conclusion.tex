\vspace{-1ex}
%%%%%%%%%%%%%%%%%%%%%%%%%%%%%%%%%%%%%%%%%%%%%%%%%%%%%%%%%%%%%%%%%%%%%%%%%%%%%%
\section{Conclusions}
%%%%%%%%%%%%%%%%%%%%%%%%%%%%%%%%%%%%%%%%%%%%%%%%%%%%%%%%%%%%%%%%%%%%%%%%%%%%%%

We have evaluated the state-of-the-art \lsa algorithms for trajectory compression, including \emph{both the optimal and the sub-optimal methods that use either \ped or \sed or \dad}. 
Using a variety of real trajectory datasets, we evaluated the performance of distinct techniques in terms of its processing time, compression ratio and average error.
Our experimental results show that: 

(1) First of all, users are freely to choose distance metric \sed or \ped or \dad, depending on application requirements. Nevertheless, the selection of distance metric would bring difference performance. The using of synchronized distance \sed saves temporal information of trajectories with a cost of approximately double-sized (\ie $200\%$) outputs compared with using \ped, in all datasets and all algorithms; \ped has obvious better compression ratios than \dad in all datasets and \sed is also better than \dad in high sampling datasets. For efficiency, \sed and \dad in batch algorithms \dpa and \tpa run a bit faster than \ped, and \sed enabled one-pass algorithm \cised runs a bit slower than \ped (\dad) enabled one-pass algorithm \siped and \operb (\interval).

(2) The optimal algorithms bring the best compression ratios, however, their efficciecies are more than tens (hundreds) of times poorer than sub-optimal (one-pass) algorithms. It seems that the optimal algorithms are impractical, especially in case that the input data set is large or computing resources are limited.

(3) When using \ped or \sed, the output sizes of evaluated sub-optimal algorithms, except \squishe, are comparable, and they are approximately $120\%$--$150\%$ of the optimal algorithm; when using \dad, the output sizes of sub-optimal algorithms \tpa and \interval are comparable, and they are approximately $101\%$--$107\%$ of the optimal algorithm. In all cases, the one-pass algorithms \siped, \operb, \cised and \interval are tens of times faster than the batch and online algorithms \tpa, \dpa and \bqsa, and more than $2$ times faster than online algorithm \squishe. Indeed, one-pass algorithms run fast and still have comparable compression ratios with batch algorithms, thus, they are more suitable for trajectory compression, especially on resource constraint mobile devices.

In the future of trajectory simplification, there may be some work worthy of exploration:

(1) The efficiency of the optimal algorithm using \sed is very poor and it is impossible to run it for a a long trajectory. Thus, a fast and resource-optimized optimal algorithm using \sed may be valuable.

(2) The effectiveness of sub-optimal algorithms remains a small room to be improved as their outputs are approximately $120\%$--$150\%$ of the optimal algorithms.

(3) The combination of trajectory simplificatoin algorithms and the self-adaptation of algorithm parameters.

%(3) The combination of trajectory simplification algorithms with other fundamental algorithms, such as location track, event detection, \etc.