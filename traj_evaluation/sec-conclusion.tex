%%%%%%%%%%%%%%%%%%%%%%%%%%%%%%%%%%%%%%%%%%%%%%%%%%%%%%%%%%%%%%%%%%%%%%%%%%%%%%
\section{Conclusions}
%%%%%%%%%%%%%%%%%%%%%%%%%%%%%%%%%%%%%%%%%%%%%%%%%%%%%%%%%%%%%%%%%%%%%%%%%%%%%%


We have evaluated the state-of-the-art \lsa algorithms for trajectory compression, including \emph{both the optimal and the sub-optimal methods that use either \ped or \sed}. 
The algorithms reviewed and evaluated in our experiments are 
(i) Douglas-Peucker\cite{Douglas:Peucker,Meratnia:Spatiotemporal} and \pavlidis~\cite{Pavlidis:Segment}, two distinct batch \lsa algorithms,
(ii) \bqsa\cite{Liu:BQS} and \squishe~\cite{Muckell:SQUISH}, two famous online \lsa algorithms,
(iii) \operb\cite{Lin:Operb}, sector intersection \cite{Williams:Longest,Sklansky:Cone,Dunham:Cone, Zhao:Sleeve} and \cised \cite{Lin:Cised}, three one-pass \lsa algorithms, and
(iv) \opt\cite{Chan:Optimal} and \nopts, an optimal and a near optimal \lsa algorithms using \ped and \sed, respectively.
%
Using a variety of real trajectory datasets, we evaluated the performance of each technique in terms of its processing time, compression ratio and average error.
Our experimental results reveal the characteristics of different techniques, based on which we provide guidelines on selecting appropriate methods and distance metrics for various scenarios.