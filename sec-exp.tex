\newcommand{\dblp}{{\sc Dblp}\xspace}
\newcommand{\hosp}{{\sc Hosp}\xspace}
\newcommand{\hcahps}{{\sc hcahps}\xspace}



\section{Experimental Study}
\label{sec-exp}
%\vspace{-1ex}

We next present an experimental study of \pCFDs and \pCINDs. Using real-life data, we conducted two sets of experiments to evaluate the efficiency and effectiveness of \pCFDs and \pCINDs vs. their counterparts \CFDs and \CINDs, separately and taken together.

\subsection{Experimental Settings}
%\vspace{-0.5ex}
We first present our experimental settings.

\stitle{Datasets}. We used two real-life datasets that were stored in an SQL Server 2012 database.

\ni(1) \hosp (Hospital Compare) is a database publicly available from U.S. Department of Health %\at{HCAHPS} and \at{HCAHPS}-\at{State}
\& Human Services~\cite{hosp}. We used two tables \at{hcahps} and \at{hcahps}-\at{state}, which record the hospital level and state level ratings of the Hospital Consumer Assessment of Healthcare Providers and Systems (HCAHPS), respectively.
%
For table \at{hcahps}, it records (a) the hospital information: \at{hid} (hospital ID), \at{hname} (hospital name), \at{addr} (address), \at{city}, \at{state}, \at{zip}, \at{county}, \at{phn} (phone number), and (b) the measure information: \at{mid} (measure ID), \at{mq} (question), \at{mad} (answer description), \at{map} (answer percent), \at{mncs} (number of completed surveys), \at{msrrp} (survey response rate percent), \at{mfn} (footnote).
%
And for table \at{hcahps}-\at{state}, it records state level measure information: \at{state}, \at{mid}, \at{mq} and \at{map}, among other things.


We designed 6 \pCFDs and 3 \pCINDs for \hosp, shown below in an informal way for easy of understanding.

\begin{footnotesize}\mat{0ex}{
$\varphi _1 $: \= \at{hcahps}(\at{zip} = `\_' \And \at{city} = `\_' $\rightarrow$ \at{state} = `\_') \\
$\varphi _2 $: \at{hcahps}(\at{hid} = `\_' $\rightarrow$ \at{hname} = `\_' \And \at{county} = `\_' \And \at{addr} = `\_' \And \\
\>  \at{phn}=  `\_') \\
$\varphi _3 $: \at{hcahps}(\at{hid} = `\_' $\rightarrow$ \at{msrrp} = `\_') \\
$\varphi _4 $: \at{hcahps}(\at{hid} = `\_' $\rightarrow$ \at{mq} = `\_' \And \at{mad} = '\_') \\
$\varphi _5 $: \at{hcahps}(\at{hid} = `\_' \And \at{mncs} = `Not Available' $\rightarrow$ \at{mfn} $\geq 1$ \And \at{mfn} $\leq 14$) \\
$\varphi _6 $: \at{hcahps}(\at{hid} = `\_' \And \at{mid} = `\_' \And \at{mncs} $\neq$ `Not Available' \And \\
\> \at{mncs} $\neq$ `Fewer than 100' $\rightarrow$ \at{map} $\geq 0$ \And \at{map} $\leq 100$) \\
$\psi _1 $: \at{hcahps}(\at{mid}, \at{mq}; \at{nil}) $\subseteq$ \at{hcahps}-\at{state}(\at{mid}, \at{mq}; \at{nil}) \\
$\psi _2 $: \at{hcahps}(\at{mid}, \at{state}; \at{nil}) $\subseteq$ \at{hcahps}-\at{state}(\at{mid}, \at{state}; \at{nil}) \\
$\psi _3 $: \at{hcahps}(\at{mid}, \at{state}; \at{mncs}$\neq$ 'Not Available') $\subseteq$ \at{hcahps}-\at{state}(\at{mid}, \at{state}; \\
\> \at{map} $\geq 0$ \And \at{map} $\leq 100$)}
\end{footnotesize}

For comparison, we also designed the \CFDs and \CINDs counterparts of the above \pCFDs and \pCINDs. Here
$\varphi_1$-$\varphi_4 $ and $\psi_1 $-$\psi_2 $ are indeed \CFDs and \CINDs, respectively, while $\varphi_5$, $\varphi_6$ and $\psi_3$ cannot.
We hence further designed $\varphi_5'$, $\varphi_6'$ and $\psi_3'$ to approximate $\varphi_5$, $\varphi_6$ and $\psi_3$, respectively.

\begin{footnotesize}\mat{0ex}{
$\varphi_5'$: \=\at{hcahps}(\at{hid} = `\_' \And \at{mncs} = `\_' $\rightarrow$ \at{mfn} = `\_') \\
$\varphi_6'$: \at{hcahps}(\at{hid} = `\_' \And \at{mid} = `\_' \And \at{mncs} = `\_' $\rightarrow$ \at{map} = `\_') \\
$\psi_3'$: \at{hcahps}(\at{mid}, \at{state}; \at{mncs} = `\_') $\subseteq$ \at{hcahps}-\at{state}(\at{mid}, \at{state}; \at{map} = `\_')
}
\end{footnotesize}


\ni(2) \dblp is a repository of computer science publications from 1946 to 2014~\cite{dblp}.
 We further transformed its XML format into two tables  \at{paper} and \at{proceeding} that record the paper and proceeding information, respectively,
such that \at{paper}(\at{key}, \at{type}, \at{title}, \at{booktitle}, \at{year}, \at{crossref}, \at{isbn}, \at{publisher}) records books, journal articles and conference papers, and \at{proceeding}(\at{key}, \at{year}, \at{isbn}, \at{publisher}) records the proceedings of conference papers, respectively.

We generated 3482 \pCFDs and 2568 \pCINDs for the DBLP data, with their representatives shown below.

\begin{footnotesize}\mat{0ex}{
$\phi_1$:  \= \at{paper}(\at{isbn} = '\_' \And \at{booktitle} = '\_' $\rightarrow$ \at{publisher} = '\_') \\
$\phi_2$: \at{paper}(\at{title} = '\_' \And \at{year} = '\_' \And \at{booktitle} = '\_' $\rightarrow$ \at{type} = '\_')  \\
$\phi_3$: \at{paper}(\at{booktitle} = 'CleanDB' $\rightarrow$ \at{year} = 2006) \\
$\phi_4$: \at{paper}(\at{booktitle} = 'VLDB' \And \at{year_L} = '\_' $\rightarrow$ \at{year_R} $\geq$ 1975 \\
\> \at{year_R}  $\leq$ 2007) \\
$\phi_5$: \at{paper}(\at{booktitle} = 'PVLDB' \And \at{year_L} = '\_' $\rightarrow$ \at{year_R} $\geq$ 2008) \\
$\rho_1$: \at{paper}(\at{crossref}, \at{isbn}, \at{publisher}; \at{type} = 'inproceedings' \And \\
\>  \at{booktitle} = 'CIKM-CNIKM') $\subseteq$ \at{proceeding}(\at{key}, \at{isbn}, \at{publisher}; \\
\>  \at{year} = 2009)  \\
$\rho_2$: \at{paper}(\at{crossref}, \at{isbn}, \at{publisher}; \at{type} = 'inproceedings' \And \\
\>  \at{booktitle} = 'VLDB') $\subseteq$ \at{proceeding}(\at{key}, \at{isbn}, \at{publisher}; \\
\>  \at{year} $\geq$ 1975 \And \at{year} $\leq$ 2007)  \\
$\rho_3$: \at{paper}(\at{crossref}, \at{isbn}, \at{publisher}; \at{type} = 'inproceedings' \And \\
\> \at{booktitle} = 'ICDE') $\subseteq$ \at{proceeding}(\at{key}, \at{isbn}, \at{publisher}; \at{year} $\geq$ 1984)
}
\end{footnotesize}

We collected all the \at{booktitle} and corresponding \at{year} from \dblp to generate the other \pCFDs and \pCINDs by instantiating the values of their \at{booktitle} and \at{year} attributes. Observe that $\phi_1$-$\phi_3$ and $\rho _1$ are \CFDs and \CINDs, respectively. For comparison,  we further designed the following \CFDs and \CINDs to approximate $\phi_4$-$\phi_5$ and $\rho_2$-$\rho_3$.

\begin{footnotesize}\mat{0ex}{
$\phi_4'$: \=\at{paper}(\at{booktitle} = 'VLDB' \And \at{year_L} = '\_' $\rightarrow$ \at{year_R} = '\_') \\
$\phi_5'$: \at{paper}(\at{booktitle} = 'PVLDB' \And \at{year_L} = '\_' $\rightarrow$ \at{year_R} = '\_') \\
$\rho_2'$: \at{paper}(\at{crossref}, \at{isbn}, \at{publisher}; \at{type}$=$'inproceedings' \And \\
\> \at{booktitle} = 'VLDB') $\subseteq$ \at{proceeding}(\at{key}, \at{isbn}, \at{publisher}; \at{year} = '\_')\\
$\rho_3'$: \at{paper}(\at{crossref}, \at{isbn}, \at{publisher}; \at{type} = 'inproceedings' \And \\
\> \at{booktitle} = 'ICDE') $\subseteq$ \at{proceeding}(\at{key}, \at{isbn}, \at{publisher}; \at{year} = '\_')
}
\end{footnotesize}


\stitle{Implementation}. All the experiments were run with an SQL Server 2012 database installed on a machine with an Intel Core i5 (3.1GHz) CPU and 8GB of RAM. Each test was repeated 5 times, and the average is reported here.

\subsection{Experimental Results}

%Efficiency of detecting CFDP violations
\begin{figure*}[tb!]
  \centering
  \subfigure[Varying $|I_{1}|$ for HOSP]{\label{fig_exp1_cfdp_hosp_I1}\includegraphics[scale=1]{./exp-fig/11a.eps}}  %, height=1.5in, width=1.5in}}
   \quad
  \subfigure[Varying $noise\%$ for \hosp]{\label{fig_exp1_cfdp_hosp_noise}\includegraphics[scale=1]{./exp-fig/11b.eps}}
  \quad
  \subfigure[Varying $|I_{1}|$ for \dblp]{\label{fig_exp1_cfdp_dblp_I1}\includegraphics[scale=1]{./exp-fig/11c.eps}}
  \quad
  \subfigure[Varying $noise\%$ for \dblp]{\label{fig_exp1_cfdp_dblp_noise}\includegraphics[scale=1]{./exp-fig/11d.eps}}
  \caption{Efficiency of detecting \pCFD violations}\label{fig_exp1_cfdp}
\end{figure*}

%Efficiency of detecting CINDP violations
\begin{figure*}[tb!]
  \centering
  \subfigure[Varying $|I_{1}|$ for \hosp]{\label{fig_exp1_cindp_hosp_I1}\includegraphics[scale=1]{./exp-fig/12a.eps}}
  \quad
  \subfigure[Varying $|I_{2}|$ for \hosp]{\label{fig_exp1_cindp_hosp_I2}\includegraphics[scale=1]{./exp-fig/12b.eps}}
  \quad
  \subfigure[Varying $noise\%$ for \hosp]{\label{fig_exp1_cindp_hosp_noise}\includegraphics[scale=1]{./exp-fig/12c.eps}}
  \quad \\
  \subfigure[Varying $|I_{1}|$ for \dblp]{\label{fig_exp1_cindp_dblp_I1}\includegraphics[scale=1]{./exp-fig/12d.eps}}
  \quad
  \subfigure[Varying $|I_{2}|$ for \dblp]{\label{fig_exp1_cindp_dblp_I2}\includegraphics[scale=1]{./exp-fig/12e.eps}}
  \quad
  \subfigure[Varying $noise\%$ for \dblp]{\label{fig_exp1_cindp_dblp_noise}\includegraphics[scale=1]{./exp-fig/12f.eps}}
  \caption{Efficiency of detecting \pCIND violations}\label{fig_exp1_cindp}
\end{figure*}

%Efficiency of detecting CFDP and CINDP violations
\begin{figure*}[tb!]
  \centering
  \subfigure[Varying $|I_{1}|$ for \hosp]{\epsfig{file=exp-fig/13a.eps}}
  \quad
  \subfigure[Varying $|I_{2}|$ for \hosp]{\epsfig{file=exp-fig/13b.eps}}
  \quad
  \subfigure[Varying $noise\%$ for \hosp]{\epsfig{file=exp-fig/13c.eps}}
  \quad \\
  \subfigure[Varying $|I_{1}|$ for \dblp]{\epsfig{file=exp-fig/13d.eps}}
  \quad
  \subfigure[Varying $|I_{2}|$ for \dblp]{\epsfig{file=exp-fig/13e.eps}}
  \quad
  \subfigure[Varying $noise\%$ for \dblp]{\epsfig{file=exp-fig/13f.eps}}

  \caption{Efficiency of detecting \pCFD and \pCIND violations}\label{fig_exp1_both}
\end{figure*}

We next present our findings.
%
Three parameters were used in our tests:
(1) $|I_1|$, the number of tuples in table \at{hcahps} of \hosp or \at{paper} of \dblp,
(2) $|I_2|$, the number of tuples in table \at{hcahps}-\at{state} of \hosp or in \at{proceeding} of \dblp, and
(3) $noise\%$, the percentage of dirty tuples in table \at{hcahps} of \hosp or  \at{paper} of \dblp, ranging from 0\% to 9\%.
A clean copy of \hosp and \dblp is also kept to tell whether a detected tuple is dirty or not.



%\subsubsection{Efficiency Tests}

\stitle{Exp-1: Efficiency}. In the first set of experiments, we evaluated the violation detection efficiency of  \pCFDs and \pCINDs vs. their counterparts \pCFDs and \pCINDs, separately and taken together.


\setitle{Exp-1.1: \pCFDs vs. \CFDs}.

\ni(1) To evaluate the impacts of $|I_1|$, we fixed $noise\% = 9\%$, and varied $|I_1|$ (a) from  10K to 90K for \hosp and (b) from 100K to 900K for \dblp, respectively. The results are reported in Figures~\ref{fig_exp1_cfdp_hosp_I1} and \ref{fig_exp1_cfdp_dblp_I1}, respectively.


\ni(2) To evaluate the impacts of $noise\%$, we fixed $|I_1|$ (a) to  $90K$ for \hosp and (b) to $900K$ for \dblp, respectively, and varied  $noise\%$ from 0\% to 9\%. The results are reported in Figures~\ref{fig_exp1_cfdp_hosp_noise} and \ref{fig_exp1_cfdp_dblp_noise}, respectively.



\setitle{Exp-1.2: \pCINDs vs. \CINDs}.


\ni(1) To evaluate the impacts of $|I_1|$, we fixed $noise\%$ = $9\%$ and $|I_2|$ = $1.6K$ for \hosp (resp. $16K$ for \dblp), and varied $|I_1|$ from  $10K$ to $90K$ for \hosp (resp. from $100K$ to $900K$ for \dblp). The results are reported in Figures~\ref{fig_exp1_cindp_hosp_I1} and \ref{fig_exp1_cindp_dblp_I1}, respectively.

\ni(2) To evaluate the impacts of $|I_2|$, we fixed $noise\%$ = $9\%$ and $|I_1|$ = $90K$ for \hosp (resp. $900K$ for \dblp), and varied $|I_2|$ from  $1K$ to $1.6K$ for \hosp (resp. from $10K$ to $16K$ for \dblp). The results are reported in Figures~\ref{fig_exp1_cindp_hosp_I2} and \ref{fig_exp1_cindp_dblp_I2}, respectively.


\ni(3) To evaluate the impacts of $noise\%$, we fixed $|I_1|$ = $90K$ for \hosp (resp. $900K$ for \dblp) and $|I_2|$ = $1.6K$ for \hosp (resp. $16K$ for \dblp), and varied  $noise\%$ from 0\% to 9\%. The results are reported in Figures~\ref{fig_exp1_cindp_hosp_noise} and \ref{fig_exp1_cindp_dblp_noise}, respectively.





\setitle{Exp-1.3:  \pCFDs + \pCINDs vs. \CFDs + \CINDs}.

Using the same setting as \setitle{Exp-1.3}







\stitle{Exp-2: Effectiveness}. In the second set of experiments, we evaluated the violation detection effectiveness of  \pCFDs and \pCINDs vs. their counterparts \pCFDs and \pCINDs, separately and taken together. Note that varying $|I_{2}|$ has no impacts on the effectiveness tests in our setting.


For each dirty tuple $t$, an attribute of $t$ in $Y$ of a \pCFD or $X \cup Y_p$ of a \pCIND was changed from a correct to incorrect value.


%Effectiveness of detecting CFDP violations
\begin{figure*}[tb!]
  \centering
  % Requires \usepackage{graphicx}
  \centering
  \subfigure[Varying $|I_{1}|$ for \hosp]{\epsfig{file=exp-fig/21a.eps}}  %, height=1.5in, width=1.5in}}
  \quad
  \subfigure[Varying $noise\%$ for \hosp]{\epsfig{file=exp-fig/21b.eps}}
  \quad
  \subfigure[Varying $|I_{1}|$ for \dblp]{\epsfig{file=exp-fig/21c.eps}}
  \quad
  \subfigure[Varying $noise\%$ for \dblp]{\epsfig{file=exp-fig/21d.eps}}
  \caption{Effectiveness of detecting \pCFD violations}\label{fig_exp2_cfdp}
\end{figure*}

\noindent Figure~\ref{fig_exp2_cfdp}(a): Fixing $noise\% = 9\%$, we varied $|I_1|$ from 10K to 90K.    \\
Figure~\ref{fig_exp2_cfdp}(b): Fixing $|I_1|=90K$, we varied $noise\%$ from 0\% to 9\%.\\
Figure~\ref{fig_exp2_cfdp}(c): Fixing $noise\% = 9\%$, we varied $|I_1|$ from 100K to 900K.  \\
Figure~\ref{fig_exp2_cfdp}(d): Fixing $|I_1|=900K$, we varied $noise\%$ from 0\% to 9\%.  \\


%Effectiveness of detecting CINDP violations
\begin{figure*}[tb!]
  \centering
  % Requires \usepackage{graphicx}
  \centering
  \subfigure[Varying $|I_{1}|$ for \hosp]{\epsfig{file=exp-fig/22a.eps}}  %, height=1.5in, width=1.5in}}
  \quad
  \subfigure[Varying $noise\%$ for \hosp]{\epsfig{file=exp-fig/22b.eps}}
  \quad
  \subfigure[Varying $|I_{1}|$ for \dblp]{\epsfig{file=exp-fig/22c.eps}}
  \quad
  \subfigure[Varying $noise\%$ for \dblp]{\epsfig{file=exp-fig/22d.eps}}
  \caption{Effectiveness of detecting \pCIND violations}\label{fig_exp2_cindp}
\end{figure*}

\noindent Figure~\ref{fig_exp2_cindp}(a): Fixing $noise\% = 9\%$ and $|I_2| = 1.6K$, we varied $|I_1|$ from 10K to 90K.\\
Figure~\ref{fig_exp2_cindp}(b): Fixing $|I_1| = 90K$ and $|I_2| = 1.6K$, we varied $noise\%$ from 0\% to 9\%.\\
Figure~\ref{fig_exp2_cindp}(c): Fixing $noise\% = 9\%$ and $|I_2| = 16K$, we varied $|I_1|$ from 100K to 900K.\\
Figure~\ref{fig_exp2_cindp}(d): Fixing $|I_1| = 900K$ and $|I_2| = 16K$, we varied $noise\%$ from 0\% to 9\%.\\



%Effectiveness of detecting CFDP and CINDP violations
\begin{figure*}[tb!]
  \centering
  % Requires \usepackage{graphicx}
  \subfigure[Varying $I_{1}|$ for \hosp]{\epsfig{file=exp-fig/23a.eps}}  %, height=1.5in, width=1.5in}}
  \quad
  \subfigure[Varying $noise\%$ for \hosp]{\epsfig{file=exp-fig/23b.eps}}
  \quad
  \subfigure[Varying $I_{1}|$ for \dblp]{\epsfig{file=exp-fig/23c.eps}}
  \quad
  \subfigure[Varying $noise\%$ for \dblp]{\epsfig{file=exp-fig/23d.eps}}
  \caption{Effectiveness of detecting \pCFD and \pCIND violations}\label{fig_exp2_both}
\end{figure*}

\noindent Figure~\ref{fig_exp2_both}(a): Fixing $noise\% = 9\%$ and $|I_2| = 1.6K$, we varied $|I_1|$ from 10K to 90K.\\
Figure~\ref{fig_exp2_both}(b): Fixing $|I_1| = 90K$ and $|I_2| = 1.6K$, we varied $noise\%$ from 0\% to 9\%.\\
Figure~\ref{fig_exp2_both}(c): Fixing $noise\% = 9\%$ and $|I_2| = 16K$, we varied $|I_1|$ from 100K to 900K.\\
Figure~\ref{fig_exp2_both}(d): Fixing $|I_1| = 900K$ and $|I_2| = 16K$, we varied $noise\%$ from 0\% to 9\%.\\



\stitle{Summary}. The experimental results show the followings.



%Accuracy of detecting CFDP violations
\begin{figure*}
  \centering
  % Requires \usepackage{graphicx}
  \centering
  \subfigure[Varying $|I_{1}|$ for HOSP]{\epsfig{file=exp-fig/31b.eps}}
  \quad
  \subfigure[Varying $|I_{1}|$ for HOSP]{\epsfig{file=exp-fig/31a.eps}}
  \quad
  \subfigure[Varying $noise\%$ for HOSP]{\epsfig{file=exp-fig/31d.eps}}
  \quad
  \subfigure[Varying $noise\%$ for HOSP]{\epsfig{file=exp-fig/31c.eps}}
  \quad
  \subfigure[Varying $|I_{1}|$ for DBLP]{\epsfig{file=exp-fig/31f.eps}}
  \quad
  \subfigure[Varying $|I_{1}|$ for DBLP]{\epsfig{file=exp-fig/31e.eps}}
  \quad
  \subfigure[Varying $noise\%$ for DBLP]{\epsfig{file=exp-fig/31h.eps}}
  \quad
  \subfigure[Varying $noise\%$ for DBLP]{\epsfig{file=exp-fig/31g.eps}}
  \caption{Accuracy of detecting \pCFDs and \pCINDs violations}\label{fig_exp2_accuracy}
\end{figure*}


%%%%%%%% end duan  2014-11-17



%%%%%%%% begin duan 2014-11-17
\begin{table*}[tbh!]
%\vspace{-2ex}
 \caption{Effectiveness of detecting \pCFD and \pCIND violations\label{tab-effectiveness}}
 \vspace{-3ex}
\begin{center}
\begin{small}
\begin{tabular}{|c|c|c|c|c|c|c|c|c|}
\hline
Dataset & $|I_1|$ & $noise\%$ & \CFDs  & \pCFDs   & \CINDs & \pCINDs & \CFDs + \CINDs & \pCFDs + \pCINDs  \\
\hline
\hosp & 10K-90K & 9\% & 0.748-0.777 & 0.855-0.888 & 0.321-0.333 & 0.446-0.489 & 0.748-0.777 & 1.0 \\
\hline
\hosp & 90K & 1\%-9\% & 0.766-0.777 & 0.875-0.888 & 0.328-0.333 & 0.445-0.472 & 0.766-0.777 & 1.0 \\
\hline
\dblp & 100K-900K & 9\% & 0.126-0.176 & 0.625-0.645 & 0.248-0.254 & 0.496-0.509 & 0.25-0.295 & 1.0 \\
\hline
\dblp & 900K & 1\%-9\% & 0.153-0.201 & 0.625-0.645 & 0.249-0.263 & 0.5-0.511 & 0.28-0.319 & 1.0 \\
\hline
\end{tabular}
\end{small}
\end{center}
\vspace{-5ex}
\end{table*}

%%%%%%%% end duan  2014-11-17







