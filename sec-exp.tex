%\vspace{-1ex}
\section{Experimental Study}
\label{sec-exp}
%\vspace{-1ex}

We next present an experimental study by using real life data. Two sets of experiments
were conducted to verify the efficiency and effectiveness of our techniques for detecting \pCFDs and \pCINDs
violations.

\stitle{Experimental data}. Real-life datasets were employed to examine the applicability of our method in practice.

(1) HOSP \textit{(Hospital Compare)} is publicly available from U.S. Department of Health
\& Human Services\footnote{https://data.medicare.gov/data/hospital-compare}. We used two tables: HCAHPS-Hospital and HCAHPS-State, which record the survey responses and the state average of survey responses of hospital patients about their experiences during a recent inpatient hospital stay, respectively. We chose 15 attributes for the schema of the relation $R_1$: \at{pid} (Provider ID), \at{hname} (Hospital Name), \at{addr} (Address), \at{city}, \at{state}, \at{zip}, \at{county}, \at{phn} (Phone Number), \at{hid} (HCAHPS Measure ID), \at{hq} (HCAHPS Question), \at{had} (HCAHPS Answer Description), \at{hap} (HCAHPS Answer Percent), \at{ncs} (Number of Completed Surveys), \at{srrp} (Survey Response Rate Percent) and \at{ft} (Footnote), and 4 attributes for the schema of the relation $R_2$: \at{state}, \at{hid}, \at{hq} and \at{hap}.

We designed 6 \pCFDs and 3 \pCINDs for the HOSP data, shown below.

\begin{footnotesize}\mat{0ex}{
$\varphi _1$: \=$R_1$(\at{zip}$=$'\_' \And \at{city}$=$'\_' $\rightarrow$ \at{state}$=$'\_') \\
$\varphi _2$: $R_1$(\at{pid}$=$'\_' $\rightarrow$ \at{hname}$=$'\_' \And \at{county}$=$'\_' \And \at{addr}$=$'\_' \\
\> \And \at{phn}$=$'\_') \\
$\varphi _3$: $R_1$(\at{pid}$=$'\_' $\rightarrow$ \at{srrp}$=$'\_') \\
$\varphi _4$: $R_1$(\at{hid}$=$'\_' $\rightarrow$ \at{hq}$=$'\_' \And \at{had}$=$'\_') \\
$\varphi _5$: $R_1$(\at{pid}$=$'\_' \And \at{ncs}$=$'Not Available' $\rightarrow$ \at{ft}$\geq 1$ \And \at{ft}$\leq 14$) \\
$\varphi _6$: $R_1$(\at{pid}$=$'\_' \And \at{hid}$=$'\_' \And \at{ncs}$\neq$'Not Available' \And \at{ncs}\\
\> $\neq$'Fewer than 100' $\rightarrow$ \at{hap}$\geq 0$ \And \at{hap}$\leq 100$) \\
$\psi _1$: $R_1$(\at{hid}, \at{hq}; \at{nil}) $\subseteq$ $R_2$(\at{hid}, \at{hq}; \at{nil}) \\
$\psi _2$: $R_1$(\at{hid}, \at{state}; \at{nil}) $\subseteq$ $R_2$(\at{hid}, \at{state}; \at{nil}) \\
$\psi _3$: $R_1$(\at{hid}, \at{state}; \at{ncs}$\neq$'Not Available') $\subseteq$ $R_2$(\at{hid}, \at{state}; \at{hap}$\geq 0$\\
\> \And \at{hap}$\leq 100$) }
\end{footnotesize}

(2) DBLP \textit{data} is from the DBLP Bibliography\footnote{http://dblp.uni-trier.de/xml/}. We first transformed the XML data into relations, and then we created two tables for \textit{publications}, including \textit{book}, \textit{article} (journals papers) and \textit{inproceedings} (conferences papers), and \textit{proceedings} (conferences) respectively. We chose 8 attributes for the schema of relation $R_1$: \at{key}, \at{type}, \at{title}, \at{booktitle}, \at{year}, \at{crossref}, \at{isbn} and \at{publisher}, and 4 attributes for the schema of $R_2$: \at{key}, \at{year}, \at{isbn} and \at{publisher}.


We designed 771 \pCFDs and 255 \pCINDs for the DBLP data, with nine representative ones as follows:\\

\begin{footnotesize}\mat{0ex}{
$\varphi _1$: \=$R_1$(\at{isbn}$=$'\_' \And \at{booktitle}$=$'\_' $\rightarrow$ \at{publisher}$=$'\_') \\
$\varphi _2$: $R_1$(\at{title}$=$'\_' \And \at{year}$=$'\_' \And \at{booktitle}$=$'\_' $\rightarrow$ \at{type}$=$'\_') \\
$\varphi _3$: $R_1$(\at{booktitle}$=$'AICT/ICIW' $\rightarrow$ \at{year}$=$2006) \\
$\varphi _4$: $R_1$(\at{booktitle}$=$'VLDB' \And \at{year_L}$=$'\_' $\rightarrow$ \at{year_R}$\geq 1975$ \\
\> \And \at{year_R}$\leq 2007$) \\
$\varphi _5$: $R_1$(\at{booktitle}$=$'PVLDB' \And \at{year_L}$=$'\_' $\rightarrow$ \at{year_R}$\geq 2008$) \\
$\psi _1$: $R_1$(\at{crossref}, \at{publisher}; \at{type}$=$'inproceedings') $\subseteq$ $R_2$(\at{key}, \\
\> \at{publisher}; \at{nil}) \\
$\psi _2$: $R_1$(\at{crossref}, \at{isbn}, \at{publisher}; \at{booktitle}$=$'ACST') $\subseteq$ $R_2$(\at{key},\\
\> \at{isbn}, \at{publisher}; \at{year}$=$2006)  \\
$\psi _3$: $R_1$(\at{crossref}, \at{isbn}, \at{publisher}; \at{booktitle}$=$'VLDB') $\subseteq$ $R_2$(\at{key},\\
\> \at{isbn}, \at{publisher}; \at{year}$\geq 1975$ \And \at{year}$\leq 2007$)  \\
$\psi _4$: $R_1$(\at{crossref}, \at{isbn}, \at{publisher}; \at{booktitle}$=$'ICDE') $\subseteq$ $R_2$(\at{key},\\
\> \at{isbn}, \at{publisher}; \at{year}$\geq 1984$)  }
\end{footnotesize}

We varied three parameters of the data in our experiments, denoted by $|I_1|$, $|I_2|$ and $noise\%$. $|I_1|$ is the number of tuples in the instance $I_1$ of relation $R_1$. Similarly, $|I_2|$ is the number of the instance $I_2$ of relation $R_2$. $noise\%$ is the percentage of dirty tuples in $I_1$ that were modified to violate a \pCFD or \pCIND, ranging from 0\% to 9\%. For each dirty tuple $t$, an attribute of $t$ in $Y$ of a \pCFD or $X \cup Y_p$ of a \pCIND was changed from a correct to incorrect value. We kept a copy of clean data before adding any noise to tell whether a tuple detected by \pCFDs or \pCINDs is a true or false dirty tuple.

\stitle{Implementation}. All methods were implemented in C\# and all data was stored in SQL Server 2012. The experiments were run on a machine with an Intel Core i5 (3.1GHz) CPU and 8GB of RAM. Each experiment was repeated 5 times and the average is reported here.

\stitle{Experimental results}. We next present our findings.

\stitle{Exp-1: Efficiency}.

\begin{figure*}
  \centering
  % Requires \usepackage{graphicx}
  \subfigure[Varying $|I_{1}|$]{\epsfig{file=exp-fig/1a.eps}} %, height=1.5in, width=2.4in
  \subfigure[Varying $|I_{1}|$]{\epsfig{file=exp-fig/1b.eps}}
  \subfigure[Varying $|I_{1}|$]{\epsfig{file=exp-fig/1c.eps}}
  \subfigure[Varying $|I_{2}|$]{\epsfig{file=exp-fig/1d.eps}}
  \subfigure[Varying $|I_{2}|$]{\epsfig{file=exp-fig/1e.eps}} \\
  \subfigure[Varying $noise\%$]{\epsfig{file=exp-fig/1f.eps}}
  \subfigure[Varying $noise\%$]{\epsfig{file=exp-fig/1g.eps}}
  \subfigure[Varying $noise$\%]{\epsfig{file=exp-fig/1h.eps}}
  \caption{Efficiency of detecting \pCFD and \pCIND violations}\label{fig_exp1}
\end{figure*}

Figure~\ref{fig_exp1}(a), (b), (c): Fixing $noise\% = 9\%$ and $|I_2| = 1.6K$, we varied $|I_1|$ from 10K to 90K.

Figure~\ref{fig_exp1}(d), (e): Fixing $noise\% = 9\%$ and $|I_1| = 90K$, we varied $|I_2|$ from 1.0K to 1.6K.

Figure~\ref{fig_exp1}(f), (g), (h): Fixing $|I_1| = 90K$ and $|I_2| = 1.6K$, we varied $noise\%$ from 0\% to 9\%.


\stitle{Exp-2: Effectiveness}.

\begin{figure*}
  \centering
  % Requires \usepackage{graphicx}
  \subfigure[Varying $|I_{1}|$]{\epsfig{file=exp-fig/2a.eps}} %, height=1.5in, width=2.4in
  \subfigure[Varying $|I_{1}|$]{\epsfig{file=exp-fig/2b.eps}}
  \subfigure[Varying $|I_{1}|$]{\epsfig{file=exp-fig/2c.eps}}
  \subfigure[Varying $|I_{2}|$]{\epsfig{file=exp-fig/2d.eps}}
  \subfigure[Varying $|I_{2}|$]{\epsfig{file=exp-fig/2e.eps}} \\
  \subfigure[Varying $noise\%$]{\epsfig{file=exp-fig/2f.eps}}
  \subfigure[Varying $noise\%$]{\epsfig{file=exp-fig/2g.eps}}
  \subfigure[Varying $noise$\%]{\epsfig{file=exp-fig/2h.eps}}
  \caption{Effectiveness of detecting \pCFD and \pCIND violations}\label{fig_exp2}
\end{figure*}

Figure~\ref{fig_exp2}(a), (b), (c): Fixing $noise\% = 9\%$ and $|I_2| = 1.6K$, we varied $|I_1|$ from 10K to 90K.

Figure~\ref{fig_exp2}(d), (e): Fixing $noise\% = 9\%$ and $|I_1| = 90K$, we varied $|I_2|$ from 1.0K to 1.6K.

Figure~\ref{fig_exp2}(f), (g), (h): Fixing $|I_1| = 90K$ and $|I_2| = 1.6K$, we varied $noise\%$ from 0\% to 9\%.

\stitle{Summary}. The experimental results show the followings.








