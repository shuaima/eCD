%\vspace{-0.5ex}
\section{Reasoning about CFD$^p$s and CIND$^p$s}
\label{sec-reasoning}
%\vspace{-1ex}

The satisfiability and implication problems are the two
classical questions associated with any dependency languages \cite{AbHuVi1995,CFDs,tcs-CINDs}.
In this section we investigate these
problems for \pCFDs and \pCINDs, separately and taken together.

%%%%%%%%%%%%%%%%%
\subsection{Satisfiability Analyses}

The satisfiability problem is to determine, given a set $\Sigma$ of
constraints, whether there exists a {\em nonempty} database that
satisfies $\Sigma$.

The satisfiability analysis of conditional dependencies is not only
of theoretical interest, but is also important in practice. Indeed,
when \pCFDs and \pCINDs are used as data quality rules, this
analysis helps one check whether the rules make sense themselves.
The need for this is particularly evident when the rules are
manually designed or discovered from various
datasets~\cite{CM08,divesh08,BauckmannALMN12}.

\stitle{Satisfiability analysis of CFD$^p$s}. Given any
\FDs, one does not need to worry about their satisfiability
as any set of \FDs is always satisfiable. However, as observed
in \cite{CFDs}, for a set $\Sigma$ of \CFDs on a relational schema
$R$, there may not exist a {\em nonempty} instance $I$ of $R$ such
that $I\models\Sigma$. As \CFDs are a special case of \pCFDs, the
same problem exists when it comes to \pCFDs.

\vspace{-0.5ex}
\begin{example}
\label{exam-sat-cfd} Consider a \pCFD $\varphi$ = $(R: A\ra
B,\ T_p)$ such that $T_p$ = $\{(\_\pa = a), (\_ \pa \ne a)\}$.
There is no {\em nonempty} instance $I$ of $R$ that
satisfies $\varphi$. Indeed, for any $R$ tuple $t$,
$\varphi$ requires that both $t[B] = a$ and
$t[B]\ne a$, which is impossible.
\end{example}
\vspace{-0.5ex}

This problem is already \NP-complete for \CFDs~\cite{CFDs}.
Below we show that it remains the
same complexity for \pCFDs despite their increased
expressive power.

%\vspace{-0.5ex}
\begin{prop}
\label{thm-sat-pcfd-fin} The satisfiability problem for \pCFDs is
\NP-complete.
\end{prop}

\begin{proof} The lower bound follows from the \NP-hardness of their \CFDs
counterparts~\cite{CFDs}, since \CFDs are a special case of \pCFDs.
The upper bound is verified by presenting an \NP algorithm that,
given a set $\Sigma$ of \pCFDs defined on a relational schema $R$,
determines whether $\Sigma$ is satisfiable.

We next present an \NP
algorithm that, given a set $\Sigma$ of \pCFDs defined on a
relational schema $R$, determines whether $\Sigma$ is satisfiable or not.
The satisfiability problem has the following small model property:
If there is a nonempty $R$ instance $I$ such that
$I\models\Sigma$, then for any tuple $t\in I$, instance $I_t$ =$\{t\}$ satisfies $\Sigma$. Thus it suffices to consider
single-tuple instances $I$ = $\{t\}$ for deciding whether $\Sigma$
is satisfiable.

Assume \kwlog that the attributes $\attrset(R)$ = $\{A_1$, $\dots$,
$A_m\}$ and the total number of pattern tuples in all pattern
tableaux $T_p$ of \pCFDs in $\Sigma$ is $h$. For each $i\in [1, m]$, define
the active domain of $A_i$ to be a set $\adom(A_i)$ = $C_0\cup C_1$,
where (1) $C_0$ consists of all constants in $T_p[A_i]$ of all
pattern tableaux $T_p$ in $\Sigma$, and if $C_0$ is empty, we further let $C_0$ = $\{a_1, a_2\}$, where
$a_1, a_2 \in\dom(A_i)$ and $a_1\ne a_2$, and (2) $C_1$ contains the
 set of constants for the attributes whose domains have
total orders, \ie involved with predicates $\ne$, $<$, $\le$, $>$ or $\ge$:


\bi
\item[(1)] Arrange all constants in $C_0$ in the increasing order, and
assume the resulting $C_0$ = $\{a_1, \ldots, a_k\}$ ($k\ge 1$);

\item[(2)] Add a constant $b_{01}\in\dom(A_i)$ to $C_1$ such that $b_{01}$ $<$ $a_1$ if there
exists one; And also add another constant $b_{02}\in\dom(A_i)$ to $C_1$
such that $b_{02} < a_1$ and $b_{02}\ne b_{01}$ if there exists one;

\item[(3)] Similarly, for each $j\in[1, k - 1]$, add a constant $b_{j1}\in\dom(A_i)$ to $C_1$
such that $a_j < b_{j1} < a_{j+1}$ if there exists
one; And also add another constant $b_{j2}\in\dom(A_i)$ to $C_1$ such
that $a_j < b_{j2} < a_{j+1}$ and $b_{j2}\ne b_{j1}$ if
there exists one;

\item[(4)] Finally, add a constant $b_{k1}\in\dom(A_i)$ to $C_1$ such that $b_{k1} > a_k$ if there exists one;
And also add another constant $b_{k2}\in\dom(A_i)$ to $C_1$ such that
$b_{k2} > a_k$ and $b_{k2} \ne b_{k1}$ if there exists one.
\ei

\noindent Moveover, the number of elements in $\adom(A_i)$ is at
most $3*h + 2$. Then one can easily verify that $\Sigma$ is
satisfiable iff there exists a mapping $\rho$ from $t[A_i]$ to
$\adom(A_i)$ ($i\in [1, m]$) such that $I$ = $\{(\rho(t[A_1]),
\ldots, \rho(t[A_m]))\}$ and $I \models \Sigma$.


We now give an \NP algorithm as follows: (1) Guess an
instance, which contains a single tuple $t$ of $R$ such that
$t[A_i]\in\adom{A_i}$ for each $i \in [1, m]$. (2) Check whether $I
\models \Sigma$. If so the algorithm returns `yes', and otherwise
it repeats steps (1) and (2). Obviously step (2) can be done in
\PTIME in the size of $\Sigma$. Hence the algorithm is in \NP, and
so is the problem.
\end{proof}

\eat{This requires the establishment of a small model property.}

It is known~\cite{CFDs} that the satisfiability problem for \CFDs is
in \PTIME when the \CFDs considered are defined over attributes that
have an infinite domain, \ie in the absence of finite domain
attributes. However, this is no longer the case for \pCFDs. This
tells us that the increased expressive power of \pCFDs does take a
toll in this special case. It should be remarked that while the
proof of Proposition~\ref{thm-sat-pcfd-fin} is an extension of its
counterpart in~\cite{CFDs}, the result below is new.


\begin{theorem}
\label{thm-sat-pcfd-infin} In the absence of finite domain
attributes, the satisfiability problem for \pCFDs remains
\NP-complete.
\end{theorem}

\begin{proof} The problem is in \NP by
Proposition~\ref{thm-sat-pcfd-fin}. Its \NP-hardness is shown by
reduction from the \kSAT\ problem, which is \NP-complete
(cf.~\cite{GaJo79}).

We next show the reduction from the \kSAT\ problem. Consider an instance $\phi = C_1 \land \cdots
\land C_n$ of \kSAT, where all the variables in $\phi$ are $x_1,
\ldots, x_m$, $C_j$ is of the form $y_{j_1} \lor y_{j_2} \lor
y_{j_3}$ such that for each $i\in[1,3]$, $y_{j_i}$ is either
$x_{p_{ji}}$ or $\overline{x_{p_{ji}}}$ for $p_{ji} \in [1, m]$.
Given $\phi$, we construct a relational schema $R$ and a set $\Sigma$
of \pCFDs defined on $R$ such that $\phi$ is satisfiable iff
$\Sigma$ is satisfiable.

\sstab (1) We first define the relational schema $R$$(X_1, \ldots, X_m,$ $C_1, \ldots,
C_n, Z)$, where all attributes share a common infinite domain $\dom$ that contains constant $a$.
Intuitively, for each $R$ tuple $t$, $t[X_1, \ldots, X_m]$ specifies a truth assignment $\xi$ for variables
$x_1, \ldots, x_m$ of $\phi$, and $t[C_i]$ ($i\in[1, n]$) and $t[Z]$ are the truth
values of clause $C_i$ and sentence $\phi$ \wrt the assignment $\xi$, respectively.

\sstab (2) We then construct the set \pCFDs $\Sigma$ =
$\Sigma_0\cup\Sigma_1\cup\ldots\cup\Sigma_n\cup\Sigma_{n+1}$, defined as follows.

\bi
\item[(a)] $\Sigma_0$ contains $n + 1$ \pCFDs, which intuitively
encode the relationships of the truth values  between the clauses $C_1,
\ldots, C_n$ and  sentence $\phi$.

For each clause $C_i$ ($i\in[1, n]$), we add to $\Sigma_0$ a \pCFD $\varphi_i$ =
$(C_1,\ldots, C_n\ra Z, T_{pi})$, in which $T_{pi} = \{t_{pi}\}$ such that
$t_{pi}[C_i,Z]$ = $(\ne a \pa \ne a)$ and $t_{pi}[C_j]$ = `\_'
for any $j\ne i$ and $j\in[1, n]$. We also add to $\Sigma_0$ a \pCFD $\varphi_{0}$ =
$(C_1,\ldots, C_n\ra Z, T_{p0})$, where $T_{p0}$ = $\{(= a, \ldots,
=a \pa = a)\}$. Intuitively, we use $\ne a$ and $= a$ to represent
{\em false} and {\em true}, respectively.

\item[(b)]  For $i\in[1, n]$, $\Sigma_i$ contains $8$ \pCFDs, which intuitively
encode the relationships of the truth values between the clause $C_i$ and its three
variables.

For clause $C_i$ = $x_{j_1} \lor \overline{x_{j_2}} \lor
x_{j_3}$ of $\phi$ with $1\le j_1$, $j_2$, $j_3 \le m$, we define
\pCFDs
(a) $\varphi_{i,0}$ = $(X_{j_1}, X_{j_2}$, $X_{j_3}\ra C_i, T_{pi,0})$ with $T_{pi,0}$ = $\{(< a, < a, <a$ $\pa = a)\}$,
(b) $\varphi_{i,1}$ = $(X_{j_1},X_{j_2},X_{j_3}\ra C_i, T_{pi,2})$ with $T_{pi,1}$ = $\{(< a, <a, \ge  a \pa =a)\}$,  and
(c) $\varphi_{i,2}$ = $(X_{j_1},X_{j_2},X_{j_3}\ra C_i, T_{pi,2})$ with $T_{pi,2}$ = $\{(< a, \ge a, < a \pa \ne a)\}$.
Similarly, we can define the rest $5$ \pCFDs $\varphi_{i,3}$, $\varphi_{i,4}$,
$\varphi_{i,5}$, $\varphi_{i,6}$ and $\varphi_{i,7}$. Intuitively,
we further use $<a$ and $\ge a$ to represent {\em false} and {\em true} for
a variable, respectively.

\item[(c)] $\Sigma_{n + 1}$ contains a single \pCFD $\varphi_{n+1}$ =
$(Z \ra Z$, $T_{p(n+1)})$ with $T_{p(n+1)}$ = $\{(\_ \pa = a)\}$.
Intuitively, $\varphi_{n+1}$ requires that for any $R$ tuple $t$,
$t[Z]$ = $a$.
\ei

Observe that $\Sigma$ contains $8*m + n + 2$ \pCFDs in total. Thus the
reduction is in \PTIME.

We now show that $\phi$ is satisfiable iff $\Sigma$ is satisfiable.

Assume first that $\Sigma$ is satisfiable, then we  show that there exists a nonempty
instance $I$ of $R$ such that $I\models\Sigma$. For any tuple $t\in
I$, (a) $\Sigma_{n+1}$ forces $t[Z]$ = $a$, (b) $\Sigma_0$ forces
$t[C_1, \ldots, C_n]$ = $(a, \ldots, a)$, and (c) for each clause $C_i$
($i\in[1, n]$) with variables $X_{j_1},X_{j_2},X_{j_3}$, $\Sigma_j$
forces that $t[X_{j_1},X_{j_2},X_{j_3}]$  does not match the \LHS of
the \pCFDs that force $t[C_i]\ne a$. From the tuple $t$, we can construct a
truth assignment $\xi$ of $\phi$ such that $\xi(x_i)$ = {\em false}
if $t[X_i]< a$ and  $\xi(x_i)$ = {\em true} if $t[X_i]\ge a$
($i\in[1, m]$). Since $\{t\}\models\Sigma$, it is easy to verify
that the truth assignment $\xi$ makes $\phi$ true.

Conversely, if $\phi$ is satisfiable, there exists a truth
assignment $\xi$ that makes $\phi$ true. We construct a tuple $t$ of
$R$ as follows: (a) $t[C_1,\ldots, C_n, Z]$ = $(a, \ldots, a)$ and
(b) for each $i\in[1, m]$, $t[X_i]$ = $a_i$ such that (a) $a_i\in\dom$ and
$a_i\ge a$ if $\xi(x_i)$ = {\em true} and (b) $a_i< a$ otherwise. Let $I
= \{t\}$, then one can easily verify that $I\models\Sigma$.


Putting these together, we have the conclusion.
\end{proof}



\stitle{Satisfiability analysis of CIND$^p$s}. Like \FDs,
one can specify arbitrary \INDs or \CINDs without worrying about
their satisfiability. Below we show that \pCINDs preserve this nice
property, by extending the proof of its counterpart in~\cite{tcs-CINDs}.

\begin{prop}
\label{thm-sat-pcind} Any set of \pCINDs is always
satisfiable.
\end{prop}

\begin{proof}
 Given a set $\Sigma$ of \pCINDs over a
database schema $\cal R$$(R_1,\ldots,R_n)$, we show that one can always construct a
{\em nonempty} instance $D$ of $\cal R$ such that $D \models
\Sigma$.

We build $D$ as follows. First, for each attribute $A$, define the
active domain of $A$ to be a set $\adom(A)$, which consists of
certain data values in $\dom(A)$. Second, using these active
domains, we construct $D$.

\sstab (1) We start with the construction of active domains. (a) For each
attribute $A$, initialize $\adom(A)$ along the same lines as the one
for \pCFDs in Proposition~\ref{thm-sat-pcfd-fin}; (b) For each
\pCIND $(R_a[A_1,A_2,\ldots,A_m;X_p] \subseteq$ $R_b[B_1,$
$\ldots,B_m;Y_p], T_p)$ in $\Sigma$, let $\adom(B_i)=$ $\adom(B_i)$
$\cup$ $\adom(A_i)$ for each $i\in [1,m]$, and this rule is
repeatedly applied until a fixpoint of $\adom(A)$ is reached for
all attributes $A$ in $\cal R$.

It is easy to verify that this process always
terminates as we start with a finite set of data values.

\sstab (2)  We next construct the database instance $D$. For each relation $R_i(A_1, \dots,$ $A_k)$ $\in
{\cal R}$, we define $I_i = \adom(A_1)\times \dots \times
\adom(A_k)$, where $\times$ is the {\em Cartesian Product}
operation~\cite{AbHuVi1995}. Let $D = \{I_1,\dots,I_n\}$, then it is easy to verify
that $D$ is nonempty and $D \models \Sigma$.
\end{proof}


\stitle{Satisfiability  analysis of CFD$^p$s and
CIND$^p$s}. The satisfiability problem for \CFDs
and \CINDs taken together is undecidable~\cite{tcs-CINDs}. Since \pCFDs and
\pCINDs subsume \CFDs and \CINDs, respectively, we immediately have the following.

\vspace{-0.5ex}
\begin{cor}
\label{thm-sat-pcfd-pcind}The satisfiability problem for \pCFDs and
\pCINDs is undecidable.
\end{cor}




%%%%%%%%%%%%%%
\subsection{Implication Analyses}

The implication problem is to determine, given a set $\Sigma$ of
dependencies and another dependency $\phi$, whether or not
$\Sigma$ entails $\phi$, denoted by $\Sigma\models\phi$. That is,
whether or not for all databases $D$, if $D\models\Sigma$ then
$D\models\phi$.

The implication analysis helps us
remove redundant rules, and thus improve the
performance of error detection and repairing based on
the rules \cite{CFDs,tcs-CINDs}.

\begin{example}
\label{exa-implication} The \pCFDs in Fig.~\ref{fig-pcfd} imply
another \pCFD $\varphi =$ \at{item}(\at{sale}, \at{price} $\ra$
\at{shipping}, $T$), where $T$ consists of a single pattern tuple
$(\at{sale} = $`F', $\at{price} = 30 \pa \at{shipping} = 6)$. Thus
in the presence of the \pCFDs in Fig.~\ref{fig-pcfd}, $\varphi$ is
redundant.
\end{example}
\vspace{-1ex}

\stitle{Implication analysis of CFD$^p$s}.~We first show that
the implication problem for \pCFDs retains
the same complexity as their \CFDs counterpart, verified by
extending the proof of its counterpart in~\cite{CFDs}.

\begin{prop}
\label{thm-imp-pcfd-fin} The implication problem for \pCFDs is
\coNP-complete.
\end{prop}



\begin{proof}
The lower bound follows from the
\coNP-hardness of their \CFDs counterpart~\cite{CFDs}, since \CFDs
are a special case of \pCFDs. The \coNP upper bound is verified
by presenting an \NP algorithm for its complement problem for determining whether $\Sigma\not\models\varphi$.



We next present the a \NP algorithm
for its complement problem. The algorithm is based on a small model
property: if $\varphi = R(X \ra Y, T_p)$ and $\Sigma
\not\models\varphi$, then there exists an instance $I$ of $R$ with
two tuples $t_1$ and $t_2$ such that $I\models\Sigma$ and $t_1[X] =
t_2[X] \asymp T_p[X]$, but either $t_1[Y]\ne t_2[Y]$ or
$t_1[Y]\not\asymp T_p[Y]$ (resp. $t_2[Y]\not\asymp T_p[Y]$). Thus it
suffices to consider instances $I$ with two tuples only for deciding
whether $\Sigma\not\models\varphi$.

Assume that the attributes $\attrset(R)$ = $\{A_1,\dots,$
$A_m\}$. For each $i\in [1, m]$, let $\adom(A_i)$ be the active domain defined in
Proposition~\ref{thm-sat-pcfd-fin}. Then one can easily verify that
$\Sigma\not\models\varphi$ iff there exist two mappings $\rho_1$ and
$\rho_2$ from all attributes $A_i$ to $\adom(A_i)$ ($i\in [1, m]$) such that $I$
= $\{(\rho_1(A_1), \ldots, \rho_1(A_m))$, $(\rho_2(A_1),
\ldots, \rho_2(A_m))\}$, $I \models\Sigma$, but
$I\not\models\varphi$.

Based on these, we give an \NP algorithm as follows: (1) Guess two
$R$ tuples $t_1$ and $t_2$ such that $t_1[A_i],t_2[A_i]
\in\adom(A_i)$ for each $i \in [1, m]$. (2) Check whether $I =
\{t_1, t_2\}$ satisfies $\Sigma$, but not $\varphi$. If so the
algorithm returns `yes', and otherwise it repeats steps (1) and
(2). Obviously step (2) can be done in \PTIME in the size of $\Sigma$.
Hence the algorithm is in \NP, and so is the problem.
\end{proof}

Similar to the satisfiability analysis,
it is known~\cite{CFDs} that the implication analysis of
\CFDs is in \PTIME when the \CFDs are defined only with attributes that
have an infinite domain. Analogous to Theorem~\ref{thm-sat-pcfd-infin},
the result below shows that this is no longer the case for \pCFDs,
which does not find a counterpart in~\cite{CFDs}.

\begin{prop}
\label{thm-imp-pcfd-infin} In the absence of finite domain
attributes, the implication problem for \pCFDs is
\coNP-complete.
\end{prop}


\begin{proof}
The problem is in \coNP by
Proposition~\ref{thm-imp-pcfd-fin}. The \coNP-hardness is shown
by reduction from the \kSAT\ problem to its complement problem, \ie
the problem for determining whether $\Sigma\not\models\varphi$.


We next show the reduction from the \kSAT\ problem to the complement
problem of the implication problem for \pCFDs, where \kSAT\ is \NP-complete
(cf.~Proposition~\ref{thm-sat-pcfd-infin}). Given an instance $\phi$
of \kSAT, we construct a relational schema $R$ and a set
$\Sigma\cup\{\varphi\}$ of \pCFDs defined on $R$ such that $\phi$
is satisfiable iff $\Sigma\not\models\varphi$.


The relational schema $R$ and the set $\Sigma$ of \pCFDs are the same
as the corresponding ones in Proposition~\ref{thm-sat-pcfd-infin}.
Moreover, $\varphi$ is defined as $(Z \ra Z, T_p)$, where $T_{p}$ =
$\{(\_ \pa \ne a)\}$. Intuitively, $\varphi$ requires that for any
$R$ tuple $t$, $t[Z] \ne a$. Along the same lines as
Proposition~\ref{thm-sat-pcfd-infin}, one can easily verify that
$\phi$ is satisfiable iff $\Sigma\not\models\varphi$. Thus the
problem is \coNP-hard.
\end{proof}




\stitle{Implication analysis of CIND$^p$s}.~We next show that \pCINDs
do not make their implication analysis harder, verified by extending the proof of
their \CINDs counterpart given in~\cite{tcs-CINDs}.

\begin{prop}
\label{thm-imp-pcind-fin}The implication problem for \pCINDs is
\EXPTIME-complete.
\end{prop}



\begin{proof}
The implication problem for \CINDs is
\EXPTIME-hard \cite{tcs-CINDs}. Since \pCINDs subsume \CINDs, the lower bound carries over to \pCINDs immediately.
The \EXPTIME upper bound is shown
by presenting an \EXPTIME algorithm that, given a set
$\Sigma\cup\{\psi\}$ of \pCINDs over a database schema $\cal R$,
determines whether $\Sigma\models\psi$ or not.

We next present the \EXPTIME algorithm.
%
Consider $\cal R$ = $(R_1,$ $\ldots,$ $R_n)$ and $\psi$ = $(R_a[X;$ $X_p]
\subseteq$ $R_b[Y;Y_p], T_p)$. And for each attribute $A$,
define the active domain $\adom(A)$ of $A$ based on
$\Sigma\cup\{\psi\}$ along the same line as the proof of
Proposition~\ref{thm-sat-pcind}. One can easily verify that if
$\Sigma\not\models\psi$, there exists a non-empty instance $D$ of ${\cal
R}$ such that (a) $D\models\Sigma$ and $D\not\models\psi$, and (b) $D$ consists of data values from the active domains only.

The detailed \EXPTIME algorithm is given as follows.

\sstab(1)
We first build a labeled directed graph $G(V, E, l)$.  Each node $u\in V$ is a possible
tuple `$R_i:t_i$' such that $t_i[A]\in \adom(A)$ for each attribute
$A$ $\in$ $\attrset(R_i)$. There is an edge $e$ = (`$R_i:t_i$',
`$R_j:t_j$') in $E$ iff there exists a \pCIND $\phi$ =
$(R_i[U; U_p]$ $\subseteq$ $ R_j[V; V_p], T_{p_{\phi}})$ in $\Sigma$
such that $t_i[U_p] \asymp T_{p_{\phi}}[U_p]$, $t_j[V]$ = $t_i[U]$
and $t_j[V_p] \asymp T_{p_{\phi}}[V_p]$, and $e$ is labeled with the \pCIND $\phi$, \ie  $\phi \in l(e)$. Note that an edge may have multiple labels.

\sstab(2)
Let $S_a$ be the set of nodes `$R_a:t_a$' such that $t_a[X_p] \asymp T_{p}[X_p]$, and $S_b$ be the
set of nodes `$R_b:t_b$' such that $t_b[Y_p] \asymp T_{p}[Y_p]$, respectively.

\sstab(3)
For each node $u$ = `$R_a:t_a$' in $S_a$, let $G_u$ be the induced  subgraph of $G$ that contains all the nodes reachable from $u$,
and exactly  the edges that appear in $G$ over the same set of nodes. We also refer to $u$ as the root of $G_u$.



\sstab(4)
For an induced subgraph $G_u$ of $G$ with root $u$ = `$R_a:t_a$', we derive another graph $G'_u$ by recursively removing edges as follows. For any $v$ in $G_u$, if $v$ has a child $v'$ from which no nodes in `$R_b:t_b$' in $S_b$ with $t_b[X] = t_b[Y]$ are reachable, then for all children $v''$ of $v$, we remove from labels $l(v, v'')$ all the labels in $l(v, v')$, and edge $(v, v'')$ is removed when $l(v, v'')$ becomes empty.

\sstab(5)
If there exists a subgraph $G'_u$ derived from  an induced subgraph $G_u$ of $G$ with root $u$ = `$R_a:t_a$' such that  no nodes `$R_b:t_b$' in $S_b$ with $t_a[X] = t_b[Y]$ are reachable from $u$, we return `no', and return `yes', otherwise.



It can be verified that (a) if the algorithm returns `no', we can
construct an instance $D$ such that $D\models\Sigma$, but not
$\psi$, by collecting those tuples attached on the end nodes of edges whole labels become empty at setp 4;
and (b) if the algorithm returns `yes', there exist no
instances $D$ such that $D\models\Sigma$, but not $\psi$.

We next show that the above algorithm indeed runs in exponential
time: (a) The number of nodes in graph $G$ is bounded by the maximum
number of tuples in a database instance on $\R$. Let
$|\Sigma\cup\{\psi\}|$ be the size of $\Sigma$ and $\psi$, and
$|\cal R|$ be the sum of arities of all relations in $\cal R$. Then
the number of tuples in a database instance is bounded by
$O(|\Sigma\cup\{\psi\}|^{|\cal
 R|})$; (b) The number of nodes in sets $S_a$ or $S_b$ is bounded by
the maximum number of tuples in a database too; (c) The induced subgraph and the reachability
testing can be done in linear-time in the size of the
input~\cite{CormenLRS01}.

Putting all these together, we have shown that the algorithm runs in
exponential time. And, hence, the problem is in \EXPTIME.
\eat{ Based on these, we give the \EXPTIME algorithm as follows: (1)
Initialize set $S_d$ = $\{D_0\}$, where $D_0$ is a database instance
with $I_a = \{t_a\}$ of $R_a$ such that $t_a[A]\in\adom(A)$ for each
attribute $A\in\attrset(R_a)$, and $I_i = \{\}$ for any other
relation $R_i$ ($1\le i\le n$) in $\cal R$. (2) For each \pCIND
$(R_i[X';$ $X'_p] \subseteq$ $R_j[Y';Y'_p], T'_p)$ in $\Sigma$ and
each tuple $t_i\in I_i$ such that $t_i\asymp T'_p[X'_p]$, add the
following set of tuples to $I_j$. Let $\attrset(R_j)\setminus Y'$ =
$E_1,\ldots,E_k$ and $S$ = $\adom(E_1)\times\ldots\times\adom(E_k)$,
where $\times$ is the {\em Cartesian Product} operation. For each
tuple $t\in S$, if $t\asymp T'_p[Y'_p]$, add a tuple $t_j$ to $I_j$
such that $t_j[\attrset(R_j)\setminus Y']$ =
$t[\attrset(R_j)\setminus Y']$ and $t_j[Y'] = t_i[X']$. This rule is
repeatedly applied until all instances reach their fixpoints. (3)
Check whether $D = \{I_1, \ldots, I_n\}$ satisfies $\Sigma$, but not
$\psi$. If so, the algorithm returns ``no'', and otherwise it
repeats steps (1) (2) and (3). If all cases in step (1) are tested
and no instance $D$ is found such that $D\models\Sigma$ and
$D\not\models\psi$, the algorithm returns `yes'.

Let $|\Sigma\cup\{\psi\}|$ be the size of $\Sigma$ and $\psi$, and
$|\cal R|$ be the sum of arities of relations in $\cal R$. Then it
is easy to know that (a) the instance $D$ in step (3) contains at
most $O(|\Sigma\cup\{\psi\}|^{|\cal R|})$ tuples, and (b) there are
at most $O(|\Sigma\cup\{\psi\}|^{|\cal R|})$ cases for step (1).
Following from these, we can conclude that the algorithm is indeed
in \EXPTIME, and thus the problem is in \EXPTIME. }
\end{proof}

It is known~\cite{tcs-CINDs} that the implication problem is
\PSPACE-complete for \CINDs defined with infinite domain
attributes. Similar to Theorem~\ref{thm-imp-pcfd-infin},
 below we show that this no longer holds for \pCINDs.


\begin{theorem}
\label{thm-imp-pcind-infin} In the absence of finite domain
attributes, the implication problem for \pCINDs remains
\EXPTIME-complete.
\end{theorem}
\vspace{-1ex}

\begin{proof} The problem is in \EXPTIME by
Proposition~\ref{thm-imp-pcind-fin}. The \EXPTIME-hardness is
shown by reduction from the implication problem for \CINDs in the
general setting, in which finite-domain attributes may be present, that is known to be \EXPTIME-complete~\cite{tcs-CINDs}.

%%%%%%%%%%%%%%%%%%%%%%%%Complexity Summary%%%%%%%%%%%%%%%%%%%%%%%%%%%%%%
\begin{table*}[tbh!]
%\vspace{-2ex}
 \caption{Summary of Complexity Results\label{tab-complexity}}
 \vspace{-2ex}
\begin{center}
\begin{small}
\begin{tabular}{|c||c|c||c|c|} \hline
&  \multicolumn{2}{c||}{\at{General\ setting}} & \multicolumn{2}{c|}{\at{Infinite\ domain\ only}}\\
\cline{2-5}
\raisebox{1.0ex}[0pt]{$\Sigma$}  & \at{Satisfiability} & \at{Implication} & \at{Satisfiability} & \at{Implication}\\
\hline\hline \CFDs~\cite{CFDs} & \NP-complete& \coNP-complete & \PTIME & \PTIME \\
\hline
\pCFDs   & \NP-complete& \coNP-complete &\NP-complete& \coNP-complete  \\
\hline
\CINDs~\cite{tcs-CINDs} & $O(1)$ & \EXPTIME-complete &  $O(1)$ & \PSPACE-complete\\
\hline
\pCINDs  &  $O(1)$  & \EXPTIME-complete & $O(1)$  & \EXPTIME-complete\\
\hline
\CFDs+ \CINDs~\cite{tcs-CINDs} & undecidable& undecidable & undecidable& undecidable\\
\hline
\pCFDs+ \pCINDs & undecidable& undecidable & undecidable& undecidable\\
\hline
\end{tabular}
\end{small}
\end{center}
\vspace{-2ex}
\end{table*}

We next present the reduction from the implication problem for
\CINDs in the general setting. Given a set $\Sigma$ $\cup$ $\{\psi\}$ of \CINDs
defined on a database schema ${\cal R}$ $(R_1,$ $\ldots,$ $R_n)$, we
construct another database schema ${\cal R'}$$(R'_1,\ldots, R'_n)$,
in which each relation $R'_i$ ($i\in[1, n]$) consists of infinite
domain attributes only, and a set $\Sigma'\cup\{\psi'\}$ of \pCINDs
on ${\cal R'}$ such that $\Sigma\models\psi$ iff
$\Sigma'\models\psi'$.

\sstab(1)
We start with constructing ${\cal R'}$. For each
$R_i(A_1,$ $\ldots, A_k)$ of ${\cal R}$, we define
$R'_i(A'_1,\ldots,A'_k)$ such that for each attribute $A'_j
(j\in[1, k])$, let $\dom(A'_j)$ = $\dom(A_j)$ if $\dom(A_j)$ is
infinite, and let $\dom(A'_j)$ be \kw{integer}, a totally ordered
infinite domain, if $\dom(A_j)$ is finite. Moreover, we define a mapping $\rho_{i,j}$  for
each finite domain $\dom(A_j)$ = $\{a_1,\ldots,a_h\}$ to
\kw{integer}: (a) Randomly choose $h$ consecutive integers $\{b_1,
\ldots, b_h\}$ such that for each $i\in[1, h-1]$, $b_{i+1} = b_i +
1$. (b) We now define the mapping $\rho_{i,j}(a_i)$ = $b_i$ for
$i\in[1, h]$. Moreover, we require two extra integers $b_0$ = $b_1 -
1$ and $b_{h+1} = b_h + 1$, denoted as $\rho_{i,j}.b_0$ and
$\rho_{i,j}.b_{h+1}$. Note that this is always doable. For clarity,
we also denote $\rho_{i,j}$ as $\rho$ when it is clear from the
context.





\sstab(2)
We next define  $\Sigma'$ and $\psi'$ on ${\cal R'}$ based on the
mappings defined above.
For each \CIND $(R_a[X;$ $A_1, \ldots,A_{m1}]
\subseteq$ $R_b[Y; B_1, \ldots, B_{m2}], T_p)$ in $\Sigma$ and each
$t_p\in T_p$, we define another \pCIND $(R'_a[X'; A'_1,\ldots,A'_{m1},X'_p] \subseteq$ $R'_b[Y';$
$B'_1,\ldots,B'_{m2},Y'_p],$ $T'_p)$, where (a) $X'$ (resp. $Y', A'_1, \ldots, A'_{m1}$ and $B'_1,\ldots,B'_{m2}$) corresponds to $X$
(resp. $Y, A_1, \ldots, A_{m1}$ and $B_1,\ldots,B_{m2}$); (b) $X'_p$
(resp. $Y'_p$) corresponds to those finite domain attributes in
$R_a$ (resp. $R_b$), but not in $A_1,\ldots,A_{m1}$ (resp.
$B_1,\ldots,B_{m2}$); and (c) $T'_p = \{t'_{p1},t'_{p2},t'_{p3}\}$
such that for each attribute $A'$ in $A'_1,\ldots,A'_{m1}$ or
$B'_1,\ldots,B'_{m2}$, (i) $t'_{p1}[A']$ = $t_p[A]$ and
$t'_{p2}[A']$ = $t'_{p3}[A']$ = `\_' if $\dom(A)$ is infinite, and
(ii) $t'_{p1}[A']$ = $\rho(t_p[A])$ and $t'_{p2}[A']$ = $t'_{p3}[A']$
= `\_' if $\dom(A)$ is finite; and (iii) for the rest attributes $A'$
in $X'_p$ or $Y'_p$, $t'_{p1}[A']$ = `\_', $t'_{p2}[A']$ =
`$>\rho.b_0$', and $t'_{p3}[A']$ = `$<\rho.b_{h+1}$'. Similarly, we
can define a set $\Sigma_{\psi}$ of \pCINDs from $\psi$ based on the
mappings.


Finally, one can easily verify that $\Sigma\models\psi$ iff
$\Sigma'\models\Sigma_{\psi}$, \ie $\Sigma'\models\psi'$ for each
\pCIND $\psi'\in\Sigma_{\psi}$. Following from this, the problem is
\EXPTIME-hard.
\end{proof}






\stitle{Implication analysis of  CFD$^p$s and
CIND$^p$s}. When \pCFDs and \pCINDs are taken together,
their  implication analysis is beyond reach in practice.
This is not surprising since
the implication problem for \FDs and
\INDs is already undecidable~\cite{AbHuVi1995}. Since
\pCFDs and \pCINDs subsume \FDs and \INDs, respectively,
from the undecidability result for \FDs and
\INDs,  the corollary below follows immediately.

\begin{cor}
\label{thm-IM-pcfd-pcind} The implication problem for \pCFDs and
\pCINDs is undecidable.
\end{cor}




%\vspace{-0.5ex}
\stitle{Summary}. The complexity bounds for reasoning about
\pCFDs and \pCINDs are summarized in Table~\ref{tab-complexity}.
To give a complete picture we also include
in Table~\ref{tab-complexity} the complexity bounds for
the static analyses of
\CFDs and \CINDs, taken from~\cite{CFDs,tcs-CINDs}. The results tell us the following.

\sstab
(1) Despite the increased expressive
power, \pCFDs and \pCINDs do not complicate the static analyses in the general case: the
satisfiability and implication problems for \pCFDs and \pCINDs have
the same complexity bounds as their
counterparts for \CFDs and \CINDs, taken separately or together.

\sstab
(2) In the special case when \pCFDs and \pCINDs are defined
with infinite domain attributes only, however, their
static analyses do not get simpler, as opposed to their counterparts for
\CFDs and \CINDs. That is, the increased expressive power of \pCFDs and \pCINDs comes at a price in this special case.
