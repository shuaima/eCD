%\vspace{-0.5ex}
\section{Validation of CFD$^p$s and CIND$^p$s}
\label{sec-detect}
%\vspace{-1ex}

If \pCFDs and \pCINDs are to be used as data quality rules, the
first question we have to settle is how to effectively detect errors
and inconsistencies as violations of these dependencies, by
leveraging functionality supported by commercial \rdms. More
specifically, consider a database schema ${\cal R}$ = $(R_1,$
$\dots,R_n)$, where $R_i$ is a relational schema for $i \in [1, n]$.
The error detection problem is stated as follows.


The {\em error detection problem} is to find, given a  set $\Sigma$
of \pCFDs and \pCINDs defined on $\R$, and a database instance $D$ =
$(I_1,\dots,I_n)$ of ${\cal R}$ as input, the subset
$(I'_1,\dots,I'_n)$ of $D$ such that for each $i\in[1,n]$,
$I'_i\subseteq I_i$ and
 each tuple in $I'_i$ violates at least one  \pCFD or
\pCIND in $\Sigma$. We denote the set as $\kw{vio}(D, \Sigma)$,
referred to it as {\em the violation set} of $D$ \wrt $\Sigma$.
%The set $\Sigma$ is partitioned into $\SCFD$ and $\SCIND$,
%consisting of \pCFDs and \pCINDs, respectively.

In this section we develop \SQL-based techniques for error detection
based on \pCFDs and \pCINDs. The main result of the section is
as follows.

%\vspace{-0.5ex}
\begin{theorem}
\label{thm-detect} Given a set $\Sigma$  of \pCFDs and \pCINDs
defined on $\R$ = $(R_1,\dots, R_n)$ and a database instance $D$ of ${\cal R}$, a set of \SQL queries can be
automatically generated such that (a) the collection of the answers
to the \SQL queries in $D$ is $\kw{vio}(D,\Sigma)$, and (b) the number
and size of the set of \SQL queries depend only on the number $n$ of
relations and their arities in $\R$, regardless of $\Sigma$.
\end{theorem}
%\vspace{-0.5ex}

We next present the main techniques for the query generation
method. Let $\SCFD^i$ be the set of all \pCFDs in $\Sigma$
defined on the same relational schema $R_i$, and $\SCIND^{(i,j)}$ the
set of all \pCINDs in $\Sigma$ from $R_i$ to $R_j$, for
$i, j \in [1, n]$. We show the following. (a) The violation
set $\kw{vio}(D, \SCFD^i)$ can be computed by {\em two}
\SQL queries. (b) Similarly,  $\kw{vio}(D, \SCIND^{(i,j)})$
can be computed by a {\em single} \SQL query.
(c) These \SQL queries encode pattern tableaux
of \pCFDs (\pCINDs) with data tables, and hence
their sizes are independent of $\Sigma$. {From}
these Theorem~\ref{thm-detect} follows immediately.


%%%%%%%%%%%%%%%%%%5
\subsection{Encoding CFD$^p$s and CIND$^p$s with Data Tables}

We first show the following, by extending the encoding
of~\cite{CFDs,icde08}.
The pattern tableaux
of all \pCFDs in $\SCFD^i$ can be encoded with {\em three data tables},
and
 the pattern tableaux of all \pCINDs in $\SCIND^{(i,j)}$ can be
represented as
{\em four data tables}, no matter how many dependencies are
in the sets. %and how large they are.

%%%%%%%
%%%%%%%%%%%%%%%%%%%%%%%%%%%%%%%%%
\begin{figure*}[tb!]
\vspace{1ex}
\begin{center}
\begin{small}
\hspace{5ex}(1)~\Enc{L} \hspace{35ex}  (2)~\Enc{R} \hspace{30ex} (3)~\Enc{\ne}\\
\vspace{0.5ex}
\begin{tabular}{|c|c|c|c|c|}
\hline \at{cid} & \at{sale} & \at{price} &
\at{price_{>}} & \at{price_{\le}}\\
\hline 2&  T & null & null &null \\
3&  F&  \_ & 20  & 40 \\
4&    T &  null & null & null \\
\hline
\end{tabular}
\hspace{2ex}
\begin{tabular}{|c|c|c|c|c|}
\hline \at{cid} & \at{shipping}
& \at{price} & \at{price_{\ge}} & \at{price_{<}}\\
\hline
2&  0 &  null & null & null \\
3&  6 &  null & null & null \\
4&  null & \_ & 2.99 & 9.99 \\
\hline
\end{tabular}
%\\
%\vspace{1ex}
\hspace{2ex}
\begin{tabular}{|c|c|c|c|}
\cline{1-4} \at{cid} & \at{pos} & \at{att} & \at{val}\\
\cline{1-4}
\end{tabular}
\end{small}
\end{center}
\vspace{-1ex} \caption{Encoding example of \pCFDs}
\label{fig-pcfd-encoding} \vspace{-1ex}
\end{figure*}
%%%%%%%%%%%%%%%%%%%%%%%%%%%%%


%%%%%%%%%%%%%%%%%%%%%%%%%%%%%%%%%
\begin{figure*}[tb!]
%\vspace{-1ex}

\begin{center}
\begin{small}
\hspace{0ex} (1)~\Enc{} \hspace{20ex} (2)~\Enc{L} \hspace{18ex} (3)~\Enc{R} \hspace{20ex}(4)~\Enc{\ne} \\
\vspace{0.5ex}
\begin{tabular}{|c|c|c|}
\hline \at{cid} & \at{state_L} & \at{state_R}\\
\hline
1& 1 & 1 \\
2& 1 & 1 \\
\hline
\end{tabular}
\hspace{8ex}
\begin{tabular}{|c|c|c|}
\hline \at{cid} & \at{type} & \at{state}\\
\hline
1&  \_ & null\\
2&  \_ & DL\\
\hline
\end{tabular}
\hspace{8ex}
\begin{tabular}{|c|c|}
\hline \at{cid} & \at{rate}\\
\hline
1 & null\\
2 & 0 \\
\hline
\end{tabular}
\hspace{8ex}
\begin{tabular}{|c|c|c|c|}
\hline \at{cid} & \at{pos} & \at{att} & \at{val}\\
\hline
1 & \LHS & type & art \\
2 & \LHS & type & art \\
\hline
\end{tabular}
\end{small}
\end{center}

\vspace{-1ex} \caption{Encoding example of \pCINDs}
\label{fig-pcind-encoding} \vspace{-1ex}
\end{figure*}
%%%%%%%%%%%%%%%%%%%%%%%%%%%%%



\stitle{Encoding CFD$^p$s}. We encode all  pattern tableaux
in $\SCFD^i$ with three tables \Enc{L}, \Enc{R} and \Enc{\ne}, where
\Enc{L} (resp.~\Enc{R}) encodes the non-negation ($=,<,
\le, >, \ge$) patterns in \LHS (resp.~\RHS),
and
\Enc{\ne} encodes those negation ($\ne$) patterns.
More specifically, we associate a unique id \at{cid} with each
\pCFDs in $\SCFD^i$, and let
\Enc{L} consist of the following
attributes: (a) \at{cid}, (b)
each attribute $A$ appearing in the \LHS of some \pCFDs in $\SCFD^i$, and
 (c) its four companion attributes $A_{>}$, $A_{\ge}$, $A_{<}$, and
 $A_{\le}$. That is, for each attribute, there are five columns in
 \Enc{L}, one for each non-negation operator.
Similarly, \Enc{R} is defined. We use an \Enc{\ne} tuple to encode a
pattern $A \ne c$ in a \pCFD, consisting of \at{cid}, \at{att},
\at{pos}, and \at{val}, encoding the \pCFD id, the attribute $A$,
the position (`\LHS' or `\RHS'), and the constant $c$, respectively.
Note that the arity of \Enc{L} (\Enc{R}) is bounded by $5*|R_i|+1$,
where $|R_i|$ is the arity of $R_i$, and the arity of \Enc{\ne} is
$4$.

Before we populate these tables, let us first describe a preferred
form of \pCFDs that would simplify the analysis to be given.
Consider a \pCFD $\varphi$ = $R(X\ra Y,\ T_p)$. If $\varphi$ is not
satisfiable we can simply drop it from $\Sigma$. Otherwise it is
equivalent to a \pCFD $\varphi'$ = $R(X\ra Y,\ T_p')$ such that for
any pattern tuples $t_p, t_p'$ in $T_p'$ and for any attribute $A$
in $X\cup Y$, (a) if $t_p[A]$ is $\op a$ and $t_p'[A]$ is $\op b$,
where \op is not $\ne$, then $a=b$, and (b) if $t_p[A]$ is `\_' then so
is $t_p'[A]$. That is, for each non-negation \op (resp.~\_), there
is a {\em unique} constant $a$ such that $t_p[A]$ = `$\op a$'
(resp.~$t_p[A]=\_$) is the only \op (resp.~\_) pattern appearing in
the $A$ column of $T'_p$. We refer to $t_p[A]$ as $T'_p(\kw{op},A)$
(resp.~$T'_p(\_,A)$), and consider \kwlog~\pCFDs of this form only.
Note that there are possibly multiple $t_p[A] \ne c$ patterns in
$T'_p$,

We populate \Enc{L}, \Enc{R} and \Enc{\ne}  as follows. For each
\pCFD $\varphi$ = $R(X\ra Y,\ T_p)$ in $\SCFD^i$,
we generate a distinct \at{cid} $\kw{id}_\varphi$ for it, and do the following.
\bi
\item[(1)] Add a tuple $t_1$ to \Enc{L} such that
(a) $t[\at{cid}] = \kw{id}_\varphi$; (b) for each $A\in X$,
$t[A]$ = `\_' if $T'_p(\_,A)$ is `\_', and for each non-negation
predicate \op, $t[A_{\kw{op}}]$ = `$a$' if $T'_p(\kw{op},A)$
is `\op $a$'; (c) we let $t[B]$ = \bd{null} for all other attributes $B$
in \Enc{L}.



\item[(2)] Similarly add a tuple $t_2$ to \Enc{R} for attributes in $Y$.


\item[(3)] For each attribute $A\in X\cup Y$ and each $\ne a$ pattern in
$T_p[A]$, add a tuple $t$ to \Enc{\ne} such that $t[\at{cid}] =
\kw{id}_\varphi$, $t[\at{att}]$ = `$A$', $t[\at{val}]$ = `a', and
$t[\at{pos}]$ = `\LHS' (resp.~$t[\at{pos}]$ = `\RHS') if attribute
$A$ appears in $X$ (resp. $Y$). \ei\vspace{-1ex}





\begin{example}
Recall from Fig.~\ref{fig-pcfd} \pCFDs $\varphi_2$, $\varphi_3$ and
$\varphi_4$ defined on relation \at{item}. The three \pCFDs are
encoded with the tables shown in Fig.~\ref{fig-pcfd-encoding}: (a)
\Enc{L} consists of attributes: \at{cid}, \at{sale}, \at{price},
\at{price_{>}} and \at{price_{\le}}; (b) \Enc{R} consists of
\at{cid}, \at{shipping}, \at{price}, \at{price_{\ge}} and
\at{price_{<}}; those attributes in a table with only `null' pattern
values do not contribute to error detection, and are thus omitted;
And (c) \Enc{\ne} is empty since all these \pCFDs have no negation
patterns. One can easily reconstruct these \pCFDs from tables
\Enc{L}, \Enc{R} and \Enc{\ne} by collating the tuples based on
\at{cid}.
\end{example}


%%%



\stitle{Encoding CIND$^p$s}. All \pCINDs in $\SCIND^{(i,j)}$
can be encoded with four tables
\Enc{}, \Enc{L}, \Enc{R} and \Enc{\ne}.
Here \Enc{L} (resp.~\Enc{R}) and \Enc{\ne} encode non-negation
patterns on relation $R_i$ (resp.~$R_j$)
and negation patterns on relations $R_i$ or $R_j$,
respectively, along the same lines as their counterparts for
\pCFDs. We use
\Enc{} to encode the \INDs\ {\em embedded} in
\pCINDs, which consists of the following attributes: (1) \at{cid}
representing the id of a \pCIND,  and (2) those $X$ attributes of $R_i$
and $Y$ attributes of $R_j$ appearing in some \pCINDs in $\SCIND^{(i,j)}$.
Note that the number of attributes in \Enc{} is bounded by $|R_i| + |R_j|
+ 1$, where $|R_i|$ is the arity of $R_i$.

\eat{We populate these tables as follows.}

For each \pCIND $\psi$ = $(R_i[A_1\ldots A_m;\ X_p]$ $\subseteq$
$R_j[B_1\ldots B_m;$ $\ Y_p],\ T_p)$ in $\SCIND^{(i,j)}$, we generate a
distinct \at{cid} $\kw{id}_\psi$ for it, and do the following.

\vspace{-0.5ex}\bi
\item[(1)] Add tuples $t_1$ and $t_2$ to \Enc{L} and \Enc{R}
based on attributes $X_p$ and $Y_p$, respectively, along the
same lines as their \pCFD counterpart.
\item[(2)] Add tuples to \Enc{\ne} in the same way as their
\pCFD counterparts.
\item[(3)] Add tuple $t$ to \Enc{} such that $t[\at{cid}] = \kw{id}_\psi$.
For each $k\in[1, m]$,  let $t[A_k]$ = $t[B_k]$ = $k$, and $t[A]$ =
\bd{null} for the rest attributes $A$ of \Enc{}. \ei \vspace{-1ex}




\begin{example} \label{exa-enc-pfind}
Figure~\ref{fig-pcind-encoding} shows the coding of \pCINDs $\psi_1$
and $\psi_2$ given in Fig.~\ref{fig-pcind}. We use \at{state_L} and
\at{state_R} in \Enc{} to denote the occurrences of attribute
\at{state} in \at{item} and \at{tax}, respectively. In
\Enc{L} and \Enc{R}, the attributes with only `null' patterns are
omitted, for the same reason as \pCFDs mentioned above.
\end{example}

Putting these together, it is easy to verify that at most $O(n^2)$
data tables are needed to encode dependencies in $\Sigma$,
regardless of the size of $\Sigma$. Recall that $n$ is the number of
relations in the database $\R$.


%%%%%%%%%%%%%%%%%%%%%%%%%%%%%%%%
\subsection{SQL-based Detection Methods}

We next show how to generate \SQL queries based on the encoding
above. For each $i\in [1, n]$, we generate {\em two} \SQL queries
that, when evaluated on the $I_i$ table of $D$, find $\kw{vio}(D,
\SCFD^i)$. Similarly, for each $i,j \in [1, n]$, we generate a {\em
single} \SQL query $Q_{(i,j)}$ that, when evaluated on $(I_i, I_j)$
of $D$, returns $\kw{vio}(D, \SCIND^{(i,j)})$. Putting these query
answers together, we get $\kw{vio}(D, \Sigma)$, the violation set of
$D$ \wrt $\Sigma$.



\eat{%%%%%%%%%%%%%%%%EAT


\stitle{SQL queries for CFD$^p$s}.
Below we first show how the \SQL queries $Q_{i}^c$ and $Q_{i}^v$ are generated for
validating \pCFDs in $\SCFD^i$, which is an extension of the
\SQL techniques for \CFDs and \eCFDs discussed in~\cite{CFDs} and ~\cite{icde08}, respectively.

The queries $Q_{i}^c$ and $Q_{i}^v$ for the violations of $\SCFD^i$ are given as follows, which capitalize on the data table \Enc{L}, \Enc{R} and \Enc{\ne} that encode \pCFDs in $\SCFD^i$.

%%%%%%%%%%%%%%%%%%%%%%%%%%%%%%% begin duan 2014-11-18
\begin{footnotesize}\mat{0ex}{
$Q_{i}^c$: \= \bd{select} \=  $R_i.*$ \bd{from} \= $R_i$, \Enc{L} $L$, \Enc{R} $R$, \Enc{\ne} $N$  \\
\>  \bd{where} \=  $L.\at{cid}=R.\at{cid}$ \bd{and}  $R_i.X\asymp L$ \bd{and} $R_i.X\asymp N$ \bd{and}\\
\> \> \bd{not} ($R_i.Y\asymp R$ \bd{and} $R_i.Y\asymp N$)}
\end{footnotesize}

\begin{footnotesize}\mat{0ex}{
$Q_{j}^v$: \= \bd{select} \= \bd{distinct } \= $X_{L}$ \\
\>  \bd{from} (\=\bd{select} \= $L.\at{cid}$ \bd{as} \at{cid}, $X_{L}$, $Y_{R}$ \bd{from} $R_i$, \Enc{L} $L$, \Enc{R} $R$, \Enc{\ne} $N$ \\
\>\> \bd{where} $L.\at{cid}=R.\at{cid}$ \bd{and} $R_i.X\asymp L$ \bd{and} \\
\>\>\> $R_i.X\asymp N$ \bd{and} $R.Y='\_'$ ) \bd{as} $M$\\
\>\>\bd{group by} \at{cid}, $X_{L}$ \bd{having count} (\bd{distinct} $Y_{R}$)$>1$}
\end{footnotesize}

\noindent Here (1) $X = \{\at{A_1,\dots, A_{m1}}\}$ and $Y = \{\at{B_1,\dots, B_{m2}}\}$ are the sets of attributes in \LHS and \RHS of $\SCFD^i$ respectively; (2) $R_i.X\asymp L$ is the conjunction of

\begin{footnotesize}\mat{0ex}{
$L.A_j$ \= \bd{is null or} $R_i.A_j$ = $L.A_j$ \bd{or} ($L.A_j$ = `\_'\\
\> \bd{and} ($L.A_{j_{>}}$ \bd{is null or} $R_i.A_j$ $>$ $L.A_{j_{>}}$)\\
\>  \bd{and} ($L.A_{j_{\ge}}$ \bd{is null or} $R_i.A_j$ $\ge$ $L.A_{j_{\ge}}$)\\
\> \bd{and} ($L.A_{j_{<}}$ \bd{is null or} $R_i.A_j$ $<$ $L.A_{j_{<}}$)\\
\> \bd{and} ($L.A_{j_{\le}}$ \bd{is null or} $R_i.A_j$ $\le$ $L.A_{j_{\le}}$))
}
\end{footnotesize}

\noindent for each $j\in[1, m_1]$; (3) $R_i.Y\asymp R$ is defined similarly for attributes in $Y$; (4) $R_i.X\asymp N$ is the conjunction of

\begin{footnotesize}\mat{1ex}{
\bd{not exists}
(\=\bd{select} $*$ \bd{from} $N$ \\
\>\bd{where} \= $L.\at{cid}$ = $N.\at{cid}$ \bd{and} $N.\at{pos}$ = `\LHS' \bd{and}\\
\>\>  $N.\at{att}$ = `$A_j$' \bd{and}
 $R_i.A_j$ = $N.\at{val}$)
}
\end{footnotesize}

\noindent for each $j\in[1, m_1]$; (5) $R_i.Y\asymp N$ is defined similarly, but with $N.\at{pos}='\RHS'$; (6) $X_{L}$ is the set of following attributes

\begin{footnotesize}\mat{3ex}{
(\bd{case when} $L.A_j$ \bd{is not null then} $R_i.A_j$ \bd{end}) \bd{as} ${A_L}_j$
}
\end{footnotesize}
\noindent for each $j \in [1, m_1]$; (7) Similarly, $Y_{R}$ is the set of
\begin{footnotesize}\mat{3ex}{
(\bd{case when} $R.B_k$ \bd{is not null then } $R_i.B_k$ \bd{end}) \bd{as} ${B_R}_k$
}
\end{footnotesize}
\noindent for each $k \in [1, m_2]$; (8) $R.Y$ = `\_' is the disjunction of $R.B_k$ = `\_' for each $k \in [1, m_2]$.

Intuitively, detecting violations of \pCFDs is a two-step process. First, query $Q_{i}^{c}$ detects single-tuple violations, \ie the tuples $t$ in $I_{i}$ that match the \LHS of a \pCFD in $\SCFD^i$, but do not match its \RHS. Second, query $Q_{i}^{v}$ finds multi-tuple violations, \ie the tuples $t$ in $I_{i}$ such that (a) there exists another tuple $t'$ in $I_{i}$, $t$ and $t'$ match and agree on the \LHS of a \pCFD in $\SCFD^i$, but do not agree on the \RHS of the \pCFD.

%For each \CFD, query $Q_{i}^{v}$ uses the \bd{group by} clause to group tuples with the same value on $X$ and it counts the number of distinct $Y$. If there is more than one instantiation, then there is a violation.

\begin{example}\label{exa-cfd-query} Using the coding of Fig.~\ref{fig-pcfd-encoding}, two SQL queries for checking \pCFDs $\varphi_2$, $\varphi_3$ and $\varphi_4$ of Fig.~\ref{fig-pcfd} are given as follows:

\begin{footnotesize}\mat{0ex}{
$Q_{1}^{c}$: \= \bd{select} \=  $R_1.*$ \= \bd{from}  \at{item} $R_1$, \Enc{L} $L$, \Enc{R} $R$, \Enc{\ne} $N$\\
 \> \bd{where} $L.\at{cid}=R.\at{cid}$ \bd{and} \\

 \> \> ($L.\at{sale}$ \bd{is null or} $R_1.\at{sale}=L.\at{sale}$ \bd{or} $L.\at{sale}='\_'$ ) \bd{and} \\
 \> \> \bd{not exists} (\bd{select * from} $N$\\
 \> \> \> \bd{where} $N.\at{cid}=L.\at{cid}$ \bd{and} $N.\at{pos}='\LHS'$ \bd{and} \\
 \> \> \> $N.\at{att}='sale'$ \bd{and} $R_{1}.\at{sale}=N.\at{val}$ ) \bd{and} \\

 \> \> ($L.\at{price}$ \bd{is null or} $R_1.\at{price}=L.\at{price}$ \bd{or} ( $L.\at{price}='\_'$ \bd{and} \\
 \> \> ($L.\at{price_>}$ \bd{is null or} $R_1.\at{price}>L.\at{price_>}$ ) \bd{and}\\
 \> \> ($L.\at{price_\leq}$ \bd{is null or} $R_1.\at{price}\leq L.\at{price_\leq}$ ))) \bd{and} \\
 \> \> \bd{not exists} ( \bd{select * from} $N$ \\
 \>\>\> \bd{where} $N.\at{cid}=L.\at{cid}$ \bd{and} $N.\at{pos}='\LHS'$ \bd{and} \\
 \> \> \> $N.\at{att}='price'$ \bd{and} $R_{1}.\at{price}=N.\at{val}$) \bd{and}\\

 \> \> \bd{not} (( $R.\at{shipping}$ \bd{is null or} $R_1.\at{shipping}=R.\at{shipping}$ \bd{or}\\
 \> \> $R.\at{shipping}='\_'$ ) \bd{and} \\
 \> \>  \bd{not exists} ( \bd{select * from} $N$  \\
 \> \> \> \bd{where} $N.\at{cid}=R.\at{cid}$ \bd{and} $N.\at{pos}='\RHS'$ \bd{and} \\
 \> \> \> $N.\at{att}='shipping'$ \bd{and} $R_{1}.\at{shipping}=N.\at{val}$) \bd{and} \\

 \> \> ($R.\at{price}$ \bd{is null or} $R_1.\at{price}=R.\at{price}$ \bd{or} ( $R.\at{price}='\_'$ \bd{and}\\
 \> \> ($R.\at{price_\geq}$ \bd{is null or} $R_1.\at{price}\geq R.\at{price_\geq}$ ) \bd{and}   \\
 \> \> ($R.\at{price_<}$ \bd{is null or} $R_1.\at{price}<R.\at{price_<}$ ))) \bd{and}   \\
 \> \> \bd{not exists} ( \bd{select * from} $N$  \\
 \> \> \> \bd{where} $N.\at{cid}=R.\at{cid}$ \bd{and} $N.\at{pos}='\RHS'$ \bd{and} \\
 \> \> \> $N.\at{att}='price'$ \bd{and} $R_{1}.\at{price}=N.\at{val}$ )) }
\end{footnotesize}

\begin{footnotesize}\mat{0ex}{
$Q_{1}^{v}$: \= \bd{select} \= \bd{distinct} \= \at{sale_L}, \=\at{price_L} \bd{from} ( \\
 \> \bd{select} $L.\at{cid}$ \bd{as} \at{cid} , \\
 \> (\bd{case when} $L.\at{sale}$ \bd{is not null then} $R_1.\at{sale}$ \bd{end}) \bd{as} \at{sale_L},\\
 \> (\bd{case when} $L.\at{price}$ \bd{is not null then} $R_1.\at{price}$ \bd{end}) \bd{as} \at{price_L},\\
 \> (\bd{case when} $R.\at{shipping}$ \bd{is not null then} $R_1.\at{shipping}$ \bd{end}) \bd{as} \at{shipping_R},  \\
 \> (\bd{case when} $R.\at{price}$ \bd{is  not null then} $R_1.\at{price}$ \bd{end}) \bd{as} \at{price_R}  \\
 \> \bd{from} \at{item} $R_1$, \Enc{L} $L$, \Enc{R} $R$, \Enc{\ne} $N$ \\
 \>\> \bd{where} $L.\at{cid}=R.\at{cid}$ \bd{and} \\

 \>\> ($L.\at{sale}$ \bd{is null or} $R_1.\at{sale}=L.\at{sale}$ \bd{or} $L.\at{sale}='\_'$) \bd{and}\\
 \> \> \bd{not exists} ( \bd{select * from} $N$\\
 \> \> \>  \bd{where} $N.\at{cid}=L.\at{cid}$ \bd{and} $N.\at{pos}='\LHS'$ \bd{and} \\
 \> \> \>  $N.\at{att}='sale'$ \bd{and} $R_{1}.\at{sale}=N.\at{val}$) \bd{and} \\

 \>\>  ($L.\at{price}$ \bd{is null or} $R_1.\at{price}=L.\at{price}$ \bd{or} ($L.\at{price}='\_'$ \bd{and} \\
 \> \> ($L.\at{price_>}$ \bd{is null or} $R_1.\at{price}>L.\at{price_>}$) \bd{and}   \\
 \> \> ($L.\at{price_\leq}$ \bd{is null or} $R_1.\at{price}\leq L.\at{price_\leq}$))) \bd{and} \\
 \> \>  \bd{not exists} (\bd{select * from} $N$ \\
 \>\>\>  \bd{where} $N.\at{cid}=L.\at{cid}$ \bd{and} $N.\at{pos}='\LHS'$ \bd{and} \\
 \> \> \>  $N.\at{att}='price'$ \bd{and} $R_{1}.\at{price}=N.\at{val}$) \bd{and}\\

 \> \> ($R.\at{shipping}='\_'$\bd{or} $R.\at{price}='\_'$ )) \bd{as} $M$ \\
 \>  \bd{group by} \at{cid}, $\at{sale_L}, \at{price_L}$ \\
 \>  \bd{having count} (\bd{distinct} $\at{shipping_R}, \at{price_R}$)$>1$ }
\end{footnotesize}
\end{example}

}%%%%%%%%%%%%%%%%EAT


\eat{
%%%%%%%%%%%%%%%%%%%%%%%%%%%%%%% end duan 2014-11-18



\begin{footnotesize}\mat{0ex}{
$Q^C$ \= \bd{select} \=  $R_i.*$ \bd{from} $R_i$, \Enc{L} $L$, \Enc{R} $R$, \Enc{\ne} $N$ \\
\>  \bd{where} $L.\at{cid}=R.\at{cid}$ \bd{and} $L.\at{cid}=N.\at{cid}$ \bd{and}  $R_i.X\asymp L$ \bd{and} \\
\> \> $R_i.X\asymp N$  \bd{and not} ($R_i.Y\asymp R$ \bd{and} $R_i.Y\asymp N$)}
\end{footnotesize}

\begin{footnotesize}\mat{0ex}{
$Q^V$ \= \bd{select} \= \bd{distinct } \= $X$ \bd{from ( select} $L.\at{cid}$, \\
\>  (\bd{case when} $L.X_j$ \bd{is null then null} \bd{else} $R_i.X_j$ \bd{end}) \bd{as} $X_j$\bd{...} \\
\>  (\bd{case when} $R.Y_k$ \bd{is null then null} \bd{else} $R_i.Y_k$ \bd{end}) \bd{as} $Y_k$\bd{...}\\
\>\> \bd{from} $R_i$, \Enc{L} $L$, \Enc{R} $R$, \Enc{\ne} $N$ \\
\>\> \bd{where} $L.\at{cid}=R.\at{cid}$ \bd{and} $L.\at{cid}=N.\at{cid}$ \bd{and} $R_i.X\asymp L$ \bd{and} \\
\>\>\> $R_i.X\asymp N$ \bd{and} ($R.Y_1='\_'$ \bd{or} $R.Y_2='\_'$ \bd{or ...} )) $M$\\
\>\>\> \bd{group by} $X$, \at{cid} \bd{having count} (\bd{distinct} $Y$)$>1$}
\end{footnotesize}

\noindent Here (1) $X$ and $Y$ are the sets of attributes in \LHS and \RHS of $\SCFD^i$ respectively; (2) $R_i.X\asymp L$ is the conjunction of

\begin{footnotesize}\mat{0ex}{
$L.A_k$ \= \bd{is null or} $R_i.A_k$ = $L.A_k$ \bd{or} ($L.A_k$ = '\_'\\
\> \bd{and} ($L.A_{k_{>}}$ \bd{is null or} $R_i.A_k$ $>$ $L.A_{k_{>}}$)\\
\>  \bd{and} ($L.A_{k_{\ge}}$ \bd{is null or} $R_i.A_k$ $\ge$ $L.A_{k_{\ge}}$)\\
\> \bd{and} ($L.A_{k_{<}}$ \bd{is null or} $R_i.A_k$ $<$ $L.A_{k_{<}}$)\\
\> \bd{and} ($L.A_{k_{\le}}$ \bd{is null or} $R_i.A_k$ $\le$ $L.A_{k_{\le}}$))
}
\end{footnotesize}

\noindent for each $A_{k} \in X$; (3) $R_i.Y\asymp R$ is defined similarly for attributes in $Y$; (4) $R_i.X\asymp N$ is the conjunction

\begin{footnotesize}\mat{1ex}{
\bd{not exists}
(\=\bd{select} $*$ \bd{from} $N$ \\
\>\bd{where} \= $L.\at{cid}$ = $N.\at{cid}$ \bd{and} $N.\at{pos}$ = `\LHS' \bd{and}\\
\>\>  $N.\at{att}$ = `$A_k$' \bd{and}
 $R_i.A_k$ = $N.\at{val}$);
}
\end{footnotesize}

\noindent for $A_{k} \in X$; (5) $R_i.Y\asymp N$ is defined similarly, but with $N.\at{pos}='\RHS'$.

Intuitively, detecting violations of \pCFDs in $\SCFD^i$ is a two-step process. Firstly, query $Q^{C}$ detects single-tuple violations, that is the tuples $t$ in $I_{i}$ that match the \LHS patterns of some \pCFDs in $\SCFD^i$, but $t$ does not match the corresponding \RHS patterns of these \pCFDs. Secondly, query $Q^{V}$ finds multi-tuple violations, that is, tuples $t$ in $I_{i}$ for which there exists a tuple $t'$ in $I_{i}$ such that $t[X]=t'[X] \asymp L$, but $t[Y] \neq t'[Y]$. Query $Q^{V}$ uses the \bd{group by} clause to group tuples with the same value on $X$ and it counts the number of distinct $Y$. If there is more than one instantiation, then there is a violation.

\begin{example}\label{exa-cfd-query} Using the coding of Fig.~\ref{fig-pcfd-encoding}, two SQL queries for checking \pCFDs $\varphi_2$, $\varphi_3$ and $\varphi_4$ of Fig.~\ref{fig-pcfd} are given as follows:

\begin{footnotesize}\mat{0ex}{
$Q^C$ \= \bd{select} \=  $R_1.*$ \= \bd{from}  \at{item} $R_1$, \Enc{L} $L$, \Enc{R} $R$, \Enc{\ne} $N$\\
 \> \bd{where} $L.\at{cid}=R.\at{cid}$ \bd{and} ($L.\at{sale}$ \bd{is null or} $R_1.\at{sale}=L.\at{sale}$\\
 \> \> \bd{or} $L.\at{sale}='\_'$) \bd{and not exists} ( \bd{select * from} $N$\\
 \> \> \> \bd{where} $N.\at{cid}=L.\at{cid}$ \bd{and} $N.\at{pos}='\LHS'$ \bd{and} \\
 \> \> \> $N.\at{att}='sale'$ \bd{and} $R_{1}.\at{sale}=N.\at{val}$) \bd{and} \\
 \>\> ($L.\at{price}$ \bd{is null or} $R_1.\at{price}=L.\at{price}$ \bd{or} ($L.\at{price}='\_'$  \\
 \> \> \bd{and} ($L.\at{price_>}$ \bd{is null or} $R_1.\at{price}>L.\at{price_>}$)  \\
 \> \> \bd{and} ($L.\at{price_\leq}$ \bd{is null or} $R_1.\at{price}\leq L.\at{price_\leq}$))) \\
 \> \> \bd{and not exists} ( \bd{select * from} $N$ \\
 \>\>\> \bd{where} $N.\at{cid}=L.\at{cid}$ \bd{and} $N.\at{pos}='\LHS'$ \bd{and} \\
 \> \> \> $N.\at{att}='price'$ \bd{and} $R_{1}.\at{price}=N.\at{val}$) \bd{and not}\\

 \> \> (($R.\at{shipping}$ \bd{is null or} $R_1.\at{shipping}=R.\at{shipping}$ \bd{or}\\
 \> \> $R.\at{shipping}='\_'$\bd{) and not exists} ( \bd{select * from} $N$  \\
 \> \> \> \bd{where} $N.\at{cid}=R.\at{cid}$ \bd{and} $N.\at{pos}='\RHS'$ \bd{and} \\
 \> \> \> $N.\at{att}='shipping'$ \bd{and} $R_{1}.\at{shipping}=N.\at{val}$) \bd{and} \\

 \> \> ($R.\at{price}$ \bd{is null or} $R_1.\at{price}=R.\at{price}$ \bd{or} ($R.\at{price}='\_'$  \\
 \> \> \bd{and} ($R.\at{price_\geq}$ \bd{is null or} $R_1.\at{price}\geq R.\at{price_\geq}$)   \\
 \> \> \bd{and} ($R.\at{price_<}$ \bd{is null or} $R_1.\at{price}<R.\at{price_<}$)))  \\
 \> \> \bd{and not exists} (\bd{select * from} $N$  \\
 \> \> \> \bd{where} $N.\at{cid}=R.\at{cid}$ \bd{and} $N.\at{pos}='\RHS'$ \bd{and} \\
 \> \> \> $N.\at{att}='price'$ \bd{and} $R_{1}.\at{price}=N.\at{val}$) ) }
\end{footnotesize}

\begin{footnotesize}\mat{0ex}{
$Q^V$\= \bd{select} \= \bd{distinct} \= \at{sale_x}, \at{price_x} \bd{from} ( \bd{select} $L.\at{cid}$, \\
 \> (\bd{case when} $L.\at{sale}$ \bd{is null then null else} $R_1.\at{sale}$ \bd{end}) \bd{as} \at{sale_x},\\
 \> (\bd{case when} $L.\at{price}$ \bd{is null then null else} $R_1.\at{price}$ \bd{end}) \bd{as} \at{price_x},\\
 \> (\bd{case when} $R.\at{shipping}$ \bd{is null then null else} $R_1.\at{shipping}$ \bd{end})\\
 \> \bd{as} \at{shipping_y},  \\
 \> (\bd{case when} $R.\at{price}$ \bd{is null then null else} $R_1.\at{price}$ \bd{end}) \bd{as} \at{price_y}  \\
 \> \bd{from} \at{item} $R_1$, \Enc{L} $L$, \Enc{R} $R$, \Enc{\ne} $N$ \\

 \> \bd{where} $L.\at{cid}=R.\at{cid}$ \bd{and} ($L.\at{sale}$ \bd{is null or} $R_1.\at{sale}=L.\at{sale}$\\
 \> \> \bd{or} $L.\at{sale}='\_'$) \bd{and not exists} ( \bd{select * from} $N$\\
 \> \> \> \bd{where} $N.\at{cid}=L.\at{cid}$ \bd{and} $N.\at{pos}='\LHS'$ \bd{and} \\
 \> \> \> $N.\at{att}='sale'$ \bd{and} $R_{1}.\at{sale}=N.\at{val}$) \bd{and} \\
 \>\> ($L.\at{price}$ \bd{is null or} $R_1.\at{price}=L.\at{price}$ \bd{or} ($L.\at{price}='\_'$  \\
 \> \> \bd{and} ($L.\at{price_>}$ \bd{is null or} $R_1.\at{price}>L.\at{price_>}$)  \\
 \> \> \bd{and} ($L.\at{price_\leq}$ \bd{is null or} $R_1.\at{price}\leq L.\at{price_\leq}$))) \\
 \> \> \bd{and not exists} (\bd{select * from} $N$ \\
 \>\>\> \bd{where} $N.\at{cid}=L.\at{cid}$ \bd{and} $N.\at{pos}='\LHS'$ \bd{and} \\
 \> \> \> $N.\at{att}='price'$ \bd{and} $R_{1}.\at{price}=N.\at{val}$) \bd{and}\\

 \> \> ($R.\at{shipping}='\_'$\bd{or} $R.\at{price}='\_'$))  $M$ \\
 \> \> \bd{group by} $\at{sale_x}, \at{price_x}$, \at{cid} \\
 \> \> \> \bd{having count} (\bd{distinct} $\at{shipping_y}, \at{price_y}$)$>1$ }
\end{footnotesize}
\end{example}


}%%%%%%%%%%%%%%%%%%%%%%%%%%%%%%%%%%%%%%


\stitle{SQL queries for CIND$^p$s}. Below we show how the \SQL query $Q_{(i,j)}$ is generated for
validating \pCINDs in $\SCIND^{(i,j)})$, which has not been studied by previous work.
For the lack of space, we put the generation of detection queries for \pCFDs in the supplementary material, which is an extension of the
\SQL techniques for \CFDs and \eCFDs discussed in~\cite{CFDs} and ~\cite{icde08}, respectively.



The query $Q_{(i,j)}$ for the validation of $\SCIND^{(i,j)}$
is given as follows, which capitalizes on the data tables
\Enc{}, \Enc{L}, \Enc{R} and \Enc{\ne} that encode \pCINDs in
$\SCIND^{(i,j)}$.



\begin{footnotesize}\mat{3ex}{
\= \=\bd{select} \=  $R_i.*$ \bd{from} $R_i$, \Enc{L} $L$, \Enc{\ne} $N$ \\
\>\> \bd{where}  $R_i.X\asymp L$ \bd{and} $R_i.X\asymp N$ \bd{and not exists} (\\
\>\>\> \bd{select} $R_j.*$ \bd{from} $R_j$, \Enc{} $H$, \Enc{R} $R$, \Enc{\ne} $N$ \\
\>\>\> \bd{where} \= $R_i.X = R_j.Y$ \bd{and} $L$.\at{cid} =
$R$.\at{cid}  \bd{and} \\
\>\>\>\>  $L$.\at{cid} = $H$.\at{cid} \bd{and} $R_j.Y\asymp R$ \bd{and} $R_j.Y\asymp N$) }
\end{footnotesize}

\noindent Here (1) $X = \{\at{A_1,\dots, A_{m1}}\}$ and $Y$ =
$\{\at{B_1,\dots, B_{m2}}\}$ are the sets of attributes of $R_i$ and
 $R_j$ appearing in $\SCIND^{(i,j)}$, respectively;
(2) $R_i.X\asymp L$ is the conjunction of


\begin{footnotesize}\mat{3ex}{
$L.A_k$\=\ \bd{is null or} $R_i.A_k$ = $L.A_k$ \bd{or} ($L.A_k$ = `\_' \bd{and} \\
\> ($L.A_{k_{>}}$ \bd{is null or} $R_i.A_k$ $>$ $L.A_{k_{>}}$) \bd{and} \\
\>  $L.A_{k_{\ge}}$ \bd{is null or} $R_i.A_k$ $\ge$ $L.A_{k_{\ge}}$) \bd{and} \\
\> ($L.A_{k_{<}}$ \bd{is null or} $R_i.A_k$ $<$ $L.A_{k_{<}}$) \bd{and} \\
\> ($L.A_{k_{\le}}$ \bd{is null or} $R_i.A_k$ $\le$ $L.A_{k_{\le}}$))
}
\end{footnotesize}

\noindent
for each $k\in[1, m_1]$;
(3) $R_j.Y\asymp R$ is defined similarly for attributes
 in $Y$;
(4) $R_i.X\asymp N$ is a shorthand for the conjunction below,
for each $k\in[1, m_1]$:

\begin{footnotesize}\mat{3ex}{
\bd{not exists}
(\=\bd{select} $*$ \bd{from} $N$ \\
\>\bd{where} \= $L.\at{cid}$ =
$N.\at{cid}$ \bd{and} $N.\at{pos}$ = `\LHS' \bd{and}\\
\>\>  $N.\at{att}$ = `$A_k$' \bd{and}
 $R_i.A_k$ = $N.\at{val}$);
}
\end{footnotesize}

\noindent
(5) $R_j.Y$ $\asymp N$ is defined similarly,
but with $N.\at{pos}$ = `\RHS' ; (6) $R_i.X$
  = $R_j.Y$ represents the following:
for each $A_k$  ($k\in[1, m_1]$) and each $B_l$ ($l\in[1, m_2]$),
($H.A_k$ \bd{is null or}
 $H.B_l$ \bd{is null} \bd{or} $H.B_l \ne H.A_k$  \bd{or} $R_i.A_k
= R_j.B_l$).



Intuitively, (1) $R_i.X\asymp L$ \bd{and} $R_i.X\asymp N$ ensure
that the $R_i$ tuples selected match the \LHS patterns of some
\pCINDs in  $\SCIND^{(i,j)}$;
(2) $R_j.Y\asymp R$ \bd{and} $R_j.Y\asymp N$ check the corresponding
 \RHS patterns of these \pCINDs on
$R_j$ tuples;
(3) $R_i.X = R_j.Y$ enforces the
{\em embedded} \INDs; (4) $L$.\at{cid} = $R$.\at{cid}
\bd{and} $L$.\at{cid} = $H$.\at{cid} assure that the \LHS and
\RHS patterns in the same \pCIND are correctly collated; and (5) \bd{not
exists} in $Q_{(i,j)}$ ensures that the $R_i$ tuples selected violate
\pCINDs in $\SCIND^{(i,j)}$.



\begin{example}\label{exa-cind-query}
Using the coding of Fig.~\ref{fig-pcind-encoding}, an \SQL query $Q$
for checking \pCINDs $\psi_1$ and $\psi_2$ of Fig.~\ref{fig-pcind}
is given as follows:


\begin{footnotesize}\mat{0ex}{
\hspace{0ex} \=  \bd{select} $R_1.*$ \bd{from} \at{item} $R_1$, \Enc{L} L, \Enc{\ne} N \\
\> \bd{where} \= ($L.$\at{type} \bd{is null or}  $R_1.$\at{type} =
$L.$\at{type} \bd{or} $L.$\at{type} = `\_') \bd{and} \\
\>\>\bd{not exist} (\=\bd{select * }\bd{from} $N$ \\
\>\>\>\bd{where} \=$N.\at{cid}$ = $L.\at{cid}$ \bd{and} $N.\at{pos} = $ `\LHS' \bd{and} \\
\>\>\>\> $N.\at{att} = $ `type' \bd{and} $R_1.\at{type}$ = $N.\at{val}$) \bd{and} \\

\>\> ($L.$\at{state} \bd{is null or}  $R_1.$\at{state} =
$L.$\at{state} \bd{or} $L.$\at{state} = `\_') \bd{and}\\

\>\> \bd{not exist} (\bd{select * }\bd{from} $N$ \\
\>\>\>\bd{where} $N.\at{cid}$ = $L.\at{cid}$ \bd{and} $N.\at{pos} = $ `\LHS' \bd{and} \\
\>\>\>\> $N.\at{att} = $ `state' \bd{and} $R_1.\at{state} = $N.\at{val}) \bd{and}\\


\> \>  \bd{not exist} (\bd{select} $R_2.*$ \bd{from} \at{tax} $R_2$, \Enc{} H, \Enc{R} R\\
\>\>\>  \bd{where}  ($H.\at{state_L}$ \bd{is null or} $H.\at{state_R}$ \bd{is null or} \\
\>\>\>\> $H.\at{state_L} != H.\at{state_R}$ \bd{or} $R_2.\at{state}$ = $R_1.\at{state}$)\\
\>\>\>\> \bd{and} $L.\at{cid}$ = $H.\at{cid}$  \bd{and} $L.\at{cid}$ = $R.\at{cid}$ \bd{and}\\
\>\>\>\>($R.$\at{rate} \bd{is null or}  $R_2.$\at{rate} = $R.$\at{rate} \bd{or} \\
\>\>\>\> $R.$\at{rate} = `\_')  \bd{and not exist} (\=\bd{select * }\bd{from} $N$ \\
\>\>\>\>\bd{where} $N.\at{cid}$ = $R.\at{cid}$ \bd{and} $N.\at{pos} =$ `\RHS'\\
\>\>\>\>  \bd{and} $N.\at{att} = $ `rate' \bd{and} $R_2.\at{rate} = $N.\at{val})) }
\end{footnotesize}



The \SQL queries generated  %for error detection
can be simplified as
follows. As shown in %Examples~\ref{exa-cfd-query} and \ref{exa-cind-query}
Example~\ref{exa-cind-query}, when checking
patterns imposed by \Enc{}, \Enc{L} or \Enc{R}, the queries need not
consider attributes $A$ if $t[A]$ is \bd{null} for each tuple $t$ in
the table. Similarly, if an attribute $A$ does not appear in any
tuple in \Enc{\ne},  the queries need not check $A$ either. From
this, it follows that we do not even need to generate those
attributes with only \bd{null} patterns for data tables \Enc{}, \Enc{L}
or \Enc{R} when encoding \pCINDs or \pCFDs.
\end{example}

\eat{

\stitle{Optimization}. The \SQL queries generated for error
detection can be simplified as follows. (1) As shown in
Example~\ref{exa-cind-query}, when checking patterns imposed by
\Enc{}, \Enc{L} or \Enc{R}, the queries need not consider attributes
$A$ if $t[A]$ is `null' for each tuple $t$ in the table. Similarly,
if an attribute $A$ does not appear in any tuple in \Enc{\ne},  the
queries need not check $A$ either. (2) One can further partition
$\SCIND^{(i,j)}$ such that each group has a bounded number of
attributes. The price that comes with it is that more than one \SQL
query will be needed. A balance between the number of groups and the
number of queries can be struck based on the dependencies and
relational schemas by using \eg~the clustering
method~\cite{HanK2000}.

\vspace{-1ex}\begin{example} An extreme case of
partitioning is to generate an \SQL query for each \pCIND.
For example,
a query $Q'$ for validating $\psi_1$ of Fig.~\ref{fig-pcind} alone
is:

\vspace{-1ex}\begin{footnotesize}\mat{2ex}{
 \=\bd{select}   $R_1.*$ \bd{from} \at{item} $R_1$ \\
\> \bd{where}  $R_1.$\at{type} $!=$ `art'  \bd{and not exists}
(\=\bd{select} $R_2.*$ \bd{from} \at{tax} $R_2$ \\
\>\>  \bd{where} $R_2.$\at{state} $=$ $R_1.$\at{state})
}\end{footnotesize}\vspace{-1.5ex}

\noindent
Contrasting $Q'$ with query $Q$ of Example~\ref{exa-cind-query},
one can see that $Q'$ is significantly simpler than $Q$.
Note that a single test $R_1.$\at{type} $!=$ `art' is used in
query $Q'$, instead of using:
\bd{not exists} (\bd{select} $N.*$
\bd{from} \Enc{\ne} $N$ \bd{where} $N.\at{cid}$ = 1 \bd{and}
$N.\at{pos}$ = `\LHS' \bd{and} $N.\at{att}$ =`type' \bd{and}
$R_1.\at{type}$ = $N.\at{val}$). This is because \Enc{\ne}
contains a single tuple, and thus the complex test in terms
of \bd{not exists} is unnecessary. In general, if only
a small number of negation patterns are present, one
can use a conjunctions of $!=$ tests instead of
the \bd{not exists} test.
\eop
\end{example}
} \vspace{-1ex}
